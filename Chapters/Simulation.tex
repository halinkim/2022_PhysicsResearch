% Chapter Template

\chapter{Numerical simulation} % Main chapter title

\label{Chapter3} % Change X to a consecutive number; for referencing this chapter elsewhere, use \ref{ChapterX}

\section{Introduction}

In this chapter, we describe the initial data set required prior to simulation, the numerical method for evolution, and the results.

\section{Schwarzschild black hole}
\subsection{Types of black holes}
Accordingly to the No-Hair theorem, all black holes solutions of the Einstein-Maxwell equation of electromagnetism in general relativity can be completely characterized by their observable classical parameters mass, electric charge and angular momentum.

The schwarzschild metric describes the spacetime geometry exterior to any spherical collapsing body. Kerr metric describes the geometry of empty spacetime around a rotating uncharged axially-symmetric black hole with quasi-spherical event horizon. There are also the Reissner-Nordström metric and the Kerr-Newman metric that describe charged black holes\parencite{vindana2018simulation, wald2010general}. In Table~\ref{tab:typeofblackholes}, it can be seen that the types of black holes are classified according to angular momentum and charge.
\begin{table}[H]
	\caption{Classifications of black holes.}
	\label{tab:typeofblackholes}
	\centering
	\begin{tabular}{c | c c}
		\toprule
		 & \textbf{Non-rotating} ($J=0$) & \textbf{Rotating} ($J>0$) \\
		\midrule
		\textbf{Uncharged} ($Q=0$) & Schwarzschild & Kerr\\
		\textbf{Charged}	($Q\neq0$) & Reissner-Nordström & Kerr-Newman\\
		\bottomrule
	\end{tabular}
\end{table}

\subsection{Isotropic coordinate}

In this report, we choose the simplest form, the Schwarzschild black hole. The original form of the Schwarzschild metric is
\begin{equation}
	\dd{s^2} = - \qty(1 - \frac{2M}{r}) \dd{t^2} + \qty(1 - \frac{2M}{r})^{-1}\dd{r^2}+r^2\qty(\dd{\theta^2}+\sin^2\theta\dd{\phi^2}).
\end{equation}
When $r$ goes $2M$, $g_{rr}$ diverges. However, this is just a coordinate singularity, like the problem that occurs at the north and south poles in the spherical coordinate system\parencite{zee2013einstein}. This can be solved by choosing another coordinate system, such as the Kruskal coordinate system.

We can avoid coordinate singularity at $r=2M$ by adopting an isotropic coordinate system by substituting $r=\bar{r}(1+M/2\bar{r})^2$, we get
\begin{equation}
	\dd{s^2} = - \qty(\frac{1-M/(2\bar{r})}{1+M/(2\bar{r})})^2 \dd{t^2} + \qty(1 + \frac{M}{2\bar{r}})^{4}\qty(\dd{\bar{r}^2}+\bar{r}^2\dd{\theta^2}+\bar{r}^2\sin^2\theta\dd{\phi^2}).
\end{equation}
This coordinate system describes the area outside the event horizon $\bar{r}=M/2$. The reason for using this coordinate system is as follows.
\begin{enumerate}
	\item A spatial metric is numerically valid in any space where $r>M/2$.
	\item Since spatial metrics are flat, they can be replaced with Cartesian coordinates, which is more useful for numerical calculations\parencite{brugmann1996adaptive}.
\end{enumerate}
So, in practice we use metric:
\begin{equation}
	\dd{s^2} = - \qty(\frac{1-M/(2r)}{1+M/(2r)})^2 \dd{t^2} + \qty(1 + \frac{M}{2r})^{4}\qty(\dd{x^2}+\dd{y^2}+\dd{z^2}),
\end{equation}
where $r = \sqrt{x^2 + y^2 + z^2}$.

\section{Gauge conditions}
Eq. (\ref{eq:hamiltonian_constraint})-(\ref{eq:evoultion_extrinsic}) does not contain any time derivative of lapse function $N$ nor of the shift vector $\shiftvec$. This means that $N$ and $\shiftvec$ are not dynamical variables. Therefore, we may choose the lapse and shift freely, without changing the physical solution $g$ of the Einstein equation\parencite{gourgoulhon20123+}.

In this simulation, we choose the lapse function and the shift vector
\begin{equation}
	N = \frac{1-M/(2r)}{1+M/(2r)}, \qquad \shiftvec = 0.
\end{equation}

\section{Initial data}
After 3+1 decomposition, we should evolve forward in time some initial data. Instead of solving Hamiltonian and momentum constraint, we use well-known initial data from isotropic coordinates of Schwarzshcild metric. So the initial spatial metric becomes $\gamma_{ij}=(1+M/(2r))^4\delta_{ij}$ and the initial extrinsic curvature becomes $K_{ij} = 0$ since there is no time-dependent and zero shift.

We will show that these data satisfies Eq. (\ref{eq:hamiltonian_constraint}, \ref{eq:momentum_constraint}). Since $E$ and $p_i$ are all $0$ in vacuum space and $K_{ij}=0$, the momentum constraint is naturally satisfied. Now, to satisfy the hamiltonian constraint, we need to show that the 3-metric Ricci scalar $R$ is $0$. Let's use a spherical coordinate system here. The non-vanishing Ricci tensor is:
\begin{align}
	R_{rr} &= - \frac{8 M}{r \left(M + 2 r\right)^{2}},\\
	R_{\theta\theta} &= \frac{4 M r}{\left(M + 2 r\right)^{2}},\\
	R_{\phi\phi} &= \frac{4 M r \sin^{2}{\left(\theta \right)}}{\left(M + 2 r\right)^{2}}.
\end{align}
Therefore, it can be seen that the 3-metric Ricci scalar $R = \gamma^{ij}R_{ij} = 0$, and it can be confirmed that the given constraint condition is well satisfied.

\section{Numerical methods}
\subsection{Finite difference method}
This simulation uses the finite difference method. The Taylor expansion of the function $f(x)$ in $x_0$ is
\begin{equation}
	f(x_0 + h) = f(x_0) + \frac{f'(x_0)}{1!}h + \frac{f^{(2)}(x_0)}{2!}h^2 + \cdots + \frac{f^{(n)}(x_0)}{n!}h^n+ \cdots .
\end{equation}
Arranging this, we get
\begin{equation}
	\begin{aligned}
		\frac{f(x_0 + h) - f(x_0)}{h} &= f'(x_0) + \frac{f^{(2)}(x_0)}{2!}h + \cdots \\
		&= f'(x_0) + \mathcal{O}(h).
	\end{aligned}
\end{equation}
Accuracy is on the order of $\mathrm{O}(h)$.

The central difference method selects the function value from $x-h$ and $x+h$, respectively, and has a more accurate error of $\mathcal{O}(h^2)$.
\begin{equation}
	f'(x_0) = \frac{f(x+ h) - f(x - h)}{2h} + \mathcal{O}(h^2).
\end{equation}

\subsection{Boundary condition}

The isotropic coordinates of the schwarzschild metric only describe the region outside the black hole horizon, i.e. $r=\frac{M}{2}$. Therefore, $\gamma_{ij}$ and $K_{ij}$ did not evolve in the region of $r\le \frac{M}{2}$.

At the boundary outside the grid, the physical quantity at the corresponding point was calculated using linear extrapolation. For example, when we need to find the derivative at $f_i$,
\begin{align}
	(\partial f)_{i - 2} &= \frac{f_{i - 1} - f_{i - 3}}{2h},\\
	(\partial f)_{i - 1} &= \frac{f_{i} - f_{i - 2}}{2h}.
\end{align}

Therefore, we get
\begin{equation}
	\begin{aligned}
			(\partial f)_{i} &= \frac{f_{i - 1} - f_{i - 3}}{2h}\\
			&= \frac{2f_i - f_{i - 1} - 2f_{i - 2} + f_{i - 3}}{2h}.
	\end{aligned}
\end{equation}

Alternatively, there is a way to use a fixed value at the grid boundary, as well as at the black hole horizon.

\subsection{Inverse matrix}

Cofactors were used to find the inverse matrix. See Appendix \ref{AppendixA}.

\section{Grid setting}

The size of the grid is $100^3$, the mass of the black hole is $0.2M$, and the grid distance is $0.01M$. Thus, the horizon of a black hole corresponds to $0.1 M$, i.e. the surface of a sphere with a radius of 10 grids.

\section{Result}

The results obtained at first do not change with time as shown in Fig. \ref{fig:0_data0}.
\begin{figure}[H]
	\centering
	%% Creator: Matplotlib, PGF backend
%%
%% To include the figure in your LaTeX document, write
%%   \input{<filename>.pgf}
%%
%% Make sure the required packages are loaded in your preamble
%%   \usepackage{pgf}
%%
%% Also ensure that all the required font packages are loaded; for instance,
%% the lmodern package is sometimes necessary when using math font.
%%   \usepackage{lmodern}
%%
%% Figures using additional raster images can only be included by \input if
%% they are in the same directory as the main LaTeX file. For loading figures
%% from other directories you can use the `import` package
%%   \usepackage{import}
%%
%% and then include the figures with
%%   \import{<path to file>}{<filename>.pgf}
%%
%% Matplotlib used the following preamble
%%   \usepackage{fontspec}
%%   \setmainfont{DejaVuSerif.ttf}[Path=\detokenize{C:/boj/venv/Lib/site-packages/matplotlib/mpl-data/fonts/ttf/}]
%%   \setsansfont{DejaVuSans.ttf}[Path=\detokenize{C:/boj/venv/Lib/site-packages/matplotlib/mpl-data/fonts/ttf/}]
%%   \setmonofont{DejaVuSansMono.ttf}[Path=\detokenize{C:/boj/venv/Lib/site-packages/matplotlib/mpl-data/fonts/ttf/}]
%%
\begingroup%
\makeatletter%
\begin{pgfpicture}%
\pgfpathrectangle{\pgfpointorigin}{\pgfqpoint{3.000000in}{3.000000in}}%
\pgfusepath{use as bounding box, clip}%
\begin{pgfscope}%
\pgfsetbuttcap%
\pgfsetmiterjoin%
\definecolor{currentfill}{rgb}{1.000000,1.000000,1.000000}%
\pgfsetfillcolor{currentfill}%
\pgfsetlinewidth{0.000000pt}%
\definecolor{currentstroke}{rgb}{1.000000,1.000000,1.000000}%
\pgfsetstrokecolor{currentstroke}%
\pgfsetdash{}{0pt}%
\pgfpathmoveto{\pgfqpoint{0.000000in}{0.000000in}}%
\pgfpathlineto{\pgfqpoint{3.000000in}{0.000000in}}%
\pgfpathlineto{\pgfqpoint{3.000000in}{3.000000in}}%
\pgfpathlineto{\pgfqpoint{0.000000in}{3.000000in}}%
\pgfpathlineto{\pgfqpoint{0.000000in}{0.000000in}}%
\pgfpathclose%
\pgfusepath{fill}%
\end{pgfscope}%
\begin{pgfscope}%
\pgfsetbuttcap%
\pgfsetmiterjoin%
\definecolor{currentfill}{rgb}{1.000000,1.000000,1.000000}%
\pgfsetfillcolor{currentfill}%
\pgfsetlinewidth{0.000000pt}%
\definecolor{currentstroke}{rgb}{0.000000,0.000000,0.000000}%
\pgfsetstrokecolor{currentstroke}%
\pgfsetstrokeopacity{0.000000}%
\pgfsetdash{}{0pt}%
\pgfpathmoveto{\pgfqpoint{0.000000in}{0.000000in}}%
\pgfpathlineto{\pgfqpoint{3.000000in}{0.000000in}}%
\pgfpathlineto{\pgfqpoint{3.000000in}{3.000000in}}%
\pgfpathlineto{\pgfqpoint{0.000000in}{3.000000in}}%
\pgfpathlineto{\pgfqpoint{0.000000in}{0.000000in}}%
\pgfpathclose%
\pgfusepath{fill}%
\end{pgfscope}%
\begin{pgfscope}%
\pgfsetbuttcap%
\pgfsetmiterjoin%
\definecolor{currentfill}{rgb}{0.950000,0.950000,0.950000}%
\pgfsetfillcolor{currentfill}%
\pgfsetfillopacity{0.500000}%
\pgfsetlinewidth{1.003750pt}%
\definecolor{currentstroke}{rgb}{0.950000,0.950000,0.950000}%
\pgfsetstrokecolor{currentstroke}%
\pgfsetstrokeopacity{0.500000}%
\pgfsetdash{}{0pt}%
\pgfpathmoveto{\pgfqpoint{0.226521in}{0.739704in}}%
\pgfpathlineto{\pgfqpoint{1.217219in}{1.570127in}}%
\pgfpathlineto{\pgfqpoint{1.203447in}{2.767745in}}%
\pgfpathlineto{\pgfqpoint{0.165339in}{2.010181in}}%
\pgfusepath{stroke,fill}%
\end{pgfscope}%
\begin{pgfscope}%
\pgfsetbuttcap%
\pgfsetmiterjoin%
\definecolor{currentfill}{rgb}{0.900000,0.900000,0.900000}%
\pgfsetfillcolor{currentfill}%
\pgfsetfillopacity{0.500000}%
\pgfsetlinewidth{1.003750pt}%
\definecolor{currentstroke}{rgb}{0.900000,0.900000,0.900000}%
\pgfsetstrokecolor{currentstroke}%
\pgfsetstrokeopacity{0.500000}%
\pgfsetdash{}{0pt}%
\pgfpathmoveto{\pgfqpoint{1.217219in}{1.570127in}}%
\pgfpathlineto{\pgfqpoint{2.806935in}{1.108059in}}%
\pgfpathlineto{\pgfqpoint{2.863667in}{2.346927in}}%
\pgfpathlineto{\pgfqpoint{1.203447in}{2.767745in}}%
\pgfusepath{stroke,fill}%
\end{pgfscope}%
\begin{pgfscope}%
\pgfsetbuttcap%
\pgfsetmiterjoin%
\definecolor{currentfill}{rgb}{0.925000,0.925000,0.925000}%
\pgfsetfillcolor{currentfill}%
\pgfsetfillopacity{0.500000}%
\pgfsetlinewidth{1.003750pt}%
\definecolor{currentstroke}{rgb}{0.925000,0.925000,0.925000}%
\pgfsetstrokecolor{currentstroke}%
\pgfsetstrokeopacity{0.500000}%
\pgfsetdash{}{0pt}%
\pgfpathmoveto{\pgfqpoint{0.226521in}{0.739704in}}%
\pgfpathlineto{\pgfqpoint{1.911698in}{0.189325in}}%
\pgfpathlineto{\pgfqpoint{2.806935in}{1.108059in}}%
\pgfpathlineto{\pgfqpoint{1.217219in}{1.570127in}}%
\pgfusepath{stroke,fill}%
\end{pgfscope}%
\begin{pgfscope}%
\pgfsetrectcap%
\pgfsetroundjoin%
\pgfsetlinewidth{0.803000pt}%
\definecolor{currentstroke}{rgb}{0.000000,0.000000,0.000000}%
\pgfsetstrokecolor{currentstroke}%
\pgfsetdash{}{0pt}%
\pgfpathmoveto{\pgfqpoint{0.226521in}{0.739704in}}%
\pgfpathlineto{\pgfqpoint{1.911698in}{0.189325in}}%
\pgfusepath{stroke}%
\end{pgfscope}%
\begin{pgfscope}%
\pgfsetbuttcap%
\pgfsetroundjoin%
\pgfsetlinewidth{0.803000pt}%
\definecolor{currentstroke}{rgb}{0.690196,0.690196,0.690196}%
\pgfsetstrokecolor{currentstroke}%
\pgfsetdash{}{0pt}%
\pgfpathmoveto{\pgfqpoint{0.328584in}{0.706370in}}%
\pgfpathlineto{\pgfqpoint{1.313900in}{1.542025in}}%
\pgfpathlineto{\pgfqpoint{1.304216in}{2.742202in}}%
\pgfusepath{stroke}%
\end{pgfscope}%
\begin{pgfscope}%
\pgfsetbuttcap%
\pgfsetroundjoin%
\pgfsetlinewidth{0.803000pt}%
\definecolor{currentstroke}{rgb}{0.690196,0.690196,0.690196}%
\pgfsetstrokecolor{currentstroke}%
\pgfsetdash{}{0pt}%
\pgfpathmoveto{\pgfqpoint{0.683655in}{0.590404in}}%
\pgfpathlineto{\pgfqpoint{1.649847in}{1.444379in}}%
\pgfpathlineto{\pgfqpoint{1.654568in}{2.653398in}}%
\pgfusepath{stroke}%
\end{pgfscope}%
\begin{pgfscope}%
\pgfsetbuttcap%
\pgfsetroundjoin%
\pgfsetlinewidth{0.803000pt}%
\definecolor{currentstroke}{rgb}{0.690196,0.690196,0.690196}%
\pgfsetstrokecolor{currentstroke}%
\pgfsetdash{}{0pt}%
\pgfpathmoveto{\pgfqpoint{1.046896in}{0.471769in}}%
\pgfpathlineto{\pgfqpoint{1.992876in}{1.344674in}}%
\pgfpathlineto{\pgfqpoint{2.012628in}{2.562641in}}%
\pgfusepath{stroke}%
\end{pgfscope}%
\begin{pgfscope}%
\pgfsetbuttcap%
\pgfsetroundjoin%
\pgfsetlinewidth{0.803000pt}%
\definecolor{currentstroke}{rgb}{0.690196,0.690196,0.690196}%
\pgfsetstrokecolor{currentstroke}%
\pgfsetdash{}{0pt}%
\pgfpathmoveto{\pgfqpoint{1.418592in}{0.350373in}}%
\pgfpathlineto{\pgfqpoint{2.343214in}{1.242844in}}%
\pgfpathlineto{\pgfqpoint{2.378655in}{2.469864in}}%
\pgfusepath{stroke}%
\end{pgfscope}%
\begin{pgfscope}%
\pgfsetbuttcap%
\pgfsetroundjoin%
\pgfsetlinewidth{0.803000pt}%
\definecolor{currentstroke}{rgb}{0.690196,0.690196,0.690196}%
\pgfsetstrokecolor{currentstroke}%
\pgfsetdash{}{0pt}%
\pgfpathmoveto{\pgfqpoint{1.799040in}{0.226119in}}%
\pgfpathlineto{\pgfqpoint{2.701096in}{1.138822in}}%
\pgfpathlineto{\pgfqpoint{2.752916in}{2.374999in}}%
\pgfusepath{stroke}%
\end{pgfscope}%
\begin{pgfscope}%
\pgfsetrectcap%
\pgfsetroundjoin%
\pgfsetlinewidth{0.803000pt}%
\definecolor{currentstroke}{rgb}{0.000000,0.000000,0.000000}%
\pgfsetstrokecolor{currentstroke}%
\pgfsetdash{}{0pt}%
\pgfpathmoveto{\pgfqpoint{0.337164in}{0.713647in}}%
\pgfpathlineto{\pgfqpoint{0.311387in}{0.691785in}}%
\pgfusepath{stroke}%
\end{pgfscope}%
\begin{pgfscope}%
\pgfsetrectcap%
\pgfsetroundjoin%
\pgfsetlinewidth{0.803000pt}%
\definecolor{currentstroke}{rgb}{0.000000,0.000000,0.000000}%
\pgfsetstrokecolor{currentstroke}%
\pgfsetdash{}{0pt}%
\pgfpathmoveto{\pgfqpoint{0.692077in}{0.597847in}}%
\pgfpathlineto{\pgfqpoint{0.666776in}{0.575485in}}%
\pgfusepath{stroke}%
\end{pgfscope}%
\begin{pgfscope}%
\pgfsetrectcap%
\pgfsetroundjoin%
\pgfsetlinewidth{0.803000pt}%
\definecolor{currentstroke}{rgb}{0.000000,0.000000,0.000000}%
\pgfsetstrokecolor{currentstroke}%
\pgfsetdash{}{0pt}%
\pgfpathmoveto{\pgfqpoint{1.055149in}{0.479385in}}%
\pgfpathlineto{\pgfqpoint{1.030354in}{0.456505in}}%
\pgfusepath{stroke}%
\end{pgfscope}%
\begin{pgfscope}%
\pgfsetrectcap%
\pgfsetroundjoin%
\pgfsetlinewidth{0.803000pt}%
\definecolor{currentstroke}{rgb}{0.000000,0.000000,0.000000}%
\pgfsetstrokecolor{currentstroke}%
\pgfsetdash{}{0pt}%
\pgfpathmoveto{\pgfqpoint{1.426666in}{0.358167in}}%
\pgfpathlineto{\pgfqpoint{1.402407in}{0.334752in}}%
\pgfusepath{stroke}%
\end{pgfscope}%
\begin{pgfscope}%
\pgfsetrectcap%
\pgfsetroundjoin%
\pgfsetlinewidth{0.803000pt}%
\definecolor{currentstroke}{rgb}{0.000000,0.000000,0.000000}%
\pgfsetstrokecolor{currentstroke}%
\pgfsetdash{}{0pt}%
\pgfpathmoveto{\pgfqpoint{1.806926in}{0.234097in}}%
\pgfpathlineto{\pgfqpoint{1.783235in}{0.210127in}}%
\pgfusepath{stroke}%
\end{pgfscope}%
\begin{pgfscope}%
\pgfsetrectcap%
\pgfsetroundjoin%
\pgfsetlinewidth{0.803000pt}%
\definecolor{currentstroke}{rgb}{0.000000,0.000000,0.000000}%
\pgfsetstrokecolor{currentstroke}%
\pgfsetdash{}{0pt}%
\pgfpathmoveto{\pgfqpoint{2.806935in}{1.108059in}}%
\pgfpathlineto{\pgfqpoint{1.911698in}{0.189325in}}%
\pgfusepath{stroke}%
\end{pgfscope}%
\begin{pgfscope}%
\pgfsetbuttcap%
\pgfsetroundjoin%
\pgfsetlinewidth{0.803000pt}%
\definecolor{currentstroke}{rgb}{0.690196,0.690196,0.690196}%
\pgfsetstrokecolor{currentstroke}%
\pgfsetdash{}{0pt}%
\pgfpathmoveto{\pgfqpoint{0.237127in}{2.062568in}}%
\pgfpathlineto{\pgfqpoint{0.294788in}{0.796927in}}%
\pgfpathlineto{\pgfqpoint{1.973642in}{0.252894in}}%
\pgfusepath{stroke}%
\end{pgfscope}%
\begin{pgfscope}%
\pgfsetbuttcap%
\pgfsetroundjoin%
\pgfsetlinewidth{0.803000pt}%
\definecolor{currentstroke}{rgb}{0.690196,0.690196,0.690196}%
\pgfsetstrokecolor{currentstroke}%
\pgfsetdash{}{0pt}%
\pgfpathmoveto{\pgfqpoint{0.476962in}{2.237589in}}%
\pgfpathlineto{\pgfqpoint{0.523122in}{0.988321in}}%
\pgfpathlineto{\pgfqpoint{2.180550in}{0.465233in}}%
\pgfusepath{stroke}%
\end{pgfscope}%
\begin{pgfscope}%
\pgfsetbuttcap%
\pgfsetroundjoin%
\pgfsetlinewidth{0.803000pt}%
\definecolor{currentstroke}{rgb}{0.690196,0.690196,0.690196}%
\pgfsetstrokecolor{currentstroke}%
\pgfsetdash{}{0pt}%
\pgfpathmoveto{\pgfqpoint{0.707485in}{2.405814in}}%
\pgfpathlineto{\pgfqpoint{0.742969in}{1.172601in}}%
\pgfpathlineto{\pgfqpoint{2.379369in}{0.669270in}}%
\pgfusepath{stroke}%
\end{pgfscope}%
\begin{pgfscope}%
\pgfsetbuttcap%
\pgfsetroundjoin%
\pgfsetlinewidth{0.803000pt}%
\definecolor{currentstroke}{rgb}{0.690196,0.690196,0.690196}%
\pgfsetstrokecolor{currentstroke}%
\pgfsetdash{}{0pt}%
\pgfpathmoveto{\pgfqpoint{0.929229in}{2.567632in}}%
\pgfpathlineto{\pgfqpoint{0.954792in}{1.350155in}}%
\pgfpathlineto{\pgfqpoint{2.570564in}{0.865483in}}%
\pgfusepath{stroke}%
\end{pgfscope}%
\begin{pgfscope}%
\pgfsetbuttcap%
\pgfsetroundjoin%
\pgfsetlinewidth{0.803000pt}%
\definecolor{currentstroke}{rgb}{0.690196,0.690196,0.690196}%
\pgfsetstrokecolor{currentstroke}%
\pgfsetdash{}{0pt}%
\pgfpathmoveto{\pgfqpoint{1.142684in}{2.723402in}}%
\pgfpathlineto{\pgfqpoint{1.159023in}{1.521346in}}%
\pgfpathlineto{\pgfqpoint{2.754565in}{1.054314in}}%
\pgfusepath{stroke}%
\end{pgfscope}%
\begin{pgfscope}%
\pgfsetrectcap%
\pgfsetroundjoin%
\pgfsetlinewidth{0.803000pt}%
\definecolor{currentstroke}{rgb}{0.000000,0.000000,0.000000}%
\pgfsetstrokecolor{currentstroke}%
\pgfsetdash{}{0pt}%
\pgfpathmoveto{\pgfqpoint{1.959494in}{0.257479in}}%
\pgfpathlineto{\pgfqpoint{2.001974in}{0.243713in}}%
\pgfusepath{stroke}%
\end{pgfscope}%
\begin{pgfscope}%
\pgfsetrectcap%
\pgfsetroundjoin%
\pgfsetlinewidth{0.803000pt}%
\definecolor{currentstroke}{rgb}{0.000000,0.000000,0.000000}%
\pgfsetstrokecolor{currentstroke}%
\pgfsetdash{}{0pt}%
\pgfpathmoveto{\pgfqpoint{2.166597in}{0.469637in}}%
\pgfpathlineto{\pgfqpoint{2.208491in}{0.456415in}}%
\pgfusepath{stroke}%
\end{pgfscope}%
\begin{pgfscope}%
\pgfsetrectcap%
\pgfsetroundjoin%
\pgfsetlinewidth{0.803000pt}%
\definecolor{currentstroke}{rgb}{0.000000,0.000000,0.000000}%
\pgfsetstrokecolor{currentstroke}%
\pgfsetdash{}{0pt}%
\pgfpathmoveto{\pgfqpoint{2.365606in}{0.673503in}}%
\pgfpathlineto{\pgfqpoint{2.406929in}{0.660793in}}%
\pgfusepath{stroke}%
\end{pgfscope}%
\begin{pgfscope}%
\pgfsetrectcap%
\pgfsetroundjoin%
\pgfsetlinewidth{0.803000pt}%
\definecolor{currentstroke}{rgb}{0.000000,0.000000,0.000000}%
\pgfsetstrokecolor{currentstroke}%
\pgfsetdash{}{0pt}%
\pgfpathmoveto{\pgfqpoint{2.556988in}{0.869556in}}%
\pgfpathlineto{\pgfqpoint{2.597750in}{0.857329in}}%
\pgfusepath{stroke}%
\end{pgfscope}%
\begin{pgfscope}%
\pgfsetrectcap%
\pgfsetroundjoin%
\pgfsetlinewidth{0.803000pt}%
\definecolor{currentstroke}{rgb}{0.000000,0.000000,0.000000}%
\pgfsetstrokecolor{currentstroke}%
\pgfsetdash{}{0pt}%
\pgfpathmoveto{\pgfqpoint{2.741171in}{1.058235in}}%
\pgfpathlineto{\pgfqpoint{2.781386in}{1.046463in}}%
\pgfusepath{stroke}%
\end{pgfscope}%
\begin{pgfscope}%
\pgfsetrectcap%
\pgfsetroundjoin%
\pgfsetlinewidth{0.803000pt}%
\definecolor{currentstroke}{rgb}{0.000000,0.000000,0.000000}%
\pgfsetstrokecolor{currentstroke}%
\pgfsetdash{}{0pt}%
\pgfpathmoveto{\pgfqpoint{2.806935in}{1.108059in}}%
\pgfpathlineto{\pgfqpoint{2.863667in}{2.346927in}}%
\pgfusepath{stroke}%
\end{pgfscope}%
\begin{pgfscope}%
\pgfsetbuttcap%
\pgfsetroundjoin%
\pgfsetlinewidth{0.803000pt}%
\definecolor{currentstroke}{rgb}{0.690196,0.690196,0.690196}%
\pgfsetstrokecolor{currentstroke}%
\pgfsetdash{}{0pt}%
\pgfpathmoveto{\pgfqpoint{2.808022in}{1.131794in}}%
\pgfpathlineto{\pgfqpoint{1.216954in}{1.593119in}}%
\pgfpathlineto{\pgfqpoint{0.225350in}{0.764005in}}%
\pgfusepath{stroke}%
\end{pgfscope}%
\begin{pgfscope}%
\pgfsetbuttcap%
\pgfsetroundjoin%
\pgfsetlinewidth{0.803000pt}%
\definecolor{currentstroke}{rgb}{0.690196,0.690196,0.690196}%
\pgfsetstrokecolor{currentstroke}%
\pgfsetdash{}{0pt}%
\pgfpathmoveto{\pgfqpoint{2.818552in}{1.361744in}}%
\pgfpathlineto{\pgfqpoint{1.214394in}{1.815776in}}%
\pgfpathlineto{\pgfqpoint{0.214009in}{0.999513in}}%
\pgfusepath{stroke}%
\end{pgfscope}%
\begin{pgfscope}%
\pgfsetbuttcap%
\pgfsetroundjoin%
\pgfsetlinewidth{0.803000pt}%
\definecolor{currentstroke}{rgb}{0.690196,0.690196,0.690196}%
\pgfsetstrokecolor{currentstroke}%
\pgfsetdash{}{0pt}%
\pgfpathmoveto{\pgfqpoint{2.829259in}{1.595546in}}%
\pgfpathlineto{\pgfqpoint{1.211793in}{2.041986in}}%
\pgfpathlineto{\pgfqpoint{0.202471in}{1.239118in}}%
\pgfusepath{stroke}%
\end{pgfscope}%
\begin{pgfscope}%
\pgfsetbuttcap%
\pgfsetroundjoin%
\pgfsetlinewidth{0.803000pt}%
\definecolor{currentstroke}{rgb}{0.690196,0.690196,0.690196}%
\pgfsetstrokecolor{currentstroke}%
\pgfsetdash{}{0pt}%
\pgfpathmoveto{\pgfqpoint{2.840146in}{1.833299in}}%
\pgfpathlineto{\pgfqpoint{1.209150in}{2.271832in}}%
\pgfpathlineto{\pgfqpoint{0.190730in}{1.482928in}}%
\pgfusepath{stroke}%
\end{pgfscope}%
\begin{pgfscope}%
\pgfsetbuttcap%
\pgfsetroundjoin%
\pgfsetlinewidth{0.803000pt}%
\definecolor{currentstroke}{rgb}{0.690196,0.690196,0.690196}%
\pgfsetstrokecolor{currentstroke}%
\pgfsetdash{}{0pt}%
\pgfpathmoveto{\pgfqpoint{2.851219in}{2.075103in}}%
\pgfpathlineto{\pgfqpoint{1.206464in}{2.505404in}}%
\pgfpathlineto{\pgfqpoint{0.178781in}{1.731054in}}%
\pgfusepath{stroke}%
\end{pgfscope}%
\begin{pgfscope}%
\pgfsetbuttcap%
\pgfsetroundjoin%
\pgfsetlinewidth{0.803000pt}%
\definecolor{currentstroke}{rgb}{0.690196,0.690196,0.690196}%
\pgfsetstrokecolor{currentstroke}%
\pgfsetdash{}{0pt}%
\pgfpathmoveto{\pgfqpoint{2.862482in}{2.321063in}}%
\pgfpathlineto{\pgfqpoint{1.203734in}{2.742793in}}%
\pgfpathlineto{\pgfqpoint{0.166618in}{1.983613in}}%
\pgfusepath{stroke}%
\end{pgfscope}%
\begin{pgfscope}%
\pgfsetrectcap%
\pgfsetroundjoin%
\pgfsetlinewidth{0.803000pt}%
\definecolor{currentstroke}{rgb}{0.000000,0.000000,0.000000}%
\pgfsetstrokecolor{currentstroke}%
\pgfsetdash{}{0pt}%
\pgfpathmoveto{\pgfqpoint{2.794668in}{1.135666in}}%
\pgfpathlineto{\pgfqpoint{2.834762in}{1.124041in}}%
\pgfusepath{stroke}%
\end{pgfscope}%
\begin{pgfscope}%
\pgfsetrectcap%
\pgfsetroundjoin%
\pgfsetlinewidth{0.803000pt}%
\definecolor{currentstroke}{rgb}{0.000000,0.000000,0.000000}%
\pgfsetstrokecolor{currentstroke}%
\pgfsetdash{}{0pt}%
\pgfpathmoveto{\pgfqpoint{2.805083in}{1.365556in}}%
\pgfpathlineto{\pgfqpoint{2.845523in}{1.354110in}}%
\pgfusepath{stroke}%
\end{pgfscope}%
\begin{pgfscope}%
\pgfsetrectcap%
\pgfsetroundjoin%
\pgfsetlinewidth{0.803000pt}%
\definecolor{currentstroke}{rgb}{0.000000,0.000000,0.000000}%
\pgfsetstrokecolor{currentstroke}%
\pgfsetdash{}{0pt}%
\pgfpathmoveto{\pgfqpoint{2.815673in}{1.599296in}}%
\pgfpathlineto{\pgfqpoint{2.856464in}{1.588037in}}%
\pgfusepath{stroke}%
\end{pgfscope}%
\begin{pgfscope}%
\pgfsetrectcap%
\pgfsetroundjoin%
\pgfsetlinewidth{0.803000pt}%
\definecolor{currentstroke}{rgb}{0.000000,0.000000,0.000000}%
\pgfsetstrokecolor{currentstroke}%
\pgfsetdash{}{0pt}%
\pgfpathmoveto{\pgfqpoint{2.826441in}{1.836984in}}%
\pgfpathlineto{\pgfqpoint{2.867590in}{1.825920in}}%
\pgfusepath{stroke}%
\end{pgfscope}%
\begin{pgfscope}%
\pgfsetrectcap%
\pgfsetroundjoin%
\pgfsetlinewidth{0.803000pt}%
\definecolor{currentstroke}{rgb}{0.000000,0.000000,0.000000}%
\pgfsetstrokecolor{currentstroke}%
\pgfsetdash{}{0pt}%
\pgfpathmoveto{\pgfqpoint{2.837393in}{2.078720in}}%
\pgfpathlineto{\pgfqpoint{2.878906in}{2.067859in}}%
\pgfusepath{stroke}%
\end{pgfscope}%
\begin{pgfscope}%
\pgfsetrectcap%
\pgfsetroundjoin%
\pgfsetlinewidth{0.803000pt}%
\definecolor{currentstroke}{rgb}{0.000000,0.000000,0.000000}%
\pgfsetstrokecolor{currentstroke}%
\pgfsetdash{}{0pt}%
\pgfpathmoveto{\pgfqpoint{2.848532in}{2.324610in}}%
\pgfpathlineto{\pgfqpoint{2.890417in}{2.313961in}}%
\pgfusepath{stroke}%
\end{pgfscope}%
\begin{pgfscope}%
\pgfpathrectangle{\pgfqpoint{0.000000in}{0.000000in}}{\pgfqpoint{3.000000in}{3.000000in}}%
\pgfusepath{clip}%
\pgfsetbuttcap%
\pgfsetroundjoin%
\definecolor{currentfill}{rgb}{0.500000,0.000000,0.000000}%
\pgfsetfillcolor{currentfill}%
\pgfsetlinewidth{0.000000pt}%
\definecolor{currentstroke}{rgb}{0.000000,0.000000,0.000000}%
\pgfsetstrokecolor{currentstroke}%
\pgfsetdash{}{0pt}%
\pgfpathmoveto{\pgfqpoint{1.582634in}{2.423611in}}%
\pgfpathlineto{\pgfqpoint{1.583474in}{2.443567in}}%
\pgfpathlineto{\pgfqpoint{1.454776in}{2.442044in}}%
\pgfpathlineto{\pgfqpoint{1.456454in}{2.422115in}}%
\pgfpathlineto{\pgfqpoint{1.582634in}{2.423611in}}%
\pgfpathclose%
\pgfusepath{fill}%
\end{pgfscope}%
\begin{pgfscope}%
\pgfpathrectangle{\pgfqpoint{0.000000in}{0.000000in}}{\pgfqpoint{3.000000in}{3.000000in}}%
\pgfusepath{clip}%
\pgfsetbuttcap%
\pgfsetroundjoin%
\definecolor{currentfill}{rgb}{0.500000,0.000000,0.000000}%
\pgfsetfillcolor{currentfill}%
\pgfsetlinewidth{0.000000pt}%
\definecolor{currentstroke}{rgb}{0.000000,0.000000,0.000000}%
\pgfsetstrokecolor{currentstroke}%
\pgfsetdash{}{0pt}%
\pgfpathmoveto{\pgfqpoint{1.707910in}{2.416145in}}%
\pgfpathlineto{\pgfqpoint{1.711253in}{2.435969in}}%
\pgfpathlineto{\pgfqpoint{1.583474in}{2.443567in}}%
\pgfpathlineto{\pgfqpoint{1.582634in}{2.423611in}}%
\pgfpathlineto{\pgfqpoint{1.707910in}{2.416145in}}%
\pgfpathclose%
\pgfusepath{fill}%
\end{pgfscope}%
\begin{pgfscope}%
\pgfpathrectangle{\pgfqpoint{0.000000in}{0.000000in}}{\pgfqpoint{3.000000in}{3.000000in}}%
\pgfusepath{clip}%
\pgfsetbuttcap%
\pgfsetroundjoin%
\definecolor{currentfill}{rgb}{0.553476,0.000000,0.000000}%
\pgfsetfillcolor{currentfill}%
\pgfsetlinewidth{0.000000pt}%
\definecolor{currentstroke}{rgb}{0.000000,0.000000,0.000000}%
\pgfsetstrokecolor{currentstroke}%
\pgfsetdash{}{0pt}%
\pgfpathmoveto{\pgfqpoint{1.581793in}{2.403467in}}%
\pgfpathlineto{\pgfqpoint{1.582634in}{2.423611in}}%
\pgfpathlineto{\pgfqpoint{1.456454in}{2.422115in}}%
\pgfpathlineto{\pgfqpoint{1.458135in}{2.401997in}}%
\pgfpathlineto{\pgfqpoint{1.581793in}{2.403467in}}%
\pgfpathclose%
\pgfusepath{fill}%
\end{pgfscope}%
\begin{pgfscope}%
\pgfpathrectangle{\pgfqpoint{0.000000in}{0.000000in}}{\pgfqpoint{3.000000in}{3.000000in}}%
\pgfusepath{clip}%
\pgfsetbuttcap%
\pgfsetroundjoin%
\definecolor{currentfill}{rgb}{0.000000,0.000000,0.500000}%
\pgfsetfillcolor{currentfill}%
\pgfsetlinewidth{0.000000pt}%
\definecolor{currentstroke}{rgb}{0.000000,0.000000,0.000000}%
\pgfsetstrokecolor{currentstroke}%
\pgfsetdash{}{0pt}%
\pgfpathmoveto{\pgfqpoint{1.549126in}{1.130168in}}%
\pgfpathlineto{\pgfqpoint{1.550003in}{1.215590in}}%
\pgfpathlineto{\pgfqpoint{1.521641in}{1.215206in}}%
\pgfpathlineto{\pgfqpoint{1.523392in}{1.129816in}}%
\pgfpathlineto{\pgfqpoint{1.549126in}{1.130168in}}%
\pgfpathclose%
\pgfusepath{fill}%
\end{pgfscope}%
\begin{pgfscope}%
\pgfpathrectangle{\pgfqpoint{0.000000in}{0.000000in}}{\pgfqpoint{3.000000in}{3.000000in}}%
\pgfusepath{clip}%
\pgfsetbuttcap%
\pgfsetroundjoin%
\definecolor{currentfill}{rgb}{0.000000,0.000000,0.500000}%
\pgfsetfillcolor{currentfill}%
\pgfsetlinewidth{0.000000pt}%
\definecolor{currentstroke}{rgb}{0.000000,0.000000,0.000000}%
\pgfsetstrokecolor{currentstroke}%
\pgfsetdash{}{0pt}%
\pgfpathmoveto{\pgfqpoint{1.574655in}{1.128414in}}%
\pgfpathlineto{\pgfqpoint{1.578139in}{1.213674in}}%
\pgfpathlineto{\pgfqpoint{1.550003in}{1.215590in}}%
\pgfpathlineto{\pgfqpoint{1.549126in}{1.130168in}}%
\pgfpathlineto{\pgfqpoint{1.574655in}{1.128414in}}%
\pgfpathclose%
\pgfusepath{fill}%
\end{pgfscope}%
\begin{pgfscope}%
\pgfpathrectangle{\pgfqpoint{0.000000in}{0.000000in}}{\pgfqpoint{3.000000in}{3.000000in}}%
\pgfusepath{clip}%
\pgfsetbuttcap%
\pgfsetroundjoin%
\definecolor{currentfill}{rgb}{0.500000,0.000000,0.000000}%
\pgfsetfillcolor{currentfill}%
\pgfsetlinewidth{0.000000pt}%
\definecolor{currentstroke}{rgb}{0.000000,0.000000,0.000000}%
\pgfsetstrokecolor{currentstroke}%
\pgfsetdash{}{0pt}%
\pgfpathmoveto{\pgfqpoint{1.456454in}{2.422115in}}%
\pgfpathlineto{\pgfqpoint{1.454776in}{2.442044in}}%
\pgfpathlineto{\pgfqpoint{1.327915in}{2.431429in}}%
\pgfpathlineto{\pgfqpoint{1.332081in}{2.411684in}}%
\pgfpathlineto{\pgfqpoint{1.456454in}{2.422115in}}%
\pgfpathclose%
\pgfusepath{fill}%
\end{pgfscope}%
\begin{pgfscope}%
\pgfpathrectangle{\pgfqpoint{0.000000in}{0.000000in}}{\pgfqpoint{3.000000in}{3.000000in}}%
\pgfusepath{clip}%
\pgfsetbuttcap%
\pgfsetroundjoin%
\definecolor{currentfill}{rgb}{0.000000,0.000000,0.500000}%
\pgfsetfillcolor{currentfill}%
\pgfsetlinewidth{0.000000pt}%
\definecolor{currentstroke}{rgb}{0.000000,0.000000,0.000000}%
\pgfsetstrokecolor{currentstroke}%
\pgfsetdash{}{0pt}%
\pgfpathmoveto{\pgfqpoint{1.523392in}{1.129816in}}%
\pgfpathlineto{\pgfqpoint{1.521641in}{1.215206in}}%
\pgfpathlineto{\pgfqpoint{1.493731in}{1.212530in}}%
\pgfpathlineto{\pgfqpoint{1.498068in}{1.127367in}}%
\pgfpathlineto{\pgfqpoint{1.523392in}{1.129816in}}%
\pgfpathclose%
\pgfusepath{fill}%
\end{pgfscope}%
\begin{pgfscope}%
\pgfpathrectangle{\pgfqpoint{0.000000in}{0.000000in}}{\pgfqpoint{3.000000in}{3.000000in}}%
\pgfusepath{clip}%
\pgfsetbuttcap%
\pgfsetroundjoin%
\definecolor{currentfill}{rgb}{0.553476,0.000000,0.000000}%
\pgfsetfillcolor{currentfill}%
\pgfsetlinewidth{0.000000pt}%
\definecolor{currentstroke}{rgb}{0.000000,0.000000,0.000000}%
\pgfsetstrokecolor{currentstroke}%
\pgfsetdash{}{0pt}%
\pgfpathmoveto{\pgfqpoint{1.704563in}{2.396133in}}%
\pgfpathlineto{\pgfqpoint{1.707910in}{2.416145in}}%
\pgfpathlineto{\pgfqpoint{1.582634in}{2.423611in}}%
\pgfpathlineto{\pgfqpoint{1.581793in}{2.403467in}}%
\pgfpathlineto{\pgfqpoint{1.704563in}{2.396133in}}%
\pgfpathclose%
\pgfusepath{fill}%
\end{pgfscope}%
\begin{pgfscope}%
\pgfpathrectangle{\pgfqpoint{0.000000in}{0.000000in}}{\pgfqpoint{3.000000in}{3.000000in}}%
\pgfusepath{clip}%
\pgfsetbuttcap%
\pgfsetroundjoin%
\definecolor{currentfill}{rgb}{0.606952,0.000000,0.000000}%
\pgfsetfillcolor{currentfill}%
\pgfsetlinewidth{0.000000pt}%
\definecolor{currentstroke}{rgb}{0.000000,0.000000,0.000000}%
\pgfsetstrokecolor{currentstroke}%
\pgfsetdash{}{0pt}%
\pgfpathmoveto{\pgfqpoint{1.580951in}{2.383125in}}%
\pgfpathlineto{\pgfqpoint{1.581793in}{2.403467in}}%
\pgfpathlineto{\pgfqpoint{1.458135in}{2.401997in}}%
\pgfpathlineto{\pgfqpoint{1.459817in}{2.381681in}}%
\pgfpathlineto{\pgfqpoint{1.580951in}{2.383125in}}%
\pgfpathclose%
\pgfusepath{fill}%
\end{pgfscope}%
\begin{pgfscope}%
\pgfpathrectangle{\pgfqpoint{0.000000in}{0.000000in}}{\pgfqpoint{3.000000in}{3.000000in}}%
\pgfusepath{clip}%
\pgfsetbuttcap%
\pgfsetroundjoin%
\definecolor{currentfill}{rgb}{0.000000,0.000000,0.500000}%
\pgfsetfillcolor{currentfill}%
\pgfsetlinewidth{0.000000pt}%
\definecolor{currentstroke}{rgb}{0.000000,0.000000,0.000000}%
\pgfsetstrokecolor{currentstroke}%
\pgfsetdash{}{0pt}%
\pgfpathmoveto{\pgfqpoint{1.599365in}{1.124595in}}%
\pgfpathlineto{\pgfqpoint{1.605375in}{1.209502in}}%
\pgfpathlineto{\pgfqpoint{1.578139in}{1.213674in}}%
\pgfpathlineto{\pgfqpoint{1.574655in}{1.128414in}}%
\pgfpathlineto{\pgfqpoint{1.599365in}{1.124595in}}%
\pgfpathclose%
\pgfusepath{fill}%
\end{pgfscope}%
\begin{pgfscope}%
\pgfpathrectangle{\pgfqpoint{0.000000in}{0.000000in}}{\pgfqpoint{3.000000in}{3.000000in}}%
\pgfusepath{clip}%
\pgfsetbuttcap%
\pgfsetroundjoin%
\definecolor{currentfill}{rgb}{0.000000,0.000000,0.500000}%
\pgfsetfillcolor{currentfill}%
\pgfsetlinewidth{0.000000pt}%
\definecolor{currentstroke}{rgb}{0.000000,0.000000,0.000000}%
\pgfsetstrokecolor{currentstroke}%
\pgfsetdash{}{0pt}%
\pgfpathmoveto{\pgfqpoint{1.498068in}{1.127367in}}%
\pgfpathlineto{\pgfqpoint{1.493731in}{1.212530in}}%
\pgfpathlineto{\pgfqpoint{1.466942in}{1.207625in}}%
\pgfpathlineto{\pgfqpoint{1.473764in}{1.122877in}}%
\pgfpathlineto{\pgfqpoint{1.498068in}{1.127367in}}%
\pgfpathclose%
\pgfusepath{fill}%
\end{pgfscope}%
\begin{pgfscope}%
\pgfpathrectangle{\pgfqpoint{0.000000in}{0.000000in}}{\pgfqpoint{3.000000in}{3.000000in}}%
\pgfusepath{clip}%
\pgfsetbuttcap%
\pgfsetroundjoin%
\definecolor{currentfill}{rgb}{0.553476,0.000000,0.000000}%
\pgfsetfillcolor{currentfill}%
\pgfsetlinewidth{0.000000pt}%
\definecolor{currentstroke}{rgb}{0.000000,0.000000,0.000000}%
\pgfsetstrokecolor{currentstroke}%
\pgfsetdash{}{0pt}%
\pgfpathmoveto{\pgfqpoint{1.458135in}{2.401997in}}%
\pgfpathlineto{\pgfqpoint{1.456454in}{2.422115in}}%
\pgfpathlineto{\pgfqpoint{1.332081in}{2.411684in}}%
\pgfpathlineto{\pgfqpoint{1.336252in}{2.391750in}}%
\pgfpathlineto{\pgfqpoint{1.458135in}{2.401997in}}%
\pgfpathclose%
\pgfusepath{fill}%
\end{pgfscope}%
\begin{pgfscope}%
\pgfpathrectangle{\pgfqpoint{0.000000in}{0.000000in}}{\pgfqpoint{3.000000in}{3.000000in}}%
\pgfusepath{clip}%
\pgfsetbuttcap%
\pgfsetroundjoin%
\definecolor{currentfill}{rgb}{0.606952,0.000000,0.000000}%
\pgfsetfillcolor{currentfill}%
\pgfsetlinewidth{0.000000pt}%
\definecolor{currentstroke}{rgb}{0.000000,0.000000,0.000000}%
\pgfsetstrokecolor{currentstroke}%
\pgfsetdash{}{0pt}%
\pgfpathmoveto{\pgfqpoint{1.701212in}{2.375923in}}%
\pgfpathlineto{\pgfqpoint{1.704563in}{2.396133in}}%
\pgfpathlineto{\pgfqpoint{1.581793in}{2.403467in}}%
\pgfpathlineto{\pgfqpoint{1.580951in}{2.383125in}}%
\pgfpathlineto{\pgfqpoint{1.701212in}{2.375923in}}%
\pgfpathclose%
\pgfusepath{fill}%
\end{pgfscope}%
\begin{pgfscope}%
\pgfpathrectangle{\pgfqpoint{0.000000in}{0.000000in}}{\pgfqpoint{3.000000in}{3.000000in}}%
\pgfusepath{clip}%
\pgfsetbuttcap%
\pgfsetroundjoin%
\definecolor{currentfill}{rgb}{0.678253,0.000000,0.000000}%
\pgfsetfillcolor{currentfill}%
\pgfsetlinewidth{0.000000pt}%
\definecolor{currentstroke}{rgb}{0.000000,0.000000,0.000000}%
\pgfsetstrokecolor{currentstroke}%
\pgfsetdash{}{0pt}%
\pgfpathmoveto{\pgfqpoint{1.580108in}{2.362575in}}%
\pgfpathlineto{\pgfqpoint{1.580951in}{2.383125in}}%
\pgfpathlineto{\pgfqpoint{1.459817in}{2.381681in}}%
\pgfpathlineto{\pgfqpoint{1.461501in}{2.361158in}}%
\pgfpathlineto{\pgfqpoint{1.580108in}{2.362575in}}%
\pgfpathclose%
\pgfusepath{fill}%
\end{pgfscope}%
\begin{pgfscope}%
\pgfpathrectangle{\pgfqpoint{0.000000in}{0.000000in}}{\pgfqpoint{3.000000in}{3.000000in}}%
\pgfusepath{clip}%
\pgfsetbuttcap%
\pgfsetroundjoin%
\definecolor{currentfill}{rgb}{0.000000,0.000000,0.500000}%
\pgfsetfillcolor{currentfill}%
\pgfsetlinewidth{0.000000pt}%
\definecolor{currentstroke}{rgb}{0.000000,0.000000,0.000000}%
\pgfsetstrokecolor{currentstroke}%
\pgfsetdash{}{0pt}%
\pgfpathmoveto{\pgfqpoint{1.622663in}{1.118801in}}%
\pgfpathlineto{\pgfqpoint{1.631057in}{1.203171in}}%
\pgfpathlineto{\pgfqpoint{1.605375in}{1.209502in}}%
\pgfpathlineto{\pgfqpoint{1.599365in}{1.124595in}}%
\pgfpathlineto{\pgfqpoint{1.622663in}{1.118801in}}%
\pgfpathclose%
\pgfusepath{fill}%
\end{pgfscope}%
\begin{pgfscope}%
\pgfpathrectangle{\pgfqpoint{0.000000in}{0.000000in}}{\pgfqpoint{3.000000in}{3.000000in}}%
\pgfusepath{clip}%
\pgfsetbuttcap%
\pgfsetroundjoin%
\definecolor{currentfill}{rgb}{0.606952,0.000000,0.000000}%
\pgfsetfillcolor{currentfill}%
\pgfsetlinewidth{0.000000pt}%
\definecolor{currentstroke}{rgb}{0.000000,0.000000,0.000000}%
\pgfsetstrokecolor{currentstroke}%
\pgfsetdash{}{0pt}%
\pgfpathmoveto{\pgfqpoint{1.459817in}{2.381681in}}%
\pgfpathlineto{\pgfqpoint{1.458135in}{2.401997in}}%
\pgfpathlineto{\pgfqpoint{1.336252in}{2.391750in}}%
\pgfpathlineto{\pgfqpoint{1.340428in}{2.371620in}}%
\pgfpathlineto{\pgfqpoint{1.459817in}{2.381681in}}%
\pgfpathclose%
\pgfusepath{fill}%
\end{pgfscope}%
\begin{pgfscope}%
\pgfpathrectangle{\pgfqpoint{0.000000in}{0.000000in}}{\pgfqpoint{3.000000in}{3.000000in}}%
\pgfusepath{clip}%
\pgfsetbuttcap%
\pgfsetroundjoin%
\definecolor{currentfill}{rgb}{0.678253,0.000000,0.000000}%
\pgfsetfillcolor{currentfill}%
\pgfsetlinewidth{0.000000pt}%
\definecolor{currentstroke}{rgb}{0.000000,0.000000,0.000000}%
\pgfsetstrokecolor{currentstroke}%
\pgfsetdash{}{0pt}%
\pgfpathmoveto{\pgfqpoint{1.697857in}{2.355506in}}%
\pgfpathlineto{\pgfqpoint{1.701212in}{2.375923in}}%
\pgfpathlineto{\pgfqpoint{1.580951in}{2.383125in}}%
\pgfpathlineto{\pgfqpoint{1.580108in}{2.362575in}}%
\pgfpathlineto{\pgfqpoint{1.697857in}{2.355506in}}%
\pgfpathclose%
\pgfusepath{fill}%
\end{pgfscope}%
\begin{pgfscope}%
\pgfpathrectangle{\pgfqpoint{0.000000in}{0.000000in}}{\pgfqpoint{3.000000in}{3.000000in}}%
\pgfusepath{clip}%
\pgfsetbuttcap%
\pgfsetroundjoin%
\definecolor{currentfill}{rgb}{0.000000,0.000000,0.500000}%
\pgfsetfillcolor{currentfill}%
\pgfsetlinewidth{0.000000pt}%
\definecolor{currentstroke}{rgb}{0.000000,0.000000,0.000000}%
\pgfsetstrokecolor{currentstroke}%
\pgfsetdash{}{0pt}%
\pgfpathmoveto{\pgfqpoint{1.473764in}{1.122877in}}%
\pgfpathlineto{\pgfqpoint{1.466942in}{1.207625in}}%
\pgfpathlineto{\pgfqpoint{1.441917in}{1.200605in}}%
\pgfpathlineto{\pgfqpoint{1.451065in}{1.116452in}}%
\pgfpathlineto{\pgfqpoint{1.473764in}{1.122877in}}%
\pgfpathclose%
\pgfusepath{fill}%
\end{pgfscope}%
\begin{pgfscope}%
\pgfpathrectangle{\pgfqpoint{0.000000in}{0.000000in}}{\pgfqpoint{3.000000in}{3.000000in}}%
\pgfusepath{clip}%
\pgfsetbuttcap%
\pgfsetroundjoin%
\definecolor{currentfill}{rgb}{0.731729,0.000000,0.000000}%
\pgfsetfillcolor{currentfill}%
\pgfsetlinewidth{0.000000pt}%
\definecolor{currentstroke}{rgb}{0.000000,0.000000,0.000000}%
\pgfsetstrokecolor{currentstroke}%
\pgfsetdash{}{0pt}%
\pgfpathmoveto{\pgfqpoint{1.579264in}{2.341807in}}%
\pgfpathlineto{\pgfqpoint{1.580108in}{2.362575in}}%
\pgfpathlineto{\pgfqpoint{1.461501in}{2.361158in}}%
\pgfpathlineto{\pgfqpoint{1.463187in}{2.340417in}}%
\pgfpathlineto{\pgfqpoint{1.579264in}{2.341807in}}%
\pgfpathclose%
\pgfusepath{fill}%
\end{pgfscope}%
\begin{pgfscope}%
\pgfpathrectangle{\pgfqpoint{0.000000in}{0.000000in}}{\pgfqpoint{3.000000in}{3.000000in}}%
\pgfusepath{clip}%
\pgfsetbuttcap%
\pgfsetroundjoin%
\definecolor{currentfill}{rgb}{0.500000,0.000000,0.000000}%
\pgfsetfillcolor{currentfill}%
\pgfsetlinewidth{0.000000pt}%
\definecolor{currentstroke}{rgb}{0.000000,0.000000,0.000000}%
\pgfsetstrokecolor{currentstroke}%
\pgfsetdash{}{0pt}%
\pgfpathmoveto{\pgfqpoint{1.829581in}{2.399853in}}%
\pgfpathlineto{\pgfqpoint{1.835366in}{2.419390in}}%
\pgfpathlineto{\pgfqpoint{1.711253in}{2.435969in}}%
\pgfpathlineto{\pgfqpoint{1.707910in}{2.416145in}}%
\pgfpathlineto{\pgfqpoint{1.829581in}{2.399853in}}%
\pgfpathclose%
\pgfusepath{fill}%
\end{pgfscope}%
\begin{pgfscope}%
\pgfpathrectangle{\pgfqpoint{0.000000in}{0.000000in}}{\pgfqpoint{3.000000in}{3.000000in}}%
\pgfusepath{clip}%
\pgfsetbuttcap%
\pgfsetroundjoin%
\definecolor{currentfill}{rgb}{0.678253,0.000000,0.000000}%
\pgfsetfillcolor{currentfill}%
\pgfsetlinewidth{0.000000pt}%
\definecolor{currentstroke}{rgb}{0.000000,0.000000,0.000000}%
\pgfsetstrokecolor{currentstroke}%
\pgfsetdash{}{0pt}%
\pgfpathmoveto{\pgfqpoint{1.461501in}{2.361158in}}%
\pgfpathlineto{\pgfqpoint{1.459817in}{2.381681in}}%
\pgfpathlineto{\pgfqpoint{1.340428in}{2.371620in}}%
\pgfpathlineto{\pgfqpoint{1.344608in}{2.351283in}}%
\pgfpathlineto{\pgfqpoint{1.461501in}{2.361158in}}%
\pgfpathclose%
\pgfusepath{fill}%
\end{pgfscope}%
\begin{pgfscope}%
\pgfpathrectangle{\pgfqpoint{0.000000in}{0.000000in}}{\pgfqpoint{3.000000in}{3.000000in}}%
\pgfusepath{clip}%
\pgfsetbuttcap%
\pgfsetroundjoin%
\definecolor{currentfill}{rgb}{0.731729,0.000000,0.000000}%
\pgfsetfillcolor{currentfill}%
\pgfsetlinewidth{0.000000pt}%
\definecolor{currentstroke}{rgb}{0.000000,0.000000,0.000000}%
\pgfsetstrokecolor{currentstroke}%
\pgfsetdash{}{0pt}%
\pgfpathmoveto{\pgfqpoint{1.694499in}{2.334872in}}%
\pgfpathlineto{\pgfqpoint{1.697857in}{2.355506in}}%
\pgfpathlineto{\pgfqpoint{1.580108in}{2.362575in}}%
\pgfpathlineto{\pgfqpoint{1.579264in}{2.341807in}}%
\pgfpathlineto{\pgfqpoint{1.694499in}{2.334872in}}%
\pgfpathclose%
\pgfusepath{fill}%
\end{pgfscope}%
\begin{pgfscope}%
\pgfpathrectangle{\pgfqpoint{0.000000in}{0.000000in}}{\pgfqpoint{3.000000in}{3.000000in}}%
\pgfusepath{clip}%
\pgfsetbuttcap%
\pgfsetroundjoin%
\definecolor{currentfill}{rgb}{0.803030,0.000000,0.000000}%
\pgfsetfillcolor{currentfill}%
\pgfsetlinewidth{0.000000pt}%
\definecolor{currentstroke}{rgb}{0.000000,0.000000,0.000000}%
\pgfsetstrokecolor{currentstroke}%
\pgfsetdash{}{0pt}%
\pgfpathmoveto{\pgfqpoint{1.578419in}{2.320810in}}%
\pgfpathlineto{\pgfqpoint{1.579264in}{2.341807in}}%
\pgfpathlineto{\pgfqpoint{1.463187in}{2.340417in}}%
\pgfpathlineto{\pgfqpoint{1.464876in}{2.319447in}}%
\pgfpathlineto{\pgfqpoint{1.578419in}{2.320810in}}%
\pgfpathclose%
\pgfusepath{fill}%
\end{pgfscope}%
\begin{pgfscope}%
\pgfpathrectangle{\pgfqpoint{0.000000in}{0.000000in}}{\pgfqpoint{3.000000in}{3.000000in}}%
\pgfusepath{clip}%
\pgfsetbuttcap%
\pgfsetroundjoin%
\definecolor{currentfill}{rgb}{0.553476,0.000000,0.000000}%
\pgfsetfillcolor{currentfill}%
\pgfsetlinewidth{0.000000pt}%
\definecolor{currentstroke}{rgb}{0.000000,0.000000,0.000000}%
\pgfsetstrokecolor{currentstroke}%
\pgfsetdash{}{0pt}%
\pgfpathmoveto{\pgfqpoint{1.823790in}{2.380130in}}%
\pgfpathlineto{\pgfqpoint{1.829581in}{2.399853in}}%
\pgfpathlineto{\pgfqpoint{1.707910in}{2.416145in}}%
\pgfpathlineto{\pgfqpoint{1.704563in}{2.396133in}}%
\pgfpathlineto{\pgfqpoint{1.823790in}{2.380130in}}%
\pgfpathclose%
\pgfusepath{fill}%
\end{pgfscope}%
\begin{pgfscope}%
\pgfpathrectangle{\pgfqpoint{0.000000in}{0.000000in}}{\pgfqpoint{3.000000in}{3.000000in}}%
\pgfusepath{clip}%
\pgfsetbuttcap%
\pgfsetroundjoin%
\definecolor{currentfill}{rgb}{0.000000,0.000000,0.500000}%
\pgfsetfillcolor{currentfill}%
\pgfsetlinewidth{0.000000pt}%
\definecolor{currentstroke}{rgb}{0.000000,0.000000,0.000000}%
\pgfsetstrokecolor{currentstroke}%
\pgfsetdash{}{0pt}%
\pgfpathmoveto{\pgfqpoint{1.643983in}{1.111167in}}%
\pgfpathlineto{\pgfqpoint{1.654566in}{1.194829in}}%
\pgfpathlineto{\pgfqpoint{1.631057in}{1.203171in}}%
\pgfpathlineto{\pgfqpoint{1.622663in}{1.118801in}}%
\pgfpathlineto{\pgfqpoint{1.643983in}{1.111167in}}%
\pgfpathclose%
\pgfusepath{fill}%
\end{pgfscope}%
\begin{pgfscope}%
\pgfpathrectangle{\pgfqpoint{0.000000in}{0.000000in}}{\pgfqpoint{3.000000in}{3.000000in}}%
\pgfusepath{clip}%
\pgfsetbuttcap%
\pgfsetroundjoin%
\definecolor{currentfill}{rgb}{0.000000,0.000000,0.838681}%
\pgfsetfillcolor{currentfill}%
\pgfsetlinewidth{0.000000pt}%
\definecolor{currentstroke}{rgb}{0.000000,0.000000,0.000000}%
\pgfsetstrokecolor{currentstroke}%
\pgfsetdash{}{0pt}%
\pgfpathmoveto{\pgfqpoint{1.550003in}{1.215590in}}%
\pgfpathlineto{\pgfqpoint{1.550879in}{1.289758in}}%
\pgfpathlineto{\pgfqpoint{1.519892in}{1.289341in}}%
\pgfpathlineto{\pgfqpoint{1.521641in}{1.215206in}}%
\pgfpathlineto{\pgfqpoint{1.550003in}{1.215590in}}%
\pgfpathclose%
\pgfusepath{fill}%
\end{pgfscope}%
\begin{pgfscope}%
\pgfpathrectangle{\pgfqpoint{0.000000in}{0.000000in}}{\pgfqpoint{3.000000in}{3.000000in}}%
\pgfusepath{clip}%
\pgfsetbuttcap%
\pgfsetroundjoin%
\definecolor{currentfill}{rgb}{0.000000,0.000000,0.838681}%
\pgfsetfillcolor{currentfill}%
\pgfsetlinewidth{0.000000pt}%
\definecolor{currentstroke}{rgb}{0.000000,0.000000,0.000000}%
\pgfsetstrokecolor{currentstroke}%
\pgfsetdash{}{0pt}%
\pgfpathmoveto{\pgfqpoint{1.578139in}{1.213674in}}%
\pgfpathlineto{\pgfqpoint{1.581618in}{1.287680in}}%
\pgfpathlineto{\pgfqpoint{1.550879in}{1.289758in}}%
\pgfpathlineto{\pgfqpoint{1.550003in}{1.215590in}}%
\pgfpathlineto{\pgfqpoint{1.578139in}{1.213674in}}%
\pgfpathclose%
\pgfusepath{fill}%
\end{pgfscope}%
\begin{pgfscope}%
\pgfpathrectangle{\pgfqpoint{0.000000in}{0.000000in}}{\pgfqpoint{3.000000in}{3.000000in}}%
\pgfusepath{clip}%
\pgfsetbuttcap%
\pgfsetroundjoin%
\definecolor{currentfill}{rgb}{0.731729,0.000000,0.000000}%
\pgfsetfillcolor{currentfill}%
\pgfsetlinewidth{0.000000pt}%
\definecolor{currentstroke}{rgb}{0.000000,0.000000,0.000000}%
\pgfsetstrokecolor{currentstroke}%
\pgfsetdash{}{0pt}%
\pgfpathmoveto{\pgfqpoint{1.463187in}{2.340417in}}%
\pgfpathlineto{\pgfqpoint{1.461501in}{2.361158in}}%
\pgfpathlineto{\pgfqpoint{1.344608in}{2.351283in}}%
\pgfpathlineto{\pgfqpoint{1.348793in}{2.330729in}}%
\pgfpathlineto{\pgfqpoint{1.463187in}{2.340417in}}%
\pgfpathclose%
\pgfusepath{fill}%
\end{pgfscope}%
\begin{pgfscope}%
\pgfpathrectangle{\pgfqpoint{0.000000in}{0.000000in}}{\pgfqpoint{3.000000in}{3.000000in}}%
\pgfusepath{clip}%
\pgfsetbuttcap%
\pgfsetroundjoin%
\definecolor{currentfill}{rgb}{0.000000,0.000000,0.838681}%
\pgfsetfillcolor{currentfill}%
\pgfsetlinewidth{0.000000pt}%
\definecolor{currentstroke}{rgb}{0.000000,0.000000,0.000000}%
\pgfsetstrokecolor{currentstroke}%
\pgfsetdash{}{0pt}%
\pgfpathmoveto{\pgfqpoint{1.521641in}{1.215206in}}%
\pgfpathlineto{\pgfqpoint{1.519892in}{1.289341in}}%
\pgfpathlineto{\pgfqpoint{1.489398in}{1.286440in}}%
\pgfpathlineto{\pgfqpoint{1.493731in}{1.212530in}}%
\pgfpathlineto{\pgfqpoint{1.521641in}{1.215206in}}%
\pgfpathclose%
\pgfusepath{fill}%
\end{pgfscope}%
\begin{pgfscope}%
\pgfpathrectangle{\pgfqpoint{0.000000in}{0.000000in}}{\pgfqpoint{3.000000in}{3.000000in}}%
\pgfusepath{clip}%
\pgfsetbuttcap%
\pgfsetroundjoin%
\definecolor{currentfill}{rgb}{0.803030,0.000000,0.000000}%
\pgfsetfillcolor{currentfill}%
\pgfsetlinewidth{0.000000pt}%
\definecolor{currentstroke}{rgb}{0.000000,0.000000,0.000000}%
\pgfsetstrokecolor{currentstroke}%
\pgfsetdash{}{0pt}%
\pgfpathmoveto{\pgfqpoint{1.691137in}{2.314010in}}%
\pgfpathlineto{\pgfqpoint{1.694499in}{2.334872in}}%
\pgfpathlineto{\pgfqpoint{1.579264in}{2.341807in}}%
\pgfpathlineto{\pgfqpoint{1.578419in}{2.320810in}}%
\pgfpathlineto{\pgfqpoint{1.691137in}{2.314010in}}%
\pgfpathclose%
\pgfusepath{fill}%
\end{pgfscope}%
\begin{pgfscope}%
\pgfpathrectangle{\pgfqpoint{0.000000in}{0.000000in}}{\pgfqpoint{3.000000in}{3.000000in}}%
\pgfusepath{clip}%
\pgfsetbuttcap%
\pgfsetroundjoin%
\definecolor{currentfill}{rgb}{0.000000,0.000000,0.500000}%
\pgfsetfillcolor{currentfill}%
\pgfsetlinewidth{0.000000pt}%
\definecolor{currentstroke}{rgb}{0.000000,0.000000,0.000000}%
\pgfsetstrokecolor{currentstroke}%
\pgfsetdash{}{0pt}%
\pgfpathmoveto{\pgfqpoint{1.451065in}{1.116452in}}%
\pgfpathlineto{\pgfqpoint{1.441917in}{1.200605in}}%
\pgfpathlineto{\pgfqpoint{1.419262in}{1.191634in}}%
\pgfpathlineto{\pgfqpoint{1.430521in}{1.108243in}}%
\pgfpathlineto{\pgfqpoint{1.451065in}{1.116452in}}%
\pgfpathclose%
\pgfusepath{fill}%
\end{pgfscope}%
\begin{pgfscope}%
\pgfpathrectangle{\pgfqpoint{0.000000in}{0.000000in}}{\pgfqpoint{3.000000in}{3.000000in}}%
\pgfusepath{clip}%
\pgfsetbuttcap%
\pgfsetroundjoin%
\definecolor{currentfill}{rgb}{0.500000,0.000000,0.000000}%
\pgfsetfillcolor{currentfill}%
\pgfsetlinewidth{0.000000pt}%
\definecolor{currentstroke}{rgb}{0.000000,0.000000,0.000000}%
\pgfsetstrokecolor{currentstroke}%
\pgfsetdash{}{0pt}%
\pgfpathmoveto{\pgfqpoint{1.332081in}{2.411684in}}%
\pgfpathlineto{\pgfqpoint{1.327915in}{2.431429in}}%
\pgfpathlineto{\pgfqpoint{1.205628in}{2.411916in}}%
\pgfpathlineto{\pgfqpoint{1.212205in}{2.392509in}}%
\pgfpathlineto{\pgfqpoint{1.332081in}{2.411684in}}%
\pgfpathclose%
\pgfusepath{fill}%
\end{pgfscope}%
\begin{pgfscope}%
\pgfpathrectangle{\pgfqpoint{0.000000in}{0.000000in}}{\pgfqpoint{3.000000in}{3.000000in}}%
\pgfusepath{clip}%
\pgfsetbuttcap%
\pgfsetroundjoin%
\definecolor{currentfill}{rgb}{0.856506,0.000000,0.000000}%
\pgfsetfillcolor{currentfill}%
\pgfsetlinewidth{0.000000pt}%
\definecolor{currentstroke}{rgb}{0.000000,0.000000,0.000000}%
\pgfsetstrokecolor{currentstroke}%
\pgfsetdash{}{0pt}%
\pgfpathmoveto{\pgfqpoint{1.577573in}{2.299571in}}%
\pgfpathlineto{\pgfqpoint{1.578419in}{2.320810in}}%
\pgfpathlineto{\pgfqpoint{1.464876in}{2.319447in}}%
\pgfpathlineto{\pgfqpoint{1.466566in}{2.298236in}}%
\pgfpathlineto{\pgfqpoint{1.577573in}{2.299571in}}%
\pgfpathclose%
\pgfusepath{fill}%
\end{pgfscope}%
\begin{pgfscope}%
\pgfpathrectangle{\pgfqpoint{0.000000in}{0.000000in}}{\pgfqpoint{3.000000in}{3.000000in}}%
\pgfusepath{clip}%
\pgfsetbuttcap%
\pgfsetroundjoin%
\definecolor{currentfill}{rgb}{0.606952,0.000000,0.000000}%
\pgfsetfillcolor{currentfill}%
\pgfsetlinewidth{0.000000pt}%
\definecolor{currentstroke}{rgb}{0.000000,0.000000,0.000000}%
\pgfsetstrokecolor{currentstroke}%
\pgfsetdash{}{0pt}%
\pgfpathmoveto{\pgfqpoint{1.817993in}{2.360210in}}%
\pgfpathlineto{\pgfqpoint{1.823790in}{2.380130in}}%
\pgfpathlineto{\pgfqpoint{1.704563in}{2.396133in}}%
\pgfpathlineto{\pgfqpoint{1.701212in}{2.375923in}}%
\pgfpathlineto{\pgfqpoint{1.817993in}{2.360210in}}%
\pgfpathclose%
\pgfusepath{fill}%
\end{pgfscope}%
\begin{pgfscope}%
\pgfpathrectangle{\pgfqpoint{0.000000in}{0.000000in}}{\pgfqpoint{3.000000in}{3.000000in}}%
\pgfusepath{clip}%
\pgfsetbuttcap%
\pgfsetroundjoin%
\definecolor{currentfill}{rgb}{0.000000,0.000000,0.838681}%
\pgfsetfillcolor{currentfill}%
\pgfsetlinewidth{0.000000pt}%
\definecolor{currentstroke}{rgb}{0.000000,0.000000,0.000000}%
\pgfsetstrokecolor{currentstroke}%
\pgfsetdash{}{0pt}%
\pgfpathmoveto{\pgfqpoint{1.605375in}{1.209502in}}%
\pgfpathlineto{\pgfqpoint{1.611378in}{1.283157in}}%
\pgfpathlineto{\pgfqpoint{1.581618in}{1.287680in}}%
\pgfpathlineto{\pgfqpoint{1.578139in}{1.213674in}}%
\pgfpathlineto{\pgfqpoint{1.605375in}{1.209502in}}%
\pgfpathclose%
\pgfusepath{fill}%
\end{pgfscope}%
\begin{pgfscope}%
\pgfpathrectangle{\pgfqpoint{0.000000in}{0.000000in}}{\pgfqpoint{3.000000in}{3.000000in}}%
\pgfusepath{clip}%
\pgfsetbuttcap%
\pgfsetroundjoin%
\definecolor{currentfill}{rgb}{0.803030,0.000000,0.000000}%
\pgfsetfillcolor{currentfill}%
\pgfsetlinewidth{0.000000pt}%
\definecolor{currentstroke}{rgb}{0.000000,0.000000,0.000000}%
\pgfsetstrokecolor{currentstroke}%
\pgfsetdash{}{0pt}%
\pgfpathmoveto{\pgfqpoint{1.464876in}{2.319447in}}%
\pgfpathlineto{\pgfqpoint{1.463187in}{2.340417in}}%
\pgfpathlineto{\pgfqpoint{1.348793in}{2.330729in}}%
\pgfpathlineto{\pgfqpoint{1.352982in}{2.309947in}}%
\pgfpathlineto{\pgfqpoint{1.464876in}{2.319447in}}%
\pgfpathclose%
\pgfusepath{fill}%
\end{pgfscope}%
\begin{pgfscope}%
\pgfpathrectangle{\pgfqpoint{0.000000in}{0.000000in}}{\pgfqpoint{3.000000in}{3.000000in}}%
\pgfusepath{clip}%
\pgfsetbuttcap%
\pgfsetroundjoin%
\definecolor{currentfill}{rgb}{0.000000,0.000000,0.838681}%
\pgfsetfillcolor{currentfill}%
\pgfsetlinewidth{0.000000pt}%
\definecolor{currentstroke}{rgb}{0.000000,0.000000,0.000000}%
\pgfsetstrokecolor{currentstroke}%
\pgfsetdash{}{0pt}%
\pgfpathmoveto{\pgfqpoint{1.493731in}{1.212530in}}%
\pgfpathlineto{\pgfqpoint{1.489398in}{1.286440in}}%
\pgfpathlineto{\pgfqpoint{1.460125in}{1.281122in}}%
\pgfpathlineto{\pgfqpoint{1.466942in}{1.207625in}}%
\pgfpathlineto{\pgfqpoint{1.493731in}{1.212530in}}%
\pgfpathclose%
\pgfusepath{fill}%
\end{pgfscope}%
\begin{pgfscope}%
\pgfpathrectangle{\pgfqpoint{0.000000in}{0.000000in}}{\pgfqpoint{3.000000in}{3.000000in}}%
\pgfusepath{clip}%
\pgfsetbuttcap%
\pgfsetroundjoin%
\definecolor{currentfill}{rgb}{0.856506,0.000000,0.000000}%
\pgfsetfillcolor{currentfill}%
\pgfsetlinewidth{0.000000pt}%
\definecolor{currentstroke}{rgb}{0.000000,0.000000,0.000000}%
\pgfsetstrokecolor{currentstroke}%
\pgfsetdash{}{0pt}%
\pgfpathmoveto{\pgfqpoint{1.687771in}{2.292907in}}%
\pgfpathlineto{\pgfqpoint{1.691137in}{2.314010in}}%
\pgfpathlineto{\pgfqpoint{1.578419in}{2.320810in}}%
\pgfpathlineto{\pgfqpoint{1.577573in}{2.299571in}}%
\pgfpathlineto{\pgfqpoint{1.687771in}{2.292907in}}%
\pgfpathclose%
\pgfusepath{fill}%
\end{pgfscope}%
\begin{pgfscope}%
\pgfpathrectangle{\pgfqpoint{0.000000in}{0.000000in}}{\pgfqpoint{3.000000in}{3.000000in}}%
\pgfusepath{clip}%
\pgfsetbuttcap%
\pgfsetroundjoin%
\definecolor{currentfill}{rgb}{0.553476,0.000000,0.000000}%
\pgfsetfillcolor{currentfill}%
\pgfsetlinewidth{0.000000pt}%
\definecolor{currentstroke}{rgb}{0.000000,0.000000,0.000000}%
\pgfsetstrokecolor{currentstroke}%
\pgfsetdash{}{0pt}%
\pgfpathmoveto{\pgfqpoint{1.336252in}{2.391750in}}%
\pgfpathlineto{\pgfqpoint{1.332081in}{2.411684in}}%
\pgfpathlineto{\pgfqpoint{1.212205in}{2.392509in}}%
\pgfpathlineto{\pgfqpoint{1.218789in}{2.372916in}}%
\pgfpathlineto{\pgfqpoint{1.336252in}{2.391750in}}%
\pgfpathclose%
\pgfusepath{fill}%
\end{pgfscope}%
\begin{pgfscope}%
\pgfpathrectangle{\pgfqpoint{0.000000in}{0.000000in}}{\pgfqpoint{3.000000in}{3.000000in}}%
\pgfusepath{clip}%
\pgfsetbuttcap%
\pgfsetroundjoin%
\definecolor{currentfill}{rgb}{0.927807,0.015251,0.000000}%
\pgfsetfillcolor{currentfill}%
\pgfsetlinewidth{0.000000pt}%
\definecolor{currentstroke}{rgb}{0.000000,0.000000,0.000000}%
\pgfsetstrokecolor{currentstroke}%
\pgfsetdash{}{0pt}%
\pgfpathmoveto{\pgfqpoint{1.576726in}{2.278078in}}%
\pgfpathlineto{\pgfqpoint{1.577573in}{2.299571in}}%
\pgfpathlineto{\pgfqpoint{1.466566in}{2.298236in}}%
\pgfpathlineto{\pgfqpoint{1.468258in}{2.276770in}}%
\pgfpathlineto{\pgfqpoint{1.576726in}{2.278078in}}%
\pgfpathclose%
\pgfusepath{fill}%
\end{pgfscope}%
\begin{pgfscope}%
\pgfpathrectangle{\pgfqpoint{0.000000in}{0.000000in}}{\pgfqpoint{3.000000in}{3.000000in}}%
\pgfusepath{clip}%
\pgfsetbuttcap%
\pgfsetroundjoin%
\definecolor{currentfill}{rgb}{0.678253,0.000000,0.000000}%
\pgfsetfillcolor{currentfill}%
\pgfsetlinewidth{0.000000pt}%
\definecolor{currentstroke}{rgb}{0.000000,0.000000,0.000000}%
\pgfsetstrokecolor{currentstroke}%
\pgfsetdash{}{0pt}%
\pgfpathmoveto{\pgfqpoint{1.812190in}{2.340085in}}%
\pgfpathlineto{\pgfqpoint{1.817993in}{2.360210in}}%
\pgfpathlineto{\pgfqpoint{1.701212in}{2.375923in}}%
\pgfpathlineto{\pgfqpoint{1.697857in}{2.355506in}}%
\pgfpathlineto{\pgfqpoint{1.812190in}{2.340085in}}%
\pgfpathclose%
\pgfusepath{fill}%
\end{pgfscope}%
\begin{pgfscope}%
\pgfpathrectangle{\pgfqpoint{0.000000in}{0.000000in}}{\pgfqpoint{3.000000in}{3.000000in}}%
\pgfusepath{clip}%
\pgfsetbuttcap%
\pgfsetroundjoin%
\definecolor{currentfill}{rgb}{0.000000,0.000000,0.500000}%
\pgfsetfillcolor{currentfill}%
\pgfsetlinewidth{0.000000pt}%
\definecolor{currentstroke}{rgb}{0.000000,0.000000,0.000000}%
\pgfsetstrokecolor{currentstroke}%
\pgfsetdash{}{0pt}%
\pgfpathmoveto{\pgfqpoint{1.662809in}{1.101874in}}%
\pgfpathlineto{\pgfqpoint{1.675331in}{1.184671in}}%
\pgfpathlineto{\pgfqpoint{1.654566in}{1.194829in}}%
\pgfpathlineto{\pgfqpoint{1.643983in}{1.111167in}}%
\pgfpathlineto{\pgfqpoint{1.662809in}{1.101874in}}%
\pgfpathclose%
\pgfusepath{fill}%
\end{pgfscope}%
\begin{pgfscope}%
\pgfpathrectangle{\pgfqpoint{0.000000in}{0.000000in}}{\pgfqpoint{3.000000in}{3.000000in}}%
\pgfusepath{clip}%
\pgfsetbuttcap%
\pgfsetroundjoin%
\definecolor{currentfill}{rgb}{0.856506,0.000000,0.000000}%
\pgfsetfillcolor{currentfill}%
\pgfsetlinewidth{0.000000pt}%
\definecolor{currentstroke}{rgb}{0.000000,0.000000,0.000000}%
\pgfsetstrokecolor{currentstroke}%
\pgfsetdash{}{0pt}%
\pgfpathmoveto{\pgfqpoint{1.466566in}{2.298236in}}%
\pgfpathlineto{\pgfqpoint{1.464876in}{2.319447in}}%
\pgfpathlineto{\pgfqpoint{1.352982in}{2.309947in}}%
\pgfpathlineto{\pgfqpoint{1.357176in}{2.288926in}}%
\pgfpathlineto{\pgfqpoint{1.466566in}{2.298236in}}%
\pgfpathclose%
\pgfusepath{fill}%
\end{pgfscope}%
\begin{pgfscope}%
\pgfpathrectangle{\pgfqpoint{0.000000in}{0.000000in}}{\pgfqpoint{3.000000in}{3.000000in}}%
\pgfusepath{clip}%
\pgfsetbuttcap%
\pgfsetroundjoin%
\definecolor{currentfill}{rgb}{0.927807,0.015251,0.000000}%
\pgfsetfillcolor{currentfill}%
\pgfsetlinewidth{0.000000pt}%
\definecolor{currentstroke}{rgb}{0.000000,0.000000,0.000000}%
\pgfsetstrokecolor{currentstroke}%
\pgfsetdash{}{0pt}%
\pgfpathmoveto{\pgfqpoint{1.684401in}{2.271550in}}%
\pgfpathlineto{\pgfqpoint{1.687771in}{2.292907in}}%
\pgfpathlineto{\pgfqpoint{1.577573in}{2.299571in}}%
\pgfpathlineto{\pgfqpoint{1.576726in}{2.278078in}}%
\pgfpathlineto{\pgfqpoint{1.684401in}{2.271550in}}%
\pgfpathclose%
\pgfusepath{fill}%
\end{pgfscope}%
\begin{pgfscope}%
\pgfpathrectangle{\pgfqpoint{0.000000in}{0.000000in}}{\pgfqpoint{3.000000in}{3.000000in}}%
\pgfusepath{clip}%
\pgfsetbuttcap%
\pgfsetroundjoin%
\definecolor{currentfill}{rgb}{0.000000,0.000000,0.838681}%
\pgfsetfillcolor{currentfill}%
\pgfsetlinewidth{0.000000pt}%
\definecolor{currentstroke}{rgb}{0.000000,0.000000,0.000000}%
\pgfsetstrokecolor{currentstroke}%
\pgfsetdash{}{0pt}%
\pgfpathmoveto{\pgfqpoint{1.631057in}{1.203171in}}%
\pgfpathlineto{\pgfqpoint{1.639445in}{1.276292in}}%
\pgfpathlineto{\pgfqpoint{1.611378in}{1.283157in}}%
\pgfpathlineto{\pgfqpoint{1.605375in}{1.209502in}}%
\pgfpathlineto{\pgfqpoint{1.631057in}{1.203171in}}%
\pgfpathclose%
\pgfusepath{fill}%
\end{pgfscope}%
\begin{pgfscope}%
\pgfpathrectangle{\pgfqpoint{0.000000in}{0.000000in}}{\pgfqpoint{3.000000in}{3.000000in}}%
\pgfusepath{clip}%
\pgfsetbuttcap%
\pgfsetroundjoin%
\definecolor{currentfill}{rgb}{0.606952,0.000000,0.000000}%
\pgfsetfillcolor{currentfill}%
\pgfsetlinewidth{0.000000pt}%
\definecolor{currentstroke}{rgb}{0.000000,0.000000,0.000000}%
\pgfsetstrokecolor{currentstroke}%
\pgfsetdash{}{0pt}%
\pgfpathmoveto{\pgfqpoint{1.340428in}{2.371620in}}%
\pgfpathlineto{\pgfqpoint{1.336252in}{2.391750in}}%
\pgfpathlineto{\pgfqpoint{1.218789in}{2.372916in}}%
\pgfpathlineto{\pgfqpoint{1.225380in}{2.353128in}}%
\pgfpathlineto{\pgfqpoint{1.340428in}{2.371620in}}%
\pgfpathclose%
\pgfusepath{fill}%
\end{pgfscope}%
\begin{pgfscope}%
\pgfpathrectangle{\pgfqpoint{0.000000in}{0.000000in}}{\pgfqpoint{3.000000in}{3.000000in}}%
\pgfusepath{clip}%
\pgfsetbuttcap%
\pgfsetroundjoin%
\definecolor{currentfill}{rgb}{0.999109,0.073348,0.000000}%
\pgfsetfillcolor{currentfill}%
\pgfsetlinewidth{0.000000pt}%
\definecolor{currentstroke}{rgb}{0.000000,0.000000,0.000000}%
\pgfsetstrokecolor{currentstroke}%
\pgfsetdash{}{0pt}%
\pgfpathmoveto{\pgfqpoint{1.575878in}{2.256316in}}%
\pgfpathlineto{\pgfqpoint{1.576726in}{2.278078in}}%
\pgfpathlineto{\pgfqpoint{1.468258in}{2.276770in}}%
\pgfpathlineto{\pgfqpoint{1.469952in}{2.255035in}}%
\pgfpathlineto{\pgfqpoint{1.575878in}{2.256316in}}%
\pgfpathclose%
\pgfusepath{fill}%
\end{pgfscope}%
\begin{pgfscope}%
\pgfpathrectangle{\pgfqpoint{0.000000in}{0.000000in}}{\pgfqpoint{3.000000in}{3.000000in}}%
\pgfusepath{clip}%
\pgfsetbuttcap%
\pgfsetroundjoin%
\definecolor{currentfill}{rgb}{0.731729,0.000000,0.000000}%
\pgfsetfillcolor{currentfill}%
\pgfsetlinewidth{0.000000pt}%
\definecolor{currentstroke}{rgb}{0.000000,0.000000,0.000000}%
\pgfsetstrokecolor{currentstroke}%
\pgfsetdash{}{0pt}%
\pgfpathmoveto{\pgfqpoint{1.806380in}{2.319745in}}%
\pgfpathlineto{\pgfqpoint{1.812190in}{2.340085in}}%
\pgfpathlineto{\pgfqpoint{1.697857in}{2.355506in}}%
\pgfpathlineto{\pgfqpoint{1.694499in}{2.334872in}}%
\pgfpathlineto{\pgfqpoint{1.806380in}{2.319745in}}%
\pgfpathclose%
\pgfusepath{fill}%
\end{pgfscope}%
\begin{pgfscope}%
\pgfpathrectangle{\pgfqpoint{0.000000in}{0.000000in}}{\pgfqpoint{3.000000in}{3.000000in}}%
\pgfusepath{clip}%
\pgfsetbuttcap%
\pgfsetroundjoin%
\definecolor{currentfill}{rgb}{0.000000,0.000000,0.500000}%
\pgfsetfillcolor{currentfill}%
\pgfsetlinewidth{0.000000pt}%
\definecolor{currentstroke}{rgb}{0.000000,0.000000,0.000000}%
\pgfsetstrokecolor{currentstroke}%
\pgfsetdash{}{0pt}%
\pgfpathmoveto{\pgfqpoint{1.430521in}{1.108243in}}%
\pgfpathlineto{\pgfqpoint{1.419262in}{1.191634in}}%
\pgfpathlineto{\pgfqpoint{1.399529in}{1.180922in}}%
\pgfpathlineto{\pgfqpoint{1.412633in}{1.098444in}}%
\pgfpathlineto{\pgfqpoint{1.430521in}{1.108243in}}%
\pgfpathclose%
\pgfusepath{fill}%
\end{pgfscope}%
\begin{pgfscope}%
\pgfpathrectangle{\pgfqpoint{0.000000in}{0.000000in}}{\pgfqpoint{3.000000in}{3.000000in}}%
\pgfusepath{clip}%
\pgfsetbuttcap%
\pgfsetroundjoin%
\definecolor{currentfill}{rgb}{0.000000,0.000000,0.838681}%
\pgfsetfillcolor{currentfill}%
\pgfsetlinewidth{0.000000pt}%
\definecolor{currentstroke}{rgb}{0.000000,0.000000,0.000000}%
\pgfsetstrokecolor{currentstroke}%
\pgfsetdash{}{0pt}%
\pgfpathmoveto{\pgfqpoint{1.466942in}{1.207625in}}%
\pgfpathlineto{\pgfqpoint{1.460125in}{1.281122in}}%
\pgfpathlineto{\pgfqpoint{1.432775in}{1.273509in}}%
\pgfpathlineto{\pgfqpoint{1.441917in}{1.200605in}}%
\pgfpathlineto{\pgfqpoint{1.466942in}{1.207625in}}%
\pgfpathclose%
\pgfusepath{fill}%
\end{pgfscope}%
\begin{pgfscope}%
\pgfpathrectangle{\pgfqpoint{0.000000in}{0.000000in}}{\pgfqpoint{3.000000in}{3.000000in}}%
\pgfusepath{clip}%
\pgfsetbuttcap%
\pgfsetroundjoin%
\definecolor{currentfill}{rgb}{0.927807,0.015251,0.000000}%
\pgfsetfillcolor{currentfill}%
\pgfsetlinewidth{0.000000pt}%
\definecolor{currentstroke}{rgb}{0.000000,0.000000,0.000000}%
\pgfsetstrokecolor{currentstroke}%
\pgfsetdash{}{0pt}%
\pgfpathmoveto{\pgfqpoint{1.468258in}{2.276770in}}%
\pgfpathlineto{\pgfqpoint{1.466566in}{2.298236in}}%
\pgfpathlineto{\pgfqpoint{1.357176in}{2.288926in}}%
\pgfpathlineto{\pgfqpoint{1.361375in}{2.267650in}}%
\pgfpathlineto{\pgfqpoint{1.468258in}{2.276770in}}%
\pgfpathclose%
\pgfusepath{fill}%
\end{pgfscope}%
\begin{pgfscope}%
\pgfpathrectangle{\pgfqpoint{0.000000in}{0.000000in}}{\pgfqpoint{3.000000in}{3.000000in}}%
\pgfusepath{clip}%
\pgfsetbuttcap%
\pgfsetroundjoin%
\definecolor{currentfill}{rgb}{0.000000,0.064706,1.000000}%
\pgfsetfillcolor{currentfill}%
\pgfsetlinewidth{0.000000pt}%
\definecolor{currentstroke}{rgb}{0.000000,0.000000,0.000000}%
\pgfsetstrokecolor{currentstroke}%
\pgfsetdash{}{0pt}%
\pgfpathmoveto{\pgfqpoint{1.550879in}{1.289758in}}%
\pgfpathlineto{\pgfqpoint{1.551753in}{1.355376in}}%
\pgfpathlineto{\pgfqpoint{1.518145in}{1.354928in}}%
\pgfpathlineto{\pgfqpoint{1.519892in}{1.289341in}}%
\pgfpathlineto{\pgfqpoint{1.550879in}{1.289758in}}%
\pgfpathclose%
\pgfusepath{fill}%
\end{pgfscope}%
\begin{pgfscope}%
\pgfpathrectangle{\pgfqpoint{0.000000in}{0.000000in}}{\pgfqpoint{3.000000in}{3.000000in}}%
\pgfusepath{clip}%
\pgfsetbuttcap%
\pgfsetroundjoin%
\definecolor{currentfill}{rgb}{0.999109,0.073348,0.000000}%
\pgfsetfillcolor{currentfill}%
\pgfsetlinewidth{0.000000pt}%
\definecolor{currentstroke}{rgb}{0.000000,0.000000,0.000000}%
\pgfsetstrokecolor{currentstroke}%
\pgfsetdash{}{0pt}%
\pgfpathmoveto{\pgfqpoint{1.681028in}{2.249926in}}%
\pgfpathlineto{\pgfqpoint{1.684401in}{2.271550in}}%
\pgfpathlineto{\pgfqpoint{1.576726in}{2.278078in}}%
\pgfpathlineto{\pgfqpoint{1.575878in}{2.256316in}}%
\pgfpathlineto{\pgfqpoint{1.681028in}{2.249926in}}%
\pgfpathclose%
\pgfusepath{fill}%
\end{pgfscope}%
\begin{pgfscope}%
\pgfpathrectangle{\pgfqpoint{0.000000in}{0.000000in}}{\pgfqpoint{3.000000in}{3.000000in}}%
\pgfusepath{clip}%
\pgfsetbuttcap%
\pgfsetroundjoin%
\definecolor{currentfill}{rgb}{0.678253,0.000000,0.000000}%
\pgfsetfillcolor{currentfill}%
\pgfsetlinewidth{0.000000pt}%
\definecolor{currentstroke}{rgb}{0.000000,0.000000,0.000000}%
\pgfsetstrokecolor{currentstroke}%
\pgfsetdash{}{0pt}%
\pgfpathmoveto{\pgfqpoint{1.344608in}{2.351283in}}%
\pgfpathlineto{\pgfqpoint{1.340428in}{2.371620in}}%
\pgfpathlineto{\pgfqpoint{1.225380in}{2.353128in}}%
\pgfpathlineto{\pgfqpoint{1.231977in}{2.333135in}}%
\pgfpathlineto{\pgfqpoint{1.344608in}{2.351283in}}%
\pgfpathclose%
\pgfusepath{fill}%
\end{pgfscope}%
\begin{pgfscope}%
\pgfpathrectangle{\pgfqpoint{0.000000in}{0.000000in}}{\pgfqpoint{3.000000in}{3.000000in}}%
\pgfusepath{clip}%
\pgfsetbuttcap%
\pgfsetroundjoin%
\definecolor{currentfill}{rgb}{0.000000,0.064706,1.000000}%
\pgfsetfillcolor{currentfill}%
\pgfsetlinewidth{0.000000pt}%
\definecolor{currentstroke}{rgb}{0.000000,0.000000,0.000000}%
\pgfsetstrokecolor{currentstroke}%
\pgfsetdash{}{0pt}%
\pgfpathmoveto{\pgfqpoint{1.581618in}{1.287680in}}%
\pgfpathlineto{\pgfqpoint{1.585095in}{1.353138in}}%
\pgfpathlineto{\pgfqpoint{1.551753in}{1.355376in}}%
\pgfpathlineto{\pgfqpoint{1.550879in}{1.289758in}}%
\pgfpathlineto{\pgfqpoint{1.581618in}{1.287680in}}%
\pgfpathclose%
\pgfusepath{fill}%
\end{pgfscope}%
\begin{pgfscope}%
\pgfpathrectangle{\pgfqpoint{0.000000in}{0.000000in}}{\pgfqpoint{3.000000in}{3.000000in}}%
\pgfusepath{clip}%
\pgfsetbuttcap%
\pgfsetroundjoin%
\definecolor{currentfill}{rgb}{1.000000,0.116921,0.000000}%
\pgfsetfillcolor{currentfill}%
\pgfsetlinewidth{0.000000pt}%
\definecolor{currentstroke}{rgb}{0.000000,0.000000,0.000000}%
\pgfsetstrokecolor{currentstroke}%
\pgfsetdash{}{0pt}%
\pgfpathmoveto{\pgfqpoint{1.575029in}{2.234270in}}%
\pgfpathlineto{\pgfqpoint{1.575878in}{2.256316in}}%
\pgfpathlineto{\pgfqpoint{1.469952in}{2.255035in}}%
\pgfpathlineto{\pgfqpoint{1.471648in}{2.233017in}}%
\pgfpathlineto{\pgfqpoint{1.575029in}{2.234270in}}%
\pgfpathclose%
\pgfusepath{fill}%
\end{pgfscope}%
\begin{pgfscope}%
\pgfpathrectangle{\pgfqpoint{0.000000in}{0.000000in}}{\pgfqpoint{3.000000in}{3.000000in}}%
\pgfusepath{clip}%
\pgfsetbuttcap%
\pgfsetroundjoin%
\definecolor{currentfill}{rgb}{0.803030,0.000000,0.000000}%
\pgfsetfillcolor{currentfill}%
\pgfsetlinewidth{0.000000pt}%
\definecolor{currentstroke}{rgb}{0.000000,0.000000,0.000000}%
\pgfsetstrokecolor{currentstroke}%
\pgfsetdash{}{0pt}%
\pgfpathmoveto{\pgfqpoint{1.800565in}{2.299177in}}%
\pgfpathlineto{\pgfqpoint{1.806380in}{2.319745in}}%
\pgfpathlineto{\pgfqpoint{1.694499in}{2.334872in}}%
\pgfpathlineto{\pgfqpoint{1.691137in}{2.314010in}}%
\pgfpathlineto{\pgfqpoint{1.800565in}{2.299177in}}%
\pgfpathclose%
\pgfusepath{fill}%
\end{pgfscope}%
\begin{pgfscope}%
\pgfpathrectangle{\pgfqpoint{0.000000in}{0.000000in}}{\pgfqpoint{3.000000in}{3.000000in}}%
\pgfusepath{clip}%
\pgfsetbuttcap%
\pgfsetroundjoin%
\definecolor{currentfill}{rgb}{0.000000,0.064706,1.000000}%
\pgfsetfillcolor{currentfill}%
\pgfsetlinewidth{0.000000pt}%
\definecolor{currentstroke}{rgb}{0.000000,0.000000,0.000000}%
\pgfsetstrokecolor{currentstroke}%
\pgfsetdash{}{0pt}%
\pgfpathmoveto{\pgfqpoint{1.519892in}{1.289341in}}%
\pgfpathlineto{\pgfqpoint{1.518145in}{1.354928in}}%
\pgfpathlineto{\pgfqpoint{1.485069in}{1.351802in}}%
\pgfpathlineto{\pgfqpoint{1.489398in}{1.286440in}}%
\pgfpathlineto{\pgfqpoint{1.519892in}{1.289341in}}%
\pgfpathclose%
\pgfusepath{fill}%
\end{pgfscope}%
\begin{pgfscope}%
\pgfpathrectangle{\pgfqpoint{0.000000in}{0.000000in}}{\pgfqpoint{3.000000in}{3.000000in}}%
\pgfusepath{clip}%
\pgfsetbuttcap%
\pgfsetroundjoin%
\definecolor{currentfill}{rgb}{0.999109,0.073348,0.000000}%
\pgfsetfillcolor{currentfill}%
\pgfsetlinewidth{0.000000pt}%
\definecolor{currentstroke}{rgb}{0.000000,0.000000,0.000000}%
\pgfsetstrokecolor{currentstroke}%
\pgfsetdash{}{0pt}%
\pgfpathmoveto{\pgfqpoint{1.469952in}{2.255035in}}%
\pgfpathlineto{\pgfqpoint{1.468258in}{2.276770in}}%
\pgfpathlineto{\pgfqpoint{1.361375in}{2.267650in}}%
\pgfpathlineto{\pgfqpoint{1.365578in}{2.246108in}}%
\pgfpathlineto{\pgfqpoint{1.469952in}{2.255035in}}%
\pgfpathclose%
\pgfusepath{fill}%
\end{pgfscope}%
\begin{pgfscope}%
\pgfpathrectangle{\pgfqpoint{0.000000in}{0.000000in}}{\pgfqpoint{3.000000in}{3.000000in}}%
\pgfusepath{clip}%
\pgfsetbuttcap%
\pgfsetroundjoin%
\definecolor{currentfill}{rgb}{1.000000,0.116921,0.000000}%
\pgfsetfillcolor{currentfill}%
\pgfsetlinewidth{0.000000pt}%
\definecolor{currentstroke}{rgb}{0.000000,0.000000,0.000000}%
\pgfsetstrokecolor{currentstroke}%
\pgfsetdash{}{0pt}%
\pgfpathmoveto{\pgfqpoint{1.677650in}{2.228017in}}%
\pgfpathlineto{\pgfqpoint{1.681028in}{2.249926in}}%
\pgfpathlineto{\pgfqpoint{1.575878in}{2.256316in}}%
\pgfpathlineto{\pgfqpoint{1.575029in}{2.234270in}}%
\pgfpathlineto{\pgfqpoint{1.677650in}{2.228017in}}%
\pgfpathclose%
\pgfusepath{fill}%
\end{pgfscope}%
\begin{pgfscope}%
\pgfpathrectangle{\pgfqpoint{0.000000in}{0.000000in}}{\pgfqpoint{3.000000in}{3.000000in}}%
\pgfusepath{clip}%
\pgfsetbuttcap%
\pgfsetroundjoin%
\definecolor{currentfill}{rgb}{0.731729,0.000000,0.000000}%
\pgfsetfillcolor{currentfill}%
\pgfsetlinewidth{0.000000pt}%
\definecolor{currentstroke}{rgb}{0.000000,0.000000,0.000000}%
\pgfsetstrokecolor{currentstroke}%
\pgfsetdash{}{0pt}%
\pgfpathmoveto{\pgfqpoint{1.348793in}{2.330729in}}%
\pgfpathlineto{\pgfqpoint{1.344608in}{2.351283in}}%
\pgfpathlineto{\pgfqpoint{1.231977in}{2.333135in}}%
\pgfpathlineto{\pgfqpoint{1.238581in}{2.312927in}}%
\pgfpathlineto{\pgfqpoint{1.348793in}{2.330729in}}%
\pgfpathclose%
\pgfusepath{fill}%
\end{pgfscope}%
\begin{pgfscope}%
\pgfpathrectangle{\pgfqpoint{0.000000in}{0.000000in}}{\pgfqpoint{3.000000in}{3.000000in}}%
\pgfusepath{clip}%
\pgfsetbuttcap%
\pgfsetroundjoin%
\definecolor{currentfill}{rgb}{0.000000,0.000000,0.838681}%
\pgfsetfillcolor{currentfill}%
\pgfsetlinewidth{0.000000pt}%
\definecolor{currentstroke}{rgb}{0.000000,0.000000,0.000000}%
\pgfsetstrokecolor{currentstroke}%
\pgfsetdash{}{0pt}%
\pgfpathmoveto{\pgfqpoint{1.654566in}{1.194829in}}%
\pgfpathlineto{\pgfqpoint{1.665142in}{1.267244in}}%
\pgfpathlineto{\pgfqpoint{1.639445in}{1.276292in}}%
\pgfpathlineto{\pgfqpoint{1.631057in}{1.203171in}}%
\pgfpathlineto{\pgfqpoint{1.654566in}{1.194829in}}%
\pgfpathclose%
\pgfusepath{fill}%
\end{pgfscope}%
\begin{pgfscope}%
\pgfpathrectangle{\pgfqpoint{0.000000in}{0.000000in}}{\pgfqpoint{3.000000in}{3.000000in}}%
\pgfusepath{clip}%
\pgfsetbuttcap%
\pgfsetroundjoin%
\definecolor{currentfill}{rgb}{0.000000,0.064706,1.000000}%
\pgfsetfillcolor{currentfill}%
\pgfsetlinewidth{0.000000pt}%
\definecolor{currentstroke}{rgb}{0.000000,0.000000,0.000000}%
\pgfsetstrokecolor{currentstroke}%
\pgfsetdash{}{0pt}%
\pgfpathmoveto{\pgfqpoint{1.611378in}{1.283157in}}%
\pgfpathlineto{\pgfqpoint{1.617376in}{1.348265in}}%
\pgfpathlineto{\pgfqpoint{1.585095in}{1.353138in}}%
\pgfpathlineto{\pgfqpoint{1.581618in}{1.287680in}}%
\pgfpathlineto{\pgfqpoint{1.611378in}{1.283157in}}%
\pgfpathclose%
\pgfusepath{fill}%
\end{pgfscope}%
\begin{pgfscope}%
\pgfpathrectangle{\pgfqpoint{0.000000in}{0.000000in}}{\pgfqpoint{3.000000in}{3.000000in}}%
\pgfusepath{clip}%
\pgfsetbuttcap%
\pgfsetroundjoin%
\definecolor{currentfill}{rgb}{1.000000,0.175018,0.000000}%
\pgfsetfillcolor{currentfill}%
\pgfsetlinewidth{0.000000pt}%
\definecolor{currentstroke}{rgb}{0.000000,0.000000,0.000000}%
\pgfsetstrokecolor{currentstroke}%
\pgfsetdash{}{0pt}%
\pgfpathmoveto{\pgfqpoint{1.574179in}{2.211922in}}%
\pgfpathlineto{\pgfqpoint{1.575029in}{2.234270in}}%
\pgfpathlineto{\pgfqpoint{1.471648in}{2.233017in}}%
\pgfpathlineto{\pgfqpoint{1.473346in}{2.210696in}}%
\pgfpathlineto{\pgfqpoint{1.574179in}{2.211922in}}%
\pgfpathclose%
\pgfusepath{fill}%
\end{pgfscope}%
\begin{pgfscope}%
\pgfpathrectangle{\pgfqpoint{0.000000in}{0.000000in}}{\pgfqpoint{3.000000in}{3.000000in}}%
\pgfusepath{clip}%
\pgfsetbuttcap%
\pgfsetroundjoin%
\definecolor{currentfill}{rgb}{0.856506,0.000000,0.000000}%
\pgfsetfillcolor{currentfill}%
\pgfsetlinewidth{0.000000pt}%
\definecolor{currentstroke}{rgb}{0.000000,0.000000,0.000000}%
\pgfsetstrokecolor{currentstroke}%
\pgfsetdash{}{0pt}%
\pgfpathmoveto{\pgfqpoint{1.794743in}{2.278371in}}%
\pgfpathlineto{\pgfqpoint{1.800565in}{2.299177in}}%
\pgfpathlineto{\pgfqpoint{1.691137in}{2.314010in}}%
\pgfpathlineto{\pgfqpoint{1.687771in}{2.292907in}}%
\pgfpathlineto{\pgfqpoint{1.794743in}{2.278371in}}%
\pgfpathclose%
\pgfusepath{fill}%
\end{pgfscope}%
\begin{pgfscope}%
\pgfpathrectangle{\pgfqpoint{0.000000in}{0.000000in}}{\pgfqpoint{3.000000in}{3.000000in}}%
\pgfusepath{clip}%
\pgfsetbuttcap%
\pgfsetroundjoin%
\definecolor{currentfill}{rgb}{0.000000,0.000000,0.500000}%
\pgfsetfillcolor{currentfill}%
\pgfsetlinewidth{0.000000pt}%
\definecolor{currentstroke}{rgb}{0.000000,0.000000,0.000000}%
\pgfsetstrokecolor{currentstroke}%
\pgfsetdash{}{0pt}%
\pgfpathmoveto{\pgfqpoint{1.678677in}{1.091141in}}%
\pgfpathlineto{\pgfqpoint{1.692841in}{1.172938in}}%
\pgfpathlineto{\pgfqpoint{1.675331in}{1.184671in}}%
\pgfpathlineto{\pgfqpoint{1.662809in}{1.101874in}}%
\pgfpathlineto{\pgfqpoint{1.678677in}{1.091141in}}%
\pgfpathclose%
\pgfusepath{fill}%
\end{pgfscope}%
\begin{pgfscope}%
\pgfpathrectangle{\pgfqpoint{0.000000in}{0.000000in}}{\pgfqpoint{3.000000in}{3.000000in}}%
\pgfusepath{clip}%
\pgfsetbuttcap%
\pgfsetroundjoin%
\definecolor{currentfill}{rgb}{0.500000,0.000000,0.000000}%
\pgfsetfillcolor{currentfill}%
\pgfsetlinewidth{0.000000pt}%
\definecolor{currentstroke}{rgb}{0.000000,0.000000,0.000000}%
\pgfsetstrokecolor{currentstroke}%
\pgfsetdash{}{0pt}%
\pgfpathmoveto{\pgfqpoint{1.944991in}{2.375038in}}%
\pgfpathlineto{\pgfqpoint{1.953109in}{2.394135in}}%
\pgfpathlineto{\pgfqpoint{1.835366in}{2.419390in}}%
\pgfpathlineto{\pgfqpoint{1.829581in}{2.399853in}}%
\pgfpathlineto{\pgfqpoint{1.944991in}{2.375038in}}%
\pgfpathclose%
\pgfusepath{fill}%
\end{pgfscope}%
\begin{pgfscope}%
\pgfpathrectangle{\pgfqpoint{0.000000in}{0.000000in}}{\pgfqpoint{3.000000in}{3.000000in}}%
\pgfusepath{clip}%
\pgfsetbuttcap%
\pgfsetroundjoin%
\definecolor{currentfill}{rgb}{0.000000,0.064706,1.000000}%
\pgfsetfillcolor{currentfill}%
\pgfsetlinewidth{0.000000pt}%
\definecolor{currentstroke}{rgb}{0.000000,0.000000,0.000000}%
\pgfsetstrokecolor{currentstroke}%
\pgfsetdash{}{0pt}%
\pgfpathmoveto{\pgfqpoint{1.489398in}{1.286440in}}%
\pgfpathlineto{\pgfqpoint{1.485069in}{1.351802in}}%
\pgfpathlineto{\pgfqpoint{1.453315in}{1.346073in}}%
\pgfpathlineto{\pgfqpoint{1.460125in}{1.281122in}}%
\pgfpathlineto{\pgfqpoint{1.489398in}{1.286440in}}%
\pgfpathclose%
\pgfusepath{fill}%
\end{pgfscope}%
\begin{pgfscope}%
\pgfpathrectangle{\pgfqpoint{0.000000in}{0.000000in}}{\pgfqpoint{3.000000in}{3.000000in}}%
\pgfusepath{clip}%
\pgfsetbuttcap%
\pgfsetroundjoin%
\definecolor{currentfill}{rgb}{1.000000,0.116921,0.000000}%
\pgfsetfillcolor{currentfill}%
\pgfsetlinewidth{0.000000pt}%
\definecolor{currentstroke}{rgb}{0.000000,0.000000,0.000000}%
\pgfsetstrokecolor{currentstroke}%
\pgfsetdash{}{0pt}%
\pgfpathmoveto{\pgfqpoint{1.471648in}{2.233017in}}%
\pgfpathlineto{\pgfqpoint{1.469952in}{2.255035in}}%
\pgfpathlineto{\pgfqpoint{1.365578in}{2.246108in}}%
\pgfpathlineto{\pgfqpoint{1.369786in}{2.224281in}}%
\pgfpathlineto{\pgfqpoint{1.471648in}{2.233017in}}%
\pgfpathclose%
\pgfusepath{fill}%
\end{pgfscope}%
\begin{pgfscope}%
\pgfpathrectangle{\pgfqpoint{0.000000in}{0.000000in}}{\pgfqpoint{3.000000in}{3.000000in}}%
\pgfusepath{clip}%
\pgfsetbuttcap%
\pgfsetroundjoin%
\definecolor{currentfill}{rgb}{0.000000,0.000000,0.838681}%
\pgfsetfillcolor{currentfill}%
\pgfsetlinewidth{0.000000pt}%
\definecolor{currentstroke}{rgb}{0.000000,0.000000,0.000000}%
\pgfsetstrokecolor{currentstroke}%
\pgfsetdash{}{0pt}%
\pgfpathmoveto{\pgfqpoint{1.441917in}{1.200605in}}%
\pgfpathlineto{\pgfqpoint{1.432775in}{1.273509in}}%
\pgfpathlineto{\pgfqpoint{1.408009in}{1.263779in}}%
\pgfpathlineto{\pgfqpoint{1.419262in}{1.191634in}}%
\pgfpathlineto{\pgfqpoint{1.441917in}{1.200605in}}%
\pgfpathclose%
\pgfusepath{fill}%
\end{pgfscope}%
\begin{pgfscope}%
\pgfpathrectangle{\pgfqpoint{0.000000in}{0.000000in}}{\pgfqpoint{3.000000in}{3.000000in}}%
\pgfusepath{clip}%
\pgfsetbuttcap%
\pgfsetroundjoin%
\definecolor{currentfill}{rgb}{1.000000,0.175018,0.000000}%
\pgfsetfillcolor{currentfill}%
\pgfsetlinewidth{0.000000pt}%
\definecolor{currentstroke}{rgb}{0.000000,0.000000,0.000000}%
\pgfsetstrokecolor{currentstroke}%
\pgfsetdash{}{0pt}%
\pgfpathmoveto{\pgfqpoint{1.674269in}{2.205807in}}%
\pgfpathlineto{\pgfqpoint{1.677650in}{2.228017in}}%
\pgfpathlineto{\pgfqpoint{1.575029in}{2.234270in}}%
\pgfpathlineto{\pgfqpoint{1.574179in}{2.211922in}}%
\pgfpathlineto{\pgfqpoint{1.674269in}{2.205807in}}%
\pgfpathclose%
\pgfusepath{fill}%
\end{pgfscope}%
\begin{pgfscope}%
\pgfpathrectangle{\pgfqpoint{0.000000in}{0.000000in}}{\pgfqpoint{3.000000in}{3.000000in}}%
\pgfusepath{clip}%
\pgfsetbuttcap%
\pgfsetroundjoin%
\definecolor{currentfill}{rgb}{0.803030,0.000000,0.000000}%
\pgfsetfillcolor{currentfill}%
\pgfsetlinewidth{0.000000pt}%
\definecolor{currentstroke}{rgb}{0.000000,0.000000,0.000000}%
\pgfsetstrokecolor{currentstroke}%
\pgfsetdash{}{0pt}%
\pgfpathmoveto{\pgfqpoint{1.352982in}{2.309947in}}%
\pgfpathlineto{\pgfqpoint{1.348793in}{2.330729in}}%
\pgfpathlineto{\pgfqpoint{1.238581in}{2.312927in}}%
\pgfpathlineto{\pgfqpoint{1.245192in}{2.292493in}}%
\pgfpathlineto{\pgfqpoint{1.352982in}{2.309947in}}%
\pgfpathclose%
\pgfusepath{fill}%
\end{pgfscope}%
\begin{pgfscope}%
\pgfpathrectangle{\pgfqpoint{0.000000in}{0.000000in}}{\pgfqpoint{3.000000in}{3.000000in}}%
\pgfusepath{clip}%
\pgfsetbuttcap%
\pgfsetroundjoin%
\definecolor{currentfill}{rgb}{1.000000,0.233115,0.000000}%
\pgfsetfillcolor{currentfill}%
\pgfsetlinewidth{0.000000pt}%
\definecolor{currentstroke}{rgb}{0.000000,0.000000,0.000000}%
\pgfsetstrokecolor{currentstroke}%
\pgfsetdash{}{0pt}%
\pgfpathmoveto{\pgfqpoint{1.573328in}{2.189253in}}%
\pgfpathlineto{\pgfqpoint{1.574179in}{2.211922in}}%
\pgfpathlineto{\pgfqpoint{1.473346in}{2.210696in}}%
\pgfpathlineto{\pgfqpoint{1.475045in}{2.188056in}}%
\pgfpathlineto{\pgfqpoint{1.573328in}{2.189253in}}%
\pgfpathclose%
\pgfusepath{fill}%
\end{pgfscope}%
\begin{pgfscope}%
\pgfpathrectangle{\pgfqpoint{0.000000in}{0.000000in}}{\pgfqpoint{3.000000in}{3.000000in}}%
\pgfusepath{clip}%
\pgfsetbuttcap%
\pgfsetroundjoin%
\definecolor{currentfill}{rgb}{0.927807,0.015251,0.000000}%
\pgfsetfillcolor{currentfill}%
\pgfsetlinewidth{0.000000pt}%
\definecolor{currentstroke}{rgb}{0.000000,0.000000,0.000000}%
\pgfsetstrokecolor{currentstroke}%
\pgfsetdash{}{0pt}%
\pgfpathmoveto{\pgfqpoint{1.788915in}{2.257312in}}%
\pgfpathlineto{\pgfqpoint{1.794743in}{2.278371in}}%
\pgfpathlineto{\pgfqpoint{1.687771in}{2.292907in}}%
\pgfpathlineto{\pgfqpoint{1.684401in}{2.271550in}}%
\pgfpathlineto{\pgfqpoint{1.788915in}{2.257312in}}%
\pgfpathclose%
\pgfusepath{fill}%
\end{pgfscope}%
\begin{pgfscope}%
\pgfpathrectangle{\pgfqpoint{0.000000in}{0.000000in}}{\pgfqpoint{3.000000in}{3.000000in}}%
\pgfusepath{clip}%
\pgfsetbuttcap%
\pgfsetroundjoin%
\definecolor{currentfill}{rgb}{0.000000,0.000000,0.500000}%
\pgfsetfillcolor{currentfill}%
\pgfsetlinewidth{0.000000pt}%
\definecolor{currentstroke}{rgb}{0.000000,0.000000,0.000000}%
\pgfsetstrokecolor{currentstroke}%
\pgfsetdash{}{0pt}%
\pgfpathmoveto{\pgfqpoint{1.412633in}{1.098444in}}%
\pgfpathlineto{\pgfqpoint{1.399529in}{1.180922in}}%
\pgfpathlineto{\pgfqpoint{1.383204in}{1.168724in}}%
\pgfpathlineto{\pgfqpoint{1.397843in}{1.087287in}}%
\pgfpathlineto{\pgfqpoint{1.412633in}{1.098444in}}%
\pgfpathclose%
\pgfusepath{fill}%
\end{pgfscope}%
\begin{pgfscope}%
\pgfpathrectangle{\pgfqpoint{0.000000in}{0.000000in}}{\pgfqpoint{3.000000in}{3.000000in}}%
\pgfusepath{clip}%
\pgfsetbuttcap%
\pgfsetroundjoin%
\definecolor{currentfill}{rgb}{0.553476,0.000000,0.000000}%
\pgfsetfillcolor{currentfill}%
\pgfsetlinewidth{0.000000pt}%
\definecolor{currentstroke}{rgb}{0.000000,0.000000,0.000000}%
\pgfsetstrokecolor{currentstroke}%
\pgfsetdash{}{0pt}%
\pgfpathmoveto{\pgfqpoint{1.936864in}{2.355757in}}%
\pgfpathlineto{\pgfqpoint{1.944991in}{2.375038in}}%
\pgfpathlineto{\pgfqpoint{1.829581in}{2.399853in}}%
\pgfpathlineto{\pgfqpoint{1.823790in}{2.380130in}}%
\pgfpathlineto{\pgfqpoint{1.936864in}{2.355757in}}%
\pgfpathclose%
\pgfusepath{fill}%
\end{pgfscope}%
\begin{pgfscope}%
\pgfpathrectangle{\pgfqpoint{0.000000in}{0.000000in}}{\pgfqpoint{3.000000in}{3.000000in}}%
\pgfusepath{clip}%
\pgfsetbuttcap%
\pgfsetroundjoin%
\definecolor{currentfill}{rgb}{1.000000,0.175018,0.000000}%
\pgfsetfillcolor{currentfill}%
\pgfsetlinewidth{0.000000pt}%
\definecolor{currentstroke}{rgb}{0.000000,0.000000,0.000000}%
\pgfsetstrokecolor{currentstroke}%
\pgfsetdash{}{0pt}%
\pgfpathmoveto{\pgfqpoint{1.473346in}{2.210696in}}%
\pgfpathlineto{\pgfqpoint{1.471648in}{2.233017in}}%
\pgfpathlineto{\pgfqpoint{1.369786in}{2.224281in}}%
\pgfpathlineto{\pgfqpoint{1.373998in}{2.202155in}}%
\pgfpathlineto{\pgfqpoint{1.473346in}{2.210696in}}%
\pgfpathclose%
\pgfusepath{fill}%
\end{pgfscope}%
\begin{pgfscope}%
\pgfpathrectangle{\pgfqpoint{0.000000in}{0.000000in}}{\pgfqpoint{3.000000in}{3.000000in}}%
\pgfusepath{clip}%
\pgfsetbuttcap%
\pgfsetroundjoin%
\definecolor{currentfill}{rgb}{0.000000,0.300000,1.000000}%
\pgfsetfillcolor{currentfill}%
\pgfsetlinewidth{0.000000pt}%
\definecolor{currentstroke}{rgb}{0.000000,0.000000,0.000000}%
\pgfsetstrokecolor{currentstroke}%
\pgfsetdash{}{0pt}%
\pgfpathmoveto{\pgfqpoint{1.551753in}{1.355376in}}%
\pgfpathlineto{\pgfqpoint{1.552627in}{1.414333in}}%
\pgfpathlineto{\pgfqpoint{1.516400in}{1.413852in}}%
\pgfpathlineto{\pgfqpoint{1.518145in}{1.354928in}}%
\pgfpathlineto{\pgfqpoint{1.551753in}{1.355376in}}%
\pgfpathclose%
\pgfusepath{fill}%
\end{pgfscope}%
\begin{pgfscope}%
\pgfpathrectangle{\pgfqpoint{0.000000in}{0.000000in}}{\pgfqpoint{3.000000in}{3.000000in}}%
\pgfusepath{clip}%
\pgfsetbuttcap%
\pgfsetroundjoin%
\definecolor{currentfill}{rgb}{0.000000,0.064706,1.000000}%
\pgfsetfillcolor{currentfill}%
\pgfsetlinewidth{0.000000pt}%
\definecolor{currentstroke}{rgb}{0.000000,0.000000,0.000000}%
\pgfsetstrokecolor{currentstroke}%
\pgfsetdash{}{0pt}%
\pgfpathmoveto{\pgfqpoint{1.639445in}{1.276292in}}%
\pgfpathlineto{\pgfqpoint{1.647825in}{1.340869in}}%
\pgfpathlineto{\pgfqpoint{1.617376in}{1.348265in}}%
\pgfpathlineto{\pgfqpoint{1.611378in}{1.283157in}}%
\pgfpathlineto{\pgfqpoint{1.639445in}{1.276292in}}%
\pgfpathclose%
\pgfusepath{fill}%
\end{pgfscope}%
\begin{pgfscope}%
\pgfpathrectangle{\pgfqpoint{0.000000in}{0.000000in}}{\pgfqpoint{3.000000in}{3.000000in}}%
\pgfusepath{clip}%
\pgfsetbuttcap%
\pgfsetroundjoin%
\definecolor{currentfill}{rgb}{1.000000,0.233115,0.000000}%
\pgfsetfillcolor{currentfill}%
\pgfsetlinewidth{0.000000pt}%
\definecolor{currentstroke}{rgb}{0.000000,0.000000,0.000000}%
\pgfsetstrokecolor{currentstroke}%
\pgfsetdash{}{0pt}%
\pgfpathmoveto{\pgfqpoint{1.670884in}{2.183278in}}%
\pgfpathlineto{\pgfqpoint{1.674269in}{2.205807in}}%
\pgfpathlineto{\pgfqpoint{1.574179in}{2.211922in}}%
\pgfpathlineto{\pgfqpoint{1.573328in}{2.189253in}}%
\pgfpathlineto{\pgfqpoint{1.670884in}{2.183278in}}%
\pgfpathclose%
\pgfusepath{fill}%
\end{pgfscope}%
\begin{pgfscope}%
\pgfpathrectangle{\pgfqpoint{0.000000in}{0.000000in}}{\pgfqpoint{3.000000in}{3.000000in}}%
\pgfusepath{clip}%
\pgfsetbuttcap%
\pgfsetroundjoin%
\definecolor{currentfill}{rgb}{0.856506,0.000000,0.000000}%
\pgfsetfillcolor{currentfill}%
\pgfsetlinewidth{0.000000pt}%
\definecolor{currentstroke}{rgb}{0.000000,0.000000,0.000000}%
\pgfsetstrokecolor{currentstroke}%
\pgfsetdash{}{0pt}%
\pgfpathmoveto{\pgfqpoint{1.357176in}{2.288926in}}%
\pgfpathlineto{\pgfqpoint{1.352982in}{2.309947in}}%
\pgfpathlineto{\pgfqpoint{1.245192in}{2.292493in}}%
\pgfpathlineto{\pgfqpoint{1.251809in}{2.271820in}}%
\pgfpathlineto{\pgfqpoint{1.357176in}{2.288926in}}%
\pgfpathclose%
\pgfusepath{fill}%
\end{pgfscope}%
\begin{pgfscope}%
\pgfpathrectangle{\pgfqpoint{0.000000in}{0.000000in}}{\pgfqpoint{3.000000in}{3.000000in}}%
\pgfusepath{clip}%
\pgfsetbuttcap%
\pgfsetroundjoin%
\definecolor{currentfill}{rgb}{0.000000,0.300000,1.000000}%
\pgfsetfillcolor{currentfill}%
\pgfsetlinewidth{0.000000pt}%
\definecolor{currentstroke}{rgb}{0.000000,0.000000,0.000000}%
\pgfsetstrokecolor{currentstroke}%
\pgfsetdash{}{0pt}%
\pgfpathmoveto{\pgfqpoint{1.585095in}{1.353138in}}%
\pgfpathlineto{\pgfqpoint{1.588568in}{1.411936in}}%
\pgfpathlineto{\pgfqpoint{1.552627in}{1.414333in}}%
\pgfpathlineto{\pgfqpoint{1.551753in}{1.355376in}}%
\pgfpathlineto{\pgfqpoint{1.585095in}{1.353138in}}%
\pgfpathclose%
\pgfusepath{fill}%
\end{pgfscope}%
\begin{pgfscope}%
\pgfpathrectangle{\pgfqpoint{0.000000in}{0.000000in}}{\pgfqpoint{3.000000in}{3.000000in}}%
\pgfusepath{clip}%
\pgfsetbuttcap%
\pgfsetroundjoin%
\definecolor{currentfill}{rgb}{1.000000,0.291213,0.000000}%
\pgfsetfillcolor{currentfill}%
\pgfsetlinewidth{0.000000pt}%
\definecolor{currentstroke}{rgb}{0.000000,0.000000,0.000000}%
\pgfsetstrokecolor{currentstroke}%
\pgfsetdash{}{0pt}%
\pgfpathmoveto{\pgfqpoint{1.572477in}{2.166244in}}%
\pgfpathlineto{\pgfqpoint{1.573328in}{2.189253in}}%
\pgfpathlineto{\pgfqpoint{1.475045in}{2.188056in}}%
\pgfpathlineto{\pgfqpoint{1.476747in}{2.165074in}}%
\pgfpathlineto{\pgfqpoint{1.572477in}{2.166244in}}%
\pgfpathclose%
\pgfusepath{fill}%
\end{pgfscope}%
\begin{pgfscope}%
\pgfpathrectangle{\pgfqpoint{0.000000in}{0.000000in}}{\pgfqpoint{3.000000in}{3.000000in}}%
\pgfusepath{clip}%
\pgfsetbuttcap%
\pgfsetroundjoin%
\definecolor{currentfill}{rgb}{0.999109,0.073348,0.000000}%
\pgfsetfillcolor{currentfill}%
\pgfsetlinewidth{0.000000pt}%
\definecolor{currentstroke}{rgb}{0.000000,0.000000,0.000000}%
\pgfsetstrokecolor{currentstroke}%
\pgfsetdash{}{0pt}%
\pgfpathmoveto{\pgfqpoint{1.783082in}{2.235987in}}%
\pgfpathlineto{\pgfqpoint{1.788915in}{2.257312in}}%
\pgfpathlineto{\pgfqpoint{1.684401in}{2.271550in}}%
\pgfpathlineto{\pgfqpoint{1.681028in}{2.249926in}}%
\pgfpathlineto{\pgfqpoint{1.783082in}{2.235987in}}%
\pgfpathclose%
\pgfusepath{fill}%
\end{pgfscope}%
\begin{pgfscope}%
\pgfpathrectangle{\pgfqpoint{0.000000in}{0.000000in}}{\pgfqpoint{3.000000in}{3.000000in}}%
\pgfusepath{clip}%
\pgfsetbuttcap%
\pgfsetroundjoin%
\definecolor{currentfill}{rgb}{0.000000,0.300000,1.000000}%
\pgfsetfillcolor{currentfill}%
\pgfsetlinewidth{0.000000pt}%
\definecolor{currentstroke}{rgb}{0.000000,0.000000,0.000000}%
\pgfsetstrokecolor{currentstroke}%
\pgfsetdash{}{0pt}%
\pgfpathmoveto{\pgfqpoint{1.518145in}{1.354928in}}%
\pgfpathlineto{\pgfqpoint{1.516400in}{1.413852in}}%
\pgfpathlineto{\pgfqpoint{1.480745in}{1.410504in}}%
\pgfpathlineto{\pgfqpoint{1.485069in}{1.351802in}}%
\pgfpathlineto{\pgfqpoint{1.518145in}{1.354928in}}%
\pgfpathclose%
\pgfusepath{fill}%
\end{pgfscope}%
\begin{pgfscope}%
\pgfpathrectangle{\pgfqpoint{0.000000in}{0.000000in}}{\pgfqpoint{3.000000in}{3.000000in}}%
\pgfusepath{clip}%
\pgfsetbuttcap%
\pgfsetroundjoin%
\definecolor{currentfill}{rgb}{0.606952,0.000000,0.000000}%
\pgfsetfillcolor{currentfill}%
\pgfsetlinewidth{0.000000pt}%
\definecolor{currentstroke}{rgb}{0.000000,0.000000,0.000000}%
\pgfsetstrokecolor{currentstroke}%
\pgfsetdash{}{0pt}%
\pgfpathmoveto{\pgfqpoint{1.928730in}{2.336282in}}%
\pgfpathlineto{\pgfqpoint{1.936864in}{2.355757in}}%
\pgfpathlineto{\pgfqpoint{1.823790in}{2.380130in}}%
\pgfpathlineto{\pgfqpoint{1.817993in}{2.360210in}}%
\pgfpathlineto{\pgfqpoint{1.928730in}{2.336282in}}%
\pgfpathclose%
\pgfusepath{fill}%
\end{pgfscope}%
\begin{pgfscope}%
\pgfpathrectangle{\pgfqpoint{0.000000in}{0.000000in}}{\pgfqpoint{3.000000in}{3.000000in}}%
\pgfusepath{clip}%
\pgfsetbuttcap%
\pgfsetroundjoin%
\definecolor{currentfill}{rgb}{0.000000,0.064706,1.000000}%
\pgfsetfillcolor{currentfill}%
\pgfsetlinewidth{0.000000pt}%
\definecolor{currentstroke}{rgb}{0.000000,0.000000,0.000000}%
\pgfsetstrokecolor{currentstroke}%
\pgfsetdash{}{0pt}%
\pgfpathmoveto{\pgfqpoint{1.460125in}{1.281122in}}%
\pgfpathlineto{\pgfqpoint{1.453315in}{1.346073in}}%
\pgfpathlineto{\pgfqpoint{1.423641in}{1.337870in}}%
\pgfpathlineto{\pgfqpoint{1.432775in}{1.273509in}}%
\pgfpathlineto{\pgfqpoint{1.460125in}{1.281122in}}%
\pgfpathclose%
\pgfusepath{fill}%
\end{pgfscope}%
\begin{pgfscope}%
\pgfpathrectangle{\pgfqpoint{0.000000in}{0.000000in}}{\pgfqpoint{3.000000in}{3.000000in}}%
\pgfusepath{clip}%
\pgfsetbuttcap%
\pgfsetroundjoin%
\definecolor{currentfill}{rgb}{1.000000,0.233115,0.000000}%
\pgfsetfillcolor{currentfill}%
\pgfsetlinewidth{0.000000pt}%
\definecolor{currentstroke}{rgb}{0.000000,0.000000,0.000000}%
\pgfsetstrokecolor{currentstroke}%
\pgfsetdash{}{0pt}%
\pgfpathmoveto{\pgfqpoint{1.475045in}{2.188056in}}%
\pgfpathlineto{\pgfqpoint{1.473346in}{2.210696in}}%
\pgfpathlineto{\pgfqpoint{1.373998in}{2.202155in}}%
\pgfpathlineto{\pgfqpoint{1.378215in}{2.179708in}}%
\pgfpathlineto{\pgfqpoint{1.475045in}{2.188056in}}%
\pgfpathclose%
\pgfusepath{fill}%
\end{pgfscope}%
\begin{pgfscope}%
\pgfpathrectangle{\pgfqpoint{0.000000in}{0.000000in}}{\pgfqpoint{3.000000in}{3.000000in}}%
\pgfusepath{clip}%
\pgfsetbuttcap%
\pgfsetroundjoin%
\definecolor{currentfill}{rgb}{0.000000,0.000000,0.838681}%
\pgfsetfillcolor{currentfill}%
\pgfsetlinewidth{0.000000pt}%
\definecolor{currentstroke}{rgb}{0.000000,0.000000,0.000000}%
\pgfsetstrokecolor{currentstroke}%
\pgfsetdash{}{0pt}%
\pgfpathmoveto{\pgfqpoint{1.675331in}{1.184671in}}%
\pgfpathlineto{\pgfqpoint{1.687847in}{1.256226in}}%
\pgfpathlineto{\pgfqpoint{1.665142in}{1.267244in}}%
\pgfpathlineto{\pgfqpoint{1.654566in}{1.194829in}}%
\pgfpathlineto{\pgfqpoint{1.675331in}{1.184671in}}%
\pgfpathclose%
\pgfusepath{fill}%
\end{pgfscope}%
\begin{pgfscope}%
\pgfpathrectangle{\pgfqpoint{0.000000in}{0.000000in}}{\pgfqpoint{3.000000in}{3.000000in}}%
\pgfusepath{clip}%
\pgfsetbuttcap%
\pgfsetroundjoin%
\definecolor{currentfill}{rgb}{1.000000,0.291213,0.000000}%
\pgfsetfillcolor{currentfill}%
\pgfsetlinewidth{0.000000pt}%
\definecolor{currentstroke}{rgb}{0.000000,0.000000,0.000000}%
\pgfsetstrokecolor{currentstroke}%
\pgfsetdash{}{0pt}%
\pgfpathmoveto{\pgfqpoint{1.667496in}{2.160408in}}%
\pgfpathlineto{\pgfqpoint{1.670884in}{2.183278in}}%
\pgfpathlineto{\pgfqpoint{1.573328in}{2.189253in}}%
\pgfpathlineto{\pgfqpoint{1.572477in}{2.166244in}}%
\pgfpathlineto{\pgfqpoint{1.667496in}{2.160408in}}%
\pgfpathclose%
\pgfusepath{fill}%
\end{pgfscope}%
\begin{pgfscope}%
\pgfpathrectangle{\pgfqpoint{0.000000in}{0.000000in}}{\pgfqpoint{3.000000in}{3.000000in}}%
\pgfusepath{clip}%
\pgfsetbuttcap%
\pgfsetroundjoin%
\definecolor{currentfill}{rgb}{0.927807,0.015251,0.000000}%
\pgfsetfillcolor{currentfill}%
\pgfsetlinewidth{0.000000pt}%
\definecolor{currentstroke}{rgb}{0.000000,0.000000,0.000000}%
\pgfsetstrokecolor{currentstroke}%
\pgfsetdash{}{0pt}%
\pgfpathmoveto{\pgfqpoint{1.361375in}{2.267650in}}%
\pgfpathlineto{\pgfqpoint{1.357176in}{2.288926in}}%
\pgfpathlineto{\pgfqpoint{1.251809in}{2.271820in}}%
\pgfpathlineto{\pgfqpoint{1.258433in}{2.250896in}}%
\pgfpathlineto{\pgfqpoint{1.361375in}{2.267650in}}%
\pgfpathclose%
\pgfusepath{fill}%
\end{pgfscope}%
\begin{pgfscope}%
\pgfpathrectangle{\pgfqpoint{0.000000in}{0.000000in}}{\pgfqpoint{3.000000in}{3.000000in}}%
\pgfusepath{clip}%
\pgfsetbuttcap%
\pgfsetroundjoin%
\definecolor{currentfill}{rgb}{1.000000,0.349310,0.000000}%
\pgfsetfillcolor{currentfill}%
\pgfsetlinewidth{0.000000pt}%
\definecolor{currentstroke}{rgb}{0.000000,0.000000,0.000000}%
\pgfsetstrokecolor{currentstroke}%
\pgfsetdash{}{0pt}%
\pgfpathmoveto{\pgfqpoint{1.571624in}{2.142870in}}%
\pgfpathlineto{\pgfqpoint{1.572477in}{2.166244in}}%
\pgfpathlineto{\pgfqpoint{1.476747in}{2.165074in}}%
\pgfpathlineto{\pgfqpoint{1.478451in}{2.141728in}}%
\pgfpathlineto{\pgfqpoint{1.571624in}{2.142870in}}%
\pgfpathclose%
\pgfusepath{fill}%
\end{pgfscope}%
\begin{pgfscope}%
\pgfpathrectangle{\pgfqpoint{0.000000in}{0.000000in}}{\pgfqpoint{3.000000in}{3.000000in}}%
\pgfusepath{clip}%
\pgfsetbuttcap%
\pgfsetroundjoin%
\definecolor{currentfill}{rgb}{0.000000,0.300000,1.000000}%
\pgfsetfillcolor{currentfill}%
\pgfsetlinewidth{0.000000pt}%
\definecolor{currentstroke}{rgb}{0.000000,0.000000,0.000000}%
\pgfsetstrokecolor{currentstroke}%
\pgfsetdash{}{0pt}%
\pgfpathmoveto{\pgfqpoint{1.617376in}{1.348265in}}%
\pgfpathlineto{\pgfqpoint{1.623368in}{1.406714in}}%
\pgfpathlineto{\pgfqpoint{1.588568in}{1.411936in}}%
\pgfpathlineto{\pgfqpoint{1.585095in}{1.353138in}}%
\pgfpathlineto{\pgfqpoint{1.617376in}{1.348265in}}%
\pgfpathclose%
\pgfusepath{fill}%
\end{pgfscope}%
\begin{pgfscope}%
\pgfpathrectangle{\pgfqpoint{0.000000in}{0.000000in}}{\pgfqpoint{3.000000in}{3.000000in}}%
\pgfusepath{clip}%
\pgfsetbuttcap%
\pgfsetroundjoin%
\definecolor{currentfill}{rgb}{1.000000,0.116921,0.000000}%
\pgfsetfillcolor{currentfill}%
\pgfsetlinewidth{0.000000pt}%
\definecolor{currentstroke}{rgb}{0.000000,0.000000,0.000000}%
\pgfsetstrokecolor{currentstroke}%
\pgfsetdash{}{0pt}%
\pgfpathmoveto{\pgfqpoint{1.777242in}{2.214379in}}%
\pgfpathlineto{\pgfqpoint{1.783082in}{2.235987in}}%
\pgfpathlineto{\pgfqpoint{1.681028in}{2.249926in}}%
\pgfpathlineto{\pgfqpoint{1.677650in}{2.228017in}}%
\pgfpathlineto{\pgfqpoint{1.777242in}{2.214379in}}%
\pgfpathclose%
\pgfusepath{fill}%
\end{pgfscope}%
\begin{pgfscope}%
\pgfpathrectangle{\pgfqpoint{0.000000in}{0.000000in}}{\pgfqpoint{3.000000in}{3.000000in}}%
\pgfusepath{clip}%
\pgfsetbuttcap%
\pgfsetroundjoin%
\definecolor{currentfill}{rgb}{0.500000,0.000000,0.000000}%
\pgfsetfillcolor{currentfill}%
\pgfsetlinewidth{0.000000pt}%
\definecolor{currentstroke}{rgb}{0.000000,0.000000,0.000000}%
\pgfsetstrokecolor{currentstroke}%
\pgfsetdash{}{0pt}%
\pgfpathmoveto{\pgfqpoint{1.212205in}{2.392509in}}%
\pgfpathlineto{\pgfqpoint{1.205628in}{2.411916in}}%
\pgfpathlineto{\pgfqpoint{1.090596in}{2.383866in}}%
\pgfpathlineto{\pgfqpoint{1.099461in}{2.364949in}}%
\pgfpathlineto{\pgfqpoint{1.212205in}{2.392509in}}%
\pgfpathclose%
\pgfusepath{fill}%
\end{pgfscope}%
\begin{pgfscope}%
\pgfpathrectangle{\pgfqpoint{0.000000in}{0.000000in}}{\pgfqpoint{3.000000in}{3.000000in}}%
\pgfusepath{clip}%
\pgfsetbuttcap%
\pgfsetroundjoin%
\definecolor{currentfill}{rgb}{0.678253,0.000000,0.000000}%
\pgfsetfillcolor{currentfill}%
\pgfsetlinewidth{0.000000pt}%
\definecolor{currentstroke}{rgb}{0.000000,0.000000,0.000000}%
\pgfsetstrokecolor{currentstroke}%
\pgfsetdash{}{0pt}%
\pgfpathmoveto{\pgfqpoint{1.920588in}{2.316604in}}%
\pgfpathlineto{\pgfqpoint{1.928730in}{2.336282in}}%
\pgfpathlineto{\pgfqpoint{1.817993in}{2.360210in}}%
\pgfpathlineto{\pgfqpoint{1.812190in}{2.340085in}}%
\pgfpathlineto{\pgfqpoint{1.920588in}{2.316604in}}%
\pgfpathclose%
\pgfusepath{fill}%
\end{pgfscope}%
\begin{pgfscope}%
\pgfpathrectangle{\pgfqpoint{0.000000in}{0.000000in}}{\pgfqpoint{3.000000in}{3.000000in}}%
\pgfusepath{clip}%
\pgfsetbuttcap%
\pgfsetroundjoin%
\definecolor{currentfill}{rgb}{1.000000,0.291213,0.000000}%
\pgfsetfillcolor{currentfill}%
\pgfsetlinewidth{0.000000pt}%
\definecolor{currentstroke}{rgb}{0.000000,0.000000,0.000000}%
\pgfsetstrokecolor{currentstroke}%
\pgfsetdash{}{0pt}%
\pgfpathmoveto{\pgfqpoint{1.476747in}{2.165074in}}%
\pgfpathlineto{\pgfqpoint{1.475045in}{2.188056in}}%
\pgfpathlineto{\pgfqpoint{1.378215in}{2.179708in}}%
\pgfpathlineto{\pgfqpoint{1.382437in}{2.156922in}}%
\pgfpathlineto{\pgfqpoint{1.476747in}{2.165074in}}%
\pgfpathclose%
\pgfusepath{fill}%
\end{pgfscope}%
\begin{pgfscope}%
\pgfpathrectangle{\pgfqpoint{0.000000in}{0.000000in}}{\pgfqpoint{3.000000in}{3.000000in}}%
\pgfusepath{clip}%
\pgfsetbuttcap%
\pgfsetroundjoin%
\definecolor{currentfill}{rgb}{0.000000,0.300000,1.000000}%
\pgfsetfillcolor{currentfill}%
\pgfsetlinewidth{0.000000pt}%
\definecolor{currentstroke}{rgb}{0.000000,0.000000,0.000000}%
\pgfsetstrokecolor{currentstroke}%
\pgfsetdash{}{0pt}%
\pgfpathmoveto{\pgfqpoint{1.485069in}{1.351802in}}%
\pgfpathlineto{\pgfqpoint{1.480745in}{1.410504in}}%
\pgfpathlineto{\pgfqpoint{1.446511in}{1.404366in}}%
\pgfpathlineto{\pgfqpoint{1.453315in}{1.346073in}}%
\pgfpathlineto{\pgfqpoint{1.485069in}{1.351802in}}%
\pgfpathclose%
\pgfusepath{fill}%
\end{pgfscope}%
\begin{pgfscope}%
\pgfpathrectangle{\pgfqpoint{0.000000in}{0.000000in}}{\pgfqpoint{3.000000in}{3.000000in}}%
\pgfusepath{clip}%
\pgfsetbuttcap%
\pgfsetroundjoin%
\definecolor{currentfill}{rgb}{0.000000,0.000000,0.500000}%
\pgfsetfillcolor{currentfill}%
\pgfsetlinewidth{0.000000pt}%
\definecolor{currentstroke}{rgb}{0.000000,0.000000,0.000000}%
\pgfsetstrokecolor{currentstroke}%
\pgfsetdash{}{0pt}%
\pgfpathmoveto{\pgfqpoint{1.691192in}{1.079224in}}%
\pgfpathlineto{\pgfqpoint{1.706660in}{1.159907in}}%
\pgfpathlineto{\pgfqpoint{1.692841in}{1.172938in}}%
\pgfpathlineto{\pgfqpoint{1.678677in}{1.091141in}}%
\pgfpathlineto{\pgfqpoint{1.691192in}{1.079224in}}%
\pgfpathclose%
\pgfusepath{fill}%
\end{pgfscope}%
\begin{pgfscope}%
\pgfpathrectangle{\pgfqpoint{0.000000in}{0.000000in}}{\pgfqpoint{3.000000in}{3.000000in}}%
\pgfusepath{clip}%
\pgfsetbuttcap%
\pgfsetroundjoin%
\definecolor{currentfill}{rgb}{1.000000,0.349310,0.000000}%
\pgfsetfillcolor{currentfill}%
\pgfsetlinewidth{0.000000pt}%
\definecolor{currentstroke}{rgb}{0.000000,0.000000,0.000000}%
\pgfsetstrokecolor{currentstroke}%
\pgfsetdash{}{0pt}%
\pgfpathmoveto{\pgfqpoint{1.664104in}{2.137175in}}%
\pgfpathlineto{\pgfqpoint{1.667496in}{2.160408in}}%
\pgfpathlineto{\pgfqpoint{1.572477in}{2.166244in}}%
\pgfpathlineto{\pgfqpoint{1.571624in}{2.142870in}}%
\pgfpathlineto{\pgfqpoint{1.664104in}{2.137175in}}%
\pgfpathclose%
\pgfusepath{fill}%
\end{pgfscope}%
\begin{pgfscope}%
\pgfpathrectangle{\pgfqpoint{0.000000in}{0.000000in}}{\pgfqpoint{3.000000in}{3.000000in}}%
\pgfusepath{clip}%
\pgfsetbuttcap%
\pgfsetroundjoin%
\definecolor{currentfill}{rgb}{0.999109,0.073348,0.000000}%
\pgfsetfillcolor{currentfill}%
\pgfsetlinewidth{0.000000pt}%
\definecolor{currentstroke}{rgb}{0.000000,0.000000,0.000000}%
\pgfsetstrokecolor{currentstroke}%
\pgfsetdash{}{0pt}%
\pgfpathmoveto{\pgfqpoint{1.365578in}{2.246108in}}%
\pgfpathlineto{\pgfqpoint{1.361375in}{2.267650in}}%
\pgfpathlineto{\pgfqpoint{1.258433in}{2.250896in}}%
\pgfpathlineto{\pgfqpoint{1.265064in}{2.229706in}}%
\pgfpathlineto{\pgfqpoint{1.365578in}{2.246108in}}%
\pgfpathclose%
\pgfusepath{fill}%
\end{pgfscope}%
\begin{pgfscope}%
\pgfpathrectangle{\pgfqpoint{0.000000in}{0.000000in}}{\pgfqpoint{3.000000in}{3.000000in}}%
\pgfusepath{clip}%
\pgfsetbuttcap%
\pgfsetroundjoin%
\definecolor{currentfill}{rgb}{0.000000,0.000000,0.838681}%
\pgfsetfillcolor{currentfill}%
\pgfsetlinewidth{0.000000pt}%
\definecolor{currentstroke}{rgb}{0.000000,0.000000,0.000000}%
\pgfsetstrokecolor{currentstroke}%
\pgfsetdash{}{0pt}%
\pgfpathmoveto{\pgfqpoint{1.419262in}{1.191634in}}%
\pgfpathlineto{\pgfqpoint{1.408009in}{1.263779in}}%
\pgfpathlineto{\pgfqpoint{1.386430in}{1.252159in}}%
\pgfpathlineto{\pgfqpoint{1.399529in}{1.180922in}}%
\pgfpathlineto{\pgfqpoint{1.419262in}{1.191634in}}%
\pgfpathclose%
\pgfusepath{fill}%
\end{pgfscope}%
\begin{pgfscope}%
\pgfpathrectangle{\pgfqpoint{0.000000in}{0.000000in}}{\pgfqpoint{3.000000in}{3.000000in}}%
\pgfusepath{clip}%
\pgfsetbuttcap%
\pgfsetroundjoin%
\definecolor{currentfill}{rgb}{1.000000,0.407407,0.000000}%
\pgfsetfillcolor{currentfill}%
\pgfsetlinewidth{0.000000pt}%
\definecolor{currentstroke}{rgb}{0.000000,0.000000,0.000000}%
\pgfsetstrokecolor{currentstroke}%
\pgfsetdash{}{0pt}%
\pgfpathmoveto{\pgfqpoint{1.570770in}{2.119107in}}%
\pgfpathlineto{\pgfqpoint{1.571624in}{2.142870in}}%
\pgfpathlineto{\pgfqpoint{1.478451in}{2.141728in}}%
\pgfpathlineto{\pgfqpoint{1.480156in}{2.117994in}}%
\pgfpathlineto{\pgfqpoint{1.570770in}{2.119107in}}%
\pgfpathclose%
\pgfusepath{fill}%
\end{pgfscope}%
\begin{pgfscope}%
\pgfpathrectangle{\pgfqpoint{0.000000in}{0.000000in}}{\pgfqpoint{3.000000in}{3.000000in}}%
\pgfusepath{clip}%
\pgfsetbuttcap%
\pgfsetroundjoin%
\definecolor{currentfill}{rgb}{0.000000,0.503922,1.000000}%
\pgfsetfillcolor{currentfill}%
\pgfsetlinewidth{0.000000pt}%
\definecolor{currentstroke}{rgb}{0.000000,0.000000,0.000000}%
\pgfsetstrokecolor{currentstroke}%
\pgfsetdash{}{0pt}%
\pgfpathmoveto{\pgfqpoint{1.552627in}{1.414333in}}%
\pgfpathlineto{\pgfqpoint{1.553500in}{1.467985in}}%
\pgfpathlineto{\pgfqpoint{1.514656in}{1.467473in}}%
\pgfpathlineto{\pgfqpoint{1.516400in}{1.413852in}}%
\pgfpathlineto{\pgfqpoint{1.552627in}{1.414333in}}%
\pgfpathclose%
\pgfusepath{fill}%
\end{pgfscope}%
\begin{pgfscope}%
\pgfpathrectangle{\pgfqpoint{0.000000in}{0.000000in}}{\pgfqpoint{3.000000in}{3.000000in}}%
\pgfusepath{clip}%
\pgfsetbuttcap%
\pgfsetroundjoin%
\definecolor{currentfill}{rgb}{1.000000,0.175018,0.000000}%
\pgfsetfillcolor{currentfill}%
\pgfsetlinewidth{0.000000pt}%
\definecolor{currentstroke}{rgb}{0.000000,0.000000,0.000000}%
\pgfsetstrokecolor{currentstroke}%
\pgfsetdash{}{0pt}%
\pgfpathmoveto{\pgfqpoint{1.771396in}{2.192473in}}%
\pgfpathlineto{\pgfqpoint{1.777242in}{2.214379in}}%
\pgfpathlineto{\pgfqpoint{1.677650in}{2.228017in}}%
\pgfpathlineto{\pgfqpoint{1.674269in}{2.205807in}}%
\pgfpathlineto{\pgfqpoint{1.771396in}{2.192473in}}%
\pgfpathclose%
\pgfusepath{fill}%
\end{pgfscope}%
\begin{pgfscope}%
\pgfpathrectangle{\pgfqpoint{0.000000in}{0.000000in}}{\pgfqpoint{3.000000in}{3.000000in}}%
\pgfusepath{clip}%
\pgfsetbuttcap%
\pgfsetroundjoin%
\definecolor{currentfill}{rgb}{0.000000,0.064706,1.000000}%
\pgfsetfillcolor{currentfill}%
\pgfsetlinewidth{0.000000pt}%
\definecolor{currentstroke}{rgb}{0.000000,0.000000,0.000000}%
\pgfsetstrokecolor{currentstroke}%
\pgfsetdash{}{0pt}%
\pgfpathmoveto{\pgfqpoint{1.665142in}{1.267244in}}%
\pgfpathlineto{\pgfqpoint{1.675711in}{1.331120in}}%
\pgfpathlineto{\pgfqpoint{1.647825in}{1.340869in}}%
\pgfpathlineto{\pgfqpoint{1.639445in}{1.276292in}}%
\pgfpathlineto{\pgfqpoint{1.665142in}{1.267244in}}%
\pgfpathclose%
\pgfusepath{fill}%
\end{pgfscope}%
\begin{pgfscope}%
\pgfpathrectangle{\pgfqpoint{0.000000in}{0.000000in}}{\pgfqpoint{3.000000in}{3.000000in}}%
\pgfusepath{clip}%
\pgfsetbuttcap%
\pgfsetroundjoin%
\definecolor{currentfill}{rgb}{0.553476,0.000000,0.000000}%
\pgfsetfillcolor{currentfill}%
\pgfsetlinewidth{0.000000pt}%
\definecolor{currentstroke}{rgb}{0.000000,0.000000,0.000000}%
\pgfsetstrokecolor{currentstroke}%
\pgfsetdash{}{0pt}%
\pgfpathmoveto{\pgfqpoint{1.218789in}{2.372916in}}%
\pgfpathlineto{\pgfqpoint{1.212205in}{2.392509in}}%
\pgfpathlineto{\pgfqpoint{1.099461in}{2.364949in}}%
\pgfpathlineto{\pgfqpoint{1.108333in}{2.345848in}}%
\pgfpathlineto{\pgfqpoint{1.218789in}{2.372916in}}%
\pgfpathclose%
\pgfusepath{fill}%
\end{pgfscope}%
\begin{pgfscope}%
\pgfpathrectangle{\pgfqpoint{0.000000in}{0.000000in}}{\pgfqpoint{3.000000in}{3.000000in}}%
\pgfusepath{clip}%
\pgfsetbuttcap%
\pgfsetroundjoin%
\definecolor{currentfill}{rgb}{0.000000,0.503922,1.000000}%
\pgfsetfillcolor{currentfill}%
\pgfsetlinewidth{0.000000pt}%
\definecolor{currentstroke}{rgb}{0.000000,0.000000,0.000000}%
\pgfsetstrokecolor{currentstroke}%
\pgfsetdash{}{0pt}%
\pgfpathmoveto{\pgfqpoint{1.588568in}{1.411936in}}%
\pgfpathlineto{\pgfqpoint{1.592037in}{1.465429in}}%
\pgfpathlineto{\pgfqpoint{1.553500in}{1.467985in}}%
\pgfpathlineto{\pgfqpoint{1.552627in}{1.414333in}}%
\pgfpathlineto{\pgfqpoint{1.588568in}{1.411936in}}%
\pgfpathclose%
\pgfusepath{fill}%
\end{pgfscope}%
\begin{pgfscope}%
\pgfpathrectangle{\pgfqpoint{0.000000in}{0.000000in}}{\pgfqpoint{3.000000in}{3.000000in}}%
\pgfusepath{clip}%
\pgfsetbuttcap%
\pgfsetroundjoin%
\definecolor{currentfill}{rgb}{0.731729,0.000000,0.000000}%
\pgfsetfillcolor{currentfill}%
\pgfsetlinewidth{0.000000pt}%
\definecolor{currentstroke}{rgb}{0.000000,0.000000,0.000000}%
\pgfsetstrokecolor{currentstroke}%
\pgfsetdash{}{0pt}%
\pgfpathmoveto{\pgfqpoint{1.912439in}{2.296712in}}%
\pgfpathlineto{\pgfqpoint{1.920588in}{2.316604in}}%
\pgfpathlineto{\pgfqpoint{1.812190in}{2.340085in}}%
\pgfpathlineto{\pgfqpoint{1.806380in}{2.319745in}}%
\pgfpathlineto{\pgfqpoint{1.912439in}{2.296712in}}%
\pgfpathclose%
\pgfusepath{fill}%
\end{pgfscope}%
\begin{pgfscope}%
\pgfpathrectangle{\pgfqpoint{0.000000in}{0.000000in}}{\pgfqpoint{3.000000in}{3.000000in}}%
\pgfusepath{clip}%
\pgfsetbuttcap%
\pgfsetroundjoin%
\definecolor{currentfill}{rgb}{1.000000,0.349310,0.000000}%
\pgfsetfillcolor{currentfill}%
\pgfsetlinewidth{0.000000pt}%
\definecolor{currentstroke}{rgb}{0.000000,0.000000,0.000000}%
\pgfsetstrokecolor{currentstroke}%
\pgfsetdash{}{0pt}%
\pgfpathmoveto{\pgfqpoint{1.478451in}{2.141728in}}%
\pgfpathlineto{\pgfqpoint{1.476747in}{2.165074in}}%
\pgfpathlineto{\pgfqpoint{1.382437in}{2.156922in}}%
\pgfpathlineto{\pgfqpoint{1.386663in}{2.133773in}}%
\pgfpathlineto{\pgfqpoint{1.478451in}{2.141728in}}%
\pgfpathclose%
\pgfusepath{fill}%
\end{pgfscope}%
\begin{pgfscope}%
\pgfpathrectangle{\pgfqpoint{0.000000in}{0.000000in}}{\pgfqpoint{3.000000in}{3.000000in}}%
\pgfusepath{clip}%
\pgfsetbuttcap%
\pgfsetroundjoin%
\definecolor{currentfill}{rgb}{0.000000,0.503922,1.000000}%
\pgfsetfillcolor{currentfill}%
\pgfsetlinewidth{0.000000pt}%
\definecolor{currentstroke}{rgb}{0.000000,0.000000,0.000000}%
\pgfsetstrokecolor{currentstroke}%
\pgfsetdash{}{0pt}%
\pgfpathmoveto{\pgfqpoint{1.516400in}{1.413852in}}%
\pgfpathlineto{\pgfqpoint{1.514656in}{1.467473in}}%
\pgfpathlineto{\pgfqpoint{1.476425in}{1.463903in}}%
\pgfpathlineto{\pgfqpoint{1.480745in}{1.410504in}}%
\pgfpathlineto{\pgfqpoint{1.516400in}{1.413852in}}%
\pgfpathclose%
\pgfusepath{fill}%
\end{pgfscope}%
\begin{pgfscope}%
\pgfpathrectangle{\pgfqpoint{0.000000in}{0.000000in}}{\pgfqpoint{3.000000in}{3.000000in}}%
\pgfusepath{clip}%
\pgfsetbuttcap%
\pgfsetroundjoin%
\definecolor{currentfill}{rgb}{1.000000,0.407407,0.000000}%
\pgfsetfillcolor{currentfill}%
\pgfsetlinewidth{0.000000pt}%
\definecolor{currentstroke}{rgb}{0.000000,0.000000,0.000000}%
\pgfsetstrokecolor{currentstroke}%
\pgfsetdash{}{0pt}%
\pgfpathmoveto{\pgfqpoint{1.660708in}{2.113553in}}%
\pgfpathlineto{\pgfqpoint{1.664104in}{2.137175in}}%
\pgfpathlineto{\pgfqpoint{1.571624in}{2.142870in}}%
\pgfpathlineto{\pgfqpoint{1.570770in}{2.119107in}}%
\pgfpathlineto{\pgfqpoint{1.660708in}{2.113553in}}%
\pgfpathclose%
\pgfusepath{fill}%
\end{pgfscope}%
\begin{pgfscope}%
\pgfpathrectangle{\pgfqpoint{0.000000in}{0.000000in}}{\pgfqpoint{3.000000in}{3.000000in}}%
\pgfusepath{clip}%
\pgfsetbuttcap%
\pgfsetroundjoin%
\definecolor{currentfill}{rgb}{1.000000,0.116921,0.000000}%
\pgfsetfillcolor{currentfill}%
\pgfsetlinewidth{0.000000pt}%
\definecolor{currentstroke}{rgb}{0.000000,0.000000,0.000000}%
\pgfsetstrokecolor{currentstroke}%
\pgfsetdash{}{0pt}%
\pgfpathmoveto{\pgfqpoint{1.369786in}{2.224281in}}%
\pgfpathlineto{\pgfqpoint{1.365578in}{2.246108in}}%
\pgfpathlineto{\pgfqpoint{1.265064in}{2.229706in}}%
\pgfpathlineto{\pgfqpoint{1.271702in}{2.208235in}}%
\pgfpathlineto{\pgfqpoint{1.369786in}{2.224281in}}%
\pgfpathclose%
\pgfusepath{fill}%
\end{pgfscope}%
\begin{pgfscope}%
\pgfpathrectangle{\pgfqpoint{0.000000in}{0.000000in}}{\pgfqpoint{3.000000in}{3.000000in}}%
\pgfusepath{clip}%
\pgfsetbuttcap%
\pgfsetroundjoin%
\definecolor{currentfill}{rgb}{1.000000,0.480029,0.000000}%
\pgfsetfillcolor{currentfill}%
\pgfsetlinewidth{0.000000pt}%
\definecolor{currentstroke}{rgb}{0.000000,0.000000,0.000000}%
\pgfsetstrokecolor{currentstroke}%
\pgfsetdash{}{0pt}%
\pgfpathmoveto{\pgfqpoint{1.569915in}{2.094926in}}%
\pgfpathlineto{\pgfqpoint{1.570770in}{2.119107in}}%
\pgfpathlineto{\pgfqpoint{1.480156in}{2.117994in}}%
\pgfpathlineto{\pgfqpoint{1.481864in}{2.093842in}}%
\pgfpathlineto{\pgfqpoint{1.569915in}{2.094926in}}%
\pgfpathclose%
\pgfusepath{fill}%
\end{pgfscope}%
\begin{pgfscope}%
\pgfpathrectangle{\pgfqpoint{0.000000in}{0.000000in}}{\pgfqpoint{3.000000in}{3.000000in}}%
\pgfusepath{clip}%
\pgfsetbuttcap%
\pgfsetroundjoin%
\definecolor{currentfill}{rgb}{0.000000,0.300000,1.000000}%
\pgfsetfillcolor{currentfill}%
\pgfsetlinewidth{0.000000pt}%
\definecolor{currentstroke}{rgb}{0.000000,0.000000,0.000000}%
\pgfsetstrokecolor{currentstroke}%
\pgfsetdash{}{0pt}%
\pgfpathmoveto{\pgfqpoint{1.647825in}{1.340869in}}%
\pgfpathlineto{\pgfqpoint{1.656199in}{1.398789in}}%
\pgfpathlineto{\pgfqpoint{1.623368in}{1.406714in}}%
\pgfpathlineto{\pgfqpoint{1.617376in}{1.348265in}}%
\pgfpathlineto{\pgfqpoint{1.647825in}{1.340869in}}%
\pgfpathclose%
\pgfusepath{fill}%
\end{pgfscope}%
\begin{pgfscope}%
\pgfpathrectangle{\pgfqpoint{0.000000in}{0.000000in}}{\pgfqpoint{3.000000in}{3.000000in}}%
\pgfusepath{clip}%
\pgfsetbuttcap%
\pgfsetroundjoin%
\definecolor{currentfill}{rgb}{0.000000,0.000000,0.500000}%
\pgfsetfillcolor{currentfill}%
\pgfsetlinewidth{0.000000pt}%
\definecolor{currentstroke}{rgb}{0.000000,0.000000,0.000000}%
\pgfsetstrokecolor{currentstroke}%
\pgfsetdash{}{0pt}%
\pgfpathmoveto{\pgfqpoint{1.397843in}{1.087287in}}%
\pgfpathlineto{\pgfqpoint{1.383204in}{1.168724in}}%
\pgfpathlineto{\pgfqpoint{1.370698in}{1.155329in}}%
\pgfpathlineto{\pgfqpoint{1.386520in}{1.075038in}}%
\pgfpathlineto{\pgfqpoint{1.397843in}{1.087287in}}%
\pgfpathclose%
\pgfusepath{fill}%
\end{pgfscope}%
\begin{pgfscope}%
\pgfpathrectangle{\pgfqpoint{0.000000in}{0.000000in}}{\pgfqpoint{3.000000in}{3.000000in}}%
\pgfusepath{clip}%
\pgfsetbuttcap%
\pgfsetroundjoin%
\definecolor{currentfill}{rgb}{1.000000,0.233115,0.000000}%
\pgfsetfillcolor{currentfill}%
\pgfsetlinewidth{0.000000pt}%
\definecolor{currentstroke}{rgb}{0.000000,0.000000,0.000000}%
\pgfsetstrokecolor{currentstroke}%
\pgfsetdash{}{0pt}%
\pgfpathmoveto{\pgfqpoint{1.765544in}{2.170248in}}%
\pgfpathlineto{\pgfqpoint{1.771396in}{2.192473in}}%
\pgfpathlineto{\pgfqpoint{1.674269in}{2.205807in}}%
\pgfpathlineto{\pgfqpoint{1.670884in}{2.183278in}}%
\pgfpathlineto{\pgfqpoint{1.765544in}{2.170248in}}%
\pgfpathclose%
\pgfusepath{fill}%
\end{pgfscope}%
\begin{pgfscope}%
\pgfpathrectangle{\pgfqpoint{0.000000in}{0.000000in}}{\pgfqpoint{3.000000in}{3.000000in}}%
\pgfusepath{clip}%
\pgfsetbuttcap%
\pgfsetroundjoin%
\definecolor{currentfill}{rgb}{0.000000,0.064706,1.000000}%
\pgfsetfillcolor{currentfill}%
\pgfsetlinewidth{0.000000pt}%
\definecolor{currentstroke}{rgb}{0.000000,0.000000,0.000000}%
\pgfsetstrokecolor{currentstroke}%
\pgfsetdash{}{0pt}%
\pgfpathmoveto{\pgfqpoint{1.432775in}{1.273509in}}%
\pgfpathlineto{\pgfqpoint{1.423641in}{1.337870in}}%
\pgfpathlineto{\pgfqpoint{1.396763in}{1.327385in}}%
\pgfpathlineto{\pgfqpoint{1.408009in}{1.263779in}}%
\pgfpathlineto{\pgfqpoint{1.432775in}{1.273509in}}%
\pgfpathclose%
\pgfusepath{fill}%
\end{pgfscope}%
\begin{pgfscope}%
\pgfpathrectangle{\pgfqpoint{0.000000in}{0.000000in}}{\pgfqpoint{3.000000in}{3.000000in}}%
\pgfusepath{clip}%
\pgfsetbuttcap%
\pgfsetroundjoin%
\definecolor{currentfill}{rgb}{0.606952,0.000000,0.000000}%
\pgfsetfillcolor{currentfill}%
\pgfsetlinewidth{0.000000pt}%
\definecolor{currentstroke}{rgb}{0.000000,0.000000,0.000000}%
\pgfsetstrokecolor{currentstroke}%
\pgfsetdash{}{0pt}%
\pgfpathmoveto{\pgfqpoint{1.225380in}{2.353128in}}%
\pgfpathlineto{\pgfqpoint{1.218789in}{2.372916in}}%
\pgfpathlineto{\pgfqpoint{1.108333in}{2.345848in}}%
\pgfpathlineto{\pgfqpoint{1.117214in}{2.326554in}}%
\pgfpathlineto{\pgfqpoint{1.225380in}{2.353128in}}%
\pgfpathclose%
\pgfusepath{fill}%
\end{pgfscope}%
\begin{pgfscope}%
\pgfpathrectangle{\pgfqpoint{0.000000in}{0.000000in}}{\pgfqpoint{3.000000in}{3.000000in}}%
\pgfusepath{clip}%
\pgfsetbuttcap%
\pgfsetroundjoin%
\definecolor{currentfill}{rgb}{0.803030,0.000000,0.000000}%
\pgfsetfillcolor{currentfill}%
\pgfsetlinewidth{0.000000pt}%
\definecolor{currentstroke}{rgb}{0.000000,0.000000,0.000000}%
\pgfsetstrokecolor{currentstroke}%
\pgfsetdash{}{0pt}%
\pgfpathmoveto{\pgfqpoint{1.904282in}{2.276596in}}%
\pgfpathlineto{\pgfqpoint{1.912439in}{2.296712in}}%
\pgfpathlineto{\pgfqpoint{1.806380in}{2.319745in}}%
\pgfpathlineto{\pgfqpoint{1.800565in}{2.299177in}}%
\pgfpathlineto{\pgfqpoint{1.904282in}{2.276596in}}%
\pgfpathclose%
\pgfusepath{fill}%
\end{pgfscope}%
\begin{pgfscope}%
\pgfpathrectangle{\pgfqpoint{0.000000in}{0.000000in}}{\pgfqpoint{3.000000in}{3.000000in}}%
\pgfusepath{clip}%
\pgfsetbuttcap%
\pgfsetroundjoin%
\definecolor{currentfill}{rgb}{1.000000,0.407407,0.000000}%
\pgfsetfillcolor{currentfill}%
\pgfsetlinewidth{0.000000pt}%
\definecolor{currentstroke}{rgb}{0.000000,0.000000,0.000000}%
\pgfsetstrokecolor{currentstroke}%
\pgfsetdash{}{0pt}%
\pgfpathmoveto{\pgfqpoint{1.480156in}{2.117994in}}%
\pgfpathlineto{\pgfqpoint{1.478451in}{2.141728in}}%
\pgfpathlineto{\pgfqpoint{1.386663in}{2.133773in}}%
\pgfpathlineto{\pgfqpoint{1.390894in}{2.110236in}}%
\pgfpathlineto{\pgfqpoint{1.480156in}{2.117994in}}%
\pgfpathclose%
\pgfusepath{fill}%
\end{pgfscope}%
\begin{pgfscope}%
\pgfpathrectangle{\pgfqpoint{0.000000in}{0.000000in}}{\pgfqpoint{3.000000in}{3.000000in}}%
\pgfusepath{clip}%
\pgfsetbuttcap%
\pgfsetroundjoin%
\definecolor{currentfill}{rgb}{1.000000,0.480029,0.000000}%
\pgfsetfillcolor{currentfill}%
\pgfsetlinewidth{0.000000pt}%
\definecolor{currentstroke}{rgb}{0.000000,0.000000,0.000000}%
\pgfsetstrokecolor{currentstroke}%
\pgfsetdash{}{0pt}%
\pgfpathmoveto{\pgfqpoint{1.657308in}{2.089515in}}%
\pgfpathlineto{\pgfqpoint{1.660708in}{2.113553in}}%
\pgfpathlineto{\pgfqpoint{1.570770in}{2.119107in}}%
\pgfpathlineto{\pgfqpoint{1.569915in}{2.094926in}}%
\pgfpathlineto{\pgfqpoint{1.657308in}{2.089515in}}%
\pgfpathclose%
\pgfusepath{fill}%
\end{pgfscope}%
\begin{pgfscope}%
\pgfpathrectangle{\pgfqpoint{0.000000in}{0.000000in}}{\pgfqpoint{3.000000in}{3.000000in}}%
\pgfusepath{clip}%
\pgfsetbuttcap%
\pgfsetroundjoin%
\definecolor{currentfill}{rgb}{0.000000,0.503922,1.000000}%
\pgfsetfillcolor{currentfill}%
\pgfsetlinewidth{0.000000pt}%
\definecolor{currentstroke}{rgb}{0.000000,0.000000,0.000000}%
\pgfsetstrokecolor{currentstroke}%
\pgfsetdash{}{0pt}%
\pgfpathmoveto{\pgfqpoint{1.623368in}{1.406714in}}%
\pgfpathlineto{\pgfqpoint{1.629355in}{1.459862in}}%
\pgfpathlineto{\pgfqpoint{1.592037in}{1.465429in}}%
\pgfpathlineto{\pgfqpoint{1.588568in}{1.411936in}}%
\pgfpathlineto{\pgfqpoint{1.623368in}{1.406714in}}%
\pgfpathclose%
\pgfusepath{fill}%
\end{pgfscope}%
\begin{pgfscope}%
\pgfpathrectangle{\pgfqpoint{0.000000in}{0.000000in}}{\pgfqpoint{3.000000in}{3.000000in}}%
\pgfusepath{clip}%
\pgfsetbuttcap%
\pgfsetroundjoin%
\definecolor{currentfill}{rgb}{1.000000,0.175018,0.000000}%
\pgfsetfillcolor{currentfill}%
\pgfsetlinewidth{0.000000pt}%
\definecolor{currentstroke}{rgb}{0.000000,0.000000,0.000000}%
\pgfsetstrokecolor{currentstroke}%
\pgfsetdash{}{0pt}%
\pgfpathmoveto{\pgfqpoint{1.373998in}{2.202155in}}%
\pgfpathlineto{\pgfqpoint{1.369786in}{2.224281in}}%
\pgfpathlineto{\pgfqpoint{1.271702in}{2.208235in}}%
\pgfpathlineto{\pgfqpoint{1.278346in}{2.186465in}}%
\pgfpathlineto{\pgfqpoint{1.373998in}{2.202155in}}%
\pgfpathclose%
\pgfusepath{fill}%
\end{pgfscope}%
\begin{pgfscope}%
\pgfpathrectangle{\pgfqpoint{0.000000in}{0.000000in}}{\pgfqpoint{3.000000in}{3.000000in}}%
\pgfusepath{clip}%
\pgfsetbuttcap%
\pgfsetroundjoin%
\definecolor{currentfill}{rgb}{1.000000,0.538126,0.000000}%
\pgfsetfillcolor{currentfill}%
\pgfsetlinewidth{0.000000pt}%
\definecolor{currentstroke}{rgb}{0.000000,0.000000,0.000000}%
\pgfsetstrokecolor{currentstroke}%
\pgfsetdash{}{0pt}%
\pgfpathmoveto{\pgfqpoint{1.569060in}{2.070298in}}%
\pgfpathlineto{\pgfqpoint{1.569915in}{2.094926in}}%
\pgfpathlineto{\pgfqpoint{1.481864in}{2.093842in}}%
\pgfpathlineto{\pgfqpoint{1.483573in}{2.069242in}}%
\pgfpathlineto{\pgfqpoint{1.569060in}{2.070298in}}%
\pgfpathclose%
\pgfusepath{fill}%
\end{pgfscope}%
\begin{pgfscope}%
\pgfpathrectangle{\pgfqpoint{0.000000in}{0.000000in}}{\pgfqpoint{3.000000in}{3.000000in}}%
\pgfusepath{clip}%
\pgfsetbuttcap%
\pgfsetroundjoin%
\definecolor{currentfill}{rgb}{0.000000,0.300000,1.000000}%
\pgfsetfillcolor{currentfill}%
\pgfsetlinewidth{0.000000pt}%
\definecolor{currentstroke}{rgb}{0.000000,0.000000,0.000000}%
\pgfsetstrokecolor{currentstroke}%
\pgfsetdash{}{0pt}%
\pgfpathmoveto{\pgfqpoint{1.453315in}{1.346073in}}%
\pgfpathlineto{\pgfqpoint{1.446511in}{1.404366in}}%
\pgfpathlineto{\pgfqpoint{1.414514in}{1.395577in}}%
\pgfpathlineto{\pgfqpoint{1.423641in}{1.337870in}}%
\pgfpathlineto{\pgfqpoint{1.453315in}{1.346073in}}%
\pgfpathclose%
\pgfusepath{fill}%
\end{pgfscope}%
\begin{pgfscope}%
\pgfpathrectangle{\pgfqpoint{0.000000in}{0.000000in}}{\pgfqpoint{3.000000in}{3.000000in}}%
\pgfusepath{clip}%
\pgfsetbuttcap%
\pgfsetroundjoin%
\definecolor{currentfill}{rgb}{1.000000,0.291213,0.000000}%
\pgfsetfillcolor{currentfill}%
\pgfsetlinewidth{0.000000pt}%
\definecolor{currentstroke}{rgb}{0.000000,0.000000,0.000000}%
\pgfsetstrokecolor{currentstroke}%
\pgfsetdash{}{0pt}%
\pgfpathmoveto{\pgfqpoint{1.759686in}{2.147684in}}%
\pgfpathlineto{\pgfqpoint{1.765544in}{2.170248in}}%
\pgfpathlineto{\pgfqpoint{1.670884in}{2.183278in}}%
\pgfpathlineto{\pgfqpoint{1.667496in}{2.160408in}}%
\pgfpathlineto{\pgfqpoint{1.759686in}{2.147684in}}%
\pgfpathclose%
\pgfusepath{fill}%
\end{pgfscope}%
\begin{pgfscope}%
\pgfpathrectangle{\pgfqpoint{0.000000in}{0.000000in}}{\pgfqpoint{3.000000in}{3.000000in}}%
\pgfusepath{clip}%
\pgfsetbuttcap%
\pgfsetroundjoin%
\definecolor{currentfill}{rgb}{0.000000,0.676471,1.000000}%
\pgfsetfillcolor{currentfill}%
\pgfsetlinewidth{0.000000pt}%
\definecolor{currentstroke}{rgb}{0.000000,0.000000,0.000000}%
\pgfsetstrokecolor{currentstroke}%
\pgfsetdash{}{0pt}%
\pgfpathmoveto{\pgfqpoint{1.553500in}{1.467985in}}%
\pgfpathlineto{\pgfqpoint{1.554372in}{1.517335in}}%
\pgfpathlineto{\pgfqpoint{1.512914in}{1.516791in}}%
\pgfpathlineto{\pgfqpoint{1.514656in}{1.467473in}}%
\pgfpathlineto{\pgfqpoint{1.553500in}{1.467985in}}%
\pgfpathclose%
\pgfusepath{fill}%
\end{pgfscope}%
\begin{pgfscope}%
\pgfpathrectangle{\pgfqpoint{0.000000in}{0.000000in}}{\pgfqpoint{3.000000in}{3.000000in}}%
\pgfusepath{clip}%
\pgfsetbuttcap%
\pgfsetroundjoin%
\definecolor{currentfill}{rgb}{0.000000,0.000000,0.838681}%
\pgfsetfillcolor{currentfill}%
\pgfsetlinewidth{0.000000pt}%
\definecolor{currentstroke}{rgb}{0.000000,0.000000,0.000000}%
\pgfsetstrokecolor{currentstroke}%
\pgfsetdash{}{0pt}%
\pgfpathmoveto{\pgfqpoint{1.692841in}{1.172938in}}%
\pgfpathlineto{\pgfqpoint{1.707001in}{1.243497in}}%
\pgfpathlineto{\pgfqpoint{1.687847in}{1.256226in}}%
\pgfpathlineto{\pgfqpoint{1.675331in}{1.184671in}}%
\pgfpathlineto{\pgfqpoint{1.692841in}{1.172938in}}%
\pgfpathclose%
\pgfusepath{fill}%
\end{pgfscope}%
\begin{pgfscope}%
\pgfpathrectangle{\pgfqpoint{0.000000in}{0.000000in}}{\pgfqpoint{3.000000in}{3.000000in}}%
\pgfusepath{clip}%
\pgfsetbuttcap%
\pgfsetroundjoin%
\definecolor{currentfill}{rgb}{0.000000,0.503922,1.000000}%
\pgfsetfillcolor{currentfill}%
\pgfsetlinewidth{0.000000pt}%
\definecolor{currentstroke}{rgb}{0.000000,0.000000,0.000000}%
\pgfsetstrokecolor{currentstroke}%
\pgfsetdash{}{0pt}%
\pgfpathmoveto{\pgfqpoint{1.480745in}{1.410504in}}%
\pgfpathlineto{\pgfqpoint{1.476425in}{1.463903in}}%
\pgfpathlineto{\pgfqpoint{1.439712in}{1.457358in}}%
\pgfpathlineto{\pgfqpoint{1.446511in}{1.404366in}}%
\pgfpathlineto{\pgfqpoint{1.480745in}{1.410504in}}%
\pgfpathclose%
\pgfusepath{fill}%
\end{pgfscope}%
\begin{pgfscope}%
\pgfpathrectangle{\pgfqpoint{0.000000in}{0.000000in}}{\pgfqpoint{3.000000in}{3.000000in}}%
\pgfusepath{clip}%
\pgfsetbuttcap%
\pgfsetroundjoin%
\definecolor{currentfill}{rgb}{0.678253,0.000000,0.000000}%
\pgfsetfillcolor{currentfill}%
\pgfsetlinewidth{0.000000pt}%
\definecolor{currentstroke}{rgb}{0.000000,0.000000,0.000000}%
\pgfsetstrokecolor{currentstroke}%
\pgfsetdash{}{0pt}%
\pgfpathmoveto{\pgfqpoint{1.231977in}{2.333135in}}%
\pgfpathlineto{\pgfqpoint{1.225380in}{2.353128in}}%
\pgfpathlineto{\pgfqpoint{1.117214in}{2.326554in}}%
\pgfpathlineto{\pgfqpoint{1.126102in}{2.307059in}}%
\pgfpathlineto{\pgfqpoint{1.231977in}{2.333135in}}%
\pgfpathclose%
\pgfusepath{fill}%
\end{pgfscope}%
\begin{pgfscope}%
\pgfpathrectangle{\pgfqpoint{0.000000in}{0.000000in}}{\pgfqpoint{3.000000in}{3.000000in}}%
\pgfusepath{clip}%
\pgfsetbuttcap%
\pgfsetroundjoin%
\definecolor{currentfill}{rgb}{0.856506,0.000000,0.000000}%
\pgfsetfillcolor{currentfill}%
\pgfsetlinewidth{0.000000pt}%
\definecolor{currentstroke}{rgb}{0.000000,0.000000,0.000000}%
\pgfsetstrokecolor{currentstroke}%
\pgfsetdash{}{0pt}%
\pgfpathmoveto{\pgfqpoint{1.896118in}{2.256243in}}%
\pgfpathlineto{\pgfqpoint{1.904282in}{2.276596in}}%
\pgfpathlineto{\pgfqpoint{1.800565in}{2.299177in}}%
\pgfpathlineto{\pgfqpoint{1.794743in}{2.278371in}}%
\pgfpathlineto{\pgfqpoint{1.896118in}{2.256243in}}%
\pgfpathclose%
\pgfusepath{fill}%
\end{pgfscope}%
\begin{pgfscope}%
\pgfpathrectangle{\pgfqpoint{0.000000in}{0.000000in}}{\pgfqpoint{3.000000in}{3.000000in}}%
\pgfusepath{clip}%
\pgfsetbuttcap%
\pgfsetroundjoin%
\definecolor{currentfill}{rgb}{1.000000,0.480029,0.000000}%
\pgfsetfillcolor{currentfill}%
\pgfsetlinewidth{0.000000pt}%
\definecolor{currentstroke}{rgb}{0.000000,0.000000,0.000000}%
\pgfsetstrokecolor{currentstroke}%
\pgfsetdash{}{0pt}%
\pgfpathmoveto{\pgfqpoint{1.481864in}{2.093842in}}%
\pgfpathlineto{\pgfqpoint{1.480156in}{2.117994in}}%
\pgfpathlineto{\pgfqpoint{1.390894in}{2.110236in}}%
\pgfpathlineto{\pgfqpoint{1.395129in}{2.086283in}}%
\pgfpathlineto{\pgfqpoint{1.481864in}{2.093842in}}%
\pgfpathclose%
\pgfusepath{fill}%
\end{pgfscope}%
\begin{pgfscope}%
\pgfpathrectangle{\pgfqpoint{0.000000in}{0.000000in}}{\pgfqpoint{3.000000in}{3.000000in}}%
\pgfusepath{clip}%
\pgfsetbuttcap%
\pgfsetroundjoin%
\definecolor{currentfill}{rgb}{0.000000,0.676471,1.000000}%
\pgfsetfillcolor{currentfill}%
\pgfsetlinewidth{0.000000pt}%
\definecolor{currentstroke}{rgb}{0.000000,0.000000,0.000000}%
\pgfsetstrokecolor{currentstroke}%
\pgfsetdash{}{0pt}%
\pgfpathmoveto{\pgfqpoint{1.592037in}{1.465429in}}%
\pgfpathlineto{\pgfqpoint{1.595503in}{1.514621in}}%
\pgfpathlineto{\pgfqpoint{1.554372in}{1.517335in}}%
\pgfpathlineto{\pgfqpoint{1.553500in}{1.467985in}}%
\pgfpathlineto{\pgfqpoint{1.592037in}{1.465429in}}%
\pgfpathclose%
\pgfusepath{fill}%
\end{pgfscope}%
\begin{pgfscope}%
\pgfpathrectangle{\pgfqpoint{0.000000in}{0.000000in}}{\pgfqpoint{3.000000in}{3.000000in}}%
\pgfusepath{clip}%
\pgfsetbuttcap%
\pgfsetroundjoin%
\definecolor{currentfill}{rgb}{1.000000,0.538126,0.000000}%
\pgfsetfillcolor{currentfill}%
\pgfsetlinewidth{0.000000pt}%
\definecolor{currentstroke}{rgb}{0.000000,0.000000,0.000000}%
\pgfsetstrokecolor{currentstroke}%
\pgfsetdash{}{0pt}%
\pgfpathmoveto{\pgfqpoint{1.653905in}{2.065030in}}%
\pgfpathlineto{\pgfqpoint{1.657308in}{2.089515in}}%
\pgfpathlineto{\pgfqpoint{1.569915in}{2.094926in}}%
\pgfpathlineto{\pgfqpoint{1.569060in}{2.070298in}}%
\pgfpathlineto{\pgfqpoint{1.653905in}{2.065030in}}%
\pgfpathclose%
\pgfusepath{fill}%
\end{pgfscope}%
\begin{pgfscope}%
\pgfpathrectangle{\pgfqpoint{0.000000in}{0.000000in}}{\pgfqpoint{3.000000in}{3.000000in}}%
\pgfusepath{clip}%
\pgfsetbuttcap%
\pgfsetroundjoin%
\definecolor{currentfill}{rgb}{1.000000,0.610748,0.000000}%
\pgfsetfillcolor{currentfill}%
\pgfsetlinewidth{0.000000pt}%
\definecolor{currentstroke}{rgb}{0.000000,0.000000,0.000000}%
\pgfsetstrokecolor{currentstroke}%
\pgfsetdash{}{0pt}%
\pgfpathmoveto{\pgfqpoint{1.568203in}{2.045185in}}%
\pgfpathlineto{\pgfqpoint{1.569060in}{2.070298in}}%
\pgfpathlineto{\pgfqpoint{1.483573in}{2.069242in}}%
\pgfpathlineto{\pgfqpoint{1.485285in}{2.044158in}}%
\pgfpathlineto{\pgfqpoint{1.568203in}{2.045185in}}%
\pgfpathclose%
\pgfusepath{fill}%
\end{pgfscope}%
\begin{pgfscope}%
\pgfpathrectangle{\pgfqpoint{0.000000in}{0.000000in}}{\pgfqpoint{3.000000in}{3.000000in}}%
\pgfusepath{clip}%
\pgfsetbuttcap%
\pgfsetroundjoin%
\definecolor{currentfill}{rgb}{1.000000,0.233115,0.000000}%
\pgfsetfillcolor{currentfill}%
\pgfsetlinewidth{0.000000pt}%
\definecolor{currentstroke}{rgb}{0.000000,0.000000,0.000000}%
\pgfsetstrokecolor{currentstroke}%
\pgfsetdash{}{0pt}%
\pgfpathmoveto{\pgfqpoint{1.378215in}{2.179708in}}%
\pgfpathlineto{\pgfqpoint{1.373998in}{2.202155in}}%
\pgfpathlineto{\pgfqpoint{1.278346in}{2.186465in}}%
\pgfpathlineto{\pgfqpoint{1.284997in}{2.164377in}}%
\pgfpathlineto{\pgfqpoint{1.378215in}{2.179708in}}%
\pgfpathclose%
\pgfusepath{fill}%
\end{pgfscope}%
\begin{pgfscope}%
\pgfpathrectangle{\pgfqpoint{0.000000in}{0.000000in}}{\pgfqpoint{3.000000in}{3.000000in}}%
\pgfusepath{clip}%
\pgfsetbuttcap%
\pgfsetroundjoin%
\definecolor{currentfill}{rgb}{0.000000,0.676471,1.000000}%
\pgfsetfillcolor{currentfill}%
\pgfsetlinewidth{0.000000pt}%
\definecolor{currentstroke}{rgb}{0.000000,0.000000,0.000000}%
\pgfsetstrokecolor{currentstroke}%
\pgfsetdash{}{0pt}%
\pgfpathmoveto{\pgfqpoint{1.514656in}{1.467473in}}%
\pgfpathlineto{\pgfqpoint{1.512914in}{1.516791in}}%
\pgfpathlineto{\pgfqpoint{1.472108in}{1.513001in}}%
\pgfpathlineto{\pgfqpoint{1.476425in}{1.463903in}}%
\pgfpathlineto{\pgfqpoint{1.514656in}{1.467473in}}%
\pgfpathclose%
\pgfusepath{fill}%
\end{pgfscope}%
\begin{pgfscope}%
\pgfpathrectangle{\pgfqpoint{0.000000in}{0.000000in}}{\pgfqpoint{3.000000in}{3.000000in}}%
\pgfusepath{clip}%
\pgfsetbuttcap%
\pgfsetroundjoin%
\definecolor{currentfill}{rgb}{1.000000,0.349310,0.000000}%
\pgfsetfillcolor{currentfill}%
\pgfsetlinewidth{0.000000pt}%
\definecolor{currentstroke}{rgb}{0.000000,0.000000,0.000000}%
\pgfsetstrokecolor{currentstroke}%
\pgfsetdash{}{0pt}%
\pgfpathmoveto{\pgfqpoint{1.753822in}{2.124758in}}%
\pgfpathlineto{\pgfqpoint{1.759686in}{2.147684in}}%
\pgfpathlineto{\pgfqpoint{1.667496in}{2.160408in}}%
\pgfpathlineto{\pgfqpoint{1.664104in}{2.137175in}}%
\pgfpathlineto{\pgfqpoint{1.753822in}{2.124758in}}%
\pgfpathclose%
\pgfusepath{fill}%
\end{pgfscope}%
\begin{pgfscope}%
\pgfpathrectangle{\pgfqpoint{0.000000in}{0.000000in}}{\pgfqpoint{3.000000in}{3.000000in}}%
\pgfusepath{clip}%
\pgfsetbuttcap%
\pgfsetroundjoin%
\definecolor{currentfill}{rgb}{1.000000,0.538126,0.000000}%
\pgfsetfillcolor{currentfill}%
\pgfsetlinewidth{0.000000pt}%
\definecolor{currentstroke}{rgb}{0.000000,0.000000,0.000000}%
\pgfsetstrokecolor{currentstroke}%
\pgfsetdash{}{0pt}%
\pgfpathmoveto{\pgfqpoint{1.483573in}{2.069242in}}%
\pgfpathlineto{\pgfqpoint{1.481864in}{2.093842in}}%
\pgfpathlineto{\pgfqpoint{1.395129in}{2.086283in}}%
\pgfpathlineto{\pgfqpoint{1.399369in}{2.061883in}}%
\pgfpathlineto{\pgfqpoint{1.483573in}{2.069242in}}%
\pgfpathclose%
\pgfusepath{fill}%
\end{pgfscope}%
\begin{pgfscope}%
\pgfpathrectangle{\pgfqpoint{0.000000in}{0.000000in}}{\pgfqpoint{3.000000in}{3.000000in}}%
\pgfusepath{clip}%
\pgfsetbuttcap%
\pgfsetroundjoin%
\definecolor{currentfill}{rgb}{0.731729,0.000000,0.000000}%
\pgfsetfillcolor{currentfill}%
\pgfsetlinewidth{0.000000pt}%
\definecolor{currentstroke}{rgb}{0.000000,0.000000,0.000000}%
\pgfsetstrokecolor{currentstroke}%
\pgfsetdash{}{0pt}%
\pgfpathmoveto{\pgfqpoint{1.238581in}{2.312927in}}%
\pgfpathlineto{\pgfqpoint{1.231977in}{2.333135in}}%
\pgfpathlineto{\pgfqpoint{1.126102in}{2.307059in}}%
\pgfpathlineto{\pgfqpoint{1.134999in}{2.287350in}}%
\pgfpathlineto{\pgfqpoint{1.238581in}{2.312927in}}%
\pgfpathclose%
\pgfusepath{fill}%
\end{pgfscope}%
\begin{pgfscope}%
\pgfpathrectangle{\pgfqpoint{0.000000in}{0.000000in}}{\pgfqpoint{3.000000in}{3.000000in}}%
\pgfusepath{clip}%
\pgfsetbuttcap%
\pgfsetroundjoin%
\definecolor{currentfill}{rgb}{0.927807,0.015251,0.000000}%
\pgfsetfillcolor{currentfill}%
\pgfsetlinewidth{0.000000pt}%
\definecolor{currentstroke}{rgb}{0.000000,0.000000,0.000000}%
\pgfsetstrokecolor{currentstroke}%
\pgfsetdash{}{0pt}%
\pgfpathmoveto{\pgfqpoint{1.887945in}{2.235640in}}%
\pgfpathlineto{\pgfqpoint{1.896118in}{2.256243in}}%
\pgfpathlineto{\pgfqpoint{1.794743in}{2.278371in}}%
\pgfpathlineto{\pgfqpoint{1.788915in}{2.257312in}}%
\pgfpathlineto{\pgfqpoint{1.887945in}{2.235640in}}%
\pgfpathclose%
\pgfusepath{fill}%
\end{pgfscope}%
\begin{pgfscope}%
\pgfpathrectangle{\pgfqpoint{0.000000in}{0.000000in}}{\pgfqpoint{3.000000in}{3.000000in}}%
\pgfusepath{clip}%
\pgfsetbuttcap%
\pgfsetroundjoin%
\definecolor{currentfill}{rgb}{1.000000,0.610748,0.000000}%
\pgfsetfillcolor{currentfill}%
\pgfsetlinewidth{0.000000pt}%
\definecolor{currentstroke}{rgb}{0.000000,0.000000,0.000000}%
\pgfsetstrokecolor{currentstroke}%
\pgfsetdash{}{0pt}%
\pgfpathmoveto{\pgfqpoint{1.650498in}{2.040061in}}%
\pgfpathlineto{\pgfqpoint{1.653905in}{2.065030in}}%
\pgfpathlineto{\pgfqpoint{1.569060in}{2.070298in}}%
\pgfpathlineto{\pgfqpoint{1.568203in}{2.045185in}}%
\pgfpathlineto{\pgfqpoint{1.650498in}{2.040061in}}%
\pgfpathclose%
\pgfusepath{fill}%
\end{pgfscope}%
\begin{pgfscope}%
\pgfpathrectangle{\pgfqpoint{0.000000in}{0.000000in}}{\pgfqpoint{3.000000in}{3.000000in}}%
\pgfusepath{clip}%
\pgfsetbuttcap%
\pgfsetroundjoin%
\definecolor{currentfill}{rgb}{1.000000,0.668845,0.000000}%
\pgfsetfillcolor{currentfill}%
\pgfsetlinewidth{0.000000pt}%
\definecolor{currentstroke}{rgb}{0.000000,0.000000,0.000000}%
\pgfsetstrokecolor{currentstroke}%
\pgfsetdash{}{0pt}%
\pgfpathmoveto{\pgfqpoint{1.567346in}{2.019551in}}%
\pgfpathlineto{\pgfqpoint{1.568203in}{2.045185in}}%
\pgfpathlineto{\pgfqpoint{1.485285in}{2.044158in}}%
\pgfpathlineto{\pgfqpoint{1.486998in}{2.018553in}}%
\pgfpathlineto{\pgfqpoint{1.567346in}{2.019551in}}%
\pgfpathclose%
\pgfusepath{fill}%
\end{pgfscope}%
\begin{pgfscope}%
\pgfpathrectangle{\pgfqpoint{0.000000in}{0.000000in}}{\pgfqpoint{3.000000in}{3.000000in}}%
\pgfusepath{clip}%
\pgfsetbuttcap%
\pgfsetroundjoin%
\definecolor{currentfill}{rgb}{0.000000,0.064706,1.000000}%
\pgfsetfillcolor{currentfill}%
\pgfsetlinewidth{0.000000pt}%
\definecolor{currentstroke}{rgb}{0.000000,0.000000,0.000000}%
\pgfsetstrokecolor{currentstroke}%
\pgfsetdash{}{0pt}%
\pgfpathmoveto{\pgfqpoint{1.687847in}{1.256226in}}%
\pgfpathlineto{\pgfqpoint{1.700357in}{1.319246in}}%
\pgfpathlineto{\pgfqpoint{1.675711in}{1.331120in}}%
\pgfpathlineto{\pgfqpoint{1.665142in}{1.267244in}}%
\pgfpathlineto{\pgfqpoint{1.687847in}{1.256226in}}%
\pgfpathclose%
\pgfusepath{fill}%
\end{pgfscope}%
\begin{pgfscope}%
\pgfpathrectangle{\pgfqpoint{0.000000in}{0.000000in}}{\pgfqpoint{3.000000in}{3.000000in}}%
\pgfusepath{clip}%
\pgfsetbuttcap%
\pgfsetroundjoin%
\definecolor{currentfill}{rgb}{0.000000,0.000000,0.838681}%
\pgfsetfillcolor{currentfill}%
\pgfsetlinewidth{0.000000pt}%
\definecolor{currentstroke}{rgb}{0.000000,0.000000,0.000000}%
\pgfsetstrokecolor{currentstroke}%
\pgfsetdash{}{0pt}%
\pgfpathmoveto{\pgfqpoint{1.399529in}{1.180922in}}%
\pgfpathlineto{\pgfqpoint{1.386430in}{1.252159in}}%
\pgfpathlineto{\pgfqpoint{1.368569in}{1.238924in}}%
\pgfpathlineto{\pgfqpoint{1.383204in}{1.168724in}}%
\pgfpathlineto{\pgfqpoint{1.399529in}{1.180922in}}%
\pgfpathclose%
\pgfusepath{fill}%
\end{pgfscope}%
\begin{pgfscope}%
\pgfpathrectangle{\pgfqpoint{0.000000in}{0.000000in}}{\pgfqpoint{3.000000in}{3.000000in}}%
\pgfusepath{clip}%
\pgfsetbuttcap%
\pgfsetroundjoin%
\definecolor{currentfill}{rgb}{1.000000,0.291213,0.000000}%
\pgfsetfillcolor{currentfill}%
\pgfsetlinewidth{0.000000pt}%
\definecolor{currentstroke}{rgb}{0.000000,0.000000,0.000000}%
\pgfsetstrokecolor{currentstroke}%
\pgfsetdash{}{0pt}%
\pgfpathmoveto{\pgfqpoint{1.382437in}{2.156922in}}%
\pgfpathlineto{\pgfqpoint{1.378215in}{2.179708in}}%
\pgfpathlineto{\pgfqpoint{1.284997in}{2.164377in}}%
\pgfpathlineto{\pgfqpoint{1.291654in}{2.141951in}}%
\pgfpathlineto{\pgfqpoint{1.382437in}{2.156922in}}%
\pgfpathclose%
\pgfusepath{fill}%
\end{pgfscope}%
\begin{pgfscope}%
\pgfpathrectangle{\pgfqpoint{0.000000in}{0.000000in}}{\pgfqpoint{3.000000in}{3.000000in}}%
\pgfusepath{clip}%
\pgfsetbuttcap%
\pgfsetroundjoin%
\definecolor{currentfill}{rgb}{0.000000,0.849020,1.000000}%
\pgfsetfillcolor{currentfill}%
\pgfsetlinewidth{0.000000pt}%
\definecolor{currentstroke}{rgb}{0.000000,0.000000,0.000000}%
\pgfsetstrokecolor{currentstroke}%
\pgfsetdash{}{0pt}%
\pgfpathmoveto{\pgfqpoint{1.554372in}{1.517335in}}%
\pgfpathlineto{\pgfqpoint{1.555243in}{1.563138in}}%
\pgfpathlineto{\pgfqpoint{1.511174in}{1.562562in}}%
\pgfpathlineto{\pgfqpoint{1.512914in}{1.516791in}}%
\pgfpathlineto{\pgfqpoint{1.554372in}{1.517335in}}%
\pgfpathclose%
\pgfusepath{fill}%
\end{pgfscope}%
\begin{pgfscope}%
\pgfpathrectangle{\pgfqpoint{0.000000in}{0.000000in}}{\pgfqpoint{3.000000in}{3.000000in}}%
\pgfusepath{clip}%
\pgfsetbuttcap%
\pgfsetroundjoin%
\definecolor{currentfill}{rgb}{0.000000,0.000000,0.500000}%
\pgfsetfillcolor{currentfill}%
\pgfsetlinewidth{0.000000pt}%
\definecolor{currentstroke}{rgb}{0.000000,0.000000,0.000000}%
\pgfsetstrokecolor{currentstroke}%
\pgfsetdash{}{0pt}%
\pgfpathmoveto{\pgfqpoint{1.700038in}{1.066410in}}%
\pgfpathlineto{\pgfqpoint{1.716438in}{1.145892in}}%
\pgfpathlineto{\pgfqpoint{1.706660in}{1.159907in}}%
\pgfpathlineto{\pgfqpoint{1.691192in}{1.079224in}}%
\pgfpathlineto{\pgfqpoint{1.700038in}{1.066410in}}%
\pgfpathclose%
\pgfusepath{fill}%
\end{pgfscope}%
\begin{pgfscope}%
\pgfpathrectangle{\pgfqpoint{0.000000in}{0.000000in}}{\pgfqpoint{3.000000in}{3.000000in}}%
\pgfusepath{clip}%
\pgfsetbuttcap%
\pgfsetroundjoin%
\definecolor{currentfill}{rgb}{0.000000,0.676471,1.000000}%
\pgfsetfillcolor{currentfill}%
\pgfsetlinewidth{0.000000pt}%
\definecolor{currentstroke}{rgb}{0.000000,0.000000,0.000000}%
\pgfsetstrokecolor{currentstroke}%
\pgfsetdash{}{0pt}%
\pgfpathmoveto{\pgfqpoint{1.629355in}{1.459862in}}%
\pgfpathlineto{\pgfqpoint{1.635337in}{1.508710in}}%
\pgfpathlineto{\pgfqpoint{1.595503in}{1.514621in}}%
\pgfpathlineto{\pgfqpoint{1.592037in}{1.465429in}}%
\pgfpathlineto{\pgfqpoint{1.629355in}{1.459862in}}%
\pgfpathclose%
\pgfusepath{fill}%
\end{pgfscope}%
\begin{pgfscope}%
\pgfpathrectangle{\pgfqpoint{0.000000in}{0.000000in}}{\pgfqpoint{3.000000in}{3.000000in}}%
\pgfusepath{clip}%
\pgfsetbuttcap%
\pgfsetroundjoin%
\definecolor{currentfill}{rgb}{1.000000,0.407407,0.000000}%
\pgfsetfillcolor{currentfill}%
\pgfsetlinewidth{0.000000pt}%
\definecolor{currentstroke}{rgb}{0.000000,0.000000,0.000000}%
\pgfsetstrokecolor{currentstroke}%
\pgfsetdash{}{0pt}%
\pgfpathmoveto{\pgfqpoint{1.747953in}{2.101446in}}%
\pgfpathlineto{\pgfqpoint{1.753822in}{2.124758in}}%
\pgfpathlineto{\pgfqpoint{1.664104in}{2.137175in}}%
\pgfpathlineto{\pgfqpoint{1.660708in}{2.113553in}}%
\pgfpathlineto{\pgfqpoint{1.747953in}{2.101446in}}%
\pgfpathclose%
\pgfusepath{fill}%
\end{pgfscope}%
\begin{pgfscope}%
\pgfpathrectangle{\pgfqpoint{0.000000in}{0.000000in}}{\pgfqpoint{3.000000in}{3.000000in}}%
\pgfusepath{clip}%
\pgfsetbuttcap%
\pgfsetroundjoin%
\definecolor{currentfill}{rgb}{0.000000,0.503922,1.000000}%
\pgfsetfillcolor{currentfill}%
\pgfsetlinewidth{0.000000pt}%
\definecolor{currentstroke}{rgb}{0.000000,0.000000,0.000000}%
\pgfsetstrokecolor{currentstroke}%
\pgfsetdash{}{0pt}%
\pgfpathmoveto{\pgfqpoint{1.656199in}{1.398789in}}%
\pgfpathlineto{\pgfqpoint{1.664567in}{1.451411in}}%
\pgfpathlineto{\pgfqpoint{1.629355in}{1.459862in}}%
\pgfpathlineto{\pgfqpoint{1.623368in}{1.406714in}}%
\pgfpathlineto{\pgfqpoint{1.656199in}{1.398789in}}%
\pgfpathclose%
\pgfusepath{fill}%
\end{pgfscope}%
\begin{pgfscope}%
\pgfpathrectangle{\pgfqpoint{0.000000in}{0.000000in}}{\pgfqpoint{3.000000in}{3.000000in}}%
\pgfusepath{clip}%
\pgfsetbuttcap%
\pgfsetroundjoin%
\definecolor{currentfill}{rgb}{0.000000,0.300000,1.000000}%
\pgfsetfillcolor{currentfill}%
\pgfsetlinewidth{0.000000pt}%
\definecolor{currentstroke}{rgb}{0.000000,0.000000,0.000000}%
\pgfsetstrokecolor{currentstroke}%
\pgfsetdash{}{0pt}%
\pgfpathmoveto{\pgfqpoint{1.675711in}{1.331120in}}%
\pgfpathlineto{\pgfqpoint{1.686273in}{1.388343in}}%
\pgfpathlineto{\pgfqpoint{1.656199in}{1.398789in}}%
\pgfpathlineto{\pgfqpoint{1.647825in}{1.340869in}}%
\pgfpathlineto{\pgfqpoint{1.675711in}{1.331120in}}%
\pgfpathclose%
\pgfusepath{fill}%
\end{pgfscope}%
\begin{pgfscope}%
\pgfpathrectangle{\pgfqpoint{0.000000in}{0.000000in}}{\pgfqpoint{3.000000in}{3.000000in}}%
\pgfusepath{clip}%
\pgfsetbuttcap%
\pgfsetroundjoin%
\definecolor{currentfill}{rgb}{0.000000,0.849020,1.000000}%
\pgfsetfillcolor{currentfill}%
\pgfsetlinewidth{0.000000pt}%
\definecolor{currentstroke}{rgb}{0.000000,0.000000,0.000000}%
\pgfsetstrokecolor{currentstroke}%
\pgfsetdash{}{0pt}%
\pgfpathmoveto{\pgfqpoint{1.595503in}{1.514621in}}%
\pgfpathlineto{\pgfqpoint{1.598966in}{1.560267in}}%
\pgfpathlineto{\pgfqpoint{1.555243in}{1.563138in}}%
\pgfpathlineto{\pgfqpoint{1.554372in}{1.517335in}}%
\pgfpathlineto{\pgfqpoint{1.595503in}{1.514621in}}%
\pgfpathclose%
\pgfusepath{fill}%
\end{pgfscope}%
\begin{pgfscope}%
\pgfpathrectangle{\pgfqpoint{0.000000in}{0.000000in}}{\pgfqpoint{3.000000in}{3.000000in}}%
\pgfusepath{clip}%
\pgfsetbuttcap%
\pgfsetroundjoin%
\definecolor{currentfill}{rgb}{1.000000,0.610748,0.000000}%
\pgfsetfillcolor{currentfill}%
\pgfsetlinewidth{0.000000pt}%
\definecolor{currentstroke}{rgb}{0.000000,0.000000,0.000000}%
\pgfsetstrokecolor{currentstroke}%
\pgfsetdash{}{0pt}%
\pgfpathmoveto{\pgfqpoint{1.485285in}{2.044158in}}%
\pgfpathlineto{\pgfqpoint{1.483573in}{2.069242in}}%
\pgfpathlineto{\pgfqpoint{1.399369in}{2.061883in}}%
\pgfpathlineto{\pgfqpoint{1.403613in}{2.037001in}}%
\pgfpathlineto{\pgfqpoint{1.485285in}{2.044158in}}%
\pgfpathclose%
\pgfusepath{fill}%
\end{pgfscope}%
\begin{pgfscope}%
\pgfpathrectangle{\pgfqpoint{0.000000in}{0.000000in}}{\pgfqpoint{3.000000in}{3.000000in}}%
\pgfusepath{clip}%
\pgfsetbuttcap%
\pgfsetroundjoin%
\definecolor{currentfill}{rgb}{1.000000,0.668845,0.000000}%
\pgfsetfillcolor{currentfill}%
\pgfsetlinewidth{0.000000pt}%
\definecolor{currentstroke}{rgb}{0.000000,0.000000,0.000000}%
\pgfsetstrokecolor{currentstroke}%
\pgfsetdash{}{0pt}%
\pgfpathmoveto{\pgfqpoint{1.647087in}{2.014571in}}%
\pgfpathlineto{\pgfqpoint{1.650498in}{2.040061in}}%
\pgfpathlineto{\pgfqpoint{1.568203in}{2.045185in}}%
\pgfpathlineto{\pgfqpoint{1.567346in}{2.019551in}}%
\pgfpathlineto{\pgfqpoint{1.647087in}{2.014571in}}%
\pgfpathclose%
\pgfusepath{fill}%
\end{pgfscope}%
\begin{pgfscope}%
\pgfpathrectangle{\pgfqpoint{0.000000in}{0.000000in}}{\pgfqpoint{3.000000in}{3.000000in}}%
\pgfusepath{clip}%
\pgfsetbuttcap%
\pgfsetroundjoin%
\definecolor{currentfill}{rgb}{1.000000,0.741467,0.000000}%
\pgfsetfillcolor{currentfill}%
\pgfsetlinewidth{0.000000pt}%
\definecolor{currentstroke}{rgb}{0.000000,0.000000,0.000000}%
\pgfsetstrokecolor{currentstroke}%
\pgfsetdash{}{0pt}%
\pgfpathmoveto{\pgfqpoint{1.566487in}{1.993350in}}%
\pgfpathlineto{\pgfqpoint{1.567346in}{2.019551in}}%
\pgfpathlineto{\pgfqpoint{1.486998in}{2.018553in}}%
\pgfpathlineto{\pgfqpoint{1.488713in}{1.992381in}}%
\pgfpathlineto{\pgfqpoint{1.566487in}{1.993350in}}%
\pgfpathclose%
\pgfusepath{fill}%
\end{pgfscope}%
\begin{pgfscope}%
\pgfpathrectangle{\pgfqpoint{0.000000in}{0.000000in}}{\pgfqpoint{3.000000in}{3.000000in}}%
\pgfusepath{clip}%
\pgfsetbuttcap%
\pgfsetroundjoin%
\definecolor{currentfill}{rgb}{0.803030,0.000000,0.000000}%
\pgfsetfillcolor{currentfill}%
\pgfsetlinewidth{0.000000pt}%
\definecolor{currentstroke}{rgb}{0.000000,0.000000,0.000000}%
\pgfsetstrokecolor{currentstroke}%
\pgfsetdash{}{0pt}%
\pgfpathmoveto{\pgfqpoint{1.245192in}{2.292493in}}%
\pgfpathlineto{\pgfqpoint{1.238581in}{2.312927in}}%
\pgfpathlineto{\pgfqpoint{1.134999in}{2.287350in}}%
\pgfpathlineto{\pgfqpoint{1.143903in}{2.267418in}}%
\pgfpathlineto{\pgfqpoint{1.245192in}{2.292493in}}%
\pgfpathclose%
\pgfusepath{fill}%
\end{pgfscope}%
\begin{pgfscope}%
\pgfpathrectangle{\pgfqpoint{0.000000in}{0.000000in}}{\pgfqpoint{3.000000in}{3.000000in}}%
\pgfusepath{clip}%
\pgfsetbuttcap%
\pgfsetroundjoin%
\definecolor{currentfill}{rgb}{0.999109,0.073348,0.000000}%
\pgfsetfillcolor{currentfill}%
\pgfsetlinewidth{0.000000pt}%
\definecolor{currentstroke}{rgb}{0.000000,0.000000,0.000000}%
\pgfsetstrokecolor{currentstroke}%
\pgfsetdash{}{0pt}%
\pgfpathmoveto{\pgfqpoint{1.879766in}{2.214773in}}%
\pgfpathlineto{\pgfqpoint{1.887945in}{2.235640in}}%
\pgfpathlineto{\pgfqpoint{1.788915in}{2.257312in}}%
\pgfpathlineto{\pgfqpoint{1.783082in}{2.235987in}}%
\pgfpathlineto{\pgfqpoint{1.879766in}{2.214773in}}%
\pgfpathclose%
\pgfusepath{fill}%
\end{pgfscope}%
\begin{pgfscope}%
\pgfpathrectangle{\pgfqpoint{0.000000in}{0.000000in}}{\pgfqpoint{3.000000in}{3.000000in}}%
\pgfusepath{clip}%
\pgfsetbuttcap%
\pgfsetroundjoin%
\definecolor{currentfill}{rgb}{0.000000,0.849020,1.000000}%
\pgfsetfillcolor{currentfill}%
\pgfsetlinewidth{0.000000pt}%
\definecolor{currentstroke}{rgb}{0.000000,0.000000,0.000000}%
\pgfsetstrokecolor{currentstroke}%
\pgfsetdash{}{0pt}%
\pgfpathmoveto{\pgfqpoint{1.512914in}{1.516791in}}%
\pgfpathlineto{\pgfqpoint{1.511174in}{1.562562in}}%
\pgfpathlineto{\pgfqpoint{1.467796in}{1.558553in}}%
\pgfpathlineto{\pgfqpoint{1.472108in}{1.513001in}}%
\pgfpathlineto{\pgfqpoint{1.512914in}{1.516791in}}%
\pgfpathclose%
\pgfusepath{fill}%
\end{pgfscope}%
\begin{pgfscope}%
\pgfpathrectangle{\pgfqpoint{0.000000in}{0.000000in}}{\pgfqpoint{3.000000in}{3.000000in}}%
\pgfusepath{clip}%
\pgfsetbuttcap%
\pgfsetroundjoin%
\definecolor{currentfill}{rgb}{0.000000,0.676471,1.000000}%
\pgfsetfillcolor{currentfill}%
\pgfsetlinewidth{0.000000pt}%
\definecolor{currentstroke}{rgb}{0.000000,0.000000,0.000000}%
\pgfsetstrokecolor{currentstroke}%
\pgfsetdash{}{0pt}%
\pgfpathmoveto{\pgfqpoint{1.476425in}{1.463903in}}%
\pgfpathlineto{\pgfqpoint{1.472108in}{1.513001in}}%
\pgfpathlineto{\pgfqpoint{1.432919in}{1.506050in}}%
\pgfpathlineto{\pgfqpoint{1.439712in}{1.457358in}}%
\pgfpathlineto{\pgfqpoint{1.476425in}{1.463903in}}%
\pgfpathclose%
\pgfusepath{fill}%
\end{pgfscope}%
\begin{pgfscope}%
\pgfpathrectangle{\pgfqpoint{0.000000in}{0.000000in}}{\pgfqpoint{3.000000in}{3.000000in}}%
\pgfusepath{clip}%
\pgfsetbuttcap%
\pgfsetroundjoin%
\definecolor{currentfill}{rgb}{1.000000,0.349310,0.000000}%
\pgfsetfillcolor{currentfill}%
\pgfsetlinewidth{0.000000pt}%
\definecolor{currentstroke}{rgb}{0.000000,0.000000,0.000000}%
\pgfsetstrokecolor{currentstroke}%
\pgfsetdash{}{0pt}%
\pgfpathmoveto{\pgfqpoint{1.386663in}{2.133773in}}%
\pgfpathlineto{\pgfqpoint{1.382437in}{2.156922in}}%
\pgfpathlineto{\pgfqpoint{1.291654in}{2.141951in}}%
\pgfpathlineto{\pgfqpoint{1.298318in}{2.119165in}}%
\pgfpathlineto{\pgfqpoint{1.386663in}{2.133773in}}%
\pgfpathclose%
\pgfusepath{fill}%
\end{pgfscope}%
\begin{pgfscope}%
\pgfpathrectangle{\pgfqpoint{0.000000in}{0.000000in}}{\pgfqpoint{3.000000in}{3.000000in}}%
\pgfusepath{clip}%
\pgfsetbuttcap%
\pgfsetroundjoin%
\definecolor{currentfill}{rgb}{1.000000,0.480029,0.000000}%
\pgfsetfillcolor{currentfill}%
\pgfsetlinewidth{0.000000pt}%
\definecolor{currentstroke}{rgb}{0.000000,0.000000,0.000000}%
\pgfsetstrokecolor{currentstroke}%
\pgfsetdash{}{0pt}%
\pgfpathmoveto{\pgfqpoint{1.742077in}{2.077718in}}%
\pgfpathlineto{\pgfqpoint{1.747953in}{2.101446in}}%
\pgfpathlineto{\pgfqpoint{1.660708in}{2.113553in}}%
\pgfpathlineto{\pgfqpoint{1.657308in}{2.089515in}}%
\pgfpathlineto{\pgfqpoint{1.742077in}{2.077718in}}%
\pgfpathclose%
\pgfusepath{fill}%
\end{pgfscope}%
\begin{pgfscope}%
\pgfpathrectangle{\pgfqpoint{0.000000in}{0.000000in}}{\pgfqpoint{3.000000in}{3.000000in}}%
\pgfusepath{clip}%
\pgfsetbuttcap%
\pgfsetroundjoin%
\definecolor{currentfill}{rgb}{0.000000,0.064706,1.000000}%
\pgfsetfillcolor{currentfill}%
\pgfsetlinewidth{0.000000pt}%
\definecolor{currentstroke}{rgb}{0.000000,0.000000,0.000000}%
\pgfsetstrokecolor{currentstroke}%
\pgfsetdash{}{0pt}%
\pgfpathmoveto{\pgfqpoint{1.408009in}{1.263779in}}%
\pgfpathlineto{\pgfqpoint{1.396763in}{1.327385in}}%
\pgfpathlineto{\pgfqpoint{1.373335in}{1.314862in}}%
\pgfpathlineto{\pgfqpoint{1.386430in}{1.252159in}}%
\pgfpathlineto{\pgfqpoint{1.408009in}{1.263779in}}%
\pgfpathclose%
\pgfusepath{fill}%
\end{pgfscope}%
\begin{pgfscope}%
\pgfpathrectangle{\pgfqpoint{0.000000in}{0.000000in}}{\pgfqpoint{3.000000in}{3.000000in}}%
\pgfusepath{clip}%
\pgfsetbuttcap%
\pgfsetroundjoin%
\definecolor{currentfill}{rgb}{0.000000,0.503922,1.000000}%
\pgfsetfillcolor{currentfill}%
\pgfsetlinewidth{0.000000pt}%
\definecolor{currentstroke}{rgb}{0.000000,0.000000,0.000000}%
\pgfsetstrokecolor{currentstroke}%
\pgfsetdash{}{0pt}%
\pgfpathmoveto{\pgfqpoint{1.446511in}{1.404366in}}%
\pgfpathlineto{\pgfqpoint{1.439712in}{1.457358in}}%
\pgfpathlineto{\pgfqpoint{1.405392in}{1.447985in}}%
\pgfpathlineto{\pgfqpoint{1.414514in}{1.395577in}}%
\pgfpathlineto{\pgfqpoint{1.446511in}{1.404366in}}%
\pgfpathclose%
\pgfusepath{fill}%
\end{pgfscope}%
\begin{pgfscope}%
\pgfpathrectangle{\pgfqpoint{0.000000in}{0.000000in}}{\pgfqpoint{3.000000in}{3.000000in}}%
\pgfusepath{clip}%
\pgfsetbuttcap%
\pgfsetroundjoin%
\definecolor{currentfill}{rgb}{1.000000,0.668845,0.000000}%
\pgfsetfillcolor{currentfill}%
\pgfsetlinewidth{0.000000pt}%
\definecolor{currentstroke}{rgb}{0.000000,0.000000,0.000000}%
\pgfsetstrokecolor{currentstroke}%
\pgfsetdash{}{0pt}%
\pgfpathmoveto{\pgfqpoint{1.486998in}{2.018553in}}%
\pgfpathlineto{\pgfqpoint{1.485285in}{2.044158in}}%
\pgfpathlineto{\pgfqpoint{1.403613in}{2.037001in}}%
\pgfpathlineto{\pgfqpoint{1.407861in}{2.011597in}}%
\pgfpathlineto{\pgfqpoint{1.486998in}{2.018553in}}%
\pgfpathclose%
\pgfusepath{fill}%
\end{pgfscope}%
\begin{pgfscope}%
\pgfpathrectangle{\pgfqpoint{0.000000in}{0.000000in}}{\pgfqpoint{3.000000in}{3.000000in}}%
\pgfusepath{clip}%
\pgfsetbuttcap%
\pgfsetroundjoin%
\definecolor{currentfill}{rgb}{1.000000,0.741467,0.000000}%
\pgfsetfillcolor{currentfill}%
\pgfsetlinewidth{0.000000pt}%
\definecolor{currentstroke}{rgb}{0.000000,0.000000,0.000000}%
\pgfsetstrokecolor{currentstroke}%
\pgfsetdash{}{0pt}%
\pgfpathmoveto{\pgfqpoint{1.643673in}{1.988516in}}%
\pgfpathlineto{\pgfqpoint{1.647087in}{2.014571in}}%
\pgfpathlineto{\pgfqpoint{1.567346in}{2.019551in}}%
\pgfpathlineto{\pgfqpoint{1.566487in}{1.993350in}}%
\pgfpathlineto{\pgfqpoint{1.643673in}{1.988516in}}%
\pgfpathclose%
\pgfusepath{fill}%
\end{pgfscope}%
\begin{pgfscope}%
\pgfpathrectangle{\pgfqpoint{0.000000in}{0.000000in}}{\pgfqpoint{3.000000in}{3.000000in}}%
\pgfusepath{clip}%
\pgfsetbuttcap%
\pgfsetroundjoin%
\definecolor{currentfill}{rgb}{1.000000,0.814089,0.000000}%
\pgfsetfillcolor{currentfill}%
\pgfsetlinewidth{0.000000pt}%
\definecolor{currentstroke}{rgb}{0.000000,0.000000,0.000000}%
\pgfsetstrokecolor{currentstroke}%
\pgfsetdash{}{0pt}%
\pgfpathmoveto{\pgfqpoint{1.565628in}{1.966533in}}%
\pgfpathlineto{\pgfqpoint{1.566487in}{1.993350in}}%
\pgfpathlineto{\pgfqpoint{1.488713in}{1.992381in}}%
\pgfpathlineto{\pgfqpoint{1.490430in}{1.965593in}}%
\pgfpathlineto{\pgfqpoint{1.565628in}{1.966533in}}%
\pgfpathclose%
\pgfusepath{fill}%
\end{pgfscope}%
\begin{pgfscope}%
\pgfpathrectangle{\pgfqpoint{0.000000in}{0.000000in}}{\pgfqpoint{3.000000in}{3.000000in}}%
\pgfusepath{clip}%
\pgfsetbuttcap%
\pgfsetroundjoin%
\definecolor{currentfill}{rgb}{0.085389,1.000000,0.882353}%
\pgfsetfillcolor{currentfill}%
\pgfsetlinewidth{0.000000pt}%
\definecolor{currentstroke}{rgb}{0.000000,0.000000,0.000000}%
\pgfsetstrokecolor{currentstroke}%
\pgfsetdash{}{0pt}%
\pgfpathmoveto{\pgfqpoint{1.555243in}{1.563138in}}%
\pgfpathlineto{\pgfqpoint{1.556113in}{1.605977in}}%
\pgfpathlineto{\pgfqpoint{1.509436in}{1.605370in}}%
\pgfpathlineto{\pgfqpoint{1.511174in}{1.562562in}}%
\pgfpathlineto{\pgfqpoint{1.555243in}{1.563138in}}%
\pgfpathclose%
\pgfusepath{fill}%
\end{pgfscope}%
\begin{pgfscope}%
\pgfpathrectangle{\pgfqpoint{0.000000in}{0.000000in}}{\pgfqpoint{3.000000in}{3.000000in}}%
\pgfusepath{clip}%
\pgfsetbuttcap%
\pgfsetroundjoin%
\definecolor{currentfill}{rgb}{0.000000,0.300000,1.000000}%
\pgfsetfillcolor{currentfill}%
\pgfsetlinewidth{0.000000pt}%
\definecolor{currentstroke}{rgb}{0.000000,0.000000,0.000000}%
\pgfsetstrokecolor{currentstroke}%
\pgfsetdash{}{0pt}%
\pgfpathmoveto{\pgfqpoint{1.423641in}{1.337870in}}%
\pgfpathlineto{\pgfqpoint{1.414514in}{1.395577in}}%
\pgfpathlineto{\pgfqpoint{1.385524in}{1.384340in}}%
\pgfpathlineto{\pgfqpoint{1.396763in}{1.327385in}}%
\pgfpathlineto{\pgfqpoint{1.423641in}{1.337870in}}%
\pgfpathclose%
\pgfusepath{fill}%
\end{pgfscope}%
\begin{pgfscope}%
\pgfpathrectangle{\pgfqpoint{0.000000in}{0.000000in}}{\pgfqpoint{3.000000in}{3.000000in}}%
\pgfusepath{clip}%
\pgfsetbuttcap%
\pgfsetroundjoin%
\definecolor{currentfill}{rgb}{0.856506,0.000000,0.000000}%
\pgfsetfillcolor{currentfill}%
\pgfsetlinewidth{0.000000pt}%
\definecolor{currentstroke}{rgb}{0.000000,0.000000,0.000000}%
\pgfsetstrokecolor{currentstroke}%
\pgfsetdash{}{0pt}%
\pgfpathmoveto{\pgfqpoint{1.251809in}{2.271820in}}%
\pgfpathlineto{\pgfqpoint{1.245192in}{2.292493in}}%
\pgfpathlineto{\pgfqpoint{1.143903in}{2.267418in}}%
\pgfpathlineto{\pgfqpoint{1.152815in}{2.247250in}}%
\pgfpathlineto{\pgfqpoint{1.251809in}{2.271820in}}%
\pgfpathclose%
\pgfusepath{fill}%
\end{pgfscope}%
\begin{pgfscope}%
\pgfpathrectangle{\pgfqpoint{0.000000in}{0.000000in}}{\pgfqpoint{3.000000in}{3.000000in}}%
\pgfusepath{clip}%
\pgfsetbuttcap%
\pgfsetroundjoin%
\definecolor{currentfill}{rgb}{1.000000,0.116921,0.000000}%
\pgfsetfillcolor{currentfill}%
\pgfsetlinewidth{0.000000pt}%
\definecolor{currentstroke}{rgb}{0.000000,0.000000,0.000000}%
\pgfsetstrokecolor{currentstroke}%
\pgfsetdash{}{0pt}%
\pgfpathmoveto{\pgfqpoint{1.871578in}{2.193626in}}%
\pgfpathlineto{\pgfqpoint{1.879766in}{2.214773in}}%
\pgfpathlineto{\pgfqpoint{1.783082in}{2.235987in}}%
\pgfpathlineto{\pgfqpoint{1.777242in}{2.214379in}}%
\pgfpathlineto{\pgfqpoint{1.871578in}{2.193626in}}%
\pgfpathclose%
\pgfusepath{fill}%
\end{pgfscope}%
\begin{pgfscope}%
\pgfpathrectangle{\pgfqpoint{0.000000in}{0.000000in}}{\pgfqpoint{3.000000in}{3.000000in}}%
\pgfusepath{clip}%
\pgfsetbuttcap%
\pgfsetroundjoin%
\definecolor{currentfill}{rgb}{0.000000,0.000000,0.500000}%
\pgfsetfillcolor{currentfill}%
\pgfsetlinewidth{0.000000pt}%
\definecolor{currentstroke}{rgb}{0.000000,0.000000,0.000000}%
\pgfsetstrokecolor{currentstroke}%
\pgfsetdash{}{0pt}%
\pgfpathmoveto{\pgfqpoint{1.386520in}{1.075038in}}%
\pgfpathlineto{\pgfqpoint{1.370698in}{1.155329in}}%
\pgfpathlineto{\pgfqpoint{1.362329in}{1.141059in}}%
\pgfpathlineto{\pgfqpoint{1.378953in}{1.061993in}}%
\pgfpathlineto{\pgfqpoint{1.386520in}{1.075038in}}%
\pgfpathclose%
\pgfusepath{fill}%
\end{pgfscope}%
\begin{pgfscope}%
\pgfpathrectangle{\pgfqpoint{0.000000in}{0.000000in}}{\pgfqpoint{3.000000in}{3.000000in}}%
\pgfusepath{clip}%
\pgfsetbuttcap%
\pgfsetroundjoin%
\definecolor{currentfill}{rgb}{0.085389,1.000000,0.882353}%
\pgfsetfillcolor{currentfill}%
\pgfsetlinewidth{0.000000pt}%
\definecolor{currentstroke}{rgb}{0.000000,0.000000,0.000000}%
\pgfsetstrokecolor{currentstroke}%
\pgfsetdash{}{0pt}%
\pgfpathmoveto{\pgfqpoint{1.598966in}{1.560267in}}%
\pgfpathlineto{\pgfqpoint{1.602426in}{1.602950in}}%
\pgfpathlineto{\pgfqpoint{1.556113in}{1.605977in}}%
\pgfpathlineto{\pgfqpoint{1.555243in}{1.563138in}}%
\pgfpathlineto{\pgfqpoint{1.598966in}{1.560267in}}%
\pgfpathclose%
\pgfusepath{fill}%
\end{pgfscope}%
\begin{pgfscope}%
\pgfpathrectangle{\pgfqpoint{0.000000in}{0.000000in}}{\pgfqpoint{3.000000in}{3.000000in}}%
\pgfusepath{clip}%
\pgfsetbuttcap%
\pgfsetroundjoin%
\definecolor{currentfill}{rgb}{1.000000,0.407407,0.000000}%
\pgfsetfillcolor{currentfill}%
\pgfsetlinewidth{0.000000pt}%
\definecolor{currentstroke}{rgb}{0.000000,0.000000,0.000000}%
\pgfsetstrokecolor{currentstroke}%
\pgfsetdash{}{0pt}%
\pgfpathmoveto{\pgfqpoint{1.390894in}{2.110236in}}%
\pgfpathlineto{\pgfqpoint{1.386663in}{2.133773in}}%
\pgfpathlineto{\pgfqpoint{1.298318in}{2.119165in}}%
\pgfpathlineto{\pgfqpoint{1.304988in}{2.095992in}}%
\pgfpathlineto{\pgfqpoint{1.390894in}{2.110236in}}%
\pgfpathclose%
\pgfusepath{fill}%
\end{pgfscope}%
\begin{pgfscope}%
\pgfpathrectangle{\pgfqpoint{0.000000in}{0.000000in}}{\pgfqpoint{3.000000in}{3.000000in}}%
\pgfusepath{clip}%
\pgfsetbuttcap%
\pgfsetroundjoin%
\definecolor{currentfill}{rgb}{1.000000,0.538126,0.000000}%
\pgfsetfillcolor{currentfill}%
\pgfsetlinewidth{0.000000pt}%
\definecolor{currentstroke}{rgb}{0.000000,0.000000,0.000000}%
\pgfsetstrokecolor{currentstroke}%
\pgfsetdash{}{0pt}%
\pgfpathmoveto{\pgfqpoint{1.736195in}{2.053545in}}%
\pgfpathlineto{\pgfqpoint{1.742077in}{2.077718in}}%
\pgfpathlineto{\pgfqpoint{1.657308in}{2.089515in}}%
\pgfpathlineto{\pgfqpoint{1.653905in}{2.065030in}}%
\pgfpathlineto{\pgfqpoint{1.736195in}{2.053545in}}%
\pgfpathclose%
\pgfusepath{fill}%
\end{pgfscope}%
\begin{pgfscope}%
\pgfpathrectangle{\pgfqpoint{0.000000in}{0.000000in}}{\pgfqpoint{3.000000in}{3.000000in}}%
\pgfusepath{clip}%
\pgfsetbuttcap%
\pgfsetroundjoin%
\definecolor{currentfill}{rgb}{0.000000,0.849020,1.000000}%
\pgfsetfillcolor{currentfill}%
\pgfsetlinewidth{0.000000pt}%
\definecolor{currentstroke}{rgb}{0.000000,0.000000,0.000000}%
\pgfsetstrokecolor{currentstroke}%
\pgfsetdash{}{0pt}%
\pgfpathmoveto{\pgfqpoint{1.635337in}{1.508710in}}%
\pgfpathlineto{\pgfqpoint{1.641313in}{1.554014in}}%
\pgfpathlineto{\pgfqpoint{1.598966in}{1.560267in}}%
\pgfpathlineto{\pgfqpoint{1.595503in}{1.514621in}}%
\pgfpathlineto{\pgfqpoint{1.635337in}{1.508710in}}%
\pgfpathclose%
\pgfusepath{fill}%
\end{pgfscope}%
\begin{pgfscope}%
\pgfpathrectangle{\pgfqpoint{0.000000in}{0.000000in}}{\pgfqpoint{3.000000in}{3.000000in}}%
\pgfusepath{clip}%
\pgfsetbuttcap%
\pgfsetroundjoin%
\definecolor{currentfill}{rgb}{1.000000,0.886710,0.000000}%
\pgfsetfillcolor{currentfill}%
\pgfsetlinewidth{0.000000pt}%
\definecolor{currentstroke}{rgb}{0.000000,0.000000,0.000000}%
\pgfsetstrokecolor{currentstroke}%
\pgfsetdash{}{0pt}%
\pgfpathmoveto{\pgfqpoint{1.564767in}{1.939043in}}%
\pgfpathlineto{\pgfqpoint{1.565628in}{1.966533in}}%
\pgfpathlineto{\pgfqpoint{1.490430in}{1.965593in}}%
\pgfpathlineto{\pgfqpoint{1.492148in}{1.938132in}}%
\pgfpathlineto{\pgfqpoint{1.564767in}{1.939043in}}%
\pgfpathclose%
\pgfusepath{fill}%
\end{pgfscope}%
\begin{pgfscope}%
\pgfpathrectangle{\pgfqpoint{0.000000in}{0.000000in}}{\pgfqpoint{3.000000in}{3.000000in}}%
\pgfusepath{clip}%
\pgfsetbuttcap%
\pgfsetroundjoin%
\definecolor{currentfill}{rgb}{1.000000,0.814089,0.000000}%
\pgfsetfillcolor{currentfill}%
\pgfsetlinewidth{0.000000pt}%
\definecolor{currentstroke}{rgb}{0.000000,0.000000,0.000000}%
\pgfsetstrokecolor{currentstroke}%
\pgfsetdash{}{0pt}%
\pgfpathmoveto{\pgfqpoint{1.640255in}{1.961845in}}%
\pgfpathlineto{\pgfqpoint{1.643673in}{1.988516in}}%
\pgfpathlineto{\pgfqpoint{1.566487in}{1.993350in}}%
\pgfpathlineto{\pgfqpoint{1.565628in}{1.966533in}}%
\pgfpathlineto{\pgfqpoint{1.640255in}{1.961845in}}%
\pgfpathclose%
\pgfusepath{fill}%
\end{pgfscope}%
\begin{pgfscope}%
\pgfpathrectangle{\pgfqpoint{0.000000in}{0.000000in}}{\pgfqpoint{3.000000in}{3.000000in}}%
\pgfusepath{clip}%
\pgfsetbuttcap%
\pgfsetroundjoin%
\definecolor{currentfill}{rgb}{1.000000,0.741467,0.000000}%
\pgfsetfillcolor{currentfill}%
\pgfsetlinewidth{0.000000pt}%
\definecolor{currentstroke}{rgb}{0.000000,0.000000,0.000000}%
\pgfsetstrokecolor{currentstroke}%
\pgfsetdash{}{0pt}%
\pgfpathmoveto{\pgfqpoint{1.488713in}{1.992381in}}%
\pgfpathlineto{\pgfqpoint{1.486998in}{2.018553in}}%
\pgfpathlineto{\pgfqpoint{1.407861in}{2.011597in}}%
\pgfpathlineto{\pgfqpoint{1.412115in}{1.985629in}}%
\pgfpathlineto{\pgfqpoint{1.488713in}{1.992381in}}%
\pgfpathclose%
\pgfusepath{fill}%
\end{pgfscope}%
\begin{pgfscope}%
\pgfpathrectangle{\pgfqpoint{0.000000in}{0.000000in}}{\pgfqpoint{3.000000in}{3.000000in}}%
\pgfusepath{clip}%
\pgfsetbuttcap%
\pgfsetroundjoin%
\definecolor{currentfill}{rgb}{0.085389,1.000000,0.882353}%
\pgfsetfillcolor{currentfill}%
\pgfsetlinewidth{0.000000pt}%
\definecolor{currentstroke}{rgb}{0.000000,0.000000,0.000000}%
\pgfsetstrokecolor{currentstroke}%
\pgfsetdash{}{0pt}%
\pgfpathmoveto{\pgfqpoint{1.511174in}{1.562562in}}%
\pgfpathlineto{\pgfqpoint{1.509436in}{1.605370in}}%
\pgfpathlineto{\pgfqpoint{1.463488in}{1.601143in}}%
\pgfpathlineto{\pgfqpoint{1.467796in}{1.558553in}}%
\pgfpathlineto{\pgfqpoint{1.511174in}{1.562562in}}%
\pgfpathclose%
\pgfusepath{fill}%
\end{pgfscope}%
\begin{pgfscope}%
\pgfpathrectangle{\pgfqpoint{0.000000in}{0.000000in}}{\pgfqpoint{3.000000in}{3.000000in}}%
\pgfusepath{clip}%
\pgfsetbuttcap%
\pgfsetroundjoin%
\definecolor{currentfill}{rgb}{0.199241,1.000000,0.768501}%
\pgfsetfillcolor{currentfill}%
\pgfsetlinewidth{0.000000pt}%
\definecolor{currentstroke}{rgb}{0.000000,0.000000,0.000000}%
\pgfsetstrokecolor{currentstroke}%
\pgfsetdash{}{0pt}%
\pgfpathmoveto{\pgfqpoint{1.556113in}{1.605977in}}%
\pgfpathlineto{\pgfqpoint{1.556983in}{1.646309in}}%
\pgfpathlineto{\pgfqpoint{1.507699in}{1.645671in}}%
\pgfpathlineto{\pgfqpoint{1.509436in}{1.605370in}}%
\pgfpathlineto{\pgfqpoint{1.556113in}{1.605977in}}%
\pgfpathclose%
\pgfusepath{fill}%
\end{pgfscope}%
\begin{pgfscope}%
\pgfpathrectangle{\pgfqpoint{0.000000in}{0.000000in}}{\pgfqpoint{3.000000in}{3.000000in}}%
\pgfusepath{clip}%
\pgfsetbuttcap%
\pgfsetroundjoin%
\definecolor{currentfill}{rgb}{1.000000,0.175018,0.000000}%
\pgfsetfillcolor{currentfill}%
\pgfsetlinewidth{0.000000pt}%
\definecolor{currentstroke}{rgb}{0.000000,0.000000,0.000000}%
\pgfsetstrokecolor{currentstroke}%
\pgfsetdash{}{0pt}%
\pgfpathmoveto{\pgfqpoint{1.863384in}{2.172182in}}%
\pgfpathlineto{\pgfqpoint{1.871578in}{2.193626in}}%
\pgfpathlineto{\pgfqpoint{1.777242in}{2.214379in}}%
\pgfpathlineto{\pgfqpoint{1.771396in}{2.192473in}}%
\pgfpathlineto{\pgfqpoint{1.863384in}{2.172182in}}%
\pgfpathclose%
\pgfusepath{fill}%
\end{pgfscope}%
\begin{pgfscope}%
\pgfpathrectangle{\pgfqpoint{0.000000in}{0.000000in}}{\pgfqpoint{3.000000in}{3.000000in}}%
\pgfusepath{clip}%
\pgfsetbuttcap%
\pgfsetroundjoin%
\definecolor{currentfill}{rgb}{0.927807,0.015251,0.000000}%
\pgfsetfillcolor{currentfill}%
\pgfsetlinewidth{0.000000pt}%
\definecolor{currentstroke}{rgb}{0.000000,0.000000,0.000000}%
\pgfsetstrokecolor{currentstroke}%
\pgfsetdash{}{0pt}%
\pgfpathmoveto{\pgfqpoint{1.258433in}{2.250896in}}%
\pgfpathlineto{\pgfqpoint{1.251809in}{2.271820in}}%
\pgfpathlineto{\pgfqpoint{1.152815in}{2.247250in}}%
\pgfpathlineto{\pgfqpoint{1.161736in}{2.226833in}}%
\pgfpathlineto{\pgfqpoint{1.258433in}{2.250896in}}%
\pgfpathclose%
\pgfusepath{fill}%
\end{pgfscope}%
\begin{pgfscope}%
\pgfpathrectangle{\pgfqpoint{0.000000in}{0.000000in}}{\pgfqpoint{3.000000in}{3.000000in}}%
\pgfusepath{clip}%
\pgfsetbuttcap%
\pgfsetroundjoin%
\definecolor{currentfill}{rgb}{0.958254,0.973856,0.009488}%
\pgfsetfillcolor{currentfill}%
\pgfsetlinewidth{0.000000pt}%
\definecolor{currentstroke}{rgb}{0.000000,0.000000,0.000000}%
\pgfsetstrokecolor{currentstroke}%
\pgfsetdash{}{0pt}%
\pgfpathmoveto{\pgfqpoint{1.563906in}{1.910815in}}%
\pgfpathlineto{\pgfqpoint{1.564767in}{1.939043in}}%
\pgfpathlineto{\pgfqpoint{1.492148in}{1.938132in}}%
\pgfpathlineto{\pgfqpoint{1.493869in}{1.909934in}}%
\pgfpathlineto{\pgfqpoint{1.563906in}{1.910815in}}%
\pgfpathclose%
\pgfusepath{fill}%
\end{pgfscope}%
\begin{pgfscope}%
\pgfpathrectangle{\pgfqpoint{0.000000in}{0.000000in}}{\pgfqpoint{3.000000in}{3.000000in}}%
\pgfusepath{clip}%
\pgfsetbuttcap%
\pgfsetroundjoin%
\definecolor{currentfill}{rgb}{0.000000,0.676471,1.000000}%
\pgfsetfillcolor{currentfill}%
\pgfsetlinewidth{0.000000pt}%
\definecolor{currentstroke}{rgb}{0.000000,0.000000,0.000000}%
\pgfsetstrokecolor{currentstroke}%
\pgfsetdash{}{0pt}%
\pgfpathmoveto{\pgfqpoint{1.664567in}{1.451411in}}%
\pgfpathlineto{\pgfqpoint{1.672928in}{1.499736in}}%
\pgfpathlineto{\pgfqpoint{1.635337in}{1.508710in}}%
\pgfpathlineto{\pgfqpoint{1.629355in}{1.459862in}}%
\pgfpathlineto{\pgfqpoint{1.664567in}{1.451411in}}%
\pgfpathclose%
\pgfusepath{fill}%
\end{pgfscope}%
\begin{pgfscope}%
\pgfpathrectangle{\pgfqpoint{0.000000in}{0.000000in}}{\pgfqpoint{3.000000in}{3.000000in}}%
\pgfusepath{clip}%
\pgfsetbuttcap%
\pgfsetroundjoin%
\definecolor{currentfill}{rgb}{1.000000,0.886710,0.000000}%
\pgfsetfillcolor{currentfill}%
\pgfsetlinewidth{0.000000pt}%
\definecolor{currentstroke}{rgb}{0.000000,0.000000,0.000000}%
\pgfsetstrokecolor{currentstroke}%
\pgfsetdash{}{0pt}%
\pgfpathmoveto{\pgfqpoint{1.636834in}{1.934502in}}%
\pgfpathlineto{\pgfqpoint{1.640255in}{1.961845in}}%
\pgfpathlineto{\pgfqpoint{1.565628in}{1.966533in}}%
\pgfpathlineto{\pgfqpoint{1.564767in}{1.939043in}}%
\pgfpathlineto{\pgfqpoint{1.636834in}{1.934502in}}%
\pgfpathclose%
\pgfusepath{fill}%
\end{pgfscope}%
\begin{pgfscope}%
\pgfpathrectangle{\pgfqpoint{0.000000in}{0.000000in}}{\pgfqpoint{3.000000in}{3.000000in}}%
\pgfusepath{clip}%
\pgfsetbuttcap%
\pgfsetroundjoin%
\definecolor{currentfill}{rgb}{1.000000,0.480029,0.000000}%
\pgfsetfillcolor{currentfill}%
\pgfsetlinewidth{0.000000pt}%
\definecolor{currentstroke}{rgb}{0.000000,0.000000,0.000000}%
\pgfsetstrokecolor{currentstroke}%
\pgfsetdash{}{0pt}%
\pgfpathmoveto{\pgfqpoint{1.395129in}{2.086283in}}%
\pgfpathlineto{\pgfqpoint{1.390894in}{2.110236in}}%
\pgfpathlineto{\pgfqpoint{1.304988in}{2.095992in}}%
\pgfpathlineto{\pgfqpoint{1.311665in}{2.072405in}}%
\pgfpathlineto{\pgfqpoint{1.395129in}{2.086283in}}%
\pgfpathclose%
\pgfusepath{fill}%
\end{pgfscope}%
\begin{pgfscope}%
\pgfpathrectangle{\pgfqpoint{0.000000in}{0.000000in}}{\pgfqpoint{3.000000in}{3.000000in}}%
\pgfusepath{clip}%
\pgfsetbuttcap%
\pgfsetroundjoin%
\definecolor{currentfill}{rgb}{1.000000,0.814089,0.000000}%
\pgfsetfillcolor{currentfill}%
\pgfsetlinewidth{0.000000pt}%
\definecolor{currentstroke}{rgb}{0.000000,0.000000,0.000000}%
\pgfsetstrokecolor{currentstroke}%
\pgfsetdash{}{0pt}%
\pgfpathmoveto{\pgfqpoint{1.490430in}{1.965593in}}%
\pgfpathlineto{\pgfqpoint{1.488713in}{1.992381in}}%
\pgfpathlineto{\pgfqpoint{1.412115in}{1.985629in}}%
\pgfpathlineto{\pgfqpoint{1.416372in}{1.959045in}}%
\pgfpathlineto{\pgfqpoint{1.490430in}{1.965593in}}%
\pgfpathclose%
\pgfusepath{fill}%
\end{pgfscope}%
\begin{pgfscope}%
\pgfpathrectangle{\pgfqpoint{0.000000in}{0.000000in}}{\pgfqpoint{3.000000in}{3.000000in}}%
\pgfusepath{clip}%
\pgfsetbuttcap%
\pgfsetroundjoin%
\definecolor{currentfill}{rgb}{0.000000,0.849020,1.000000}%
\pgfsetfillcolor{currentfill}%
\pgfsetlinewidth{0.000000pt}%
\definecolor{currentstroke}{rgb}{0.000000,0.000000,0.000000}%
\pgfsetstrokecolor{currentstroke}%
\pgfsetdash{}{0pt}%
\pgfpathmoveto{\pgfqpoint{1.472108in}{1.513001in}}%
\pgfpathlineto{\pgfqpoint{1.467796in}{1.558553in}}%
\pgfpathlineto{\pgfqpoint{1.426132in}{1.551201in}}%
\pgfpathlineto{\pgfqpoint{1.432919in}{1.506050in}}%
\pgfpathlineto{\pgfqpoint{1.472108in}{1.513001in}}%
\pgfpathclose%
\pgfusepath{fill}%
\end{pgfscope}%
\begin{pgfscope}%
\pgfpathrectangle{\pgfqpoint{0.000000in}{0.000000in}}{\pgfqpoint{3.000000in}{3.000000in}}%
\pgfusepath{clip}%
\pgfsetbuttcap%
\pgfsetroundjoin%
\definecolor{currentfill}{rgb}{1.000000,0.610748,0.000000}%
\pgfsetfillcolor{currentfill}%
\pgfsetlinewidth{0.000000pt}%
\definecolor{currentstroke}{rgb}{0.000000,0.000000,0.000000}%
\pgfsetstrokecolor{currentstroke}%
\pgfsetdash{}{0pt}%
\pgfpathmoveto{\pgfqpoint{1.730308in}{2.028891in}}%
\pgfpathlineto{\pgfqpoint{1.736195in}{2.053545in}}%
\pgfpathlineto{\pgfqpoint{1.653905in}{2.065030in}}%
\pgfpathlineto{\pgfqpoint{1.650498in}{2.040061in}}%
\pgfpathlineto{\pgfqpoint{1.730308in}{2.028891in}}%
\pgfpathclose%
\pgfusepath{fill}%
\end{pgfscope}%
\begin{pgfscope}%
\pgfpathrectangle{\pgfqpoint{0.000000in}{0.000000in}}{\pgfqpoint{3.000000in}{3.000000in}}%
\pgfusepath{clip}%
\pgfsetbuttcap%
\pgfsetroundjoin%
\definecolor{currentfill}{rgb}{0.199241,1.000000,0.768501}%
\pgfsetfillcolor{currentfill}%
\pgfsetlinewidth{0.000000pt}%
\definecolor{currentstroke}{rgb}{0.000000,0.000000,0.000000}%
\pgfsetstrokecolor{currentstroke}%
\pgfsetdash{}{0pt}%
\pgfpathmoveto{\pgfqpoint{1.602426in}{1.602950in}}%
\pgfpathlineto{\pgfqpoint{1.605882in}{1.643126in}}%
\pgfpathlineto{\pgfqpoint{1.556983in}{1.646309in}}%
\pgfpathlineto{\pgfqpoint{1.556113in}{1.605977in}}%
\pgfpathlineto{\pgfqpoint{1.602426in}{1.602950in}}%
\pgfpathclose%
\pgfusepath{fill}%
\end{pgfscope}%
\begin{pgfscope}%
\pgfpathrectangle{\pgfqpoint{0.000000in}{0.000000in}}{\pgfqpoint{3.000000in}{3.000000in}}%
\pgfusepath{clip}%
\pgfsetbuttcap%
\pgfsetroundjoin%
\definecolor{currentfill}{rgb}{0.000000,0.000000,0.838681}%
\pgfsetfillcolor{currentfill}%
\pgfsetlinewidth{0.000000pt}%
\definecolor{currentstroke}{rgb}{0.000000,0.000000,0.000000}%
\pgfsetstrokecolor{currentstroke}%
\pgfsetdash{}{0pt}%
\pgfpathmoveto{\pgfqpoint{1.706660in}{1.159907in}}%
\pgfpathlineto{\pgfqpoint{1.722128in}{1.229356in}}%
\pgfpathlineto{\pgfqpoint{1.707001in}{1.243497in}}%
\pgfpathlineto{\pgfqpoint{1.692841in}{1.172938in}}%
\pgfpathlineto{\pgfqpoint{1.706660in}{1.159907in}}%
\pgfpathclose%
\pgfusepath{fill}%
\end{pgfscope}%
\begin{pgfscope}%
\pgfpathrectangle{\pgfqpoint{0.000000in}{0.000000in}}{\pgfqpoint{3.000000in}{3.000000in}}%
\pgfusepath{clip}%
\pgfsetbuttcap%
\pgfsetroundjoin%
\definecolor{currentfill}{rgb}{0.300443,1.000000,0.667299}%
\pgfsetfillcolor{currentfill}%
\pgfsetlinewidth{0.000000pt}%
\definecolor{currentstroke}{rgb}{0.000000,0.000000,0.000000}%
\pgfsetstrokecolor{currentstroke}%
\pgfsetdash{}{0pt}%
\pgfpathmoveto{\pgfqpoint{1.556983in}{1.646309in}}%
\pgfpathlineto{\pgfqpoint{1.557851in}{1.684495in}}%
\pgfpathlineto{\pgfqpoint{1.505964in}{1.683826in}}%
\pgfpathlineto{\pgfqpoint{1.507699in}{1.645671in}}%
\pgfpathlineto{\pgfqpoint{1.556983in}{1.646309in}}%
\pgfpathclose%
\pgfusepath{fill}%
\end{pgfscope}%
\begin{pgfscope}%
\pgfpathrectangle{\pgfqpoint{0.000000in}{0.000000in}}{\pgfqpoint{3.000000in}{3.000000in}}%
\pgfusepath{clip}%
\pgfsetbuttcap%
\pgfsetroundjoin%
\definecolor{currentfill}{rgb}{0.895003,1.000000,0.072739}%
\pgfsetfillcolor{currentfill}%
\pgfsetlinewidth{0.000000pt}%
\definecolor{currentstroke}{rgb}{0.000000,0.000000,0.000000}%
\pgfsetstrokecolor{currentstroke}%
\pgfsetdash{}{0pt}%
\pgfpathmoveto{\pgfqpoint{1.563044in}{1.881775in}}%
\pgfpathlineto{\pgfqpoint{1.563906in}{1.910815in}}%
\pgfpathlineto{\pgfqpoint{1.493869in}{1.909934in}}%
\pgfpathlineto{\pgfqpoint{1.495591in}{1.880924in}}%
\pgfpathlineto{\pgfqpoint{1.563044in}{1.881775in}}%
\pgfpathclose%
\pgfusepath{fill}%
\end{pgfscope}%
\begin{pgfscope}%
\pgfpathrectangle{\pgfqpoint{0.000000in}{0.000000in}}{\pgfqpoint{3.000000in}{3.000000in}}%
\pgfusepath{clip}%
\pgfsetbuttcap%
\pgfsetroundjoin%
\definecolor{currentfill}{rgb}{0.500000,0.000000,0.000000}%
\pgfsetfillcolor{currentfill}%
\pgfsetlinewidth{0.000000pt}%
\definecolor{currentstroke}{rgb}{0.000000,0.000000,0.000000}%
\pgfsetstrokecolor{currentstroke}%
\pgfsetdash{}{0pt}%
\pgfpathmoveto{\pgfqpoint{2.051560in}{2.342169in}}%
\pgfpathlineto{\pgfqpoint{2.061858in}{2.360678in}}%
\pgfpathlineto{\pgfqpoint{1.953109in}{2.394135in}}%
\pgfpathlineto{\pgfqpoint{1.944991in}{2.375038in}}%
\pgfpathlineto{\pgfqpoint{2.051560in}{2.342169in}}%
\pgfpathclose%
\pgfusepath{fill}%
\end{pgfscope}%
\begin{pgfscope}%
\pgfpathrectangle{\pgfqpoint{0.000000in}{0.000000in}}{\pgfqpoint{3.000000in}{3.000000in}}%
\pgfusepath{clip}%
\pgfsetbuttcap%
\pgfsetroundjoin%
\definecolor{currentfill}{rgb}{0.199241,1.000000,0.768501}%
\pgfsetfillcolor{currentfill}%
\pgfsetlinewidth{0.000000pt}%
\definecolor{currentstroke}{rgb}{0.000000,0.000000,0.000000}%
\pgfsetstrokecolor{currentstroke}%
\pgfsetdash{}{0pt}%
\pgfpathmoveto{\pgfqpoint{1.509436in}{1.605370in}}%
\pgfpathlineto{\pgfqpoint{1.507699in}{1.645671in}}%
\pgfpathlineto{\pgfqpoint{1.459184in}{1.641227in}}%
\pgfpathlineto{\pgfqpoint{1.463488in}{1.601143in}}%
\pgfpathlineto{\pgfqpoint{1.509436in}{1.605370in}}%
\pgfpathclose%
\pgfusepath{fill}%
\end{pgfscope}%
\begin{pgfscope}%
\pgfpathrectangle{\pgfqpoint{0.000000in}{0.000000in}}{\pgfqpoint{3.000000in}{3.000000in}}%
\pgfusepath{clip}%
\pgfsetbuttcap%
\pgfsetroundjoin%
\definecolor{currentfill}{rgb}{0.958254,0.973856,0.009488}%
\pgfsetfillcolor{currentfill}%
\pgfsetlinewidth{0.000000pt}%
\definecolor{currentstroke}{rgb}{0.000000,0.000000,0.000000}%
\pgfsetstrokecolor{currentstroke}%
\pgfsetdash{}{0pt}%
\pgfpathmoveto{\pgfqpoint{1.633408in}{1.906421in}}%
\pgfpathlineto{\pgfqpoint{1.636834in}{1.934502in}}%
\pgfpathlineto{\pgfqpoint{1.564767in}{1.939043in}}%
\pgfpathlineto{\pgfqpoint{1.563906in}{1.910815in}}%
\pgfpathlineto{\pgfqpoint{1.633408in}{1.906421in}}%
\pgfpathclose%
\pgfusepath{fill}%
\end{pgfscope}%
\begin{pgfscope}%
\pgfpathrectangle{\pgfqpoint{0.000000in}{0.000000in}}{\pgfqpoint{3.000000in}{3.000000in}}%
\pgfusepath{clip}%
\pgfsetbuttcap%
\pgfsetroundjoin%
\definecolor{currentfill}{rgb}{1.000000,0.886710,0.000000}%
\pgfsetfillcolor{currentfill}%
\pgfsetlinewidth{0.000000pt}%
\definecolor{currentstroke}{rgb}{0.000000,0.000000,0.000000}%
\pgfsetstrokecolor{currentstroke}%
\pgfsetdash{}{0pt}%
\pgfpathmoveto{\pgfqpoint{1.492148in}{1.938132in}}%
\pgfpathlineto{\pgfqpoint{1.490430in}{1.965593in}}%
\pgfpathlineto{\pgfqpoint{1.416372in}{1.959045in}}%
\pgfpathlineto{\pgfqpoint{1.420634in}{1.931790in}}%
\pgfpathlineto{\pgfqpoint{1.492148in}{1.938132in}}%
\pgfpathclose%
\pgfusepath{fill}%
\end{pgfscope}%
\begin{pgfscope}%
\pgfpathrectangle{\pgfqpoint{0.000000in}{0.000000in}}{\pgfqpoint{3.000000in}{3.000000in}}%
\pgfusepath{clip}%
\pgfsetbuttcap%
\pgfsetroundjoin%
\definecolor{currentfill}{rgb}{0.085389,1.000000,0.882353}%
\pgfsetfillcolor{currentfill}%
\pgfsetlinewidth{0.000000pt}%
\definecolor{currentstroke}{rgb}{0.000000,0.000000,0.000000}%
\pgfsetstrokecolor{currentstroke}%
\pgfsetdash{}{0pt}%
\pgfpathmoveto{\pgfqpoint{1.641313in}{1.554014in}}%
\pgfpathlineto{\pgfqpoint{1.647285in}{1.596357in}}%
\pgfpathlineto{\pgfqpoint{1.602426in}{1.602950in}}%
\pgfpathlineto{\pgfqpoint{1.598966in}{1.560267in}}%
\pgfpathlineto{\pgfqpoint{1.641313in}{1.554014in}}%
\pgfpathclose%
\pgfusepath{fill}%
\end{pgfscope}%
\begin{pgfscope}%
\pgfpathrectangle{\pgfqpoint{0.000000in}{0.000000in}}{\pgfqpoint{3.000000in}{3.000000in}}%
\pgfusepath{clip}%
\pgfsetbuttcap%
\pgfsetroundjoin%
\definecolor{currentfill}{rgb}{1.000000,0.233115,0.000000}%
\pgfsetfillcolor{currentfill}%
\pgfsetlinewidth{0.000000pt}%
\definecolor{currentstroke}{rgb}{0.000000,0.000000,0.000000}%
\pgfsetstrokecolor{currentstroke}%
\pgfsetdash{}{0pt}%
\pgfpathmoveto{\pgfqpoint{1.855182in}{2.150422in}}%
\pgfpathlineto{\pgfqpoint{1.863384in}{2.172182in}}%
\pgfpathlineto{\pgfqpoint{1.771396in}{2.192473in}}%
\pgfpathlineto{\pgfqpoint{1.765544in}{2.170248in}}%
\pgfpathlineto{\pgfqpoint{1.855182in}{2.150422in}}%
\pgfpathclose%
\pgfusepath{fill}%
\end{pgfscope}%
\begin{pgfscope}%
\pgfpathrectangle{\pgfqpoint{0.000000in}{0.000000in}}{\pgfqpoint{3.000000in}{3.000000in}}%
\pgfusepath{clip}%
\pgfsetbuttcap%
\pgfsetroundjoin%
\definecolor{currentfill}{rgb}{0.000000,0.503922,1.000000}%
\pgfsetfillcolor{currentfill}%
\pgfsetlinewidth{0.000000pt}%
\definecolor{currentstroke}{rgb}{0.000000,0.000000,0.000000}%
\pgfsetstrokecolor{currentstroke}%
\pgfsetdash{}{0pt}%
\pgfpathmoveto{\pgfqpoint{1.686273in}{1.388343in}}%
\pgfpathlineto{\pgfqpoint{1.696829in}{1.440270in}}%
\pgfpathlineto{\pgfqpoint{1.664567in}{1.451411in}}%
\pgfpathlineto{\pgfqpoint{1.656199in}{1.398789in}}%
\pgfpathlineto{\pgfqpoint{1.686273in}{1.388343in}}%
\pgfpathclose%
\pgfusepath{fill}%
\end{pgfscope}%
\begin{pgfscope}%
\pgfpathrectangle{\pgfqpoint{0.000000in}{0.000000in}}{\pgfqpoint{3.000000in}{3.000000in}}%
\pgfusepath{clip}%
\pgfsetbuttcap%
\pgfsetroundjoin%
\definecolor{currentfill}{rgb}{0.819102,1.000000,0.148640}%
\pgfsetfillcolor{currentfill}%
\pgfsetlinewidth{0.000000pt}%
\definecolor{currentstroke}{rgb}{0.000000,0.000000,0.000000}%
\pgfsetstrokecolor{currentstroke}%
\pgfsetdash{}{0pt}%
\pgfpathmoveto{\pgfqpoint{1.562181in}{1.851837in}}%
\pgfpathlineto{\pgfqpoint{1.563044in}{1.881775in}}%
\pgfpathlineto{\pgfqpoint{1.495591in}{1.880924in}}%
\pgfpathlineto{\pgfqpoint{1.497316in}{1.851016in}}%
\pgfpathlineto{\pgfqpoint{1.562181in}{1.851837in}}%
\pgfpathclose%
\pgfusepath{fill}%
\end{pgfscope}%
\begin{pgfscope}%
\pgfpathrectangle{\pgfqpoint{0.000000in}{0.000000in}}{\pgfqpoint{3.000000in}{3.000000in}}%
\pgfusepath{clip}%
\pgfsetbuttcap%
\pgfsetroundjoin%
\definecolor{currentfill}{rgb}{0.999109,0.073348,0.000000}%
\pgfsetfillcolor{currentfill}%
\pgfsetlinewidth{0.000000pt}%
\definecolor{currentstroke}{rgb}{0.000000,0.000000,0.000000}%
\pgfsetstrokecolor{currentstroke}%
\pgfsetdash{}{0pt}%
\pgfpathmoveto{\pgfqpoint{1.265064in}{2.229706in}}%
\pgfpathlineto{\pgfqpoint{1.258433in}{2.250896in}}%
\pgfpathlineto{\pgfqpoint{1.161736in}{2.226833in}}%
\pgfpathlineto{\pgfqpoint{1.170663in}{2.206153in}}%
\pgfpathlineto{\pgfqpoint{1.265064in}{2.229706in}}%
\pgfpathclose%
\pgfusepath{fill}%
\end{pgfscope}%
\begin{pgfscope}%
\pgfpathrectangle{\pgfqpoint{0.000000in}{0.000000in}}{\pgfqpoint{3.000000in}{3.000000in}}%
\pgfusepath{clip}%
\pgfsetbuttcap%
\pgfsetroundjoin%
\definecolor{currentfill}{rgb}{1.000000,0.668845,0.000000}%
\pgfsetfillcolor{currentfill}%
\pgfsetlinewidth{0.000000pt}%
\definecolor{currentstroke}{rgb}{0.000000,0.000000,0.000000}%
\pgfsetstrokecolor{currentstroke}%
\pgfsetdash{}{0pt}%
\pgfpathmoveto{\pgfqpoint{1.724414in}{2.003717in}}%
\pgfpathlineto{\pgfqpoint{1.730308in}{2.028891in}}%
\pgfpathlineto{\pgfqpoint{1.650498in}{2.040061in}}%
\pgfpathlineto{\pgfqpoint{1.647087in}{2.014571in}}%
\pgfpathlineto{\pgfqpoint{1.724414in}{2.003717in}}%
\pgfpathclose%
\pgfusepath{fill}%
\end{pgfscope}%
\begin{pgfscope}%
\pgfpathrectangle{\pgfqpoint{0.000000in}{0.000000in}}{\pgfqpoint{3.000000in}{3.000000in}}%
\pgfusepath{clip}%
\pgfsetbuttcap%
\pgfsetroundjoin%
\definecolor{currentfill}{rgb}{1.000000,0.538126,0.000000}%
\pgfsetfillcolor{currentfill}%
\pgfsetlinewidth{0.000000pt}%
\definecolor{currentstroke}{rgb}{0.000000,0.000000,0.000000}%
\pgfsetstrokecolor{currentstroke}%
\pgfsetdash{}{0pt}%
\pgfpathmoveto{\pgfqpoint{1.399369in}{2.061883in}}%
\pgfpathlineto{\pgfqpoint{1.395129in}{2.086283in}}%
\pgfpathlineto{\pgfqpoint{1.311665in}{2.072405in}}%
\pgfpathlineto{\pgfqpoint{1.318349in}{2.048373in}}%
\pgfpathlineto{\pgfqpoint{1.399369in}{2.061883in}}%
\pgfpathclose%
\pgfusepath{fill}%
\end{pgfscope}%
\begin{pgfscope}%
\pgfpathrectangle{\pgfqpoint{0.000000in}{0.000000in}}{\pgfqpoint{3.000000in}{3.000000in}}%
\pgfusepath{clip}%
\pgfsetbuttcap%
\pgfsetroundjoin%
\definecolor{currentfill}{rgb}{0.300443,1.000000,0.667299}%
\pgfsetfillcolor{currentfill}%
\pgfsetlinewidth{0.000000pt}%
\definecolor{currentstroke}{rgb}{0.000000,0.000000,0.000000}%
\pgfsetstrokecolor{currentstroke}%
\pgfsetdash{}{0pt}%
\pgfpathmoveto{\pgfqpoint{1.605882in}{1.643126in}}%
\pgfpathlineto{\pgfqpoint{1.609335in}{1.681158in}}%
\pgfpathlineto{\pgfqpoint{1.557851in}{1.684495in}}%
\pgfpathlineto{\pgfqpoint{1.556983in}{1.646309in}}%
\pgfpathlineto{\pgfqpoint{1.605882in}{1.643126in}}%
\pgfpathclose%
\pgfusepath{fill}%
\end{pgfscope}%
\begin{pgfscope}%
\pgfpathrectangle{\pgfqpoint{0.000000in}{0.000000in}}{\pgfqpoint{3.000000in}{3.000000in}}%
\pgfusepath{clip}%
\pgfsetbuttcap%
\pgfsetroundjoin%
\definecolor{currentfill}{rgb}{0.401645,1.000000,0.566097}%
\pgfsetfillcolor{currentfill}%
\pgfsetlinewidth{0.000000pt}%
\definecolor{currentstroke}{rgb}{0.000000,0.000000,0.000000}%
\pgfsetstrokecolor{currentstroke}%
\pgfsetdash{}{0pt}%
\pgfpathmoveto{\pgfqpoint{1.557851in}{1.684495in}}%
\pgfpathlineto{\pgfqpoint{1.558719in}{1.720829in}}%
\pgfpathlineto{\pgfqpoint{1.504231in}{1.720129in}}%
\pgfpathlineto{\pgfqpoint{1.505964in}{1.683826in}}%
\pgfpathlineto{\pgfqpoint{1.557851in}{1.684495in}}%
\pgfpathclose%
\pgfusepath{fill}%
\end{pgfscope}%
\begin{pgfscope}%
\pgfpathrectangle{\pgfqpoint{0.000000in}{0.000000in}}{\pgfqpoint{3.000000in}{3.000000in}}%
\pgfusepath{clip}%
\pgfsetbuttcap%
\pgfsetroundjoin%
\definecolor{currentfill}{rgb}{0.000000,0.676471,1.000000}%
\pgfsetfillcolor{currentfill}%
\pgfsetlinewidth{0.000000pt}%
\definecolor{currentstroke}{rgb}{0.000000,0.000000,0.000000}%
\pgfsetstrokecolor{currentstroke}%
\pgfsetdash{}{0pt}%
\pgfpathmoveto{\pgfqpoint{1.439712in}{1.457358in}}%
\pgfpathlineto{\pgfqpoint{1.432919in}{1.506050in}}%
\pgfpathlineto{\pgfqpoint{1.396278in}{1.496098in}}%
\pgfpathlineto{\pgfqpoint{1.405392in}{1.447985in}}%
\pgfpathlineto{\pgfqpoint{1.439712in}{1.457358in}}%
\pgfpathclose%
\pgfusepath{fill}%
\end{pgfscope}%
\begin{pgfscope}%
\pgfpathrectangle{\pgfqpoint{0.000000in}{0.000000in}}{\pgfqpoint{3.000000in}{3.000000in}}%
\pgfusepath{clip}%
\pgfsetbuttcap%
\pgfsetroundjoin%
\definecolor{currentfill}{rgb}{0.895003,1.000000,0.072739}%
\pgfsetfillcolor{currentfill}%
\pgfsetlinewidth{0.000000pt}%
\definecolor{currentstroke}{rgb}{0.000000,0.000000,0.000000}%
\pgfsetstrokecolor{currentstroke}%
\pgfsetdash{}{0pt}%
\pgfpathmoveto{\pgfqpoint{1.629980in}{1.877530in}}%
\pgfpathlineto{\pgfqpoint{1.633408in}{1.906421in}}%
\pgfpathlineto{\pgfqpoint{1.563906in}{1.910815in}}%
\pgfpathlineto{\pgfqpoint{1.563044in}{1.881775in}}%
\pgfpathlineto{\pgfqpoint{1.629980in}{1.877530in}}%
\pgfpathclose%
\pgfusepath{fill}%
\end{pgfscope}%
\begin{pgfscope}%
\pgfpathrectangle{\pgfqpoint{0.000000in}{0.000000in}}{\pgfqpoint{3.000000in}{3.000000in}}%
\pgfusepath{clip}%
\pgfsetbuttcap%
\pgfsetroundjoin%
\definecolor{currentfill}{rgb}{0.743201,1.000000,0.224541}%
\pgfsetfillcolor{currentfill}%
\pgfsetlinewidth{0.000000pt}%
\definecolor{currentstroke}{rgb}{0.000000,0.000000,0.000000}%
\pgfsetstrokecolor{currentstroke}%
\pgfsetdash{}{0pt}%
\pgfpathmoveto{\pgfqpoint{1.561317in}{1.820902in}}%
\pgfpathlineto{\pgfqpoint{1.562181in}{1.851837in}}%
\pgfpathlineto{\pgfqpoint{1.497316in}{1.851016in}}%
\pgfpathlineto{\pgfqpoint{1.499042in}{1.820111in}}%
\pgfpathlineto{\pgfqpoint{1.561317in}{1.820902in}}%
\pgfpathclose%
\pgfusepath{fill}%
\end{pgfscope}%
\begin{pgfscope}%
\pgfpathrectangle{\pgfqpoint{0.000000in}{0.000000in}}{\pgfqpoint{3.000000in}{3.000000in}}%
\pgfusepath{clip}%
\pgfsetbuttcap%
\pgfsetroundjoin%
\definecolor{currentfill}{rgb}{0.490196,1.000000,0.477546}%
\pgfsetfillcolor{currentfill}%
\pgfsetlinewidth{0.000000pt}%
\definecolor{currentstroke}{rgb}{0.000000,0.000000,0.000000}%
\pgfsetstrokecolor{currentstroke}%
\pgfsetdash{}{0pt}%
\pgfpathmoveto{\pgfqpoint{1.558719in}{1.720829in}}%
\pgfpathlineto{\pgfqpoint{1.559586in}{1.755549in}}%
\pgfpathlineto{\pgfqpoint{1.502499in}{1.754818in}}%
\pgfpathlineto{\pgfqpoint{1.504231in}{1.720129in}}%
\pgfpathlineto{\pgfqpoint{1.558719in}{1.720829in}}%
\pgfpathclose%
\pgfusepath{fill}%
\end{pgfscope}%
\begin{pgfscope}%
\pgfpathrectangle{\pgfqpoint{0.000000in}{0.000000in}}{\pgfqpoint{3.000000in}{3.000000in}}%
\pgfusepath{clip}%
\pgfsetbuttcap%
\pgfsetroundjoin%
\definecolor{currentfill}{rgb}{0.958254,0.973856,0.009488}%
\pgfsetfillcolor{currentfill}%
\pgfsetlinewidth{0.000000pt}%
\definecolor{currentstroke}{rgb}{0.000000,0.000000,0.000000}%
\pgfsetstrokecolor{currentstroke}%
\pgfsetdash{}{0pt}%
\pgfpathmoveto{\pgfqpoint{1.493869in}{1.909934in}}%
\pgfpathlineto{\pgfqpoint{1.492148in}{1.938132in}}%
\pgfpathlineto{\pgfqpoint{1.420634in}{1.931790in}}%
\pgfpathlineto{\pgfqpoint{1.424900in}{1.903798in}}%
\pgfpathlineto{\pgfqpoint{1.493869in}{1.909934in}}%
\pgfpathclose%
\pgfusepath{fill}%
\end{pgfscope}%
\begin{pgfscope}%
\pgfpathrectangle{\pgfqpoint{0.000000in}{0.000000in}}{\pgfqpoint{3.000000in}{3.000000in}}%
\pgfusepath{clip}%
\pgfsetbuttcap%
\pgfsetroundjoin%
\definecolor{currentfill}{rgb}{0.667299,1.000000,0.300443}%
\pgfsetfillcolor{currentfill}%
\pgfsetlinewidth{0.000000pt}%
\definecolor{currentstroke}{rgb}{0.000000,0.000000,0.000000}%
\pgfsetstrokecolor{currentstroke}%
\pgfsetdash{}{0pt}%
\pgfpathmoveto{\pgfqpoint{1.560452in}{1.788852in}}%
\pgfpathlineto{\pgfqpoint{1.561317in}{1.820902in}}%
\pgfpathlineto{\pgfqpoint{1.499042in}{1.820111in}}%
\pgfpathlineto{\pgfqpoint{1.500770in}{1.788091in}}%
\pgfpathlineto{\pgfqpoint{1.560452in}{1.788852in}}%
\pgfpathclose%
\pgfusepath{fill}%
\end{pgfscope}%
\begin{pgfscope}%
\pgfpathrectangle{\pgfqpoint{0.000000in}{0.000000in}}{\pgfqpoint{3.000000in}{3.000000in}}%
\pgfusepath{clip}%
\pgfsetbuttcap%
\pgfsetroundjoin%
\definecolor{currentfill}{rgb}{0.578748,1.000000,0.388994}%
\pgfsetfillcolor{currentfill}%
\pgfsetlinewidth{0.000000pt}%
\definecolor{currentstroke}{rgb}{0.000000,0.000000,0.000000}%
\pgfsetstrokecolor{currentstroke}%
\pgfsetdash{}{0pt}%
\pgfpathmoveto{\pgfqpoint{1.559586in}{1.755549in}}%
\pgfpathlineto{\pgfqpoint{1.560452in}{1.788852in}}%
\pgfpathlineto{\pgfqpoint{1.500770in}{1.788091in}}%
\pgfpathlineto{\pgfqpoint{1.502499in}{1.754818in}}%
\pgfpathlineto{\pgfqpoint{1.559586in}{1.755549in}}%
\pgfpathclose%
\pgfusepath{fill}%
\end{pgfscope}%
\begin{pgfscope}%
\pgfpathrectangle{\pgfqpoint{0.000000in}{0.000000in}}{\pgfqpoint{3.000000in}{3.000000in}}%
\pgfusepath{clip}%
\pgfsetbuttcap%
\pgfsetroundjoin%
\definecolor{currentfill}{rgb}{0.300443,1.000000,0.667299}%
\pgfsetfillcolor{currentfill}%
\pgfsetlinewidth{0.000000pt}%
\definecolor{currentstroke}{rgb}{0.000000,0.000000,0.000000}%
\pgfsetstrokecolor{currentstroke}%
\pgfsetdash{}{0pt}%
\pgfpathmoveto{\pgfqpoint{1.507699in}{1.645671in}}%
\pgfpathlineto{\pgfqpoint{1.505964in}{1.683826in}}%
\pgfpathlineto{\pgfqpoint{1.454884in}{1.679166in}}%
\pgfpathlineto{\pgfqpoint{1.459184in}{1.641227in}}%
\pgfpathlineto{\pgfqpoint{1.507699in}{1.645671in}}%
\pgfpathclose%
\pgfusepath{fill}%
\end{pgfscope}%
\begin{pgfscope}%
\pgfpathrectangle{\pgfqpoint{0.000000in}{0.000000in}}{\pgfqpoint{3.000000in}{3.000000in}}%
\pgfusepath{clip}%
\pgfsetbuttcap%
\pgfsetroundjoin%
\definecolor{currentfill}{rgb}{0.000000,0.064706,1.000000}%
\pgfsetfillcolor{currentfill}%
\pgfsetlinewidth{0.000000pt}%
\definecolor{currentstroke}{rgb}{0.000000,0.000000,0.000000}%
\pgfsetstrokecolor{currentstroke}%
\pgfsetdash{}{0pt}%
\pgfpathmoveto{\pgfqpoint{1.707001in}{1.243497in}}%
\pgfpathlineto{\pgfqpoint{1.721159in}{1.305523in}}%
\pgfpathlineto{\pgfqpoint{1.700357in}{1.319246in}}%
\pgfpathlineto{\pgfqpoint{1.687847in}{1.256226in}}%
\pgfpathlineto{\pgfqpoint{1.707001in}{1.243497in}}%
\pgfpathclose%
\pgfusepath{fill}%
\end{pgfscope}%
\begin{pgfscope}%
\pgfpathrectangle{\pgfqpoint{0.000000in}{0.000000in}}{\pgfqpoint{3.000000in}{3.000000in}}%
\pgfusepath{clip}%
\pgfsetbuttcap%
\pgfsetroundjoin%
\definecolor{currentfill}{rgb}{0.401645,1.000000,0.566097}%
\pgfsetfillcolor{currentfill}%
\pgfsetlinewidth{0.000000pt}%
\definecolor{currentstroke}{rgb}{0.000000,0.000000,0.000000}%
\pgfsetstrokecolor{currentstroke}%
\pgfsetdash{}{0pt}%
\pgfpathmoveto{\pgfqpoint{1.609335in}{1.681158in}}%
\pgfpathlineto{\pgfqpoint{1.612784in}{1.717339in}}%
\pgfpathlineto{\pgfqpoint{1.558719in}{1.720829in}}%
\pgfpathlineto{\pgfqpoint{1.557851in}{1.684495in}}%
\pgfpathlineto{\pgfqpoint{1.609335in}{1.681158in}}%
\pgfpathclose%
\pgfusepath{fill}%
\end{pgfscope}%
\begin{pgfscope}%
\pgfpathrectangle{\pgfqpoint{0.000000in}{0.000000in}}{\pgfqpoint{3.000000in}{3.000000in}}%
\pgfusepath{clip}%
\pgfsetbuttcap%
\pgfsetroundjoin%
\definecolor{currentfill}{rgb}{0.000000,0.300000,1.000000}%
\pgfsetfillcolor{currentfill}%
\pgfsetlinewidth{0.000000pt}%
\definecolor{currentstroke}{rgb}{0.000000,0.000000,0.000000}%
\pgfsetstrokecolor{currentstroke}%
\pgfsetdash{}{0pt}%
\pgfpathmoveto{\pgfqpoint{1.700357in}{1.319246in}}%
\pgfpathlineto{\pgfqpoint{1.712862in}{1.375616in}}%
\pgfpathlineto{\pgfqpoint{1.686273in}{1.388343in}}%
\pgfpathlineto{\pgfqpoint{1.675711in}{1.331120in}}%
\pgfpathlineto{\pgfqpoint{1.700357in}{1.319246in}}%
\pgfpathclose%
\pgfusepath{fill}%
\end{pgfscope}%
\begin{pgfscope}%
\pgfpathrectangle{\pgfqpoint{0.000000in}{0.000000in}}{\pgfqpoint{3.000000in}{3.000000in}}%
\pgfusepath{clip}%
\pgfsetbuttcap%
\pgfsetroundjoin%
\definecolor{currentfill}{rgb}{0.819102,1.000000,0.148640}%
\pgfsetfillcolor{currentfill}%
\pgfsetlinewidth{0.000000pt}%
\definecolor{currentstroke}{rgb}{0.000000,0.000000,0.000000}%
\pgfsetstrokecolor{currentstroke}%
\pgfsetdash{}{0pt}%
\pgfpathmoveto{\pgfqpoint{1.626548in}{1.847742in}}%
\pgfpathlineto{\pgfqpoint{1.629980in}{1.877530in}}%
\pgfpathlineto{\pgfqpoint{1.563044in}{1.881775in}}%
\pgfpathlineto{\pgfqpoint{1.562181in}{1.851837in}}%
\pgfpathlineto{\pgfqpoint{1.626548in}{1.847742in}}%
\pgfpathclose%
\pgfusepath{fill}%
\end{pgfscope}%
\begin{pgfscope}%
\pgfpathrectangle{\pgfqpoint{0.000000in}{0.000000in}}{\pgfqpoint{3.000000in}{3.000000in}}%
\pgfusepath{clip}%
\pgfsetbuttcap%
\pgfsetroundjoin%
\definecolor{currentfill}{rgb}{0.553476,0.000000,0.000000}%
\pgfsetfillcolor{currentfill}%
\pgfsetlinewidth{0.000000pt}%
\definecolor{currentstroke}{rgb}{0.000000,0.000000,0.000000}%
\pgfsetstrokecolor{currentstroke}%
\pgfsetdash{}{0pt}%
\pgfpathmoveto{\pgfqpoint{2.041254in}{2.323477in}}%
\pgfpathlineto{\pgfqpoint{2.051560in}{2.342169in}}%
\pgfpathlineto{\pgfqpoint{1.944991in}{2.375038in}}%
\pgfpathlineto{\pgfqpoint{1.936864in}{2.355757in}}%
\pgfpathlineto{\pgfqpoint{2.041254in}{2.323477in}}%
\pgfpathclose%
\pgfusepath{fill}%
\end{pgfscope}%
\begin{pgfscope}%
\pgfpathrectangle{\pgfqpoint{0.000000in}{0.000000in}}{\pgfqpoint{3.000000in}{3.000000in}}%
\pgfusepath{clip}%
\pgfsetbuttcap%
\pgfsetroundjoin%
\definecolor{currentfill}{rgb}{1.000000,0.741467,0.000000}%
\pgfsetfillcolor{currentfill}%
\pgfsetlinewidth{0.000000pt}%
\definecolor{currentstroke}{rgb}{0.000000,0.000000,0.000000}%
\pgfsetstrokecolor{currentstroke}%
\pgfsetdash{}{0pt}%
\pgfpathmoveto{\pgfqpoint{1.718515in}{1.977979in}}%
\pgfpathlineto{\pgfqpoint{1.724414in}{2.003717in}}%
\pgfpathlineto{\pgfqpoint{1.647087in}{2.014571in}}%
\pgfpathlineto{\pgfqpoint{1.643673in}{1.988516in}}%
\pgfpathlineto{\pgfqpoint{1.718515in}{1.977979in}}%
\pgfpathclose%
\pgfusepath{fill}%
\end{pgfscope}%
\begin{pgfscope}%
\pgfpathrectangle{\pgfqpoint{0.000000in}{0.000000in}}{\pgfqpoint{3.000000in}{3.000000in}}%
\pgfusepath{clip}%
\pgfsetbuttcap%
\pgfsetroundjoin%
\definecolor{currentfill}{rgb}{0.085389,1.000000,0.882353}%
\pgfsetfillcolor{currentfill}%
\pgfsetlinewidth{0.000000pt}%
\definecolor{currentstroke}{rgb}{0.000000,0.000000,0.000000}%
\pgfsetstrokecolor{currentstroke}%
\pgfsetdash{}{0pt}%
\pgfpathmoveto{\pgfqpoint{1.467796in}{1.558553in}}%
\pgfpathlineto{\pgfqpoint{1.463488in}{1.601143in}}%
\pgfpathlineto{\pgfqpoint{1.419351in}{1.593390in}}%
\pgfpathlineto{\pgfqpoint{1.426132in}{1.551201in}}%
\pgfpathlineto{\pgfqpoint{1.467796in}{1.558553in}}%
\pgfpathclose%
\pgfusepath{fill}%
\end{pgfscope}%
\begin{pgfscope}%
\pgfpathrectangle{\pgfqpoint{0.000000in}{0.000000in}}{\pgfqpoint{3.000000in}{3.000000in}}%
\pgfusepath{clip}%
\pgfsetbuttcap%
\pgfsetroundjoin%
\definecolor{currentfill}{rgb}{1.000000,0.610748,0.000000}%
\pgfsetfillcolor{currentfill}%
\pgfsetlinewidth{0.000000pt}%
\definecolor{currentstroke}{rgb}{0.000000,0.000000,0.000000}%
\pgfsetstrokecolor{currentstroke}%
\pgfsetdash{}{0pt}%
\pgfpathmoveto{\pgfqpoint{1.403613in}{2.037001in}}%
\pgfpathlineto{\pgfqpoint{1.399369in}{2.061883in}}%
\pgfpathlineto{\pgfqpoint{1.318349in}{2.048373in}}%
\pgfpathlineto{\pgfqpoint{1.325038in}{2.023860in}}%
\pgfpathlineto{\pgfqpoint{1.403613in}{2.037001in}}%
\pgfpathclose%
\pgfusepath{fill}%
\end{pgfscope}%
\begin{pgfscope}%
\pgfpathrectangle{\pgfqpoint{0.000000in}{0.000000in}}{\pgfqpoint{3.000000in}{3.000000in}}%
\pgfusepath{clip}%
\pgfsetbuttcap%
\pgfsetroundjoin%
\definecolor{currentfill}{rgb}{1.000000,0.291213,0.000000}%
\pgfsetfillcolor{currentfill}%
\pgfsetlinewidth{0.000000pt}%
\definecolor{currentstroke}{rgb}{0.000000,0.000000,0.000000}%
\pgfsetstrokecolor{currentstroke}%
\pgfsetdash{}{0pt}%
\pgfpathmoveto{\pgfqpoint{1.846972in}{2.128325in}}%
\pgfpathlineto{\pgfqpoint{1.855182in}{2.150422in}}%
\pgfpathlineto{\pgfqpoint{1.765544in}{2.170248in}}%
\pgfpathlineto{\pgfqpoint{1.759686in}{2.147684in}}%
\pgfpathlineto{\pgfqpoint{1.846972in}{2.128325in}}%
\pgfpathclose%
\pgfusepath{fill}%
\end{pgfscope}%
\begin{pgfscope}%
\pgfpathrectangle{\pgfqpoint{0.000000in}{0.000000in}}{\pgfqpoint{3.000000in}{3.000000in}}%
\pgfusepath{clip}%
\pgfsetbuttcap%
\pgfsetroundjoin%
\definecolor{currentfill}{rgb}{0.895003,1.000000,0.072739}%
\pgfsetfillcolor{currentfill}%
\pgfsetlinewidth{0.000000pt}%
\definecolor{currentstroke}{rgb}{0.000000,0.000000,0.000000}%
\pgfsetstrokecolor{currentstroke}%
\pgfsetdash{}{0pt}%
\pgfpathmoveto{\pgfqpoint{1.495591in}{1.880924in}}%
\pgfpathlineto{\pgfqpoint{1.493869in}{1.909934in}}%
\pgfpathlineto{\pgfqpoint{1.424900in}{1.903798in}}%
\pgfpathlineto{\pgfqpoint{1.429171in}{1.874995in}}%
\pgfpathlineto{\pgfqpoint{1.495591in}{1.880924in}}%
\pgfpathclose%
\pgfusepath{fill}%
\end{pgfscope}%
\begin{pgfscope}%
\pgfpathrectangle{\pgfqpoint{0.000000in}{0.000000in}}{\pgfqpoint{3.000000in}{3.000000in}}%
\pgfusepath{clip}%
\pgfsetbuttcap%
\pgfsetroundjoin%
\definecolor{currentfill}{rgb}{0.743201,1.000000,0.224541}%
\pgfsetfillcolor{currentfill}%
\pgfsetlinewidth{0.000000pt}%
\definecolor{currentstroke}{rgb}{0.000000,0.000000,0.000000}%
\pgfsetstrokecolor{currentstroke}%
\pgfsetdash{}{0pt}%
\pgfpathmoveto{\pgfqpoint{1.623112in}{1.816956in}}%
\pgfpathlineto{\pgfqpoint{1.626548in}{1.847742in}}%
\pgfpathlineto{\pgfqpoint{1.562181in}{1.851837in}}%
\pgfpathlineto{\pgfqpoint{1.561317in}{1.820902in}}%
\pgfpathlineto{\pgfqpoint{1.623112in}{1.816956in}}%
\pgfpathclose%
\pgfusepath{fill}%
\end{pgfscope}%
\begin{pgfscope}%
\pgfpathrectangle{\pgfqpoint{0.000000in}{0.000000in}}{\pgfqpoint{3.000000in}{3.000000in}}%
\pgfusepath{clip}%
\pgfsetbuttcap%
\pgfsetroundjoin%
\definecolor{currentfill}{rgb}{0.490196,1.000000,0.477546}%
\pgfsetfillcolor{currentfill}%
\pgfsetlinewidth{0.000000pt}%
\definecolor{currentstroke}{rgb}{0.000000,0.000000,0.000000}%
\pgfsetstrokecolor{currentstroke}%
\pgfsetdash{}{0pt}%
\pgfpathmoveto{\pgfqpoint{1.612784in}{1.717339in}}%
\pgfpathlineto{\pgfqpoint{1.616230in}{1.751906in}}%
\pgfpathlineto{\pgfqpoint{1.559586in}{1.755549in}}%
\pgfpathlineto{\pgfqpoint{1.558719in}{1.720829in}}%
\pgfpathlineto{\pgfqpoint{1.612784in}{1.717339in}}%
\pgfpathclose%
\pgfusepath{fill}%
\end{pgfscope}%
\begin{pgfscope}%
\pgfpathrectangle{\pgfqpoint{0.000000in}{0.000000in}}{\pgfqpoint{3.000000in}{3.000000in}}%
\pgfusepath{clip}%
\pgfsetbuttcap%
\pgfsetroundjoin%
\definecolor{currentfill}{rgb}{1.000000,0.116921,0.000000}%
\pgfsetfillcolor{currentfill}%
\pgfsetlinewidth{0.000000pt}%
\definecolor{currentstroke}{rgb}{0.000000,0.000000,0.000000}%
\pgfsetstrokecolor{currentstroke}%
\pgfsetdash{}{0pt}%
\pgfpathmoveto{\pgfqpoint{1.271702in}{2.208235in}}%
\pgfpathlineto{\pgfqpoint{1.265064in}{2.229706in}}%
\pgfpathlineto{\pgfqpoint{1.170663in}{2.206153in}}%
\pgfpathlineto{\pgfqpoint{1.179599in}{2.185194in}}%
\pgfpathlineto{\pgfqpoint{1.271702in}{2.208235in}}%
\pgfpathclose%
\pgfusepath{fill}%
\end{pgfscope}%
\begin{pgfscope}%
\pgfpathrectangle{\pgfqpoint{0.000000in}{0.000000in}}{\pgfqpoint{3.000000in}{3.000000in}}%
\pgfusepath{clip}%
\pgfsetbuttcap%
\pgfsetroundjoin%
\definecolor{currentfill}{rgb}{0.000000,0.000000,0.838681}%
\pgfsetfillcolor{currentfill}%
\pgfsetlinewidth{0.000000pt}%
\definecolor{currentstroke}{rgb}{0.000000,0.000000,0.000000}%
\pgfsetstrokecolor{currentstroke}%
\pgfsetdash{}{0pt}%
\pgfpathmoveto{\pgfqpoint{1.383204in}{1.168724in}}%
\pgfpathlineto{\pgfqpoint{1.368569in}{1.238924in}}%
\pgfpathlineto{\pgfqpoint{1.354875in}{1.224387in}}%
\pgfpathlineto{\pgfqpoint{1.370698in}{1.155329in}}%
\pgfpathlineto{\pgfqpoint{1.383204in}{1.168724in}}%
\pgfpathclose%
\pgfusepath{fill}%
\end{pgfscope}%
\begin{pgfscope}%
\pgfpathrectangle{\pgfqpoint{0.000000in}{0.000000in}}{\pgfqpoint{3.000000in}{3.000000in}}%
\pgfusepath{clip}%
\pgfsetbuttcap%
\pgfsetroundjoin%
\definecolor{currentfill}{rgb}{0.667299,1.000000,0.300443}%
\pgfsetfillcolor{currentfill}%
\pgfsetlinewidth{0.000000pt}%
\definecolor{currentstroke}{rgb}{0.000000,0.000000,0.000000}%
\pgfsetstrokecolor{currentstroke}%
\pgfsetdash{}{0pt}%
\pgfpathmoveto{\pgfqpoint{1.619673in}{1.785057in}}%
\pgfpathlineto{\pgfqpoint{1.623112in}{1.816956in}}%
\pgfpathlineto{\pgfqpoint{1.561317in}{1.820902in}}%
\pgfpathlineto{\pgfqpoint{1.560452in}{1.788852in}}%
\pgfpathlineto{\pgfqpoint{1.619673in}{1.785057in}}%
\pgfpathclose%
\pgfusepath{fill}%
\end{pgfscope}%
\begin{pgfscope}%
\pgfpathrectangle{\pgfqpoint{0.000000in}{0.000000in}}{\pgfqpoint{3.000000in}{3.000000in}}%
\pgfusepath{clip}%
\pgfsetbuttcap%
\pgfsetroundjoin%
\definecolor{currentfill}{rgb}{0.578748,1.000000,0.388994}%
\pgfsetfillcolor{currentfill}%
\pgfsetlinewidth{0.000000pt}%
\definecolor{currentstroke}{rgb}{0.000000,0.000000,0.000000}%
\pgfsetstrokecolor{currentstroke}%
\pgfsetdash{}{0pt}%
\pgfpathmoveto{\pgfqpoint{1.616230in}{1.751906in}}%
\pgfpathlineto{\pgfqpoint{1.619673in}{1.785057in}}%
\pgfpathlineto{\pgfqpoint{1.560452in}{1.788852in}}%
\pgfpathlineto{\pgfqpoint{1.559586in}{1.755549in}}%
\pgfpathlineto{\pgfqpoint{1.616230in}{1.751906in}}%
\pgfpathclose%
\pgfusepath{fill}%
\end{pgfscope}%
\begin{pgfscope}%
\pgfpathrectangle{\pgfqpoint{0.000000in}{0.000000in}}{\pgfqpoint{3.000000in}{3.000000in}}%
\pgfusepath{clip}%
\pgfsetbuttcap%
\pgfsetroundjoin%
\definecolor{currentfill}{rgb}{0.199241,1.000000,0.768501}%
\pgfsetfillcolor{currentfill}%
\pgfsetlinewidth{0.000000pt}%
\definecolor{currentstroke}{rgb}{0.000000,0.000000,0.000000}%
\pgfsetstrokecolor{currentstroke}%
\pgfsetdash{}{0pt}%
\pgfpathmoveto{\pgfqpoint{1.647285in}{1.596357in}}%
\pgfpathlineto{\pgfqpoint{1.653251in}{1.636195in}}%
\pgfpathlineto{\pgfqpoint{1.605882in}{1.643126in}}%
\pgfpathlineto{\pgfqpoint{1.602426in}{1.602950in}}%
\pgfpathlineto{\pgfqpoint{1.647285in}{1.596357in}}%
\pgfpathclose%
\pgfusepath{fill}%
\end{pgfscope}%
\begin{pgfscope}%
\pgfpathrectangle{\pgfqpoint{0.000000in}{0.000000in}}{\pgfqpoint{3.000000in}{3.000000in}}%
\pgfusepath{clip}%
\pgfsetbuttcap%
\pgfsetroundjoin%
\definecolor{currentfill}{rgb}{0.401645,1.000000,0.566097}%
\pgfsetfillcolor{currentfill}%
\pgfsetlinewidth{0.000000pt}%
\definecolor{currentstroke}{rgb}{0.000000,0.000000,0.000000}%
\pgfsetstrokecolor{currentstroke}%
\pgfsetdash{}{0pt}%
\pgfpathmoveto{\pgfqpoint{1.505964in}{1.683826in}}%
\pgfpathlineto{\pgfqpoint{1.504231in}{1.720129in}}%
\pgfpathlineto{\pgfqpoint{1.450588in}{1.715255in}}%
\pgfpathlineto{\pgfqpoint{1.454884in}{1.679166in}}%
\pgfpathlineto{\pgfqpoint{1.505964in}{1.683826in}}%
\pgfpathclose%
\pgfusepath{fill}%
\end{pgfscope}%
\begin{pgfscope}%
\pgfpathrectangle{\pgfqpoint{0.000000in}{0.000000in}}{\pgfqpoint{3.000000in}{3.000000in}}%
\pgfusepath{clip}%
\pgfsetbuttcap%
\pgfsetroundjoin%
\definecolor{currentfill}{rgb}{0.000000,0.849020,1.000000}%
\pgfsetfillcolor{currentfill}%
\pgfsetlinewidth{0.000000pt}%
\definecolor{currentstroke}{rgb}{0.000000,0.000000,0.000000}%
\pgfsetstrokecolor{currentstroke}%
\pgfsetdash{}{0pt}%
\pgfpathmoveto{\pgfqpoint{1.672928in}{1.499736in}}%
\pgfpathlineto{\pgfqpoint{1.681283in}{1.544520in}}%
\pgfpathlineto{\pgfqpoint{1.641313in}{1.554014in}}%
\pgfpathlineto{\pgfqpoint{1.635337in}{1.508710in}}%
\pgfpathlineto{\pgfqpoint{1.672928in}{1.499736in}}%
\pgfpathclose%
\pgfusepath{fill}%
\end{pgfscope}%
\begin{pgfscope}%
\pgfpathrectangle{\pgfqpoint{0.000000in}{0.000000in}}{\pgfqpoint{3.000000in}{3.000000in}}%
\pgfusepath{clip}%
\pgfsetbuttcap%
\pgfsetroundjoin%
\definecolor{currentfill}{rgb}{0.819102,1.000000,0.148640}%
\pgfsetfillcolor{currentfill}%
\pgfsetlinewidth{0.000000pt}%
\definecolor{currentstroke}{rgb}{0.000000,0.000000,0.000000}%
\pgfsetstrokecolor{currentstroke}%
\pgfsetdash{}{0pt}%
\pgfpathmoveto{\pgfqpoint{1.497316in}{1.851016in}}%
\pgfpathlineto{\pgfqpoint{1.495591in}{1.880924in}}%
\pgfpathlineto{\pgfqpoint{1.429171in}{1.874995in}}%
\pgfpathlineto{\pgfqpoint{1.433446in}{1.845296in}}%
\pgfpathlineto{\pgfqpoint{1.497316in}{1.851016in}}%
\pgfpathclose%
\pgfusepath{fill}%
\end{pgfscope}%
\begin{pgfscope}%
\pgfpathrectangle{\pgfqpoint{0.000000in}{0.000000in}}{\pgfqpoint{3.000000in}{3.000000in}}%
\pgfusepath{clip}%
\pgfsetbuttcap%
\pgfsetroundjoin%
\definecolor{currentfill}{rgb}{0.000000,0.000000,0.500000}%
\pgfsetfillcolor{currentfill}%
\pgfsetlinewidth{0.000000pt}%
\definecolor{currentstroke}{rgb}{0.000000,0.000000,0.000000}%
\pgfsetstrokecolor{currentstroke}%
\pgfsetdash{}{0pt}%
\pgfpathmoveto{\pgfqpoint{1.704983in}{1.053009in}}%
\pgfpathlineto{\pgfqpoint{1.721917in}{1.131229in}}%
\pgfpathlineto{\pgfqpoint{1.716438in}{1.145892in}}%
\pgfpathlineto{\pgfqpoint{1.700038in}{1.066410in}}%
\pgfpathlineto{\pgfqpoint{1.704983in}{1.053009in}}%
\pgfpathclose%
\pgfusepath{fill}%
\end{pgfscope}%
\begin{pgfscope}%
\pgfpathrectangle{\pgfqpoint{0.000000in}{0.000000in}}{\pgfqpoint{3.000000in}{3.000000in}}%
\pgfusepath{clip}%
\pgfsetbuttcap%
\pgfsetroundjoin%
\definecolor{currentfill}{rgb}{0.000000,0.503922,1.000000}%
\pgfsetfillcolor{currentfill}%
\pgfsetlinewidth{0.000000pt}%
\definecolor{currentstroke}{rgb}{0.000000,0.000000,0.000000}%
\pgfsetstrokecolor{currentstroke}%
\pgfsetdash{}{0pt}%
\pgfpathmoveto{\pgfqpoint{1.414514in}{1.395577in}}%
\pgfpathlineto{\pgfqpoint{1.405392in}{1.447985in}}%
\pgfpathlineto{\pgfqpoint{1.374290in}{1.436001in}}%
\pgfpathlineto{\pgfqpoint{1.385524in}{1.384340in}}%
\pgfpathlineto{\pgfqpoint{1.414514in}{1.395577in}}%
\pgfpathclose%
\pgfusepath{fill}%
\end{pgfscope}%
\begin{pgfscope}%
\pgfpathrectangle{\pgfqpoint{0.000000in}{0.000000in}}{\pgfqpoint{3.000000in}{3.000000in}}%
\pgfusepath{clip}%
\pgfsetbuttcap%
\pgfsetroundjoin%
\definecolor{currentfill}{rgb}{1.000000,0.814089,0.000000}%
\pgfsetfillcolor{currentfill}%
\pgfsetlinewidth{0.000000pt}%
\definecolor{currentstroke}{rgb}{0.000000,0.000000,0.000000}%
\pgfsetstrokecolor{currentstroke}%
\pgfsetdash{}{0pt}%
\pgfpathmoveto{\pgfqpoint{1.712610in}{1.951627in}}%
\pgfpathlineto{\pgfqpoint{1.718515in}{1.977979in}}%
\pgfpathlineto{\pgfqpoint{1.643673in}{1.988516in}}%
\pgfpathlineto{\pgfqpoint{1.640255in}{1.961845in}}%
\pgfpathlineto{\pgfqpoint{1.712610in}{1.951627in}}%
\pgfpathclose%
\pgfusepath{fill}%
\end{pgfscope}%
\begin{pgfscope}%
\pgfpathrectangle{\pgfqpoint{0.000000in}{0.000000in}}{\pgfqpoint{3.000000in}{3.000000in}}%
\pgfusepath{clip}%
\pgfsetbuttcap%
\pgfsetroundjoin%
\definecolor{currentfill}{rgb}{0.490196,1.000000,0.477546}%
\pgfsetfillcolor{currentfill}%
\pgfsetlinewidth{0.000000pt}%
\definecolor{currentstroke}{rgb}{0.000000,0.000000,0.000000}%
\pgfsetstrokecolor{currentstroke}%
\pgfsetdash{}{0pt}%
\pgfpathmoveto{\pgfqpoint{1.504231in}{1.720129in}}%
\pgfpathlineto{\pgfqpoint{1.502499in}{1.754818in}}%
\pgfpathlineto{\pgfqpoint{1.446296in}{1.749731in}}%
\pgfpathlineto{\pgfqpoint{1.450588in}{1.715255in}}%
\pgfpathlineto{\pgfqpoint{1.504231in}{1.720129in}}%
\pgfpathclose%
\pgfusepath{fill}%
\end{pgfscope}%
\begin{pgfscope}%
\pgfpathrectangle{\pgfqpoint{0.000000in}{0.000000in}}{\pgfqpoint{3.000000in}{3.000000in}}%
\pgfusepath{clip}%
\pgfsetbuttcap%
\pgfsetroundjoin%
\definecolor{currentfill}{rgb}{0.743201,1.000000,0.224541}%
\pgfsetfillcolor{currentfill}%
\pgfsetlinewidth{0.000000pt}%
\definecolor{currentstroke}{rgb}{0.000000,0.000000,0.000000}%
\pgfsetstrokecolor{currentstroke}%
\pgfsetdash{}{0pt}%
\pgfpathmoveto{\pgfqpoint{1.499042in}{1.820111in}}%
\pgfpathlineto{\pgfqpoint{1.497316in}{1.851016in}}%
\pgfpathlineto{\pgfqpoint{1.433446in}{1.845296in}}%
\pgfpathlineto{\pgfqpoint{1.437725in}{1.814601in}}%
\pgfpathlineto{\pgfqpoint{1.499042in}{1.820111in}}%
\pgfpathclose%
\pgfusepath{fill}%
\end{pgfscope}%
\begin{pgfscope}%
\pgfpathrectangle{\pgfqpoint{0.000000in}{0.000000in}}{\pgfqpoint{3.000000in}{3.000000in}}%
\pgfusepath{clip}%
\pgfsetbuttcap%
\pgfsetroundjoin%
\definecolor{currentfill}{rgb}{1.000000,0.668845,0.000000}%
\pgfsetfillcolor{currentfill}%
\pgfsetlinewidth{0.000000pt}%
\definecolor{currentstroke}{rgb}{0.000000,0.000000,0.000000}%
\pgfsetstrokecolor{currentstroke}%
\pgfsetdash{}{0pt}%
\pgfpathmoveto{\pgfqpoint{1.407861in}{2.011597in}}%
\pgfpathlineto{\pgfqpoint{1.403613in}{2.037001in}}%
\pgfpathlineto{\pgfqpoint{1.325038in}{2.023860in}}%
\pgfpathlineto{\pgfqpoint{1.331735in}{1.998829in}}%
\pgfpathlineto{\pgfqpoint{1.407861in}{2.011597in}}%
\pgfpathclose%
\pgfusepath{fill}%
\end{pgfscope}%
\begin{pgfscope}%
\pgfpathrectangle{\pgfqpoint{0.000000in}{0.000000in}}{\pgfqpoint{3.000000in}{3.000000in}}%
\pgfusepath{clip}%
\pgfsetbuttcap%
\pgfsetroundjoin%
\definecolor{currentfill}{rgb}{0.606952,0.000000,0.000000}%
\pgfsetfillcolor{currentfill}%
\pgfsetlinewidth{0.000000pt}%
\definecolor{currentstroke}{rgb}{0.000000,0.000000,0.000000}%
\pgfsetstrokecolor{currentstroke}%
\pgfsetdash{}{0pt}%
\pgfpathmoveto{\pgfqpoint{2.030939in}{2.304595in}}%
\pgfpathlineto{\pgfqpoint{2.041254in}{2.323477in}}%
\pgfpathlineto{\pgfqpoint{1.936864in}{2.355757in}}%
\pgfpathlineto{\pgfqpoint{1.928730in}{2.336282in}}%
\pgfpathlineto{\pgfqpoint{2.030939in}{2.304595in}}%
\pgfpathclose%
\pgfusepath{fill}%
\end{pgfscope}%
\begin{pgfscope}%
\pgfpathrectangle{\pgfqpoint{0.000000in}{0.000000in}}{\pgfqpoint{3.000000in}{3.000000in}}%
\pgfusepath{clip}%
\pgfsetbuttcap%
\pgfsetroundjoin%
\definecolor{currentfill}{rgb}{0.578748,1.000000,0.388994}%
\pgfsetfillcolor{currentfill}%
\pgfsetlinewidth{0.000000pt}%
\definecolor{currentstroke}{rgb}{0.000000,0.000000,0.000000}%
\pgfsetstrokecolor{currentstroke}%
\pgfsetdash{}{0pt}%
\pgfpathmoveto{\pgfqpoint{1.502499in}{1.754818in}}%
\pgfpathlineto{\pgfqpoint{1.500770in}{1.788091in}}%
\pgfpathlineto{\pgfqpoint{1.442008in}{1.782792in}}%
\pgfpathlineto{\pgfqpoint{1.446296in}{1.749731in}}%
\pgfpathlineto{\pgfqpoint{1.502499in}{1.754818in}}%
\pgfpathclose%
\pgfusepath{fill}%
\end{pgfscope}%
\begin{pgfscope}%
\pgfpathrectangle{\pgfqpoint{0.000000in}{0.000000in}}{\pgfqpoint{3.000000in}{3.000000in}}%
\pgfusepath{clip}%
\pgfsetbuttcap%
\pgfsetroundjoin%
\definecolor{currentfill}{rgb}{0.667299,1.000000,0.300443}%
\pgfsetfillcolor{currentfill}%
\pgfsetlinewidth{0.000000pt}%
\definecolor{currentstroke}{rgb}{0.000000,0.000000,0.000000}%
\pgfsetstrokecolor{currentstroke}%
\pgfsetdash{}{0pt}%
\pgfpathmoveto{\pgfqpoint{1.500770in}{1.788091in}}%
\pgfpathlineto{\pgfqpoint{1.499042in}{1.820111in}}%
\pgfpathlineto{\pgfqpoint{1.437725in}{1.814601in}}%
\pgfpathlineto{\pgfqpoint{1.442008in}{1.782792in}}%
\pgfpathlineto{\pgfqpoint{1.500770in}{1.788091in}}%
\pgfpathclose%
\pgfusepath{fill}%
\end{pgfscope}%
\begin{pgfscope}%
\pgfpathrectangle{\pgfqpoint{0.000000in}{0.000000in}}{\pgfqpoint{3.000000in}{3.000000in}}%
\pgfusepath{clip}%
\pgfsetbuttcap%
\pgfsetroundjoin%
\definecolor{currentfill}{rgb}{1.000000,0.349310,0.000000}%
\pgfsetfillcolor{currentfill}%
\pgfsetlinewidth{0.000000pt}%
\definecolor{currentstroke}{rgb}{0.000000,0.000000,0.000000}%
\pgfsetstrokecolor{currentstroke}%
\pgfsetdash{}{0pt}%
\pgfpathmoveto{\pgfqpoint{1.838755in}{2.105869in}}%
\pgfpathlineto{\pgfqpoint{1.846972in}{2.128325in}}%
\pgfpathlineto{\pgfqpoint{1.759686in}{2.147684in}}%
\pgfpathlineto{\pgfqpoint{1.753822in}{2.124758in}}%
\pgfpathlineto{\pgfqpoint{1.838755in}{2.105869in}}%
\pgfpathclose%
\pgfusepath{fill}%
\end{pgfscope}%
\begin{pgfscope}%
\pgfpathrectangle{\pgfqpoint{0.000000in}{0.000000in}}{\pgfqpoint{3.000000in}{3.000000in}}%
\pgfusepath{clip}%
\pgfsetbuttcap%
\pgfsetroundjoin%
\definecolor{currentfill}{rgb}{1.000000,0.175018,0.000000}%
\pgfsetfillcolor{currentfill}%
\pgfsetlinewidth{0.000000pt}%
\definecolor{currentstroke}{rgb}{0.000000,0.000000,0.000000}%
\pgfsetstrokecolor{currentstroke}%
\pgfsetdash{}{0pt}%
\pgfpathmoveto{\pgfqpoint{1.278346in}{2.186465in}}%
\pgfpathlineto{\pgfqpoint{1.271702in}{2.208235in}}%
\pgfpathlineto{\pgfqpoint{1.179599in}{2.185194in}}%
\pgfpathlineto{\pgfqpoint{1.188543in}{2.163938in}}%
\pgfpathlineto{\pgfqpoint{1.278346in}{2.186465in}}%
\pgfpathclose%
\pgfusepath{fill}%
\end{pgfscope}%
\begin{pgfscope}%
\pgfpathrectangle{\pgfqpoint{0.000000in}{0.000000in}}{\pgfqpoint{3.000000in}{3.000000in}}%
\pgfusepath{clip}%
\pgfsetbuttcap%
\pgfsetroundjoin%
\definecolor{currentfill}{rgb}{1.000000,0.886710,0.000000}%
\pgfsetfillcolor{currentfill}%
\pgfsetlinewidth{0.000000pt}%
\definecolor{currentstroke}{rgb}{0.000000,0.000000,0.000000}%
\pgfsetstrokecolor{currentstroke}%
\pgfsetdash{}{0pt}%
\pgfpathmoveto{\pgfqpoint{1.706700in}{1.924604in}}%
\pgfpathlineto{\pgfqpoint{1.712610in}{1.951627in}}%
\pgfpathlineto{\pgfqpoint{1.640255in}{1.961845in}}%
\pgfpathlineto{\pgfqpoint{1.636834in}{1.934502in}}%
\pgfpathlineto{\pgfqpoint{1.706700in}{1.924604in}}%
\pgfpathclose%
\pgfusepath{fill}%
\end{pgfscope}%
\begin{pgfscope}%
\pgfpathrectangle{\pgfqpoint{0.000000in}{0.000000in}}{\pgfqpoint{3.000000in}{3.000000in}}%
\pgfusepath{clip}%
\pgfsetbuttcap%
\pgfsetroundjoin%
\definecolor{currentfill}{rgb}{0.300443,1.000000,0.667299}%
\pgfsetfillcolor{currentfill}%
\pgfsetlinewidth{0.000000pt}%
\definecolor{currentstroke}{rgb}{0.000000,0.000000,0.000000}%
\pgfsetstrokecolor{currentstroke}%
\pgfsetdash{}{0pt}%
\pgfpathmoveto{\pgfqpoint{1.653251in}{1.636195in}}%
\pgfpathlineto{\pgfqpoint{1.659211in}{1.673890in}}%
\pgfpathlineto{\pgfqpoint{1.609335in}{1.681158in}}%
\pgfpathlineto{\pgfqpoint{1.605882in}{1.643126in}}%
\pgfpathlineto{\pgfqpoint{1.653251in}{1.636195in}}%
\pgfpathclose%
\pgfusepath{fill}%
\end{pgfscope}%
\begin{pgfscope}%
\pgfpathrectangle{\pgfqpoint{0.000000in}{0.000000in}}{\pgfqpoint{3.000000in}{3.000000in}}%
\pgfusepath{clip}%
\pgfsetbuttcap%
\pgfsetroundjoin%
\definecolor{currentfill}{rgb}{0.199241,1.000000,0.768501}%
\pgfsetfillcolor{currentfill}%
\pgfsetlinewidth{0.000000pt}%
\definecolor{currentstroke}{rgb}{0.000000,0.000000,0.000000}%
\pgfsetstrokecolor{currentstroke}%
\pgfsetdash{}{0pt}%
\pgfpathmoveto{\pgfqpoint{1.463488in}{1.601143in}}%
\pgfpathlineto{\pgfqpoint{1.459184in}{1.641227in}}%
\pgfpathlineto{\pgfqpoint{1.412575in}{1.633075in}}%
\pgfpathlineto{\pgfqpoint{1.419351in}{1.593390in}}%
\pgfpathlineto{\pgfqpoint{1.463488in}{1.601143in}}%
\pgfpathclose%
\pgfusepath{fill}%
\end{pgfscope}%
\begin{pgfscope}%
\pgfpathrectangle{\pgfqpoint{0.000000in}{0.000000in}}{\pgfqpoint{3.000000in}{3.000000in}}%
\pgfusepath{clip}%
\pgfsetbuttcap%
\pgfsetroundjoin%
\definecolor{currentfill}{rgb}{0.000000,0.300000,1.000000}%
\pgfsetfillcolor{currentfill}%
\pgfsetlinewidth{0.000000pt}%
\definecolor{currentstroke}{rgb}{0.000000,0.000000,0.000000}%
\pgfsetstrokecolor{currentstroke}%
\pgfsetdash{}{0pt}%
\pgfpathmoveto{\pgfqpoint{1.396763in}{1.327385in}}%
\pgfpathlineto{\pgfqpoint{1.385524in}{1.384340in}}%
\pgfpathlineto{\pgfqpoint{1.360246in}{1.370917in}}%
\pgfpathlineto{\pgfqpoint{1.373335in}{1.314862in}}%
\pgfpathlineto{\pgfqpoint{1.396763in}{1.327385in}}%
\pgfpathclose%
\pgfusepath{fill}%
\end{pgfscope}%
\begin{pgfscope}%
\pgfpathrectangle{\pgfqpoint{0.000000in}{0.000000in}}{\pgfqpoint{3.000000in}{3.000000in}}%
\pgfusepath{clip}%
\pgfsetbuttcap%
\pgfsetroundjoin%
\definecolor{currentfill}{rgb}{0.000000,0.064706,1.000000}%
\pgfsetfillcolor{currentfill}%
\pgfsetlinewidth{0.000000pt}%
\definecolor{currentstroke}{rgb}{0.000000,0.000000,0.000000}%
\pgfsetstrokecolor{currentstroke}%
\pgfsetdash{}{0pt}%
\pgfpathmoveto{\pgfqpoint{1.386430in}{1.252159in}}%
\pgfpathlineto{\pgfqpoint{1.373335in}{1.314862in}}%
\pgfpathlineto{\pgfqpoint{1.353936in}{1.300593in}}%
\pgfpathlineto{\pgfqpoint{1.368569in}{1.238924in}}%
\pgfpathlineto{\pgfqpoint{1.386430in}{1.252159in}}%
\pgfpathclose%
\pgfusepath{fill}%
\end{pgfscope}%
\begin{pgfscope}%
\pgfpathrectangle{\pgfqpoint{0.000000in}{0.000000in}}{\pgfqpoint{3.000000in}{3.000000in}}%
\pgfusepath{clip}%
\pgfsetbuttcap%
\pgfsetroundjoin%
\definecolor{currentfill}{rgb}{1.000000,0.741467,0.000000}%
\pgfsetfillcolor{currentfill}%
\pgfsetlinewidth{0.000000pt}%
\definecolor{currentstroke}{rgb}{0.000000,0.000000,0.000000}%
\pgfsetstrokecolor{currentstroke}%
\pgfsetdash{}{0pt}%
\pgfpathmoveto{\pgfqpoint{1.412115in}{1.985629in}}%
\pgfpathlineto{\pgfqpoint{1.407861in}{2.011597in}}%
\pgfpathlineto{\pgfqpoint{1.331735in}{1.998829in}}%
\pgfpathlineto{\pgfqpoint{1.338437in}{1.973234in}}%
\pgfpathlineto{\pgfqpoint{1.412115in}{1.985629in}}%
\pgfpathclose%
\pgfusepath{fill}%
\end{pgfscope}%
\begin{pgfscope}%
\pgfpathrectangle{\pgfqpoint{0.000000in}{0.000000in}}{\pgfqpoint{3.000000in}{3.000000in}}%
\pgfusepath{clip}%
\pgfsetbuttcap%
\pgfsetroundjoin%
\definecolor{currentfill}{rgb}{0.000000,0.849020,1.000000}%
\pgfsetfillcolor{currentfill}%
\pgfsetlinewidth{0.000000pt}%
\definecolor{currentstroke}{rgb}{0.000000,0.000000,0.000000}%
\pgfsetstrokecolor{currentstroke}%
\pgfsetdash{}{0pt}%
\pgfpathmoveto{\pgfqpoint{1.432919in}{1.506050in}}%
\pgfpathlineto{\pgfqpoint{1.426132in}{1.551201in}}%
\pgfpathlineto{\pgfqpoint{1.387170in}{1.540670in}}%
\pgfpathlineto{\pgfqpoint{1.396278in}{1.496098in}}%
\pgfpathlineto{\pgfqpoint{1.432919in}{1.506050in}}%
\pgfpathclose%
\pgfusepath{fill}%
\end{pgfscope}%
\begin{pgfscope}%
\pgfpathrectangle{\pgfqpoint{0.000000in}{0.000000in}}{\pgfqpoint{3.000000in}{3.000000in}}%
\pgfusepath{clip}%
\pgfsetbuttcap%
\pgfsetroundjoin%
\definecolor{currentfill}{rgb}{0.958254,0.973856,0.009488}%
\pgfsetfillcolor{currentfill}%
\pgfsetlinewidth{0.000000pt}%
\definecolor{currentstroke}{rgb}{0.000000,0.000000,0.000000}%
\pgfsetstrokecolor{currentstroke}%
\pgfsetdash{}{0pt}%
\pgfpathmoveto{\pgfqpoint{1.700783in}{1.896847in}}%
\pgfpathlineto{\pgfqpoint{1.706700in}{1.924604in}}%
\pgfpathlineto{\pgfqpoint{1.636834in}{1.934502in}}%
\pgfpathlineto{\pgfqpoint{1.633408in}{1.906421in}}%
\pgfpathlineto{\pgfqpoint{1.700783in}{1.896847in}}%
\pgfpathclose%
\pgfusepath{fill}%
\end{pgfscope}%
\begin{pgfscope}%
\pgfpathrectangle{\pgfqpoint{0.000000in}{0.000000in}}{\pgfqpoint{3.000000in}{3.000000in}}%
\pgfusepath{clip}%
\pgfsetbuttcap%
\pgfsetroundjoin%
\definecolor{currentfill}{rgb}{0.678253,0.000000,0.000000}%
\pgfsetfillcolor{currentfill}%
\pgfsetlinewidth{0.000000pt}%
\definecolor{currentstroke}{rgb}{0.000000,0.000000,0.000000}%
\pgfsetstrokecolor{currentstroke}%
\pgfsetdash{}{0pt}%
\pgfpathmoveto{\pgfqpoint{2.020616in}{2.285513in}}%
\pgfpathlineto{\pgfqpoint{2.030939in}{2.304595in}}%
\pgfpathlineto{\pgfqpoint{1.928730in}{2.336282in}}%
\pgfpathlineto{\pgfqpoint{1.920588in}{2.316604in}}%
\pgfpathlineto{\pgfqpoint{2.020616in}{2.285513in}}%
\pgfpathclose%
\pgfusepath{fill}%
\end{pgfscope}%
\begin{pgfscope}%
\pgfpathrectangle{\pgfqpoint{0.000000in}{0.000000in}}{\pgfqpoint{3.000000in}{3.000000in}}%
\pgfusepath{clip}%
\pgfsetbuttcap%
\pgfsetroundjoin%
\definecolor{currentfill}{rgb}{1.000000,0.407407,0.000000}%
\pgfsetfillcolor{currentfill}%
\pgfsetlinewidth{0.000000pt}%
\definecolor{currentstroke}{rgb}{0.000000,0.000000,0.000000}%
\pgfsetstrokecolor{currentstroke}%
\pgfsetdash{}{0pt}%
\pgfpathmoveto{\pgfqpoint{1.830530in}{2.083029in}}%
\pgfpathlineto{\pgfqpoint{1.838755in}{2.105869in}}%
\pgfpathlineto{\pgfqpoint{1.753822in}{2.124758in}}%
\pgfpathlineto{\pgfqpoint{1.747953in}{2.101446in}}%
\pgfpathlineto{\pgfqpoint{1.830530in}{2.083029in}}%
\pgfpathclose%
\pgfusepath{fill}%
\end{pgfscope}%
\begin{pgfscope}%
\pgfpathrectangle{\pgfqpoint{0.000000in}{0.000000in}}{\pgfqpoint{3.000000in}{3.000000in}}%
\pgfusepath{clip}%
\pgfsetbuttcap%
\pgfsetroundjoin%
\definecolor{currentfill}{rgb}{0.401645,1.000000,0.566097}%
\pgfsetfillcolor{currentfill}%
\pgfsetlinewidth{0.000000pt}%
\definecolor{currentstroke}{rgb}{0.000000,0.000000,0.000000}%
\pgfsetstrokecolor{currentstroke}%
\pgfsetdash{}{0pt}%
\pgfpathmoveto{\pgfqpoint{1.659211in}{1.673890in}}%
\pgfpathlineto{\pgfqpoint{1.665167in}{1.709736in}}%
\pgfpathlineto{\pgfqpoint{1.612784in}{1.717339in}}%
\pgfpathlineto{\pgfqpoint{1.609335in}{1.681158in}}%
\pgfpathlineto{\pgfqpoint{1.659211in}{1.673890in}}%
\pgfpathclose%
\pgfusepath{fill}%
\end{pgfscope}%
\begin{pgfscope}%
\pgfpathrectangle{\pgfqpoint{0.000000in}{0.000000in}}{\pgfqpoint{3.000000in}{3.000000in}}%
\pgfusepath{clip}%
\pgfsetbuttcap%
\pgfsetroundjoin%
\definecolor{currentfill}{rgb}{1.000000,0.233115,0.000000}%
\pgfsetfillcolor{currentfill}%
\pgfsetlinewidth{0.000000pt}%
\definecolor{currentstroke}{rgb}{0.000000,0.000000,0.000000}%
\pgfsetstrokecolor{currentstroke}%
\pgfsetdash{}{0pt}%
\pgfpathmoveto{\pgfqpoint{1.284997in}{2.164377in}}%
\pgfpathlineto{\pgfqpoint{1.278346in}{2.186465in}}%
\pgfpathlineto{\pgfqpoint{1.188543in}{2.163938in}}%
\pgfpathlineto{\pgfqpoint{1.197494in}{2.142368in}}%
\pgfpathlineto{\pgfqpoint{1.284997in}{2.164377in}}%
\pgfpathclose%
\pgfusepath{fill}%
\end{pgfscope}%
\begin{pgfscope}%
\pgfpathrectangle{\pgfqpoint{0.000000in}{0.000000in}}{\pgfqpoint{3.000000in}{3.000000in}}%
\pgfusepath{clip}%
\pgfsetbuttcap%
\pgfsetroundjoin%
\definecolor{currentfill}{rgb}{0.000000,0.676471,1.000000}%
\pgfsetfillcolor{currentfill}%
\pgfsetlinewidth{0.000000pt}%
\definecolor{currentstroke}{rgb}{0.000000,0.000000,0.000000}%
\pgfsetstrokecolor{currentstroke}%
\pgfsetdash{}{0pt}%
\pgfpathmoveto{\pgfqpoint{1.696829in}{1.440270in}}%
\pgfpathlineto{\pgfqpoint{1.707379in}{1.487903in}}%
\pgfpathlineto{\pgfqpoint{1.672928in}{1.499736in}}%
\pgfpathlineto{\pgfqpoint{1.664567in}{1.451411in}}%
\pgfpathlineto{\pgfqpoint{1.696829in}{1.440270in}}%
\pgfpathclose%
\pgfusepath{fill}%
\end{pgfscope}%
\begin{pgfscope}%
\pgfpathrectangle{\pgfqpoint{0.000000in}{0.000000in}}{\pgfqpoint{3.000000in}{3.000000in}}%
\pgfusepath{clip}%
\pgfsetbuttcap%
\pgfsetroundjoin%
\definecolor{currentfill}{rgb}{1.000000,0.814089,0.000000}%
\pgfsetfillcolor{currentfill}%
\pgfsetlinewidth{0.000000pt}%
\definecolor{currentstroke}{rgb}{0.000000,0.000000,0.000000}%
\pgfsetstrokecolor{currentstroke}%
\pgfsetdash{}{0pt}%
\pgfpathmoveto{\pgfqpoint{1.416372in}{1.959045in}}%
\pgfpathlineto{\pgfqpoint{1.412115in}{1.985629in}}%
\pgfpathlineto{\pgfqpoint{1.338437in}{1.973234in}}%
\pgfpathlineto{\pgfqpoint{1.345146in}{1.947026in}}%
\pgfpathlineto{\pgfqpoint{1.416372in}{1.959045in}}%
\pgfpathclose%
\pgfusepath{fill}%
\end{pgfscope}%
\begin{pgfscope}%
\pgfpathrectangle{\pgfqpoint{0.000000in}{0.000000in}}{\pgfqpoint{3.000000in}{3.000000in}}%
\pgfusepath{clip}%
\pgfsetbuttcap%
\pgfsetroundjoin%
\definecolor{currentfill}{rgb}{0.895003,1.000000,0.072739}%
\pgfsetfillcolor{currentfill}%
\pgfsetlinewidth{0.000000pt}%
\definecolor{currentstroke}{rgb}{0.000000,0.000000,0.000000}%
\pgfsetstrokecolor{currentstroke}%
\pgfsetdash{}{0pt}%
\pgfpathmoveto{\pgfqpoint{1.694861in}{1.868280in}}%
\pgfpathlineto{\pgfqpoint{1.700783in}{1.896847in}}%
\pgfpathlineto{\pgfqpoint{1.633408in}{1.906421in}}%
\pgfpathlineto{\pgfqpoint{1.629980in}{1.877530in}}%
\pgfpathlineto{\pgfqpoint{1.694861in}{1.868280in}}%
\pgfpathclose%
\pgfusepath{fill}%
\end{pgfscope}%
\begin{pgfscope}%
\pgfpathrectangle{\pgfqpoint{0.000000in}{0.000000in}}{\pgfqpoint{3.000000in}{3.000000in}}%
\pgfusepath{clip}%
\pgfsetbuttcap%
\pgfsetroundjoin%
\definecolor{currentfill}{rgb}{0.085389,1.000000,0.882353}%
\pgfsetfillcolor{currentfill}%
\pgfsetlinewidth{0.000000pt}%
\definecolor{currentstroke}{rgb}{0.000000,0.000000,0.000000}%
\pgfsetstrokecolor{currentstroke}%
\pgfsetdash{}{0pt}%
\pgfpathmoveto{\pgfqpoint{1.681283in}{1.544520in}}%
\pgfpathlineto{\pgfqpoint{1.689632in}{1.586345in}}%
\pgfpathlineto{\pgfqpoint{1.647285in}{1.596357in}}%
\pgfpathlineto{\pgfqpoint{1.641313in}{1.554014in}}%
\pgfpathlineto{\pgfqpoint{1.681283in}{1.544520in}}%
\pgfpathclose%
\pgfusepath{fill}%
\end{pgfscope}%
\begin{pgfscope}%
\pgfpathrectangle{\pgfqpoint{0.000000in}{0.000000in}}{\pgfqpoint{3.000000in}{3.000000in}}%
\pgfusepath{clip}%
\pgfsetbuttcap%
\pgfsetroundjoin%
\definecolor{currentfill}{rgb}{0.000000,0.000000,0.500000}%
\pgfsetfillcolor{currentfill}%
\pgfsetlinewidth{0.000000pt}%
\definecolor{currentstroke}{rgb}{0.000000,0.000000,0.000000}%
\pgfsetstrokecolor{currentstroke}%
\pgfsetdash{}{0pt}%
\pgfpathmoveto{\pgfqpoint{1.378953in}{1.061993in}}%
\pgfpathlineto{\pgfqpoint{1.362329in}{1.141059in}}%
\pgfpathlineto{\pgfqpoint{1.358322in}{1.126259in}}%
\pgfpathlineto{\pgfqpoint{1.375343in}{1.048467in}}%
\pgfpathlineto{\pgfqpoint{1.378953in}{1.061993in}}%
\pgfpathclose%
\pgfusepath{fill}%
\end{pgfscope}%
\begin{pgfscope}%
\pgfpathrectangle{\pgfqpoint{0.000000in}{0.000000in}}{\pgfqpoint{3.000000in}{3.000000in}}%
\pgfusepath{clip}%
\pgfsetbuttcap%
\pgfsetroundjoin%
\definecolor{currentfill}{rgb}{0.300443,1.000000,0.667299}%
\pgfsetfillcolor{currentfill}%
\pgfsetlinewidth{0.000000pt}%
\definecolor{currentstroke}{rgb}{0.000000,0.000000,0.000000}%
\pgfsetstrokecolor{currentstroke}%
\pgfsetdash{}{0pt}%
\pgfpathmoveto{\pgfqpoint{1.459184in}{1.641227in}}%
\pgfpathlineto{\pgfqpoint{1.454884in}{1.679166in}}%
\pgfpathlineto{\pgfqpoint{1.405805in}{1.670619in}}%
\pgfpathlineto{\pgfqpoint{1.412575in}{1.633075in}}%
\pgfpathlineto{\pgfqpoint{1.459184in}{1.641227in}}%
\pgfpathclose%
\pgfusepath{fill}%
\end{pgfscope}%
\begin{pgfscope}%
\pgfpathrectangle{\pgfqpoint{0.000000in}{0.000000in}}{\pgfqpoint{3.000000in}{3.000000in}}%
\pgfusepath{clip}%
\pgfsetbuttcap%
\pgfsetroundjoin%
\definecolor{currentfill}{rgb}{0.490196,1.000000,0.477546}%
\pgfsetfillcolor{currentfill}%
\pgfsetlinewidth{0.000000pt}%
\definecolor{currentstroke}{rgb}{0.000000,0.000000,0.000000}%
\pgfsetstrokecolor{currentstroke}%
\pgfsetdash{}{0pt}%
\pgfpathmoveto{\pgfqpoint{1.665167in}{1.709736in}}%
\pgfpathlineto{\pgfqpoint{1.671117in}{1.743970in}}%
\pgfpathlineto{\pgfqpoint{1.616230in}{1.751906in}}%
\pgfpathlineto{\pgfqpoint{1.612784in}{1.717339in}}%
\pgfpathlineto{\pgfqpoint{1.665167in}{1.709736in}}%
\pgfpathclose%
\pgfusepath{fill}%
\end{pgfscope}%
\begin{pgfscope}%
\pgfpathrectangle{\pgfqpoint{0.000000in}{0.000000in}}{\pgfqpoint{3.000000in}{3.000000in}}%
\pgfusepath{clip}%
\pgfsetbuttcap%
\pgfsetroundjoin%
\definecolor{currentfill}{rgb}{0.819102,1.000000,0.148640}%
\pgfsetfillcolor{currentfill}%
\pgfsetlinewidth{0.000000pt}%
\definecolor{currentstroke}{rgb}{0.000000,0.000000,0.000000}%
\pgfsetstrokecolor{currentstroke}%
\pgfsetdash{}{0pt}%
\pgfpathmoveto{\pgfqpoint{1.688933in}{1.838817in}}%
\pgfpathlineto{\pgfqpoint{1.694861in}{1.868280in}}%
\pgfpathlineto{\pgfqpoint{1.629980in}{1.877530in}}%
\pgfpathlineto{\pgfqpoint{1.626548in}{1.847742in}}%
\pgfpathlineto{\pgfqpoint{1.688933in}{1.838817in}}%
\pgfpathclose%
\pgfusepath{fill}%
\end{pgfscope}%
\begin{pgfscope}%
\pgfpathrectangle{\pgfqpoint{0.000000in}{0.000000in}}{\pgfqpoint{3.000000in}{3.000000in}}%
\pgfusepath{clip}%
\pgfsetbuttcap%
\pgfsetroundjoin%
\definecolor{currentfill}{rgb}{0.578748,1.000000,0.388994}%
\pgfsetfillcolor{currentfill}%
\pgfsetlinewidth{0.000000pt}%
\definecolor{currentstroke}{rgb}{0.000000,0.000000,0.000000}%
\pgfsetstrokecolor{currentstroke}%
\pgfsetdash{}{0pt}%
\pgfpathmoveto{\pgfqpoint{1.671117in}{1.743970in}}%
\pgfpathlineto{\pgfqpoint{1.677061in}{1.776790in}}%
\pgfpathlineto{\pgfqpoint{1.619673in}{1.785057in}}%
\pgfpathlineto{\pgfqpoint{1.616230in}{1.751906in}}%
\pgfpathlineto{\pgfqpoint{1.671117in}{1.743970in}}%
\pgfpathclose%
\pgfusepath{fill}%
\end{pgfscope}%
\begin{pgfscope}%
\pgfpathrectangle{\pgfqpoint{0.000000in}{0.000000in}}{\pgfqpoint{3.000000in}{3.000000in}}%
\pgfusepath{clip}%
\pgfsetbuttcap%
\pgfsetroundjoin%
\definecolor{currentfill}{rgb}{0.743201,1.000000,0.224541}%
\pgfsetfillcolor{currentfill}%
\pgfsetlinewidth{0.000000pt}%
\definecolor{currentstroke}{rgb}{0.000000,0.000000,0.000000}%
\pgfsetstrokecolor{currentstroke}%
\pgfsetdash{}{0pt}%
\pgfpathmoveto{\pgfqpoint{1.683000in}{1.808360in}}%
\pgfpathlineto{\pgfqpoint{1.688933in}{1.838817in}}%
\pgfpathlineto{\pgfqpoint{1.626548in}{1.847742in}}%
\pgfpathlineto{\pgfqpoint{1.623112in}{1.816956in}}%
\pgfpathlineto{\pgfqpoint{1.683000in}{1.808360in}}%
\pgfpathclose%
\pgfusepath{fill}%
\end{pgfscope}%
\begin{pgfscope}%
\pgfpathrectangle{\pgfqpoint{0.000000in}{0.000000in}}{\pgfqpoint{3.000000in}{3.000000in}}%
\pgfusepath{clip}%
\pgfsetbuttcap%
\pgfsetroundjoin%
\definecolor{currentfill}{rgb}{1.000000,0.480029,0.000000}%
\pgfsetfillcolor{currentfill}%
\pgfsetlinewidth{0.000000pt}%
\definecolor{currentstroke}{rgb}{0.000000,0.000000,0.000000}%
\pgfsetstrokecolor{currentstroke}%
\pgfsetdash{}{0pt}%
\pgfpathmoveto{\pgfqpoint{1.822299in}{2.059776in}}%
\pgfpathlineto{\pgfqpoint{1.830530in}{2.083029in}}%
\pgfpathlineto{\pgfqpoint{1.747953in}{2.101446in}}%
\pgfpathlineto{\pgfqpoint{1.742077in}{2.077718in}}%
\pgfpathlineto{\pgfqpoint{1.822299in}{2.059776in}}%
\pgfpathclose%
\pgfusepath{fill}%
\end{pgfscope}%
\begin{pgfscope}%
\pgfpathrectangle{\pgfqpoint{0.000000in}{0.000000in}}{\pgfqpoint{3.000000in}{3.000000in}}%
\pgfusepath{clip}%
\pgfsetbuttcap%
\pgfsetroundjoin%
\definecolor{currentfill}{rgb}{1.000000,0.886710,0.000000}%
\pgfsetfillcolor{currentfill}%
\pgfsetlinewidth{0.000000pt}%
\definecolor{currentstroke}{rgb}{0.000000,0.000000,0.000000}%
\pgfsetstrokecolor{currentstroke}%
\pgfsetdash{}{0pt}%
\pgfpathmoveto{\pgfqpoint{1.420634in}{1.931790in}}%
\pgfpathlineto{\pgfqpoint{1.416372in}{1.959045in}}%
\pgfpathlineto{\pgfqpoint{1.345146in}{1.947026in}}%
\pgfpathlineto{\pgfqpoint{1.351862in}{1.920148in}}%
\pgfpathlineto{\pgfqpoint{1.420634in}{1.931790in}}%
\pgfpathclose%
\pgfusepath{fill}%
\end{pgfscope}%
\begin{pgfscope}%
\pgfpathrectangle{\pgfqpoint{0.000000in}{0.000000in}}{\pgfqpoint{3.000000in}{3.000000in}}%
\pgfusepath{clip}%
\pgfsetbuttcap%
\pgfsetroundjoin%
\definecolor{currentfill}{rgb}{0.667299,1.000000,0.300443}%
\pgfsetfillcolor{currentfill}%
\pgfsetlinewidth{0.000000pt}%
\definecolor{currentstroke}{rgb}{0.000000,0.000000,0.000000}%
\pgfsetstrokecolor{currentstroke}%
\pgfsetdash{}{0pt}%
\pgfpathmoveto{\pgfqpoint{1.677061in}{1.776790in}}%
\pgfpathlineto{\pgfqpoint{1.683000in}{1.808360in}}%
\pgfpathlineto{\pgfqpoint{1.623112in}{1.816956in}}%
\pgfpathlineto{\pgfqpoint{1.619673in}{1.785057in}}%
\pgfpathlineto{\pgfqpoint{1.677061in}{1.776790in}}%
\pgfpathclose%
\pgfusepath{fill}%
\end{pgfscope}%
\begin{pgfscope}%
\pgfpathrectangle{\pgfqpoint{0.000000in}{0.000000in}}{\pgfqpoint{3.000000in}{3.000000in}}%
\pgfusepath{clip}%
\pgfsetbuttcap%
\pgfsetroundjoin%
\definecolor{currentfill}{rgb}{0.731729,0.000000,0.000000}%
\pgfsetfillcolor{currentfill}%
\pgfsetlinewidth{0.000000pt}%
\definecolor{currentstroke}{rgb}{0.000000,0.000000,0.000000}%
\pgfsetstrokecolor{currentstroke}%
\pgfsetdash{}{0pt}%
\pgfpathmoveto{\pgfqpoint{2.010286in}{2.266220in}}%
\pgfpathlineto{\pgfqpoint{2.020616in}{2.285513in}}%
\pgfpathlineto{\pgfqpoint{1.920588in}{2.316604in}}%
\pgfpathlineto{\pgfqpoint{1.912439in}{2.296712in}}%
\pgfpathlineto{\pgfqpoint{2.010286in}{2.266220in}}%
\pgfpathclose%
\pgfusepath{fill}%
\end{pgfscope}%
\begin{pgfscope}%
\pgfpathrectangle{\pgfqpoint{0.000000in}{0.000000in}}{\pgfqpoint{3.000000in}{3.000000in}}%
\pgfusepath{clip}%
\pgfsetbuttcap%
\pgfsetroundjoin%
\definecolor{currentfill}{rgb}{1.000000,0.291213,0.000000}%
\pgfsetfillcolor{currentfill}%
\pgfsetlinewidth{0.000000pt}%
\definecolor{currentstroke}{rgb}{0.000000,0.000000,0.000000}%
\pgfsetstrokecolor{currentstroke}%
\pgfsetdash{}{0pt}%
\pgfpathmoveto{\pgfqpoint{1.291654in}{2.141951in}}%
\pgfpathlineto{\pgfqpoint{1.284997in}{2.164377in}}%
\pgfpathlineto{\pgfqpoint{1.197494in}{2.142368in}}%
\pgfpathlineto{\pgfqpoint{1.206452in}{2.120462in}}%
\pgfpathlineto{\pgfqpoint{1.291654in}{2.141951in}}%
\pgfpathclose%
\pgfusepath{fill}%
\end{pgfscope}%
\begin{pgfscope}%
\pgfpathrectangle{\pgfqpoint{0.000000in}{0.000000in}}{\pgfqpoint{3.000000in}{3.000000in}}%
\pgfusepath{clip}%
\pgfsetbuttcap%
\pgfsetroundjoin%
\definecolor{currentfill}{rgb}{0.401645,1.000000,0.566097}%
\pgfsetfillcolor{currentfill}%
\pgfsetlinewidth{0.000000pt}%
\definecolor{currentstroke}{rgb}{0.000000,0.000000,0.000000}%
\pgfsetstrokecolor{currentstroke}%
\pgfsetdash{}{0pt}%
\pgfpathmoveto{\pgfqpoint{1.454884in}{1.679166in}}%
\pgfpathlineto{\pgfqpoint{1.450588in}{1.715255in}}%
\pgfpathlineto{\pgfqpoint{1.399041in}{1.706314in}}%
\pgfpathlineto{\pgfqpoint{1.405805in}{1.670619in}}%
\pgfpathlineto{\pgfqpoint{1.454884in}{1.679166in}}%
\pgfpathclose%
\pgfusepath{fill}%
\end{pgfscope}%
\begin{pgfscope}%
\pgfpathrectangle{\pgfqpoint{0.000000in}{0.000000in}}{\pgfqpoint{3.000000in}{3.000000in}}%
\pgfusepath{clip}%
\pgfsetbuttcap%
\pgfsetroundjoin%
\definecolor{currentfill}{rgb}{0.958254,0.973856,0.009488}%
\pgfsetfillcolor{currentfill}%
\pgfsetlinewidth{0.000000pt}%
\definecolor{currentstroke}{rgb}{0.000000,0.000000,0.000000}%
\pgfsetstrokecolor{currentstroke}%
\pgfsetdash{}{0pt}%
\pgfpathmoveto{\pgfqpoint{1.424900in}{1.903798in}}%
\pgfpathlineto{\pgfqpoint{1.420634in}{1.931790in}}%
\pgfpathlineto{\pgfqpoint{1.351862in}{1.920148in}}%
\pgfpathlineto{\pgfqpoint{1.358583in}{1.892536in}}%
\pgfpathlineto{\pgfqpoint{1.424900in}{1.903798in}}%
\pgfpathclose%
\pgfusepath{fill}%
\end{pgfscope}%
\begin{pgfscope}%
\pgfpathrectangle{\pgfqpoint{0.000000in}{0.000000in}}{\pgfqpoint{3.000000in}{3.000000in}}%
\pgfusepath{clip}%
\pgfsetbuttcap%
\pgfsetroundjoin%
\definecolor{currentfill}{rgb}{1.000000,0.538126,0.000000}%
\pgfsetfillcolor{currentfill}%
\pgfsetlinewidth{0.000000pt}%
\definecolor{currentstroke}{rgb}{0.000000,0.000000,0.000000}%
\pgfsetstrokecolor{currentstroke}%
\pgfsetdash{}{0pt}%
\pgfpathmoveto{\pgfqpoint{1.814060in}{2.036080in}}%
\pgfpathlineto{\pgfqpoint{1.822299in}{2.059776in}}%
\pgfpathlineto{\pgfqpoint{1.742077in}{2.077718in}}%
\pgfpathlineto{\pgfqpoint{1.736195in}{2.053545in}}%
\pgfpathlineto{\pgfqpoint{1.814060in}{2.036080in}}%
\pgfpathclose%
\pgfusepath{fill}%
\end{pgfscope}%
\begin{pgfscope}%
\pgfpathrectangle{\pgfqpoint{0.000000in}{0.000000in}}{\pgfqpoint{3.000000in}{3.000000in}}%
\pgfusepath{clip}%
\pgfsetbuttcap%
\pgfsetroundjoin%
\definecolor{currentfill}{rgb}{0.490196,1.000000,0.477546}%
\pgfsetfillcolor{currentfill}%
\pgfsetlinewidth{0.000000pt}%
\definecolor{currentstroke}{rgb}{0.000000,0.000000,0.000000}%
\pgfsetstrokecolor{currentstroke}%
\pgfsetdash{}{0pt}%
\pgfpathmoveto{\pgfqpoint{1.450588in}{1.715255in}}%
\pgfpathlineto{\pgfqpoint{1.446296in}{1.749731in}}%
\pgfpathlineto{\pgfqpoint{1.392283in}{1.740398in}}%
\pgfpathlineto{\pgfqpoint{1.399041in}{1.706314in}}%
\pgfpathlineto{\pgfqpoint{1.450588in}{1.715255in}}%
\pgfpathclose%
\pgfusepath{fill}%
\end{pgfscope}%
\begin{pgfscope}%
\pgfpathrectangle{\pgfqpoint{0.000000in}{0.000000in}}{\pgfqpoint{3.000000in}{3.000000in}}%
\pgfusepath{clip}%
\pgfsetbuttcap%
\pgfsetroundjoin%
\definecolor{currentfill}{rgb}{0.895003,1.000000,0.072739}%
\pgfsetfillcolor{currentfill}%
\pgfsetlinewidth{0.000000pt}%
\definecolor{currentstroke}{rgb}{0.000000,0.000000,0.000000}%
\pgfsetstrokecolor{currentstroke}%
\pgfsetdash{}{0pt}%
\pgfpathmoveto{\pgfqpoint{1.429171in}{1.874995in}}%
\pgfpathlineto{\pgfqpoint{1.424900in}{1.903798in}}%
\pgfpathlineto{\pgfqpoint{1.358583in}{1.892536in}}%
\pgfpathlineto{\pgfqpoint{1.365311in}{1.864115in}}%
\pgfpathlineto{\pgfqpoint{1.429171in}{1.874995in}}%
\pgfpathclose%
\pgfusepath{fill}%
\end{pgfscope}%
\begin{pgfscope}%
\pgfpathrectangle{\pgfqpoint{0.000000in}{0.000000in}}{\pgfqpoint{3.000000in}{3.000000in}}%
\pgfusepath{clip}%
\pgfsetbuttcap%
\pgfsetroundjoin%
\definecolor{currentfill}{rgb}{0.085389,1.000000,0.882353}%
\pgfsetfillcolor{currentfill}%
\pgfsetlinewidth{0.000000pt}%
\definecolor{currentstroke}{rgb}{0.000000,0.000000,0.000000}%
\pgfsetstrokecolor{currentstroke}%
\pgfsetdash{}{0pt}%
\pgfpathmoveto{\pgfqpoint{1.426132in}{1.551201in}}%
\pgfpathlineto{\pgfqpoint{1.419351in}{1.593390in}}%
\pgfpathlineto{\pgfqpoint{1.378068in}{1.582285in}}%
\pgfpathlineto{\pgfqpoint{1.387170in}{1.540670in}}%
\pgfpathlineto{\pgfqpoint{1.426132in}{1.551201in}}%
\pgfpathclose%
\pgfusepath{fill}%
\end{pgfscope}%
\begin{pgfscope}%
\pgfpathrectangle{\pgfqpoint{0.000000in}{0.000000in}}{\pgfqpoint{3.000000in}{3.000000in}}%
\pgfusepath{clip}%
\pgfsetbuttcap%
\pgfsetroundjoin%
\definecolor{currentfill}{rgb}{0.199241,1.000000,0.768501}%
\pgfsetfillcolor{currentfill}%
\pgfsetlinewidth{0.000000pt}%
\definecolor{currentstroke}{rgb}{0.000000,0.000000,0.000000}%
\pgfsetstrokecolor{currentstroke}%
\pgfsetdash{}{0pt}%
\pgfpathmoveto{\pgfqpoint{1.689632in}{1.586345in}}%
\pgfpathlineto{\pgfqpoint{1.697974in}{1.625668in}}%
\pgfpathlineto{\pgfqpoint{1.653251in}{1.636195in}}%
\pgfpathlineto{\pgfqpoint{1.647285in}{1.596357in}}%
\pgfpathlineto{\pgfqpoint{1.689632in}{1.586345in}}%
\pgfpathclose%
\pgfusepath{fill}%
\end{pgfscope}%
\begin{pgfscope}%
\pgfpathrectangle{\pgfqpoint{0.000000in}{0.000000in}}{\pgfqpoint{3.000000in}{3.000000in}}%
\pgfusepath{clip}%
\pgfsetbuttcap%
\pgfsetroundjoin%
\definecolor{currentfill}{rgb}{0.000000,0.676471,1.000000}%
\pgfsetfillcolor{currentfill}%
\pgfsetlinewidth{0.000000pt}%
\definecolor{currentstroke}{rgb}{0.000000,0.000000,0.000000}%
\pgfsetstrokecolor{currentstroke}%
\pgfsetdash{}{0pt}%
\pgfpathmoveto{\pgfqpoint{1.405392in}{1.447985in}}%
\pgfpathlineto{\pgfqpoint{1.396278in}{1.496098in}}%
\pgfpathlineto{\pgfqpoint{1.363063in}{1.483369in}}%
\pgfpathlineto{\pgfqpoint{1.374290in}{1.436001in}}%
\pgfpathlineto{\pgfqpoint{1.405392in}{1.447985in}}%
\pgfpathclose%
\pgfusepath{fill}%
\end{pgfscope}%
\begin{pgfscope}%
\pgfpathrectangle{\pgfqpoint{0.000000in}{0.000000in}}{\pgfqpoint{3.000000in}{3.000000in}}%
\pgfusepath{clip}%
\pgfsetbuttcap%
\pgfsetroundjoin%
\definecolor{currentfill}{rgb}{0.500000,0.000000,0.000000}%
\pgfsetfillcolor{currentfill}%
\pgfsetlinewidth{0.000000pt}%
\definecolor{currentstroke}{rgb}{0.000000,0.000000,0.000000}%
\pgfsetstrokecolor{currentstroke}%
\pgfsetdash{}{0pt}%
\pgfpathmoveto{\pgfqpoint{1.099461in}{2.364949in}}%
\pgfpathlineto{\pgfqpoint{1.090596in}{2.383866in}}%
\pgfpathlineto{\pgfqpoint{0.985410in}{2.347810in}}%
\pgfpathlineto{\pgfqpoint{0.996392in}{2.329527in}}%
\pgfpathlineto{\pgfqpoint{1.099461in}{2.364949in}}%
\pgfpathclose%
\pgfusepath{fill}%
\end{pgfscope}%
\begin{pgfscope}%
\pgfpathrectangle{\pgfqpoint{0.000000in}{0.000000in}}{\pgfqpoint{3.000000in}{3.000000in}}%
\pgfusepath{clip}%
\pgfsetbuttcap%
\pgfsetroundjoin%
\definecolor{currentfill}{rgb}{0.000000,0.503922,1.000000}%
\pgfsetfillcolor{currentfill}%
\pgfsetlinewidth{0.000000pt}%
\definecolor{currentstroke}{rgb}{0.000000,0.000000,0.000000}%
\pgfsetstrokecolor{currentstroke}%
\pgfsetdash{}{0pt}%
\pgfpathmoveto{\pgfqpoint{1.712862in}{1.375616in}}%
\pgfpathlineto{\pgfqpoint{1.725362in}{1.426694in}}%
\pgfpathlineto{\pgfqpoint{1.696829in}{1.440270in}}%
\pgfpathlineto{\pgfqpoint{1.686273in}{1.388343in}}%
\pgfpathlineto{\pgfqpoint{1.712862in}{1.375616in}}%
\pgfpathclose%
\pgfusepath{fill}%
\end{pgfscope}%
\begin{pgfscope}%
\pgfpathrectangle{\pgfqpoint{0.000000in}{0.000000in}}{\pgfqpoint{3.000000in}{3.000000in}}%
\pgfusepath{clip}%
\pgfsetbuttcap%
\pgfsetroundjoin%
\definecolor{currentfill}{rgb}{0.000000,0.000000,0.838681}%
\pgfsetfillcolor{currentfill}%
\pgfsetlinewidth{0.000000pt}%
\definecolor{currentstroke}{rgb}{0.000000,0.000000,0.000000}%
\pgfsetstrokecolor{currentstroke}%
\pgfsetdash{}{0pt}%
\pgfpathmoveto{\pgfqpoint{1.716438in}{1.145892in}}%
\pgfpathlineto{\pgfqpoint{1.732842in}{1.214143in}}%
\pgfpathlineto{\pgfqpoint{1.722128in}{1.229356in}}%
\pgfpathlineto{\pgfqpoint{1.706660in}{1.159907in}}%
\pgfpathlineto{\pgfqpoint{1.716438in}{1.145892in}}%
\pgfpathclose%
\pgfusepath{fill}%
\end{pgfscope}%
\begin{pgfscope}%
\pgfpathrectangle{\pgfqpoint{0.000000in}{0.000000in}}{\pgfqpoint{3.000000in}{3.000000in}}%
\pgfusepath{clip}%
\pgfsetbuttcap%
\pgfsetroundjoin%
\definecolor{currentfill}{rgb}{0.803030,0.000000,0.000000}%
\pgfsetfillcolor{currentfill}%
\pgfsetlinewidth{0.000000pt}%
\definecolor{currentstroke}{rgb}{0.000000,0.000000,0.000000}%
\pgfsetstrokecolor{currentstroke}%
\pgfsetdash{}{0pt}%
\pgfpathmoveto{\pgfqpoint{1.999947in}{2.246705in}}%
\pgfpathlineto{\pgfqpoint{2.010286in}{2.266220in}}%
\pgfpathlineto{\pgfqpoint{1.912439in}{2.296712in}}%
\pgfpathlineto{\pgfqpoint{1.904282in}{2.276596in}}%
\pgfpathlineto{\pgfqpoint{1.999947in}{2.246705in}}%
\pgfpathclose%
\pgfusepath{fill}%
\end{pgfscope}%
\begin{pgfscope}%
\pgfpathrectangle{\pgfqpoint{0.000000in}{0.000000in}}{\pgfqpoint{3.000000in}{3.000000in}}%
\pgfusepath{clip}%
\pgfsetbuttcap%
\pgfsetroundjoin%
\definecolor{currentfill}{rgb}{1.000000,0.349310,0.000000}%
\pgfsetfillcolor{currentfill}%
\pgfsetlinewidth{0.000000pt}%
\definecolor{currentstroke}{rgb}{0.000000,0.000000,0.000000}%
\pgfsetstrokecolor{currentstroke}%
\pgfsetdash{}{0pt}%
\pgfpathmoveto{\pgfqpoint{1.298318in}{2.119165in}}%
\pgfpathlineto{\pgfqpoint{1.291654in}{2.141951in}}%
\pgfpathlineto{\pgfqpoint{1.206452in}{2.120462in}}%
\pgfpathlineto{\pgfqpoint{1.215419in}{2.098198in}}%
\pgfpathlineto{\pgfqpoint{1.298318in}{2.119165in}}%
\pgfpathclose%
\pgfusepath{fill}%
\end{pgfscope}%
\begin{pgfscope}%
\pgfpathrectangle{\pgfqpoint{0.000000in}{0.000000in}}{\pgfqpoint{3.000000in}{3.000000in}}%
\pgfusepath{clip}%
\pgfsetbuttcap%
\pgfsetroundjoin%
\definecolor{currentfill}{rgb}{0.819102,1.000000,0.148640}%
\pgfsetfillcolor{currentfill}%
\pgfsetlinewidth{0.000000pt}%
\definecolor{currentstroke}{rgb}{0.000000,0.000000,0.000000}%
\pgfsetstrokecolor{currentstroke}%
\pgfsetdash{}{0pt}%
\pgfpathmoveto{\pgfqpoint{1.433446in}{1.845296in}}%
\pgfpathlineto{\pgfqpoint{1.429171in}{1.874995in}}%
\pgfpathlineto{\pgfqpoint{1.365311in}{1.864115in}}%
\pgfpathlineto{\pgfqpoint{1.372045in}{1.834800in}}%
\pgfpathlineto{\pgfqpoint{1.433446in}{1.845296in}}%
\pgfpathclose%
\pgfusepath{fill}%
\end{pgfscope}%
\begin{pgfscope}%
\pgfpathrectangle{\pgfqpoint{0.000000in}{0.000000in}}{\pgfqpoint{3.000000in}{3.000000in}}%
\pgfusepath{clip}%
\pgfsetbuttcap%
\pgfsetroundjoin%
\definecolor{currentfill}{rgb}{0.578748,1.000000,0.388994}%
\pgfsetfillcolor{currentfill}%
\pgfsetlinewidth{0.000000pt}%
\definecolor{currentstroke}{rgb}{0.000000,0.000000,0.000000}%
\pgfsetstrokecolor{currentstroke}%
\pgfsetdash{}{0pt}%
\pgfpathmoveto{\pgfqpoint{1.446296in}{1.749731in}}%
\pgfpathlineto{\pgfqpoint{1.442008in}{1.782792in}}%
\pgfpathlineto{\pgfqpoint{1.385531in}{1.773069in}}%
\pgfpathlineto{\pgfqpoint{1.392283in}{1.740398in}}%
\pgfpathlineto{\pgfqpoint{1.446296in}{1.749731in}}%
\pgfpathclose%
\pgfusepath{fill}%
\end{pgfscope}%
\begin{pgfscope}%
\pgfpathrectangle{\pgfqpoint{0.000000in}{0.000000in}}{\pgfqpoint{3.000000in}{3.000000in}}%
\pgfusepath{clip}%
\pgfsetbuttcap%
\pgfsetroundjoin%
\definecolor{currentfill}{rgb}{0.743201,1.000000,0.224541}%
\pgfsetfillcolor{currentfill}%
\pgfsetlinewidth{0.000000pt}%
\definecolor{currentstroke}{rgb}{0.000000,0.000000,0.000000}%
\pgfsetstrokecolor{currentstroke}%
\pgfsetdash{}{0pt}%
\pgfpathmoveto{\pgfqpoint{1.437725in}{1.814601in}}%
\pgfpathlineto{\pgfqpoint{1.433446in}{1.845296in}}%
\pgfpathlineto{\pgfqpoint{1.372045in}{1.834800in}}%
\pgfpathlineto{\pgfqpoint{1.378785in}{1.804490in}}%
\pgfpathlineto{\pgfqpoint{1.437725in}{1.814601in}}%
\pgfpathclose%
\pgfusepath{fill}%
\end{pgfscope}%
\begin{pgfscope}%
\pgfpathrectangle{\pgfqpoint{0.000000in}{0.000000in}}{\pgfqpoint{3.000000in}{3.000000in}}%
\pgfusepath{clip}%
\pgfsetbuttcap%
\pgfsetroundjoin%
\definecolor{currentfill}{rgb}{0.667299,1.000000,0.300443}%
\pgfsetfillcolor{currentfill}%
\pgfsetlinewidth{0.000000pt}%
\definecolor{currentstroke}{rgb}{0.000000,0.000000,0.000000}%
\pgfsetstrokecolor{currentstroke}%
\pgfsetdash{}{0pt}%
\pgfpathmoveto{\pgfqpoint{1.442008in}{1.782792in}}%
\pgfpathlineto{\pgfqpoint{1.437725in}{1.814601in}}%
\pgfpathlineto{\pgfqpoint{1.378785in}{1.804490in}}%
\pgfpathlineto{\pgfqpoint{1.385531in}{1.773069in}}%
\pgfpathlineto{\pgfqpoint{1.442008in}{1.782792in}}%
\pgfpathclose%
\pgfusepath{fill}%
\end{pgfscope}%
\begin{pgfscope}%
\pgfpathrectangle{\pgfqpoint{0.000000in}{0.000000in}}{\pgfqpoint{3.000000in}{3.000000in}}%
\pgfusepath{clip}%
\pgfsetbuttcap%
\pgfsetroundjoin%
\definecolor{currentfill}{rgb}{1.000000,0.610748,0.000000}%
\pgfsetfillcolor{currentfill}%
\pgfsetlinewidth{0.000000pt}%
\definecolor{currentstroke}{rgb}{0.000000,0.000000,0.000000}%
\pgfsetstrokecolor{currentstroke}%
\pgfsetdash{}{0pt}%
\pgfpathmoveto{\pgfqpoint{1.805813in}{2.011905in}}%
\pgfpathlineto{\pgfqpoint{1.814060in}{2.036080in}}%
\pgfpathlineto{\pgfqpoint{1.736195in}{2.053545in}}%
\pgfpathlineto{\pgfqpoint{1.730308in}{2.028891in}}%
\pgfpathlineto{\pgfqpoint{1.805813in}{2.011905in}}%
\pgfpathclose%
\pgfusepath{fill}%
\end{pgfscope}%
\begin{pgfscope}%
\pgfpathrectangle{\pgfqpoint{0.000000in}{0.000000in}}{\pgfqpoint{3.000000in}{3.000000in}}%
\pgfusepath{clip}%
\pgfsetbuttcap%
\pgfsetroundjoin%
\definecolor{currentfill}{rgb}{1.000000,0.407407,0.000000}%
\pgfsetfillcolor{currentfill}%
\pgfsetlinewidth{0.000000pt}%
\definecolor{currentstroke}{rgb}{0.000000,0.000000,0.000000}%
\pgfsetstrokecolor{currentstroke}%
\pgfsetdash{}{0pt}%
\pgfpathmoveto{\pgfqpoint{1.304988in}{2.095992in}}%
\pgfpathlineto{\pgfqpoint{1.298318in}{2.119165in}}%
\pgfpathlineto{\pgfqpoint{1.215419in}{2.098198in}}%
\pgfpathlineto{\pgfqpoint{1.224393in}{2.075550in}}%
\pgfpathlineto{\pgfqpoint{1.304988in}{2.095992in}}%
\pgfpathclose%
\pgfusepath{fill}%
\end{pgfscope}%
\begin{pgfscope}%
\pgfpathrectangle{\pgfqpoint{0.000000in}{0.000000in}}{\pgfqpoint{3.000000in}{3.000000in}}%
\pgfusepath{clip}%
\pgfsetbuttcap%
\pgfsetroundjoin%
\definecolor{currentfill}{rgb}{0.000000,0.849020,1.000000}%
\pgfsetfillcolor{currentfill}%
\pgfsetlinewidth{0.000000pt}%
\definecolor{currentstroke}{rgb}{0.000000,0.000000,0.000000}%
\pgfsetstrokecolor{currentstroke}%
\pgfsetdash{}{0pt}%
\pgfpathmoveto{\pgfqpoint{1.707379in}{1.487903in}}%
\pgfpathlineto{\pgfqpoint{1.717922in}{1.531999in}}%
\pgfpathlineto{\pgfqpoint{1.681283in}{1.544520in}}%
\pgfpathlineto{\pgfqpoint{1.672928in}{1.499736in}}%
\pgfpathlineto{\pgfqpoint{1.707379in}{1.487903in}}%
\pgfpathclose%
\pgfusepath{fill}%
\end{pgfscope}%
\begin{pgfscope}%
\pgfpathrectangle{\pgfqpoint{0.000000in}{0.000000in}}{\pgfqpoint{3.000000in}{3.000000in}}%
\pgfusepath{clip}%
\pgfsetbuttcap%
\pgfsetroundjoin%
\definecolor{currentfill}{rgb}{0.553476,0.000000,0.000000}%
\pgfsetfillcolor{currentfill}%
\pgfsetlinewidth{0.000000pt}%
\definecolor{currentstroke}{rgb}{0.000000,0.000000,0.000000}%
\pgfsetstrokecolor{currentstroke}%
\pgfsetdash{}{0pt}%
\pgfpathmoveto{\pgfqpoint{1.108333in}{2.345848in}}%
\pgfpathlineto{\pgfqpoint{1.099461in}{2.364949in}}%
\pgfpathlineto{\pgfqpoint{0.996392in}{2.329527in}}%
\pgfpathlineto{\pgfqpoint{1.007383in}{2.311064in}}%
\pgfpathlineto{\pgfqpoint{1.108333in}{2.345848in}}%
\pgfpathclose%
\pgfusepath{fill}%
\end{pgfscope}%
\begin{pgfscope}%
\pgfpathrectangle{\pgfqpoint{0.000000in}{0.000000in}}{\pgfqpoint{3.000000in}{3.000000in}}%
\pgfusepath{clip}%
\pgfsetbuttcap%
\pgfsetroundjoin%
\definecolor{currentfill}{rgb}{0.856506,0.000000,0.000000}%
\pgfsetfillcolor{currentfill}%
\pgfsetlinewidth{0.000000pt}%
\definecolor{currentstroke}{rgb}{0.000000,0.000000,0.000000}%
\pgfsetstrokecolor{currentstroke}%
\pgfsetdash{}{0pt}%
\pgfpathmoveto{\pgfqpoint{1.989600in}{2.226956in}}%
\pgfpathlineto{\pgfqpoint{1.999947in}{2.246705in}}%
\pgfpathlineto{\pgfqpoint{1.904282in}{2.276596in}}%
\pgfpathlineto{\pgfqpoint{1.896118in}{2.256243in}}%
\pgfpathlineto{\pgfqpoint{1.989600in}{2.226956in}}%
\pgfpathclose%
\pgfusepath{fill}%
\end{pgfscope}%
\begin{pgfscope}%
\pgfpathrectangle{\pgfqpoint{0.000000in}{0.000000in}}{\pgfqpoint{3.000000in}{3.000000in}}%
\pgfusepath{clip}%
\pgfsetbuttcap%
\pgfsetroundjoin%
\definecolor{currentfill}{rgb}{0.300443,1.000000,0.667299}%
\pgfsetfillcolor{currentfill}%
\pgfsetlinewidth{0.000000pt}%
\definecolor{currentstroke}{rgb}{0.000000,0.000000,0.000000}%
\pgfsetstrokecolor{currentstroke}%
\pgfsetdash{}{0pt}%
\pgfpathmoveto{\pgfqpoint{1.697974in}{1.625668in}}%
\pgfpathlineto{\pgfqpoint{1.706310in}{1.662851in}}%
\pgfpathlineto{\pgfqpoint{1.659211in}{1.673890in}}%
\pgfpathlineto{\pgfqpoint{1.653251in}{1.636195in}}%
\pgfpathlineto{\pgfqpoint{1.697974in}{1.625668in}}%
\pgfpathclose%
\pgfusepath{fill}%
\end{pgfscope}%
\begin{pgfscope}%
\pgfpathrectangle{\pgfqpoint{0.000000in}{0.000000in}}{\pgfqpoint{3.000000in}{3.000000in}}%
\pgfusepath{clip}%
\pgfsetbuttcap%
\pgfsetroundjoin%
\definecolor{currentfill}{rgb}{0.000000,0.300000,1.000000}%
\pgfsetfillcolor{currentfill}%
\pgfsetlinewidth{0.000000pt}%
\definecolor{currentstroke}{rgb}{0.000000,0.000000,0.000000}%
\pgfsetstrokecolor{currentstroke}%
\pgfsetdash{}{0pt}%
\pgfpathmoveto{\pgfqpoint{1.721159in}{1.305523in}}%
\pgfpathlineto{\pgfqpoint{1.735313in}{1.360905in}}%
\pgfpathlineto{\pgfqpoint{1.712862in}{1.375616in}}%
\pgfpathlineto{\pgfqpoint{1.700357in}{1.319246in}}%
\pgfpathlineto{\pgfqpoint{1.721159in}{1.305523in}}%
\pgfpathclose%
\pgfusepath{fill}%
\end{pgfscope}%
\begin{pgfscope}%
\pgfpathrectangle{\pgfqpoint{0.000000in}{0.000000in}}{\pgfqpoint{3.000000in}{3.000000in}}%
\pgfusepath{clip}%
\pgfsetbuttcap%
\pgfsetroundjoin%
\definecolor{currentfill}{rgb}{1.000000,0.668845,0.000000}%
\pgfsetfillcolor{currentfill}%
\pgfsetlinewidth{0.000000pt}%
\definecolor{currentstroke}{rgb}{0.000000,0.000000,0.000000}%
\pgfsetstrokecolor{currentstroke}%
\pgfsetdash{}{0pt}%
\pgfpathmoveto{\pgfqpoint{1.797560in}{1.987213in}}%
\pgfpathlineto{\pgfqpoint{1.805813in}{2.011905in}}%
\pgfpathlineto{\pgfqpoint{1.730308in}{2.028891in}}%
\pgfpathlineto{\pgfqpoint{1.724414in}{2.003717in}}%
\pgfpathlineto{\pgfqpoint{1.797560in}{1.987213in}}%
\pgfpathclose%
\pgfusepath{fill}%
\end{pgfscope}%
\begin{pgfscope}%
\pgfpathrectangle{\pgfqpoint{0.000000in}{0.000000in}}{\pgfqpoint{3.000000in}{3.000000in}}%
\pgfusepath{clip}%
\pgfsetbuttcap%
\pgfsetroundjoin%
\definecolor{currentfill}{rgb}{0.000000,0.064706,1.000000}%
\pgfsetfillcolor{currentfill}%
\pgfsetlinewidth{0.000000pt}%
\definecolor{currentstroke}{rgb}{0.000000,0.000000,0.000000}%
\pgfsetstrokecolor{currentstroke}%
\pgfsetdash{}{0pt}%
\pgfpathmoveto{\pgfqpoint{1.722128in}{1.229356in}}%
\pgfpathlineto{\pgfqpoint{1.737596in}{1.290277in}}%
\pgfpathlineto{\pgfqpoint{1.721159in}{1.305523in}}%
\pgfpathlineto{\pgfqpoint{1.707001in}{1.243497in}}%
\pgfpathlineto{\pgfqpoint{1.722128in}{1.229356in}}%
\pgfpathclose%
\pgfusepath{fill}%
\end{pgfscope}%
\begin{pgfscope}%
\pgfpathrectangle{\pgfqpoint{0.000000in}{0.000000in}}{\pgfqpoint{3.000000in}{3.000000in}}%
\pgfusepath{clip}%
\pgfsetbuttcap%
\pgfsetroundjoin%
\definecolor{currentfill}{rgb}{0.199241,1.000000,0.768501}%
\pgfsetfillcolor{currentfill}%
\pgfsetlinewidth{0.000000pt}%
\definecolor{currentstroke}{rgb}{0.000000,0.000000,0.000000}%
\pgfsetstrokecolor{currentstroke}%
\pgfsetdash{}{0pt}%
\pgfpathmoveto{\pgfqpoint{1.419351in}{1.593390in}}%
\pgfpathlineto{\pgfqpoint{1.412575in}{1.633075in}}%
\pgfpathlineto{\pgfqpoint{1.368973in}{1.621399in}}%
\pgfpathlineto{\pgfqpoint{1.378068in}{1.582285in}}%
\pgfpathlineto{\pgfqpoint{1.419351in}{1.593390in}}%
\pgfpathclose%
\pgfusepath{fill}%
\end{pgfscope}%
\begin{pgfscope}%
\pgfpathrectangle{\pgfqpoint{0.000000in}{0.000000in}}{\pgfqpoint{3.000000in}{3.000000in}}%
\pgfusepath{clip}%
\pgfsetbuttcap%
\pgfsetroundjoin%
\definecolor{currentfill}{rgb}{1.000000,0.480029,0.000000}%
\pgfsetfillcolor{currentfill}%
\pgfsetlinewidth{0.000000pt}%
\definecolor{currentstroke}{rgb}{0.000000,0.000000,0.000000}%
\pgfsetstrokecolor{currentstroke}%
\pgfsetdash{}{0pt}%
\pgfpathmoveto{\pgfqpoint{1.311665in}{2.072405in}}%
\pgfpathlineto{\pgfqpoint{1.304988in}{2.095992in}}%
\pgfpathlineto{\pgfqpoint{1.224393in}{2.075550in}}%
\pgfpathlineto{\pgfqpoint{1.233375in}{2.052490in}}%
\pgfpathlineto{\pgfqpoint{1.311665in}{2.072405in}}%
\pgfpathclose%
\pgfusepath{fill}%
\end{pgfscope}%
\begin{pgfscope}%
\pgfpathrectangle{\pgfqpoint{0.000000in}{0.000000in}}{\pgfqpoint{3.000000in}{3.000000in}}%
\pgfusepath{clip}%
\pgfsetbuttcap%
\pgfsetroundjoin%
\definecolor{currentfill}{rgb}{1.000000,0.741467,0.000000}%
\pgfsetfillcolor{currentfill}%
\pgfsetlinewidth{0.000000pt}%
\definecolor{currentstroke}{rgb}{0.000000,0.000000,0.000000}%
\pgfsetstrokecolor{currentstroke}%
\pgfsetdash{}{0pt}%
\pgfpathmoveto{\pgfqpoint{1.789299in}{1.961959in}}%
\pgfpathlineto{\pgfqpoint{1.797560in}{1.987213in}}%
\pgfpathlineto{\pgfqpoint{1.724414in}{2.003717in}}%
\pgfpathlineto{\pgfqpoint{1.718515in}{1.977979in}}%
\pgfpathlineto{\pgfqpoint{1.789299in}{1.961959in}}%
\pgfpathclose%
\pgfusepath{fill}%
\end{pgfscope}%
\begin{pgfscope}%
\pgfpathrectangle{\pgfqpoint{0.000000in}{0.000000in}}{\pgfqpoint{3.000000in}{3.000000in}}%
\pgfusepath{clip}%
\pgfsetbuttcap%
\pgfsetroundjoin%
\definecolor{currentfill}{rgb}{0.000000,0.503922,1.000000}%
\pgfsetfillcolor{currentfill}%
\pgfsetlinewidth{0.000000pt}%
\definecolor{currentstroke}{rgb}{0.000000,0.000000,0.000000}%
\pgfsetstrokecolor{currentstroke}%
\pgfsetdash{}{0pt}%
\pgfpathmoveto{\pgfqpoint{1.385524in}{1.384340in}}%
\pgfpathlineto{\pgfqpoint{1.374290in}{1.436001in}}%
\pgfpathlineto{\pgfqpoint{1.347161in}{1.421681in}}%
\pgfpathlineto{\pgfqpoint{1.360246in}{1.370917in}}%
\pgfpathlineto{\pgfqpoint{1.385524in}{1.384340in}}%
\pgfpathclose%
\pgfusepath{fill}%
\end{pgfscope}%
\begin{pgfscope}%
\pgfpathrectangle{\pgfqpoint{0.000000in}{0.000000in}}{\pgfqpoint{3.000000in}{3.000000in}}%
\pgfusepath{clip}%
\pgfsetbuttcap%
\pgfsetroundjoin%
\definecolor{currentfill}{rgb}{0.606952,0.000000,0.000000}%
\pgfsetfillcolor{currentfill}%
\pgfsetlinewidth{0.000000pt}%
\definecolor{currentstroke}{rgb}{0.000000,0.000000,0.000000}%
\pgfsetstrokecolor{currentstroke}%
\pgfsetdash{}{0pt}%
\pgfpathmoveto{\pgfqpoint{1.117214in}{2.326554in}}%
\pgfpathlineto{\pgfqpoint{1.108333in}{2.345848in}}%
\pgfpathlineto{\pgfqpoint{1.007383in}{2.311064in}}%
\pgfpathlineto{\pgfqpoint{1.018381in}{2.292412in}}%
\pgfpathlineto{\pgfqpoint{1.117214in}{2.326554in}}%
\pgfpathclose%
\pgfusepath{fill}%
\end{pgfscope}%
\begin{pgfscope}%
\pgfpathrectangle{\pgfqpoint{0.000000in}{0.000000in}}{\pgfqpoint{3.000000in}{3.000000in}}%
\pgfusepath{clip}%
\pgfsetbuttcap%
\pgfsetroundjoin%
\definecolor{currentfill}{rgb}{0.927807,0.015251,0.000000}%
\pgfsetfillcolor{currentfill}%
\pgfsetlinewidth{0.000000pt}%
\definecolor{currentstroke}{rgb}{0.000000,0.000000,0.000000}%
\pgfsetstrokecolor{currentstroke}%
\pgfsetdash{}{0pt}%
\pgfpathmoveto{\pgfqpoint{1.979245in}{2.206961in}}%
\pgfpathlineto{\pgfqpoint{1.989600in}{2.226956in}}%
\pgfpathlineto{\pgfqpoint{1.896118in}{2.256243in}}%
\pgfpathlineto{\pgfqpoint{1.887945in}{2.235640in}}%
\pgfpathlineto{\pgfqpoint{1.979245in}{2.206961in}}%
\pgfpathclose%
\pgfusepath{fill}%
\end{pgfscope}%
\begin{pgfscope}%
\pgfpathrectangle{\pgfqpoint{0.000000in}{0.000000in}}{\pgfqpoint{3.000000in}{3.000000in}}%
\pgfusepath{clip}%
\pgfsetbuttcap%
\pgfsetroundjoin%
\definecolor{currentfill}{rgb}{0.000000,0.000000,0.838681}%
\pgfsetfillcolor{currentfill}%
\pgfsetlinewidth{0.000000pt}%
\definecolor{currentstroke}{rgb}{0.000000,0.000000,0.000000}%
\pgfsetstrokecolor{currentstroke}%
\pgfsetdash{}{0pt}%
\pgfpathmoveto{\pgfqpoint{1.370698in}{1.155329in}}%
\pgfpathlineto{\pgfqpoint{1.354875in}{1.224387in}}%
\pgfpathlineto{\pgfqpoint{1.345701in}{1.208896in}}%
\pgfpathlineto{\pgfqpoint{1.362329in}{1.141059in}}%
\pgfpathlineto{\pgfqpoint{1.370698in}{1.155329in}}%
\pgfpathclose%
\pgfusepath{fill}%
\end{pgfscope}%
\begin{pgfscope}%
\pgfpathrectangle{\pgfqpoint{0.000000in}{0.000000in}}{\pgfqpoint{3.000000in}{3.000000in}}%
\pgfusepath{clip}%
\pgfsetbuttcap%
\pgfsetroundjoin%
\definecolor{currentfill}{rgb}{0.401645,1.000000,0.566097}%
\pgfsetfillcolor{currentfill}%
\pgfsetlinewidth{0.000000pt}%
\definecolor{currentstroke}{rgb}{0.000000,0.000000,0.000000}%
\pgfsetstrokecolor{currentstroke}%
\pgfsetdash{}{0pt}%
\pgfpathmoveto{\pgfqpoint{1.706310in}{1.662851in}}%
\pgfpathlineto{\pgfqpoint{1.714639in}{1.698187in}}%
\pgfpathlineto{\pgfqpoint{1.665167in}{1.709736in}}%
\pgfpathlineto{\pgfqpoint{1.659211in}{1.673890in}}%
\pgfpathlineto{\pgfqpoint{1.706310in}{1.662851in}}%
\pgfpathclose%
\pgfusepath{fill}%
\end{pgfscope}%
\begin{pgfscope}%
\pgfpathrectangle{\pgfqpoint{0.000000in}{0.000000in}}{\pgfqpoint{3.000000in}{3.000000in}}%
\pgfusepath{clip}%
\pgfsetbuttcap%
\pgfsetroundjoin%
\definecolor{currentfill}{rgb}{0.000000,0.000000,0.500000}%
\pgfsetfillcolor{currentfill}%
\pgfsetlinewidth{0.000000pt}%
\definecolor{currentstroke}{rgb}{0.000000,0.000000,0.000000}%
\pgfsetstrokecolor{currentstroke}%
\pgfsetdash{}{0pt}%
\pgfpathmoveto{\pgfqpoint{1.705891in}{1.039346in}}%
\pgfpathlineto{\pgfqpoint{1.722945in}{1.116277in}}%
\pgfpathlineto{\pgfqpoint{1.721917in}{1.131229in}}%
\pgfpathlineto{\pgfqpoint{1.704983in}{1.053009in}}%
\pgfpathlineto{\pgfqpoint{1.705891in}{1.039346in}}%
\pgfpathclose%
\pgfusepath{fill}%
\end{pgfscope}%
\begin{pgfscope}%
\pgfpathrectangle{\pgfqpoint{0.000000in}{0.000000in}}{\pgfqpoint{3.000000in}{3.000000in}}%
\pgfusepath{clip}%
\pgfsetbuttcap%
\pgfsetroundjoin%
\definecolor{currentfill}{rgb}{1.000000,0.814089,0.000000}%
\pgfsetfillcolor{currentfill}%
\pgfsetlinewidth{0.000000pt}%
\definecolor{currentstroke}{rgb}{0.000000,0.000000,0.000000}%
\pgfsetstrokecolor{currentstroke}%
\pgfsetdash{}{0pt}%
\pgfpathmoveto{\pgfqpoint{1.781031in}{1.936094in}}%
\pgfpathlineto{\pgfqpoint{1.789299in}{1.961959in}}%
\pgfpathlineto{\pgfqpoint{1.718515in}{1.977979in}}%
\pgfpathlineto{\pgfqpoint{1.712610in}{1.951627in}}%
\pgfpathlineto{\pgfqpoint{1.781031in}{1.936094in}}%
\pgfpathclose%
\pgfusepath{fill}%
\end{pgfscope}%
\begin{pgfscope}%
\pgfpathrectangle{\pgfqpoint{0.000000in}{0.000000in}}{\pgfqpoint{3.000000in}{3.000000in}}%
\pgfusepath{clip}%
\pgfsetbuttcap%
\pgfsetroundjoin%
\definecolor{currentfill}{rgb}{1.000000,0.538126,0.000000}%
\pgfsetfillcolor{currentfill}%
\pgfsetlinewidth{0.000000pt}%
\definecolor{currentstroke}{rgb}{0.000000,0.000000,0.000000}%
\pgfsetstrokecolor{currentstroke}%
\pgfsetdash{}{0pt}%
\pgfpathmoveto{\pgfqpoint{1.318349in}{2.048373in}}%
\pgfpathlineto{\pgfqpoint{1.311665in}{2.072405in}}%
\pgfpathlineto{\pgfqpoint{1.233375in}{2.052490in}}%
\pgfpathlineto{\pgfqpoint{1.242364in}{2.028988in}}%
\pgfpathlineto{\pgfqpoint{1.318349in}{2.048373in}}%
\pgfpathclose%
\pgfusepath{fill}%
\end{pgfscope}%
\begin{pgfscope}%
\pgfpathrectangle{\pgfqpoint{0.000000in}{0.000000in}}{\pgfqpoint{3.000000in}{3.000000in}}%
\pgfusepath{clip}%
\pgfsetbuttcap%
\pgfsetroundjoin%
\definecolor{currentfill}{rgb}{0.000000,0.849020,1.000000}%
\pgfsetfillcolor{currentfill}%
\pgfsetlinewidth{0.000000pt}%
\definecolor{currentstroke}{rgb}{0.000000,0.000000,0.000000}%
\pgfsetstrokecolor{currentstroke}%
\pgfsetdash{}{0pt}%
\pgfpathmoveto{\pgfqpoint{1.396278in}{1.496098in}}%
\pgfpathlineto{\pgfqpoint{1.387170in}{1.540670in}}%
\pgfpathlineto{\pgfqpoint{1.351842in}{1.527201in}}%
\pgfpathlineto{\pgfqpoint{1.363063in}{1.483369in}}%
\pgfpathlineto{\pgfqpoint{1.396278in}{1.496098in}}%
\pgfpathclose%
\pgfusepath{fill}%
\end{pgfscope}%
\begin{pgfscope}%
\pgfpathrectangle{\pgfqpoint{0.000000in}{0.000000in}}{\pgfqpoint{3.000000in}{3.000000in}}%
\pgfusepath{clip}%
\pgfsetbuttcap%
\pgfsetroundjoin%
\definecolor{currentfill}{rgb}{0.300443,1.000000,0.667299}%
\pgfsetfillcolor{currentfill}%
\pgfsetlinewidth{0.000000pt}%
\definecolor{currentstroke}{rgb}{0.000000,0.000000,0.000000}%
\pgfsetstrokecolor{currentstroke}%
\pgfsetdash{}{0pt}%
\pgfpathmoveto{\pgfqpoint{1.412575in}{1.633075in}}%
\pgfpathlineto{\pgfqpoint{1.405805in}{1.670619in}}%
\pgfpathlineto{\pgfqpoint{1.359884in}{1.658374in}}%
\pgfpathlineto{\pgfqpoint{1.368973in}{1.621399in}}%
\pgfpathlineto{\pgfqpoint{1.412575in}{1.633075in}}%
\pgfpathclose%
\pgfusepath{fill}%
\end{pgfscope}%
\begin{pgfscope}%
\pgfpathrectangle{\pgfqpoint{0.000000in}{0.000000in}}{\pgfqpoint{3.000000in}{3.000000in}}%
\pgfusepath{clip}%
\pgfsetbuttcap%
\pgfsetroundjoin%
\definecolor{currentfill}{rgb}{0.490196,1.000000,0.477546}%
\pgfsetfillcolor{currentfill}%
\pgfsetlinewidth{0.000000pt}%
\definecolor{currentstroke}{rgb}{0.000000,0.000000,0.000000}%
\pgfsetstrokecolor{currentstroke}%
\pgfsetdash{}{0pt}%
\pgfpathmoveto{\pgfqpoint{1.714639in}{1.698187in}}%
\pgfpathlineto{\pgfqpoint{1.722962in}{1.731914in}}%
\pgfpathlineto{\pgfqpoint{1.671117in}{1.743970in}}%
\pgfpathlineto{\pgfqpoint{1.665167in}{1.709736in}}%
\pgfpathlineto{\pgfqpoint{1.714639in}{1.698187in}}%
\pgfpathclose%
\pgfusepath{fill}%
\end{pgfscope}%
\begin{pgfscope}%
\pgfpathrectangle{\pgfqpoint{0.000000in}{0.000000in}}{\pgfqpoint{3.000000in}{3.000000in}}%
\pgfusepath{clip}%
\pgfsetbuttcap%
\pgfsetroundjoin%
\definecolor{currentfill}{rgb}{1.000000,0.886710,0.000000}%
\pgfsetfillcolor{currentfill}%
\pgfsetlinewidth{0.000000pt}%
\definecolor{currentstroke}{rgb}{0.000000,0.000000,0.000000}%
\pgfsetstrokecolor{currentstroke}%
\pgfsetdash{}{0pt}%
\pgfpathmoveto{\pgfqpoint{1.772756in}{1.909561in}}%
\pgfpathlineto{\pgfqpoint{1.781031in}{1.936094in}}%
\pgfpathlineto{\pgfqpoint{1.712610in}{1.951627in}}%
\pgfpathlineto{\pgfqpoint{1.706700in}{1.924604in}}%
\pgfpathlineto{\pgfqpoint{1.772756in}{1.909561in}}%
\pgfpathclose%
\pgfusepath{fill}%
\end{pgfscope}%
\begin{pgfscope}%
\pgfpathrectangle{\pgfqpoint{0.000000in}{0.000000in}}{\pgfqpoint{3.000000in}{3.000000in}}%
\pgfusepath{clip}%
\pgfsetbuttcap%
\pgfsetroundjoin%
\definecolor{currentfill}{rgb}{0.999109,0.073348,0.000000}%
\pgfsetfillcolor{currentfill}%
\pgfsetlinewidth{0.000000pt}%
\definecolor{currentstroke}{rgb}{0.000000,0.000000,0.000000}%
\pgfsetstrokecolor{currentstroke}%
\pgfsetdash{}{0pt}%
\pgfpathmoveto{\pgfqpoint{1.968883in}{2.186703in}}%
\pgfpathlineto{\pgfqpoint{1.979245in}{2.206961in}}%
\pgfpathlineto{\pgfqpoint{1.887945in}{2.235640in}}%
\pgfpathlineto{\pgfqpoint{1.879766in}{2.214773in}}%
\pgfpathlineto{\pgfqpoint{1.968883in}{2.186703in}}%
\pgfpathclose%
\pgfusepath{fill}%
\end{pgfscope}%
\begin{pgfscope}%
\pgfpathrectangle{\pgfqpoint{0.000000in}{0.000000in}}{\pgfqpoint{3.000000in}{3.000000in}}%
\pgfusepath{clip}%
\pgfsetbuttcap%
\pgfsetroundjoin%
\definecolor{currentfill}{rgb}{0.678253,0.000000,0.000000}%
\pgfsetfillcolor{currentfill}%
\pgfsetlinewidth{0.000000pt}%
\definecolor{currentstroke}{rgb}{0.000000,0.000000,0.000000}%
\pgfsetstrokecolor{currentstroke}%
\pgfsetdash{}{0pt}%
\pgfpathmoveto{\pgfqpoint{1.126102in}{2.307059in}}%
\pgfpathlineto{\pgfqpoint{1.117214in}{2.326554in}}%
\pgfpathlineto{\pgfqpoint{1.018381in}{2.292412in}}%
\pgfpathlineto{\pgfqpoint{1.029387in}{2.273559in}}%
\pgfpathlineto{\pgfqpoint{1.126102in}{2.307059in}}%
\pgfpathclose%
\pgfusepath{fill}%
\end{pgfscope}%
\begin{pgfscope}%
\pgfpathrectangle{\pgfqpoint{0.000000in}{0.000000in}}{\pgfqpoint{3.000000in}{3.000000in}}%
\pgfusepath{clip}%
\pgfsetbuttcap%
\pgfsetroundjoin%
\definecolor{currentfill}{rgb}{0.085389,1.000000,0.882353}%
\pgfsetfillcolor{currentfill}%
\pgfsetlinewidth{0.000000pt}%
\definecolor{currentstroke}{rgb}{0.000000,0.000000,0.000000}%
\pgfsetstrokecolor{currentstroke}%
\pgfsetdash{}{0pt}%
\pgfpathmoveto{\pgfqpoint{1.717922in}{1.531999in}}%
\pgfpathlineto{\pgfqpoint{1.728459in}{1.573140in}}%
\pgfpathlineto{\pgfqpoint{1.689632in}{1.586345in}}%
\pgfpathlineto{\pgfqpoint{1.681283in}{1.544520in}}%
\pgfpathlineto{\pgfqpoint{1.717922in}{1.531999in}}%
\pgfpathclose%
\pgfusepath{fill}%
\end{pgfscope}%
\begin{pgfscope}%
\pgfpathrectangle{\pgfqpoint{0.000000in}{0.000000in}}{\pgfqpoint{3.000000in}{3.000000in}}%
\pgfusepath{clip}%
\pgfsetbuttcap%
\pgfsetroundjoin%
\definecolor{currentfill}{rgb}{1.000000,0.610748,0.000000}%
\pgfsetfillcolor{currentfill}%
\pgfsetlinewidth{0.000000pt}%
\definecolor{currentstroke}{rgb}{0.000000,0.000000,0.000000}%
\pgfsetstrokecolor{currentstroke}%
\pgfsetdash{}{0pt}%
\pgfpathmoveto{\pgfqpoint{1.325038in}{2.023860in}}%
\pgfpathlineto{\pgfqpoint{1.318349in}{2.048373in}}%
\pgfpathlineto{\pgfqpoint{1.242364in}{2.028988in}}%
\pgfpathlineto{\pgfqpoint{1.251361in}{2.005009in}}%
\pgfpathlineto{\pgfqpoint{1.325038in}{2.023860in}}%
\pgfpathclose%
\pgfusepath{fill}%
\end{pgfscope}%
\begin{pgfscope}%
\pgfpathrectangle{\pgfqpoint{0.000000in}{0.000000in}}{\pgfqpoint{3.000000in}{3.000000in}}%
\pgfusepath{clip}%
\pgfsetbuttcap%
\pgfsetroundjoin%
\definecolor{currentfill}{rgb}{0.000000,0.300000,1.000000}%
\pgfsetfillcolor{currentfill}%
\pgfsetlinewidth{0.000000pt}%
\definecolor{currentstroke}{rgb}{0.000000,0.000000,0.000000}%
\pgfsetstrokecolor{currentstroke}%
\pgfsetdash{}{0pt}%
\pgfpathmoveto{\pgfqpoint{1.373335in}{1.314862in}}%
\pgfpathlineto{\pgfqpoint{1.360246in}{1.370917in}}%
\pgfpathlineto{\pgfqpoint{1.339304in}{1.355619in}}%
\pgfpathlineto{\pgfqpoint{1.353936in}{1.300593in}}%
\pgfpathlineto{\pgfqpoint{1.373335in}{1.314862in}}%
\pgfpathclose%
\pgfusepath{fill}%
\end{pgfscope}%
\begin{pgfscope}%
\pgfpathrectangle{\pgfqpoint{0.000000in}{0.000000in}}{\pgfqpoint{3.000000in}{3.000000in}}%
\pgfusepath{clip}%
\pgfsetbuttcap%
\pgfsetroundjoin%
\definecolor{currentfill}{rgb}{0.958254,0.973856,0.009488}%
\pgfsetfillcolor{currentfill}%
\pgfsetlinewidth{0.000000pt}%
\definecolor{currentstroke}{rgb}{0.000000,0.000000,0.000000}%
\pgfsetstrokecolor{currentstroke}%
\pgfsetdash{}{0pt}%
\pgfpathmoveto{\pgfqpoint{1.764474in}{1.882295in}}%
\pgfpathlineto{\pgfqpoint{1.772756in}{1.909561in}}%
\pgfpathlineto{\pgfqpoint{1.706700in}{1.924604in}}%
\pgfpathlineto{\pgfqpoint{1.700783in}{1.896847in}}%
\pgfpathlineto{\pgfqpoint{1.764474in}{1.882295in}}%
\pgfpathclose%
\pgfusepath{fill}%
\end{pgfscope}%
\begin{pgfscope}%
\pgfpathrectangle{\pgfqpoint{0.000000in}{0.000000in}}{\pgfqpoint{3.000000in}{3.000000in}}%
\pgfusepath{clip}%
\pgfsetbuttcap%
\pgfsetroundjoin%
\definecolor{currentfill}{rgb}{0.578748,1.000000,0.388994}%
\pgfsetfillcolor{currentfill}%
\pgfsetlinewidth{0.000000pt}%
\definecolor{currentstroke}{rgb}{0.000000,0.000000,0.000000}%
\pgfsetstrokecolor{currentstroke}%
\pgfsetdash{}{0pt}%
\pgfpathmoveto{\pgfqpoint{1.722962in}{1.731914in}}%
\pgfpathlineto{\pgfqpoint{1.731278in}{1.764230in}}%
\pgfpathlineto{\pgfqpoint{1.677061in}{1.776790in}}%
\pgfpathlineto{\pgfqpoint{1.671117in}{1.743970in}}%
\pgfpathlineto{\pgfqpoint{1.722962in}{1.731914in}}%
\pgfpathclose%
\pgfusepath{fill}%
\end{pgfscope}%
\begin{pgfscope}%
\pgfpathrectangle{\pgfqpoint{0.000000in}{0.000000in}}{\pgfqpoint{3.000000in}{3.000000in}}%
\pgfusepath{clip}%
\pgfsetbuttcap%
\pgfsetroundjoin%
\definecolor{currentfill}{rgb}{0.000000,0.064706,1.000000}%
\pgfsetfillcolor{currentfill}%
\pgfsetlinewidth{0.000000pt}%
\definecolor{currentstroke}{rgb}{0.000000,0.000000,0.000000}%
\pgfsetstrokecolor{currentstroke}%
\pgfsetdash{}{0pt}%
\pgfpathmoveto{\pgfqpoint{1.368569in}{1.238924in}}%
\pgfpathlineto{\pgfqpoint{1.353936in}{1.300593in}}%
\pgfpathlineto{\pgfqpoint{1.339051in}{1.284918in}}%
\pgfpathlineto{\pgfqpoint{1.354875in}{1.224387in}}%
\pgfpathlineto{\pgfqpoint{1.368569in}{1.238924in}}%
\pgfpathclose%
\pgfusepath{fill}%
\end{pgfscope}%
\begin{pgfscope}%
\pgfpathrectangle{\pgfqpoint{0.000000in}{0.000000in}}{\pgfqpoint{3.000000in}{3.000000in}}%
\pgfusepath{clip}%
\pgfsetbuttcap%
\pgfsetroundjoin%
\definecolor{currentfill}{rgb}{0.895003,1.000000,0.072739}%
\pgfsetfillcolor{currentfill}%
\pgfsetlinewidth{0.000000pt}%
\definecolor{currentstroke}{rgb}{0.000000,0.000000,0.000000}%
\pgfsetstrokecolor{currentstroke}%
\pgfsetdash{}{0pt}%
\pgfpathmoveto{\pgfqpoint{1.756186in}{1.854222in}}%
\pgfpathlineto{\pgfqpoint{1.764474in}{1.882295in}}%
\pgfpathlineto{\pgfqpoint{1.700783in}{1.896847in}}%
\pgfpathlineto{\pgfqpoint{1.694861in}{1.868280in}}%
\pgfpathlineto{\pgfqpoint{1.756186in}{1.854222in}}%
\pgfpathclose%
\pgfusepath{fill}%
\end{pgfscope}%
\begin{pgfscope}%
\pgfpathrectangle{\pgfqpoint{0.000000in}{0.000000in}}{\pgfqpoint{3.000000in}{3.000000in}}%
\pgfusepath{clip}%
\pgfsetbuttcap%
\pgfsetroundjoin%
\definecolor{currentfill}{rgb}{0.667299,1.000000,0.300443}%
\pgfsetfillcolor{currentfill}%
\pgfsetlinewidth{0.000000pt}%
\definecolor{currentstroke}{rgb}{0.000000,0.000000,0.000000}%
\pgfsetstrokecolor{currentstroke}%
\pgfsetdash{}{0pt}%
\pgfpathmoveto{\pgfqpoint{1.731278in}{1.764230in}}%
\pgfpathlineto{\pgfqpoint{1.739587in}{1.795298in}}%
\pgfpathlineto{\pgfqpoint{1.683000in}{1.808360in}}%
\pgfpathlineto{\pgfqpoint{1.677061in}{1.776790in}}%
\pgfpathlineto{\pgfqpoint{1.731278in}{1.764230in}}%
\pgfpathclose%
\pgfusepath{fill}%
\end{pgfscope}%
\begin{pgfscope}%
\pgfpathrectangle{\pgfqpoint{0.000000in}{0.000000in}}{\pgfqpoint{3.000000in}{3.000000in}}%
\pgfusepath{clip}%
\pgfsetbuttcap%
\pgfsetroundjoin%
\definecolor{currentfill}{rgb}{0.000000,0.676471,1.000000}%
\pgfsetfillcolor{currentfill}%
\pgfsetlinewidth{0.000000pt}%
\definecolor{currentstroke}{rgb}{0.000000,0.000000,0.000000}%
\pgfsetstrokecolor{currentstroke}%
\pgfsetdash{}{0pt}%
\pgfpathmoveto{\pgfqpoint{1.725362in}{1.426694in}}%
\pgfpathlineto{\pgfqpoint{1.737856in}{1.473483in}}%
\pgfpathlineto{\pgfqpoint{1.707379in}{1.487903in}}%
\pgfpathlineto{\pgfqpoint{1.696829in}{1.440270in}}%
\pgfpathlineto{\pgfqpoint{1.725362in}{1.426694in}}%
\pgfpathclose%
\pgfusepath{fill}%
\end{pgfscope}%
\begin{pgfscope}%
\pgfpathrectangle{\pgfqpoint{0.000000in}{0.000000in}}{\pgfqpoint{3.000000in}{3.000000in}}%
\pgfusepath{clip}%
\pgfsetbuttcap%
\pgfsetroundjoin%
\definecolor{currentfill}{rgb}{0.819102,1.000000,0.148640}%
\pgfsetfillcolor{currentfill}%
\pgfsetlinewidth{0.000000pt}%
\definecolor{currentstroke}{rgb}{0.000000,0.000000,0.000000}%
\pgfsetstrokecolor{currentstroke}%
\pgfsetdash{}{0pt}%
\pgfpathmoveto{\pgfqpoint{1.747890in}{1.825256in}}%
\pgfpathlineto{\pgfqpoint{1.756186in}{1.854222in}}%
\pgfpathlineto{\pgfqpoint{1.694861in}{1.868280in}}%
\pgfpathlineto{\pgfqpoint{1.688933in}{1.838817in}}%
\pgfpathlineto{\pgfqpoint{1.747890in}{1.825256in}}%
\pgfpathclose%
\pgfusepath{fill}%
\end{pgfscope}%
\begin{pgfscope}%
\pgfpathrectangle{\pgfqpoint{0.000000in}{0.000000in}}{\pgfqpoint{3.000000in}{3.000000in}}%
\pgfusepath{clip}%
\pgfsetbuttcap%
\pgfsetroundjoin%
\definecolor{currentfill}{rgb}{0.743201,1.000000,0.224541}%
\pgfsetfillcolor{currentfill}%
\pgfsetlinewidth{0.000000pt}%
\definecolor{currentstroke}{rgb}{0.000000,0.000000,0.000000}%
\pgfsetstrokecolor{currentstroke}%
\pgfsetdash{}{0pt}%
\pgfpathmoveto{\pgfqpoint{1.739587in}{1.795298in}}%
\pgfpathlineto{\pgfqpoint{1.747890in}{1.825256in}}%
\pgfpathlineto{\pgfqpoint{1.688933in}{1.838817in}}%
\pgfpathlineto{\pgfqpoint{1.683000in}{1.808360in}}%
\pgfpathlineto{\pgfqpoint{1.739587in}{1.795298in}}%
\pgfpathclose%
\pgfusepath{fill}%
\end{pgfscope}%
\begin{pgfscope}%
\pgfpathrectangle{\pgfqpoint{0.000000in}{0.000000in}}{\pgfqpoint{3.000000in}{3.000000in}}%
\pgfusepath{clip}%
\pgfsetbuttcap%
\pgfsetroundjoin%
\definecolor{currentfill}{rgb}{0.401645,1.000000,0.566097}%
\pgfsetfillcolor{currentfill}%
\pgfsetlinewidth{0.000000pt}%
\definecolor{currentstroke}{rgb}{0.000000,0.000000,0.000000}%
\pgfsetstrokecolor{currentstroke}%
\pgfsetdash{}{0pt}%
\pgfpathmoveto{\pgfqpoint{1.405805in}{1.670619in}}%
\pgfpathlineto{\pgfqpoint{1.399041in}{1.706314in}}%
\pgfpathlineto{\pgfqpoint{1.350802in}{1.693502in}}%
\pgfpathlineto{\pgfqpoint{1.359884in}{1.658374in}}%
\pgfpathlineto{\pgfqpoint{1.405805in}{1.670619in}}%
\pgfpathclose%
\pgfusepath{fill}%
\end{pgfscope}%
\begin{pgfscope}%
\pgfpathrectangle{\pgfqpoint{0.000000in}{0.000000in}}{\pgfqpoint{3.000000in}{3.000000in}}%
\pgfusepath{clip}%
\pgfsetbuttcap%
\pgfsetroundjoin%
\definecolor{currentfill}{rgb}{1.000000,0.668845,0.000000}%
\pgfsetfillcolor{currentfill}%
\pgfsetlinewidth{0.000000pt}%
\definecolor{currentstroke}{rgb}{0.000000,0.000000,0.000000}%
\pgfsetstrokecolor{currentstroke}%
\pgfsetdash{}{0pt}%
\pgfpathmoveto{\pgfqpoint{1.331735in}{1.998829in}}%
\pgfpathlineto{\pgfqpoint{1.325038in}{2.023860in}}%
\pgfpathlineto{\pgfqpoint{1.251361in}{2.005009in}}%
\pgfpathlineto{\pgfqpoint{1.260365in}{1.980513in}}%
\pgfpathlineto{\pgfqpoint{1.331735in}{1.998829in}}%
\pgfpathclose%
\pgfusepath{fill}%
\end{pgfscope}%
\begin{pgfscope}%
\pgfpathrectangle{\pgfqpoint{0.000000in}{0.000000in}}{\pgfqpoint{3.000000in}{3.000000in}}%
\pgfusepath{clip}%
\pgfsetbuttcap%
\pgfsetroundjoin%
\definecolor{currentfill}{rgb}{1.000000,0.116921,0.000000}%
\pgfsetfillcolor{currentfill}%
\pgfsetlinewidth{0.000000pt}%
\definecolor{currentstroke}{rgb}{0.000000,0.000000,0.000000}%
\pgfsetstrokecolor{currentstroke}%
\pgfsetdash{}{0pt}%
\pgfpathmoveto{\pgfqpoint{1.958512in}{2.166169in}}%
\pgfpathlineto{\pgfqpoint{1.968883in}{2.186703in}}%
\pgfpathlineto{\pgfqpoint{1.879766in}{2.214773in}}%
\pgfpathlineto{\pgfqpoint{1.871578in}{2.193626in}}%
\pgfpathlineto{\pgfqpoint{1.958512in}{2.166169in}}%
\pgfpathclose%
\pgfusepath{fill}%
\end{pgfscope}%
\begin{pgfscope}%
\pgfpathrectangle{\pgfqpoint{0.000000in}{0.000000in}}{\pgfqpoint{3.000000in}{3.000000in}}%
\pgfusepath{clip}%
\pgfsetbuttcap%
\pgfsetroundjoin%
\definecolor{currentfill}{rgb}{0.731729,0.000000,0.000000}%
\pgfsetfillcolor{currentfill}%
\pgfsetlinewidth{0.000000pt}%
\definecolor{currentstroke}{rgb}{0.000000,0.000000,0.000000}%
\pgfsetstrokecolor{currentstroke}%
\pgfsetdash{}{0pt}%
\pgfpathmoveto{\pgfqpoint{1.134999in}{2.287350in}}%
\pgfpathlineto{\pgfqpoint{1.126102in}{2.307059in}}%
\pgfpathlineto{\pgfqpoint{1.029387in}{2.273559in}}%
\pgfpathlineto{\pgfqpoint{1.040402in}{2.254498in}}%
\pgfpathlineto{\pgfqpoint{1.134999in}{2.287350in}}%
\pgfpathclose%
\pgfusepath{fill}%
\end{pgfscope}%
\begin{pgfscope}%
\pgfpathrectangle{\pgfqpoint{0.000000in}{0.000000in}}{\pgfqpoint{3.000000in}{3.000000in}}%
\pgfusepath{clip}%
\pgfsetbuttcap%
\pgfsetroundjoin%
\definecolor{currentfill}{rgb}{0.000000,0.000000,0.500000}%
\pgfsetfillcolor{currentfill}%
\pgfsetlinewidth{0.000000pt}%
\definecolor{currentstroke}{rgb}{0.000000,0.000000,0.000000}%
\pgfsetstrokecolor{currentstroke}%
\pgfsetdash{}{0pt}%
\pgfpathmoveto{\pgfqpoint{1.375343in}{1.048467in}}%
\pgfpathlineto{\pgfqpoint{1.358322in}{1.126259in}}%
\pgfpathlineto{\pgfqpoint{1.358794in}{1.111292in}}%
\pgfpathlineto{\pgfqpoint{1.375794in}{1.034791in}}%
\pgfpathlineto{\pgfqpoint{1.375343in}{1.048467in}}%
\pgfpathclose%
\pgfusepath{fill}%
\end{pgfscope}%
\begin{pgfscope}%
\pgfpathrectangle{\pgfqpoint{0.000000in}{0.000000in}}{\pgfqpoint{3.000000in}{3.000000in}}%
\pgfusepath{clip}%
\pgfsetbuttcap%
\pgfsetroundjoin%
\definecolor{currentfill}{rgb}{1.000000,0.741467,0.000000}%
\pgfsetfillcolor{currentfill}%
\pgfsetlinewidth{0.000000pt}%
\definecolor{currentstroke}{rgb}{0.000000,0.000000,0.000000}%
\pgfsetstrokecolor{currentstroke}%
\pgfsetdash{}{0pt}%
\pgfpathmoveto{\pgfqpoint{1.338437in}{1.973234in}}%
\pgfpathlineto{\pgfqpoint{1.331735in}{1.998829in}}%
\pgfpathlineto{\pgfqpoint{1.260365in}{1.980513in}}%
\pgfpathlineto{\pgfqpoint{1.269376in}{1.955456in}}%
\pgfpathlineto{\pgfqpoint{1.338437in}{1.973234in}}%
\pgfpathclose%
\pgfusepath{fill}%
\end{pgfscope}%
\begin{pgfscope}%
\pgfpathrectangle{\pgfqpoint{0.000000in}{0.000000in}}{\pgfqpoint{3.000000in}{3.000000in}}%
\pgfusepath{clip}%
\pgfsetbuttcap%
\pgfsetroundjoin%
\definecolor{currentfill}{rgb}{0.490196,1.000000,0.477546}%
\pgfsetfillcolor{currentfill}%
\pgfsetlinewidth{0.000000pt}%
\definecolor{currentstroke}{rgb}{0.000000,0.000000,0.000000}%
\pgfsetstrokecolor{currentstroke}%
\pgfsetdash{}{0pt}%
\pgfpathmoveto{\pgfqpoint{1.399041in}{1.706314in}}%
\pgfpathlineto{\pgfqpoint{1.392283in}{1.740398in}}%
\pgfpathlineto{\pgfqpoint{1.341727in}{1.727024in}}%
\pgfpathlineto{\pgfqpoint{1.350802in}{1.693502in}}%
\pgfpathlineto{\pgfqpoint{1.399041in}{1.706314in}}%
\pgfpathclose%
\pgfusepath{fill}%
\end{pgfscope}%
\begin{pgfscope}%
\pgfpathrectangle{\pgfqpoint{0.000000in}{0.000000in}}{\pgfqpoint{3.000000in}{3.000000in}}%
\pgfusepath{clip}%
\pgfsetbuttcap%
\pgfsetroundjoin%
\definecolor{currentfill}{rgb}{1.000000,0.175018,0.000000}%
\pgfsetfillcolor{currentfill}%
\pgfsetlinewidth{0.000000pt}%
\definecolor{currentstroke}{rgb}{0.000000,0.000000,0.000000}%
\pgfsetstrokecolor{currentstroke}%
\pgfsetdash{}{0pt}%
\pgfpathmoveto{\pgfqpoint{1.948133in}{2.145341in}}%
\pgfpathlineto{\pgfqpoint{1.958512in}{2.166169in}}%
\pgfpathlineto{\pgfqpoint{1.871578in}{2.193626in}}%
\pgfpathlineto{\pgfqpoint{1.863384in}{2.172182in}}%
\pgfpathlineto{\pgfqpoint{1.948133in}{2.145341in}}%
\pgfpathclose%
\pgfusepath{fill}%
\end{pgfscope}%
\begin{pgfscope}%
\pgfpathrectangle{\pgfqpoint{0.000000in}{0.000000in}}{\pgfqpoint{3.000000in}{3.000000in}}%
\pgfusepath{clip}%
\pgfsetbuttcap%
\pgfsetroundjoin%
\definecolor{currentfill}{rgb}{0.803030,0.000000,0.000000}%
\pgfsetfillcolor{currentfill}%
\pgfsetlinewidth{0.000000pt}%
\definecolor{currentstroke}{rgb}{0.000000,0.000000,0.000000}%
\pgfsetstrokecolor{currentstroke}%
\pgfsetdash{}{0pt}%
\pgfpathmoveto{\pgfqpoint{1.143903in}{2.267418in}}%
\pgfpathlineto{\pgfqpoint{1.134999in}{2.287350in}}%
\pgfpathlineto{\pgfqpoint{1.040402in}{2.254498in}}%
\pgfpathlineto{\pgfqpoint{1.051424in}{2.235215in}}%
\pgfpathlineto{\pgfqpoint{1.143903in}{2.267418in}}%
\pgfpathclose%
\pgfusepath{fill}%
\end{pgfscope}%
\begin{pgfscope}%
\pgfpathrectangle{\pgfqpoint{0.000000in}{0.000000in}}{\pgfqpoint{3.000000in}{3.000000in}}%
\pgfusepath{clip}%
\pgfsetbuttcap%
\pgfsetroundjoin%
\definecolor{currentfill}{rgb}{1.000000,0.814089,0.000000}%
\pgfsetfillcolor{currentfill}%
\pgfsetlinewidth{0.000000pt}%
\definecolor{currentstroke}{rgb}{0.000000,0.000000,0.000000}%
\pgfsetstrokecolor{currentstroke}%
\pgfsetdash{}{0pt}%
\pgfpathmoveto{\pgfqpoint{1.345146in}{1.947026in}}%
\pgfpathlineto{\pgfqpoint{1.338437in}{1.973234in}}%
\pgfpathlineto{\pgfqpoint{1.269376in}{1.955456in}}%
\pgfpathlineto{\pgfqpoint{1.278395in}{1.929789in}}%
\pgfpathlineto{\pgfqpoint{1.345146in}{1.947026in}}%
\pgfpathclose%
\pgfusepath{fill}%
\end{pgfscope}%
\begin{pgfscope}%
\pgfpathrectangle{\pgfqpoint{0.000000in}{0.000000in}}{\pgfqpoint{3.000000in}{3.000000in}}%
\pgfusepath{clip}%
\pgfsetbuttcap%
\pgfsetroundjoin%
\definecolor{currentfill}{rgb}{0.199241,1.000000,0.768501}%
\pgfsetfillcolor{currentfill}%
\pgfsetlinewidth{0.000000pt}%
\definecolor{currentstroke}{rgb}{0.000000,0.000000,0.000000}%
\pgfsetstrokecolor{currentstroke}%
\pgfsetdash{}{0pt}%
\pgfpathmoveto{\pgfqpoint{1.728459in}{1.573140in}}%
\pgfpathlineto{\pgfqpoint{1.738989in}{1.611782in}}%
\pgfpathlineto{\pgfqpoint{1.697974in}{1.625668in}}%
\pgfpathlineto{\pgfqpoint{1.689632in}{1.586345in}}%
\pgfpathlineto{\pgfqpoint{1.728459in}{1.573140in}}%
\pgfpathclose%
\pgfusepath{fill}%
\end{pgfscope}%
\begin{pgfscope}%
\pgfpathrectangle{\pgfqpoint{0.000000in}{0.000000in}}{\pgfqpoint{3.000000in}{3.000000in}}%
\pgfusepath{clip}%
\pgfsetbuttcap%
\pgfsetroundjoin%
\definecolor{currentfill}{rgb}{0.085389,1.000000,0.882353}%
\pgfsetfillcolor{currentfill}%
\pgfsetlinewidth{0.000000pt}%
\definecolor{currentstroke}{rgb}{0.000000,0.000000,0.000000}%
\pgfsetstrokecolor{currentstroke}%
\pgfsetdash{}{0pt}%
\pgfpathmoveto{\pgfqpoint{1.387170in}{1.540670in}}%
\pgfpathlineto{\pgfqpoint{1.378068in}{1.582285in}}%
\pgfpathlineto{\pgfqpoint{1.340626in}{1.568079in}}%
\pgfpathlineto{\pgfqpoint{1.351842in}{1.527201in}}%
\pgfpathlineto{\pgfqpoint{1.387170in}{1.540670in}}%
\pgfpathclose%
\pgfusepath{fill}%
\end{pgfscope}%
\begin{pgfscope}%
\pgfpathrectangle{\pgfqpoint{0.000000in}{0.000000in}}{\pgfqpoint{3.000000in}{3.000000in}}%
\pgfusepath{clip}%
\pgfsetbuttcap%
\pgfsetroundjoin%
\definecolor{currentfill}{rgb}{0.578748,1.000000,0.388994}%
\pgfsetfillcolor{currentfill}%
\pgfsetlinewidth{0.000000pt}%
\definecolor{currentstroke}{rgb}{0.000000,0.000000,0.000000}%
\pgfsetstrokecolor{currentstroke}%
\pgfsetdash{}{0pt}%
\pgfpathmoveto{\pgfqpoint{1.392283in}{1.740398in}}%
\pgfpathlineto{\pgfqpoint{1.385531in}{1.773069in}}%
\pgfpathlineto{\pgfqpoint{1.332659in}{1.759134in}}%
\pgfpathlineto{\pgfqpoint{1.341727in}{1.727024in}}%
\pgfpathlineto{\pgfqpoint{1.392283in}{1.740398in}}%
\pgfpathclose%
\pgfusepath{fill}%
\end{pgfscope}%
\begin{pgfscope}%
\pgfpathrectangle{\pgfqpoint{0.000000in}{0.000000in}}{\pgfqpoint{3.000000in}{3.000000in}}%
\pgfusepath{clip}%
\pgfsetbuttcap%
\pgfsetroundjoin%
\definecolor{currentfill}{rgb}{1.000000,0.886710,0.000000}%
\pgfsetfillcolor{currentfill}%
\pgfsetlinewidth{0.000000pt}%
\definecolor{currentstroke}{rgb}{0.000000,0.000000,0.000000}%
\pgfsetstrokecolor{currentstroke}%
\pgfsetdash{}{0pt}%
\pgfpathmoveto{\pgfqpoint{1.351862in}{1.920148in}}%
\pgfpathlineto{\pgfqpoint{1.345146in}{1.947026in}}%
\pgfpathlineto{\pgfqpoint{1.278395in}{1.929789in}}%
\pgfpathlineto{\pgfqpoint{1.287421in}{1.903455in}}%
\pgfpathlineto{\pgfqpoint{1.351862in}{1.920148in}}%
\pgfpathclose%
\pgfusepath{fill}%
\end{pgfscope}%
\begin{pgfscope}%
\pgfpathrectangle{\pgfqpoint{0.000000in}{0.000000in}}{\pgfqpoint{3.000000in}{3.000000in}}%
\pgfusepath{clip}%
\pgfsetbuttcap%
\pgfsetroundjoin%
\definecolor{currentfill}{rgb}{0.667299,1.000000,0.300443}%
\pgfsetfillcolor{currentfill}%
\pgfsetlinewidth{0.000000pt}%
\definecolor{currentstroke}{rgb}{0.000000,0.000000,0.000000}%
\pgfsetstrokecolor{currentstroke}%
\pgfsetdash{}{0pt}%
\pgfpathmoveto{\pgfqpoint{1.385531in}{1.773069in}}%
\pgfpathlineto{\pgfqpoint{1.378785in}{1.804490in}}%
\pgfpathlineto{\pgfqpoint{1.323597in}{1.789998in}}%
\pgfpathlineto{\pgfqpoint{1.332659in}{1.759134in}}%
\pgfpathlineto{\pgfqpoint{1.385531in}{1.773069in}}%
\pgfpathclose%
\pgfusepath{fill}%
\end{pgfscope}%
\begin{pgfscope}%
\pgfpathrectangle{\pgfqpoint{0.000000in}{0.000000in}}{\pgfqpoint{3.000000in}{3.000000in}}%
\pgfusepath{clip}%
\pgfsetbuttcap%
\pgfsetroundjoin%
\definecolor{currentfill}{rgb}{1.000000,0.233115,0.000000}%
\pgfsetfillcolor{currentfill}%
\pgfsetlinewidth{0.000000pt}%
\definecolor{currentstroke}{rgb}{0.000000,0.000000,0.000000}%
\pgfsetstrokecolor{currentstroke}%
\pgfsetdash{}{0pt}%
\pgfpathmoveto{\pgfqpoint{1.937747in}{2.124200in}}%
\pgfpathlineto{\pgfqpoint{1.948133in}{2.145341in}}%
\pgfpathlineto{\pgfqpoint{1.863384in}{2.172182in}}%
\pgfpathlineto{\pgfqpoint{1.855182in}{2.150422in}}%
\pgfpathlineto{\pgfqpoint{1.937747in}{2.124200in}}%
\pgfpathclose%
\pgfusepath{fill}%
\end{pgfscope}%
\begin{pgfscope}%
\pgfpathrectangle{\pgfqpoint{0.000000in}{0.000000in}}{\pgfqpoint{3.000000in}{3.000000in}}%
\pgfusepath{clip}%
\pgfsetbuttcap%
\pgfsetroundjoin%
\definecolor{currentfill}{rgb}{0.958254,0.973856,0.009488}%
\pgfsetfillcolor{currentfill}%
\pgfsetlinewidth{0.000000pt}%
\definecolor{currentstroke}{rgb}{0.000000,0.000000,0.000000}%
\pgfsetstrokecolor{currentstroke}%
\pgfsetdash{}{0pt}%
\pgfpathmoveto{\pgfqpoint{1.358583in}{1.892536in}}%
\pgfpathlineto{\pgfqpoint{1.351862in}{1.920148in}}%
\pgfpathlineto{\pgfqpoint{1.287421in}{1.903455in}}%
\pgfpathlineto{\pgfqpoint{1.296455in}{1.876389in}}%
\pgfpathlineto{\pgfqpoint{1.358583in}{1.892536in}}%
\pgfpathclose%
\pgfusepath{fill}%
\end{pgfscope}%
\begin{pgfscope}%
\pgfpathrectangle{\pgfqpoint{0.000000in}{0.000000in}}{\pgfqpoint{3.000000in}{3.000000in}}%
\pgfusepath{clip}%
\pgfsetbuttcap%
\pgfsetroundjoin%
\definecolor{currentfill}{rgb}{0.000000,0.676471,1.000000}%
\pgfsetfillcolor{currentfill}%
\pgfsetlinewidth{0.000000pt}%
\definecolor{currentstroke}{rgb}{0.000000,0.000000,0.000000}%
\pgfsetstrokecolor{currentstroke}%
\pgfsetdash{}{0pt}%
\pgfpathmoveto{\pgfqpoint{1.374290in}{1.436001in}}%
\pgfpathlineto{\pgfqpoint{1.363063in}{1.483369in}}%
\pgfpathlineto{\pgfqpoint{1.334081in}{1.468157in}}%
\pgfpathlineto{\pgfqpoint{1.347161in}{1.421681in}}%
\pgfpathlineto{\pgfqpoint{1.374290in}{1.436001in}}%
\pgfpathclose%
\pgfusepath{fill}%
\end{pgfscope}%
\begin{pgfscope}%
\pgfpathrectangle{\pgfqpoint{0.000000in}{0.000000in}}{\pgfqpoint{3.000000in}{3.000000in}}%
\pgfusepath{clip}%
\pgfsetbuttcap%
\pgfsetroundjoin%
\definecolor{currentfill}{rgb}{0.743201,1.000000,0.224541}%
\pgfsetfillcolor{currentfill}%
\pgfsetlinewidth{0.000000pt}%
\definecolor{currentstroke}{rgb}{0.000000,0.000000,0.000000}%
\pgfsetstrokecolor{currentstroke}%
\pgfsetdash{}{0pt}%
\pgfpathmoveto{\pgfqpoint{1.378785in}{1.804490in}}%
\pgfpathlineto{\pgfqpoint{1.372045in}{1.834800in}}%
\pgfpathlineto{\pgfqpoint{1.314543in}{1.819753in}}%
\pgfpathlineto{\pgfqpoint{1.323597in}{1.789998in}}%
\pgfpathlineto{\pgfqpoint{1.378785in}{1.804490in}}%
\pgfpathclose%
\pgfusepath{fill}%
\end{pgfscope}%
\begin{pgfscope}%
\pgfpathrectangle{\pgfqpoint{0.000000in}{0.000000in}}{\pgfqpoint{3.000000in}{3.000000in}}%
\pgfusepath{clip}%
\pgfsetbuttcap%
\pgfsetroundjoin%
\definecolor{currentfill}{rgb}{0.895003,1.000000,0.072739}%
\pgfsetfillcolor{currentfill}%
\pgfsetlinewidth{0.000000pt}%
\definecolor{currentstroke}{rgb}{0.000000,0.000000,0.000000}%
\pgfsetstrokecolor{currentstroke}%
\pgfsetdash{}{0pt}%
\pgfpathmoveto{\pgfqpoint{1.365311in}{1.864115in}}%
\pgfpathlineto{\pgfqpoint{1.358583in}{1.892536in}}%
\pgfpathlineto{\pgfqpoint{1.296455in}{1.876389in}}%
\pgfpathlineto{\pgfqpoint{1.305495in}{1.848517in}}%
\pgfpathlineto{\pgfqpoint{1.365311in}{1.864115in}}%
\pgfpathclose%
\pgfusepath{fill}%
\end{pgfscope}%
\begin{pgfscope}%
\pgfpathrectangle{\pgfqpoint{0.000000in}{0.000000in}}{\pgfqpoint{3.000000in}{3.000000in}}%
\pgfusepath{clip}%
\pgfsetbuttcap%
\pgfsetroundjoin%
\definecolor{currentfill}{rgb}{0.856506,0.000000,0.000000}%
\pgfsetfillcolor{currentfill}%
\pgfsetlinewidth{0.000000pt}%
\definecolor{currentstroke}{rgb}{0.000000,0.000000,0.000000}%
\pgfsetstrokecolor{currentstroke}%
\pgfsetdash{}{0pt}%
\pgfpathmoveto{\pgfqpoint{1.152815in}{2.247250in}}%
\pgfpathlineto{\pgfqpoint{1.143903in}{2.267418in}}%
\pgfpathlineto{\pgfqpoint{1.051424in}{2.235215in}}%
\pgfpathlineto{\pgfqpoint{1.062454in}{2.215700in}}%
\pgfpathlineto{\pgfqpoint{1.152815in}{2.247250in}}%
\pgfpathclose%
\pgfusepath{fill}%
\end{pgfscope}%
\begin{pgfscope}%
\pgfpathrectangle{\pgfqpoint{0.000000in}{0.000000in}}{\pgfqpoint{3.000000in}{3.000000in}}%
\pgfusepath{clip}%
\pgfsetbuttcap%
\pgfsetroundjoin%
\definecolor{currentfill}{rgb}{0.819102,1.000000,0.148640}%
\pgfsetfillcolor{currentfill}%
\pgfsetlinewidth{0.000000pt}%
\definecolor{currentstroke}{rgb}{0.000000,0.000000,0.000000}%
\pgfsetstrokecolor{currentstroke}%
\pgfsetdash{}{0pt}%
\pgfpathmoveto{\pgfqpoint{1.372045in}{1.834800in}}%
\pgfpathlineto{\pgfqpoint{1.365311in}{1.864115in}}%
\pgfpathlineto{\pgfqpoint{1.305495in}{1.848517in}}%
\pgfpathlineto{\pgfqpoint{1.314543in}{1.819753in}}%
\pgfpathlineto{\pgfqpoint{1.372045in}{1.834800in}}%
\pgfpathclose%
\pgfusepath{fill}%
\end{pgfscope}%
\begin{pgfscope}%
\pgfpathrectangle{\pgfqpoint{0.000000in}{0.000000in}}{\pgfqpoint{3.000000in}{3.000000in}}%
\pgfusepath{clip}%
\pgfsetbuttcap%
\pgfsetroundjoin%
\definecolor{currentfill}{rgb}{0.000000,0.503922,1.000000}%
\pgfsetfillcolor{currentfill}%
\pgfsetlinewidth{0.000000pt}%
\definecolor{currentstroke}{rgb}{0.000000,0.000000,0.000000}%
\pgfsetstrokecolor{currentstroke}%
\pgfsetdash{}{0pt}%
\pgfpathmoveto{\pgfqpoint{1.735313in}{1.360905in}}%
\pgfpathlineto{\pgfqpoint{1.749464in}{1.411000in}}%
\pgfpathlineto{\pgfqpoint{1.725362in}{1.426694in}}%
\pgfpathlineto{\pgfqpoint{1.712862in}{1.375616in}}%
\pgfpathlineto{\pgfqpoint{1.735313in}{1.360905in}}%
\pgfpathclose%
\pgfusepath{fill}%
\end{pgfscope}%
\begin{pgfscope}%
\pgfpathrectangle{\pgfqpoint{0.000000in}{0.000000in}}{\pgfqpoint{3.000000in}{3.000000in}}%
\pgfusepath{clip}%
\pgfsetbuttcap%
\pgfsetroundjoin%
\definecolor{currentfill}{rgb}{0.000000,0.000000,0.838681}%
\pgfsetfillcolor{currentfill}%
\pgfsetlinewidth{0.000000pt}%
\definecolor{currentstroke}{rgb}{0.000000,0.000000,0.000000}%
\pgfsetstrokecolor{currentstroke}%
\pgfsetdash{}{0pt}%
\pgfpathmoveto{\pgfqpoint{1.721917in}{1.131229in}}%
\pgfpathlineto{\pgfqpoint{1.738860in}{1.198223in}}%
\pgfpathlineto{\pgfqpoint{1.732842in}{1.214143in}}%
\pgfpathlineto{\pgfqpoint{1.716438in}{1.145892in}}%
\pgfpathlineto{\pgfqpoint{1.721917in}{1.131229in}}%
\pgfpathclose%
\pgfusepath{fill}%
\end{pgfscope}%
\begin{pgfscope}%
\pgfpathrectangle{\pgfqpoint{0.000000in}{0.000000in}}{\pgfqpoint{3.000000in}{3.000000in}}%
\pgfusepath{clip}%
\pgfsetbuttcap%
\pgfsetroundjoin%
\definecolor{currentfill}{rgb}{1.000000,0.291213,0.000000}%
\pgfsetfillcolor{currentfill}%
\pgfsetlinewidth{0.000000pt}%
\definecolor{currentstroke}{rgb}{0.000000,0.000000,0.000000}%
\pgfsetstrokecolor{currentstroke}%
\pgfsetdash{}{0pt}%
\pgfpathmoveto{\pgfqpoint{1.927353in}{2.102725in}}%
\pgfpathlineto{\pgfqpoint{1.937747in}{2.124200in}}%
\pgfpathlineto{\pgfqpoint{1.855182in}{2.150422in}}%
\pgfpathlineto{\pgfqpoint{1.846972in}{2.128325in}}%
\pgfpathlineto{\pgfqpoint{1.927353in}{2.102725in}}%
\pgfpathclose%
\pgfusepath{fill}%
\end{pgfscope}%
\begin{pgfscope}%
\pgfpathrectangle{\pgfqpoint{0.000000in}{0.000000in}}{\pgfqpoint{3.000000in}{3.000000in}}%
\pgfusepath{clip}%
\pgfsetbuttcap%
\pgfsetroundjoin%
\definecolor{currentfill}{rgb}{0.300443,1.000000,0.667299}%
\pgfsetfillcolor{currentfill}%
\pgfsetlinewidth{0.000000pt}%
\definecolor{currentstroke}{rgb}{0.000000,0.000000,0.000000}%
\pgfsetstrokecolor{currentstroke}%
\pgfsetdash{}{0pt}%
\pgfpathmoveto{\pgfqpoint{1.738989in}{1.611782in}}%
\pgfpathlineto{\pgfqpoint{1.749513in}{1.648287in}}%
\pgfpathlineto{\pgfqpoint{1.706310in}{1.662851in}}%
\pgfpathlineto{\pgfqpoint{1.697974in}{1.625668in}}%
\pgfpathlineto{\pgfqpoint{1.738989in}{1.611782in}}%
\pgfpathclose%
\pgfusepath{fill}%
\end{pgfscope}%
\begin{pgfscope}%
\pgfpathrectangle{\pgfqpoint{0.000000in}{0.000000in}}{\pgfqpoint{3.000000in}{3.000000in}}%
\pgfusepath{clip}%
\pgfsetbuttcap%
\pgfsetroundjoin%
\definecolor{currentfill}{rgb}{0.927807,0.015251,0.000000}%
\pgfsetfillcolor{currentfill}%
\pgfsetlinewidth{0.000000pt}%
\definecolor{currentstroke}{rgb}{0.000000,0.000000,0.000000}%
\pgfsetstrokecolor{currentstroke}%
\pgfsetdash{}{0pt}%
\pgfpathmoveto{\pgfqpoint{1.161736in}{2.226833in}}%
\pgfpathlineto{\pgfqpoint{1.152815in}{2.247250in}}%
\pgfpathlineto{\pgfqpoint{1.062454in}{2.215700in}}%
\pgfpathlineto{\pgfqpoint{1.073493in}{2.195939in}}%
\pgfpathlineto{\pgfqpoint{1.161736in}{2.226833in}}%
\pgfpathclose%
\pgfusepath{fill}%
\end{pgfscope}%
\begin{pgfscope}%
\pgfpathrectangle{\pgfqpoint{0.000000in}{0.000000in}}{\pgfqpoint{3.000000in}{3.000000in}}%
\pgfusepath{clip}%
\pgfsetbuttcap%
\pgfsetroundjoin%
\definecolor{currentfill}{rgb}{0.000000,0.849020,1.000000}%
\pgfsetfillcolor{currentfill}%
\pgfsetlinewidth{0.000000pt}%
\definecolor{currentstroke}{rgb}{0.000000,0.000000,0.000000}%
\pgfsetstrokecolor{currentstroke}%
\pgfsetdash{}{0pt}%
\pgfpathmoveto{\pgfqpoint{1.737856in}{1.473483in}}%
\pgfpathlineto{\pgfqpoint{1.750346in}{1.516738in}}%
\pgfpathlineto{\pgfqpoint{1.717922in}{1.531999in}}%
\pgfpathlineto{\pgfqpoint{1.707379in}{1.487903in}}%
\pgfpathlineto{\pgfqpoint{1.737856in}{1.473483in}}%
\pgfpathclose%
\pgfusepath{fill}%
\end{pgfscope}%
\begin{pgfscope}%
\pgfpathrectangle{\pgfqpoint{0.000000in}{0.000000in}}{\pgfqpoint{3.000000in}{3.000000in}}%
\pgfusepath{clip}%
\pgfsetbuttcap%
\pgfsetroundjoin%
\definecolor{currentfill}{rgb}{0.199241,1.000000,0.768501}%
\pgfsetfillcolor{currentfill}%
\pgfsetlinewidth{0.000000pt}%
\definecolor{currentstroke}{rgb}{0.000000,0.000000,0.000000}%
\pgfsetstrokecolor{currentstroke}%
\pgfsetdash{}{0pt}%
\pgfpathmoveto{\pgfqpoint{1.378068in}{1.582285in}}%
\pgfpathlineto{\pgfqpoint{1.368973in}{1.621399in}}%
\pgfpathlineto{\pgfqpoint{1.329417in}{1.606459in}}%
\pgfpathlineto{\pgfqpoint{1.340626in}{1.568079in}}%
\pgfpathlineto{\pgfqpoint{1.378068in}{1.582285in}}%
\pgfpathclose%
\pgfusepath{fill}%
\end{pgfscope}%
\begin{pgfscope}%
\pgfpathrectangle{\pgfqpoint{0.000000in}{0.000000in}}{\pgfqpoint{3.000000in}{3.000000in}}%
\pgfusepath{clip}%
\pgfsetbuttcap%
\pgfsetroundjoin%
\definecolor{currentfill}{rgb}{1.000000,0.349310,0.000000}%
\pgfsetfillcolor{currentfill}%
\pgfsetlinewidth{0.000000pt}%
\definecolor{currentstroke}{rgb}{0.000000,0.000000,0.000000}%
\pgfsetstrokecolor{currentstroke}%
\pgfsetdash{}{0pt}%
\pgfpathmoveto{\pgfqpoint{1.916951in}{2.080894in}}%
\pgfpathlineto{\pgfqpoint{1.927353in}{2.102725in}}%
\pgfpathlineto{\pgfqpoint{1.846972in}{2.128325in}}%
\pgfpathlineto{\pgfqpoint{1.838755in}{2.105869in}}%
\pgfpathlineto{\pgfqpoint{1.916951in}{2.080894in}}%
\pgfpathclose%
\pgfusepath{fill}%
\end{pgfscope}%
\begin{pgfscope}%
\pgfpathrectangle{\pgfqpoint{0.000000in}{0.000000in}}{\pgfqpoint{3.000000in}{3.000000in}}%
\pgfusepath{clip}%
\pgfsetbuttcap%
\pgfsetroundjoin%
\definecolor{currentfill}{rgb}{0.999109,0.073348,0.000000}%
\pgfsetfillcolor{currentfill}%
\pgfsetlinewidth{0.000000pt}%
\definecolor{currentstroke}{rgb}{0.000000,0.000000,0.000000}%
\pgfsetstrokecolor{currentstroke}%
\pgfsetdash{}{0pt}%
\pgfpathmoveto{\pgfqpoint{1.170663in}{2.206153in}}%
\pgfpathlineto{\pgfqpoint{1.161736in}{2.226833in}}%
\pgfpathlineto{\pgfqpoint{1.073493in}{2.195939in}}%
\pgfpathlineto{\pgfqpoint{1.084538in}{2.175917in}}%
\pgfpathlineto{\pgfqpoint{1.170663in}{2.206153in}}%
\pgfpathclose%
\pgfusepath{fill}%
\end{pgfscope}%
\begin{pgfscope}%
\pgfpathrectangle{\pgfqpoint{0.000000in}{0.000000in}}{\pgfqpoint{3.000000in}{3.000000in}}%
\pgfusepath{clip}%
\pgfsetbuttcap%
\pgfsetroundjoin%
\definecolor{currentfill}{rgb}{0.000000,0.300000,1.000000}%
\pgfsetfillcolor{currentfill}%
\pgfsetlinewidth{0.000000pt}%
\definecolor{currentstroke}{rgb}{0.000000,0.000000,0.000000}%
\pgfsetstrokecolor{currentstroke}%
\pgfsetdash{}{0pt}%
\pgfpathmoveto{\pgfqpoint{1.737596in}{1.290277in}}%
\pgfpathlineto{\pgfqpoint{1.753065in}{1.344557in}}%
\pgfpathlineto{\pgfqpoint{1.735313in}{1.360905in}}%
\pgfpathlineto{\pgfqpoint{1.721159in}{1.305523in}}%
\pgfpathlineto{\pgfqpoint{1.737596in}{1.290277in}}%
\pgfpathclose%
\pgfusepath{fill}%
\end{pgfscope}%
\begin{pgfscope}%
\pgfpathrectangle{\pgfqpoint{0.000000in}{0.000000in}}{\pgfqpoint{3.000000in}{3.000000in}}%
\pgfusepath{clip}%
\pgfsetbuttcap%
\pgfsetroundjoin%
\definecolor{currentfill}{rgb}{0.000000,0.064706,1.000000}%
\pgfsetfillcolor{currentfill}%
\pgfsetlinewidth{0.000000pt}%
\definecolor{currentstroke}{rgb}{0.000000,0.000000,0.000000}%
\pgfsetstrokecolor{currentstroke}%
\pgfsetdash{}{0pt}%
\pgfpathmoveto{\pgfqpoint{1.732842in}{1.214143in}}%
\pgfpathlineto{\pgfqpoint{1.749251in}{1.273869in}}%
\pgfpathlineto{\pgfqpoint{1.737596in}{1.290277in}}%
\pgfpathlineto{\pgfqpoint{1.722128in}{1.229356in}}%
\pgfpathlineto{\pgfqpoint{1.732842in}{1.214143in}}%
\pgfpathclose%
\pgfusepath{fill}%
\end{pgfscope}%
\begin{pgfscope}%
\pgfpathrectangle{\pgfqpoint{0.000000in}{0.000000in}}{\pgfqpoint{3.000000in}{3.000000in}}%
\pgfusepath{clip}%
\pgfsetbuttcap%
\pgfsetroundjoin%
\definecolor{currentfill}{rgb}{0.401645,1.000000,0.566097}%
\pgfsetfillcolor{currentfill}%
\pgfsetlinewidth{0.000000pt}%
\definecolor{currentstroke}{rgb}{0.000000,0.000000,0.000000}%
\pgfsetstrokecolor{currentstroke}%
\pgfsetdash{}{0pt}%
\pgfpathmoveto{\pgfqpoint{1.749513in}{1.648287in}}%
\pgfpathlineto{\pgfqpoint{1.760031in}{1.682949in}}%
\pgfpathlineto{\pgfqpoint{1.714639in}{1.698187in}}%
\pgfpathlineto{\pgfqpoint{1.706310in}{1.662851in}}%
\pgfpathlineto{\pgfqpoint{1.749513in}{1.648287in}}%
\pgfpathclose%
\pgfusepath{fill}%
\end{pgfscope}%
\begin{pgfscope}%
\pgfpathrectangle{\pgfqpoint{0.000000in}{0.000000in}}{\pgfqpoint{3.000000in}{3.000000in}}%
\pgfusepath{clip}%
\pgfsetbuttcap%
\pgfsetroundjoin%
\definecolor{currentfill}{rgb}{1.000000,0.407407,0.000000}%
\pgfsetfillcolor{currentfill}%
\pgfsetlinewidth{0.000000pt}%
\definecolor{currentstroke}{rgb}{0.000000,0.000000,0.000000}%
\pgfsetstrokecolor{currentstroke}%
\pgfsetdash{}{0pt}%
\pgfpathmoveto{\pgfqpoint{1.906541in}{2.058681in}}%
\pgfpathlineto{\pgfqpoint{1.916951in}{2.080894in}}%
\pgfpathlineto{\pgfqpoint{1.838755in}{2.105869in}}%
\pgfpathlineto{\pgfqpoint{1.830530in}{2.083029in}}%
\pgfpathlineto{\pgfqpoint{1.906541in}{2.058681in}}%
\pgfpathclose%
\pgfusepath{fill}%
\end{pgfscope}%
\begin{pgfscope}%
\pgfpathrectangle{\pgfqpoint{0.000000in}{0.000000in}}{\pgfqpoint{3.000000in}{3.000000in}}%
\pgfusepath{clip}%
\pgfsetbuttcap%
\pgfsetroundjoin%
\definecolor{currentfill}{rgb}{0.000000,0.000000,0.500000}%
\pgfsetfillcolor{currentfill}%
\pgfsetlinewidth{0.000000pt}%
\definecolor{currentstroke}{rgb}{0.000000,0.000000,0.000000}%
\pgfsetstrokecolor{currentstroke}%
\pgfsetdash{}{0pt}%
\pgfpathmoveto{\pgfqpoint{1.702723in}{1.025758in}}%
\pgfpathlineto{\pgfqpoint{1.719476in}{1.101402in}}%
\pgfpathlineto{\pgfqpoint{1.722945in}{1.116277in}}%
\pgfpathlineto{\pgfqpoint{1.705891in}{1.039346in}}%
\pgfpathlineto{\pgfqpoint{1.702723in}{1.025758in}}%
\pgfpathclose%
\pgfusepath{fill}%
\end{pgfscope}%
\begin{pgfscope}%
\pgfpathrectangle{\pgfqpoint{0.000000in}{0.000000in}}{\pgfqpoint{3.000000in}{3.000000in}}%
\pgfusepath{clip}%
\pgfsetbuttcap%
\pgfsetroundjoin%
\definecolor{currentfill}{rgb}{0.000000,0.503922,1.000000}%
\pgfsetfillcolor{currentfill}%
\pgfsetlinewidth{0.000000pt}%
\definecolor{currentstroke}{rgb}{0.000000,0.000000,0.000000}%
\pgfsetstrokecolor{currentstroke}%
\pgfsetdash{}{0pt}%
\pgfpathmoveto{\pgfqpoint{1.360246in}{1.370917in}}%
\pgfpathlineto{\pgfqpoint{1.347161in}{1.421681in}}%
\pgfpathlineto{\pgfqpoint{1.324674in}{1.405359in}}%
\pgfpathlineto{\pgfqpoint{1.339304in}{1.355619in}}%
\pgfpathlineto{\pgfqpoint{1.360246in}{1.370917in}}%
\pgfpathclose%
\pgfusepath{fill}%
\end{pgfscope}%
\begin{pgfscope}%
\pgfpathrectangle{\pgfqpoint{0.000000in}{0.000000in}}{\pgfqpoint{3.000000in}{3.000000in}}%
\pgfusepath{clip}%
\pgfsetbuttcap%
\pgfsetroundjoin%
\definecolor{currentfill}{rgb}{0.000000,0.000000,0.838681}%
\pgfsetfillcolor{currentfill}%
\pgfsetlinewidth{0.000000pt}%
\definecolor{currentstroke}{rgb}{0.000000,0.000000,0.000000}%
\pgfsetstrokecolor{currentstroke}%
\pgfsetdash{}{0pt}%
\pgfpathmoveto{\pgfqpoint{1.362329in}{1.141059in}}%
\pgfpathlineto{\pgfqpoint{1.345701in}{1.208896in}}%
\pgfpathlineto{\pgfqpoint{1.341291in}{1.192826in}}%
\pgfpathlineto{\pgfqpoint{1.358322in}{1.126259in}}%
\pgfpathlineto{\pgfqpoint{1.362329in}{1.141059in}}%
\pgfpathclose%
\pgfusepath{fill}%
\end{pgfscope}%
\begin{pgfscope}%
\pgfpathrectangle{\pgfqpoint{0.000000in}{0.000000in}}{\pgfqpoint{3.000000in}{3.000000in}}%
\pgfusepath{clip}%
\pgfsetbuttcap%
\pgfsetroundjoin%
\definecolor{currentfill}{rgb}{1.000000,0.116921,0.000000}%
\pgfsetfillcolor{currentfill}%
\pgfsetlinewidth{0.000000pt}%
\definecolor{currentstroke}{rgb}{0.000000,0.000000,0.000000}%
\pgfsetstrokecolor{currentstroke}%
\pgfsetdash{}{0pt}%
\pgfpathmoveto{\pgfqpoint{1.179599in}{2.185194in}}%
\pgfpathlineto{\pgfqpoint{1.170663in}{2.206153in}}%
\pgfpathlineto{\pgfqpoint{1.084538in}{2.175917in}}%
\pgfpathlineto{\pgfqpoint{1.095592in}{2.155620in}}%
\pgfpathlineto{\pgfqpoint{1.179599in}{2.185194in}}%
\pgfpathclose%
\pgfusepath{fill}%
\end{pgfscope}%
\begin{pgfscope}%
\pgfpathrectangle{\pgfqpoint{0.000000in}{0.000000in}}{\pgfqpoint{3.000000in}{3.000000in}}%
\pgfusepath{clip}%
\pgfsetbuttcap%
\pgfsetroundjoin%
\definecolor{currentfill}{rgb}{1.000000,0.480029,0.000000}%
\pgfsetfillcolor{currentfill}%
\pgfsetlinewidth{0.000000pt}%
\definecolor{currentstroke}{rgb}{0.000000,0.000000,0.000000}%
\pgfsetstrokecolor{currentstroke}%
\pgfsetdash{}{0pt}%
\pgfpathmoveto{\pgfqpoint{1.896124in}{2.036059in}}%
\pgfpathlineto{\pgfqpoint{1.906541in}{2.058681in}}%
\pgfpathlineto{\pgfqpoint{1.830530in}{2.083029in}}%
\pgfpathlineto{\pgfqpoint{1.822299in}{2.059776in}}%
\pgfpathlineto{\pgfqpoint{1.896124in}{2.036059in}}%
\pgfpathclose%
\pgfusepath{fill}%
\end{pgfscope}%
\begin{pgfscope}%
\pgfpathrectangle{\pgfqpoint{0.000000in}{0.000000in}}{\pgfqpoint{3.000000in}{3.000000in}}%
\pgfusepath{clip}%
\pgfsetbuttcap%
\pgfsetroundjoin%
\definecolor{currentfill}{rgb}{0.300443,1.000000,0.667299}%
\pgfsetfillcolor{currentfill}%
\pgfsetlinewidth{0.000000pt}%
\definecolor{currentstroke}{rgb}{0.000000,0.000000,0.000000}%
\pgfsetstrokecolor{currentstroke}%
\pgfsetdash{}{0pt}%
\pgfpathmoveto{\pgfqpoint{1.368973in}{1.621399in}}%
\pgfpathlineto{\pgfqpoint{1.359884in}{1.658374in}}%
\pgfpathlineto{\pgfqpoint{1.318214in}{1.642704in}}%
\pgfpathlineto{\pgfqpoint{1.329417in}{1.606459in}}%
\pgfpathlineto{\pgfqpoint{1.368973in}{1.621399in}}%
\pgfpathclose%
\pgfusepath{fill}%
\end{pgfscope}%
\begin{pgfscope}%
\pgfpathrectangle{\pgfqpoint{0.000000in}{0.000000in}}{\pgfqpoint{3.000000in}{3.000000in}}%
\pgfusepath{clip}%
\pgfsetbuttcap%
\pgfsetroundjoin%
\definecolor{currentfill}{rgb}{0.490196,1.000000,0.477546}%
\pgfsetfillcolor{currentfill}%
\pgfsetlinewidth{0.000000pt}%
\definecolor{currentstroke}{rgb}{0.000000,0.000000,0.000000}%
\pgfsetstrokecolor{currentstroke}%
\pgfsetdash{}{0pt}%
\pgfpathmoveto{\pgfqpoint{1.760031in}{1.682949in}}%
\pgfpathlineto{\pgfqpoint{1.770542in}{1.716005in}}%
\pgfpathlineto{\pgfqpoint{1.722962in}{1.731914in}}%
\pgfpathlineto{\pgfqpoint{1.714639in}{1.698187in}}%
\pgfpathlineto{\pgfqpoint{1.760031in}{1.682949in}}%
\pgfpathclose%
\pgfusepath{fill}%
\end{pgfscope}%
\begin{pgfscope}%
\pgfpathrectangle{\pgfqpoint{0.000000in}{0.000000in}}{\pgfqpoint{3.000000in}{3.000000in}}%
\pgfusepath{clip}%
\pgfsetbuttcap%
\pgfsetroundjoin%
\definecolor{currentfill}{rgb}{0.000000,0.849020,1.000000}%
\pgfsetfillcolor{currentfill}%
\pgfsetlinewidth{0.000000pt}%
\definecolor{currentstroke}{rgb}{0.000000,0.000000,0.000000}%
\pgfsetstrokecolor{currentstroke}%
\pgfsetdash{}{0pt}%
\pgfpathmoveto{\pgfqpoint{1.363063in}{1.483369in}}%
\pgfpathlineto{\pgfqpoint{1.351842in}{1.527201in}}%
\pgfpathlineto{\pgfqpoint{1.321004in}{1.511101in}}%
\pgfpathlineto{\pgfqpoint{1.334081in}{1.468157in}}%
\pgfpathlineto{\pgfqpoint{1.363063in}{1.483369in}}%
\pgfpathclose%
\pgfusepath{fill}%
\end{pgfscope}%
\begin{pgfscope}%
\pgfpathrectangle{\pgfqpoint{0.000000in}{0.000000in}}{\pgfqpoint{3.000000in}{3.000000in}}%
\pgfusepath{clip}%
\pgfsetbuttcap%
\pgfsetroundjoin%
\definecolor{currentfill}{rgb}{1.000000,0.538126,0.000000}%
\pgfsetfillcolor{currentfill}%
\pgfsetlinewidth{0.000000pt}%
\definecolor{currentstroke}{rgb}{0.000000,0.000000,0.000000}%
\pgfsetstrokecolor{currentstroke}%
\pgfsetdash{}{0pt}%
\pgfpathmoveto{\pgfqpoint{1.885699in}{2.012997in}}%
\pgfpathlineto{\pgfqpoint{1.896124in}{2.036059in}}%
\pgfpathlineto{\pgfqpoint{1.822299in}{2.059776in}}%
\pgfpathlineto{\pgfqpoint{1.814060in}{2.036080in}}%
\pgfpathlineto{\pgfqpoint{1.885699in}{2.012997in}}%
\pgfpathclose%
\pgfusepath{fill}%
\end{pgfscope}%
\begin{pgfscope}%
\pgfpathrectangle{\pgfqpoint{0.000000in}{0.000000in}}{\pgfqpoint{3.000000in}{3.000000in}}%
\pgfusepath{clip}%
\pgfsetbuttcap%
\pgfsetroundjoin%
\definecolor{currentfill}{rgb}{1.000000,0.175018,0.000000}%
\pgfsetfillcolor{currentfill}%
\pgfsetlinewidth{0.000000pt}%
\definecolor{currentstroke}{rgb}{0.000000,0.000000,0.000000}%
\pgfsetstrokecolor{currentstroke}%
\pgfsetdash{}{0pt}%
\pgfpathmoveto{\pgfqpoint{1.188543in}{2.163938in}}%
\pgfpathlineto{\pgfqpoint{1.179599in}{2.185194in}}%
\pgfpathlineto{\pgfqpoint{1.095592in}{2.155620in}}%
\pgfpathlineto{\pgfqpoint{1.106654in}{2.135029in}}%
\pgfpathlineto{\pgfqpoint{1.188543in}{2.163938in}}%
\pgfpathclose%
\pgfusepath{fill}%
\end{pgfscope}%
\begin{pgfscope}%
\pgfpathrectangle{\pgfqpoint{0.000000in}{0.000000in}}{\pgfqpoint{3.000000in}{3.000000in}}%
\pgfusepath{clip}%
\pgfsetbuttcap%
\pgfsetroundjoin%
\definecolor{currentfill}{rgb}{0.085389,1.000000,0.882353}%
\pgfsetfillcolor{currentfill}%
\pgfsetlinewidth{0.000000pt}%
\definecolor{currentstroke}{rgb}{0.000000,0.000000,0.000000}%
\pgfsetstrokecolor{currentstroke}%
\pgfsetdash{}{0pt}%
\pgfpathmoveto{\pgfqpoint{1.750346in}{1.516738in}}%
\pgfpathlineto{\pgfqpoint{1.762830in}{1.557042in}}%
\pgfpathlineto{\pgfqpoint{1.728459in}{1.573140in}}%
\pgfpathlineto{\pgfqpoint{1.717922in}{1.531999in}}%
\pgfpathlineto{\pgfqpoint{1.750346in}{1.516738in}}%
\pgfpathclose%
\pgfusepath{fill}%
\end{pgfscope}%
\begin{pgfscope}%
\pgfpathrectangle{\pgfqpoint{0.000000in}{0.000000in}}{\pgfqpoint{3.000000in}{3.000000in}}%
\pgfusepath{clip}%
\pgfsetbuttcap%
\pgfsetroundjoin%
\definecolor{currentfill}{rgb}{1.000000,0.610748,0.000000}%
\pgfsetfillcolor{currentfill}%
\pgfsetlinewidth{0.000000pt}%
\definecolor{currentstroke}{rgb}{0.000000,0.000000,0.000000}%
\pgfsetstrokecolor{currentstroke}%
\pgfsetdash{}{0pt}%
\pgfpathmoveto{\pgfqpoint{1.875267in}{1.989458in}}%
\pgfpathlineto{\pgfqpoint{1.885699in}{2.012997in}}%
\pgfpathlineto{\pgfqpoint{1.814060in}{2.036080in}}%
\pgfpathlineto{\pgfqpoint{1.805813in}{2.011905in}}%
\pgfpathlineto{\pgfqpoint{1.875267in}{1.989458in}}%
\pgfpathclose%
\pgfusepath{fill}%
\end{pgfscope}%
\begin{pgfscope}%
\pgfpathrectangle{\pgfqpoint{0.000000in}{0.000000in}}{\pgfqpoint{3.000000in}{3.000000in}}%
\pgfusepath{clip}%
\pgfsetbuttcap%
\pgfsetroundjoin%
\definecolor{currentfill}{rgb}{0.578748,1.000000,0.388994}%
\pgfsetfillcolor{currentfill}%
\pgfsetlinewidth{0.000000pt}%
\definecolor{currentstroke}{rgb}{0.000000,0.000000,0.000000}%
\pgfsetstrokecolor{currentstroke}%
\pgfsetdash{}{0pt}%
\pgfpathmoveto{\pgfqpoint{1.770542in}{1.716005in}}%
\pgfpathlineto{\pgfqpoint{1.781046in}{1.747652in}}%
\pgfpathlineto{\pgfqpoint{1.731278in}{1.764230in}}%
\pgfpathlineto{\pgfqpoint{1.722962in}{1.731914in}}%
\pgfpathlineto{\pgfqpoint{1.770542in}{1.716005in}}%
\pgfpathclose%
\pgfusepath{fill}%
\end{pgfscope}%
\begin{pgfscope}%
\pgfpathrectangle{\pgfqpoint{0.000000in}{0.000000in}}{\pgfqpoint{3.000000in}{3.000000in}}%
\pgfusepath{clip}%
\pgfsetbuttcap%
\pgfsetroundjoin%
\definecolor{currentfill}{rgb}{0.000000,0.300000,1.000000}%
\pgfsetfillcolor{currentfill}%
\pgfsetlinewidth{0.000000pt}%
\definecolor{currentstroke}{rgb}{0.000000,0.000000,0.000000}%
\pgfsetstrokecolor{currentstroke}%
\pgfsetdash{}{0pt}%
\pgfpathmoveto{\pgfqpoint{1.353936in}{1.300593in}}%
\pgfpathlineto{\pgfqpoint{1.339304in}{1.355619in}}%
\pgfpathlineto{\pgfqpoint{1.323224in}{1.338809in}}%
\pgfpathlineto{\pgfqpoint{1.339051in}{1.284918in}}%
\pgfpathlineto{\pgfqpoint{1.353936in}{1.300593in}}%
\pgfpathclose%
\pgfusepath{fill}%
\end{pgfscope}%
\begin{pgfscope}%
\pgfpathrectangle{\pgfqpoint{0.000000in}{0.000000in}}{\pgfqpoint{3.000000in}{3.000000in}}%
\pgfusepath{clip}%
\pgfsetbuttcap%
\pgfsetroundjoin%
\definecolor{currentfill}{rgb}{1.000000,0.233115,0.000000}%
\pgfsetfillcolor{currentfill}%
\pgfsetlinewidth{0.000000pt}%
\definecolor{currentstroke}{rgb}{0.000000,0.000000,0.000000}%
\pgfsetstrokecolor{currentstroke}%
\pgfsetdash{}{0pt}%
\pgfpathmoveto{\pgfqpoint{1.197494in}{2.142368in}}%
\pgfpathlineto{\pgfqpoint{1.188543in}{2.163938in}}%
\pgfpathlineto{\pgfqpoint{1.106654in}{2.135029in}}%
\pgfpathlineto{\pgfqpoint{1.117723in}{2.114127in}}%
\pgfpathlineto{\pgfqpoint{1.197494in}{2.142368in}}%
\pgfpathclose%
\pgfusepath{fill}%
\end{pgfscope}%
\begin{pgfscope}%
\pgfpathrectangle{\pgfqpoint{0.000000in}{0.000000in}}{\pgfqpoint{3.000000in}{3.000000in}}%
\pgfusepath{clip}%
\pgfsetbuttcap%
\pgfsetroundjoin%
\definecolor{currentfill}{rgb}{0.000000,0.064706,1.000000}%
\pgfsetfillcolor{currentfill}%
\pgfsetlinewidth{0.000000pt}%
\definecolor{currentstroke}{rgb}{0.000000,0.000000,0.000000}%
\pgfsetstrokecolor{currentstroke}%
\pgfsetdash{}{0pt}%
\pgfpathmoveto{\pgfqpoint{1.354875in}{1.224387in}}%
\pgfpathlineto{\pgfqpoint{1.339051in}{1.284918in}}%
\pgfpathlineto{\pgfqpoint{1.329065in}{1.268209in}}%
\pgfpathlineto{\pgfqpoint{1.345701in}{1.208896in}}%
\pgfpathlineto{\pgfqpoint{1.354875in}{1.224387in}}%
\pgfpathclose%
\pgfusepath{fill}%
\end{pgfscope}%
\begin{pgfscope}%
\pgfpathrectangle{\pgfqpoint{0.000000in}{0.000000in}}{\pgfqpoint{3.000000in}{3.000000in}}%
\pgfusepath{clip}%
\pgfsetbuttcap%
\pgfsetroundjoin%
\definecolor{currentfill}{rgb}{1.000000,0.668845,0.000000}%
\pgfsetfillcolor{currentfill}%
\pgfsetlinewidth{0.000000pt}%
\definecolor{currentstroke}{rgb}{0.000000,0.000000,0.000000}%
\pgfsetstrokecolor{currentstroke}%
\pgfsetdash{}{0pt}%
\pgfpathmoveto{\pgfqpoint{1.864827in}{1.965406in}}%
\pgfpathlineto{\pgfqpoint{1.875267in}{1.989458in}}%
\pgfpathlineto{\pgfqpoint{1.805813in}{2.011905in}}%
\pgfpathlineto{\pgfqpoint{1.797560in}{1.987213in}}%
\pgfpathlineto{\pgfqpoint{1.864827in}{1.965406in}}%
\pgfpathclose%
\pgfusepath{fill}%
\end{pgfscope}%
\begin{pgfscope}%
\pgfpathrectangle{\pgfqpoint{0.000000in}{0.000000in}}{\pgfqpoint{3.000000in}{3.000000in}}%
\pgfusepath{clip}%
\pgfsetbuttcap%
\pgfsetroundjoin%
\definecolor{currentfill}{rgb}{0.000000,0.000000,0.500000}%
\pgfsetfillcolor{currentfill}%
\pgfsetlinewidth{0.000000pt}%
\definecolor{currentstroke}{rgb}{0.000000,0.000000,0.000000}%
\pgfsetstrokecolor{currentstroke}%
\pgfsetdash{}{0pt}%
\pgfpathmoveto{\pgfqpoint{1.375794in}{1.034791in}}%
\pgfpathlineto{\pgfqpoint{1.358794in}{1.111292in}}%
\pgfpathlineto{\pgfqpoint{1.363753in}{1.096525in}}%
\pgfpathlineto{\pgfqpoint{1.380312in}{1.021303in}}%
\pgfpathlineto{\pgfqpoint{1.375794in}{1.034791in}}%
\pgfpathclose%
\pgfusepath{fill}%
\end{pgfscope}%
\begin{pgfscope}%
\pgfpathrectangle{\pgfqpoint{0.000000in}{0.000000in}}{\pgfqpoint{3.000000in}{3.000000in}}%
\pgfusepath{clip}%
\pgfsetbuttcap%
\pgfsetroundjoin%
\definecolor{currentfill}{rgb}{0.401645,1.000000,0.566097}%
\pgfsetfillcolor{currentfill}%
\pgfsetlinewidth{0.000000pt}%
\definecolor{currentstroke}{rgb}{0.000000,0.000000,0.000000}%
\pgfsetstrokecolor{currentstroke}%
\pgfsetdash{}{0pt}%
\pgfpathmoveto{\pgfqpoint{1.359884in}{1.658374in}}%
\pgfpathlineto{\pgfqpoint{1.350802in}{1.693502in}}%
\pgfpathlineto{\pgfqpoint{1.307018in}{1.677106in}}%
\pgfpathlineto{\pgfqpoint{1.318214in}{1.642704in}}%
\pgfpathlineto{\pgfqpoint{1.359884in}{1.658374in}}%
\pgfpathclose%
\pgfusepath{fill}%
\end{pgfscope}%
\begin{pgfscope}%
\pgfpathrectangle{\pgfqpoint{0.000000in}{0.000000in}}{\pgfqpoint{3.000000in}{3.000000in}}%
\pgfusepath{clip}%
\pgfsetbuttcap%
\pgfsetroundjoin%
\definecolor{currentfill}{rgb}{0.000000,0.676471,1.000000}%
\pgfsetfillcolor{currentfill}%
\pgfsetlinewidth{0.000000pt}%
\definecolor{currentstroke}{rgb}{0.000000,0.000000,0.000000}%
\pgfsetstrokecolor{currentstroke}%
\pgfsetdash{}{0pt}%
\pgfpathmoveto{\pgfqpoint{1.749464in}{1.411000in}}%
\pgfpathlineto{\pgfqpoint{1.763613in}{1.456808in}}%
\pgfpathlineto{\pgfqpoint{1.737856in}{1.473483in}}%
\pgfpathlineto{\pgfqpoint{1.725362in}{1.426694in}}%
\pgfpathlineto{\pgfqpoint{1.749464in}{1.411000in}}%
\pgfpathclose%
\pgfusepath{fill}%
\end{pgfscope}%
\begin{pgfscope}%
\pgfpathrectangle{\pgfqpoint{0.000000in}{0.000000in}}{\pgfqpoint{3.000000in}{3.000000in}}%
\pgfusepath{clip}%
\pgfsetbuttcap%
\pgfsetroundjoin%
\definecolor{currentfill}{rgb}{0.667299,1.000000,0.300443}%
\pgfsetfillcolor{currentfill}%
\pgfsetlinewidth{0.000000pt}%
\definecolor{currentstroke}{rgb}{0.000000,0.000000,0.000000}%
\pgfsetstrokecolor{currentstroke}%
\pgfsetdash{}{0pt}%
\pgfpathmoveto{\pgfqpoint{1.781046in}{1.747652in}}%
\pgfpathlineto{\pgfqpoint{1.791543in}{1.778055in}}%
\pgfpathlineto{\pgfqpoint{1.739587in}{1.795298in}}%
\pgfpathlineto{\pgfqpoint{1.731278in}{1.764230in}}%
\pgfpathlineto{\pgfqpoint{1.781046in}{1.747652in}}%
\pgfpathclose%
\pgfusepath{fill}%
\end{pgfscope}%
\begin{pgfscope}%
\pgfpathrectangle{\pgfqpoint{0.000000in}{0.000000in}}{\pgfqpoint{3.000000in}{3.000000in}}%
\pgfusepath{clip}%
\pgfsetbuttcap%
\pgfsetroundjoin%
\definecolor{currentfill}{rgb}{1.000000,0.741467,0.000000}%
\pgfsetfillcolor{currentfill}%
\pgfsetlinewidth{0.000000pt}%
\definecolor{currentstroke}{rgb}{0.000000,0.000000,0.000000}%
\pgfsetstrokecolor{currentstroke}%
\pgfsetdash{}{0pt}%
\pgfpathmoveto{\pgfqpoint{1.854379in}{1.940795in}}%
\pgfpathlineto{\pgfqpoint{1.864827in}{1.965406in}}%
\pgfpathlineto{\pgfqpoint{1.797560in}{1.987213in}}%
\pgfpathlineto{\pgfqpoint{1.789299in}{1.961959in}}%
\pgfpathlineto{\pgfqpoint{1.854379in}{1.940795in}}%
\pgfpathclose%
\pgfusepath{fill}%
\end{pgfscope}%
\begin{pgfscope}%
\pgfpathrectangle{\pgfqpoint{0.000000in}{0.000000in}}{\pgfqpoint{3.000000in}{3.000000in}}%
\pgfusepath{clip}%
\pgfsetbuttcap%
\pgfsetroundjoin%
\definecolor{currentfill}{rgb}{1.000000,0.291213,0.000000}%
\pgfsetfillcolor{currentfill}%
\pgfsetlinewidth{0.000000pt}%
\definecolor{currentstroke}{rgb}{0.000000,0.000000,0.000000}%
\pgfsetstrokecolor{currentstroke}%
\pgfsetdash{}{0pt}%
\pgfpathmoveto{\pgfqpoint{1.206452in}{2.120462in}}%
\pgfpathlineto{\pgfqpoint{1.197494in}{2.142368in}}%
\pgfpathlineto{\pgfqpoint{1.117723in}{2.114127in}}%
\pgfpathlineto{\pgfqpoint{1.128799in}{2.092892in}}%
\pgfpathlineto{\pgfqpoint{1.206452in}{2.120462in}}%
\pgfpathclose%
\pgfusepath{fill}%
\end{pgfscope}%
\begin{pgfscope}%
\pgfpathrectangle{\pgfqpoint{0.000000in}{0.000000in}}{\pgfqpoint{3.000000in}{3.000000in}}%
\pgfusepath{clip}%
\pgfsetbuttcap%
\pgfsetroundjoin%
\definecolor{currentfill}{rgb}{0.743201,1.000000,0.224541}%
\pgfsetfillcolor{currentfill}%
\pgfsetlinewidth{0.000000pt}%
\definecolor{currentstroke}{rgb}{0.000000,0.000000,0.000000}%
\pgfsetstrokecolor{currentstroke}%
\pgfsetdash{}{0pt}%
\pgfpathmoveto{\pgfqpoint{1.791543in}{1.778055in}}%
\pgfpathlineto{\pgfqpoint{1.802034in}{1.807352in}}%
\pgfpathlineto{\pgfqpoint{1.747890in}{1.825256in}}%
\pgfpathlineto{\pgfqpoint{1.739587in}{1.795298in}}%
\pgfpathlineto{\pgfqpoint{1.791543in}{1.778055in}}%
\pgfpathclose%
\pgfusepath{fill}%
\end{pgfscope}%
\begin{pgfscope}%
\pgfpathrectangle{\pgfqpoint{0.000000in}{0.000000in}}{\pgfqpoint{3.000000in}{3.000000in}}%
\pgfusepath{clip}%
\pgfsetbuttcap%
\pgfsetroundjoin%
\definecolor{currentfill}{rgb}{1.000000,0.814089,0.000000}%
\pgfsetfillcolor{currentfill}%
\pgfsetlinewidth{0.000000pt}%
\definecolor{currentstroke}{rgb}{0.000000,0.000000,0.000000}%
\pgfsetstrokecolor{currentstroke}%
\pgfsetdash{}{0pt}%
\pgfpathmoveto{\pgfqpoint{1.843925in}{1.915575in}}%
\pgfpathlineto{\pgfqpoint{1.854379in}{1.940795in}}%
\pgfpathlineto{\pgfqpoint{1.789299in}{1.961959in}}%
\pgfpathlineto{\pgfqpoint{1.781031in}{1.936094in}}%
\pgfpathlineto{\pgfqpoint{1.843925in}{1.915575in}}%
\pgfpathclose%
\pgfusepath{fill}%
\end{pgfscope}%
\begin{pgfscope}%
\pgfpathrectangle{\pgfqpoint{0.000000in}{0.000000in}}{\pgfqpoint{3.000000in}{3.000000in}}%
\pgfusepath{clip}%
\pgfsetbuttcap%
\pgfsetroundjoin%
\definecolor{currentfill}{rgb}{0.819102,1.000000,0.148640}%
\pgfsetfillcolor{currentfill}%
\pgfsetlinewidth{0.000000pt}%
\definecolor{currentstroke}{rgb}{0.000000,0.000000,0.000000}%
\pgfsetstrokecolor{currentstroke}%
\pgfsetdash{}{0pt}%
\pgfpathmoveto{\pgfqpoint{1.802034in}{1.807352in}}%
\pgfpathlineto{\pgfqpoint{1.812517in}{1.835660in}}%
\pgfpathlineto{\pgfqpoint{1.756186in}{1.854222in}}%
\pgfpathlineto{\pgfqpoint{1.747890in}{1.825256in}}%
\pgfpathlineto{\pgfqpoint{1.802034in}{1.807352in}}%
\pgfpathclose%
\pgfusepath{fill}%
\end{pgfscope}%
\begin{pgfscope}%
\pgfpathrectangle{\pgfqpoint{0.000000in}{0.000000in}}{\pgfqpoint{3.000000in}{3.000000in}}%
\pgfusepath{clip}%
\pgfsetbuttcap%
\pgfsetroundjoin%
\definecolor{currentfill}{rgb}{1.000000,0.886710,0.000000}%
\pgfsetfillcolor{currentfill}%
\pgfsetlinewidth{0.000000pt}%
\definecolor{currentstroke}{rgb}{0.000000,0.000000,0.000000}%
\pgfsetstrokecolor{currentstroke}%
\pgfsetdash{}{0pt}%
\pgfpathmoveto{\pgfqpoint{1.833463in}{1.889691in}}%
\pgfpathlineto{\pgfqpoint{1.843925in}{1.915575in}}%
\pgfpathlineto{\pgfqpoint{1.781031in}{1.936094in}}%
\pgfpathlineto{\pgfqpoint{1.772756in}{1.909561in}}%
\pgfpathlineto{\pgfqpoint{1.833463in}{1.889691in}}%
\pgfpathclose%
\pgfusepath{fill}%
\end{pgfscope}%
\begin{pgfscope}%
\pgfpathrectangle{\pgfqpoint{0.000000in}{0.000000in}}{\pgfqpoint{3.000000in}{3.000000in}}%
\pgfusepath{clip}%
\pgfsetbuttcap%
\pgfsetroundjoin%
\definecolor{currentfill}{rgb}{0.895003,1.000000,0.072739}%
\pgfsetfillcolor{currentfill}%
\pgfsetlinewidth{0.000000pt}%
\definecolor{currentstroke}{rgb}{0.000000,0.000000,0.000000}%
\pgfsetstrokecolor{currentstroke}%
\pgfsetdash{}{0pt}%
\pgfpathmoveto{\pgfqpoint{1.812517in}{1.835660in}}%
\pgfpathlineto{\pgfqpoint{1.822993in}{1.863077in}}%
\pgfpathlineto{\pgfqpoint{1.764474in}{1.882295in}}%
\pgfpathlineto{\pgfqpoint{1.756186in}{1.854222in}}%
\pgfpathlineto{\pgfqpoint{1.812517in}{1.835660in}}%
\pgfpathclose%
\pgfusepath{fill}%
\end{pgfscope}%
\begin{pgfscope}%
\pgfpathrectangle{\pgfqpoint{0.000000in}{0.000000in}}{\pgfqpoint{3.000000in}{3.000000in}}%
\pgfusepath{clip}%
\pgfsetbuttcap%
\pgfsetroundjoin%
\definecolor{currentfill}{rgb}{0.958254,0.973856,0.009488}%
\pgfsetfillcolor{currentfill}%
\pgfsetlinewidth{0.000000pt}%
\definecolor{currentstroke}{rgb}{0.000000,0.000000,0.000000}%
\pgfsetstrokecolor{currentstroke}%
\pgfsetdash{}{0pt}%
\pgfpathmoveto{\pgfqpoint{1.822993in}{1.863077in}}%
\pgfpathlineto{\pgfqpoint{1.833463in}{1.889691in}}%
\pgfpathlineto{\pgfqpoint{1.772756in}{1.909561in}}%
\pgfpathlineto{\pgfqpoint{1.764474in}{1.882295in}}%
\pgfpathlineto{\pgfqpoint{1.822993in}{1.863077in}}%
\pgfpathclose%
\pgfusepath{fill}%
\end{pgfscope}%
\begin{pgfscope}%
\pgfpathrectangle{\pgfqpoint{0.000000in}{0.000000in}}{\pgfqpoint{3.000000in}{3.000000in}}%
\pgfusepath{clip}%
\pgfsetbuttcap%
\pgfsetroundjoin%
\definecolor{currentfill}{rgb}{1.000000,0.349310,0.000000}%
\pgfsetfillcolor{currentfill}%
\pgfsetlinewidth{0.000000pt}%
\definecolor{currentstroke}{rgb}{0.000000,0.000000,0.000000}%
\pgfsetstrokecolor{currentstroke}%
\pgfsetdash{}{0pt}%
\pgfpathmoveto{\pgfqpoint{1.215419in}{2.098198in}}%
\pgfpathlineto{\pgfqpoint{1.206452in}{2.120462in}}%
\pgfpathlineto{\pgfqpoint{1.128799in}{2.092892in}}%
\pgfpathlineto{\pgfqpoint{1.139884in}{2.071302in}}%
\pgfpathlineto{\pgfqpoint{1.215419in}{2.098198in}}%
\pgfpathclose%
\pgfusepath{fill}%
\end{pgfscope}%
\begin{pgfscope}%
\pgfpathrectangle{\pgfqpoint{0.000000in}{0.000000in}}{\pgfqpoint{3.000000in}{3.000000in}}%
\pgfusepath{clip}%
\pgfsetbuttcap%
\pgfsetroundjoin%
\definecolor{currentfill}{rgb}{0.490196,1.000000,0.477546}%
\pgfsetfillcolor{currentfill}%
\pgfsetlinewidth{0.000000pt}%
\definecolor{currentstroke}{rgb}{0.000000,0.000000,0.000000}%
\pgfsetstrokecolor{currentstroke}%
\pgfsetdash{}{0pt}%
\pgfpathmoveto{\pgfqpoint{1.350802in}{1.693502in}}%
\pgfpathlineto{\pgfqpoint{1.341727in}{1.727024in}}%
\pgfpathlineto{\pgfqpoint{1.295827in}{1.709904in}}%
\pgfpathlineto{\pgfqpoint{1.307018in}{1.677106in}}%
\pgfpathlineto{\pgfqpoint{1.350802in}{1.693502in}}%
\pgfpathclose%
\pgfusepath{fill}%
\end{pgfscope}%
\begin{pgfscope}%
\pgfpathrectangle{\pgfqpoint{0.000000in}{0.000000in}}{\pgfqpoint{3.000000in}{3.000000in}}%
\pgfusepath{clip}%
\pgfsetbuttcap%
\pgfsetroundjoin%
\definecolor{currentfill}{rgb}{0.199241,1.000000,0.768501}%
\pgfsetfillcolor{currentfill}%
\pgfsetlinewidth{0.000000pt}%
\definecolor{currentstroke}{rgb}{0.000000,0.000000,0.000000}%
\pgfsetstrokecolor{currentstroke}%
\pgfsetdash{}{0pt}%
\pgfpathmoveto{\pgfqpoint{1.762830in}{1.557042in}}%
\pgfpathlineto{\pgfqpoint{1.775310in}{1.594850in}}%
\pgfpathlineto{\pgfqpoint{1.738989in}{1.611782in}}%
\pgfpathlineto{\pgfqpoint{1.728459in}{1.573140in}}%
\pgfpathlineto{\pgfqpoint{1.762830in}{1.557042in}}%
\pgfpathclose%
\pgfusepath{fill}%
\end{pgfscope}%
\begin{pgfscope}%
\pgfpathrectangle{\pgfqpoint{0.000000in}{0.000000in}}{\pgfqpoint{3.000000in}{3.000000in}}%
\pgfusepath{clip}%
\pgfsetbuttcap%
\pgfsetroundjoin%
\definecolor{currentfill}{rgb}{0.085389,1.000000,0.882353}%
\pgfsetfillcolor{currentfill}%
\pgfsetlinewidth{0.000000pt}%
\definecolor{currentstroke}{rgb}{0.000000,0.000000,0.000000}%
\pgfsetstrokecolor{currentstroke}%
\pgfsetdash{}{0pt}%
\pgfpathmoveto{\pgfqpoint{1.351842in}{1.527201in}}%
\pgfpathlineto{\pgfqpoint{1.340626in}{1.568079in}}%
\pgfpathlineto{\pgfqpoint{1.307932in}{1.551094in}}%
\pgfpathlineto{\pgfqpoint{1.321004in}{1.511101in}}%
\pgfpathlineto{\pgfqpoint{1.351842in}{1.527201in}}%
\pgfpathclose%
\pgfusepath{fill}%
\end{pgfscope}%
\begin{pgfscope}%
\pgfpathrectangle{\pgfqpoint{0.000000in}{0.000000in}}{\pgfqpoint{3.000000in}{3.000000in}}%
\pgfusepath{clip}%
\pgfsetbuttcap%
\pgfsetroundjoin%
\definecolor{currentfill}{rgb}{1.000000,0.407407,0.000000}%
\pgfsetfillcolor{currentfill}%
\pgfsetlinewidth{0.000000pt}%
\definecolor{currentstroke}{rgb}{0.000000,0.000000,0.000000}%
\pgfsetstrokecolor{currentstroke}%
\pgfsetdash{}{0pt}%
\pgfpathmoveto{\pgfqpoint{1.224393in}{2.075550in}}%
\pgfpathlineto{\pgfqpoint{1.215419in}{2.098198in}}%
\pgfpathlineto{\pgfqpoint{1.139884in}{2.071302in}}%
\pgfpathlineto{\pgfqpoint{1.150976in}{2.049332in}}%
\pgfpathlineto{\pgfqpoint{1.224393in}{2.075550in}}%
\pgfpathclose%
\pgfusepath{fill}%
\end{pgfscope}%
\begin{pgfscope}%
\pgfpathrectangle{\pgfqpoint{0.000000in}{0.000000in}}{\pgfqpoint{3.000000in}{3.000000in}}%
\pgfusepath{clip}%
\pgfsetbuttcap%
\pgfsetroundjoin%
\definecolor{currentfill}{rgb}{0.500000,0.000000,0.000000}%
\pgfsetfillcolor{currentfill}%
\pgfsetlinewidth{0.000000pt}%
\definecolor{currentstroke}{rgb}{0.000000,0.000000,0.000000}%
\pgfsetstrokecolor{currentstroke}%
\pgfsetdash{}{0pt}%
\pgfpathmoveto{\pgfqpoint{2.146828in}{2.301880in}}%
\pgfpathlineto{\pgfqpoint{2.159104in}{2.319662in}}%
\pgfpathlineto{\pgfqpoint{2.061858in}{2.360678in}}%
\pgfpathlineto{\pgfqpoint{2.051560in}{2.342169in}}%
\pgfpathlineto{\pgfqpoint{2.146828in}{2.301880in}}%
\pgfpathclose%
\pgfusepath{fill}%
\end{pgfscope}%
\begin{pgfscope}%
\pgfpathrectangle{\pgfqpoint{0.000000in}{0.000000in}}{\pgfqpoint{3.000000in}{3.000000in}}%
\pgfusepath{clip}%
\pgfsetbuttcap%
\pgfsetroundjoin%
\definecolor{currentfill}{rgb}{0.578748,1.000000,0.388994}%
\pgfsetfillcolor{currentfill}%
\pgfsetlinewidth{0.000000pt}%
\definecolor{currentstroke}{rgb}{0.000000,0.000000,0.000000}%
\pgfsetstrokecolor{currentstroke}%
\pgfsetdash{}{0pt}%
\pgfpathmoveto{\pgfqpoint{1.341727in}{1.727024in}}%
\pgfpathlineto{\pgfqpoint{1.332659in}{1.759134in}}%
\pgfpathlineto{\pgfqpoint{1.284644in}{1.741294in}}%
\pgfpathlineto{\pgfqpoint{1.295827in}{1.709904in}}%
\pgfpathlineto{\pgfqpoint{1.341727in}{1.727024in}}%
\pgfpathclose%
\pgfusepath{fill}%
\end{pgfscope}%
\begin{pgfscope}%
\pgfpathrectangle{\pgfqpoint{0.000000in}{0.000000in}}{\pgfqpoint{3.000000in}{3.000000in}}%
\pgfusepath{clip}%
\pgfsetbuttcap%
\pgfsetroundjoin%
\definecolor{currentfill}{rgb}{1.000000,0.480029,0.000000}%
\pgfsetfillcolor{currentfill}%
\pgfsetlinewidth{0.000000pt}%
\definecolor{currentstroke}{rgb}{0.000000,0.000000,0.000000}%
\pgfsetstrokecolor{currentstroke}%
\pgfsetdash{}{0pt}%
\pgfpathmoveto{\pgfqpoint{1.233375in}{2.052490in}}%
\pgfpathlineto{\pgfqpoint{1.224393in}{2.075550in}}%
\pgfpathlineto{\pgfqpoint{1.150976in}{2.049332in}}%
\pgfpathlineto{\pgfqpoint{1.162075in}{2.026953in}}%
\pgfpathlineto{\pgfqpoint{1.233375in}{2.052490in}}%
\pgfpathclose%
\pgfusepath{fill}%
\end{pgfscope}%
\begin{pgfscope}%
\pgfpathrectangle{\pgfqpoint{0.000000in}{0.000000in}}{\pgfqpoint{3.000000in}{3.000000in}}%
\pgfusepath{clip}%
\pgfsetbuttcap%
\pgfsetroundjoin%
\definecolor{currentfill}{rgb}{0.000000,0.676471,1.000000}%
\pgfsetfillcolor{currentfill}%
\pgfsetlinewidth{0.000000pt}%
\definecolor{currentstroke}{rgb}{0.000000,0.000000,0.000000}%
\pgfsetstrokecolor{currentstroke}%
\pgfsetdash{}{0pt}%
\pgfpathmoveto{\pgfqpoint{1.347161in}{1.421681in}}%
\pgfpathlineto{\pgfqpoint{1.334081in}{1.468157in}}%
\pgfpathlineto{\pgfqpoint{1.310046in}{1.450815in}}%
\pgfpathlineto{\pgfqpoint{1.324674in}{1.405359in}}%
\pgfpathlineto{\pgfqpoint{1.347161in}{1.421681in}}%
\pgfpathclose%
\pgfusepath{fill}%
\end{pgfscope}%
\begin{pgfscope}%
\pgfpathrectangle{\pgfqpoint{0.000000in}{0.000000in}}{\pgfqpoint{3.000000in}{3.000000in}}%
\pgfusepath{clip}%
\pgfsetbuttcap%
\pgfsetroundjoin%
\definecolor{currentfill}{rgb}{0.000000,0.000000,0.838681}%
\pgfsetfillcolor{currentfill}%
\pgfsetlinewidth{0.000000pt}%
\definecolor{currentstroke}{rgb}{0.000000,0.000000,0.000000}%
\pgfsetstrokecolor{currentstroke}%
\pgfsetdash{}{0pt}%
\pgfpathmoveto{\pgfqpoint{1.722945in}{1.116277in}}%
\pgfpathlineto{\pgfqpoint{1.740013in}{1.181983in}}%
\pgfpathlineto{\pgfqpoint{1.738860in}{1.198223in}}%
\pgfpathlineto{\pgfqpoint{1.721917in}{1.131229in}}%
\pgfpathlineto{\pgfqpoint{1.722945in}{1.116277in}}%
\pgfpathclose%
\pgfusepath{fill}%
\end{pgfscope}%
\begin{pgfscope}%
\pgfpathrectangle{\pgfqpoint{0.000000in}{0.000000in}}{\pgfqpoint{3.000000in}{3.000000in}}%
\pgfusepath{clip}%
\pgfsetbuttcap%
\pgfsetroundjoin%
\definecolor{currentfill}{rgb}{0.553476,0.000000,0.000000}%
\pgfsetfillcolor{currentfill}%
\pgfsetlinewidth{0.000000pt}%
\definecolor{currentstroke}{rgb}{0.000000,0.000000,0.000000}%
\pgfsetstrokecolor{currentstroke}%
\pgfsetdash{}{0pt}%
\pgfpathmoveto{\pgfqpoint{2.134543in}{2.283919in}}%
\pgfpathlineto{\pgfqpoint{2.146828in}{2.301880in}}%
\pgfpathlineto{\pgfqpoint{2.051560in}{2.342169in}}%
\pgfpathlineto{\pgfqpoint{2.041254in}{2.323477in}}%
\pgfpathlineto{\pgfqpoint{2.134543in}{2.283919in}}%
\pgfpathclose%
\pgfusepath{fill}%
\end{pgfscope}%
\begin{pgfscope}%
\pgfpathrectangle{\pgfqpoint{0.000000in}{0.000000in}}{\pgfqpoint{3.000000in}{3.000000in}}%
\pgfusepath{clip}%
\pgfsetbuttcap%
\pgfsetroundjoin%
\definecolor{currentfill}{rgb}{1.000000,0.538126,0.000000}%
\pgfsetfillcolor{currentfill}%
\pgfsetlinewidth{0.000000pt}%
\definecolor{currentstroke}{rgb}{0.000000,0.000000,0.000000}%
\pgfsetstrokecolor{currentstroke}%
\pgfsetdash{}{0pt}%
\pgfpathmoveto{\pgfqpoint{1.242364in}{2.028988in}}%
\pgfpathlineto{\pgfqpoint{1.233375in}{2.052490in}}%
\pgfpathlineto{\pgfqpoint{1.162075in}{2.026953in}}%
\pgfpathlineto{\pgfqpoint{1.173182in}{2.004134in}}%
\pgfpathlineto{\pgfqpoint{1.242364in}{2.028988in}}%
\pgfpathclose%
\pgfusepath{fill}%
\end{pgfscope}%
\begin{pgfscope}%
\pgfpathrectangle{\pgfqpoint{0.000000in}{0.000000in}}{\pgfqpoint{3.000000in}{3.000000in}}%
\pgfusepath{clip}%
\pgfsetbuttcap%
\pgfsetroundjoin%
\definecolor{currentfill}{rgb}{0.667299,1.000000,0.300443}%
\pgfsetfillcolor{currentfill}%
\pgfsetlinewidth{0.000000pt}%
\definecolor{currentstroke}{rgb}{0.000000,0.000000,0.000000}%
\pgfsetstrokecolor{currentstroke}%
\pgfsetdash{}{0pt}%
\pgfpathmoveto{\pgfqpoint{1.332659in}{1.759134in}}%
\pgfpathlineto{\pgfqpoint{1.323597in}{1.789998in}}%
\pgfpathlineto{\pgfqpoint{1.273466in}{1.771442in}}%
\pgfpathlineto{\pgfqpoint{1.284644in}{1.741294in}}%
\pgfpathlineto{\pgfqpoint{1.332659in}{1.759134in}}%
\pgfpathclose%
\pgfusepath{fill}%
\end{pgfscope}%
\begin{pgfscope}%
\pgfpathrectangle{\pgfqpoint{0.000000in}{0.000000in}}{\pgfqpoint{3.000000in}{3.000000in}}%
\pgfusepath{clip}%
\pgfsetbuttcap%
\pgfsetroundjoin%
\definecolor{currentfill}{rgb}{0.000000,0.503922,1.000000}%
\pgfsetfillcolor{currentfill}%
\pgfsetlinewidth{0.000000pt}%
\definecolor{currentstroke}{rgb}{0.000000,0.000000,0.000000}%
\pgfsetstrokecolor{currentstroke}%
\pgfsetdash{}{0pt}%
\pgfpathmoveto{\pgfqpoint{1.753065in}{1.344557in}}%
\pgfpathlineto{\pgfqpoint{1.768535in}{1.393553in}}%
\pgfpathlineto{\pgfqpoint{1.749464in}{1.411000in}}%
\pgfpathlineto{\pgfqpoint{1.735313in}{1.360905in}}%
\pgfpathlineto{\pgfqpoint{1.753065in}{1.344557in}}%
\pgfpathclose%
\pgfusepath{fill}%
\end{pgfscope}%
\begin{pgfscope}%
\pgfpathrectangle{\pgfqpoint{0.000000in}{0.000000in}}{\pgfqpoint{3.000000in}{3.000000in}}%
\pgfusepath{clip}%
\pgfsetbuttcap%
\pgfsetroundjoin%
\definecolor{currentfill}{rgb}{1.000000,0.610748,0.000000}%
\pgfsetfillcolor{currentfill}%
\pgfsetlinewidth{0.000000pt}%
\definecolor{currentstroke}{rgb}{0.000000,0.000000,0.000000}%
\pgfsetstrokecolor{currentstroke}%
\pgfsetdash{}{0pt}%
\pgfpathmoveto{\pgfqpoint{1.251361in}{2.005009in}}%
\pgfpathlineto{\pgfqpoint{1.242364in}{2.028988in}}%
\pgfpathlineto{\pgfqpoint{1.173182in}{2.004134in}}%
\pgfpathlineto{\pgfqpoint{1.184296in}{1.980841in}}%
\pgfpathlineto{\pgfqpoint{1.251361in}{2.005009in}}%
\pgfpathclose%
\pgfusepath{fill}%
\end{pgfscope}%
\begin{pgfscope}%
\pgfpathrectangle{\pgfqpoint{0.000000in}{0.000000in}}{\pgfqpoint{3.000000in}{3.000000in}}%
\pgfusepath{clip}%
\pgfsetbuttcap%
\pgfsetroundjoin%
\definecolor{currentfill}{rgb}{0.743201,1.000000,0.224541}%
\pgfsetfillcolor{currentfill}%
\pgfsetlinewidth{0.000000pt}%
\definecolor{currentstroke}{rgb}{0.000000,0.000000,0.000000}%
\pgfsetstrokecolor{currentstroke}%
\pgfsetdash{}{0pt}%
\pgfpathmoveto{\pgfqpoint{1.323597in}{1.789998in}}%
\pgfpathlineto{\pgfqpoint{1.314543in}{1.819753in}}%
\pgfpathlineto{\pgfqpoint{1.262296in}{1.800484in}}%
\pgfpathlineto{\pgfqpoint{1.273466in}{1.771442in}}%
\pgfpathlineto{\pgfqpoint{1.323597in}{1.789998in}}%
\pgfpathclose%
\pgfusepath{fill}%
\end{pgfscope}%
\begin{pgfscope}%
\pgfpathrectangle{\pgfqpoint{0.000000in}{0.000000in}}{\pgfqpoint{3.000000in}{3.000000in}}%
\pgfusepath{clip}%
\pgfsetbuttcap%
\pgfsetroundjoin%
\definecolor{currentfill}{rgb}{0.300443,1.000000,0.667299}%
\pgfsetfillcolor{currentfill}%
\pgfsetlinewidth{0.000000pt}%
\definecolor{currentstroke}{rgb}{0.000000,0.000000,0.000000}%
\pgfsetstrokecolor{currentstroke}%
\pgfsetdash{}{0pt}%
\pgfpathmoveto{\pgfqpoint{1.775310in}{1.594850in}}%
\pgfpathlineto{\pgfqpoint{1.787784in}{1.630526in}}%
\pgfpathlineto{\pgfqpoint{1.749513in}{1.648287in}}%
\pgfpathlineto{\pgfqpoint{1.738989in}{1.611782in}}%
\pgfpathlineto{\pgfqpoint{1.775310in}{1.594850in}}%
\pgfpathclose%
\pgfusepath{fill}%
\end{pgfscope}%
\begin{pgfscope}%
\pgfpathrectangle{\pgfqpoint{0.000000in}{0.000000in}}{\pgfqpoint{3.000000in}{3.000000in}}%
\pgfusepath{clip}%
\pgfsetbuttcap%
\pgfsetroundjoin%
\definecolor{currentfill}{rgb}{1.000000,0.668845,0.000000}%
\pgfsetfillcolor{currentfill}%
\pgfsetlinewidth{0.000000pt}%
\definecolor{currentstroke}{rgb}{0.000000,0.000000,0.000000}%
\pgfsetstrokecolor{currentstroke}%
\pgfsetdash{}{0pt}%
\pgfpathmoveto{\pgfqpoint{1.260365in}{1.980513in}}%
\pgfpathlineto{\pgfqpoint{1.251361in}{2.005009in}}%
\pgfpathlineto{\pgfqpoint{1.184296in}{1.980841in}}%
\pgfpathlineto{\pgfqpoint{1.195417in}{1.957035in}}%
\pgfpathlineto{\pgfqpoint{1.260365in}{1.980513in}}%
\pgfpathclose%
\pgfusepath{fill}%
\end{pgfscope}%
\begin{pgfscope}%
\pgfpathrectangle{\pgfqpoint{0.000000in}{0.000000in}}{\pgfqpoint{3.000000in}{3.000000in}}%
\pgfusepath{clip}%
\pgfsetbuttcap%
\pgfsetroundjoin%
\definecolor{currentfill}{rgb}{0.606952,0.000000,0.000000}%
\pgfsetfillcolor{currentfill}%
\pgfsetlinewidth{0.000000pt}%
\definecolor{currentstroke}{rgb}{0.000000,0.000000,0.000000}%
\pgfsetstrokecolor{currentstroke}%
\pgfsetdash{}{0pt}%
\pgfpathmoveto{\pgfqpoint{2.122252in}{2.265770in}}%
\pgfpathlineto{\pgfqpoint{2.134543in}{2.283919in}}%
\pgfpathlineto{\pgfqpoint{2.041254in}{2.323477in}}%
\pgfpathlineto{\pgfqpoint{2.030939in}{2.304595in}}%
\pgfpathlineto{\pgfqpoint{2.122252in}{2.265770in}}%
\pgfpathclose%
\pgfusepath{fill}%
\end{pgfscope}%
\begin{pgfscope}%
\pgfpathrectangle{\pgfqpoint{0.000000in}{0.000000in}}{\pgfqpoint{3.000000in}{3.000000in}}%
\pgfusepath{clip}%
\pgfsetbuttcap%
\pgfsetroundjoin%
\definecolor{currentfill}{rgb}{0.000000,0.849020,1.000000}%
\pgfsetfillcolor{currentfill}%
\pgfsetlinewidth{0.000000pt}%
\definecolor{currentstroke}{rgb}{0.000000,0.000000,0.000000}%
\pgfsetstrokecolor{currentstroke}%
\pgfsetdash{}{0pt}%
\pgfpathmoveto{\pgfqpoint{1.763613in}{1.456808in}}%
\pgfpathlineto{\pgfqpoint{1.777760in}{1.499087in}}%
\pgfpathlineto{\pgfqpoint{1.750346in}{1.516738in}}%
\pgfpathlineto{\pgfqpoint{1.737856in}{1.473483in}}%
\pgfpathlineto{\pgfqpoint{1.763613in}{1.456808in}}%
\pgfpathclose%
\pgfusepath{fill}%
\end{pgfscope}%
\begin{pgfscope}%
\pgfpathrectangle{\pgfqpoint{0.000000in}{0.000000in}}{\pgfqpoint{3.000000in}{3.000000in}}%
\pgfusepath{clip}%
\pgfsetbuttcap%
\pgfsetroundjoin%
\definecolor{currentfill}{rgb}{0.819102,1.000000,0.148640}%
\pgfsetfillcolor{currentfill}%
\pgfsetlinewidth{0.000000pt}%
\definecolor{currentstroke}{rgb}{0.000000,0.000000,0.000000}%
\pgfsetstrokecolor{currentstroke}%
\pgfsetdash{}{0pt}%
\pgfpathmoveto{\pgfqpoint{1.314543in}{1.819753in}}%
\pgfpathlineto{\pgfqpoint{1.305495in}{1.848517in}}%
\pgfpathlineto{\pgfqpoint{1.251132in}{1.828538in}}%
\pgfpathlineto{\pgfqpoint{1.262296in}{1.800484in}}%
\pgfpathlineto{\pgfqpoint{1.314543in}{1.819753in}}%
\pgfpathclose%
\pgfusepath{fill}%
\end{pgfscope}%
\begin{pgfscope}%
\pgfpathrectangle{\pgfqpoint{0.000000in}{0.000000in}}{\pgfqpoint{3.000000in}{3.000000in}}%
\pgfusepath{clip}%
\pgfsetbuttcap%
\pgfsetroundjoin%
\definecolor{currentfill}{rgb}{1.000000,0.741467,0.000000}%
\pgfsetfillcolor{currentfill}%
\pgfsetlinewidth{0.000000pt}%
\definecolor{currentstroke}{rgb}{0.000000,0.000000,0.000000}%
\pgfsetstrokecolor{currentstroke}%
\pgfsetdash{}{0pt}%
\pgfpathmoveto{\pgfqpoint{1.269376in}{1.955456in}}%
\pgfpathlineto{\pgfqpoint{1.260365in}{1.980513in}}%
\pgfpathlineto{\pgfqpoint{1.195417in}{1.957035in}}%
\pgfpathlineto{\pgfqpoint{1.206546in}{1.932672in}}%
\pgfpathlineto{\pgfqpoint{1.269376in}{1.955456in}}%
\pgfpathclose%
\pgfusepath{fill}%
\end{pgfscope}%
\begin{pgfscope}%
\pgfpathrectangle{\pgfqpoint{0.000000in}{0.000000in}}{\pgfqpoint{3.000000in}{3.000000in}}%
\pgfusepath{clip}%
\pgfsetbuttcap%
\pgfsetroundjoin%
\definecolor{currentfill}{rgb}{0.000000,0.000000,0.500000}%
\pgfsetfillcolor{currentfill}%
\pgfsetlinewidth{0.000000pt}%
\definecolor{currentstroke}{rgb}{0.000000,0.000000,0.000000}%
\pgfsetstrokecolor{currentstroke}%
\pgfsetdash{}{0pt}%
\pgfpathmoveto{\pgfqpoint{1.695540in}{1.012581in}}%
\pgfpathlineto{\pgfqpoint{1.711574in}{1.086974in}}%
\pgfpathlineto{\pgfqpoint{1.719476in}{1.101402in}}%
\pgfpathlineto{\pgfqpoint{1.702723in}{1.025758in}}%
\pgfpathlineto{\pgfqpoint{1.695540in}{1.012581in}}%
\pgfpathclose%
\pgfusepath{fill}%
\end{pgfscope}%
\begin{pgfscope}%
\pgfpathrectangle{\pgfqpoint{0.000000in}{0.000000in}}{\pgfqpoint{3.000000in}{3.000000in}}%
\pgfusepath{clip}%
\pgfsetbuttcap%
\pgfsetroundjoin%
\definecolor{currentfill}{rgb}{0.895003,1.000000,0.072739}%
\pgfsetfillcolor{currentfill}%
\pgfsetlinewidth{0.000000pt}%
\definecolor{currentstroke}{rgb}{0.000000,0.000000,0.000000}%
\pgfsetstrokecolor{currentstroke}%
\pgfsetdash{}{0pt}%
\pgfpathmoveto{\pgfqpoint{1.305495in}{1.848517in}}%
\pgfpathlineto{\pgfqpoint{1.296455in}{1.876389in}}%
\pgfpathlineto{\pgfqpoint{1.239975in}{1.855704in}}%
\pgfpathlineto{\pgfqpoint{1.251132in}{1.828538in}}%
\pgfpathlineto{\pgfqpoint{1.305495in}{1.848517in}}%
\pgfpathclose%
\pgfusepath{fill}%
\end{pgfscope}%
\begin{pgfscope}%
\pgfpathrectangle{\pgfqpoint{0.000000in}{0.000000in}}{\pgfqpoint{3.000000in}{3.000000in}}%
\pgfusepath{clip}%
\pgfsetbuttcap%
\pgfsetroundjoin%
\definecolor{currentfill}{rgb}{1.000000,0.814089,0.000000}%
\pgfsetfillcolor{currentfill}%
\pgfsetlinewidth{0.000000pt}%
\definecolor{currentstroke}{rgb}{0.000000,0.000000,0.000000}%
\pgfsetstrokecolor{currentstroke}%
\pgfsetdash{}{0pt}%
\pgfpathmoveto{\pgfqpoint{1.278395in}{1.929789in}}%
\pgfpathlineto{\pgfqpoint{1.269376in}{1.955456in}}%
\pgfpathlineto{\pgfqpoint{1.206546in}{1.932672in}}%
\pgfpathlineto{\pgfqpoint{1.217682in}{1.907701in}}%
\pgfpathlineto{\pgfqpoint{1.278395in}{1.929789in}}%
\pgfpathclose%
\pgfusepath{fill}%
\end{pgfscope}%
\begin{pgfscope}%
\pgfpathrectangle{\pgfqpoint{0.000000in}{0.000000in}}{\pgfqpoint{3.000000in}{3.000000in}}%
\pgfusepath{clip}%
\pgfsetbuttcap%
\pgfsetroundjoin%
\definecolor{currentfill}{rgb}{0.958254,0.973856,0.009488}%
\pgfsetfillcolor{currentfill}%
\pgfsetlinewidth{0.000000pt}%
\definecolor{currentstroke}{rgb}{0.000000,0.000000,0.000000}%
\pgfsetstrokecolor{currentstroke}%
\pgfsetdash{}{0pt}%
\pgfpathmoveto{\pgfqpoint{1.296455in}{1.876389in}}%
\pgfpathlineto{\pgfqpoint{1.287421in}{1.903455in}}%
\pgfpathlineto{\pgfqpoint{1.228825in}{1.882067in}}%
\pgfpathlineto{\pgfqpoint{1.239975in}{1.855704in}}%
\pgfpathlineto{\pgfqpoint{1.296455in}{1.876389in}}%
\pgfpathclose%
\pgfusepath{fill}%
\end{pgfscope}%
\begin{pgfscope}%
\pgfpathrectangle{\pgfqpoint{0.000000in}{0.000000in}}{\pgfqpoint{3.000000in}{3.000000in}}%
\pgfusepath{clip}%
\pgfsetbuttcap%
\pgfsetroundjoin%
\definecolor{currentfill}{rgb}{1.000000,0.886710,0.000000}%
\pgfsetfillcolor{currentfill}%
\pgfsetlinewidth{0.000000pt}%
\definecolor{currentstroke}{rgb}{0.000000,0.000000,0.000000}%
\pgfsetstrokecolor{currentstroke}%
\pgfsetdash{}{0pt}%
\pgfpathmoveto{\pgfqpoint{1.287421in}{1.903455in}}%
\pgfpathlineto{\pgfqpoint{1.278395in}{1.929789in}}%
\pgfpathlineto{\pgfqpoint{1.217682in}{1.907701in}}%
\pgfpathlineto{\pgfqpoint{1.228825in}{1.882067in}}%
\pgfpathlineto{\pgfqpoint{1.287421in}{1.903455in}}%
\pgfpathclose%
\pgfusepath{fill}%
\end{pgfscope}%
\begin{pgfscope}%
\pgfpathrectangle{\pgfqpoint{0.000000in}{0.000000in}}{\pgfqpoint{3.000000in}{3.000000in}}%
\pgfusepath{clip}%
\pgfsetbuttcap%
\pgfsetroundjoin%
\definecolor{currentfill}{rgb}{0.199241,1.000000,0.768501}%
\pgfsetfillcolor{currentfill}%
\pgfsetlinewidth{0.000000pt}%
\definecolor{currentstroke}{rgb}{0.000000,0.000000,0.000000}%
\pgfsetstrokecolor{currentstroke}%
\pgfsetdash{}{0pt}%
\pgfpathmoveto{\pgfqpoint{1.340626in}{1.568079in}}%
\pgfpathlineto{\pgfqpoint{1.329417in}{1.606459in}}%
\pgfpathlineto{\pgfqpoint{1.294865in}{1.588594in}}%
\pgfpathlineto{\pgfqpoint{1.307932in}{1.551094in}}%
\pgfpathlineto{\pgfqpoint{1.340626in}{1.568079in}}%
\pgfpathclose%
\pgfusepath{fill}%
\end{pgfscope}%
\begin{pgfscope}%
\pgfpathrectangle{\pgfqpoint{0.000000in}{0.000000in}}{\pgfqpoint{3.000000in}{3.000000in}}%
\pgfusepath{clip}%
\pgfsetbuttcap%
\pgfsetroundjoin%
\definecolor{currentfill}{rgb}{0.678253,0.000000,0.000000}%
\pgfsetfillcolor{currentfill}%
\pgfsetlinewidth{0.000000pt}%
\definecolor{currentstroke}{rgb}{0.000000,0.000000,0.000000}%
\pgfsetstrokecolor{currentstroke}%
\pgfsetdash{}{0pt}%
\pgfpathmoveto{\pgfqpoint{2.109953in}{2.247424in}}%
\pgfpathlineto{\pgfqpoint{2.122252in}{2.265770in}}%
\pgfpathlineto{\pgfqpoint{2.030939in}{2.304595in}}%
\pgfpathlineto{\pgfqpoint{2.020616in}{2.285513in}}%
\pgfpathlineto{\pgfqpoint{2.109953in}{2.247424in}}%
\pgfpathclose%
\pgfusepath{fill}%
\end{pgfscope}%
\begin{pgfscope}%
\pgfpathrectangle{\pgfqpoint{0.000000in}{0.000000in}}{\pgfqpoint{3.000000in}{3.000000in}}%
\pgfusepath{clip}%
\pgfsetbuttcap%
\pgfsetroundjoin%
\definecolor{currentfill}{rgb}{0.000000,0.300000,1.000000}%
\pgfsetfillcolor{currentfill}%
\pgfsetlinewidth{0.000000pt}%
\definecolor{currentstroke}{rgb}{0.000000,0.000000,0.000000}%
\pgfsetstrokecolor{currentstroke}%
\pgfsetdash{}{0pt}%
\pgfpathmoveto{\pgfqpoint{1.749251in}{1.273869in}}%
\pgfpathlineto{\pgfqpoint{1.765664in}{1.326959in}}%
\pgfpathlineto{\pgfqpoint{1.753065in}{1.344557in}}%
\pgfpathlineto{\pgfqpoint{1.737596in}{1.290277in}}%
\pgfpathlineto{\pgfqpoint{1.749251in}{1.273869in}}%
\pgfpathclose%
\pgfusepath{fill}%
\end{pgfscope}%
\begin{pgfscope}%
\pgfpathrectangle{\pgfqpoint{0.000000in}{0.000000in}}{\pgfqpoint{3.000000in}{3.000000in}}%
\pgfusepath{clip}%
\pgfsetbuttcap%
\pgfsetroundjoin%
\definecolor{currentfill}{rgb}{0.000000,0.064706,1.000000}%
\pgfsetfillcolor{currentfill}%
\pgfsetlinewidth{0.000000pt}%
\definecolor{currentstroke}{rgb}{0.000000,0.000000,0.000000}%
\pgfsetstrokecolor{currentstroke}%
\pgfsetdash{}{0pt}%
\pgfpathmoveto{\pgfqpoint{1.738860in}{1.198223in}}%
\pgfpathlineto{\pgfqpoint{1.755813in}{1.256695in}}%
\pgfpathlineto{\pgfqpoint{1.749251in}{1.273869in}}%
\pgfpathlineto{\pgfqpoint{1.732842in}{1.214143in}}%
\pgfpathlineto{\pgfqpoint{1.738860in}{1.198223in}}%
\pgfpathclose%
\pgfusepath{fill}%
\end{pgfscope}%
\begin{pgfscope}%
\pgfpathrectangle{\pgfqpoint{0.000000in}{0.000000in}}{\pgfqpoint{3.000000in}{3.000000in}}%
\pgfusepath{clip}%
\pgfsetbuttcap%
\pgfsetroundjoin%
\definecolor{currentfill}{rgb}{0.000000,0.000000,0.838681}%
\pgfsetfillcolor{currentfill}%
\pgfsetlinewidth{0.000000pt}%
\definecolor{currentstroke}{rgb}{0.000000,0.000000,0.000000}%
\pgfsetstrokecolor{currentstroke}%
\pgfsetdash{}{0pt}%
\pgfpathmoveto{\pgfqpoint{1.358322in}{1.126259in}}%
\pgfpathlineto{\pgfqpoint{1.341291in}{1.192826in}}%
\pgfpathlineto{\pgfqpoint{1.341778in}{1.176568in}}%
\pgfpathlineto{\pgfqpoint{1.358794in}{1.111292in}}%
\pgfpathlineto{\pgfqpoint{1.358322in}{1.126259in}}%
\pgfpathclose%
\pgfusepath{fill}%
\end{pgfscope}%
\begin{pgfscope}%
\pgfpathrectangle{\pgfqpoint{0.000000in}{0.000000in}}{\pgfqpoint{3.000000in}{3.000000in}}%
\pgfusepath{clip}%
\pgfsetbuttcap%
\pgfsetroundjoin%
\definecolor{currentfill}{rgb}{0.731729,0.000000,0.000000}%
\pgfsetfillcolor{currentfill}%
\pgfsetlinewidth{0.000000pt}%
\definecolor{currentstroke}{rgb}{0.000000,0.000000,0.000000}%
\pgfsetstrokecolor{currentstroke}%
\pgfsetdash{}{0pt}%
\pgfpathmoveto{\pgfqpoint{2.097647in}{2.228871in}}%
\pgfpathlineto{\pgfqpoint{2.109953in}{2.247424in}}%
\pgfpathlineto{\pgfqpoint{2.020616in}{2.285513in}}%
\pgfpathlineto{\pgfqpoint{2.010286in}{2.266220in}}%
\pgfpathlineto{\pgfqpoint{2.097647in}{2.228871in}}%
\pgfpathclose%
\pgfusepath{fill}%
\end{pgfscope}%
\begin{pgfscope}%
\pgfpathrectangle{\pgfqpoint{0.000000in}{0.000000in}}{\pgfqpoint{3.000000in}{3.000000in}}%
\pgfusepath{clip}%
\pgfsetbuttcap%
\pgfsetroundjoin%
\definecolor{currentfill}{rgb}{0.401645,1.000000,0.566097}%
\pgfsetfillcolor{currentfill}%
\pgfsetlinewidth{0.000000pt}%
\definecolor{currentstroke}{rgb}{0.000000,0.000000,0.000000}%
\pgfsetstrokecolor{currentstroke}%
\pgfsetdash{}{0pt}%
\pgfpathmoveto{\pgfqpoint{1.787784in}{1.630526in}}%
\pgfpathlineto{\pgfqpoint{1.800252in}{1.664361in}}%
\pgfpathlineto{\pgfqpoint{1.760031in}{1.682949in}}%
\pgfpathlineto{\pgfqpoint{1.749513in}{1.648287in}}%
\pgfpathlineto{\pgfqpoint{1.787784in}{1.630526in}}%
\pgfpathclose%
\pgfusepath{fill}%
\end{pgfscope}%
\begin{pgfscope}%
\pgfpathrectangle{\pgfqpoint{0.000000in}{0.000000in}}{\pgfqpoint{3.000000in}{3.000000in}}%
\pgfusepath{clip}%
\pgfsetbuttcap%
\pgfsetroundjoin%
\definecolor{currentfill}{rgb}{0.000000,0.503922,1.000000}%
\pgfsetfillcolor{currentfill}%
\pgfsetlinewidth{0.000000pt}%
\definecolor{currentstroke}{rgb}{0.000000,0.000000,0.000000}%
\pgfsetstrokecolor{currentstroke}%
\pgfsetdash{}{0pt}%
\pgfpathmoveto{\pgfqpoint{1.339304in}{1.355619in}}%
\pgfpathlineto{\pgfqpoint{1.324674in}{1.405359in}}%
\pgfpathlineto{\pgfqpoint{1.307396in}{1.387419in}}%
\pgfpathlineto{\pgfqpoint{1.323224in}{1.338809in}}%
\pgfpathlineto{\pgfqpoint{1.339304in}{1.355619in}}%
\pgfpathclose%
\pgfusepath{fill}%
\end{pgfscope}%
\begin{pgfscope}%
\pgfpathrectangle{\pgfqpoint{0.000000in}{0.000000in}}{\pgfqpoint{3.000000in}{3.000000in}}%
\pgfusepath{clip}%
\pgfsetbuttcap%
\pgfsetroundjoin%
\definecolor{currentfill}{rgb}{0.803030,0.000000,0.000000}%
\pgfsetfillcolor{currentfill}%
\pgfsetlinewidth{0.000000pt}%
\definecolor{currentstroke}{rgb}{0.000000,0.000000,0.000000}%
\pgfsetstrokecolor{currentstroke}%
\pgfsetdash{}{0pt}%
\pgfpathmoveto{\pgfqpoint{2.085333in}{2.210099in}}%
\pgfpathlineto{\pgfqpoint{2.097647in}{2.228871in}}%
\pgfpathlineto{\pgfqpoint{2.010286in}{2.266220in}}%
\pgfpathlineto{\pgfqpoint{1.999947in}{2.246705in}}%
\pgfpathlineto{\pgfqpoint{2.085333in}{2.210099in}}%
\pgfpathclose%
\pgfusepath{fill}%
\end{pgfscope}%
\begin{pgfscope}%
\pgfpathrectangle{\pgfqpoint{0.000000in}{0.000000in}}{\pgfqpoint{3.000000in}{3.000000in}}%
\pgfusepath{clip}%
\pgfsetbuttcap%
\pgfsetroundjoin%
\definecolor{currentfill}{rgb}{0.000000,0.000000,0.500000}%
\pgfsetfillcolor{currentfill}%
\pgfsetlinewidth{0.000000pt}%
\definecolor{currentstroke}{rgb}{0.000000,0.000000,0.000000}%
\pgfsetstrokecolor{currentstroke}%
\pgfsetdash{}{0pt}%
\pgfpathmoveto{\pgfqpoint{1.380312in}{1.021303in}}%
\pgfpathlineto{\pgfqpoint{1.363753in}{1.096525in}}%
\pgfpathlineto{\pgfqpoint{1.373098in}{1.082326in}}%
\pgfpathlineto{\pgfqpoint{1.388802in}{1.008337in}}%
\pgfpathlineto{\pgfqpoint{1.380312in}{1.021303in}}%
\pgfpathclose%
\pgfusepath{fill}%
\end{pgfscope}%
\begin{pgfscope}%
\pgfpathrectangle{\pgfqpoint{0.000000in}{0.000000in}}{\pgfqpoint{3.000000in}{3.000000in}}%
\pgfusepath{clip}%
\pgfsetbuttcap%
\pgfsetroundjoin%
\definecolor{currentfill}{rgb}{0.300443,1.000000,0.667299}%
\pgfsetfillcolor{currentfill}%
\pgfsetlinewidth{0.000000pt}%
\definecolor{currentstroke}{rgb}{0.000000,0.000000,0.000000}%
\pgfsetstrokecolor{currentstroke}%
\pgfsetdash{}{0pt}%
\pgfpathmoveto{\pgfqpoint{1.329417in}{1.606459in}}%
\pgfpathlineto{\pgfqpoint{1.318214in}{1.642704in}}%
\pgfpathlineto{\pgfqpoint{1.281802in}{1.623962in}}%
\pgfpathlineto{\pgfqpoint{1.294865in}{1.588594in}}%
\pgfpathlineto{\pgfqpoint{1.329417in}{1.606459in}}%
\pgfpathclose%
\pgfusepath{fill}%
\end{pgfscope}%
\begin{pgfscope}%
\pgfpathrectangle{\pgfqpoint{0.000000in}{0.000000in}}{\pgfqpoint{3.000000in}{3.000000in}}%
\pgfusepath{clip}%
\pgfsetbuttcap%
\pgfsetroundjoin%
\definecolor{currentfill}{rgb}{0.000000,0.849020,1.000000}%
\pgfsetfillcolor{currentfill}%
\pgfsetlinewidth{0.000000pt}%
\definecolor{currentstroke}{rgb}{0.000000,0.000000,0.000000}%
\pgfsetstrokecolor{currentstroke}%
\pgfsetdash{}{0pt}%
\pgfpathmoveto{\pgfqpoint{1.334081in}{1.468157in}}%
\pgfpathlineto{\pgfqpoint{1.321004in}{1.511101in}}%
\pgfpathlineto{\pgfqpoint{1.295418in}{1.492742in}}%
\pgfpathlineto{\pgfqpoint{1.310046in}{1.450815in}}%
\pgfpathlineto{\pgfqpoint{1.334081in}{1.468157in}}%
\pgfpathclose%
\pgfusepath{fill}%
\end{pgfscope}%
\begin{pgfscope}%
\pgfpathrectangle{\pgfqpoint{0.000000in}{0.000000in}}{\pgfqpoint{3.000000in}{3.000000in}}%
\pgfusepath{clip}%
\pgfsetbuttcap%
\pgfsetroundjoin%
\definecolor{currentfill}{rgb}{0.856506,0.000000,0.000000}%
\pgfsetfillcolor{currentfill}%
\pgfsetlinewidth{0.000000pt}%
\definecolor{currentstroke}{rgb}{0.000000,0.000000,0.000000}%
\pgfsetstrokecolor{currentstroke}%
\pgfsetdash{}{0pt}%
\pgfpathmoveto{\pgfqpoint{2.073013in}{2.191096in}}%
\pgfpathlineto{\pgfqpoint{2.085333in}{2.210099in}}%
\pgfpathlineto{\pgfqpoint{1.999947in}{2.246705in}}%
\pgfpathlineto{\pgfqpoint{1.989600in}{2.226956in}}%
\pgfpathlineto{\pgfqpoint{2.073013in}{2.191096in}}%
\pgfpathclose%
\pgfusepath{fill}%
\end{pgfscope}%
\begin{pgfscope}%
\pgfpathrectangle{\pgfqpoint{0.000000in}{0.000000in}}{\pgfqpoint{3.000000in}{3.000000in}}%
\pgfusepath{clip}%
\pgfsetbuttcap%
\pgfsetroundjoin%
\definecolor{currentfill}{rgb}{0.490196,1.000000,0.477546}%
\pgfsetfillcolor{currentfill}%
\pgfsetlinewidth{0.000000pt}%
\definecolor{currentstroke}{rgb}{0.000000,0.000000,0.000000}%
\pgfsetstrokecolor{currentstroke}%
\pgfsetdash{}{0pt}%
\pgfpathmoveto{\pgfqpoint{1.800252in}{1.664361in}}%
\pgfpathlineto{\pgfqpoint{1.812716in}{1.696595in}}%
\pgfpathlineto{\pgfqpoint{1.770542in}{1.716005in}}%
\pgfpathlineto{\pgfqpoint{1.760031in}{1.682949in}}%
\pgfpathlineto{\pgfqpoint{1.800252in}{1.664361in}}%
\pgfpathclose%
\pgfusepath{fill}%
\end{pgfscope}%
\begin{pgfscope}%
\pgfpathrectangle{\pgfqpoint{0.000000in}{0.000000in}}{\pgfqpoint{3.000000in}{3.000000in}}%
\pgfusepath{clip}%
\pgfsetbuttcap%
\pgfsetroundjoin%
\definecolor{currentfill}{rgb}{0.085389,1.000000,0.882353}%
\pgfsetfillcolor{currentfill}%
\pgfsetlinewidth{0.000000pt}%
\definecolor{currentstroke}{rgb}{0.000000,0.000000,0.000000}%
\pgfsetstrokecolor{currentstroke}%
\pgfsetdash{}{0pt}%
\pgfpathmoveto{\pgfqpoint{1.777760in}{1.499087in}}%
\pgfpathlineto{\pgfqpoint{1.791904in}{1.538419in}}%
\pgfpathlineto{\pgfqpoint{1.762830in}{1.557042in}}%
\pgfpathlineto{\pgfqpoint{1.750346in}{1.516738in}}%
\pgfpathlineto{\pgfqpoint{1.777760in}{1.499087in}}%
\pgfpathclose%
\pgfusepath{fill}%
\end{pgfscope}%
\begin{pgfscope}%
\pgfpathrectangle{\pgfqpoint{0.000000in}{0.000000in}}{\pgfqpoint{3.000000in}{3.000000in}}%
\pgfusepath{clip}%
\pgfsetbuttcap%
\pgfsetroundjoin%
\definecolor{currentfill}{rgb}{0.000000,0.300000,1.000000}%
\pgfsetfillcolor{currentfill}%
\pgfsetlinewidth{0.000000pt}%
\definecolor{currentstroke}{rgb}{0.000000,0.000000,0.000000}%
\pgfsetstrokecolor{currentstroke}%
\pgfsetdash{}{0pt}%
\pgfpathmoveto{\pgfqpoint{1.339051in}{1.284918in}}%
\pgfpathlineto{\pgfqpoint{1.323224in}{1.338809in}}%
\pgfpathlineto{\pgfqpoint{1.312424in}{1.320887in}}%
\pgfpathlineto{\pgfqpoint{1.329065in}{1.268209in}}%
\pgfpathlineto{\pgfqpoint{1.339051in}{1.284918in}}%
\pgfpathclose%
\pgfusepath{fill}%
\end{pgfscope}%
\begin{pgfscope}%
\pgfpathrectangle{\pgfqpoint{0.000000in}{0.000000in}}{\pgfqpoint{3.000000in}{3.000000in}}%
\pgfusepath{clip}%
\pgfsetbuttcap%
\pgfsetroundjoin%
\definecolor{currentfill}{rgb}{0.000000,0.064706,1.000000}%
\pgfsetfillcolor{currentfill}%
\pgfsetlinewidth{0.000000pt}%
\definecolor{currentstroke}{rgb}{0.000000,0.000000,0.000000}%
\pgfsetstrokecolor{currentstroke}%
\pgfsetdash{}{0pt}%
\pgfpathmoveto{\pgfqpoint{1.345701in}{1.208896in}}%
\pgfpathlineto{\pgfqpoint{1.329065in}{1.268209in}}%
\pgfpathlineto{\pgfqpoint{1.324249in}{1.250871in}}%
\pgfpathlineto{\pgfqpoint{1.341291in}{1.192826in}}%
\pgfpathlineto{\pgfqpoint{1.345701in}{1.208896in}}%
\pgfpathclose%
\pgfusepath{fill}%
\end{pgfscope}%
\begin{pgfscope}%
\pgfpathrectangle{\pgfqpoint{0.000000in}{0.000000in}}{\pgfqpoint{3.000000in}{3.000000in}}%
\pgfusepath{clip}%
\pgfsetbuttcap%
\pgfsetroundjoin%
\definecolor{currentfill}{rgb}{0.927807,0.015251,0.000000}%
\pgfsetfillcolor{currentfill}%
\pgfsetlinewidth{0.000000pt}%
\definecolor{currentstroke}{rgb}{0.000000,0.000000,0.000000}%
\pgfsetstrokecolor{currentstroke}%
\pgfsetdash{}{0pt}%
\pgfpathmoveto{\pgfqpoint{2.060685in}{2.171850in}}%
\pgfpathlineto{\pgfqpoint{2.073013in}{2.191096in}}%
\pgfpathlineto{\pgfqpoint{1.989600in}{2.226956in}}%
\pgfpathlineto{\pgfqpoint{1.979245in}{2.206961in}}%
\pgfpathlineto{\pgfqpoint{2.060685in}{2.171850in}}%
\pgfpathclose%
\pgfusepath{fill}%
\end{pgfscope}%
\begin{pgfscope}%
\pgfpathrectangle{\pgfqpoint{0.000000in}{0.000000in}}{\pgfqpoint{3.000000in}{3.000000in}}%
\pgfusepath{clip}%
\pgfsetbuttcap%
\pgfsetroundjoin%
\definecolor{currentfill}{rgb}{0.000000,0.676471,1.000000}%
\pgfsetfillcolor{currentfill}%
\pgfsetlinewidth{0.000000pt}%
\definecolor{currentstroke}{rgb}{0.000000,0.000000,0.000000}%
\pgfsetstrokecolor{currentstroke}%
\pgfsetdash{}{0pt}%
\pgfpathmoveto{\pgfqpoint{1.768535in}{1.393553in}}%
\pgfpathlineto{\pgfqpoint{1.784006in}{1.438268in}}%
\pgfpathlineto{\pgfqpoint{1.763613in}{1.456808in}}%
\pgfpathlineto{\pgfqpoint{1.749464in}{1.411000in}}%
\pgfpathlineto{\pgfqpoint{1.768535in}{1.393553in}}%
\pgfpathclose%
\pgfusepath{fill}%
\end{pgfscope}%
\begin{pgfscope}%
\pgfpathrectangle{\pgfqpoint{0.000000in}{0.000000in}}{\pgfqpoint{3.000000in}{3.000000in}}%
\pgfusepath{clip}%
\pgfsetbuttcap%
\pgfsetroundjoin%
\definecolor{currentfill}{rgb}{0.578748,1.000000,0.388994}%
\pgfsetfillcolor{currentfill}%
\pgfsetlinewidth{0.000000pt}%
\definecolor{currentstroke}{rgb}{0.000000,0.000000,0.000000}%
\pgfsetstrokecolor{currentstroke}%
\pgfsetdash{}{0pt}%
\pgfpathmoveto{\pgfqpoint{1.812716in}{1.696595in}}%
\pgfpathlineto{\pgfqpoint{1.825173in}{1.727424in}}%
\pgfpathlineto{\pgfqpoint{1.781046in}{1.747652in}}%
\pgfpathlineto{\pgfqpoint{1.770542in}{1.716005in}}%
\pgfpathlineto{\pgfqpoint{1.812716in}{1.696595in}}%
\pgfpathclose%
\pgfusepath{fill}%
\end{pgfscope}%
\begin{pgfscope}%
\pgfpathrectangle{\pgfqpoint{0.000000in}{0.000000in}}{\pgfqpoint{3.000000in}{3.000000in}}%
\pgfusepath{clip}%
\pgfsetbuttcap%
\pgfsetroundjoin%
\definecolor{currentfill}{rgb}{0.999109,0.073348,0.000000}%
\pgfsetfillcolor{currentfill}%
\pgfsetlinewidth{0.000000pt}%
\definecolor{currentstroke}{rgb}{0.000000,0.000000,0.000000}%
\pgfsetstrokecolor{currentstroke}%
\pgfsetdash{}{0pt}%
\pgfpathmoveto{\pgfqpoint{2.048350in}{2.152346in}}%
\pgfpathlineto{\pgfqpoint{2.060685in}{2.171850in}}%
\pgfpathlineto{\pgfqpoint{1.979245in}{2.206961in}}%
\pgfpathlineto{\pgfqpoint{1.968883in}{2.186703in}}%
\pgfpathlineto{\pgfqpoint{2.048350in}{2.152346in}}%
\pgfpathclose%
\pgfusepath{fill}%
\end{pgfscope}%
\begin{pgfscope}%
\pgfpathrectangle{\pgfqpoint{0.000000in}{0.000000in}}{\pgfqpoint{3.000000in}{3.000000in}}%
\pgfusepath{clip}%
\pgfsetbuttcap%
\pgfsetroundjoin%
\definecolor{currentfill}{rgb}{0.401645,1.000000,0.566097}%
\pgfsetfillcolor{currentfill}%
\pgfsetlinewidth{0.000000pt}%
\definecolor{currentstroke}{rgb}{0.000000,0.000000,0.000000}%
\pgfsetstrokecolor{currentstroke}%
\pgfsetdash{}{0pt}%
\pgfpathmoveto{\pgfqpoint{1.318214in}{1.642704in}}%
\pgfpathlineto{\pgfqpoint{1.307018in}{1.677106in}}%
\pgfpathlineto{\pgfqpoint{1.268744in}{1.657492in}}%
\pgfpathlineto{\pgfqpoint{1.281802in}{1.623962in}}%
\pgfpathlineto{\pgfqpoint{1.318214in}{1.642704in}}%
\pgfpathclose%
\pgfusepath{fill}%
\end{pgfscope}%
\begin{pgfscope}%
\pgfpathrectangle{\pgfqpoint{0.000000in}{0.000000in}}{\pgfqpoint{3.000000in}{3.000000in}}%
\pgfusepath{clip}%
\pgfsetbuttcap%
\pgfsetroundjoin%
\definecolor{currentfill}{rgb}{1.000000,0.116921,0.000000}%
\pgfsetfillcolor{currentfill}%
\pgfsetlinewidth{0.000000pt}%
\definecolor{currentstroke}{rgb}{0.000000,0.000000,0.000000}%
\pgfsetstrokecolor{currentstroke}%
\pgfsetdash{}{0pt}%
\pgfpathmoveto{\pgfqpoint{2.036008in}{2.132568in}}%
\pgfpathlineto{\pgfqpoint{2.048350in}{2.152346in}}%
\pgfpathlineto{\pgfqpoint{1.968883in}{2.186703in}}%
\pgfpathlineto{\pgfqpoint{1.958512in}{2.166169in}}%
\pgfpathlineto{\pgfqpoint{2.036008in}{2.132568in}}%
\pgfpathclose%
\pgfusepath{fill}%
\end{pgfscope}%
\begin{pgfscope}%
\pgfpathrectangle{\pgfqpoint{0.000000in}{0.000000in}}{\pgfqpoint{3.000000in}{3.000000in}}%
\pgfusepath{clip}%
\pgfsetbuttcap%
\pgfsetroundjoin%
\definecolor{currentfill}{rgb}{0.667299,1.000000,0.300443}%
\pgfsetfillcolor{currentfill}%
\pgfsetlinewidth{0.000000pt}%
\definecolor{currentstroke}{rgb}{0.000000,0.000000,0.000000}%
\pgfsetstrokecolor{currentstroke}%
\pgfsetdash{}{0pt}%
\pgfpathmoveto{\pgfqpoint{1.825173in}{1.727424in}}%
\pgfpathlineto{\pgfqpoint{1.837625in}{1.757013in}}%
\pgfpathlineto{\pgfqpoint{1.791543in}{1.778055in}}%
\pgfpathlineto{\pgfqpoint{1.781046in}{1.747652in}}%
\pgfpathlineto{\pgfqpoint{1.825173in}{1.727424in}}%
\pgfpathclose%
\pgfusepath{fill}%
\end{pgfscope}%
\begin{pgfscope}%
\pgfpathrectangle{\pgfqpoint{0.000000in}{0.000000in}}{\pgfqpoint{3.000000in}{3.000000in}}%
\pgfusepath{clip}%
\pgfsetbuttcap%
\pgfsetroundjoin%
\definecolor{currentfill}{rgb}{0.500000,0.000000,0.000000}%
\pgfsetfillcolor{currentfill}%
\pgfsetlinewidth{0.000000pt}%
\definecolor{currentstroke}{rgb}{0.000000,0.000000,0.000000}%
\pgfsetstrokecolor{currentstroke}%
\pgfsetdash{}{0pt}%
\pgfpathmoveto{\pgfqpoint{0.996392in}{2.329527in}}%
\pgfpathlineto{\pgfqpoint{0.985410in}{2.347810in}}%
\pgfpathlineto{\pgfqpoint{0.892529in}{2.304446in}}%
\pgfpathlineto{\pgfqpoint{0.905412in}{2.286935in}}%
\pgfpathlineto{\pgfqpoint{0.996392in}{2.329527in}}%
\pgfpathclose%
\pgfusepath{fill}%
\end{pgfscope}%
\begin{pgfscope}%
\pgfpathrectangle{\pgfqpoint{0.000000in}{0.000000in}}{\pgfqpoint{3.000000in}{3.000000in}}%
\pgfusepath{clip}%
\pgfsetbuttcap%
\pgfsetroundjoin%
\definecolor{currentfill}{rgb}{1.000000,0.175018,0.000000}%
\pgfsetfillcolor{currentfill}%
\pgfsetlinewidth{0.000000pt}%
\definecolor{currentstroke}{rgb}{0.000000,0.000000,0.000000}%
\pgfsetstrokecolor{currentstroke}%
\pgfsetdash{}{0pt}%
\pgfpathmoveto{\pgfqpoint{2.023659in}{2.112499in}}%
\pgfpathlineto{\pgfqpoint{2.036008in}{2.132568in}}%
\pgfpathlineto{\pgfqpoint{1.958512in}{2.166169in}}%
\pgfpathlineto{\pgfqpoint{1.948133in}{2.145341in}}%
\pgfpathlineto{\pgfqpoint{2.023659in}{2.112499in}}%
\pgfpathclose%
\pgfusepath{fill}%
\end{pgfscope}%
\begin{pgfscope}%
\pgfpathrectangle{\pgfqpoint{0.000000in}{0.000000in}}{\pgfqpoint{3.000000in}{3.000000in}}%
\pgfusepath{clip}%
\pgfsetbuttcap%
\pgfsetroundjoin%
\definecolor{currentfill}{rgb}{0.000000,0.000000,0.838681}%
\pgfsetfillcolor{currentfill}%
\pgfsetlinewidth{0.000000pt}%
\definecolor{currentstroke}{rgb}{0.000000,0.000000,0.000000}%
\pgfsetstrokecolor{currentstroke}%
\pgfsetdash{}{0pt}%
\pgfpathmoveto{\pgfqpoint{1.719476in}{1.101402in}}%
\pgfpathlineto{\pgfqpoint{1.736246in}{1.165824in}}%
\pgfpathlineto{\pgfqpoint{1.740013in}{1.181983in}}%
\pgfpathlineto{\pgfqpoint{1.722945in}{1.116277in}}%
\pgfpathlineto{\pgfqpoint{1.719476in}{1.101402in}}%
\pgfpathclose%
\pgfusepath{fill}%
\end{pgfscope}%
\begin{pgfscope}%
\pgfpathrectangle{\pgfqpoint{0.000000in}{0.000000in}}{\pgfqpoint{3.000000in}{3.000000in}}%
\pgfusepath{clip}%
\pgfsetbuttcap%
\pgfsetroundjoin%
\definecolor{currentfill}{rgb}{0.553476,0.000000,0.000000}%
\pgfsetfillcolor{currentfill}%
\pgfsetlinewidth{0.000000pt}%
\definecolor{currentstroke}{rgb}{0.000000,0.000000,0.000000}%
\pgfsetstrokecolor{currentstroke}%
\pgfsetdash{}{0pt}%
\pgfpathmoveto{\pgfqpoint{1.007383in}{2.311064in}}%
\pgfpathlineto{\pgfqpoint{0.996392in}{2.329527in}}%
\pgfpathlineto{\pgfqpoint{0.905412in}{2.286935in}}%
\pgfpathlineto{\pgfqpoint{0.918303in}{2.269247in}}%
\pgfpathlineto{\pgfqpoint{1.007383in}{2.311064in}}%
\pgfpathclose%
\pgfusepath{fill}%
\end{pgfscope}%
\begin{pgfscope}%
\pgfpathrectangle{\pgfqpoint{0.000000in}{0.000000in}}{\pgfqpoint{3.000000in}{3.000000in}}%
\pgfusepath{clip}%
\pgfsetbuttcap%
\pgfsetroundjoin%
\definecolor{currentfill}{rgb}{1.000000,0.233115,0.000000}%
\pgfsetfillcolor{currentfill}%
\pgfsetlinewidth{0.000000pt}%
\definecolor{currentstroke}{rgb}{0.000000,0.000000,0.000000}%
\pgfsetstrokecolor{currentstroke}%
\pgfsetdash{}{0pt}%
\pgfpathmoveto{\pgfqpoint{2.011303in}{2.092120in}}%
\pgfpathlineto{\pgfqpoint{2.023659in}{2.112499in}}%
\pgfpathlineto{\pgfqpoint{1.948133in}{2.145341in}}%
\pgfpathlineto{\pgfqpoint{1.937747in}{2.124200in}}%
\pgfpathlineto{\pgfqpoint{2.011303in}{2.092120in}}%
\pgfpathclose%
\pgfusepath{fill}%
\end{pgfscope}%
\begin{pgfscope}%
\pgfpathrectangle{\pgfqpoint{0.000000in}{0.000000in}}{\pgfqpoint{3.000000in}{3.000000in}}%
\pgfusepath{clip}%
\pgfsetbuttcap%
\pgfsetroundjoin%
\definecolor{currentfill}{rgb}{0.743201,1.000000,0.224541}%
\pgfsetfillcolor{currentfill}%
\pgfsetlinewidth{0.000000pt}%
\definecolor{currentstroke}{rgb}{0.000000,0.000000,0.000000}%
\pgfsetstrokecolor{currentstroke}%
\pgfsetdash{}{0pt}%
\pgfpathmoveto{\pgfqpoint{1.837625in}{1.757013in}}%
\pgfpathlineto{\pgfqpoint{1.850071in}{1.785498in}}%
\pgfpathlineto{\pgfqpoint{1.802034in}{1.807352in}}%
\pgfpathlineto{\pgfqpoint{1.791543in}{1.778055in}}%
\pgfpathlineto{\pgfqpoint{1.837625in}{1.757013in}}%
\pgfpathclose%
\pgfusepath{fill}%
\end{pgfscope}%
\begin{pgfscope}%
\pgfpathrectangle{\pgfqpoint{0.000000in}{0.000000in}}{\pgfqpoint{3.000000in}{3.000000in}}%
\pgfusepath{clip}%
\pgfsetbuttcap%
\pgfsetroundjoin%
\definecolor{currentfill}{rgb}{0.490196,1.000000,0.477546}%
\pgfsetfillcolor{currentfill}%
\pgfsetlinewidth{0.000000pt}%
\definecolor{currentstroke}{rgb}{0.000000,0.000000,0.000000}%
\pgfsetstrokecolor{currentstroke}%
\pgfsetdash{}{0pt}%
\pgfpathmoveto{\pgfqpoint{1.307018in}{1.677106in}}%
\pgfpathlineto{\pgfqpoint{1.295827in}{1.709904in}}%
\pgfpathlineto{\pgfqpoint{1.255691in}{1.689421in}}%
\pgfpathlineto{\pgfqpoint{1.268744in}{1.657492in}}%
\pgfpathlineto{\pgfqpoint{1.307018in}{1.677106in}}%
\pgfpathclose%
\pgfusepath{fill}%
\end{pgfscope}%
\begin{pgfscope}%
\pgfpathrectangle{\pgfqpoint{0.000000in}{0.000000in}}{\pgfqpoint{3.000000in}{3.000000in}}%
\pgfusepath{clip}%
\pgfsetbuttcap%
\pgfsetroundjoin%
\definecolor{currentfill}{rgb}{0.199241,1.000000,0.768501}%
\pgfsetfillcolor{currentfill}%
\pgfsetlinewidth{0.000000pt}%
\definecolor{currentstroke}{rgb}{0.000000,0.000000,0.000000}%
\pgfsetstrokecolor{currentstroke}%
\pgfsetdash{}{0pt}%
\pgfpathmoveto{\pgfqpoint{1.791904in}{1.538419in}}%
\pgfpathlineto{\pgfqpoint{1.806045in}{1.575259in}}%
\pgfpathlineto{\pgfqpoint{1.775310in}{1.594850in}}%
\pgfpathlineto{\pgfqpoint{1.762830in}{1.557042in}}%
\pgfpathlineto{\pgfqpoint{1.791904in}{1.538419in}}%
\pgfpathclose%
\pgfusepath{fill}%
\end{pgfscope}%
\begin{pgfscope}%
\pgfpathrectangle{\pgfqpoint{0.000000in}{0.000000in}}{\pgfqpoint{3.000000in}{3.000000in}}%
\pgfusepath{clip}%
\pgfsetbuttcap%
\pgfsetroundjoin%
\definecolor{currentfill}{rgb}{0.085389,1.000000,0.882353}%
\pgfsetfillcolor{currentfill}%
\pgfsetlinewidth{0.000000pt}%
\definecolor{currentstroke}{rgb}{0.000000,0.000000,0.000000}%
\pgfsetstrokecolor{currentstroke}%
\pgfsetdash{}{0pt}%
\pgfpathmoveto{\pgfqpoint{1.321004in}{1.511101in}}%
\pgfpathlineto{\pgfqpoint{1.307932in}{1.551094in}}%
\pgfpathlineto{\pgfqpoint{1.280793in}{1.531723in}}%
\pgfpathlineto{\pgfqpoint{1.295418in}{1.492742in}}%
\pgfpathlineto{\pgfqpoint{1.321004in}{1.511101in}}%
\pgfpathclose%
\pgfusepath{fill}%
\end{pgfscope}%
\begin{pgfscope}%
\pgfpathrectangle{\pgfqpoint{0.000000in}{0.000000in}}{\pgfqpoint{3.000000in}{3.000000in}}%
\pgfusepath{clip}%
\pgfsetbuttcap%
\pgfsetroundjoin%
\definecolor{currentfill}{rgb}{0.000000,0.000000,0.500000}%
\pgfsetfillcolor{currentfill}%
\pgfsetlinewidth{0.000000pt}%
\definecolor{currentstroke}{rgb}{0.000000,0.000000,0.000000}%
\pgfsetstrokecolor{currentstroke}%
\pgfsetdash{}{0pt}%
\pgfpathmoveto{\pgfqpoint{1.684505in}{1.000145in}}%
\pgfpathlineto{\pgfqpoint{1.699418in}{1.073353in}}%
\pgfpathlineto{\pgfqpoint{1.711574in}{1.086974in}}%
\pgfpathlineto{\pgfqpoint{1.695540in}{1.012581in}}%
\pgfpathlineto{\pgfqpoint{1.684505in}{1.000145in}}%
\pgfpathclose%
\pgfusepath{fill}%
\end{pgfscope}%
\begin{pgfscope}%
\pgfpathrectangle{\pgfqpoint{0.000000in}{0.000000in}}{\pgfqpoint{3.000000in}{3.000000in}}%
\pgfusepath{clip}%
\pgfsetbuttcap%
\pgfsetroundjoin%
\definecolor{currentfill}{rgb}{1.000000,0.291213,0.000000}%
\pgfsetfillcolor{currentfill}%
\pgfsetlinewidth{0.000000pt}%
\definecolor{currentstroke}{rgb}{0.000000,0.000000,0.000000}%
\pgfsetstrokecolor{currentstroke}%
\pgfsetdash{}{0pt}%
\pgfpathmoveto{\pgfqpoint{1.998940in}{2.071411in}}%
\pgfpathlineto{\pgfqpoint{2.011303in}{2.092120in}}%
\pgfpathlineto{\pgfqpoint{1.937747in}{2.124200in}}%
\pgfpathlineto{\pgfqpoint{1.927353in}{2.102725in}}%
\pgfpathlineto{\pgfqpoint{1.998940in}{2.071411in}}%
\pgfpathclose%
\pgfusepath{fill}%
\end{pgfscope}%
\begin{pgfscope}%
\pgfpathrectangle{\pgfqpoint{0.000000in}{0.000000in}}{\pgfqpoint{3.000000in}{3.000000in}}%
\pgfusepath{clip}%
\pgfsetbuttcap%
\pgfsetroundjoin%
\definecolor{currentfill}{rgb}{0.606952,0.000000,0.000000}%
\pgfsetfillcolor{currentfill}%
\pgfsetlinewidth{0.000000pt}%
\definecolor{currentstroke}{rgb}{0.000000,0.000000,0.000000}%
\pgfsetstrokecolor{currentstroke}%
\pgfsetdash{}{0pt}%
\pgfpathmoveto{\pgfqpoint{1.018381in}{2.292412in}}%
\pgfpathlineto{\pgfqpoint{1.007383in}{2.311064in}}%
\pgfpathlineto{\pgfqpoint{0.918303in}{2.269247in}}%
\pgfpathlineto{\pgfqpoint{0.931200in}{2.251372in}}%
\pgfpathlineto{\pgfqpoint{1.018381in}{2.292412in}}%
\pgfpathclose%
\pgfusepath{fill}%
\end{pgfscope}%
\begin{pgfscope}%
\pgfpathrectangle{\pgfqpoint{0.000000in}{0.000000in}}{\pgfqpoint{3.000000in}{3.000000in}}%
\pgfusepath{clip}%
\pgfsetbuttcap%
\pgfsetroundjoin%
\definecolor{currentfill}{rgb}{0.819102,1.000000,0.148640}%
\pgfsetfillcolor{currentfill}%
\pgfsetlinewidth{0.000000pt}%
\definecolor{currentstroke}{rgb}{0.000000,0.000000,0.000000}%
\pgfsetstrokecolor{currentstroke}%
\pgfsetdash{}{0pt}%
\pgfpathmoveto{\pgfqpoint{1.850071in}{1.785498in}}%
\pgfpathlineto{\pgfqpoint{1.862512in}{1.812998in}}%
\pgfpathlineto{\pgfqpoint{1.812517in}{1.835660in}}%
\pgfpathlineto{\pgfqpoint{1.802034in}{1.807352in}}%
\pgfpathlineto{\pgfqpoint{1.850071in}{1.785498in}}%
\pgfpathclose%
\pgfusepath{fill}%
\end{pgfscope}%
\begin{pgfscope}%
\pgfpathrectangle{\pgfqpoint{0.000000in}{0.000000in}}{\pgfqpoint{3.000000in}{3.000000in}}%
\pgfusepath{clip}%
\pgfsetbuttcap%
\pgfsetroundjoin%
\definecolor{currentfill}{rgb}{0.000000,0.676471,1.000000}%
\pgfsetfillcolor{currentfill}%
\pgfsetlinewidth{0.000000pt}%
\definecolor{currentstroke}{rgb}{0.000000,0.000000,0.000000}%
\pgfsetstrokecolor{currentstroke}%
\pgfsetdash{}{0pt}%
\pgfpathmoveto{\pgfqpoint{1.324674in}{1.405359in}}%
\pgfpathlineto{\pgfqpoint{1.310046in}{1.450815in}}%
\pgfpathlineto{\pgfqpoint{1.291564in}{1.431748in}}%
\pgfpathlineto{\pgfqpoint{1.307396in}{1.387419in}}%
\pgfpathlineto{\pgfqpoint{1.324674in}{1.405359in}}%
\pgfpathclose%
\pgfusepath{fill}%
\end{pgfscope}%
\begin{pgfscope}%
\pgfpathrectangle{\pgfqpoint{0.000000in}{0.000000in}}{\pgfqpoint{3.000000in}{3.000000in}}%
\pgfusepath{clip}%
\pgfsetbuttcap%
\pgfsetroundjoin%
\definecolor{currentfill}{rgb}{1.000000,0.349310,0.000000}%
\pgfsetfillcolor{currentfill}%
\pgfsetlinewidth{0.000000pt}%
\definecolor{currentstroke}{rgb}{0.000000,0.000000,0.000000}%
\pgfsetstrokecolor{currentstroke}%
\pgfsetdash{}{0pt}%
\pgfpathmoveto{\pgfqpoint{1.986571in}{2.050350in}}%
\pgfpathlineto{\pgfqpoint{1.998940in}{2.071411in}}%
\pgfpathlineto{\pgfqpoint{1.927353in}{2.102725in}}%
\pgfpathlineto{\pgfqpoint{1.916951in}{2.080894in}}%
\pgfpathlineto{\pgfqpoint{1.986571in}{2.050350in}}%
\pgfpathclose%
\pgfusepath{fill}%
\end{pgfscope}%
\begin{pgfscope}%
\pgfpathrectangle{\pgfqpoint{0.000000in}{0.000000in}}{\pgfqpoint{3.000000in}{3.000000in}}%
\pgfusepath{clip}%
\pgfsetbuttcap%
\pgfsetroundjoin%
\definecolor{currentfill}{rgb}{0.000000,0.503922,1.000000}%
\pgfsetfillcolor{currentfill}%
\pgfsetlinewidth{0.000000pt}%
\definecolor{currentstroke}{rgb}{0.000000,0.000000,0.000000}%
\pgfsetstrokecolor{currentstroke}%
\pgfsetdash{}{0pt}%
\pgfpathmoveto{\pgfqpoint{1.765664in}{1.326959in}}%
\pgfpathlineto{\pgfqpoint{1.782084in}{1.374768in}}%
\pgfpathlineto{\pgfqpoint{1.768535in}{1.393553in}}%
\pgfpathlineto{\pgfqpoint{1.753065in}{1.344557in}}%
\pgfpathlineto{\pgfqpoint{1.765664in}{1.326959in}}%
\pgfpathclose%
\pgfusepath{fill}%
\end{pgfscope}%
\begin{pgfscope}%
\pgfpathrectangle{\pgfqpoint{0.000000in}{0.000000in}}{\pgfqpoint{3.000000in}{3.000000in}}%
\pgfusepath{clip}%
\pgfsetbuttcap%
\pgfsetroundjoin%
\definecolor{currentfill}{rgb}{0.895003,1.000000,0.072739}%
\pgfsetfillcolor{currentfill}%
\pgfsetlinewidth{0.000000pt}%
\definecolor{currentstroke}{rgb}{0.000000,0.000000,0.000000}%
\pgfsetstrokecolor{currentstroke}%
\pgfsetdash{}{0pt}%
\pgfpathmoveto{\pgfqpoint{1.862512in}{1.812998in}}%
\pgfpathlineto{\pgfqpoint{1.874946in}{1.839612in}}%
\pgfpathlineto{\pgfqpoint{1.822993in}{1.863077in}}%
\pgfpathlineto{\pgfqpoint{1.812517in}{1.835660in}}%
\pgfpathlineto{\pgfqpoint{1.862512in}{1.812998in}}%
\pgfpathclose%
\pgfusepath{fill}%
\end{pgfscope}%
\begin{pgfscope}%
\pgfpathrectangle{\pgfqpoint{0.000000in}{0.000000in}}{\pgfqpoint{3.000000in}{3.000000in}}%
\pgfusepath{clip}%
\pgfsetbuttcap%
\pgfsetroundjoin%
\definecolor{currentfill}{rgb}{1.000000,0.407407,0.000000}%
\pgfsetfillcolor{currentfill}%
\pgfsetlinewidth{0.000000pt}%
\definecolor{currentstroke}{rgb}{0.000000,0.000000,0.000000}%
\pgfsetstrokecolor{currentstroke}%
\pgfsetdash{}{0pt}%
\pgfpathmoveto{\pgfqpoint{1.974194in}{2.028910in}}%
\pgfpathlineto{\pgfqpoint{1.986571in}{2.050350in}}%
\pgfpathlineto{\pgfqpoint{1.916951in}{2.080894in}}%
\pgfpathlineto{\pgfqpoint{1.906541in}{2.058681in}}%
\pgfpathlineto{\pgfqpoint{1.974194in}{2.028910in}}%
\pgfpathclose%
\pgfusepath{fill}%
\end{pgfscope}%
\begin{pgfscope}%
\pgfpathrectangle{\pgfqpoint{0.000000in}{0.000000in}}{\pgfqpoint{3.000000in}{3.000000in}}%
\pgfusepath{clip}%
\pgfsetbuttcap%
\pgfsetroundjoin%
\definecolor{currentfill}{rgb}{0.678253,0.000000,0.000000}%
\pgfsetfillcolor{currentfill}%
\pgfsetlinewidth{0.000000pt}%
\definecolor{currentstroke}{rgb}{0.000000,0.000000,0.000000}%
\pgfsetstrokecolor{currentstroke}%
\pgfsetdash{}{0pt}%
\pgfpathmoveto{\pgfqpoint{1.029387in}{2.273559in}}%
\pgfpathlineto{\pgfqpoint{1.018381in}{2.292412in}}%
\pgfpathlineto{\pgfqpoint{0.931200in}{2.251372in}}%
\pgfpathlineto{\pgfqpoint{0.944104in}{2.233301in}}%
\pgfpathlineto{\pgfqpoint{1.029387in}{2.273559in}}%
\pgfpathclose%
\pgfusepath{fill}%
\end{pgfscope}%
\begin{pgfscope}%
\pgfpathrectangle{\pgfqpoint{0.000000in}{0.000000in}}{\pgfqpoint{3.000000in}{3.000000in}}%
\pgfusepath{clip}%
\pgfsetbuttcap%
\pgfsetroundjoin%
\definecolor{currentfill}{rgb}{0.578748,1.000000,0.388994}%
\pgfsetfillcolor{currentfill}%
\pgfsetlinewidth{0.000000pt}%
\definecolor{currentstroke}{rgb}{0.000000,0.000000,0.000000}%
\pgfsetstrokecolor{currentstroke}%
\pgfsetdash{}{0pt}%
\pgfpathmoveto{\pgfqpoint{1.295827in}{1.709904in}}%
\pgfpathlineto{\pgfqpoint{1.284644in}{1.741294in}}%
\pgfpathlineto{\pgfqpoint{1.242642in}{1.719946in}}%
\pgfpathlineto{\pgfqpoint{1.255691in}{1.689421in}}%
\pgfpathlineto{\pgfqpoint{1.295827in}{1.709904in}}%
\pgfpathclose%
\pgfusepath{fill}%
\end{pgfscope}%
\begin{pgfscope}%
\pgfpathrectangle{\pgfqpoint{0.000000in}{0.000000in}}{\pgfqpoint{3.000000in}{3.000000in}}%
\pgfusepath{clip}%
\pgfsetbuttcap%
\pgfsetroundjoin%
\definecolor{currentfill}{rgb}{1.000000,0.480029,0.000000}%
\pgfsetfillcolor{currentfill}%
\pgfsetlinewidth{0.000000pt}%
\definecolor{currentstroke}{rgb}{0.000000,0.000000,0.000000}%
\pgfsetstrokecolor{currentstroke}%
\pgfsetdash{}{0pt}%
\pgfpathmoveto{\pgfqpoint{1.961811in}{2.007064in}}%
\pgfpathlineto{\pgfqpoint{1.974194in}{2.028910in}}%
\pgfpathlineto{\pgfqpoint{1.906541in}{2.058681in}}%
\pgfpathlineto{\pgfqpoint{1.896124in}{2.036059in}}%
\pgfpathlineto{\pgfqpoint{1.961811in}{2.007064in}}%
\pgfpathclose%
\pgfusepath{fill}%
\end{pgfscope}%
\begin{pgfscope}%
\pgfpathrectangle{\pgfqpoint{0.000000in}{0.000000in}}{\pgfqpoint{3.000000in}{3.000000in}}%
\pgfusepath{clip}%
\pgfsetbuttcap%
\pgfsetroundjoin%
\definecolor{currentfill}{rgb}{0.958254,0.973856,0.009488}%
\pgfsetfillcolor{currentfill}%
\pgfsetlinewidth{0.000000pt}%
\definecolor{currentstroke}{rgb}{0.000000,0.000000,0.000000}%
\pgfsetstrokecolor{currentstroke}%
\pgfsetdash{}{0pt}%
\pgfpathmoveto{\pgfqpoint{1.874946in}{1.839612in}}%
\pgfpathlineto{\pgfqpoint{1.887374in}{1.865425in}}%
\pgfpathlineto{\pgfqpoint{1.833463in}{1.889691in}}%
\pgfpathlineto{\pgfqpoint{1.822993in}{1.863077in}}%
\pgfpathlineto{\pgfqpoint{1.874946in}{1.839612in}}%
\pgfpathclose%
\pgfusepath{fill}%
\end{pgfscope}%
\begin{pgfscope}%
\pgfpathrectangle{\pgfqpoint{0.000000in}{0.000000in}}{\pgfqpoint{3.000000in}{3.000000in}}%
\pgfusepath{clip}%
\pgfsetbuttcap%
\pgfsetroundjoin%
\definecolor{currentfill}{rgb}{1.000000,0.538126,0.000000}%
\pgfsetfillcolor{currentfill}%
\pgfsetlinewidth{0.000000pt}%
\definecolor{currentstroke}{rgb}{0.000000,0.000000,0.000000}%
\pgfsetstrokecolor{currentstroke}%
\pgfsetdash{}{0pt}%
\pgfpathmoveto{\pgfqpoint{1.949421in}{1.984781in}}%
\pgfpathlineto{\pgfqpoint{1.961811in}{2.007064in}}%
\pgfpathlineto{\pgfqpoint{1.896124in}{2.036059in}}%
\pgfpathlineto{\pgfqpoint{1.885699in}{2.012997in}}%
\pgfpathlineto{\pgfqpoint{1.949421in}{1.984781in}}%
\pgfpathclose%
\pgfusepath{fill}%
\end{pgfscope}%
\begin{pgfscope}%
\pgfpathrectangle{\pgfqpoint{0.000000in}{0.000000in}}{\pgfqpoint{3.000000in}{3.000000in}}%
\pgfusepath{clip}%
\pgfsetbuttcap%
\pgfsetroundjoin%
\definecolor{currentfill}{rgb}{1.000000,0.886710,0.000000}%
\pgfsetfillcolor{currentfill}%
\pgfsetlinewidth{0.000000pt}%
\definecolor{currentstroke}{rgb}{0.000000,0.000000,0.000000}%
\pgfsetstrokecolor{currentstroke}%
\pgfsetdash{}{0pt}%
\pgfpathmoveto{\pgfqpoint{1.887374in}{1.865425in}}%
\pgfpathlineto{\pgfqpoint{1.899796in}{1.890512in}}%
\pgfpathlineto{\pgfqpoint{1.843925in}{1.915575in}}%
\pgfpathlineto{\pgfqpoint{1.833463in}{1.889691in}}%
\pgfpathlineto{\pgfqpoint{1.887374in}{1.865425in}}%
\pgfpathclose%
\pgfusepath{fill}%
\end{pgfscope}%
\begin{pgfscope}%
\pgfpathrectangle{\pgfqpoint{0.000000in}{0.000000in}}{\pgfqpoint{3.000000in}{3.000000in}}%
\pgfusepath{clip}%
\pgfsetbuttcap%
\pgfsetroundjoin%
\definecolor{currentfill}{rgb}{0.731729,0.000000,0.000000}%
\pgfsetfillcolor{currentfill}%
\pgfsetlinewidth{0.000000pt}%
\definecolor{currentstroke}{rgb}{0.000000,0.000000,0.000000}%
\pgfsetstrokecolor{currentstroke}%
\pgfsetdash{}{0pt}%
\pgfpathmoveto{\pgfqpoint{1.040402in}{2.254498in}}%
\pgfpathlineto{\pgfqpoint{1.029387in}{2.273559in}}%
\pgfpathlineto{\pgfqpoint{0.944104in}{2.233301in}}%
\pgfpathlineto{\pgfqpoint{0.957015in}{2.215024in}}%
\pgfpathlineto{\pgfqpoint{1.040402in}{2.254498in}}%
\pgfpathclose%
\pgfusepath{fill}%
\end{pgfscope}%
\begin{pgfscope}%
\pgfpathrectangle{\pgfqpoint{0.000000in}{0.000000in}}{\pgfqpoint{3.000000in}{3.000000in}}%
\pgfusepath{clip}%
\pgfsetbuttcap%
\pgfsetroundjoin%
\definecolor{currentfill}{rgb}{1.000000,0.610748,0.000000}%
\pgfsetfillcolor{currentfill}%
\pgfsetlinewidth{0.000000pt}%
\definecolor{currentstroke}{rgb}{0.000000,0.000000,0.000000}%
\pgfsetstrokecolor{currentstroke}%
\pgfsetdash{}{0pt}%
\pgfpathmoveto{\pgfqpoint{1.937025in}{1.962026in}}%
\pgfpathlineto{\pgfqpoint{1.949421in}{1.984781in}}%
\pgfpathlineto{\pgfqpoint{1.885699in}{2.012997in}}%
\pgfpathlineto{\pgfqpoint{1.875267in}{1.989458in}}%
\pgfpathlineto{\pgfqpoint{1.937025in}{1.962026in}}%
\pgfpathclose%
\pgfusepath{fill}%
\end{pgfscope}%
\begin{pgfscope}%
\pgfpathrectangle{\pgfqpoint{0.000000in}{0.000000in}}{\pgfqpoint{3.000000in}{3.000000in}}%
\pgfusepath{clip}%
\pgfsetbuttcap%
\pgfsetroundjoin%
\definecolor{currentfill}{rgb}{1.000000,0.814089,0.000000}%
\pgfsetfillcolor{currentfill}%
\pgfsetlinewidth{0.000000pt}%
\definecolor{currentstroke}{rgb}{0.000000,0.000000,0.000000}%
\pgfsetstrokecolor{currentstroke}%
\pgfsetdash{}{0pt}%
\pgfpathmoveto{\pgfqpoint{1.899796in}{1.890512in}}%
\pgfpathlineto{\pgfqpoint{1.912212in}{1.914938in}}%
\pgfpathlineto{\pgfqpoint{1.854379in}{1.940795in}}%
\pgfpathlineto{\pgfqpoint{1.843925in}{1.915575in}}%
\pgfpathlineto{\pgfqpoint{1.899796in}{1.890512in}}%
\pgfpathclose%
\pgfusepath{fill}%
\end{pgfscope}%
\begin{pgfscope}%
\pgfpathrectangle{\pgfqpoint{0.000000in}{0.000000in}}{\pgfqpoint{3.000000in}{3.000000in}}%
\pgfusepath{clip}%
\pgfsetbuttcap%
\pgfsetroundjoin%
\definecolor{currentfill}{rgb}{0.000000,0.000000,0.500000}%
\pgfsetfillcolor{currentfill}%
\pgfsetlinewidth{0.000000pt}%
\definecolor{currentstroke}{rgb}{0.000000,0.000000,0.000000}%
\pgfsetstrokecolor{currentstroke}%
\pgfsetdash{}{0pt}%
\pgfpathmoveto{\pgfqpoint{1.388802in}{1.008337in}}%
\pgfpathlineto{\pgfqpoint{1.373098in}{1.082326in}}%
\pgfpathlineto{\pgfqpoint{1.386614in}{1.069052in}}%
\pgfpathlineto{\pgfqpoint{1.401068in}{0.996219in}}%
\pgfpathlineto{\pgfqpoint{1.388802in}{1.008337in}}%
\pgfpathclose%
\pgfusepath{fill}%
\end{pgfscope}%
\begin{pgfscope}%
\pgfpathrectangle{\pgfqpoint{0.000000in}{0.000000in}}{\pgfqpoint{3.000000in}{3.000000in}}%
\pgfusepath{clip}%
\pgfsetbuttcap%
\pgfsetroundjoin%
\definecolor{currentfill}{rgb}{1.000000,0.668845,0.000000}%
\pgfsetfillcolor{currentfill}%
\pgfsetlinewidth{0.000000pt}%
\definecolor{currentstroke}{rgb}{0.000000,0.000000,0.000000}%
\pgfsetstrokecolor{currentstroke}%
\pgfsetdash{}{0pt}%
\pgfpathmoveto{\pgfqpoint{1.924622in}{1.938760in}}%
\pgfpathlineto{\pgfqpoint{1.937025in}{1.962026in}}%
\pgfpathlineto{\pgfqpoint{1.875267in}{1.989458in}}%
\pgfpathlineto{\pgfqpoint{1.864827in}{1.965406in}}%
\pgfpathlineto{\pgfqpoint{1.924622in}{1.938760in}}%
\pgfpathclose%
\pgfusepath{fill}%
\end{pgfscope}%
\begin{pgfscope}%
\pgfpathrectangle{\pgfqpoint{0.000000in}{0.000000in}}{\pgfqpoint{3.000000in}{3.000000in}}%
\pgfusepath{clip}%
\pgfsetbuttcap%
\pgfsetroundjoin%
\definecolor{currentfill}{rgb}{1.000000,0.741467,0.000000}%
\pgfsetfillcolor{currentfill}%
\pgfsetlinewidth{0.000000pt}%
\definecolor{currentstroke}{rgb}{0.000000,0.000000,0.000000}%
\pgfsetstrokecolor{currentstroke}%
\pgfsetdash{}{0pt}%
\pgfpathmoveto{\pgfqpoint{1.912212in}{1.914938in}}%
\pgfpathlineto{\pgfqpoint{1.924622in}{1.938760in}}%
\pgfpathlineto{\pgfqpoint{1.864827in}{1.965406in}}%
\pgfpathlineto{\pgfqpoint{1.854379in}{1.940795in}}%
\pgfpathlineto{\pgfqpoint{1.912212in}{1.914938in}}%
\pgfpathclose%
\pgfusepath{fill}%
\end{pgfscope}%
\begin{pgfscope}%
\pgfpathrectangle{\pgfqpoint{0.000000in}{0.000000in}}{\pgfqpoint{3.000000in}{3.000000in}}%
\pgfusepath{clip}%
\pgfsetbuttcap%
\pgfsetroundjoin%
\definecolor{currentfill}{rgb}{0.000000,0.000000,0.838681}%
\pgfsetfillcolor{currentfill}%
\pgfsetlinewidth{0.000000pt}%
\definecolor{currentstroke}{rgb}{0.000000,0.000000,0.000000}%
\pgfsetstrokecolor{currentstroke}%
\pgfsetdash{}{0pt}%
\pgfpathmoveto{\pgfqpoint{1.358794in}{1.111292in}}%
\pgfpathlineto{\pgfqpoint{1.341778in}{1.176568in}}%
\pgfpathlineto{\pgfqpoint{1.347175in}{1.160524in}}%
\pgfpathlineto{\pgfqpoint{1.363753in}{1.096525in}}%
\pgfpathlineto{\pgfqpoint{1.358794in}{1.111292in}}%
\pgfpathclose%
\pgfusepath{fill}%
\end{pgfscope}%
\begin{pgfscope}%
\pgfpathrectangle{\pgfqpoint{0.000000in}{0.000000in}}{\pgfqpoint{3.000000in}{3.000000in}}%
\pgfusepath{clip}%
\pgfsetbuttcap%
\pgfsetroundjoin%
\definecolor{currentfill}{rgb}{0.000000,0.849020,1.000000}%
\pgfsetfillcolor{currentfill}%
\pgfsetlinewidth{0.000000pt}%
\definecolor{currentstroke}{rgb}{0.000000,0.000000,0.000000}%
\pgfsetstrokecolor{currentstroke}%
\pgfsetdash{}{0pt}%
\pgfpathmoveto{\pgfqpoint{1.784006in}{1.438268in}}%
\pgfpathlineto{\pgfqpoint{1.799478in}{1.479457in}}%
\pgfpathlineto{\pgfqpoint{1.777760in}{1.499087in}}%
\pgfpathlineto{\pgfqpoint{1.763613in}{1.456808in}}%
\pgfpathlineto{\pgfqpoint{1.784006in}{1.438268in}}%
\pgfpathclose%
\pgfusepath{fill}%
\end{pgfscope}%
\begin{pgfscope}%
\pgfpathrectangle{\pgfqpoint{0.000000in}{0.000000in}}{\pgfqpoint{3.000000in}{3.000000in}}%
\pgfusepath{clip}%
\pgfsetbuttcap%
\pgfsetroundjoin%
\definecolor{currentfill}{rgb}{0.000000,0.064706,1.000000}%
\pgfsetfillcolor{currentfill}%
\pgfsetlinewidth{0.000000pt}%
\definecolor{currentstroke}{rgb}{0.000000,0.000000,0.000000}%
\pgfsetstrokecolor{currentstroke}%
\pgfsetdash{}{0pt}%
\pgfpathmoveto{\pgfqpoint{1.740013in}{1.181983in}}%
\pgfpathlineto{\pgfqpoint{1.757095in}{1.239171in}}%
\pgfpathlineto{\pgfqpoint{1.755813in}{1.256695in}}%
\pgfpathlineto{\pgfqpoint{1.738860in}{1.198223in}}%
\pgfpathlineto{\pgfqpoint{1.740013in}{1.181983in}}%
\pgfpathclose%
\pgfusepath{fill}%
\end{pgfscope}%
\begin{pgfscope}%
\pgfpathrectangle{\pgfqpoint{0.000000in}{0.000000in}}{\pgfqpoint{3.000000in}{3.000000in}}%
\pgfusepath{clip}%
\pgfsetbuttcap%
\pgfsetroundjoin%
\definecolor{currentfill}{rgb}{0.667299,1.000000,0.300443}%
\pgfsetfillcolor{currentfill}%
\pgfsetlinewidth{0.000000pt}%
\definecolor{currentstroke}{rgb}{0.000000,0.000000,0.000000}%
\pgfsetstrokecolor{currentstroke}%
\pgfsetdash{}{0pt}%
\pgfpathmoveto{\pgfqpoint{1.284644in}{1.741294in}}%
\pgfpathlineto{\pgfqpoint{1.273466in}{1.771442in}}%
\pgfpathlineto{\pgfqpoint{1.229599in}{1.749233in}}%
\pgfpathlineto{\pgfqpoint{1.242642in}{1.719946in}}%
\pgfpathlineto{\pgfqpoint{1.284644in}{1.741294in}}%
\pgfpathclose%
\pgfusepath{fill}%
\end{pgfscope}%
\begin{pgfscope}%
\pgfpathrectangle{\pgfqpoint{0.000000in}{0.000000in}}{\pgfqpoint{3.000000in}{3.000000in}}%
\pgfusepath{clip}%
\pgfsetbuttcap%
\pgfsetroundjoin%
\definecolor{currentfill}{rgb}{0.300443,1.000000,0.667299}%
\pgfsetfillcolor{currentfill}%
\pgfsetlinewidth{0.000000pt}%
\definecolor{currentstroke}{rgb}{0.000000,0.000000,0.000000}%
\pgfsetstrokecolor{currentstroke}%
\pgfsetdash{}{0pt}%
\pgfpathmoveto{\pgfqpoint{1.806045in}{1.575259in}}%
\pgfpathlineto{\pgfqpoint{1.820183in}{1.609970in}}%
\pgfpathlineto{\pgfqpoint{1.787784in}{1.630526in}}%
\pgfpathlineto{\pgfqpoint{1.775310in}{1.594850in}}%
\pgfpathlineto{\pgfqpoint{1.806045in}{1.575259in}}%
\pgfpathclose%
\pgfusepath{fill}%
\end{pgfscope}%
\begin{pgfscope}%
\pgfpathrectangle{\pgfqpoint{0.000000in}{0.000000in}}{\pgfqpoint{3.000000in}{3.000000in}}%
\pgfusepath{clip}%
\pgfsetbuttcap%
\pgfsetroundjoin%
\definecolor{currentfill}{rgb}{0.803030,0.000000,0.000000}%
\pgfsetfillcolor{currentfill}%
\pgfsetlinewidth{0.000000pt}%
\definecolor{currentstroke}{rgb}{0.000000,0.000000,0.000000}%
\pgfsetstrokecolor{currentstroke}%
\pgfsetdash{}{0pt}%
\pgfpathmoveto{\pgfqpoint{1.051424in}{2.235215in}}%
\pgfpathlineto{\pgfqpoint{1.040402in}{2.254498in}}%
\pgfpathlineto{\pgfqpoint{0.957015in}{2.215024in}}%
\pgfpathlineto{\pgfqpoint{0.969932in}{2.196529in}}%
\pgfpathlineto{\pgfqpoint{1.051424in}{2.235215in}}%
\pgfpathclose%
\pgfusepath{fill}%
\end{pgfscope}%
\begin{pgfscope}%
\pgfpathrectangle{\pgfqpoint{0.000000in}{0.000000in}}{\pgfqpoint{3.000000in}{3.000000in}}%
\pgfusepath{clip}%
\pgfsetbuttcap%
\pgfsetroundjoin%
\definecolor{currentfill}{rgb}{0.000000,0.300000,1.000000}%
\pgfsetfillcolor{currentfill}%
\pgfsetlinewidth{0.000000pt}%
\definecolor{currentstroke}{rgb}{0.000000,0.000000,0.000000}%
\pgfsetstrokecolor{currentstroke}%
\pgfsetdash{}{0pt}%
\pgfpathmoveto{\pgfqpoint{1.755813in}{1.256695in}}%
\pgfpathlineto{\pgfqpoint{1.772775in}{1.308533in}}%
\pgfpathlineto{\pgfqpoint{1.765664in}{1.326959in}}%
\pgfpathlineto{\pgfqpoint{1.749251in}{1.273869in}}%
\pgfpathlineto{\pgfqpoint{1.755813in}{1.256695in}}%
\pgfpathclose%
\pgfusepath{fill}%
\end{pgfscope}%
\begin{pgfscope}%
\pgfpathrectangle{\pgfqpoint{0.000000in}{0.000000in}}{\pgfqpoint{3.000000in}{3.000000in}}%
\pgfusepath{clip}%
\pgfsetbuttcap%
\pgfsetroundjoin%
\definecolor{currentfill}{rgb}{0.199241,1.000000,0.768501}%
\pgfsetfillcolor{currentfill}%
\pgfsetlinewidth{0.000000pt}%
\definecolor{currentstroke}{rgb}{0.000000,0.000000,0.000000}%
\pgfsetstrokecolor{currentstroke}%
\pgfsetdash{}{0pt}%
\pgfpathmoveto{\pgfqpoint{1.307932in}{1.551094in}}%
\pgfpathlineto{\pgfqpoint{1.294865in}{1.588594in}}%
\pgfpathlineto{\pgfqpoint{1.266169in}{1.568214in}}%
\pgfpathlineto{\pgfqpoint{1.280793in}{1.531723in}}%
\pgfpathlineto{\pgfqpoint{1.307932in}{1.551094in}}%
\pgfpathclose%
\pgfusepath{fill}%
\end{pgfscope}%
\begin{pgfscope}%
\pgfpathrectangle{\pgfqpoint{0.000000in}{0.000000in}}{\pgfqpoint{3.000000in}{3.000000in}}%
\pgfusepath{clip}%
\pgfsetbuttcap%
\pgfsetroundjoin%
\definecolor{currentfill}{rgb}{0.856506,0.000000,0.000000}%
\pgfsetfillcolor{currentfill}%
\pgfsetlinewidth{0.000000pt}%
\definecolor{currentstroke}{rgb}{0.000000,0.000000,0.000000}%
\pgfsetstrokecolor{currentstroke}%
\pgfsetdash{}{0pt}%
\pgfpathmoveto{\pgfqpoint{1.062454in}{2.215700in}}%
\pgfpathlineto{\pgfqpoint{1.051424in}{2.235215in}}%
\pgfpathlineto{\pgfqpoint{0.969932in}{2.196529in}}%
\pgfpathlineto{\pgfqpoint{0.982857in}{2.177805in}}%
\pgfpathlineto{\pgfqpoint{1.062454in}{2.215700in}}%
\pgfpathclose%
\pgfusepath{fill}%
\end{pgfscope}%
\begin{pgfscope}%
\pgfpathrectangle{\pgfqpoint{0.000000in}{0.000000in}}{\pgfqpoint{3.000000in}{3.000000in}}%
\pgfusepath{clip}%
\pgfsetbuttcap%
\pgfsetroundjoin%
\definecolor{currentfill}{rgb}{0.743201,1.000000,0.224541}%
\pgfsetfillcolor{currentfill}%
\pgfsetlinewidth{0.000000pt}%
\definecolor{currentstroke}{rgb}{0.000000,0.000000,0.000000}%
\pgfsetstrokecolor{currentstroke}%
\pgfsetdash{}{0pt}%
\pgfpathmoveto{\pgfqpoint{1.273466in}{1.771442in}}%
\pgfpathlineto{\pgfqpoint{1.262296in}{1.800484in}}%
\pgfpathlineto{\pgfqpoint{1.216561in}{1.777417in}}%
\pgfpathlineto{\pgfqpoint{1.229599in}{1.749233in}}%
\pgfpathlineto{\pgfqpoint{1.273466in}{1.771442in}}%
\pgfpathclose%
\pgfusepath{fill}%
\end{pgfscope}%
\begin{pgfscope}%
\pgfpathrectangle{\pgfqpoint{0.000000in}{0.000000in}}{\pgfqpoint{3.000000in}{3.000000in}}%
\pgfusepath{clip}%
\pgfsetbuttcap%
\pgfsetroundjoin%
\definecolor{currentfill}{rgb}{0.927807,0.015251,0.000000}%
\pgfsetfillcolor{currentfill}%
\pgfsetlinewidth{0.000000pt}%
\definecolor{currentstroke}{rgb}{0.000000,0.000000,0.000000}%
\pgfsetstrokecolor{currentstroke}%
\pgfsetdash{}{0pt}%
\pgfpathmoveto{\pgfqpoint{1.073493in}{2.195939in}}%
\pgfpathlineto{\pgfqpoint{1.062454in}{2.215700in}}%
\pgfpathlineto{\pgfqpoint{0.982857in}{2.177805in}}%
\pgfpathlineto{\pgfqpoint{0.995787in}{2.158838in}}%
\pgfpathlineto{\pgfqpoint{1.073493in}{2.195939in}}%
\pgfpathclose%
\pgfusepath{fill}%
\end{pgfscope}%
\begin{pgfscope}%
\pgfpathrectangle{\pgfqpoint{0.000000in}{0.000000in}}{\pgfqpoint{3.000000in}{3.000000in}}%
\pgfusepath{clip}%
\pgfsetbuttcap%
\pgfsetroundjoin%
\definecolor{currentfill}{rgb}{0.000000,0.503922,1.000000}%
\pgfsetfillcolor{currentfill}%
\pgfsetlinewidth{0.000000pt}%
\definecolor{currentstroke}{rgb}{0.000000,0.000000,0.000000}%
\pgfsetstrokecolor{currentstroke}%
\pgfsetdash{}{0pt}%
\pgfpathmoveto{\pgfqpoint{1.323224in}{1.338809in}}%
\pgfpathlineto{\pgfqpoint{1.307396in}{1.387419in}}%
\pgfpathlineto{\pgfqpoint{1.295775in}{1.368286in}}%
\pgfpathlineto{\pgfqpoint{1.312424in}{1.320887in}}%
\pgfpathlineto{\pgfqpoint{1.323224in}{1.338809in}}%
\pgfpathclose%
\pgfusepath{fill}%
\end{pgfscope}%
\begin{pgfscope}%
\pgfpathrectangle{\pgfqpoint{0.000000in}{0.000000in}}{\pgfqpoint{3.000000in}{3.000000in}}%
\pgfusepath{clip}%
\pgfsetbuttcap%
\pgfsetroundjoin%
\definecolor{currentfill}{rgb}{0.999109,0.073348,0.000000}%
\pgfsetfillcolor{currentfill}%
\pgfsetlinewidth{0.000000pt}%
\definecolor{currentstroke}{rgb}{0.000000,0.000000,0.000000}%
\pgfsetstrokecolor{currentstroke}%
\pgfsetdash{}{0pt}%
\pgfpathmoveto{\pgfqpoint{1.084538in}{2.175917in}}%
\pgfpathlineto{\pgfqpoint{1.073493in}{2.195939in}}%
\pgfpathlineto{\pgfqpoint{0.995787in}{2.158838in}}%
\pgfpathlineto{\pgfqpoint{1.008725in}{2.139614in}}%
\pgfpathlineto{\pgfqpoint{1.084538in}{2.175917in}}%
\pgfpathclose%
\pgfusepath{fill}%
\end{pgfscope}%
\begin{pgfscope}%
\pgfpathrectangle{\pgfqpoint{0.000000in}{0.000000in}}{\pgfqpoint{3.000000in}{3.000000in}}%
\pgfusepath{clip}%
\pgfsetbuttcap%
\pgfsetroundjoin%
\definecolor{currentfill}{rgb}{0.819102,1.000000,0.148640}%
\pgfsetfillcolor{currentfill}%
\pgfsetlinewidth{0.000000pt}%
\definecolor{currentstroke}{rgb}{0.000000,0.000000,0.000000}%
\pgfsetstrokecolor{currentstroke}%
\pgfsetdash{}{0pt}%
\pgfpathmoveto{\pgfqpoint{1.262296in}{1.800484in}}%
\pgfpathlineto{\pgfqpoint{1.251132in}{1.828538in}}%
\pgfpathlineto{\pgfqpoint{1.203528in}{1.804617in}}%
\pgfpathlineto{\pgfqpoint{1.216561in}{1.777417in}}%
\pgfpathlineto{\pgfqpoint{1.262296in}{1.800484in}}%
\pgfpathclose%
\pgfusepath{fill}%
\end{pgfscope}%
\begin{pgfscope}%
\pgfpathrectangle{\pgfqpoint{0.000000in}{0.000000in}}{\pgfqpoint{3.000000in}{3.000000in}}%
\pgfusepath{clip}%
\pgfsetbuttcap%
\pgfsetroundjoin%
\definecolor{currentfill}{rgb}{1.000000,0.116921,0.000000}%
\pgfsetfillcolor{currentfill}%
\pgfsetlinewidth{0.000000pt}%
\definecolor{currentstroke}{rgb}{0.000000,0.000000,0.000000}%
\pgfsetstrokecolor{currentstroke}%
\pgfsetdash{}{0pt}%
\pgfpathmoveto{\pgfqpoint{1.095592in}{2.155620in}}%
\pgfpathlineto{\pgfqpoint{1.084538in}{2.175917in}}%
\pgfpathlineto{\pgfqpoint{1.008725in}{2.139614in}}%
\pgfpathlineto{\pgfqpoint{1.021668in}{2.120118in}}%
\pgfpathlineto{\pgfqpoint{1.095592in}{2.155620in}}%
\pgfpathclose%
\pgfusepath{fill}%
\end{pgfscope}%
\begin{pgfscope}%
\pgfpathrectangle{\pgfqpoint{0.000000in}{0.000000in}}{\pgfqpoint{3.000000in}{3.000000in}}%
\pgfusepath{clip}%
\pgfsetbuttcap%
\pgfsetroundjoin%
\definecolor{currentfill}{rgb}{0.401645,1.000000,0.566097}%
\pgfsetfillcolor{currentfill}%
\pgfsetlinewidth{0.000000pt}%
\definecolor{currentstroke}{rgb}{0.000000,0.000000,0.000000}%
\pgfsetstrokecolor{currentstroke}%
\pgfsetdash{}{0pt}%
\pgfpathmoveto{\pgfqpoint{1.820183in}{1.609970in}}%
\pgfpathlineto{\pgfqpoint{1.834319in}{1.642845in}}%
\pgfpathlineto{\pgfqpoint{1.800252in}{1.664361in}}%
\pgfpathlineto{\pgfqpoint{1.787784in}{1.630526in}}%
\pgfpathlineto{\pgfqpoint{1.820183in}{1.609970in}}%
\pgfpathclose%
\pgfusepath{fill}%
\end{pgfscope}%
\begin{pgfscope}%
\pgfpathrectangle{\pgfqpoint{0.000000in}{0.000000in}}{\pgfqpoint{3.000000in}{3.000000in}}%
\pgfusepath{clip}%
\pgfsetbuttcap%
\pgfsetroundjoin%
\definecolor{currentfill}{rgb}{0.895003,1.000000,0.072739}%
\pgfsetfillcolor{currentfill}%
\pgfsetlinewidth{0.000000pt}%
\definecolor{currentstroke}{rgb}{0.000000,0.000000,0.000000}%
\pgfsetstrokecolor{currentstroke}%
\pgfsetdash{}{0pt}%
\pgfpathmoveto{\pgfqpoint{1.251132in}{1.828538in}}%
\pgfpathlineto{\pgfqpoint{1.239975in}{1.855704in}}%
\pgfpathlineto{\pgfqpoint{1.190500in}{1.830933in}}%
\pgfpathlineto{\pgfqpoint{1.203528in}{1.804617in}}%
\pgfpathlineto{\pgfqpoint{1.251132in}{1.828538in}}%
\pgfpathclose%
\pgfusepath{fill}%
\end{pgfscope}%
\begin{pgfscope}%
\pgfpathrectangle{\pgfqpoint{0.000000in}{0.000000in}}{\pgfqpoint{3.000000in}{3.000000in}}%
\pgfusepath{clip}%
\pgfsetbuttcap%
\pgfsetroundjoin%
\definecolor{currentfill}{rgb}{0.000000,0.064706,1.000000}%
\pgfsetfillcolor{currentfill}%
\pgfsetlinewidth{0.000000pt}%
\definecolor{currentstroke}{rgb}{0.000000,0.000000,0.000000}%
\pgfsetstrokecolor{currentstroke}%
\pgfsetdash{}{0pt}%
\pgfpathmoveto{\pgfqpoint{1.341291in}{1.192826in}}%
\pgfpathlineto{\pgfqpoint{1.324249in}{1.250871in}}%
\pgfpathlineto{\pgfqpoint{1.324747in}{1.233326in}}%
\pgfpathlineto{\pgfqpoint{1.341778in}{1.176568in}}%
\pgfpathlineto{\pgfqpoint{1.341291in}{1.192826in}}%
\pgfpathclose%
\pgfusepath{fill}%
\end{pgfscope}%
\begin{pgfscope}%
\pgfpathrectangle{\pgfqpoint{0.000000in}{0.000000in}}{\pgfqpoint{3.000000in}{3.000000in}}%
\pgfusepath{clip}%
\pgfsetbuttcap%
\pgfsetroundjoin%
\definecolor{currentfill}{rgb}{1.000000,0.175018,0.000000}%
\pgfsetfillcolor{currentfill}%
\pgfsetlinewidth{0.000000pt}%
\definecolor{currentstroke}{rgb}{0.000000,0.000000,0.000000}%
\pgfsetstrokecolor{currentstroke}%
\pgfsetdash{}{0pt}%
\pgfpathmoveto{\pgfqpoint{1.106654in}{2.135029in}}%
\pgfpathlineto{\pgfqpoint{1.095592in}{2.155620in}}%
\pgfpathlineto{\pgfqpoint{1.021668in}{2.120118in}}%
\pgfpathlineto{\pgfqpoint{1.034619in}{2.100332in}}%
\pgfpathlineto{\pgfqpoint{1.106654in}{2.135029in}}%
\pgfpathclose%
\pgfusepath{fill}%
\end{pgfscope}%
\begin{pgfscope}%
\pgfpathrectangle{\pgfqpoint{0.000000in}{0.000000in}}{\pgfqpoint{3.000000in}{3.000000in}}%
\pgfusepath{clip}%
\pgfsetbuttcap%
\pgfsetroundjoin%
\definecolor{currentfill}{rgb}{0.000000,0.849020,1.000000}%
\pgfsetfillcolor{currentfill}%
\pgfsetlinewidth{0.000000pt}%
\definecolor{currentstroke}{rgb}{0.000000,0.000000,0.000000}%
\pgfsetstrokecolor{currentstroke}%
\pgfsetdash{}{0pt}%
\pgfpathmoveto{\pgfqpoint{1.310046in}{1.450815in}}%
\pgfpathlineto{\pgfqpoint{1.295418in}{1.492742in}}%
\pgfpathlineto{\pgfqpoint{1.275731in}{1.472552in}}%
\pgfpathlineto{\pgfqpoint{1.291564in}{1.431748in}}%
\pgfpathlineto{\pgfqpoint{1.310046in}{1.450815in}}%
\pgfpathclose%
\pgfusepath{fill}%
\end{pgfscope}%
\begin{pgfscope}%
\pgfpathrectangle{\pgfqpoint{0.000000in}{0.000000in}}{\pgfqpoint{3.000000in}{3.000000in}}%
\pgfusepath{clip}%
\pgfsetbuttcap%
\pgfsetroundjoin%
\definecolor{currentfill}{rgb}{0.000000,0.300000,1.000000}%
\pgfsetfillcolor{currentfill}%
\pgfsetlinewidth{0.000000pt}%
\definecolor{currentstroke}{rgb}{0.000000,0.000000,0.000000}%
\pgfsetstrokecolor{currentstroke}%
\pgfsetdash{}{0pt}%
\pgfpathmoveto{\pgfqpoint{1.329065in}{1.268209in}}%
\pgfpathlineto{\pgfqpoint{1.312424in}{1.320887in}}%
\pgfpathlineto{\pgfqpoint{1.307195in}{1.302284in}}%
\pgfpathlineto{\pgfqpoint{1.324249in}{1.250871in}}%
\pgfpathlineto{\pgfqpoint{1.329065in}{1.268209in}}%
\pgfpathclose%
\pgfusepath{fill}%
\end{pgfscope}%
\begin{pgfscope}%
\pgfpathrectangle{\pgfqpoint{0.000000in}{0.000000in}}{\pgfqpoint{3.000000in}{3.000000in}}%
\pgfusepath{clip}%
\pgfsetbuttcap%
\pgfsetroundjoin%
\definecolor{currentfill}{rgb}{1.000000,0.233115,0.000000}%
\pgfsetfillcolor{currentfill}%
\pgfsetlinewidth{0.000000pt}%
\definecolor{currentstroke}{rgb}{0.000000,0.000000,0.000000}%
\pgfsetstrokecolor{currentstroke}%
\pgfsetdash{}{0pt}%
\pgfpathmoveto{\pgfqpoint{1.117723in}{2.114127in}}%
\pgfpathlineto{\pgfqpoint{1.106654in}{2.135029in}}%
\pgfpathlineto{\pgfqpoint{1.034619in}{2.100332in}}%
\pgfpathlineto{\pgfqpoint{1.047575in}{2.080238in}}%
\pgfpathlineto{\pgfqpoint{1.117723in}{2.114127in}}%
\pgfpathclose%
\pgfusepath{fill}%
\end{pgfscope}%
\begin{pgfscope}%
\pgfpathrectangle{\pgfqpoint{0.000000in}{0.000000in}}{\pgfqpoint{3.000000in}{3.000000in}}%
\pgfusepath{clip}%
\pgfsetbuttcap%
\pgfsetroundjoin%
\definecolor{currentfill}{rgb}{0.958254,0.973856,0.009488}%
\pgfsetfillcolor{currentfill}%
\pgfsetlinewidth{0.000000pt}%
\definecolor{currentstroke}{rgb}{0.000000,0.000000,0.000000}%
\pgfsetstrokecolor{currentstroke}%
\pgfsetdash{}{0pt}%
\pgfpathmoveto{\pgfqpoint{1.239975in}{1.855704in}}%
\pgfpathlineto{\pgfqpoint{1.228825in}{1.882067in}}%
\pgfpathlineto{\pgfqpoint{1.177478in}{1.856449in}}%
\pgfpathlineto{\pgfqpoint{1.190500in}{1.830933in}}%
\pgfpathlineto{\pgfqpoint{1.239975in}{1.855704in}}%
\pgfpathclose%
\pgfusepath{fill}%
\end{pgfscope}%
\begin{pgfscope}%
\pgfpathrectangle{\pgfqpoint{0.000000in}{0.000000in}}{\pgfqpoint{3.000000in}{3.000000in}}%
\pgfusepath{clip}%
\pgfsetbuttcap%
\pgfsetroundjoin%
\definecolor{currentfill}{rgb}{0.000000,0.000000,0.500000}%
\pgfsetfillcolor{currentfill}%
\pgfsetlinewidth{0.000000pt}%
\definecolor{currentstroke}{rgb}{0.000000,0.000000,0.000000}%
\pgfsetstrokecolor{currentstroke}%
\pgfsetdash{}{0pt}%
\pgfpathmoveto{\pgfqpoint{1.669879in}{0.988763in}}%
\pgfpathlineto{\pgfqpoint{1.683294in}{1.060883in}}%
\pgfpathlineto{\pgfqpoint{1.699418in}{1.073353in}}%
\pgfpathlineto{\pgfqpoint{1.684505in}{1.000145in}}%
\pgfpathlineto{\pgfqpoint{1.669879in}{0.988763in}}%
\pgfpathclose%
\pgfusepath{fill}%
\end{pgfscope}%
\begin{pgfscope}%
\pgfpathrectangle{\pgfqpoint{0.000000in}{0.000000in}}{\pgfqpoint{3.000000in}{3.000000in}}%
\pgfusepath{clip}%
\pgfsetbuttcap%
\pgfsetroundjoin%
\definecolor{currentfill}{rgb}{0.300443,1.000000,0.667299}%
\pgfsetfillcolor{currentfill}%
\pgfsetlinewidth{0.000000pt}%
\definecolor{currentstroke}{rgb}{0.000000,0.000000,0.000000}%
\pgfsetstrokecolor{currentstroke}%
\pgfsetdash{}{0pt}%
\pgfpathmoveto{\pgfqpoint{1.294865in}{1.588594in}}%
\pgfpathlineto{\pgfqpoint{1.281802in}{1.623962in}}%
\pgfpathlineto{\pgfqpoint{1.251547in}{1.602577in}}%
\pgfpathlineto{\pgfqpoint{1.266169in}{1.568214in}}%
\pgfpathlineto{\pgfqpoint{1.294865in}{1.588594in}}%
\pgfpathclose%
\pgfusepath{fill}%
\end{pgfscope}%
\begin{pgfscope}%
\pgfpathrectangle{\pgfqpoint{0.000000in}{0.000000in}}{\pgfqpoint{3.000000in}{3.000000in}}%
\pgfusepath{clip}%
\pgfsetbuttcap%
\pgfsetroundjoin%
\definecolor{currentfill}{rgb}{1.000000,0.291213,0.000000}%
\pgfsetfillcolor{currentfill}%
\pgfsetlinewidth{0.000000pt}%
\definecolor{currentstroke}{rgb}{0.000000,0.000000,0.000000}%
\pgfsetstrokecolor{currentstroke}%
\pgfsetdash{}{0pt}%
\pgfpathmoveto{\pgfqpoint{1.128799in}{2.092892in}}%
\pgfpathlineto{\pgfqpoint{1.117723in}{2.114127in}}%
\pgfpathlineto{\pgfqpoint{1.047575in}{2.080238in}}%
\pgfpathlineto{\pgfqpoint{1.060538in}{2.059814in}}%
\pgfpathlineto{\pgfqpoint{1.128799in}{2.092892in}}%
\pgfpathclose%
\pgfusepath{fill}%
\end{pgfscope}%
\begin{pgfscope}%
\pgfpathrectangle{\pgfqpoint{0.000000in}{0.000000in}}{\pgfqpoint{3.000000in}{3.000000in}}%
\pgfusepath{clip}%
\pgfsetbuttcap%
\pgfsetroundjoin%
\definecolor{currentfill}{rgb}{1.000000,0.886710,0.000000}%
\pgfsetfillcolor{currentfill}%
\pgfsetlinewidth{0.000000pt}%
\definecolor{currentstroke}{rgb}{0.000000,0.000000,0.000000}%
\pgfsetstrokecolor{currentstroke}%
\pgfsetdash{}{0pt}%
\pgfpathmoveto{\pgfqpoint{1.228825in}{1.882067in}}%
\pgfpathlineto{\pgfqpoint{1.217682in}{1.907701in}}%
\pgfpathlineto{\pgfqpoint{1.164461in}{1.881240in}}%
\pgfpathlineto{\pgfqpoint{1.177478in}{1.856449in}}%
\pgfpathlineto{\pgfqpoint{1.228825in}{1.882067in}}%
\pgfpathclose%
\pgfusepath{fill}%
\end{pgfscope}%
\begin{pgfscope}%
\pgfpathrectangle{\pgfqpoint{0.000000in}{0.000000in}}{\pgfqpoint{3.000000in}{3.000000in}}%
\pgfusepath{clip}%
\pgfsetbuttcap%
\pgfsetroundjoin%
\definecolor{currentfill}{rgb}{1.000000,0.349310,0.000000}%
\pgfsetfillcolor{currentfill}%
\pgfsetlinewidth{0.000000pt}%
\definecolor{currentstroke}{rgb}{0.000000,0.000000,0.000000}%
\pgfsetstrokecolor{currentstroke}%
\pgfsetdash{}{0pt}%
\pgfpathmoveto{\pgfqpoint{1.139884in}{2.071302in}}%
\pgfpathlineto{\pgfqpoint{1.128799in}{2.092892in}}%
\pgfpathlineto{\pgfqpoint{1.060538in}{2.059814in}}%
\pgfpathlineto{\pgfqpoint{1.073507in}{2.039039in}}%
\pgfpathlineto{\pgfqpoint{1.139884in}{2.071302in}}%
\pgfpathclose%
\pgfusepath{fill}%
\end{pgfscope}%
\begin{pgfscope}%
\pgfpathrectangle{\pgfqpoint{0.000000in}{0.000000in}}{\pgfqpoint{3.000000in}{3.000000in}}%
\pgfusepath{clip}%
\pgfsetbuttcap%
\pgfsetroundjoin%
\definecolor{currentfill}{rgb}{0.085389,1.000000,0.882353}%
\pgfsetfillcolor{currentfill}%
\pgfsetlinewidth{0.000000pt}%
\definecolor{currentstroke}{rgb}{0.000000,0.000000,0.000000}%
\pgfsetstrokecolor{currentstroke}%
\pgfsetdash{}{0pt}%
\pgfpathmoveto{\pgfqpoint{1.799478in}{1.479457in}}%
\pgfpathlineto{\pgfqpoint{1.814951in}{1.517702in}}%
\pgfpathlineto{\pgfqpoint{1.791904in}{1.538419in}}%
\pgfpathlineto{\pgfqpoint{1.777760in}{1.499087in}}%
\pgfpathlineto{\pgfqpoint{1.799478in}{1.479457in}}%
\pgfpathclose%
\pgfusepath{fill}%
\end{pgfscope}%
\begin{pgfscope}%
\pgfpathrectangle{\pgfqpoint{0.000000in}{0.000000in}}{\pgfqpoint{3.000000in}{3.000000in}}%
\pgfusepath{clip}%
\pgfsetbuttcap%
\pgfsetroundjoin%
\definecolor{currentfill}{rgb}{1.000000,0.814089,0.000000}%
\pgfsetfillcolor{currentfill}%
\pgfsetlinewidth{0.000000pt}%
\definecolor{currentstroke}{rgb}{0.000000,0.000000,0.000000}%
\pgfsetstrokecolor{currentstroke}%
\pgfsetdash{}{0pt}%
\pgfpathmoveto{\pgfqpoint{1.217682in}{1.907701in}}%
\pgfpathlineto{\pgfqpoint{1.206546in}{1.932672in}}%
\pgfpathlineto{\pgfqpoint{1.151450in}{1.905371in}}%
\pgfpathlineto{\pgfqpoint{1.164461in}{1.881240in}}%
\pgfpathlineto{\pgfqpoint{1.217682in}{1.907701in}}%
\pgfpathclose%
\pgfusepath{fill}%
\end{pgfscope}%
\begin{pgfscope}%
\pgfpathrectangle{\pgfqpoint{0.000000in}{0.000000in}}{\pgfqpoint{3.000000in}{3.000000in}}%
\pgfusepath{clip}%
\pgfsetbuttcap%
\pgfsetroundjoin%
\definecolor{currentfill}{rgb}{1.000000,0.407407,0.000000}%
\pgfsetfillcolor{currentfill}%
\pgfsetlinewidth{0.000000pt}%
\definecolor{currentstroke}{rgb}{0.000000,0.000000,0.000000}%
\pgfsetstrokecolor{currentstroke}%
\pgfsetdash{}{0pt}%
\pgfpathmoveto{\pgfqpoint{1.150976in}{2.049332in}}%
\pgfpathlineto{\pgfqpoint{1.139884in}{2.071302in}}%
\pgfpathlineto{\pgfqpoint{1.073507in}{2.039039in}}%
\pgfpathlineto{\pgfqpoint{1.086483in}{2.017887in}}%
\pgfpathlineto{\pgfqpoint{1.150976in}{2.049332in}}%
\pgfpathclose%
\pgfusepath{fill}%
\end{pgfscope}%
\begin{pgfscope}%
\pgfpathrectangle{\pgfqpoint{0.000000in}{0.000000in}}{\pgfqpoint{3.000000in}{3.000000in}}%
\pgfusepath{clip}%
\pgfsetbuttcap%
\pgfsetroundjoin%
\definecolor{currentfill}{rgb}{1.000000,0.741467,0.000000}%
\pgfsetfillcolor{currentfill}%
\pgfsetlinewidth{0.000000pt}%
\definecolor{currentstroke}{rgb}{0.000000,0.000000,0.000000}%
\pgfsetstrokecolor{currentstroke}%
\pgfsetdash{}{0pt}%
\pgfpathmoveto{\pgfqpoint{1.206546in}{1.932672in}}%
\pgfpathlineto{\pgfqpoint{1.195417in}{1.957035in}}%
\pgfpathlineto{\pgfqpoint{1.138445in}{1.928899in}}%
\pgfpathlineto{\pgfqpoint{1.151450in}{1.905371in}}%
\pgfpathlineto{\pgfqpoint{1.206546in}{1.932672in}}%
\pgfpathclose%
\pgfusepath{fill}%
\end{pgfscope}%
\begin{pgfscope}%
\pgfpathrectangle{\pgfqpoint{0.000000in}{0.000000in}}{\pgfqpoint{3.000000in}{3.000000in}}%
\pgfusepath{clip}%
\pgfsetbuttcap%
\pgfsetroundjoin%
\definecolor{currentfill}{rgb}{1.000000,0.480029,0.000000}%
\pgfsetfillcolor{currentfill}%
\pgfsetlinewidth{0.000000pt}%
\definecolor{currentstroke}{rgb}{0.000000,0.000000,0.000000}%
\pgfsetstrokecolor{currentstroke}%
\pgfsetdash{}{0pt}%
\pgfpathmoveto{\pgfqpoint{1.162075in}{2.026953in}}%
\pgfpathlineto{\pgfqpoint{1.150976in}{2.049332in}}%
\pgfpathlineto{\pgfqpoint{1.086483in}{2.017887in}}%
\pgfpathlineto{\pgfqpoint{1.099464in}{1.996330in}}%
\pgfpathlineto{\pgfqpoint{1.162075in}{2.026953in}}%
\pgfpathclose%
\pgfusepath{fill}%
\end{pgfscope}%
\begin{pgfscope}%
\pgfpathrectangle{\pgfqpoint{0.000000in}{0.000000in}}{\pgfqpoint{3.000000in}{3.000000in}}%
\pgfusepath{clip}%
\pgfsetbuttcap%
\pgfsetroundjoin%
\definecolor{currentfill}{rgb}{1.000000,0.668845,0.000000}%
\pgfsetfillcolor{currentfill}%
\pgfsetlinewidth{0.000000pt}%
\definecolor{currentstroke}{rgb}{0.000000,0.000000,0.000000}%
\pgfsetstrokecolor{currentstroke}%
\pgfsetdash{}{0pt}%
\pgfpathmoveto{\pgfqpoint{1.195417in}{1.957035in}}%
\pgfpathlineto{\pgfqpoint{1.184296in}{1.980841in}}%
\pgfpathlineto{\pgfqpoint{1.125445in}{1.951873in}}%
\pgfpathlineto{\pgfqpoint{1.138445in}{1.928899in}}%
\pgfpathlineto{\pgfqpoint{1.195417in}{1.957035in}}%
\pgfpathclose%
\pgfusepath{fill}%
\end{pgfscope}%
\begin{pgfscope}%
\pgfpathrectangle{\pgfqpoint{0.000000in}{0.000000in}}{\pgfqpoint{3.000000in}{3.000000in}}%
\pgfusepath{clip}%
\pgfsetbuttcap%
\pgfsetroundjoin%
\definecolor{currentfill}{rgb}{0.000000,0.676471,1.000000}%
\pgfsetfillcolor{currentfill}%
\pgfsetlinewidth{0.000000pt}%
\definecolor{currentstroke}{rgb}{0.000000,0.000000,0.000000}%
\pgfsetstrokecolor{currentstroke}%
\pgfsetdash{}{0pt}%
\pgfpathmoveto{\pgfqpoint{1.782084in}{1.374768in}}%
\pgfpathlineto{\pgfqpoint{1.798509in}{1.418300in}}%
\pgfpathlineto{\pgfqpoint{1.784006in}{1.438268in}}%
\pgfpathlineto{\pgfqpoint{1.768535in}{1.393553in}}%
\pgfpathlineto{\pgfqpoint{1.782084in}{1.374768in}}%
\pgfpathclose%
\pgfusepath{fill}%
\end{pgfscope}%
\begin{pgfscope}%
\pgfpathrectangle{\pgfqpoint{0.000000in}{0.000000in}}{\pgfqpoint{3.000000in}{3.000000in}}%
\pgfusepath{clip}%
\pgfsetbuttcap%
\pgfsetroundjoin%
\definecolor{currentfill}{rgb}{1.000000,0.538126,0.000000}%
\pgfsetfillcolor{currentfill}%
\pgfsetlinewidth{0.000000pt}%
\definecolor{currentstroke}{rgb}{0.000000,0.000000,0.000000}%
\pgfsetstrokecolor{currentstroke}%
\pgfsetdash{}{0pt}%
\pgfpathmoveto{\pgfqpoint{1.173182in}{2.004134in}}%
\pgfpathlineto{\pgfqpoint{1.162075in}{2.026953in}}%
\pgfpathlineto{\pgfqpoint{1.099464in}{1.996330in}}%
\pgfpathlineto{\pgfqpoint{1.112452in}{1.974337in}}%
\pgfpathlineto{\pgfqpoint{1.173182in}{2.004134in}}%
\pgfpathclose%
\pgfusepath{fill}%
\end{pgfscope}%
\begin{pgfscope}%
\pgfpathrectangle{\pgfqpoint{0.000000in}{0.000000in}}{\pgfqpoint{3.000000in}{3.000000in}}%
\pgfusepath{clip}%
\pgfsetbuttcap%
\pgfsetroundjoin%
\definecolor{currentfill}{rgb}{1.000000,0.610748,0.000000}%
\pgfsetfillcolor{currentfill}%
\pgfsetlinewidth{0.000000pt}%
\definecolor{currentstroke}{rgb}{0.000000,0.000000,0.000000}%
\pgfsetstrokecolor{currentstroke}%
\pgfsetdash{}{0pt}%
\pgfpathmoveto{\pgfqpoint{1.184296in}{1.980841in}}%
\pgfpathlineto{\pgfqpoint{1.173182in}{2.004134in}}%
\pgfpathlineto{\pgfqpoint{1.112452in}{1.974337in}}%
\pgfpathlineto{\pgfqpoint{1.125445in}{1.951873in}}%
\pgfpathlineto{\pgfqpoint{1.184296in}{1.980841in}}%
\pgfpathclose%
\pgfusepath{fill}%
\end{pgfscope}%
\begin{pgfscope}%
\pgfpathrectangle{\pgfqpoint{0.000000in}{0.000000in}}{\pgfqpoint{3.000000in}{3.000000in}}%
\pgfusepath{clip}%
\pgfsetbuttcap%
\pgfsetroundjoin%
\definecolor{currentfill}{rgb}{0.000000,0.000000,0.838681}%
\pgfsetfillcolor{currentfill}%
\pgfsetlinewidth{0.000000pt}%
\definecolor{currentstroke}{rgb}{0.000000,0.000000,0.000000}%
\pgfsetstrokecolor{currentstroke}%
\pgfsetdash{}{0pt}%
\pgfpathmoveto{\pgfqpoint{1.711574in}{1.086974in}}%
\pgfpathlineto{\pgfqpoint{1.727630in}{1.150145in}}%
\pgfpathlineto{\pgfqpoint{1.736246in}{1.165824in}}%
\pgfpathlineto{\pgfqpoint{1.719476in}{1.101402in}}%
\pgfpathlineto{\pgfqpoint{1.711574in}{1.086974in}}%
\pgfpathclose%
\pgfusepath{fill}%
\end{pgfscope}%
\begin{pgfscope}%
\pgfpathrectangle{\pgfqpoint{0.000000in}{0.000000in}}{\pgfqpoint{3.000000in}{3.000000in}}%
\pgfusepath{clip}%
\pgfsetbuttcap%
\pgfsetroundjoin%
\definecolor{currentfill}{rgb}{0.490196,1.000000,0.477546}%
\pgfsetfillcolor{currentfill}%
\pgfsetlinewidth{0.000000pt}%
\definecolor{currentstroke}{rgb}{0.000000,0.000000,0.000000}%
\pgfsetstrokecolor{currentstroke}%
\pgfsetdash{}{0pt}%
\pgfpathmoveto{\pgfqpoint{1.834319in}{1.642845in}}%
\pgfpathlineto{\pgfqpoint{1.848451in}{1.674123in}}%
\pgfpathlineto{\pgfqpoint{1.812716in}{1.696595in}}%
\pgfpathlineto{\pgfqpoint{1.800252in}{1.664361in}}%
\pgfpathlineto{\pgfqpoint{1.834319in}{1.642845in}}%
\pgfpathclose%
\pgfusepath{fill}%
\end{pgfscope}%
\begin{pgfscope}%
\pgfpathrectangle{\pgfqpoint{0.000000in}{0.000000in}}{\pgfqpoint{3.000000in}{3.000000in}}%
\pgfusepath{clip}%
\pgfsetbuttcap%
\pgfsetroundjoin%
\definecolor{currentfill}{rgb}{0.000000,0.000000,0.500000}%
\pgfsetfillcolor{currentfill}%
\pgfsetlinewidth{0.000000pt}%
\definecolor{currentstroke}{rgb}{0.000000,0.000000,0.000000}%
\pgfsetstrokecolor{currentstroke}%
\pgfsetdash{}{0pt}%
\pgfpathmoveto{\pgfqpoint{1.401068in}{0.996219in}}%
\pgfpathlineto{\pgfqpoint{1.386614in}{1.069052in}}%
\pgfpathlineto{\pgfqpoint{1.403980in}{1.057038in}}%
\pgfpathlineto{\pgfqpoint{1.416818in}{0.985253in}}%
\pgfpathlineto{\pgfqpoint{1.401068in}{0.996219in}}%
\pgfpathclose%
\pgfusepath{fill}%
\end{pgfscope}%
\begin{pgfscope}%
\pgfpathrectangle{\pgfqpoint{0.000000in}{0.000000in}}{\pgfqpoint{3.000000in}{3.000000in}}%
\pgfusepath{clip}%
\pgfsetbuttcap%
\pgfsetroundjoin%
\definecolor{currentfill}{rgb}{0.401645,1.000000,0.566097}%
\pgfsetfillcolor{currentfill}%
\pgfsetlinewidth{0.000000pt}%
\definecolor{currentstroke}{rgb}{0.000000,0.000000,0.000000}%
\pgfsetstrokecolor{currentstroke}%
\pgfsetdash{}{0pt}%
\pgfpathmoveto{\pgfqpoint{1.281802in}{1.623962in}}%
\pgfpathlineto{\pgfqpoint{1.268744in}{1.657492in}}%
\pgfpathlineto{\pgfqpoint{1.236926in}{1.635106in}}%
\pgfpathlineto{\pgfqpoint{1.251547in}{1.602577in}}%
\pgfpathlineto{\pgfqpoint{1.281802in}{1.623962in}}%
\pgfpathclose%
\pgfusepath{fill}%
\end{pgfscope}%
\begin{pgfscope}%
\pgfpathrectangle{\pgfqpoint{0.000000in}{0.000000in}}{\pgfqpoint{3.000000in}{3.000000in}}%
\pgfusepath{clip}%
\pgfsetbuttcap%
\pgfsetroundjoin%
\definecolor{currentfill}{rgb}{0.578748,1.000000,0.388994}%
\pgfsetfillcolor{currentfill}%
\pgfsetlinewidth{0.000000pt}%
\definecolor{currentstroke}{rgb}{0.000000,0.000000,0.000000}%
\pgfsetstrokecolor{currentstroke}%
\pgfsetdash{}{0pt}%
\pgfpathmoveto{\pgfqpoint{1.848451in}{1.674123in}}%
\pgfpathlineto{\pgfqpoint{1.862581in}{1.704000in}}%
\pgfpathlineto{\pgfqpoint{1.825173in}{1.727424in}}%
\pgfpathlineto{\pgfqpoint{1.812716in}{1.696595in}}%
\pgfpathlineto{\pgfqpoint{1.848451in}{1.674123in}}%
\pgfpathclose%
\pgfusepath{fill}%
\end{pgfscope}%
\begin{pgfscope}%
\pgfpathrectangle{\pgfqpoint{0.000000in}{0.000000in}}{\pgfqpoint{3.000000in}{3.000000in}}%
\pgfusepath{clip}%
\pgfsetbuttcap%
\pgfsetroundjoin%
\definecolor{currentfill}{rgb}{0.000000,0.000000,0.838681}%
\pgfsetfillcolor{currentfill}%
\pgfsetlinewidth{0.000000pt}%
\definecolor{currentstroke}{rgb}{0.000000,0.000000,0.000000}%
\pgfsetstrokecolor{currentstroke}%
\pgfsetdash{}{0pt}%
\pgfpathmoveto{\pgfqpoint{1.363753in}{1.096525in}}%
\pgfpathlineto{\pgfqpoint{1.347175in}{1.160524in}}%
\pgfpathlineto{\pgfqpoint{1.357371in}{1.145093in}}%
\pgfpathlineto{\pgfqpoint{1.373098in}{1.082326in}}%
\pgfpathlineto{\pgfqpoint{1.363753in}{1.096525in}}%
\pgfpathclose%
\pgfusepath{fill}%
\end{pgfscope}%
\begin{pgfscope}%
\pgfpathrectangle{\pgfqpoint{0.000000in}{0.000000in}}{\pgfqpoint{3.000000in}{3.000000in}}%
\pgfusepath{clip}%
\pgfsetbuttcap%
\pgfsetroundjoin%
\definecolor{currentfill}{rgb}{0.085389,1.000000,0.882353}%
\pgfsetfillcolor{currentfill}%
\pgfsetlinewidth{0.000000pt}%
\definecolor{currentstroke}{rgb}{0.000000,0.000000,0.000000}%
\pgfsetstrokecolor{currentstroke}%
\pgfsetdash{}{0pt}%
\pgfpathmoveto{\pgfqpoint{1.295418in}{1.492742in}}%
\pgfpathlineto{\pgfqpoint{1.280793in}{1.531723in}}%
\pgfpathlineto{\pgfqpoint{1.259894in}{1.510414in}}%
\pgfpathlineto{\pgfqpoint{1.275731in}{1.472552in}}%
\pgfpathlineto{\pgfqpoint{1.295418in}{1.492742in}}%
\pgfpathclose%
\pgfusepath{fill}%
\end{pgfscope}%
\begin{pgfscope}%
\pgfpathrectangle{\pgfqpoint{0.000000in}{0.000000in}}{\pgfqpoint{3.000000in}{3.000000in}}%
\pgfusepath{clip}%
\pgfsetbuttcap%
\pgfsetroundjoin%
\definecolor{currentfill}{rgb}{0.199241,1.000000,0.768501}%
\pgfsetfillcolor{currentfill}%
\pgfsetlinewidth{0.000000pt}%
\definecolor{currentstroke}{rgb}{0.000000,0.000000,0.000000}%
\pgfsetstrokecolor{currentstroke}%
\pgfsetdash{}{0pt}%
\pgfpathmoveto{\pgfqpoint{1.814951in}{1.517702in}}%
\pgfpathlineto{\pgfqpoint{1.830425in}{1.553460in}}%
\pgfpathlineto{\pgfqpoint{1.806045in}{1.575259in}}%
\pgfpathlineto{\pgfqpoint{1.791904in}{1.538419in}}%
\pgfpathlineto{\pgfqpoint{1.814951in}{1.517702in}}%
\pgfpathclose%
\pgfusepath{fill}%
\end{pgfscope}%
\begin{pgfscope}%
\pgfpathrectangle{\pgfqpoint{0.000000in}{0.000000in}}{\pgfqpoint{3.000000in}{3.000000in}}%
\pgfusepath{clip}%
\pgfsetbuttcap%
\pgfsetroundjoin%
\definecolor{currentfill}{rgb}{0.000000,0.676471,1.000000}%
\pgfsetfillcolor{currentfill}%
\pgfsetlinewidth{0.000000pt}%
\definecolor{currentstroke}{rgb}{0.000000,0.000000,0.000000}%
\pgfsetstrokecolor{currentstroke}%
\pgfsetdash{}{0pt}%
\pgfpathmoveto{\pgfqpoint{1.307396in}{1.387419in}}%
\pgfpathlineto{\pgfqpoint{1.291564in}{1.431748in}}%
\pgfpathlineto{\pgfqpoint{1.279119in}{1.411408in}}%
\pgfpathlineto{\pgfqpoint{1.295775in}{1.368286in}}%
\pgfpathlineto{\pgfqpoint{1.307396in}{1.387419in}}%
\pgfpathclose%
\pgfusepath{fill}%
\end{pgfscope}%
\begin{pgfscope}%
\pgfpathrectangle{\pgfqpoint{0.000000in}{0.000000in}}{\pgfqpoint{3.000000in}{3.000000in}}%
\pgfusepath{clip}%
\pgfsetbuttcap%
\pgfsetroundjoin%
\definecolor{currentfill}{rgb}{0.000000,0.503922,1.000000}%
\pgfsetfillcolor{currentfill}%
\pgfsetlinewidth{0.000000pt}%
\definecolor{currentstroke}{rgb}{0.000000,0.000000,0.000000}%
\pgfsetstrokecolor{currentstroke}%
\pgfsetdash{}{0pt}%
\pgfpathmoveto{\pgfqpoint{1.772775in}{1.308533in}}%
\pgfpathlineto{\pgfqpoint{1.789748in}{1.355095in}}%
\pgfpathlineto{\pgfqpoint{1.782084in}{1.374768in}}%
\pgfpathlineto{\pgfqpoint{1.765664in}{1.326959in}}%
\pgfpathlineto{\pgfqpoint{1.772775in}{1.308533in}}%
\pgfpathclose%
\pgfusepath{fill}%
\end{pgfscope}%
\begin{pgfscope}%
\pgfpathrectangle{\pgfqpoint{0.000000in}{0.000000in}}{\pgfqpoint{3.000000in}{3.000000in}}%
\pgfusepath{clip}%
\pgfsetbuttcap%
\pgfsetroundjoin%
\definecolor{currentfill}{rgb}{0.667299,1.000000,0.300443}%
\pgfsetfillcolor{currentfill}%
\pgfsetlinewidth{0.000000pt}%
\definecolor{currentstroke}{rgb}{0.000000,0.000000,0.000000}%
\pgfsetstrokecolor{currentstroke}%
\pgfsetdash{}{0pt}%
\pgfpathmoveto{\pgfqpoint{1.862581in}{1.704000in}}%
\pgfpathlineto{\pgfqpoint{1.876707in}{1.732639in}}%
\pgfpathlineto{\pgfqpoint{1.837625in}{1.757013in}}%
\pgfpathlineto{\pgfqpoint{1.825173in}{1.727424in}}%
\pgfpathlineto{\pgfqpoint{1.862581in}{1.704000in}}%
\pgfpathclose%
\pgfusepath{fill}%
\end{pgfscope}%
\begin{pgfscope}%
\pgfpathrectangle{\pgfqpoint{0.000000in}{0.000000in}}{\pgfqpoint{3.000000in}{3.000000in}}%
\pgfusepath{clip}%
\pgfsetbuttcap%
\pgfsetroundjoin%
\definecolor{currentfill}{rgb}{0.000000,0.064706,1.000000}%
\pgfsetfillcolor{currentfill}%
\pgfsetlinewidth{0.000000pt}%
\definecolor{currentstroke}{rgb}{0.000000,0.000000,0.000000}%
\pgfsetstrokecolor{currentstroke}%
\pgfsetdash{}{0pt}%
\pgfpathmoveto{\pgfqpoint{1.736246in}{1.165824in}}%
\pgfpathlineto{\pgfqpoint{1.753036in}{1.221728in}}%
\pgfpathlineto{\pgfqpoint{1.757095in}{1.239171in}}%
\pgfpathlineto{\pgfqpoint{1.740013in}{1.181983in}}%
\pgfpathlineto{\pgfqpoint{1.736246in}{1.165824in}}%
\pgfpathclose%
\pgfusepath{fill}%
\end{pgfscope}%
\begin{pgfscope}%
\pgfpathrectangle{\pgfqpoint{0.000000in}{0.000000in}}{\pgfqpoint{3.000000in}{3.000000in}}%
\pgfusepath{clip}%
\pgfsetbuttcap%
\pgfsetroundjoin%
\definecolor{currentfill}{rgb}{0.490196,1.000000,0.477546}%
\pgfsetfillcolor{currentfill}%
\pgfsetlinewidth{0.000000pt}%
\definecolor{currentstroke}{rgb}{0.000000,0.000000,0.000000}%
\pgfsetstrokecolor{currentstroke}%
\pgfsetdash{}{0pt}%
\pgfpathmoveto{\pgfqpoint{1.268744in}{1.657492in}}%
\pgfpathlineto{\pgfqpoint{1.255691in}{1.689421in}}%
\pgfpathlineto{\pgfqpoint{1.222308in}{1.666038in}}%
\pgfpathlineto{\pgfqpoint{1.236926in}{1.635106in}}%
\pgfpathlineto{\pgfqpoint{1.268744in}{1.657492in}}%
\pgfpathclose%
\pgfusepath{fill}%
\end{pgfscope}%
\begin{pgfscope}%
\pgfpathrectangle{\pgfqpoint{0.000000in}{0.000000in}}{\pgfqpoint{3.000000in}{3.000000in}}%
\pgfusepath{clip}%
\pgfsetbuttcap%
\pgfsetroundjoin%
\definecolor{currentfill}{rgb}{0.000000,0.000000,0.500000}%
\pgfsetfillcolor{currentfill}%
\pgfsetlinewidth{0.000000pt}%
\definecolor{currentstroke}{rgb}{0.000000,0.000000,0.000000}%
\pgfsetstrokecolor{currentstroke}%
\pgfsetdash{}{0pt}%
\pgfpathmoveto{\pgfqpoint{1.652015in}{0.978722in}}%
\pgfpathlineto{\pgfqpoint{1.663590in}{1.049881in}}%
\pgfpathlineto{\pgfqpoint{1.683294in}{1.060883in}}%
\pgfpathlineto{\pgfqpoint{1.669879in}{0.988763in}}%
\pgfpathlineto{\pgfqpoint{1.652015in}{0.978722in}}%
\pgfpathclose%
\pgfusepath{fill}%
\end{pgfscope}%
\begin{pgfscope}%
\pgfpathrectangle{\pgfqpoint{0.000000in}{0.000000in}}{\pgfqpoint{3.000000in}{3.000000in}}%
\pgfusepath{clip}%
\pgfsetbuttcap%
\pgfsetroundjoin%
\definecolor{currentfill}{rgb}{0.000000,0.300000,1.000000}%
\pgfsetfillcolor{currentfill}%
\pgfsetlinewidth{0.000000pt}%
\definecolor{currentstroke}{rgb}{0.000000,0.000000,0.000000}%
\pgfsetstrokecolor{currentstroke}%
\pgfsetdash{}{0pt}%
\pgfpathmoveto{\pgfqpoint{1.757095in}{1.239171in}}%
\pgfpathlineto{\pgfqpoint{1.774192in}{1.289728in}}%
\pgfpathlineto{\pgfqpoint{1.772775in}{1.308533in}}%
\pgfpathlineto{\pgfqpoint{1.755813in}{1.256695in}}%
\pgfpathlineto{\pgfqpoint{1.757095in}{1.239171in}}%
\pgfpathclose%
\pgfusepath{fill}%
\end{pgfscope}%
\begin{pgfscope}%
\pgfpathrectangle{\pgfqpoint{0.000000in}{0.000000in}}{\pgfqpoint{3.000000in}{3.000000in}}%
\pgfusepath{clip}%
\pgfsetbuttcap%
\pgfsetroundjoin%
\definecolor{currentfill}{rgb}{0.743201,1.000000,0.224541}%
\pgfsetfillcolor{currentfill}%
\pgfsetlinewidth{0.000000pt}%
\definecolor{currentstroke}{rgb}{0.000000,0.000000,0.000000}%
\pgfsetstrokecolor{currentstroke}%
\pgfsetdash{}{0pt}%
\pgfpathmoveto{\pgfqpoint{1.876707in}{1.732639in}}%
\pgfpathlineto{\pgfqpoint{1.890829in}{1.760180in}}%
\pgfpathlineto{\pgfqpoint{1.850071in}{1.785498in}}%
\pgfpathlineto{\pgfqpoint{1.837625in}{1.757013in}}%
\pgfpathlineto{\pgfqpoint{1.876707in}{1.732639in}}%
\pgfpathclose%
\pgfusepath{fill}%
\end{pgfscope}%
\begin{pgfscope}%
\pgfpathrectangle{\pgfqpoint{0.000000in}{0.000000in}}{\pgfqpoint{3.000000in}{3.000000in}}%
\pgfusepath{clip}%
\pgfsetbuttcap%
\pgfsetroundjoin%
\definecolor{currentfill}{rgb}{0.000000,0.849020,1.000000}%
\pgfsetfillcolor{currentfill}%
\pgfsetlinewidth{0.000000pt}%
\definecolor{currentstroke}{rgb}{0.000000,0.000000,0.000000}%
\pgfsetstrokecolor{currentstroke}%
\pgfsetdash{}{0pt}%
\pgfpathmoveto{\pgfqpoint{1.798509in}{1.418300in}}%
\pgfpathlineto{\pgfqpoint{1.814939in}{1.458310in}}%
\pgfpathlineto{\pgfqpoint{1.799478in}{1.479457in}}%
\pgfpathlineto{\pgfqpoint{1.784006in}{1.438268in}}%
\pgfpathlineto{\pgfqpoint{1.798509in}{1.418300in}}%
\pgfpathclose%
\pgfusepath{fill}%
\end{pgfscope}%
\begin{pgfscope}%
\pgfpathrectangle{\pgfqpoint{0.000000in}{0.000000in}}{\pgfqpoint{3.000000in}{3.000000in}}%
\pgfusepath{clip}%
\pgfsetbuttcap%
\pgfsetroundjoin%
\definecolor{currentfill}{rgb}{0.578748,1.000000,0.388994}%
\pgfsetfillcolor{currentfill}%
\pgfsetlinewidth{0.000000pt}%
\definecolor{currentstroke}{rgb}{0.000000,0.000000,0.000000}%
\pgfsetstrokecolor{currentstroke}%
\pgfsetdash{}{0pt}%
\pgfpathmoveto{\pgfqpoint{1.255691in}{1.689421in}}%
\pgfpathlineto{\pgfqpoint{1.242642in}{1.719946in}}%
\pgfpathlineto{\pgfqpoint{1.207692in}{1.695571in}}%
\pgfpathlineto{\pgfqpoint{1.222308in}{1.666038in}}%
\pgfpathlineto{\pgfqpoint{1.255691in}{1.689421in}}%
\pgfpathclose%
\pgfusepath{fill}%
\end{pgfscope}%
\begin{pgfscope}%
\pgfpathrectangle{\pgfqpoint{0.000000in}{0.000000in}}{\pgfqpoint{3.000000in}{3.000000in}}%
\pgfusepath{clip}%
\pgfsetbuttcap%
\pgfsetroundjoin%
\definecolor{currentfill}{rgb}{0.000000,0.000000,0.500000}%
\pgfsetfillcolor{currentfill}%
\pgfsetlinewidth{0.000000pt}%
\definecolor{currentstroke}{rgb}{0.000000,0.000000,0.000000}%
\pgfsetstrokecolor{currentstroke}%
\pgfsetdash{}{0pt}%
\pgfpathmoveto{\pgfqpoint{1.416818in}{0.985253in}}%
\pgfpathlineto{\pgfqpoint{1.403980in}{1.057038in}}%
\pgfpathlineto{\pgfqpoint{1.424774in}{1.046590in}}%
\pgfpathlineto{\pgfqpoint{1.435667in}{0.975719in}}%
\pgfpathlineto{\pgfqpoint{1.416818in}{0.985253in}}%
\pgfpathclose%
\pgfusepath{fill}%
\end{pgfscope}%
\begin{pgfscope}%
\pgfpathrectangle{\pgfqpoint{0.000000in}{0.000000in}}{\pgfqpoint{3.000000in}{3.000000in}}%
\pgfusepath{clip}%
\pgfsetbuttcap%
\pgfsetroundjoin%
\definecolor{currentfill}{rgb}{0.819102,1.000000,0.148640}%
\pgfsetfillcolor{currentfill}%
\pgfsetlinewidth{0.000000pt}%
\definecolor{currentstroke}{rgb}{0.000000,0.000000,0.000000}%
\pgfsetstrokecolor{currentstroke}%
\pgfsetdash{}{0pt}%
\pgfpathmoveto{\pgfqpoint{1.890829in}{1.760180in}}%
\pgfpathlineto{\pgfqpoint{1.904948in}{1.786739in}}%
\pgfpathlineto{\pgfqpoint{1.862512in}{1.812998in}}%
\pgfpathlineto{\pgfqpoint{1.850071in}{1.785498in}}%
\pgfpathlineto{\pgfqpoint{1.890829in}{1.760180in}}%
\pgfpathclose%
\pgfusepath{fill}%
\end{pgfscope}%
\begin{pgfscope}%
\pgfpathrectangle{\pgfqpoint{0.000000in}{0.000000in}}{\pgfqpoint{3.000000in}{3.000000in}}%
\pgfusepath{clip}%
\pgfsetbuttcap%
\pgfsetroundjoin%
\definecolor{currentfill}{rgb}{0.300443,1.000000,0.667299}%
\pgfsetfillcolor{currentfill}%
\pgfsetlinewidth{0.000000pt}%
\definecolor{currentstroke}{rgb}{0.000000,0.000000,0.000000}%
\pgfsetstrokecolor{currentstroke}%
\pgfsetdash{}{0pt}%
\pgfpathmoveto{\pgfqpoint{1.830425in}{1.553460in}}%
\pgfpathlineto{\pgfqpoint{1.845899in}{1.587094in}}%
\pgfpathlineto{\pgfqpoint{1.820183in}{1.609970in}}%
\pgfpathlineto{\pgfqpoint{1.806045in}{1.575259in}}%
\pgfpathlineto{\pgfqpoint{1.830425in}{1.553460in}}%
\pgfpathclose%
\pgfusepath{fill}%
\end{pgfscope}%
\begin{pgfscope}%
\pgfpathrectangle{\pgfqpoint{0.000000in}{0.000000in}}{\pgfqpoint{3.000000in}{3.000000in}}%
\pgfusepath{clip}%
\pgfsetbuttcap%
\pgfsetroundjoin%
\definecolor{currentfill}{rgb}{0.000000,0.503922,1.000000}%
\pgfsetfillcolor{currentfill}%
\pgfsetlinewidth{0.000000pt}%
\definecolor{currentstroke}{rgb}{0.000000,0.000000,0.000000}%
\pgfsetstrokecolor{currentstroke}%
\pgfsetdash{}{0pt}%
\pgfpathmoveto{\pgfqpoint{1.312424in}{1.320887in}}%
\pgfpathlineto{\pgfqpoint{1.295775in}{1.368286in}}%
\pgfpathlineto{\pgfqpoint{1.290129in}{1.348422in}}%
\pgfpathlineto{\pgfqpoint{1.307195in}{1.302284in}}%
\pgfpathlineto{\pgfqpoint{1.312424in}{1.320887in}}%
\pgfpathclose%
\pgfusepath{fill}%
\end{pgfscope}%
\begin{pgfscope}%
\pgfpathrectangle{\pgfqpoint{0.000000in}{0.000000in}}{\pgfqpoint{3.000000in}{3.000000in}}%
\pgfusepath{clip}%
\pgfsetbuttcap%
\pgfsetroundjoin%
\definecolor{currentfill}{rgb}{0.000000,0.064706,1.000000}%
\pgfsetfillcolor{currentfill}%
\pgfsetlinewidth{0.000000pt}%
\definecolor{currentstroke}{rgb}{0.000000,0.000000,0.000000}%
\pgfsetstrokecolor{currentstroke}%
\pgfsetdash{}{0pt}%
\pgfpathmoveto{\pgfqpoint{1.341778in}{1.176568in}}%
\pgfpathlineto{\pgfqpoint{1.324747in}{1.233326in}}%
\pgfpathlineto{\pgfqpoint{1.330576in}{1.216007in}}%
\pgfpathlineto{\pgfqpoint{1.347175in}{1.160524in}}%
\pgfpathlineto{\pgfqpoint{1.341778in}{1.176568in}}%
\pgfpathclose%
\pgfusepath{fill}%
\end{pgfscope}%
\begin{pgfscope}%
\pgfpathrectangle{\pgfqpoint{0.000000in}{0.000000in}}{\pgfqpoint{3.000000in}{3.000000in}}%
\pgfusepath{clip}%
\pgfsetbuttcap%
\pgfsetroundjoin%
\definecolor{currentfill}{rgb}{0.199241,1.000000,0.768501}%
\pgfsetfillcolor{currentfill}%
\pgfsetlinewidth{0.000000pt}%
\definecolor{currentstroke}{rgb}{0.000000,0.000000,0.000000}%
\pgfsetstrokecolor{currentstroke}%
\pgfsetdash{}{0pt}%
\pgfpathmoveto{\pgfqpoint{1.280793in}{1.531723in}}%
\pgfpathlineto{\pgfqpoint{1.266169in}{1.568214in}}%
\pgfpathlineto{\pgfqpoint{1.244056in}{1.545791in}}%
\pgfpathlineto{\pgfqpoint{1.259894in}{1.510414in}}%
\pgfpathlineto{\pgfqpoint{1.280793in}{1.531723in}}%
\pgfpathclose%
\pgfusepath{fill}%
\end{pgfscope}%
\begin{pgfscope}%
\pgfpathrectangle{\pgfqpoint{0.000000in}{0.000000in}}{\pgfqpoint{3.000000in}{3.000000in}}%
\pgfusepath{clip}%
\pgfsetbuttcap%
\pgfsetroundjoin%
\definecolor{currentfill}{rgb}{0.895003,1.000000,0.072739}%
\pgfsetfillcolor{currentfill}%
\pgfsetlinewidth{0.000000pt}%
\definecolor{currentstroke}{rgb}{0.000000,0.000000,0.000000}%
\pgfsetstrokecolor{currentstroke}%
\pgfsetdash{}{0pt}%
\pgfpathmoveto{\pgfqpoint{1.904948in}{1.786739in}}%
\pgfpathlineto{\pgfqpoint{1.919064in}{1.812415in}}%
\pgfpathlineto{\pgfqpoint{1.874946in}{1.839612in}}%
\pgfpathlineto{\pgfqpoint{1.862512in}{1.812998in}}%
\pgfpathlineto{\pgfqpoint{1.904948in}{1.786739in}}%
\pgfpathclose%
\pgfusepath{fill}%
\end{pgfscope}%
\begin{pgfscope}%
\pgfpathrectangle{\pgfqpoint{0.000000in}{0.000000in}}{\pgfqpoint{3.000000in}{3.000000in}}%
\pgfusepath{clip}%
\pgfsetbuttcap%
\pgfsetroundjoin%
\definecolor{currentfill}{rgb}{0.000000,0.000000,0.838681}%
\pgfsetfillcolor{currentfill}%
\pgfsetlinewidth{0.000000pt}%
\definecolor{currentstroke}{rgb}{0.000000,0.000000,0.000000}%
\pgfsetstrokecolor{currentstroke}%
\pgfsetdash{}{0pt}%
\pgfpathmoveto{\pgfqpoint{1.699418in}{1.073353in}}%
\pgfpathlineto{\pgfqpoint{1.714355in}{1.135339in}}%
\pgfpathlineto{\pgfqpoint{1.727630in}{1.150145in}}%
\pgfpathlineto{\pgfqpoint{1.711574in}{1.086974in}}%
\pgfpathlineto{\pgfqpoint{1.699418in}{1.073353in}}%
\pgfpathclose%
\pgfusepath{fill}%
\end{pgfscope}%
\begin{pgfscope}%
\pgfpathrectangle{\pgfqpoint{0.000000in}{0.000000in}}{\pgfqpoint{3.000000in}{3.000000in}}%
\pgfusepath{clip}%
\pgfsetbuttcap%
\pgfsetroundjoin%
\definecolor{currentfill}{rgb}{0.667299,1.000000,0.300443}%
\pgfsetfillcolor{currentfill}%
\pgfsetlinewidth{0.000000pt}%
\definecolor{currentstroke}{rgb}{0.000000,0.000000,0.000000}%
\pgfsetstrokecolor{currentstroke}%
\pgfsetdash{}{0pt}%
\pgfpathmoveto{\pgfqpoint{1.242642in}{1.719946in}}%
\pgfpathlineto{\pgfqpoint{1.229599in}{1.749233in}}%
\pgfpathlineto{\pgfqpoint{1.193078in}{1.723868in}}%
\pgfpathlineto{\pgfqpoint{1.207692in}{1.695571in}}%
\pgfpathlineto{\pgfqpoint{1.242642in}{1.719946in}}%
\pgfpathclose%
\pgfusepath{fill}%
\end{pgfscope}%
\begin{pgfscope}%
\pgfpathrectangle{\pgfqpoint{0.000000in}{0.000000in}}{\pgfqpoint{3.000000in}{3.000000in}}%
\pgfusepath{clip}%
\pgfsetbuttcap%
\pgfsetroundjoin%
\definecolor{currentfill}{rgb}{0.000000,0.300000,1.000000}%
\pgfsetfillcolor{currentfill}%
\pgfsetlinewidth{0.000000pt}%
\definecolor{currentstroke}{rgb}{0.000000,0.000000,0.000000}%
\pgfsetstrokecolor{currentstroke}%
\pgfsetdash{}{0pt}%
\pgfpathmoveto{\pgfqpoint{1.324249in}{1.250871in}}%
\pgfpathlineto{\pgfqpoint{1.307195in}{1.302284in}}%
\pgfpathlineto{\pgfqpoint{1.307699in}{1.283454in}}%
\pgfpathlineto{\pgfqpoint{1.324747in}{1.233326in}}%
\pgfpathlineto{\pgfqpoint{1.324249in}{1.250871in}}%
\pgfpathclose%
\pgfusepath{fill}%
\end{pgfscope}%
\begin{pgfscope}%
\pgfpathrectangle{\pgfqpoint{0.000000in}{0.000000in}}{\pgfqpoint{3.000000in}{3.000000in}}%
\pgfusepath{clip}%
\pgfsetbuttcap%
\pgfsetroundjoin%
\definecolor{currentfill}{rgb}{0.958254,0.973856,0.009488}%
\pgfsetfillcolor{currentfill}%
\pgfsetlinewidth{0.000000pt}%
\definecolor{currentstroke}{rgb}{0.000000,0.000000,0.000000}%
\pgfsetstrokecolor{currentstroke}%
\pgfsetdash{}{0pt}%
\pgfpathmoveto{\pgfqpoint{1.919064in}{1.812415in}}%
\pgfpathlineto{\pgfqpoint{1.933176in}{1.837295in}}%
\pgfpathlineto{\pgfqpoint{1.887374in}{1.865425in}}%
\pgfpathlineto{\pgfqpoint{1.874946in}{1.839612in}}%
\pgfpathlineto{\pgfqpoint{1.919064in}{1.812415in}}%
\pgfpathclose%
\pgfusepath{fill}%
\end{pgfscope}%
\begin{pgfscope}%
\pgfpathrectangle{\pgfqpoint{0.000000in}{0.000000in}}{\pgfqpoint{3.000000in}{3.000000in}}%
\pgfusepath{clip}%
\pgfsetbuttcap%
\pgfsetroundjoin%
\definecolor{currentfill}{rgb}{0.500000,0.000000,0.000000}%
\pgfsetfillcolor{currentfill}%
\pgfsetlinewidth{0.000000pt}%
\definecolor{currentstroke}{rgb}{0.000000,0.000000,0.000000}%
\pgfsetstrokecolor{currentstroke}%
\pgfsetdash{}{0pt}%
\pgfpathmoveto{\pgfqpoint{2.228493in}{2.254972in}}%
\pgfpathlineto{\pgfqpoint{2.242501in}{2.271899in}}%
\pgfpathlineto{\pgfqpoint{2.159104in}{2.319662in}}%
\pgfpathlineto{\pgfqpoint{2.146828in}{2.301880in}}%
\pgfpathlineto{\pgfqpoint{2.228493in}{2.254972in}}%
\pgfpathclose%
\pgfusepath{fill}%
\end{pgfscope}%
\begin{pgfscope}%
\pgfpathrectangle{\pgfqpoint{0.000000in}{0.000000in}}{\pgfqpoint{3.000000in}{3.000000in}}%
\pgfusepath{clip}%
\pgfsetbuttcap%
\pgfsetroundjoin%
\definecolor{currentfill}{rgb}{0.000000,0.849020,1.000000}%
\pgfsetfillcolor{currentfill}%
\pgfsetlinewidth{0.000000pt}%
\definecolor{currentstroke}{rgb}{0.000000,0.000000,0.000000}%
\pgfsetstrokecolor{currentstroke}%
\pgfsetdash{}{0pt}%
\pgfpathmoveto{\pgfqpoint{1.291564in}{1.431748in}}%
\pgfpathlineto{\pgfqpoint{1.275731in}{1.472552in}}%
\pgfpathlineto{\pgfqpoint{1.262455in}{1.451010in}}%
\pgfpathlineto{\pgfqpoint{1.279119in}{1.411408in}}%
\pgfpathlineto{\pgfqpoint{1.291564in}{1.431748in}}%
\pgfpathclose%
\pgfusepath{fill}%
\end{pgfscope}%
\begin{pgfscope}%
\pgfpathrectangle{\pgfqpoint{0.000000in}{0.000000in}}{\pgfqpoint{3.000000in}{3.000000in}}%
\pgfusepath{clip}%
\pgfsetbuttcap%
\pgfsetroundjoin%
\definecolor{currentfill}{rgb}{1.000000,0.886710,0.000000}%
\pgfsetfillcolor{currentfill}%
\pgfsetlinewidth{0.000000pt}%
\definecolor{currentstroke}{rgb}{0.000000,0.000000,0.000000}%
\pgfsetstrokecolor{currentstroke}%
\pgfsetdash{}{0pt}%
\pgfpathmoveto{\pgfqpoint{1.933176in}{1.837295in}}%
\pgfpathlineto{\pgfqpoint{1.947283in}{1.861453in}}%
\pgfpathlineto{\pgfqpoint{1.899796in}{1.890512in}}%
\pgfpathlineto{\pgfqpoint{1.887374in}{1.865425in}}%
\pgfpathlineto{\pgfqpoint{1.933176in}{1.837295in}}%
\pgfpathclose%
\pgfusepath{fill}%
\end{pgfscope}%
\begin{pgfscope}%
\pgfpathrectangle{\pgfqpoint{0.000000in}{0.000000in}}{\pgfqpoint{3.000000in}{3.000000in}}%
\pgfusepath{clip}%
\pgfsetbuttcap%
\pgfsetroundjoin%
\definecolor{currentfill}{rgb}{0.553476,0.000000,0.000000}%
\pgfsetfillcolor{currentfill}%
\pgfsetlinewidth{0.000000pt}%
\definecolor{currentstroke}{rgb}{0.000000,0.000000,0.000000}%
\pgfsetstrokecolor{currentstroke}%
\pgfsetdash{}{0pt}%
\pgfpathmoveto{\pgfqpoint{2.214479in}{2.237871in}}%
\pgfpathlineto{\pgfqpoint{2.228493in}{2.254972in}}%
\pgfpathlineto{\pgfqpoint{2.146828in}{2.301880in}}%
\pgfpathlineto{\pgfqpoint{2.134543in}{2.283919in}}%
\pgfpathlineto{\pgfqpoint{2.214479in}{2.237871in}}%
\pgfpathclose%
\pgfusepath{fill}%
\end{pgfscope}%
\begin{pgfscope}%
\pgfpathrectangle{\pgfqpoint{0.000000in}{0.000000in}}{\pgfqpoint{3.000000in}{3.000000in}}%
\pgfusepath{clip}%
\pgfsetbuttcap%
\pgfsetroundjoin%
\definecolor{currentfill}{rgb}{0.000000,0.000000,0.500000}%
\pgfsetfillcolor{currentfill}%
\pgfsetlinewidth{0.000000pt}%
\definecolor{currentstroke}{rgb}{0.000000,0.000000,0.000000}%
\pgfsetstrokecolor{currentstroke}%
\pgfsetdash{}{0pt}%
\pgfpathmoveto{\pgfqpoint{1.631352in}{0.970281in}}%
\pgfpathlineto{\pgfqpoint{1.640791in}{1.040629in}}%
\pgfpathlineto{\pgfqpoint{1.663590in}{1.049881in}}%
\pgfpathlineto{\pgfqpoint{1.652015in}{0.978722in}}%
\pgfpathlineto{\pgfqpoint{1.631352in}{0.970281in}}%
\pgfpathclose%
\pgfusepath{fill}%
\end{pgfscope}%
\begin{pgfscope}%
\pgfpathrectangle{\pgfqpoint{0.000000in}{0.000000in}}{\pgfqpoint{3.000000in}{3.000000in}}%
\pgfusepath{clip}%
\pgfsetbuttcap%
\pgfsetroundjoin%
\definecolor{currentfill}{rgb}{0.743201,1.000000,0.224541}%
\pgfsetfillcolor{currentfill}%
\pgfsetlinewidth{0.000000pt}%
\definecolor{currentstroke}{rgb}{0.000000,0.000000,0.000000}%
\pgfsetstrokecolor{currentstroke}%
\pgfsetdash{}{0pt}%
\pgfpathmoveto{\pgfqpoint{1.229599in}{1.749233in}}%
\pgfpathlineto{\pgfqpoint{1.216561in}{1.777417in}}%
\pgfpathlineto{\pgfqpoint{1.178466in}{1.751068in}}%
\pgfpathlineto{\pgfqpoint{1.193078in}{1.723868in}}%
\pgfpathlineto{\pgfqpoint{1.229599in}{1.749233in}}%
\pgfpathclose%
\pgfusepath{fill}%
\end{pgfscope}%
\begin{pgfscope}%
\pgfpathrectangle{\pgfqpoint{0.000000in}{0.000000in}}{\pgfqpoint{3.000000in}{3.000000in}}%
\pgfusepath{clip}%
\pgfsetbuttcap%
\pgfsetroundjoin%
\definecolor{currentfill}{rgb}{0.401645,1.000000,0.566097}%
\pgfsetfillcolor{currentfill}%
\pgfsetlinewidth{0.000000pt}%
\definecolor{currentstroke}{rgb}{0.000000,0.000000,0.000000}%
\pgfsetstrokecolor{currentstroke}%
\pgfsetdash{}{0pt}%
\pgfpathmoveto{\pgfqpoint{1.845899in}{1.587094in}}%
\pgfpathlineto{\pgfqpoint{1.861375in}{1.618895in}}%
\pgfpathlineto{\pgfqpoint{1.834319in}{1.642845in}}%
\pgfpathlineto{\pgfqpoint{1.820183in}{1.609970in}}%
\pgfpathlineto{\pgfqpoint{1.845899in}{1.587094in}}%
\pgfpathclose%
\pgfusepath{fill}%
\end{pgfscope}%
\begin{pgfscope}%
\pgfpathrectangle{\pgfqpoint{0.000000in}{0.000000in}}{\pgfqpoint{3.000000in}{3.000000in}}%
\pgfusepath{clip}%
\pgfsetbuttcap%
\pgfsetroundjoin%
\definecolor{currentfill}{rgb}{0.000000,0.000000,0.838681}%
\pgfsetfillcolor{currentfill}%
\pgfsetlinewidth{0.000000pt}%
\definecolor{currentstroke}{rgb}{0.000000,0.000000,0.000000}%
\pgfsetstrokecolor{currentstroke}%
\pgfsetdash{}{0pt}%
\pgfpathmoveto{\pgfqpoint{1.373098in}{1.082326in}}%
\pgfpathlineto{\pgfqpoint{1.357371in}{1.145093in}}%
\pgfpathlineto{\pgfqpoint{1.372135in}{1.130663in}}%
\pgfpathlineto{\pgfqpoint{1.386614in}{1.069052in}}%
\pgfpathlineto{\pgfqpoint{1.373098in}{1.082326in}}%
\pgfpathclose%
\pgfusepath{fill}%
\end{pgfscope}%
\begin{pgfscope}%
\pgfpathrectangle{\pgfqpoint{0.000000in}{0.000000in}}{\pgfqpoint{3.000000in}{3.000000in}}%
\pgfusepath{clip}%
\pgfsetbuttcap%
\pgfsetroundjoin%
\definecolor{currentfill}{rgb}{0.606952,0.000000,0.000000}%
\pgfsetfillcolor{currentfill}%
\pgfsetlinewidth{0.000000pt}%
\definecolor{currentstroke}{rgb}{0.000000,0.000000,0.000000}%
\pgfsetstrokecolor{currentstroke}%
\pgfsetdash{}{0pt}%
\pgfpathmoveto{\pgfqpoint{2.200461in}{2.220585in}}%
\pgfpathlineto{\pgfqpoint{2.214479in}{2.237871in}}%
\pgfpathlineto{\pgfqpoint{2.134543in}{2.283919in}}%
\pgfpathlineto{\pgfqpoint{2.122252in}{2.265770in}}%
\pgfpathlineto{\pgfqpoint{2.200461in}{2.220585in}}%
\pgfpathclose%
\pgfusepath{fill}%
\end{pgfscope}%
\begin{pgfscope}%
\pgfpathrectangle{\pgfqpoint{0.000000in}{0.000000in}}{\pgfqpoint{3.000000in}{3.000000in}}%
\pgfusepath{clip}%
\pgfsetbuttcap%
\pgfsetroundjoin%
\definecolor{currentfill}{rgb}{1.000000,0.814089,0.000000}%
\pgfsetfillcolor{currentfill}%
\pgfsetlinewidth{0.000000pt}%
\definecolor{currentstroke}{rgb}{0.000000,0.000000,0.000000}%
\pgfsetstrokecolor{currentstroke}%
\pgfsetdash{}{0pt}%
\pgfpathmoveto{\pgfqpoint{1.947283in}{1.861453in}}%
\pgfpathlineto{\pgfqpoint{1.961387in}{1.884952in}}%
\pgfpathlineto{\pgfqpoint{1.912212in}{1.914938in}}%
\pgfpathlineto{\pgfqpoint{1.899796in}{1.890512in}}%
\pgfpathlineto{\pgfqpoint{1.947283in}{1.861453in}}%
\pgfpathclose%
\pgfusepath{fill}%
\end{pgfscope}%
\begin{pgfscope}%
\pgfpathrectangle{\pgfqpoint{0.000000in}{0.000000in}}{\pgfqpoint{3.000000in}{3.000000in}}%
\pgfusepath{clip}%
\pgfsetbuttcap%
\pgfsetroundjoin%
\definecolor{currentfill}{rgb}{0.678253,0.000000,0.000000}%
\pgfsetfillcolor{currentfill}%
\pgfsetlinewidth{0.000000pt}%
\definecolor{currentstroke}{rgb}{0.000000,0.000000,0.000000}%
\pgfsetstrokecolor{currentstroke}%
\pgfsetdash{}{0pt}%
\pgfpathmoveto{\pgfqpoint{2.186437in}{2.203105in}}%
\pgfpathlineto{\pgfqpoint{2.200461in}{2.220585in}}%
\pgfpathlineto{\pgfqpoint{2.122252in}{2.265770in}}%
\pgfpathlineto{\pgfqpoint{2.109953in}{2.247424in}}%
\pgfpathlineto{\pgfqpoint{2.186437in}{2.203105in}}%
\pgfpathclose%
\pgfusepath{fill}%
\end{pgfscope}%
\begin{pgfscope}%
\pgfpathrectangle{\pgfqpoint{0.000000in}{0.000000in}}{\pgfqpoint{3.000000in}{3.000000in}}%
\pgfusepath{clip}%
\pgfsetbuttcap%
\pgfsetroundjoin%
\definecolor{currentfill}{rgb}{0.000000,0.676471,1.000000}%
\pgfsetfillcolor{currentfill}%
\pgfsetlinewidth{0.000000pt}%
\definecolor{currentstroke}{rgb}{0.000000,0.000000,0.000000}%
\pgfsetstrokecolor{currentstroke}%
\pgfsetdash{}{0pt}%
\pgfpathmoveto{\pgfqpoint{1.789748in}{1.355095in}}%
\pgfpathlineto{\pgfqpoint{1.806732in}{1.397382in}}%
\pgfpathlineto{\pgfqpoint{1.798509in}{1.418300in}}%
\pgfpathlineto{\pgfqpoint{1.782084in}{1.374768in}}%
\pgfpathlineto{\pgfqpoint{1.789748in}{1.355095in}}%
\pgfpathclose%
\pgfusepath{fill}%
\end{pgfscope}%
\begin{pgfscope}%
\pgfpathrectangle{\pgfqpoint{0.000000in}{0.000000in}}{\pgfqpoint{3.000000in}{3.000000in}}%
\pgfusepath{clip}%
\pgfsetbuttcap%
\pgfsetroundjoin%
\definecolor{currentfill}{rgb}{0.085389,1.000000,0.882353}%
\pgfsetfillcolor{currentfill}%
\pgfsetlinewidth{0.000000pt}%
\definecolor{currentstroke}{rgb}{0.000000,0.000000,0.000000}%
\pgfsetstrokecolor{currentstroke}%
\pgfsetdash{}{0pt}%
\pgfpathmoveto{\pgfqpoint{1.814939in}{1.458310in}}%
\pgfpathlineto{\pgfqpoint{1.831375in}{1.495379in}}%
\pgfpathlineto{\pgfqpoint{1.814951in}{1.517702in}}%
\pgfpathlineto{\pgfqpoint{1.799478in}{1.479457in}}%
\pgfpathlineto{\pgfqpoint{1.814939in}{1.458310in}}%
\pgfpathclose%
\pgfusepath{fill}%
\end{pgfscope}%
\begin{pgfscope}%
\pgfpathrectangle{\pgfqpoint{0.000000in}{0.000000in}}{\pgfqpoint{3.000000in}{3.000000in}}%
\pgfusepath{clip}%
\pgfsetbuttcap%
\pgfsetroundjoin%
\definecolor{currentfill}{rgb}{1.000000,0.741467,0.000000}%
\pgfsetfillcolor{currentfill}%
\pgfsetlinewidth{0.000000pt}%
\definecolor{currentstroke}{rgb}{0.000000,0.000000,0.000000}%
\pgfsetstrokecolor{currentstroke}%
\pgfsetdash{}{0pt}%
\pgfpathmoveto{\pgfqpoint{1.961387in}{1.884952in}}%
\pgfpathlineto{\pgfqpoint{1.975487in}{1.907851in}}%
\pgfpathlineto{\pgfqpoint{1.924622in}{1.938760in}}%
\pgfpathlineto{\pgfqpoint{1.912212in}{1.914938in}}%
\pgfpathlineto{\pgfqpoint{1.961387in}{1.884952in}}%
\pgfpathclose%
\pgfusepath{fill}%
\end{pgfscope}%
\begin{pgfscope}%
\pgfpathrectangle{\pgfqpoint{0.000000in}{0.000000in}}{\pgfqpoint{3.000000in}{3.000000in}}%
\pgfusepath{clip}%
\pgfsetbuttcap%
\pgfsetroundjoin%
\definecolor{currentfill}{rgb}{0.731729,0.000000,0.000000}%
\pgfsetfillcolor{currentfill}%
\pgfsetlinewidth{0.000000pt}%
\definecolor{currentstroke}{rgb}{0.000000,0.000000,0.000000}%
\pgfsetstrokecolor{currentstroke}%
\pgfsetdash{}{0pt}%
\pgfpathmoveto{\pgfqpoint{2.172408in}{2.185422in}}%
\pgfpathlineto{\pgfqpoint{2.186437in}{2.203105in}}%
\pgfpathlineto{\pgfqpoint{2.109953in}{2.247424in}}%
\pgfpathlineto{\pgfqpoint{2.097647in}{2.228871in}}%
\pgfpathlineto{\pgfqpoint{2.172408in}{2.185422in}}%
\pgfpathclose%
\pgfusepath{fill}%
\end{pgfscope}%
\begin{pgfscope}%
\pgfpathrectangle{\pgfqpoint{0.000000in}{0.000000in}}{\pgfqpoint{3.000000in}{3.000000in}}%
\pgfusepath{clip}%
\pgfsetbuttcap%
\pgfsetroundjoin%
\definecolor{currentfill}{rgb}{0.300443,1.000000,0.667299}%
\pgfsetfillcolor{currentfill}%
\pgfsetlinewidth{0.000000pt}%
\definecolor{currentstroke}{rgb}{0.000000,0.000000,0.000000}%
\pgfsetstrokecolor{currentstroke}%
\pgfsetdash{}{0pt}%
\pgfpathmoveto{\pgfqpoint{1.266169in}{1.568214in}}%
\pgfpathlineto{\pgfqpoint{1.251547in}{1.602577in}}%
\pgfpathlineto{\pgfqpoint{1.228215in}{1.579043in}}%
\pgfpathlineto{\pgfqpoint{1.244056in}{1.545791in}}%
\pgfpathlineto{\pgfqpoint{1.266169in}{1.568214in}}%
\pgfpathclose%
\pgfusepath{fill}%
\end{pgfscope}%
\begin{pgfscope}%
\pgfpathrectangle{\pgfqpoint{0.000000in}{0.000000in}}{\pgfqpoint{3.000000in}{3.000000in}}%
\pgfusepath{clip}%
\pgfsetbuttcap%
\pgfsetroundjoin%
\definecolor{currentfill}{rgb}{0.000000,0.000000,0.500000}%
\pgfsetfillcolor{currentfill}%
\pgfsetlinewidth{0.000000pt}%
\definecolor{currentstroke}{rgb}{0.000000,0.000000,0.000000}%
\pgfsetstrokecolor{currentstroke}%
\pgfsetdash{}{0pt}%
\pgfpathmoveto{\pgfqpoint{1.435667in}{0.975719in}}%
\pgfpathlineto{\pgfqpoint{1.424774in}{1.046590in}}%
\pgfpathlineto{\pgfqpoint{1.448482in}{1.037976in}}%
\pgfpathlineto{\pgfqpoint{1.457152in}{0.967861in}}%
\pgfpathlineto{\pgfqpoint{1.435667in}{0.975719in}}%
\pgfpathclose%
\pgfusepath{fill}%
\end{pgfscope}%
\begin{pgfscope}%
\pgfpathrectangle{\pgfqpoint{0.000000in}{0.000000in}}{\pgfqpoint{3.000000in}{3.000000in}}%
\pgfusepath{clip}%
\pgfsetbuttcap%
\pgfsetroundjoin%
\definecolor{currentfill}{rgb}{0.803030,0.000000,0.000000}%
\pgfsetfillcolor{currentfill}%
\pgfsetlinewidth{0.000000pt}%
\definecolor{currentstroke}{rgb}{0.000000,0.000000,0.000000}%
\pgfsetstrokecolor{currentstroke}%
\pgfsetdash{}{0pt}%
\pgfpathmoveto{\pgfqpoint{2.158374in}{2.167523in}}%
\pgfpathlineto{\pgfqpoint{2.172408in}{2.185422in}}%
\pgfpathlineto{\pgfqpoint{2.097647in}{2.228871in}}%
\pgfpathlineto{\pgfqpoint{2.085333in}{2.210099in}}%
\pgfpathlineto{\pgfqpoint{2.158374in}{2.167523in}}%
\pgfpathclose%
\pgfusepath{fill}%
\end{pgfscope}%
\begin{pgfscope}%
\pgfpathrectangle{\pgfqpoint{0.000000in}{0.000000in}}{\pgfqpoint{3.000000in}{3.000000in}}%
\pgfusepath{clip}%
\pgfsetbuttcap%
\pgfsetroundjoin%
\definecolor{currentfill}{rgb}{1.000000,0.668845,0.000000}%
\pgfsetfillcolor{currentfill}%
\pgfsetlinewidth{0.000000pt}%
\definecolor{currentstroke}{rgb}{0.000000,0.000000,0.000000}%
\pgfsetstrokecolor{currentstroke}%
\pgfsetdash{}{0pt}%
\pgfpathmoveto{\pgfqpoint{1.975487in}{1.907851in}}%
\pgfpathlineto{\pgfqpoint{1.989582in}{1.930198in}}%
\pgfpathlineto{\pgfqpoint{1.937025in}{1.962026in}}%
\pgfpathlineto{\pgfqpoint{1.924622in}{1.938760in}}%
\pgfpathlineto{\pgfqpoint{1.975487in}{1.907851in}}%
\pgfpathclose%
\pgfusepath{fill}%
\end{pgfscope}%
\begin{pgfscope}%
\pgfpathrectangle{\pgfqpoint{0.000000in}{0.000000in}}{\pgfqpoint{3.000000in}{3.000000in}}%
\pgfusepath{clip}%
\pgfsetbuttcap%
\pgfsetroundjoin%
\definecolor{currentfill}{rgb}{0.819102,1.000000,0.148640}%
\pgfsetfillcolor{currentfill}%
\pgfsetlinewidth{0.000000pt}%
\definecolor{currentstroke}{rgb}{0.000000,0.000000,0.000000}%
\pgfsetstrokecolor{currentstroke}%
\pgfsetdash{}{0pt}%
\pgfpathmoveto{\pgfqpoint{1.216561in}{1.777417in}}%
\pgfpathlineto{\pgfqpoint{1.203528in}{1.804617in}}%
\pgfpathlineto{\pgfqpoint{1.163857in}{1.777286in}}%
\pgfpathlineto{\pgfqpoint{1.178466in}{1.751068in}}%
\pgfpathlineto{\pgfqpoint{1.216561in}{1.777417in}}%
\pgfpathclose%
\pgfusepath{fill}%
\end{pgfscope}%
\begin{pgfscope}%
\pgfpathrectangle{\pgfqpoint{0.000000in}{0.000000in}}{\pgfqpoint{3.000000in}{3.000000in}}%
\pgfusepath{clip}%
\pgfsetbuttcap%
\pgfsetroundjoin%
\definecolor{currentfill}{rgb}{0.856506,0.000000,0.000000}%
\pgfsetfillcolor{currentfill}%
\pgfsetlinewidth{0.000000pt}%
\definecolor{currentstroke}{rgb}{0.000000,0.000000,0.000000}%
\pgfsetstrokecolor{currentstroke}%
\pgfsetdash{}{0pt}%
\pgfpathmoveto{\pgfqpoint{2.144334in}{2.149397in}}%
\pgfpathlineto{\pgfqpoint{2.158374in}{2.167523in}}%
\pgfpathlineto{\pgfqpoint{2.085333in}{2.210099in}}%
\pgfpathlineto{\pgfqpoint{2.073013in}{2.191096in}}%
\pgfpathlineto{\pgfqpoint{2.144334in}{2.149397in}}%
\pgfpathclose%
\pgfusepath{fill}%
\end{pgfscope}%
\begin{pgfscope}%
\pgfpathrectangle{\pgfqpoint{0.000000in}{0.000000in}}{\pgfqpoint{3.000000in}{3.000000in}}%
\pgfusepath{clip}%
\pgfsetbuttcap%
\pgfsetroundjoin%
\definecolor{currentfill}{rgb}{1.000000,0.610748,0.000000}%
\pgfsetfillcolor{currentfill}%
\pgfsetlinewidth{0.000000pt}%
\definecolor{currentstroke}{rgb}{0.000000,0.000000,0.000000}%
\pgfsetstrokecolor{currentstroke}%
\pgfsetdash{}{0pt}%
\pgfpathmoveto{\pgfqpoint{1.989582in}{1.930198in}}%
\pgfpathlineto{\pgfqpoint{2.003674in}{1.952038in}}%
\pgfpathlineto{\pgfqpoint{1.949421in}{1.984781in}}%
\pgfpathlineto{\pgfqpoint{1.937025in}{1.962026in}}%
\pgfpathlineto{\pgfqpoint{1.989582in}{1.930198in}}%
\pgfpathclose%
\pgfusepath{fill}%
\end{pgfscope}%
\begin{pgfscope}%
\pgfpathrectangle{\pgfqpoint{0.000000in}{0.000000in}}{\pgfqpoint{3.000000in}{3.000000in}}%
\pgfusepath{clip}%
\pgfsetbuttcap%
\pgfsetroundjoin%
\definecolor{currentfill}{rgb}{0.927807,0.015251,0.000000}%
\pgfsetfillcolor{currentfill}%
\pgfsetlinewidth{0.000000pt}%
\definecolor{currentstroke}{rgb}{0.000000,0.000000,0.000000}%
\pgfsetstrokecolor{currentstroke}%
\pgfsetdash{}{0pt}%
\pgfpathmoveto{\pgfqpoint{2.130290in}{2.131031in}}%
\pgfpathlineto{\pgfqpoint{2.144334in}{2.149397in}}%
\pgfpathlineto{\pgfqpoint{2.073013in}{2.191096in}}%
\pgfpathlineto{\pgfqpoint{2.060685in}{2.171850in}}%
\pgfpathlineto{\pgfqpoint{2.130290in}{2.131031in}}%
\pgfpathclose%
\pgfusepath{fill}%
\end{pgfscope}%
\begin{pgfscope}%
\pgfpathrectangle{\pgfqpoint{0.000000in}{0.000000in}}{\pgfqpoint{3.000000in}{3.000000in}}%
\pgfusepath{clip}%
\pgfsetbuttcap%
\pgfsetroundjoin%
\definecolor{currentfill}{rgb}{1.000000,0.538126,0.000000}%
\pgfsetfillcolor{currentfill}%
\pgfsetlinewidth{0.000000pt}%
\definecolor{currentstroke}{rgb}{0.000000,0.000000,0.000000}%
\pgfsetstrokecolor{currentstroke}%
\pgfsetdash{}{0pt}%
\pgfpathmoveto{\pgfqpoint{2.003674in}{1.952038in}}%
\pgfpathlineto{\pgfqpoint{2.017761in}{1.973410in}}%
\pgfpathlineto{\pgfqpoint{1.961811in}{2.007064in}}%
\pgfpathlineto{\pgfqpoint{1.949421in}{1.984781in}}%
\pgfpathlineto{\pgfqpoint{2.003674in}{1.952038in}}%
\pgfpathclose%
\pgfusepath{fill}%
\end{pgfscope}%
\begin{pgfscope}%
\pgfpathrectangle{\pgfqpoint{0.000000in}{0.000000in}}{\pgfqpoint{3.000000in}{3.000000in}}%
\pgfusepath{clip}%
\pgfsetbuttcap%
\pgfsetroundjoin%
\definecolor{currentfill}{rgb}{0.999109,0.073348,0.000000}%
\pgfsetfillcolor{currentfill}%
\pgfsetlinewidth{0.000000pt}%
\definecolor{currentstroke}{rgb}{0.000000,0.000000,0.000000}%
\pgfsetstrokecolor{currentstroke}%
\pgfsetdash{}{0pt}%
\pgfpathmoveto{\pgfqpoint{2.116241in}{2.112410in}}%
\pgfpathlineto{\pgfqpoint{2.130290in}{2.131031in}}%
\pgfpathlineto{\pgfqpoint{2.060685in}{2.171850in}}%
\pgfpathlineto{\pgfqpoint{2.048350in}{2.152346in}}%
\pgfpathlineto{\pgfqpoint{2.116241in}{2.112410in}}%
\pgfpathclose%
\pgfusepath{fill}%
\end{pgfscope}%
\begin{pgfscope}%
\pgfpathrectangle{\pgfqpoint{0.000000in}{0.000000in}}{\pgfqpoint{3.000000in}{3.000000in}}%
\pgfusepath{clip}%
\pgfsetbuttcap%
\pgfsetroundjoin%
\definecolor{currentfill}{rgb}{1.000000,0.480029,0.000000}%
\pgfsetfillcolor{currentfill}%
\pgfsetlinewidth{0.000000pt}%
\definecolor{currentstroke}{rgb}{0.000000,0.000000,0.000000}%
\pgfsetstrokecolor{currentstroke}%
\pgfsetdash{}{0pt}%
\pgfpathmoveto{\pgfqpoint{2.017761in}{1.973410in}}%
\pgfpathlineto{\pgfqpoint{2.031843in}{1.994348in}}%
\pgfpathlineto{\pgfqpoint{1.974194in}{2.028910in}}%
\pgfpathlineto{\pgfqpoint{1.961811in}{2.007064in}}%
\pgfpathlineto{\pgfqpoint{2.017761in}{1.973410in}}%
\pgfpathclose%
\pgfusepath{fill}%
\end{pgfscope}%
\begin{pgfscope}%
\pgfpathrectangle{\pgfqpoint{0.000000in}{0.000000in}}{\pgfqpoint{3.000000in}{3.000000in}}%
\pgfusepath{clip}%
\pgfsetbuttcap%
\pgfsetroundjoin%
\definecolor{currentfill}{rgb}{1.000000,0.116921,0.000000}%
\pgfsetfillcolor{currentfill}%
\pgfsetlinewidth{0.000000pt}%
\definecolor{currentstroke}{rgb}{0.000000,0.000000,0.000000}%
\pgfsetstrokecolor{currentstroke}%
\pgfsetdash{}{0pt}%
\pgfpathmoveto{\pgfqpoint{2.102186in}{2.093518in}}%
\pgfpathlineto{\pgfqpoint{2.116241in}{2.112410in}}%
\pgfpathlineto{\pgfqpoint{2.048350in}{2.152346in}}%
\pgfpathlineto{\pgfqpoint{2.036008in}{2.132568in}}%
\pgfpathlineto{\pgfqpoint{2.102186in}{2.093518in}}%
\pgfpathclose%
\pgfusepath{fill}%
\end{pgfscope}%
\begin{pgfscope}%
\pgfpathrectangle{\pgfqpoint{0.000000in}{0.000000in}}{\pgfqpoint{3.000000in}{3.000000in}}%
\pgfusepath{clip}%
\pgfsetbuttcap%
\pgfsetroundjoin%
\definecolor{currentfill}{rgb}{1.000000,0.407407,0.000000}%
\pgfsetfillcolor{currentfill}%
\pgfsetlinewidth{0.000000pt}%
\definecolor{currentstroke}{rgb}{0.000000,0.000000,0.000000}%
\pgfsetstrokecolor{currentstroke}%
\pgfsetdash{}{0pt}%
\pgfpathmoveto{\pgfqpoint{2.031843in}{1.994348in}}%
\pgfpathlineto{\pgfqpoint{2.045921in}{2.014883in}}%
\pgfpathlineto{\pgfqpoint{1.986571in}{2.050350in}}%
\pgfpathlineto{\pgfqpoint{1.974194in}{2.028910in}}%
\pgfpathlineto{\pgfqpoint{2.031843in}{1.994348in}}%
\pgfpathclose%
\pgfusepath{fill}%
\end{pgfscope}%
\begin{pgfscope}%
\pgfpathrectangle{\pgfqpoint{0.000000in}{0.000000in}}{\pgfqpoint{3.000000in}{3.000000in}}%
\pgfusepath{clip}%
\pgfsetbuttcap%
\pgfsetroundjoin%
\definecolor{currentfill}{rgb}{1.000000,0.175018,0.000000}%
\pgfsetfillcolor{currentfill}%
\pgfsetlinewidth{0.000000pt}%
\definecolor{currentstroke}{rgb}{0.000000,0.000000,0.000000}%
\pgfsetstrokecolor{currentstroke}%
\pgfsetdash{}{0pt}%
\pgfpathmoveto{\pgfqpoint{2.088127in}{2.074340in}}%
\pgfpathlineto{\pgfqpoint{2.102186in}{2.093518in}}%
\pgfpathlineto{\pgfqpoint{2.036008in}{2.132568in}}%
\pgfpathlineto{\pgfqpoint{2.023659in}{2.112499in}}%
\pgfpathlineto{\pgfqpoint{2.088127in}{2.074340in}}%
\pgfpathclose%
\pgfusepath{fill}%
\end{pgfscope}%
\begin{pgfscope}%
\pgfpathrectangle{\pgfqpoint{0.000000in}{0.000000in}}{\pgfqpoint{3.000000in}{3.000000in}}%
\pgfusepath{clip}%
\pgfsetbuttcap%
\pgfsetroundjoin%
\definecolor{currentfill}{rgb}{1.000000,0.233115,0.000000}%
\pgfsetfillcolor{currentfill}%
\pgfsetlinewidth{0.000000pt}%
\definecolor{currentstroke}{rgb}{0.000000,0.000000,0.000000}%
\pgfsetstrokecolor{currentstroke}%
\pgfsetdash{}{0pt}%
\pgfpathmoveto{\pgfqpoint{2.074063in}{2.054855in}}%
\pgfpathlineto{\pgfqpoint{2.088127in}{2.074340in}}%
\pgfpathlineto{\pgfqpoint{2.023659in}{2.112499in}}%
\pgfpathlineto{\pgfqpoint{2.011303in}{2.092120in}}%
\pgfpathlineto{\pgfqpoint{2.074063in}{2.054855in}}%
\pgfpathclose%
\pgfusepath{fill}%
\end{pgfscope}%
\begin{pgfscope}%
\pgfpathrectangle{\pgfqpoint{0.000000in}{0.000000in}}{\pgfqpoint{3.000000in}{3.000000in}}%
\pgfusepath{clip}%
\pgfsetbuttcap%
\pgfsetroundjoin%
\definecolor{currentfill}{rgb}{1.000000,0.349310,0.000000}%
\pgfsetfillcolor{currentfill}%
\pgfsetlinewidth{0.000000pt}%
\definecolor{currentstroke}{rgb}{0.000000,0.000000,0.000000}%
\pgfsetstrokecolor{currentstroke}%
\pgfsetdash{}{0pt}%
\pgfpathmoveto{\pgfqpoint{2.045921in}{2.014883in}}%
\pgfpathlineto{\pgfqpoint{2.059995in}{2.035044in}}%
\pgfpathlineto{\pgfqpoint{1.998940in}{2.071411in}}%
\pgfpathlineto{\pgfqpoint{1.986571in}{2.050350in}}%
\pgfpathlineto{\pgfqpoint{2.045921in}{2.014883in}}%
\pgfpathclose%
\pgfusepath{fill}%
\end{pgfscope}%
\begin{pgfscope}%
\pgfpathrectangle{\pgfqpoint{0.000000in}{0.000000in}}{\pgfqpoint{3.000000in}{3.000000in}}%
\pgfusepath{clip}%
\pgfsetbuttcap%
\pgfsetroundjoin%
\definecolor{currentfill}{rgb}{1.000000,0.291213,0.000000}%
\pgfsetfillcolor{currentfill}%
\pgfsetlinewidth{0.000000pt}%
\definecolor{currentstroke}{rgb}{0.000000,0.000000,0.000000}%
\pgfsetstrokecolor{currentstroke}%
\pgfsetdash{}{0pt}%
\pgfpathmoveto{\pgfqpoint{2.059995in}{2.035044in}}%
\pgfpathlineto{\pgfqpoint{2.074063in}{2.054855in}}%
\pgfpathlineto{\pgfqpoint{2.011303in}{2.092120in}}%
\pgfpathlineto{\pgfqpoint{1.998940in}{2.071411in}}%
\pgfpathlineto{\pgfqpoint{2.059995in}{2.035044in}}%
\pgfpathclose%
\pgfusepath{fill}%
\end{pgfscope}%
\begin{pgfscope}%
\pgfpathrectangle{\pgfqpoint{0.000000in}{0.000000in}}{\pgfqpoint{3.000000in}{3.000000in}}%
\pgfusepath{clip}%
\pgfsetbuttcap%
\pgfsetroundjoin%
\definecolor{currentfill}{rgb}{0.000000,0.064706,1.000000}%
\pgfsetfillcolor{currentfill}%
\pgfsetlinewidth{0.000000pt}%
\definecolor{currentstroke}{rgb}{0.000000,0.000000,0.000000}%
\pgfsetstrokecolor{currentstroke}%
\pgfsetdash{}{0pt}%
\pgfpathmoveto{\pgfqpoint{1.727630in}{1.150145in}}%
\pgfpathlineto{\pgfqpoint{1.743709in}{1.204800in}}%
\pgfpathlineto{\pgfqpoint{1.753036in}{1.221728in}}%
\pgfpathlineto{\pgfqpoint{1.736246in}{1.165824in}}%
\pgfpathlineto{\pgfqpoint{1.727630in}{1.150145in}}%
\pgfpathclose%
\pgfusepath{fill}%
\end{pgfscope}%
\begin{pgfscope}%
\pgfpathrectangle{\pgfqpoint{0.000000in}{0.000000in}}{\pgfqpoint{3.000000in}{3.000000in}}%
\pgfusepath{clip}%
\pgfsetbuttcap%
\pgfsetroundjoin%
\definecolor{currentfill}{rgb}{0.895003,1.000000,0.072739}%
\pgfsetfillcolor{currentfill}%
\pgfsetlinewidth{0.000000pt}%
\definecolor{currentstroke}{rgb}{0.000000,0.000000,0.000000}%
\pgfsetstrokecolor{currentstroke}%
\pgfsetdash{}{0pt}%
\pgfpathmoveto{\pgfqpoint{1.203528in}{1.804617in}}%
\pgfpathlineto{\pgfqpoint{1.190500in}{1.830933in}}%
\pgfpathlineto{\pgfqpoint{1.149250in}{1.802624in}}%
\pgfpathlineto{\pgfqpoint{1.163857in}{1.777286in}}%
\pgfpathlineto{\pgfqpoint{1.203528in}{1.804617in}}%
\pgfpathclose%
\pgfusepath{fill}%
\end{pgfscope}%
\begin{pgfscope}%
\pgfpathrectangle{\pgfqpoint{0.000000in}{0.000000in}}{\pgfqpoint{3.000000in}{3.000000in}}%
\pgfusepath{clip}%
\pgfsetbuttcap%
\pgfsetroundjoin%
\definecolor{currentfill}{rgb}{0.490196,1.000000,0.477546}%
\pgfsetfillcolor{currentfill}%
\pgfsetlinewidth{0.000000pt}%
\definecolor{currentstroke}{rgb}{0.000000,0.000000,0.000000}%
\pgfsetstrokecolor{currentstroke}%
\pgfsetdash{}{0pt}%
\pgfpathmoveto{\pgfqpoint{1.861375in}{1.618895in}}%
\pgfpathlineto{\pgfqpoint{1.876850in}{1.649102in}}%
\pgfpathlineto{\pgfqpoint{1.848451in}{1.674123in}}%
\pgfpathlineto{\pgfqpoint{1.834319in}{1.642845in}}%
\pgfpathlineto{\pgfqpoint{1.861375in}{1.618895in}}%
\pgfpathclose%
\pgfusepath{fill}%
\end{pgfscope}%
\begin{pgfscope}%
\pgfpathrectangle{\pgfqpoint{0.000000in}{0.000000in}}{\pgfqpoint{3.000000in}{3.000000in}}%
\pgfusepath{clip}%
\pgfsetbuttcap%
\pgfsetroundjoin%
\definecolor{currentfill}{rgb}{0.000000,0.000000,0.500000}%
\pgfsetfillcolor{currentfill}%
\pgfsetlinewidth{0.000000pt}%
\definecolor{currentstroke}{rgb}{0.000000,0.000000,0.000000}%
\pgfsetstrokecolor{currentstroke}%
\pgfsetdash{}{0pt}%
\pgfpathmoveto{\pgfqpoint{1.608404in}{0.963654in}}%
\pgfpathlineto{\pgfqpoint{1.615462in}{1.033364in}}%
\pgfpathlineto{\pgfqpoint{1.640791in}{1.040629in}}%
\pgfpathlineto{\pgfqpoint{1.631352in}{0.970281in}}%
\pgfpathlineto{\pgfqpoint{1.608404in}{0.963654in}}%
\pgfpathclose%
\pgfusepath{fill}%
\end{pgfscope}%
\begin{pgfscope}%
\pgfpathrectangle{\pgfqpoint{0.000000in}{0.000000in}}{\pgfqpoint{3.000000in}{3.000000in}}%
\pgfusepath{clip}%
\pgfsetbuttcap%
\pgfsetroundjoin%
\definecolor{currentfill}{rgb}{0.000000,0.503922,1.000000}%
\pgfsetfillcolor{currentfill}%
\pgfsetlinewidth{0.000000pt}%
\definecolor{currentstroke}{rgb}{0.000000,0.000000,0.000000}%
\pgfsetstrokecolor{currentstroke}%
\pgfsetdash{}{0pt}%
\pgfpathmoveto{\pgfqpoint{1.774192in}{1.289728in}}%
\pgfpathlineto{\pgfqpoint{1.791304in}{1.335010in}}%
\pgfpathlineto{\pgfqpoint{1.789748in}{1.355095in}}%
\pgfpathlineto{\pgfqpoint{1.772775in}{1.308533in}}%
\pgfpathlineto{\pgfqpoint{1.774192in}{1.289728in}}%
\pgfpathclose%
\pgfusepath{fill}%
\end{pgfscope}%
\begin{pgfscope}%
\pgfpathrectangle{\pgfqpoint{0.000000in}{0.000000in}}{\pgfqpoint{3.000000in}{3.000000in}}%
\pgfusepath{clip}%
\pgfsetbuttcap%
\pgfsetroundjoin%
\definecolor{currentfill}{rgb}{0.958254,0.973856,0.009488}%
\pgfsetfillcolor{currentfill}%
\pgfsetlinewidth{0.000000pt}%
\definecolor{currentstroke}{rgb}{0.000000,0.000000,0.000000}%
\pgfsetstrokecolor{currentstroke}%
\pgfsetdash{}{0pt}%
\pgfpathmoveto{\pgfqpoint{1.190500in}{1.830933in}}%
\pgfpathlineto{\pgfqpoint{1.177478in}{1.856449in}}%
\pgfpathlineto{\pgfqpoint{1.134646in}{1.827166in}}%
\pgfpathlineto{\pgfqpoint{1.149250in}{1.802624in}}%
\pgfpathlineto{\pgfqpoint{1.190500in}{1.830933in}}%
\pgfpathclose%
\pgfusepath{fill}%
\end{pgfscope}%
\begin{pgfscope}%
\pgfpathrectangle{\pgfqpoint{0.000000in}{0.000000in}}{\pgfqpoint{3.000000in}{3.000000in}}%
\pgfusepath{clip}%
\pgfsetbuttcap%
\pgfsetroundjoin%
\definecolor{currentfill}{rgb}{0.000000,0.676471,1.000000}%
\pgfsetfillcolor{currentfill}%
\pgfsetlinewidth{0.000000pt}%
\definecolor{currentstroke}{rgb}{0.000000,0.000000,0.000000}%
\pgfsetstrokecolor{currentstroke}%
\pgfsetdash{}{0pt}%
\pgfpathmoveto{\pgfqpoint{1.295775in}{1.368286in}}%
\pgfpathlineto{\pgfqpoint{1.279119in}{1.411408in}}%
\pgfpathlineto{\pgfqpoint{1.273051in}{1.390285in}}%
\pgfpathlineto{\pgfqpoint{1.290129in}{1.348422in}}%
\pgfpathlineto{\pgfqpoint{1.295775in}{1.368286in}}%
\pgfpathclose%
\pgfusepath{fill}%
\end{pgfscope}%
\begin{pgfscope}%
\pgfpathrectangle{\pgfqpoint{0.000000in}{0.000000in}}{\pgfqpoint{3.000000in}{3.000000in}}%
\pgfusepath{clip}%
\pgfsetbuttcap%
\pgfsetroundjoin%
\definecolor{currentfill}{rgb}{0.401645,1.000000,0.566097}%
\pgfsetfillcolor{currentfill}%
\pgfsetlinewidth{0.000000pt}%
\definecolor{currentstroke}{rgb}{0.000000,0.000000,0.000000}%
\pgfsetstrokecolor{currentstroke}%
\pgfsetdash{}{0pt}%
\pgfpathmoveto{\pgfqpoint{1.251547in}{1.602577in}}%
\pgfpathlineto{\pgfqpoint{1.236926in}{1.635106in}}%
\pgfpathlineto{\pgfqpoint{1.212373in}{1.610465in}}%
\pgfpathlineto{\pgfqpoint{1.228215in}{1.579043in}}%
\pgfpathlineto{\pgfqpoint{1.251547in}{1.602577in}}%
\pgfpathclose%
\pgfusepath{fill}%
\end{pgfscope}%
\begin{pgfscope}%
\pgfpathrectangle{\pgfqpoint{0.000000in}{0.000000in}}{\pgfqpoint{3.000000in}{3.000000in}}%
\pgfusepath{clip}%
\pgfsetbuttcap%
\pgfsetroundjoin%
\definecolor{currentfill}{rgb}{0.000000,0.300000,1.000000}%
\pgfsetfillcolor{currentfill}%
\pgfsetlinewidth{0.000000pt}%
\definecolor{currentstroke}{rgb}{0.000000,0.000000,0.000000}%
\pgfsetstrokecolor{currentstroke}%
\pgfsetdash{}{0pt}%
\pgfpathmoveto{\pgfqpoint{1.753036in}{1.221728in}}%
\pgfpathlineto{\pgfqpoint{1.769845in}{1.271004in}}%
\pgfpathlineto{\pgfqpoint{1.774192in}{1.289728in}}%
\pgfpathlineto{\pgfqpoint{1.757095in}{1.239171in}}%
\pgfpathlineto{\pgfqpoint{1.753036in}{1.221728in}}%
\pgfpathclose%
\pgfusepath{fill}%
\end{pgfscope}%
\begin{pgfscope}%
\pgfpathrectangle{\pgfqpoint{0.000000in}{0.000000in}}{\pgfqpoint{3.000000in}{3.000000in}}%
\pgfusepath{clip}%
\pgfsetbuttcap%
\pgfsetroundjoin%
\definecolor{currentfill}{rgb}{0.085389,1.000000,0.882353}%
\pgfsetfillcolor{currentfill}%
\pgfsetlinewidth{0.000000pt}%
\definecolor{currentstroke}{rgb}{0.000000,0.000000,0.000000}%
\pgfsetstrokecolor{currentstroke}%
\pgfsetdash{}{0pt}%
\pgfpathmoveto{\pgfqpoint{1.275731in}{1.472552in}}%
\pgfpathlineto{\pgfqpoint{1.259894in}{1.510414in}}%
\pgfpathlineto{\pgfqpoint{1.245785in}{1.487672in}}%
\pgfpathlineto{\pgfqpoint{1.262455in}{1.451010in}}%
\pgfpathlineto{\pgfqpoint{1.275731in}{1.472552in}}%
\pgfpathclose%
\pgfusepath{fill}%
\end{pgfscope}%
\begin{pgfscope}%
\pgfpathrectangle{\pgfqpoint{0.000000in}{0.000000in}}{\pgfqpoint{3.000000in}{3.000000in}}%
\pgfusepath{clip}%
\pgfsetbuttcap%
\pgfsetroundjoin%
\definecolor{currentfill}{rgb}{0.000000,0.000000,0.838681}%
\pgfsetfillcolor{currentfill}%
\pgfsetlinewidth{0.000000pt}%
\definecolor{currentstroke}{rgb}{0.000000,0.000000,0.000000}%
\pgfsetstrokecolor{currentstroke}%
\pgfsetdash{}{0pt}%
\pgfpathmoveto{\pgfqpoint{1.683294in}{1.060883in}}%
\pgfpathlineto{\pgfqpoint{1.696734in}{1.121781in}}%
\pgfpathlineto{\pgfqpoint{1.714355in}{1.135339in}}%
\pgfpathlineto{\pgfqpoint{1.699418in}{1.073353in}}%
\pgfpathlineto{\pgfqpoint{1.683294in}{1.060883in}}%
\pgfpathclose%
\pgfusepath{fill}%
\end{pgfscope}%
\begin{pgfscope}%
\pgfpathrectangle{\pgfqpoint{0.000000in}{0.000000in}}{\pgfqpoint{3.000000in}{3.000000in}}%
\pgfusepath{clip}%
\pgfsetbuttcap%
\pgfsetroundjoin%
\definecolor{currentfill}{rgb}{0.000000,0.000000,0.500000}%
\pgfsetfillcolor{currentfill}%
\pgfsetlinewidth{0.000000pt}%
\definecolor{currentstroke}{rgb}{0.000000,0.000000,0.000000}%
\pgfsetstrokecolor{currentstroke}%
\pgfsetdash{}{0pt}%
\pgfpathmoveto{\pgfqpoint{1.457152in}{0.967861in}}%
\pgfpathlineto{\pgfqpoint{1.448482in}{1.037976in}}%
\pgfpathlineto{\pgfqpoint{1.474515in}{1.031419in}}%
\pgfpathlineto{\pgfqpoint{1.480737in}{0.961879in}}%
\pgfpathlineto{\pgfqpoint{1.457152in}{0.967861in}}%
\pgfpathclose%
\pgfusepath{fill}%
\end{pgfscope}%
\begin{pgfscope}%
\pgfpathrectangle{\pgfqpoint{0.000000in}{0.000000in}}{\pgfqpoint{3.000000in}{3.000000in}}%
\pgfusepath{clip}%
\pgfsetbuttcap%
\pgfsetroundjoin%
\definecolor{currentfill}{rgb}{1.000000,0.886710,0.000000}%
\pgfsetfillcolor{currentfill}%
\pgfsetlinewidth{0.000000pt}%
\definecolor{currentstroke}{rgb}{0.000000,0.000000,0.000000}%
\pgfsetstrokecolor{currentstroke}%
\pgfsetdash{}{0pt}%
\pgfpathmoveto{\pgfqpoint{1.177478in}{1.856449in}}%
\pgfpathlineto{\pgfqpoint{1.164461in}{1.881240in}}%
\pgfpathlineto{\pgfqpoint{1.120045in}{1.850987in}}%
\pgfpathlineto{\pgfqpoint{1.134646in}{1.827166in}}%
\pgfpathlineto{\pgfqpoint{1.177478in}{1.856449in}}%
\pgfpathclose%
\pgfusepath{fill}%
\end{pgfscope}%
\begin{pgfscope}%
\pgfpathrectangle{\pgfqpoint{0.000000in}{0.000000in}}{\pgfqpoint{3.000000in}{3.000000in}}%
\pgfusepath{clip}%
\pgfsetbuttcap%
\pgfsetroundjoin%
\definecolor{currentfill}{rgb}{0.199241,1.000000,0.768501}%
\pgfsetfillcolor{currentfill}%
\pgfsetlinewidth{0.000000pt}%
\definecolor{currentstroke}{rgb}{0.000000,0.000000,0.000000}%
\pgfsetstrokecolor{currentstroke}%
\pgfsetdash{}{0pt}%
\pgfpathmoveto{\pgfqpoint{1.831375in}{1.495379in}}%
\pgfpathlineto{\pgfqpoint{1.847816in}{1.529965in}}%
\pgfpathlineto{\pgfqpoint{1.830425in}{1.553460in}}%
\pgfpathlineto{\pgfqpoint{1.814951in}{1.517702in}}%
\pgfpathlineto{\pgfqpoint{1.831375in}{1.495379in}}%
\pgfpathclose%
\pgfusepath{fill}%
\end{pgfscope}%
\begin{pgfscope}%
\pgfpathrectangle{\pgfqpoint{0.000000in}{0.000000in}}{\pgfqpoint{3.000000in}{3.000000in}}%
\pgfusepath{clip}%
\pgfsetbuttcap%
\pgfsetroundjoin%
\definecolor{currentfill}{rgb}{0.578748,1.000000,0.388994}%
\pgfsetfillcolor{currentfill}%
\pgfsetlinewidth{0.000000pt}%
\definecolor{currentstroke}{rgb}{0.000000,0.000000,0.000000}%
\pgfsetstrokecolor{currentstroke}%
\pgfsetdash{}{0pt}%
\pgfpathmoveto{\pgfqpoint{1.876850in}{1.649102in}}%
\pgfpathlineto{\pgfqpoint{1.892327in}{1.677912in}}%
\pgfpathlineto{\pgfqpoint{1.862581in}{1.704000in}}%
\pgfpathlineto{\pgfqpoint{1.848451in}{1.674123in}}%
\pgfpathlineto{\pgfqpoint{1.876850in}{1.649102in}}%
\pgfpathclose%
\pgfusepath{fill}%
\end{pgfscope}%
\begin{pgfscope}%
\pgfpathrectangle{\pgfqpoint{0.000000in}{0.000000in}}{\pgfqpoint{3.000000in}{3.000000in}}%
\pgfusepath{clip}%
\pgfsetbuttcap%
\pgfsetroundjoin%
\definecolor{currentfill}{rgb}{0.000000,0.064706,1.000000}%
\pgfsetfillcolor{currentfill}%
\pgfsetlinewidth{0.000000pt}%
\definecolor{currentstroke}{rgb}{0.000000,0.000000,0.000000}%
\pgfsetstrokecolor{currentstroke}%
\pgfsetdash{}{0pt}%
\pgfpathmoveto{\pgfqpoint{1.347175in}{1.160524in}}%
\pgfpathlineto{\pgfqpoint{1.330576in}{1.216007in}}%
\pgfpathlineto{\pgfqpoint{1.341621in}{1.199345in}}%
\pgfpathlineto{\pgfqpoint{1.357371in}{1.145093in}}%
\pgfpathlineto{\pgfqpoint{1.347175in}{1.160524in}}%
\pgfpathclose%
\pgfusepath{fill}%
\end{pgfscope}%
\begin{pgfscope}%
\pgfpathrectangle{\pgfqpoint{0.000000in}{0.000000in}}{\pgfqpoint{3.000000in}{3.000000in}}%
\pgfusepath{clip}%
\pgfsetbuttcap%
\pgfsetroundjoin%
\definecolor{currentfill}{rgb}{1.000000,0.814089,0.000000}%
\pgfsetfillcolor{currentfill}%
\pgfsetlinewidth{0.000000pt}%
\definecolor{currentstroke}{rgb}{0.000000,0.000000,0.000000}%
\pgfsetstrokecolor{currentstroke}%
\pgfsetdash{}{0pt}%
\pgfpathmoveto{\pgfqpoint{1.164461in}{1.881240in}}%
\pgfpathlineto{\pgfqpoint{1.151450in}{1.905371in}}%
\pgfpathlineto{\pgfqpoint{1.105447in}{1.874152in}}%
\pgfpathlineto{\pgfqpoint{1.120045in}{1.850987in}}%
\pgfpathlineto{\pgfqpoint{1.164461in}{1.881240in}}%
\pgfpathclose%
\pgfusepath{fill}%
\end{pgfscope}%
\begin{pgfscope}%
\pgfpathrectangle{\pgfqpoint{0.000000in}{0.000000in}}{\pgfqpoint{3.000000in}{3.000000in}}%
\pgfusepath{clip}%
\pgfsetbuttcap%
\pgfsetroundjoin%
\definecolor{currentfill}{rgb}{1.000000,0.741467,0.000000}%
\pgfsetfillcolor{currentfill}%
\pgfsetlinewidth{0.000000pt}%
\definecolor{currentstroke}{rgb}{0.000000,0.000000,0.000000}%
\pgfsetstrokecolor{currentstroke}%
\pgfsetdash{}{0pt}%
\pgfpathmoveto{\pgfqpoint{1.151450in}{1.905371in}}%
\pgfpathlineto{\pgfqpoint{1.138445in}{1.928899in}}%
\pgfpathlineto{\pgfqpoint{1.090852in}{1.896716in}}%
\pgfpathlineto{\pgfqpoint{1.105447in}{1.874152in}}%
\pgfpathlineto{\pgfqpoint{1.151450in}{1.905371in}}%
\pgfpathclose%
\pgfusepath{fill}%
\end{pgfscope}%
\begin{pgfscope}%
\pgfpathrectangle{\pgfqpoint{0.000000in}{0.000000in}}{\pgfqpoint{3.000000in}{3.000000in}}%
\pgfusepath{clip}%
\pgfsetbuttcap%
\pgfsetroundjoin%
\definecolor{currentfill}{rgb}{0.000000,0.000000,0.500000}%
\pgfsetfillcolor{currentfill}%
\pgfsetlinewidth{0.000000pt}%
\definecolor{currentstroke}{rgb}{0.000000,0.000000,0.000000}%
\pgfsetstrokecolor{currentstroke}%
\pgfsetdash{}{0pt}%
\pgfpathmoveto{\pgfqpoint{1.583744in}{0.959013in}}%
\pgfpathlineto{\pgfqpoint{1.588240in}{1.028276in}}%
\pgfpathlineto{\pgfqpoint{1.615462in}{1.033364in}}%
\pgfpathlineto{\pgfqpoint{1.608404in}{0.963654in}}%
\pgfpathlineto{\pgfqpoint{1.583744in}{0.959013in}}%
\pgfpathclose%
\pgfusepath{fill}%
\end{pgfscope}%
\begin{pgfscope}%
\pgfpathrectangle{\pgfqpoint{0.000000in}{0.000000in}}{\pgfqpoint{3.000000in}{3.000000in}}%
\pgfusepath{clip}%
\pgfsetbuttcap%
\pgfsetroundjoin%
\definecolor{currentfill}{rgb}{0.000000,0.000000,0.838681}%
\pgfsetfillcolor{currentfill}%
\pgfsetlinewidth{0.000000pt}%
\definecolor{currentstroke}{rgb}{0.000000,0.000000,0.000000}%
\pgfsetstrokecolor{currentstroke}%
\pgfsetdash{}{0pt}%
\pgfpathmoveto{\pgfqpoint{1.386614in}{1.069052in}}%
\pgfpathlineto{\pgfqpoint{1.372135in}{1.130663in}}%
\pgfpathlineto{\pgfqpoint{1.391117in}{1.117600in}}%
\pgfpathlineto{\pgfqpoint{1.403980in}{1.057038in}}%
\pgfpathlineto{\pgfqpoint{1.386614in}{1.069052in}}%
\pgfpathclose%
\pgfusepath{fill}%
\end{pgfscope}%
\begin{pgfscope}%
\pgfpathrectangle{\pgfqpoint{0.000000in}{0.000000in}}{\pgfqpoint{3.000000in}{3.000000in}}%
\pgfusepath{clip}%
\pgfsetbuttcap%
\pgfsetroundjoin%
\definecolor{currentfill}{rgb}{0.000000,0.849020,1.000000}%
\pgfsetfillcolor{currentfill}%
\pgfsetlinewidth{0.000000pt}%
\definecolor{currentstroke}{rgb}{0.000000,0.000000,0.000000}%
\pgfsetstrokecolor{currentstroke}%
\pgfsetdash{}{0pt}%
\pgfpathmoveto{\pgfqpoint{1.806732in}{1.397382in}}%
\pgfpathlineto{\pgfqpoint{1.823725in}{1.436150in}}%
\pgfpathlineto{\pgfqpoint{1.814939in}{1.458310in}}%
\pgfpathlineto{\pgfqpoint{1.798509in}{1.418300in}}%
\pgfpathlineto{\pgfqpoint{1.806732in}{1.397382in}}%
\pgfpathclose%
\pgfusepath{fill}%
\end{pgfscope}%
\begin{pgfscope}%
\pgfpathrectangle{\pgfqpoint{0.000000in}{0.000000in}}{\pgfqpoint{3.000000in}{3.000000in}}%
\pgfusepath{clip}%
\pgfsetbuttcap%
\pgfsetroundjoin%
\definecolor{currentfill}{rgb}{0.000000,0.503922,1.000000}%
\pgfsetfillcolor{currentfill}%
\pgfsetlinewidth{0.000000pt}%
\definecolor{currentstroke}{rgb}{0.000000,0.000000,0.000000}%
\pgfsetstrokecolor{currentstroke}%
\pgfsetdash{}{0pt}%
\pgfpathmoveto{\pgfqpoint{1.307195in}{1.302284in}}%
\pgfpathlineto{\pgfqpoint{1.290129in}{1.348422in}}%
\pgfpathlineto{\pgfqpoint{1.290634in}{1.328309in}}%
\pgfpathlineto{\pgfqpoint{1.307699in}{1.283454in}}%
\pgfpathlineto{\pgfqpoint{1.307195in}{1.302284in}}%
\pgfpathclose%
\pgfusepath{fill}%
\end{pgfscope}%
\begin{pgfscope}%
\pgfpathrectangle{\pgfqpoint{0.000000in}{0.000000in}}{\pgfqpoint{3.000000in}{3.000000in}}%
\pgfusepath{clip}%
\pgfsetbuttcap%
\pgfsetroundjoin%
\definecolor{currentfill}{rgb}{1.000000,0.668845,0.000000}%
\pgfsetfillcolor{currentfill}%
\pgfsetlinewidth{0.000000pt}%
\definecolor{currentstroke}{rgb}{0.000000,0.000000,0.000000}%
\pgfsetstrokecolor{currentstroke}%
\pgfsetdash{}{0pt}%
\pgfpathmoveto{\pgfqpoint{1.138445in}{1.928899in}}%
\pgfpathlineto{\pgfqpoint{1.125445in}{1.951873in}}%
\pgfpathlineto{\pgfqpoint{1.076260in}{1.918731in}}%
\pgfpathlineto{\pgfqpoint{1.090852in}{1.896716in}}%
\pgfpathlineto{\pgfqpoint{1.138445in}{1.928899in}}%
\pgfpathclose%
\pgfusepath{fill}%
\end{pgfscope}%
\begin{pgfscope}%
\pgfpathrectangle{\pgfqpoint{0.000000in}{0.000000in}}{\pgfqpoint{3.000000in}{3.000000in}}%
\pgfusepath{clip}%
\pgfsetbuttcap%
\pgfsetroundjoin%
\definecolor{currentfill}{rgb}{0.490196,1.000000,0.477546}%
\pgfsetfillcolor{currentfill}%
\pgfsetlinewidth{0.000000pt}%
\definecolor{currentstroke}{rgb}{0.000000,0.000000,0.000000}%
\pgfsetstrokecolor{currentstroke}%
\pgfsetdash{}{0pt}%
\pgfpathmoveto{\pgfqpoint{1.236926in}{1.635106in}}%
\pgfpathlineto{\pgfqpoint{1.222308in}{1.666038in}}%
\pgfpathlineto{\pgfqpoint{1.196528in}{1.640293in}}%
\pgfpathlineto{\pgfqpoint{1.212373in}{1.610465in}}%
\pgfpathlineto{\pgfqpoint{1.236926in}{1.635106in}}%
\pgfpathclose%
\pgfusepath{fill}%
\end{pgfscope}%
\begin{pgfscope}%
\pgfpathrectangle{\pgfqpoint{0.000000in}{0.000000in}}{\pgfqpoint{3.000000in}{3.000000in}}%
\pgfusepath{clip}%
\pgfsetbuttcap%
\pgfsetroundjoin%
\definecolor{currentfill}{rgb}{0.000000,0.000000,0.500000}%
\pgfsetfillcolor{currentfill}%
\pgfsetlinewidth{0.000000pt}%
\definecolor{currentstroke}{rgb}{0.000000,0.000000,0.000000}%
\pgfsetstrokecolor{currentstroke}%
\pgfsetdash{}{0pt}%
\pgfpathmoveto{\pgfqpoint{1.480737in}{0.961879in}}%
\pgfpathlineto{\pgfqpoint{1.474515in}{1.031419in}}%
\pgfpathlineto{\pgfqpoint{1.502218in}{1.027088in}}%
\pgfpathlineto{\pgfqpoint{1.505831in}{0.957929in}}%
\pgfpathlineto{\pgfqpoint{1.480737in}{0.961879in}}%
\pgfpathclose%
\pgfusepath{fill}%
\end{pgfscope}%
\begin{pgfscope}%
\pgfpathrectangle{\pgfqpoint{0.000000in}{0.000000in}}{\pgfqpoint{3.000000in}{3.000000in}}%
\pgfusepath{clip}%
\pgfsetbuttcap%
\pgfsetroundjoin%
\definecolor{currentfill}{rgb}{0.667299,1.000000,0.300443}%
\pgfsetfillcolor{currentfill}%
\pgfsetlinewidth{0.000000pt}%
\definecolor{currentstroke}{rgb}{0.000000,0.000000,0.000000}%
\pgfsetstrokecolor{currentstroke}%
\pgfsetdash{}{0pt}%
\pgfpathmoveto{\pgfqpoint{1.892327in}{1.677912in}}%
\pgfpathlineto{\pgfqpoint{1.907803in}{1.705489in}}%
\pgfpathlineto{\pgfqpoint{1.876707in}{1.732639in}}%
\pgfpathlineto{\pgfqpoint{1.862581in}{1.704000in}}%
\pgfpathlineto{\pgfqpoint{1.892327in}{1.677912in}}%
\pgfpathclose%
\pgfusepath{fill}%
\end{pgfscope}%
\begin{pgfscope}%
\pgfpathrectangle{\pgfqpoint{0.000000in}{0.000000in}}{\pgfqpoint{3.000000in}{3.000000in}}%
\pgfusepath{clip}%
\pgfsetbuttcap%
\pgfsetroundjoin%
\definecolor{currentfill}{rgb}{0.000000,0.300000,1.000000}%
\pgfsetfillcolor{currentfill}%
\pgfsetlinewidth{0.000000pt}%
\definecolor{currentstroke}{rgb}{0.000000,0.000000,0.000000}%
\pgfsetstrokecolor{currentstroke}%
\pgfsetdash{}{0pt}%
\pgfpathmoveto{\pgfqpoint{1.324747in}{1.233326in}}%
\pgfpathlineto{\pgfqpoint{1.307699in}{1.283454in}}%
\pgfpathlineto{\pgfqpoint{1.313957in}{1.264861in}}%
\pgfpathlineto{\pgfqpoint{1.330576in}{1.216007in}}%
\pgfpathlineto{\pgfqpoint{1.324747in}{1.233326in}}%
\pgfpathclose%
\pgfusepath{fill}%
\end{pgfscope}%
\begin{pgfscope}%
\pgfpathrectangle{\pgfqpoint{0.000000in}{0.000000in}}{\pgfqpoint{3.000000in}{3.000000in}}%
\pgfusepath{clip}%
\pgfsetbuttcap%
\pgfsetroundjoin%
\definecolor{currentfill}{rgb}{1.000000,0.610748,0.000000}%
\pgfsetfillcolor{currentfill}%
\pgfsetlinewidth{0.000000pt}%
\definecolor{currentstroke}{rgb}{0.000000,0.000000,0.000000}%
\pgfsetstrokecolor{currentstroke}%
\pgfsetdash{}{0pt}%
\pgfpathmoveto{\pgfqpoint{1.125445in}{1.951873in}}%
\pgfpathlineto{\pgfqpoint{1.112452in}{1.974337in}}%
\pgfpathlineto{\pgfqpoint{1.061671in}{1.940239in}}%
\pgfpathlineto{\pgfqpoint{1.076260in}{1.918731in}}%
\pgfpathlineto{\pgfqpoint{1.125445in}{1.951873in}}%
\pgfpathclose%
\pgfusepath{fill}%
\end{pgfscope}%
\begin{pgfscope}%
\pgfpathrectangle{\pgfqpoint{0.000000in}{0.000000in}}{\pgfqpoint{3.000000in}{3.000000in}}%
\pgfusepath{clip}%
\pgfsetbuttcap%
\pgfsetroundjoin%
\definecolor{currentfill}{rgb}{1.000000,0.538126,0.000000}%
\pgfsetfillcolor{currentfill}%
\pgfsetlinewidth{0.000000pt}%
\definecolor{currentstroke}{rgb}{0.000000,0.000000,0.000000}%
\pgfsetstrokecolor{currentstroke}%
\pgfsetdash{}{0pt}%
\pgfpathmoveto{\pgfqpoint{1.112452in}{1.974337in}}%
\pgfpathlineto{\pgfqpoint{1.099464in}{1.996330in}}%
\pgfpathlineto{\pgfqpoint{1.047086in}{1.961280in}}%
\pgfpathlineto{\pgfqpoint{1.061671in}{1.940239in}}%
\pgfpathlineto{\pgfqpoint{1.112452in}{1.974337in}}%
\pgfpathclose%
\pgfusepath{fill}%
\end{pgfscope}%
\begin{pgfscope}%
\pgfpathrectangle{\pgfqpoint{0.000000in}{0.000000in}}{\pgfqpoint{3.000000in}{3.000000in}}%
\pgfusepath{clip}%
\pgfsetbuttcap%
\pgfsetroundjoin%
\definecolor{currentfill}{rgb}{0.500000,0.000000,0.000000}%
\pgfsetfillcolor{currentfill}%
\pgfsetlinewidth{0.000000pt}%
\definecolor{currentstroke}{rgb}{0.000000,0.000000,0.000000}%
\pgfsetstrokecolor{currentstroke}%
\pgfsetdash{}{0pt}%
\pgfpathmoveto{\pgfqpoint{0.905412in}{2.286935in}}%
\pgfpathlineto{\pgfqpoint{0.892529in}{2.304446in}}%
\pgfpathlineto{\pgfqpoint{0.814235in}{2.254642in}}%
\pgfpathlineto{\pgfqpoint{0.828758in}{2.238028in}}%
\pgfpathlineto{\pgfqpoint{0.905412in}{2.286935in}}%
\pgfpathclose%
\pgfusepath{fill}%
\end{pgfscope}%
\begin{pgfscope}%
\pgfpathrectangle{\pgfqpoint{0.000000in}{0.000000in}}{\pgfqpoint{3.000000in}{3.000000in}}%
\pgfusepath{clip}%
\pgfsetbuttcap%
\pgfsetroundjoin%
\definecolor{currentfill}{rgb}{0.000000,0.000000,0.500000}%
\pgfsetfillcolor{currentfill}%
\pgfsetlinewidth{0.000000pt}%
\definecolor{currentstroke}{rgb}{0.000000,0.000000,0.000000}%
\pgfsetstrokecolor{currentstroke}%
\pgfsetdash{}{0pt}%
\pgfpathmoveto{\pgfqpoint{1.557994in}{0.956477in}}%
\pgfpathlineto{\pgfqpoint{1.559810in}{1.025496in}}%
\pgfpathlineto{\pgfqpoint{1.588240in}{1.028276in}}%
\pgfpathlineto{\pgfqpoint{1.583744in}{0.959013in}}%
\pgfpathlineto{\pgfqpoint{1.557994in}{0.956477in}}%
\pgfpathclose%
\pgfusepath{fill}%
\end{pgfscope}%
\begin{pgfscope}%
\pgfpathrectangle{\pgfqpoint{0.000000in}{0.000000in}}{\pgfqpoint{3.000000in}{3.000000in}}%
\pgfusepath{clip}%
\pgfsetbuttcap%
\pgfsetroundjoin%
\definecolor{currentfill}{rgb}{1.000000,0.480029,0.000000}%
\pgfsetfillcolor{currentfill}%
\pgfsetlinewidth{0.000000pt}%
\definecolor{currentstroke}{rgb}{0.000000,0.000000,0.000000}%
\pgfsetstrokecolor{currentstroke}%
\pgfsetdash{}{0pt}%
\pgfpathmoveto{\pgfqpoint{1.099464in}{1.996330in}}%
\pgfpathlineto{\pgfqpoint{1.086483in}{2.017887in}}%
\pgfpathlineto{\pgfqpoint{1.032504in}{1.981889in}}%
\pgfpathlineto{\pgfqpoint{1.047086in}{1.961280in}}%
\pgfpathlineto{\pgfqpoint{1.099464in}{1.996330in}}%
\pgfpathclose%
\pgfusepath{fill}%
\end{pgfscope}%
\begin{pgfscope}%
\pgfpathrectangle{\pgfqpoint{0.000000in}{0.000000in}}{\pgfqpoint{3.000000in}{3.000000in}}%
\pgfusepath{clip}%
\pgfsetbuttcap%
\pgfsetroundjoin%
\definecolor{currentfill}{rgb}{0.553476,0.000000,0.000000}%
\pgfsetfillcolor{currentfill}%
\pgfsetlinewidth{0.000000pt}%
\definecolor{currentstroke}{rgb}{0.000000,0.000000,0.000000}%
\pgfsetstrokecolor{currentstroke}%
\pgfsetdash{}{0pt}%
\pgfpathmoveto{\pgfqpoint{0.918303in}{2.269247in}}%
\pgfpathlineto{\pgfqpoint{0.905412in}{2.286935in}}%
\pgfpathlineto{\pgfqpoint{0.828758in}{2.238028in}}%
\pgfpathlineto{\pgfqpoint{0.843284in}{2.221239in}}%
\pgfpathlineto{\pgfqpoint{0.918303in}{2.269247in}}%
\pgfpathclose%
\pgfusepath{fill}%
\end{pgfscope}%
\begin{pgfscope}%
\pgfpathrectangle{\pgfqpoint{0.000000in}{0.000000in}}{\pgfqpoint{3.000000in}{3.000000in}}%
\pgfusepath{clip}%
\pgfsetbuttcap%
\pgfsetroundjoin%
\definecolor{currentfill}{rgb}{0.000000,0.000000,0.500000}%
\pgfsetfillcolor{currentfill}%
\pgfsetlinewidth{0.000000pt}%
\definecolor{currentstroke}{rgb}{0.000000,0.000000,0.000000}%
\pgfsetstrokecolor{currentstroke}%
\pgfsetdash{}{0pt}%
\pgfpathmoveto{\pgfqpoint{1.505831in}{0.957929in}}%
\pgfpathlineto{\pgfqpoint{1.502218in}{1.027088in}}%
\pgfpathlineto{\pgfqpoint{1.530892in}{1.025096in}}%
\pgfpathlineto{\pgfqpoint{1.531802in}{0.956113in}}%
\pgfpathlineto{\pgfqpoint{1.505831in}{0.957929in}}%
\pgfpathclose%
\pgfusepath{fill}%
\end{pgfscope}%
\begin{pgfscope}%
\pgfpathrectangle{\pgfqpoint{0.000000in}{0.000000in}}{\pgfqpoint{3.000000in}{3.000000in}}%
\pgfusepath{clip}%
\pgfsetbuttcap%
\pgfsetroundjoin%
\definecolor{currentfill}{rgb}{0.606952,0.000000,0.000000}%
\pgfsetfillcolor{currentfill}%
\pgfsetlinewidth{0.000000pt}%
\definecolor{currentstroke}{rgb}{0.000000,0.000000,0.000000}%
\pgfsetstrokecolor{currentstroke}%
\pgfsetdash{}{0pt}%
\pgfpathmoveto{\pgfqpoint{0.931200in}{2.251372in}}%
\pgfpathlineto{\pgfqpoint{0.918303in}{2.269247in}}%
\pgfpathlineto{\pgfqpoint{0.843284in}{2.221239in}}%
\pgfpathlineto{\pgfqpoint{0.857815in}{2.204267in}}%
\pgfpathlineto{\pgfqpoint{0.931200in}{2.251372in}}%
\pgfpathclose%
\pgfusepath{fill}%
\end{pgfscope}%
\begin{pgfscope}%
\pgfpathrectangle{\pgfqpoint{0.000000in}{0.000000in}}{\pgfqpoint{3.000000in}{3.000000in}}%
\pgfusepath{clip}%
\pgfsetbuttcap%
\pgfsetroundjoin%
\definecolor{currentfill}{rgb}{1.000000,0.407407,0.000000}%
\pgfsetfillcolor{currentfill}%
\pgfsetlinewidth{0.000000pt}%
\definecolor{currentstroke}{rgb}{0.000000,0.000000,0.000000}%
\pgfsetstrokecolor{currentstroke}%
\pgfsetdash{}{0pt}%
\pgfpathmoveto{\pgfqpoint{1.086483in}{2.017887in}}%
\pgfpathlineto{\pgfqpoint{1.073507in}{2.039039in}}%
\pgfpathlineto{\pgfqpoint{1.017925in}{2.002097in}}%
\pgfpathlineto{\pgfqpoint{1.032504in}{1.981889in}}%
\pgfpathlineto{\pgfqpoint{1.086483in}{2.017887in}}%
\pgfpathclose%
\pgfusepath{fill}%
\end{pgfscope}%
\begin{pgfscope}%
\pgfpathrectangle{\pgfqpoint{0.000000in}{0.000000in}}{\pgfqpoint{3.000000in}{3.000000in}}%
\pgfusepath{clip}%
\pgfsetbuttcap%
\pgfsetroundjoin%
\definecolor{currentfill}{rgb}{0.678253,0.000000,0.000000}%
\pgfsetfillcolor{currentfill}%
\pgfsetlinewidth{0.000000pt}%
\definecolor{currentstroke}{rgb}{0.000000,0.000000,0.000000}%
\pgfsetstrokecolor{currentstroke}%
\pgfsetdash{}{0pt}%
\pgfpathmoveto{\pgfqpoint{0.944104in}{2.233301in}}%
\pgfpathlineto{\pgfqpoint{0.931200in}{2.251372in}}%
\pgfpathlineto{\pgfqpoint{0.857815in}{2.204267in}}%
\pgfpathlineto{\pgfqpoint{0.872350in}{2.187103in}}%
\pgfpathlineto{\pgfqpoint{0.944104in}{2.233301in}}%
\pgfpathclose%
\pgfusepath{fill}%
\end{pgfscope}%
\begin{pgfscope}%
\pgfpathrectangle{\pgfqpoint{0.000000in}{0.000000in}}{\pgfqpoint{3.000000in}{3.000000in}}%
\pgfusepath{clip}%
\pgfsetbuttcap%
\pgfsetroundjoin%
\definecolor{currentfill}{rgb}{0.000000,0.000000,0.500000}%
\pgfsetfillcolor{currentfill}%
\pgfsetlinewidth{0.000000pt}%
\definecolor{currentstroke}{rgb}{0.000000,0.000000,0.000000}%
\pgfsetstrokecolor{currentstroke}%
\pgfsetdash{}{0pt}%
\pgfpathmoveto{\pgfqpoint{1.531802in}{0.956113in}}%
\pgfpathlineto{\pgfqpoint{1.530892in}{1.025096in}}%
\pgfpathlineto{\pgfqpoint{1.559810in}{1.025496in}}%
\pgfpathlineto{\pgfqpoint{1.557994in}{0.956477in}}%
\pgfpathlineto{\pgfqpoint{1.531802in}{0.956113in}}%
\pgfpathclose%
\pgfusepath{fill}%
\end{pgfscope}%
\begin{pgfscope}%
\pgfpathrectangle{\pgfqpoint{0.000000in}{0.000000in}}{\pgfqpoint{3.000000in}{3.000000in}}%
\pgfusepath{clip}%
\pgfsetbuttcap%
\pgfsetroundjoin%
\definecolor{currentfill}{rgb}{0.300443,1.000000,0.667299}%
\pgfsetfillcolor{currentfill}%
\pgfsetlinewidth{0.000000pt}%
\definecolor{currentstroke}{rgb}{0.000000,0.000000,0.000000}%
\pgfsetstrokecolor{currentstroke}%
\pgfsetdash{}{0pt}%
\pgfpathmoveto{\pgfqpoint{1.847816in}{1.529965in}}%
\pgfpathlineto{\pgfqpoint{1.864263in}{1.562430in}}%
\pgfpathlineto{\pgfqpoint{1.845899in}{1.587094in}}%
\pgfpathlineto{\pgfqpoint{1.830425in}{1.553460in}}%
\pgfpathlineto{\pgfqpoint{1.847816in}{1.529965in}}%
\pgfpathclose%
\pgfusepath{fill}%
\end{pgfscope}%
\begin{pgfscope}%
\pgfpathrectangle{\pgfqpoint{0.000000in}{0.000000in}}{\pgfqpoint{3.000000in}{3.000000in}}%
\pgfusepath{clip}%
\pgfsetbuttcap%
\pgfsetroundjoin%
\definecolor{currentfill}{rgb}{0.199241,1.000000,0.768501}%
\pgfsetfillcolor{currentfill}%
\pgfsetlinewidth{0.000000pt}%
\definecolor{currentstroke}{rgb}{0.000000,0.000000,0.000000}%
\pgfsetstrokecolor{currentstroke}%
\pgfsetdash{}{0pt}%
\pgfpathmoveto{\pgfqpoint{1.259894in}{1.510414in}}%
\pgfpathlineto{\pgfqpoint{1.244056in}{1.545791in}}%
\pgfpathlineto{\pgfqpoint{1.229108in}{1.521852in}}%
\pgfpathlineto{\pgfqpoint{1.245785in}{1.487672in}}%
\pgfpathlineto{\pgfqpoint{1.259894in}{1.510414in}}%
\pgfpathclose%
\pgfusepath{fill}%
\end{pgfscope}%
\begin{pgfscope}%
\pgfpathrectangle{\pgfqpoint{0.000000in}{0.000000in}}{\pgfqpoint{3.000000in}{3.000000in}}%
\pgfusepath{clip}%
\pgfsetbuttcap%
\pgfsetroundjoin%
\definecolor{currentfill}{rgb}{1.000000,0.349310,0.000000}%
\pgfsetfillcolor{currentfill}%
\pgfsetlinewidth{0.000000pt}%
\definecolor{currentstroke}{rgb}{0.000000,0.000000,0.000000}%
\pgfsetstrokecolor{currentstroke}%
\pgfsetdash{}{0pt}%
\pgfpathmoveto{\pgfqpoint{1.073507in}{2.039039in}}%
\pgfpathlineto{\pgfqpoint{1.060538in}{2.059814in}}%
\pgfpathlineto{\pgfqpoint{1.003350in}{2.021930in}}%
\pgfpathlineto{\pgfqpoint{1.017925in}{2.002097in}}%
\pgfpathlineto{\pgfqpoint{1.073507in}{2.039039in}}%
\pgfpathclose%
\pgfusepath{fill}%
\end{pgfscope}%
\begin{pgfscope}%
\pgfpathrectangle{\pgfqpoint{0.000000in}{0.000000in}}{\pgfqpoint{3.000000in}{3.000000in}}%
\pgfusepath{clip}%
\pgfsetbuttcap%
\pgfsetroundjoin%
\definecolor{currentfill}{rgb}{0.731729,0.000000,0.000000}%
\pgfsetfillcolor{currentfill}%
\pgfsetlinewidth{0.000000pt}%
\definecolor{currentstroke}{rgb}{0.000000,0.000000,0.000000}%
\pgfsetstrokecolor{currentstroke}%
\pgfsetdash{}{0pt}%
\pgfpathmoveto{\pgfqpoint{0.957015in}{2.215024in}}%
\pgfpathlineto{\pgfqpoint{0.944104in}{2.233301in}}%
\pgfpathlineto{\pgfqpoint{0.872350in}{2.187103in}}%
\pgfpathlineto{\pgfqpoint{0.886890in}{2.169736in}}%
\pgfpathlineto{\pgfqpoint{0.957015in}{2.215024in}}%
\pgfpathclose%
\pgfusepath{fill}%
\end{pgfscope}%
\begin{pgfscope}%
\pgfpathrectangle{\pgfqpoint{0.000000in}{0.000000in}}{\pgfqpoint{3.000000in}{3.000000in}}%
\pgfusepath{clip}%
\pgfsetbuttcap%
\pgfsetroundjoin%
\definecolor{currentfill}{rgb}{1.000000,0.291213,0.000000}%
\pgfsetfillcolor{currentfill}%
\pgfsetlinewidth{0.000000pt}%
\definecolor{currentstroke}{rgb}{0.000000,0.000000,0.000000}%
\pgfsetstrokecolor{currentstroke}%
\pgfsetdash{}{0pt}%
\pgfpathmoveto{\pgfqpoint{1.060538in}{2.059814in}}%
\pgfpathlineto{\pgfqpoint{1.047575in}{2.080238in}}%
\pgfpathlineto{\pgfqpoint{0.988779in}{2.041416in}}%
\pgfpathlineto{\pgfqpoint{1.003350in}{2.021930in}}%
\pgfpathlineto{\pgfqpoint{1.060538in}{2.059814in}}%
\pgfpathclose%
\pgfusepath{fill}%
\end{pgfscope}%
\begin{pgfscope}%
\pgfpathrectangle{\pgfqpoint{0.000000in}{0.000000in}}{\pgfqpoint{3.000000in}{3.000000in}}%
\pgfusepath{clip}%
\pgfsetbuttcap%
\pgfsetroundjoin%
\definecolor{currentfill}{rgb}{0.743201,1.000000,0.224541}%
\pgfsetfillcolor{currentfill}%
\pgfsetlinewidth{0.000000pt}%
\definecolor{currentstroke}{rgb}{0.000000,0.000000,0.000000}%
\pgfsetstrokecolor{currentstroke}%
\pgfsetdash{}{0pt}%
\pgfpathmoveto{\pgfqpoint{1.907803in}{1.705489in}}%
\pgfpathlineto{\pgfqpoint{1.923280in}{1.731970in}}%
\pgfpathlineto{\pgfqpoint{1.890829in}{1.760180in}}%
\pgfpathlineto{\pgfqpoint{1.876707in}{1.732639in}}%
\pgfpathlineto{\pgfqpoint{1.907803in}{1.705489in}}%
\pgfpathclose%
\pgfusepath{fill}%
\end{pgfscope}%
\begin{pgfscope}%
\pgfpathrectangle{\pgfqpoint{0.000000in}{0.000000in}}{\pgfqpoint{3.000000in}{3.000000in}}%
\pgfusepath{clip}%
\pgfsetbuttcap%
\pgfsetroundjoin%
\definecolor{currentfill}{rgb}{0.803030,0.000000,0.000000}%
\pgfsetfillcolor{currentfill}%
\pgfsetlinewidth{0.000000pt}%
\definecolor{currentstroke}{rgb}{0.000000,0.000000,0.000000}%
\pgfsetstrokecolor{currentstroke}%
\pgfsetdash{}{0pt}%
\pgfpathmoveto{\pgfqpoint{0.969932in}{2.196529in}}%
\pgfpathlineto{\pgfqpoint{0.957015in}{2.215024in}}%
\pgfpathlineto{\pgfqpoint{0.886890in}{2.169736in}}%
\pgfpathlineto{\pgfqpoint{0.901433in}{2.152155in}}%
\pgfpathlineto{\pgfqpoint{0.969932in}{2.196529in}}%
\pgfpathclose%
\pgfusepath{fill}%
\end{pgfscope}%
\begin{pgfscope}%
\pgfpathrectangle{\pgfqpoint{0.000000in}{0.000000in}}{\pgfqpoint{3.000000in}{3.000000in}}%
\pgfusepath{clip}%
\pgfsetbuttcap%
\pgfsetroundjoin%
\definecolor{currentfill}{rgb}{1.000000,0.233115,0.000000}%
\pgfsetfillcolor{currentfill}%
\pgfsetlinewidth{0.000000pt}%
\definecolor{currentstroke}{rgb}{0.000000,0.000000,0.000000}%
\pgfsetstrokecolor{currentstroke}%
\pgfsetdash{}{0pt}%
\pgfpathmoveto{\pgfqpoint{1.047575in}{2.080238in}}%
\pgfpathlineto{\pgfqpoint{1.034619in}{2.100332in}}%
\pgfpathlineto{\pgfqpoint{0.974212in}{2.060576in}}%
\pgfpathlineto{\pgfqpoint{0.988779in}{2.041416in}}%
\pgfpathlineto{\pgfqpoint{1.047575in}{2.080238in}}%
\pgfpathclose%
\pgfusepath{fill}%
\end{pgfscope}%
\begin{pgfscope}%
\pgfpathrectangle{\pgfqpoint{0.000000in}{0.000000in}}{\pgfqpoint{3.000000in}{3.000000in}}%
\pgfusepath{clip}%
\pgfsetbuttcap%
\pgfsetroundjoin%
\definecolor{currentfill}{rgb}{0.856506,0.000000,0.000000}%
\pgfsetfillcolor{currentfill}%
\pgfsetlinewidth{0.000000pt}%
\definecolor{currentstroke}{rgb}{0.000000,0.000000,0.000000}%
\pgfsetstrokecolor{currentstroke}%
\pgfsetdash{}{0pt}%
\pgfpathmoveto{\pgfqpoint{0.982857in}{2.177805in}}%
\pgfpathlineto{\pgfqpoint{0.969932in}{2.196529in}}%
\pgfpathlineto{\pgfqpoint{0.901433in}{2.152155in}}%
\pgfpathlineto{\pgfqpoint{0.915981in}{2.134347in}}%
\pgfpathlineto{\pgfqpoint{0.982857in}{2.177805in}}%
\pgfpathclose%
\pgfusepath{fill}%
\end{pgfscope}%
\begin{pgfscope}%
\pgfpathrectangle{\pgfqpoint{0.000000in}{0.000000in}}{\pgfqpoint{3.000000in}{3.000000in}}%
\pgfusepath{clip}%
\pgfsetbuttcap%
\pgfsetroundjoin%
\definecolor{currentfill}{rgb}{1.000000,0.175018,0.000000}%
\pgfsetfillcolor{currentfill}%
\pgfsetlinewidth{0.000000pt}%
\definecolor{currentstroke}{rgb}{0.000000,0.000000,0.000000}%
\pgfsetstrokecolor{currentstroke}%
\pgfsetdash{}{0pt}%
\pgfpathmoveto{\pgfqpoint{1.034619in}{2.100332in}}%
\pgfpathlineto{\pgfqpoint{1.021668in}{2.120118in}}%
\pgfpathlineto{\pgfqpoint{0.959648in}{2.079432in}}%
\pgfpathlineto{\pgfqpoint{0.974212in}{2.060576in}}%
\pgfpathlineto{\pgfqpoint{1.034619in}{2.100332in}}%
\pgfpathclose%
\pgfusepath{fill}%
\end{pgfscope}%
\begin{pgfscope}%
\pgfpathrectangle{\pgfqpoint{0.000000in}{0.000000in}}{\pgfqpoint{3.000000in}{3.000000in}}%
\pgfusepath{clip}%
\pgfsetbuttcap%
\pgfsetroundjoin%
\definecolor{currentfill}{rgb}{0.927807,0.015251,0.000000}%
\pgfsetfillcolor{currentfill}%
\pgfsetlinewidth{0.000000pt}%
\definecolor{currentstroke}{rgb}{0.000000,0.000000,0.000000}%
\pgfsetstrokecolor{currentstroke}%
\pgfsetdash{}{0pt}%
\pgfpathmoveto{\pgfqpoint{0.995787in}{2.158838in}}%
\pgfpathlineto{\pgfqpoint{0.982857in}{2.177805in}}%
\pgfpathlineto{\pgfqpoint{0.915981in}{2.134347in}}%
\pgfpathlineto{\pgfqpoint{0.930533in}{2.116301in}}%
\pgfpathlineto{\pgfqpoint{0.995787in}{2.158838in}}%
\pgfpathclose%
\pgfusepath{fill}%
\end{pgfscope}%
\begin{pgfscope}%
\pgfpathrectangle{\pgfqpoint{0.000000in}{0.000000in}}{\pgfqpoint{3.000000in}{3.000000in}}%
\pgfusepath{clip}%
\pgfsetbuttcap%
\pgfsetroundjoin%
\definecolor{currentfill}{rgb}{1.000000,0.116921,0.000000}%
\pgfsetfillcolor{currentfill}%
\pgfsetlinewidth{0.000000pt}%
\definecolor{currentstroke}{rgb}{0.000000,0.000000,0.000000}%
\pgfsetstrokecolor{currentstroke}%
\pgfsetdash{}{0pt}%
\pgfpathmoveto{\pgfqpoint{1.021668in}{2.120118in}}%
\pgfpathlineto{\pgfqpoint{1.008725in}{2.139614in}}%
\pgfpathlineto{\pgfqpoint{0.945088in}{2.098001in}}%
\pgfpathlineto{\pgfqpoint{0.959648in}{2.079432in}}%
\pgfpathlineto{\pgfqpoint{1.021668in}{2.120118in}}%
\pgfpathclose%
\pgfusepath{fill}%
\end{pgfscope}%
\begin{pgfscope}%
\pgfpathrectangle{\pgfqpoint{0.000000in}{0.000000in}}{\pgfqpoint{3.000000in}{3.000000in}}%
\pgfusepath{clip}%
\pgfsetbuttcap%
\pgfsetroundjoin%
\definecolor{currentfill}{rgb}{0.999109,0.073348,0.000000}%
\pgfsetfillcolor{currentfill}%
\pgfsetlinewidth{0.000000pt}%
\definecolor{currentstroke}{rgb}{0.000000,0.000000,0.000000}%
\pgfsetstrokecolor{currentstroke}%
\pgfsetdash{}{0pt}%
\pgfpathmoveto{\pgfqpoint{1.008725in}{2.139614in}}%
\pgfpathlineto{\pgfqpoint{0.995787in}{2.158838in}}%
\pgfpathlineto{\pgfqpoint{0.930533in}{2.116301in}}%
\pgfpathlineto{\pgfqpoint{0.945088in}{2.098001in}}%
\pgfpathlineto{\pgfqpoint{1.008725in}{2.139614in}}%
\pgfpathclose%
\pgfusepath{fill}%
\end{pgfscope}%
\begin{pgfscope}%
\pgfpathrectangle{\pgfqpoint{0.000000in}{0.000000in}}{\pgfqpoint{3.000000in}{3.000000in}}%
\pgfusepath{clip}%
\pgfsetbuttcap%
\pgfsetroundjoin%
\definecolor{currentfill}{rgb}{0.578748,1.000000,0.388994}%
\pgfsetfillcolor{currentfill}%
\pgfsetlinewidth{0.000000pt}%
\definecolor{currentstroke}{rgb}{0.000000,0.000000,0.000000}%
\pgfsetstrokecolor{currentstroke}%
\pgfsetdash{}{0pt}%
\pgfpathmoveto{\pgfqpoint{1.222308in}{1.666038in}}%
\pgfpathlineto{\pgfqpoint{1.207692in}{1.695571in}}%
\pgfpathlineto{\pgfqpoint{1.180681in}{1.668726in}}%
\pgfpathlineto{\pgfqpoint{1.196528in}{1.640293in}}%
\pgfpathlineto{\pgfqpoint{1.222308in}{1.666038in}}%
\pgfpathclose%
\pgfusepath{fill}%
\end{pgfscope}%
\begin{pgfscope}%
\pgfpathrectangle{\pgfqpoint{0.000000in}{0.000000in}}{\pgfqpoint{3.000000in}{3.000000in}}%
\pgfusepath{clip}%
\pgfsetbuttcap%
\pgfsetroundjoin%
\definecolor{currentfill}{rgb}{0.000000,0.849020,1.000000}%
\pgfsetfillcolor{currentfill}%
\pgfsetlinewidth{0.000000pt}%
\definecolor{currentstroke}{rgb}{0.000000,0.000000,0.000000}%
\pgfsetstrokecolor{currentstroke}%
\pgfsetdash{}{0pt}%
\pgfpathmoveto{\pgfqpoint{1.279119in}{1.411408in}}%
\pgfpathlineto{\pgfqpoint{1.262455in}{1.451010in}}%
\pgfpathlineto{\pgfqpoint{1.255960in}{1.428630in}}%
\pgfpathlineto{\pgfqpoint{1.273051in}{1.390285in}}%
\pgfpathlineto{\pgfqpoint{1.279119in}{1.411408in}}%
\pgfpathclose%
\pgfusepath{fill}%
\end{pgfscope}%
\begin{pgfscope}%
\pgfpathrectangle{\pgfqpoint{0.000000in}{0.000000in}}{\pgfqpoint{3.000000in}{3.000000in}}%
\pgfusepath{clip}%
\pgfsetbuttcap%
\pgfsetroundjoin%
\definecolor{currentfill}{rgb}{0.000000,0.064706,1.000000}%
\pgfsetfillcolor{currentfill}%
\pgfsetlinewidth{0.000000pt}%
\definecolor{currentstroke}{rgb}{0.000000,0.000000,0.000000}%
\pgfsetstrokecolor{currentstroke}%
\pgfsetdash{}{0pt}%
\pgfpathmoveto{\pgfqpoint{1.714355in}{1.135339in}}%
\pgfpathlineto{\pgfqpoint{1.729318in}{1.188809in}}%
\pgfpathlineto{\pgfqpoint{1.743709in}{1.204800in}}%
\pgfpathlineto{\pgfqpoint{1.727630in}{1.150145in}}%
\pgfpathlineto{\pgfqpoint{1.714355in}{1.135339in}}%
\pgfpathclose%
\pgfusepath{fill}%
\end{pgfscope}%
\begin{pgfscope}%
\pgfpathrectangle{\pgfqpoint{0.000000in}{0.000000in}}{\pgfqpoint{3.000000in}{3.000000in}}%
\pgfusepath{clip}%
\pgfsetbuttcap%
\pgfsetroundjoin%
\definecolor{currentfill}{rgb}{0.000000,0.000000,0.838681}%
\pgfsetfillcolor{currentfill}%
\pgfsetlinewidth{0.000000pt}%
\definecolor{currentstroke}{rgb}{0.000000,0.000000,0.000000}%
\pgfsetstrokecolor{currentstroke}%
\pgfsetdash{}{0pt}%
\pgfpathmoveto{\pgfqpoint{1.663590in}{1.049881in}}%
\pgfpathlineto{\pgfqpoint{1.675189in}{1.109816in}}%
\pgfpathlineto{\pgfqpoint{1.696734in}{1.121781in}}%
\pgfpathlineto{\pgfqpoint{1.683294in}{1.060883in}}%
\pgfpathlineto{\pgfqpoint{1.663590in}{1.049881in}}%
\pgfpathclose%
\pgfusepath{fill}%
\end{pgfscope}%
\begin{pgfscope}%
\pgfpathrectangle{\pgfqpoint{0.000000in}{0.000000in}}{\pgfqpoint{3.000000in}{3.000000in}}%
\pgfusepath{clip}%
\pgfsetbuttcap%
\pgfsetroundjoin%
\definecolor{currentfill}{rgb}{0.000000,0.676471,1.000000}%
\pgfsetfillcolor{currentfill}%
\pgfsetlinewidth{0.000000pt}%
\definecolor{currentstroke}{rgb}{0.000000,0.000000,0.000000}%
\pgfsetstrokecolor{currentstroke}%
\pgfsetdash{}{0pt}%
\pgfpathmoveto{\pgfqpoint{1.791304in}{1.335010in}}%
\pgfpathlineto{\pgfqpoint{1.808432in}{1.376020in}}%
\pgfpathlineto{\pgfqpoint{1.806732in}{1.397382in}}%
\pgfpathlineto{\pgfqpoint{1.789748in}{1.355095in}}%
\pgfpathlineto{\pgfqpoint{1.791304in}{1.335010in}}%
\pgfpathclose%
\pgfusepath{fill}%
\end{pgfscope}%
\begin{pgfscope}%
\pgfpathrectangle{\pgfqpoint{0.000000in}{0.000000in}}{\pgfqpoint{3.000000in}{3.000000in}}%
\pgfusepath{clip}%
\pgfsetbuttcap%
\pgfsetroundjoin%
\definecolor{currentfill}{rgb}{0.819102,1.000000,0.148640}%
\pgfsetfillcolor{currentfill}%
\pgfsetlinewidth{0.000000pt}%
\definecolor{currentstroke}{rgb}{0.000000,0.000000,0.000000}%
\pgfsetstrokecolor{currentstroke}%
\pgfsetdash{}{0pt}%
\pgfpathmoveto{\pgfqpoint{1.923280in}{1.731970in}}%
\pgfpathlineto{\pgfqpoint{1.938756in}{1.757473in}}%
\pgfpathlineto{\pgfqpoint{1.904948in}{1.786739in}}%
\pgfpathlineto{\pgfqpoint{1.890829in}{1.760180in}}%
\pgfpathlineto{\pgfqpoint{1.923280in}{1.731970in}}%
\pgfpathclose%
\pgfusepath{fill}%
\end{pgfscope}%
\begin{pgfscope}%
\pgfpathrectangle{\pgfqpoint{0.000000in}{0.000000in}}{\pgfqpoint{3.000000in}{3.000000in}}%
\pgfusepath{clip}%
\pgfsetbuttcap%
\pgfsetroundjoin%
\definecolor{currentfill}{rgb}{0.667299,1.000000,0.300443}%
\pgfsetfillcolor{currentfill}%
\pgfsetlinewidth{0.000000pt}%
\definecolor{currentstroke}{rgb}{0.000000,0.000000,0.000000}%
\pgfsetstrokecolor{currentstroke}%
\pgfsetdash{}{0pt}%
\pgfpathmoveto{\pgfqpoint{1.207692in}{1.695571in}}%
\pgfpathlineto{\pgfqpoint{1.193078in}{1.723868in}}%
\pgfpathlineto{\pgfqpoint{1.164833in}{1.695927in}}%
\pgfpathlineto{\pgfqpoint{1.180681in}{1.668726in}}%
\pgfpathlineto{\pgfqpoint{1.207692in}{1.695571in}}%
\pgfpathclose%
\pgfusepath{fill}%
\end{pgfscope}%
\begin{pgfscope}%
\pgfpathrectangle{\pgfqpoint{0.000000in}{0.000000in}}{\pgfqpoint{3.000000in}{3.000000in}}%
\pgfusepath{clip}%
\pgfsetbuttcap%
\pgfsetroundjoin%
\definecolor{currentfill}{rgb}{0.085389,1.000000,0.882353}%
\pgfsetfillcolor{currentfill}%
\pgfsetlinewidth{0.000000pt}%
\definecolor{currentstroke}{rgb}{0.000000,0.000000,0.000000}%
\pgfsetstrokecolor{currentstroke}%
\pgfsetdash{}{0pt}%
\pgfpathmoveto{\pgfqpoint{1.823725in}{1.436150in}}%
\pgfpathlineto{\pgfqpoint{1.840730in}{1.471981in}}%
\pgfpathlineto{\pgfqpoint{1.831375in}{1.495379in}}%
\pgfpathlineto{\pgfqpoint{1.814939in}{1.458310in}}%
\pgfpathlineto{\pgfqpoint{1.823725in}{1.436150in}}%
\pgfpathclose%
\pgfusepath{fill}%
\end{pgfscope}%
\begin{pgfscope}%
\pgfpathrectangle{\pgfqpoint{0.000000in}{0.000000in}}{\pgfqpoint{3.000000in}{3.000000in}}%
\pgfusepath{clip}%
\pgfsetbuttcap%
\pgfsetroundjoin%
\definecolor{currentfill}{rgb}{0.000000,0.000000,0.838681}%
\pgfsetfillcolor{currentfill}%
\pgfsetlinewidth{0.000000pt}%
\definecolor{currentstroke}{rgb}{0.000000,0.000000,0.000000}%
\pgfsetstrokecolor{currentstroke}%
\pgfsetdash{}{0pt}%
\pgfpathmoveto{\pgfqpoint{1.403980in}{1.057038in}}%
\pgfpathlineto{\pgfqpoint{1.391117in}{1.117600in}}%
\pgfpathlineto{\pgfqpoint{1.413857in}{1.106237in}}%
\pgfpathlineto{\pgfqpoint{1.424774in}{1.046590in}}%
\pgfpathlineto{\pgfqpoint{1.403980in}{1.057038in}}%
\pgfpathclose%
\pgfusepath{fill}%
\end{pgfscope}%
\begin{pgfscope}%
\pgfpathrectangle{\pgfqpoint{0.000000in}{0.000000in}}{\pgfqpoint{3.000000in}{3.000000in}}%
\pgfusepath{clip}%
\pgfsetbuttcap%
\pgfsetroundjoin%
\definecolor{currentfill}{rgb}{0.401645,1.000000,0.566097}%
\pgfsetfillcolor{currentfill}%
\pgfsetlinewidth{0.000000pt}%
\definecolor{currentstroke}{rgb}{0.000000,0.000000,0.000000}%
\pgfsetstrokecolor{currentstroke}%
\pgfsetdash{}{0pt}%
\pgfpathmoveto{\pgfqpoint{1.864263in}{1.562430in}}%
\pgfpathlineto{\pgfqpoint{1.880714in}{1.593066in}}%
\pgfpathlineto{\pgfqpoint{1.861375in}{1.618895in}}%
\pgfpathlineto{\pgfqpoint{1.845899in}{1.587094in}}%
\pgfpathlineto{\pgfqpoint{1.864263in}{1.562430in}}%
\pgfpathclose%
\pgfusepath{fill}%
\end{pgfscope}%
\begin{pgfscope}%
\pgfpathrectangle{\pgfqpoint{0.000000in}{0.000000in}}{\pgfqpoint{3.000000in}{3.000000in}}%
\pgfusepath{clip}%
\pgfsetbuttcap%
\pgfsetroundjoin%
\definecolor{currentfill}{rgb}{0.895003,1.000000,0.072739}%
\pgfsetfillcolor{currentfill}%
\pgfsetlinewidth{0.000000pt}%
\definecolor{currentstroke}{rgb}{0.000000,0.000000,0.000000}%
\pgfsetstrokecolor{currentstroke}%
\pgfsetdash{}{0pt}%
\pgfpathmoveto{\pgfqpoint{1.938756in}{1.757473in}}%
\pgfpathlineto{\pgfqpoint{1.954233in}{1.782098in}}%
\pgfpathlineto{\pgfqpoint{1.919064in}{1.812415in}}%
\pgfpathlineto{\pgfqpoint{1.904948in}{1.786739in}}%
\pgfpathlineto{\pgfqpoint{1.938756in}{1.757473in}}%
\pgfpathclose%
\pgfusepath{fill}%
\end{pgfscope}%
\begin{pgfscope}%
\pgfpathrectangle{\pgfqpoint{0.000000in}{0.000000in}}{\pgfqpoint{3.000000in}{3.000000in}}%
\pgfusepath{clip}%
\pgfsetbuttcap%
\pgfsetroundjoin%
\definecolor{currentfill}{rgb}{0.000000,0.064706,1.000000}%
\pgfsetfillcolor{currentfill}%
\pgfsetlinewidth{0.000000pt}%
\definecolor{currentstroke}{rgb}{0.000000,0.000000,0.000000}%
\pgfsetstrokecolor{currentstroke}%
\pgfsetdash{}{0pt}%
\pgfpathmoveto{\pgfqpoint{1.357371in}{1.145093in}}%
\pgfpathlineto{\pgfqpoint{1.341621in}{1.199345in}}%
\pgfpathlineto{\pgfqpoint{1.357631in}{1.183759in}}%
\pgfpathlineto{\pgfqpoint{1.372135in}{1.130663in}}%
\pgfpathlineto{\pgfqpoint{1.357371in}{1.145093in}}%
\pgfpathclose%
\pgfusepath{fill}%
\end{pgfscope}%
\begin{pgfscope}%
\pgfpathrectangle{\pgfqpoint{0.000000in}{0.000000in}}{\pgfqpoint{3.000000in}{3.000000in}}%
\pgfusepath{clip}%
\pgfsetbuttcap%
\pgfsetroundjoin%
\definecolor{currentfill}{rgb}{0.300443,1.000000,0.667299}%
\pgfsetfillcolor{currentfill}%
\pgfsetlinewidth{0.000000pt}%
\definecolor{currentstroke}{rgb}{0.000000,0.000000,0.000000}%
\pgfsetstrokecolor{currentstroke}%
\pgfsetdash{}{0pt}%
\pgfpathmoveto{\pgfqpoint{1.244056in}{1.545791in}}%
\pgfpathlineto{\pgfqpoint{1.228215in}{1.579043in}}%
\pgfpathlineto{\pgfqpoint{1.212424in}{1.553912in}}%
\pgfpathlineto{\pgfqpoint{1.229108in}{1.521852in}}%
\pgfpathlineto{\pgfqpoint{1.244056in}{1.545791in}}%
\pgfpathclose%
\pgfusepath{fill}%
\end{pgfscope}%
\begin{pgfscope}%
\pgfpathrectangle{\pgfqpoint{0.000000in}{0.000000in}}{\pgfqpoint{3.000000in}{3.000000in}}%
\pgfusepath{clip}%
\pgfsetbuttcap%
\pgfsetroundjoin%
\definecolor{currentfill}{rgb}{0.000000,0.300000,1.000000}%
\pgfsetfillcolor{currentfill}%
\pgfsetlinewidth{0.000000pt}%
\definecolor{currentstroke}{rgb}{0.000000,0.000000,0.000000}%
\pgfsetstrokecolor{currentstroke}%
\pgfsetdash{}{0pt}%
\pgfpathmoveto{\pgfqpoint{1.743709in}{1.204800in}}%
\pgfpathlineto{\pgfqpoint{1.759810in}{1.252826in}}%
\pgfpathlineto{\pgfqpoint{1.769845in}{1.271004in}}%
\pgfpathlineto{\pgfqpoint{1.753036in}{1.221728in}}%
\pgfpathlineto{\pgfqpoint{1.743709in}{1.204800in}}%
\pgfpathclose%
\pgfusepath{fill}%
\end{pgfscope}%
\begin{pgfscope}%
\pgfpathrectangle{\pgfqpoint{0.000000in}{0.000000in}}{\pgfqpoint{3.000000in}{3.000000in}}%
\pgfusepath{clip}%
\pgfsetbuttcap%
\pgfsetroundjoin%
\definecolor{currentfill}{rgb}{0.000000,0.503922,1.000000}%
\pgfsetfillcolor{currentfill}%
\pgfsetlinewidth{0.000000pt}%
\definecolor{currentstroke}{rgb}{0.000000,0.000000,0.000000}%
\pgfsetstrokecolor{currentstroke}%
\pgfsetdash{}{0pt}%
\pgfpathmoveto{\pgfqpoint{1.769845in}{1.271004in}}%
\pgfpathlineto{\pgfqpoint{1.786674in}{1.315006in}}%
\pgfpathlineto{\pgfqpoint{1.791304in}{1.335010in}}%
\pgfpathlineto{\pgfqpoint{1.774192in}{1.289728in}}%
\pgfpathlineto{\pgfqpoint{1.769845in}{1.271004in}}%
\pgfpathclose%
\pgfusepath{fill}%
\end{pgfscope}%
\begin{pgfscope}%
\pgfpathrectangle{\pgfqpoint{0.000000in}{0.000000in}}{\pgfqpoint{3.000000in}{3.000000in}}%
\pgfusepath{clip}%
\pgfsetbuttcap%
\pgfsetroundjoin%
\definecolor{currentfill}{rgb}{0.743201,1.000000,0.224541}%
\pgfsetfillcolor{currentfill}%
\pgfsetlinewidth{0.000000pt}%
\definecolor{currentstroke}{rgb}{0.000000,0.000000,0.000000}%
\pgfsetstrokecolor{currentstroke}%
\pgfsetdash{}{0pt}%
\pgfpathmoveto{\pgfqpoint{1.193078in}{1.723868in}}%
\pgfpathlineto{\pgfqpoint{1.178466in}{1.751068in}}%
\pgfpathlineto{\pgfqpoint{1.148984in}{1.722034in}}%
\pgfpathlineto{\pgfqpoint{1.164833in}{1.695927in}}%
\pgfpathlineto{\pgfqpoint{1.193078in}{1.723868in}}%
\pgfpathclose%
\pgfusepath{fill}%
\end{pgfscope}%
\begin{pgfscope}%
\pgfpathrectangle{\pgfqpoint{0.000000in}{0.000000in}}{\pgfqpoint{3.000000in}{3.000000in}}%
\pgfusepath{clip}%
\pgfsetbuttcap%
\pgfsetroundjoin%
\definecolor{currentfill}{rgb}{0.958254,0.973856,0.009488}%
\pgfsetfillcolor{currentfill}%
\pgfsetlinewidth{0.000000pt}%
\definecolor{currentstroke}{rgb}{0.000000,0.000000,0.000000}%
\pgfsetstrokecolor{currentstroke}%
\pgfsetdash{}{0pt}%
\pgfpathmoveto{\pgfqpoint{1.954233in}{1.782098in}}%
\pgfpathlineto{\pgfqpoint{1.969709in}{1.805930in}}%
\pgfpathlineto{\pgfqpoint{1.933176in}{1.837295in}}%
\pgfpathlineto{\pgfqpoint{1.919064in}{1.812415in}}%
\pgfpathlineto{\pgfqpoint{1.954233in}{1.782098in}}%
\pgfpathclose%
\pgfusepath{fill}%
\end{pgfscope}%
\begin{pgfscope}%
\pgfpathrectangle{\pgfqpoint{0.000000in}{0.000000in}}{\pgfqpoint{3.000000in}{3.000000in}}%
\pgfusepath{clip}%
\pgfsetbuttcap%
\pgfsetroundjoin%
\definecolor{currentfill}{rgb}{0.000000,0.676471,1.000000}%
\pgfsetfillcolor{currentfill}%
\pgfsetlinewidth{0.000000pt}%
\definecolor{currentstroke}{rgb}{0.000000,0.000000,0.000000}%
\pgfsetstrokecolor{currentstroke}%
\pgfsetdash{}{0pt}%
\pgfpathmoveto{\pgfqpoint{1.290129in}{1.348422in}}%
\pgfpathlineto{\pgfqpoint{1.273051in}{1.390285in}}%
\pgfpathlineto{\pgfqpoint{1.273552in}{1.368891in}}%
\pgfpathlineto{\pgfqpoint{1.290634in}{1.328309in}}%
\pgfpathlineto{\pgfqpoint{1.290129in}{1.348422in}}%
\pgfpathclose%
\pgfusepath{fill}%
\end{pgfscope}%
\begin{pgfscope}%
\pgfpathrectangle{\pgfqpoint{0.000000in}{0.000000in}}{\pgfqpoint{3.000000in}{3.000000in}}%
\pgfusepath{clip}%
\pgfsetbuttcap%
\pgfsetroundjoin%
\definecolor{currentfill}{rgb}{1.000000,0.886710,0.000000}%
\pgfsetfillcolor{currentfill}%
\pgfsetlinewidth{0.000000pt}%
\definecolor{currentstroke}{rgb}{0.000000,0.000000,0.000000}%
\pgfsetstrokecolor{currentstroke}%
\pgfsetdash{}{0pt}%
\pgfpathmoveto{\pgfqpoint{1.969709in}{1.805930in}}%
\pgfpathlineto{\pgfqpoint{1.985185in}{1.829043in}}%
\pgfpathlineto{\pgfqpoint{1.947283in}{1.861453in}}%
\pgfpathlineto{\pgfqpoint{1.933176in}{1.837295in}}%
\pgfpathlineto{\pgfqpoint{1.969709in}{1.805930in}}%
\pgfpathclose%
\pgfusepath{fill}%
\end{pgfscope}%
\begin{pgfscope}%
\pgfpathrectangle{\pgfqpoint{0.000000in}{0.000000in}}{\pgfqpoint{3.000000in}{3.000000in}}%
\pgfusepath{clip}%
\pgfsetbuttcap%
\pgfsetroundjoin%
\definecolor{currentfill}{rgb}{0.000000,0.000000,0.838681}%
\pgfsetfillcolor{currentfill}%
\pgfsetlinewidth{0.000000pt}%
\definecolor{currentstroke}{rgb}{0.000000,0.000000,0.000000}%
\pgfsetstrokecolor{currentstroke}%
\pgfsetdash{}{0pt}%
\pgfpathmoveto{\pgfqpoint{1.640791in}{1.040629in}}%
\pgfpathlineto{\pgfqpoint{1.650250in}{1.099753in}}%
\pgfpathlineto{\pgfqpoint{1.675189in}{1.109816in}}%
\pgfpathlineto{\pgfqpoint{1.663590in}{1.049881in}}%
\pgfpathlineto{\pgfqpoint{1.640791in}{1.040629in}}%
\pgfpathclose%
\pgfusepath{fill}%
\end{pgfscope}%
\begin{pgfscope}%
\pgfpathrectangle{\pgfqpoint{0.000000in}{0.000000in}}{\pgfqpoint{3.000000in}{3.000000in}}%
\pgfusepath{clip}%
\pgfsetbuttcap%
\pgfsetroundjoin%
\definecolor{currentfill}{rgb}{0.490196,1.000000,0.477546}%
\pgfsetfillcolor{currentfill}%
\pgfsetlinewidth{0.000000pt}%
\definecolor{currentstroke}{rgb}{0.000000,0.000000,0.000000}%
\pgfsetstrokecolor{currentstroke}%
\pgfsetdash{}{0pt}%
\pgfpathmoveto{\pgfqpoint{1.880714in}{1.593066in}}%
\pgfpathlineto{\pgfqpoint{1.897171in}{1.622112in}}%
\pgfpathlineto{\pgfqpoint{1.876850in}{1.649102in}}%
\pgfpathlineto{\pgfqpoint{1.861375in}{1.618895in}}%
\pgfpathlineto{\pgfqpoint{1.880714in}{1.593066in}}%
\pgfpathclose%
\pgfusepath{fill}%
\end{pgfscope}%
\begin{pgfscope}%
\pgfpathrectangle{\pgfqpoint{0.000000in}{0.000000in}}{\pgfqpoint{3.000000in}{3.000000in}}%
\pgfusepath{clip}%
\pgfsetbuttcap%
\pgfsetroundjoin%
\definecolor{currentfill}{rgb}{0.085389,1.000000,0.882353}%
\pgfsetfillcolor{currentfill}%
\pgfsetlinewidth{0.000000pt}%
\definecolor{currentstroke}{rgb}{0.000000,0.000000,0.000000}%
\pgfsetstrokecolor{currentstroke}%
\pgfsetdash{}{0pt}%
\pgfpathmoveto{\pgfqpoint{1.262455in}{1.451010in}}%
\pgfpathlineto{\pgfqpoint{1.245785in}{1.487672in}}%
\pgfpathlineto{\pgfqpoint{1.238858in}{1.464040in}}%
\pgfpathlineto{\pgfqpoint{1.255960in}{1.428630in}}%
\pgfpathlineto{\pgfqpoint{1.262455in}{1.451010in}}%
\pgfpathclose%
\pgfusepath{fill}%
\end{pgfscope}%
\begin{pgfscope}%
\pgfpathrectangle{\pgfqpoint{0.000000in}{0.000000in}}{\pgfqpoint{3.000000in}{3.000000in}}%
\pgfusepath{clip}%
\pgfsetbuttcap%
\pgfsetroundjoin%
\definecolor{currentfill}{rgb}{0.000000,0.300000,1.000000}%
\pgfsetfillcolor{currentfill}%
\pgfsetlinewidth{0.000000pt}%
\definecolor{currentstroke}{rgb}{0.000000,0.000000,0.000000}%
\pgfsetstrokecolor{currentstroke}%
\pgfsetdash{}{0pt}%
\pgfpathmoveto{\pgfqpoint{1.330576in}{1.216007in}}%
\pgfpathlineto{\pgfqpoint{1.313957in}{1.264861in}}%
\pgfpathlineto{\pgfqpoint{1.325846in}{1.246968in}}%
\pgfpathlineto{\pgfqpoint{1.341621in}{1.199345in}}%
\pgfpathlineto{\pgfqpoint{1.330576in}{1.216007in}}%
\pgfpathclose%
\pgfusepath{fill}%
\end{pgfscope}%
\begin{pgfscope}%
\pgfpathrectangle{\pgfqpoint{0.000000in}{0.000000in}}{\pgfqpoint{3.000000in}{3.000000in}}%
\pgfusepath{clip}%
\pgfsetbuttcap%
\pgfsetroundjoin%
\definecolor{currentfill}{rgb}{0.819102,1.000000,0.148640}%
\pgfsetfillcolor{currentfill}%
\pgfsetlinewidth{0.000000pt}%
\definecolor{currentstroke}{rgb}{0.000000,0.000000,0.000000}%
\pgfsetstrokecolor{currentstroke}%
\pgfsetdash{}{0pt}%
\pgfpathmoveto{\pgfqpoint{1.178466in}{1.751068in}}%
\pgfpathlineto{\pgfqpoint{1.163857in}{1.777286in}}%
\pgfpathlineto{\pgfqpoint{1.133133in}{1.747164in}}%
\pgfpathlineto{\pgfqpoint{1.148984in}{1.722034in}}%
\pgfpathlineto{\pgfqpoint{1.178466in}{1.751068in}}%
\pgfpathclose%
\pgfusepath{fill}%
\end{pgfscope}%
\begin{pgfscope}%
\pgfpathrectangle{\pgfqpoint{0.000000in}{0.000000in}}{\pgfqpoint{3.000000in}{3.000000in}}%
\pgfusepath{clip}%
\pgfsetbuttcap%
\pgfsetroundjoin%
\definecolor{currentfill}{rgb}{0.000000,0.503922,1.000000}%
\pgfsetfillcolor{currentfill}%
\pgfsetlinewidth{0.000000pt}%
\definecolor{currentstroke}{rgb}{0.000000,0.000000,0.000000}%
\pgfsetstrokecolor{currentstroke}%
\pgfsetdash{}{0pt}%
\pgfpathmoveto{\pgfqpoint{1.307699in}{1.283454in}}%
\pgfpathlineto{\pgfqpoint{1.290634in}{1.328309in}}%
\pgfpathlineto{\pgfqpoint{1.297316in}{1.308443in}}%
\pgfpathlineto{\pgfqpoint{1.313957in}{1.264861in}}%
\pgfpathlineto{\pgfqpoint{1.307699in}{1.283454in}}%
\pgfpathclose%
\pgfusepath{fill}%
\end{pgfscope}%
\begin{pgfscope}%
\pgfpathrectangle{\pgfqpoint{0.000000in}{0.000000in}}{\pgfqpoint{3.000000in}{3.000000in}}%
\pgfusepath{clip}%
\pgfsetbuttcap%
\pgfsetroundjoin%
\definecolor{currentfill}{rgb}{1.000000,0.814089,0.000000}%
\pgfsetfillcolor{currentfill}%
\pgfsetlinewidth{0.000000pt}%
\definecolor{currentstroke}{rgb}{0.000000,0.000000,0.000000}%
\pgfsetstrokecolor{currentstroke}%
\pgfsetdash{}{0pt}%
\pgfpathmoveto{\pgfqpoint{1.985185in}{1.829043in}}%
\pgfpathlineto{\pgfqpoint{2.000660in}{1.851502in}}%
\pgfpathlineto{\pgfqpoint{1.961387in}{1.884952in}}%
\pgfpathlineto{\pgfqpoint{1.947283in}{1.861453in}}%
\pgfpathlineto{\pgfqpoint{1.985185in}{1.829043in}}%
\pgfpathclose%
\pgfusepath{fill}%
\end{pgfscope}%
\begin{pgfscope}%
\pgfpathrectangle{\pgfqpoint{0.000000in}{0.000000in}}{\pgfqpoint{3.000000in}{3.000000in}}%
\pgfusepath{clip}%
\pgfsetbuttcap%
\pgfsetroundjoin%
\definecolor{currentfill}{rgb}{0.401645,1.000000,0.566097}%
\pgfsetfillcolor{currentfill}%
\pgfsetlinewidth{0.000000pt}%
\definecolor{currentstroke}{rgb}{0.000000,0.000000,0.000000}%
\pgfsetstrokecolor{currentstroke}%
\pgfsetdash{}{0pt}%
\pgfpathmoveto{\pgfqpoint{1.228215in}{1.579043in}}%
\pgfpathlineto{\pgfqpoint{1.212373in}{1.610465in}}%
\pgfpathlineto{\pgfqpoint{1.195733in}{1.584144in}}%
\pgfpathlineto{\pgfqpoint{1.212424in}{1.553912in}}%
\pgfpathlineto{\pgfqpoint{1.228215in}{1.579043in}}%
\pgfpathclose%
\pgfusepath{fill}%
\end{pgfscope}%
\begin{pgfscope}%
\pgfpathrectangle{\pgfqpoint{0.000000in}{0.000000in}}{\pgfqpoint{3.000000in}{3.000000in}}%
\pgfusepath{clip}%
\pgfsetbuttcap%
\pgfsetroundjoin%
\definecolor{currentfill}{rgb}{0.199241,1.000000,0.768501}%
\pgfsetfillcolor{currentfill}%
\pgfsetlinewidth{0.000000pt}%
\definecolor{currentstroke}{rgb}{0.000000,0.000000,0.000000}%
\pgfsetstrokecolor{currentstroke}%
\pgfsetdash{}{0pt}%
\pgfpathmoveto{\pgfqpoint{1.840730in}{1.471981in}}%
\pgfpathlineto{\pgfqpoint{1.857744in}{1.505332in}}%
\pgfpathlineto{\pgfqpoint{1.847816in}{1.529965in}}%
\pgfpathlineto{\pgfqpoint{1.831375in}{1.495379in}}%
\pgfpathlineto{\pgfqpoint{1.840730in}{1.471981in}}%
\pgfpathclose%
\pgfusepath{fill}%
\end{pgfscope}%
\begin{pgfscope}%
\pgfpathrectangle{\pgfqpoint{0.000000in}{0.000000in}}{\pgfqpoint{3.000000in}{3.000000in}}%
\pgfusepath{clip}%
\pgfsetbuttcap%
\pgfsetroundjoin%
\definecolor{currentfill}{rgb}{0.000000,0.000000,0.838681}%
\pgfsetfillcolor{currentfill}%
\pgfsetlinewidth{0.000000pt}%
\definecolor{currentstroke}{rgb}{0.000000,0.000000,0.000000}%
\pgfsetstrokecolor{currentstroke}%
\pgfsetdash{}{0pt}%
\pgfpathmoveto{\pgfqpoint{1.424774in}{1.046590in}}%
\pgfpathlineto{\pgfqpoint{1.413857in}{1.106237in}}%
\pgfpathlineto{\pgfqpoint{1.439793in}{1.096867in}}%
\pgfpathlineto{\pgfqpoint{1.448482in}{1.037976in}}%
\pgfpathlineto{\pgfqpoint{1.424774in}{1.046590in}}%
\pgfpathclose%
\pgfusepath{fill}%
\end{pgfscope}%
\begin{pgfscope}%
\pgfpathrectangle{\pgfqpoint{0.000000in}{0.000000in}}{\pgfqpoint{3.000000in}{3.000000in}}%
\pgfusepath{clip}%
\pgfsetbuttcap%
\pgfsetroundjoin%
\definecolor{currentfill}{rgb}{0.000000,0.849020,1.000000}%
\pgfsetfillcolor{currentfill}%
\pgfsetlinewidth{0.000000pt}%
\definecolor{currentstroke}{rgb}{0.000000,0.000000,0.000000}%
\pgfsetstrokecolor{currentstroke}%
\pgfsetdash{}{0pt}%
\pgfpathmoveto{\pgfqpoint{1.808432in}{1.376020in}}%
\pgfpathlineto{\pgfqpoint{1.825575in}{1.413514in}}%
\pgfpathlineto{\pgfqpoint{1.823725in}{1.436150in}}%
\pgfpathlineto{\pgfqpoint{1.806732in}{1.397382in}}%
\pgfpathlineto{\pgfqpoint{1.808432in}{1.376020in}}%
\pgfpathclose%
\pgfusepath{fill}%
\end{pgfscope}%
\begin{pgfscope}%
\pgfpathrectangle{\pgfqpoint{0.000000in}{0.000000in}}{\pgfqpoint{3.000000in}{3.000000in}}%
\pgfusepath{clip}%
\pgfsetbuttcap%
\pgfsetroundjoin%
\definecolor{currentfill}{rgb}{0.000000,0.064706,1.000000}%
\pgfsetfillcolor{currentfill}%
\pgfsetlinewidth{0.000000pt}%
\definecolor{currentstroke}{rgb}{0.000000,0.000000,0.000000}%
\pgfsetstrokecolor{currentstroke}%
\pgfsetdash{}{0pt}%
\pgfpathmoveto{\pgfqpoint{1.696734in}{1.121781in}}%
\pgfpathlineto{\pgfqpoint{1.710199in}{1.174163in}}%
\pgfpathlineto{\pgfqpoint{1.729318in}{1.188809in}}%
\pgfpathlineto{\pgfqpoint{1.714355in}{1.135339in}}%
\pgfpathlineto{\pgfqpoint{1.696734in}{1.121781in}}%
\pgfpathclose%
\pgfusepath{fill}%
\end{pgfscope}%
\begin{pgfscope}%
\pgfpathrectangle{\pgfqpoint{0.000000in}{0.000000in}}{\pgfqpoint{3.000000in}{3.000000in}}%
\pgfusepath{clip}%
\pgfsetbuttcap%
\pgfsetroundjoin%
\definecolor{currentfill}{rgb}{1.000000,0.741467,0.000000}%
\pgfsetfillcolor{currentfill}%
\pgfsetlinewidth{0.000000pt}%
\definecolor{currentstroke}{rgb}{0.000000,0.000000,0.000000}%
\pgfsetstrokecolor{currentstroke}%
\pgfsetdash{}{0pt}%
\pgfpathmoveto{\pgfqpoint{2.000660in}{1.851502in}}%
\pgfpathlineto{\pgfqpoint{2.016135in}{1.873363in}}%
\pgfpathlineto{\pgfqpoint{1.975487in}{1.907851in}}%
\pgfpathlineto{\pgfqpoint{1.961387in}{1.884952in}}%
\pgfpathlineto{\pgfqpoint{2.000660in}{1.851502in}}%
\pgfpathclose%
\pgfusepath{fill}%
\end{pgfscope}%
\begin{pgfscope}%
\pgfpathrectangle{\pgfqpoint{0.000000in}{0.000000in}}{\pgfqpoint{3.000000in}{3.000000in}}%
\pgfusepath{clip}%
\pgfsetbuttcap%
\pgfsetroundjoin%
\definecolor{currentfill}{rgb}{0.895003,1.000000,0.072739}%
\pgfsetfillcolor{currentfill}%
\pgfsetlinewidth{0.000000pt}%
\definecolor{currentstroke}{rgb}{0.000000,0.000000,0.000000}%
\pgfsetstrokecolor{currentstroke}%
\pgfsetdash{}{0pt}%
\pgfpathmoveto{\pgfqpoint{1.163857in}{1.777286in}}%
\pgfpathlineto{\pgfqpoint{1.149250in}{1.802624in}}%
\pgfpathlineto{\pgfqpoint{1.117281in}{1.771416in}}%
\pgfpathlineto{\pgfqpoint{1.133133in}{1.747164in}}%
\pgfpathlineto{\pgfqpoint{1.163857in}{1.777286in}}%
\pgfpathclose%
\pgfusepath{fill}%
\end{pgfscope}%
\begin{pgfscope}%
\pgfpathrectangle{\pgfqpoint{0.000000in}{0.000000in}}{\pgfqpoint{3.000000in}{3.000000in}}%
\pgfusepath{clip}%
\pgfsetbuttcap%
\pgfsetroundjoin%
\definecolor{currentfill}{rgb}{0.578748,1.000000,0.388994}%
\pgfsetfillcolor{currentfill}%
\pgfsetlinewidth{0.000000pt}%
\definecolor{currentstroke}{rgb}{0.000000,0.000000,0.000000}%
\pgfsetstrokecolor{currentstroke}%
\pgfsetdash{}{0pt}%
\pgfpathmoveto{\pgfqpoint{1.897171in}{1.622112in}}%
\pgfpathlineto{\pgfqpoint{1.913632in}{1.649764in}}%
\pgfpathlineto{\pgfqpoint{1.892327in}{1.677912in}}%
\pgfpathlineto{\pgfqpoint{1.876850in}{1.649102in}}%
\pgfpathlineto{\pgfqpoint{1.897171in}{1.622112in}}%
\pgfpathclose%
\pgfusepath{fill}%
\end{pgfscope}%
\begin{pgfscope}%
\pgfpathrectangle{\pgfqpoint{0.000000in}{0.000000in}}{\pgfqpoint{3.000000in}{3.000000in}}%
\pgfusepath{clip}%
\pgfsetbuttcap%
\pgfsetroundjoin%
\definecolor{currentfill}{rgb}{1.000000,0.668845,0.000000}%
\pgfsetfillcolor{currentfill}%
\pgfsetlinewidth{0.000000pt}%
\definecolor{currentstroke}{rgb}{0.000000,0.000000,0.000000}%
\pgfsetstrokecolor{currentstroke}%
\pgfsetdash{}{0pt}%
\pgfpathmoveto{\pgfqpoint{2.016135in}{1.873363in}}%
\pgfpathlineto{\pgfqpoint{2.031608in}{1.894677in}}%
\pgfpathlineto{\pgfqpoint{1.989582in}{1.930198in}}%
\pgfpathlineto{\pgfqpoint{1.975487in}{1.907851in}}%
\pgfpathlineto{\pgfqpoint{2.016135in}{1.873363in}}%
\pgfpathclose%
\pgfusepath{fill}%
\end{pgfscope}%
\begin{pgfscope}%
\pgfpathrectangle{\pgfqpoint{0.000000in}{0.000000in}}{\pgfqpoint{3.000000in}{3.000000in}}%
\pgfusepath{clip}%
\pgfsetbuttcap%
\pgfsetroundjoin%
\definecolor{currentfill}{rgb}{0.958254,0.973856,0.009488}%
\pgfsetfillcolor{currentfill}%
\pgfsetlinewidth{0.000000pt}%
\definecolor{currentstroke}{rgb}{0.000000,0.000000,0.000000}%
\pgfsetstrokecolor{currentstroke}%
\pgfsetdash{}{0pt}%
\pgfpathmoveto{\pgfqpoint{1.149250in}{1.802624in}}%
\pgfpathlineto{\pgfqpoint{1.134646in}{1.827166in}}%
\pgfpathlineto{\pgfqpoint{1.101427in}{1.794877in}}%
\pgfpathlineto{\pgfqpoint{1.117281in}{1.771416in}}%
\pgfpathlineto{\pgfqpoint{1.149250in}{1.802624in}}%
\pgfpathclose%
\pgfusepath{fill}%
\end{pgfscope}%
\begin{pgfscope}%
\pgfpathrectangle{\pgfqpoint{0.000000in}{0.000000in}}{\pgfqpoint{3.000000in}{3.000000in}}%
\pgfusepath{clip}%
\pgfsetbuttcap%
\pgfsetroundjoin%
\definecolor{currentfill}{rgb}{0.000000,0.000000,0.838681}%
\pgfsetfillcolor{currentfill}%
\pgfsetlinewidth{0.000000pt}%
\definecolor{currentstroke}{rgb}{0.000000,0.000000,0.000000}%
\pgfsetstrokecolor{currentstroke}%
\pgfsetdash{}{0pt}%
\pgfpathmoveto{\pgfqpoint{1.615462in}{1.033364in}}%
\pgfpathlineto{\pgfqpoint{1.622538in}{1.091850in}}%
\pgfpathlineto{\pgfqpoint{1.650250in}{1.099753in}}%
\pgfpathlineto{\pgfqpoint{1.640791in}{1.040629in}}%
\pgfpathlineto{\pgfqpoint{1.615462in}{1.033364in}}%
\pgfpathclose%
\pgfusepath{fill}%
\end{pgfscope}%
\begin{pgfscope}%
\pgfpathrectangle{\pgfqpoint{0.000000in}{0.000000in}}{\pgfqpoint{3.000000in}{3.000000in}}%
\pgfusepath{clip}%
\pgfsetbuttcap%
\pgfsetroundjoin%
\definecolor{currentfill}{rgb}{1.000000,0.610748,0.000000}%
\pgfsetfillcolor{currentfill}%
\pgfsetlinewidth{0.000000pt}%
\definecolor{currentstroke}{rgb}{0.000000,0.000000,0.000000}%
\pgfsetstrokecolor{currentstroke}%
\pgfsetdash{}{0pt}%
\pgfpathmoveto{\pgfqpoint{2.031608in}{1.894677in}}%
\pgfpathlineto{\pgfqpoint{2.047081in}{1.915487in}}%
\pgfpathlineto{\pgfqpoint{2.003674in}{1.952038in}}%
\pgfpathlineto{\pgfqpoint{1.989582in}{1.930198in}}%
\pgfpathlineto{\pgfqpoint{2.031608in}{1.894677in}}%
\pgfpathclose%
\pgfusepath{fill}%
\end{pgfscope}%
\begin{pgfscope}%
\pgfpathrectangle{\pgfqpoint{0.000000in}{0.000000in}}{\pgfqpoint{3.000000in}{3.000000in}}%
\pgfusepath{clip}%
\pgfsetbuttcap%
\pgfsetroundjoin%
\definecolor{currentfill}{rgb}{0.000000,0.064706,1.000000}%
\pgfsetfillcolor{currentfill}%
\pgfsetlinewidth{0.000000pt}%
\definecolor{currentstroke}{rgb}{0.000000,0.000000,0.000000}%
\pgfsetstrokecolor{currentstroke}%
\pgfsetdash{}{0pt}%
\pgfpathmoveto{\pgfqpoint{1.372135in}{1.130663in}}%
\pgfpathlineto{\pgfqpoint{1.357631in}{1.183759in}}%
\pgfpathlineto{\pgfqpoint{1.378229in}{1.169645in}}%
\pgfpathlineto{\pgfqpoint{1.391117in}{1.117600in}}%
\pgfpathlineto{\pgfqpoint{1.372135in}{1.130663in}}%
\pgfpathclose%
\pgfusepath{fill}%
\end{pgfscope}%
\begin{pgfscope}%
\pgfpathrectangle{\pgfqpoint{0.000000in}{0.000000in}}{\pgfqpoint{3.000000in}{3.000000in}}%
\pgfusepath{clip}%
\pgfsetbuttcap%
\pgfsetroundjoin%
\definecolor{currentfill}{rgb}{0.490196,1.000000,0.477546}%
\pgfsetfillcolor{currentfill}%
\pgfsetlinewidth{0.000000pt}%
\definecolor{currentstroke}{rgb}{0.000000,0.000000,0.000000}%
\pgfsetstrokecolor{currentstroke}%
\pgfsetdash{}{0pt}%
\pgfpathmoveto{\pgfqpoint{1.212373in}{1.610465in}}%
\pgfpathlineto{\pgfqpoint{1.196528in}{1.640293in}}%
\pgfpathlineto{\pgfqpoint{1.179035in}{1.612787in}}%
\pgfpathlineto{\pgfqpoint{1.195733in}{1.584144in}}%
\pgfpathlineto{\pgfqpoint{1.212373in}{1.610465in}}%
\pgfpathclose%
\pgfusepath{fill}%
\end{pgfscope}%
\begin{pgfscope}%
\pgfpathrectangle{\pgfqpoint{0.000000in}{0.000000in}}{\pgfqpoint{3.000000in}{3.000000in}}%
\pgfusepath{clip}%
\pgfsetbuttcap%
\pgfsetroundjoin%
\definecolor{currentfill}{rgb}{1.000000,0.538126,0.000000}%
\pgfsetfillcolor{currentfill}%
\pgfsetlinewidth{0.000000pt}%
\definecolor{currentstroke}{rgb}{0.000000,0.000000,0.000000}%
\pgfsetstrokecolor{currentstroke}%
\pgfsetdash{}{0pt}%
\pgfpathmoveto{\pgfqpoint{2.047081in}{1.915487in}}%
\pgfpathlineto{\pgfqpoint{2.062554in}{1.935832in}}%
\pgfpathlineto{\pgfqpoint{2.017761in}{1.973410in}}%
\pgfpathlineto{\pgfqpoint{2.003674in}{1.952038in}}%
\pgfpathlineto{\pgfqpoint{2.047081in}{1.915487in}}%
\pgfpathclose%
\pgfusepath{fill}%
\end{pgfscope}%
\begin{pgfscope}%
\pgfpathrectangle{\pgfqpoint{0.000000in}{0.000000in}}{\pgfqpoint{3.000000in}{3.000000in}}%
\pgfusepath{clip}%
\pgfsetbuttcap%
\pgfsetroundjoin%
\definecolor{currentfill}{rgb}{0.000000,0.000000,0.838681}%
\pgfsetfillcolor{currentfill}%
\pgfsetlinewidth{0.000000pt}%
\definecolor{currentstroke}{rgb}{0.000000,0.000000,0.000000}%
\pgfsetstrokecolor{currentstroke}%
\pgfsetdash{}{0pt}%
\pgfpathmoveto{\pgfqpoint{1.448482in}{1.037976in}}%
\pgfpathlineto{\pgfqpoint{1.439793in}{1.096867in}}%
\pgfpathlineto{\pgfqpoint{1.468279in}{1.089733in}}%
\pgfpathlineto{\pgfqpoint{1.474515in}{1.031419in}}%
\pgfpathlineto{\pgfqpoint{1.448482in}{1.037976in}}%
\pgfpathclose%
\pgfusepath{fill}%
\end{pgfscope}%
\begin{pgfscope}%
\pgfpathrectangle{\pgfqpoint{0.000000in}{0.000000in}}{\pgfqpoint{3.000000in}{3.000000in}}%
\pgfusepath{clip}%
\pgfsetbuttcap%
\pgfsetroundjoin%
\definecolor{currentfill}{rgb}{0.199241,1.000000,0.768501}%
\pgfsetfillcolor{currentfill}%
\pgfsetlinewidth{0.000000pt}%
\definecolor{currentstroke}{rgb}{0.000000,0.000000,0.000000}%
\pgfsetstrokecolor{currentstroke}%
\pgfsetdash{}{0pt}%
\pgfpathmoveto{\pgfqpoint{1.245785in}{1.487672in}}%
\pgfpathlineto{\pgfqpoint{1.229108in}{1.521852in}}%
\pgfpathlineto{\pgfqpoint{1.221743in}{1.496970in}}%
\pgfpathlineto{\pgfqpoint{1.238858in}{1.464040in}}%
\pgfpathlineto{\pgfqpoint{1.245785in}{1.487672in}}%
\pgfpathclose%
\pgfusepath{fill}%
\end{pgfscope}%
\begin{pgfscope}%
\pgfpathrectangle{\pgfqpoint{0.000000in}{0.000000in}}{\pgfqpoint{3.000000in}{3.000000in}}%
\pgfusepath{clip}%
\pgfsetbuttcap%
\pgfsetroundjoin%
\definecolor{currentfill}{rgb}{1.000000,0.886710,0.000000}%
\pgfsetfillcolor{currentfill}%
\pgfsetlinewidth{0.000000pt}%
\definecolor{currentstroke}{rgb}{0.000000,0.000000,0.000000}%
\pgfsetstrokecolor{currentstroke}%
\pgfsetdash{}{0pt}%
\pgfpathmoveto{\pgfqpoint{1.134646in}{1.827166in}}%
\pgfpathlineto{\pgfqpoint{1.120045in}{1.850987in}}%
\pgfpathlineto{\pgfqpoint{1.085573in}{1.817620in}}%
\pgfpathlineto{\pgfqpoint{1.101427in}{1.794877in}}%
\pgfpathlineto{\pgfqpoint{1.134646in}{1.827166in}}%
\pgfpathclose%
\pgfusepath{fill}%
\end{pgfscope}%
\begin{pgfscope}%
\pgfpathrectangle{\pgfqpoint{0.000000in}{0.000000in}}{\pgfqpoint{3.000000in}{3.000000in}}%
\pgfusepath{clip}%
\pgfsetbuttcap%
\pgfsetroundjoin%
\definecolor{currentfill}{rgb}{0.000000,0.676471,1.000000}%
\pgfsetfillcolor{currentfill}%
\pgfsetlinewidth{0.000000pt}%
\definecolor{currentstroke}{rgb}{0.000000,0.000000,0.000000}%
\pgfsetstrokecolor{currentstroke}%
\pgfsetdash{}{0pt}%
\pgfpathmoveto{\pgfqpoint{1.786674in}{1.315006in}}%
\pgfpathlineto{\pgfqpoint{1.803523in}{1.354739in}}%
\pgfpathlineto{\pgfqpoint{1.808432in}{1.376020in}}%
\pgfpathlineto{\pgfqpoint{1.791304in}{1.335010in}}%
\pgfpathlineto{\pgfqpoint{1.786674in}{1.315006in}}%
\pgfpathclose%
\pgfusepath{fill}%
\end{pgfscope}%
\begin{pgfscope}%
\pgfpathrectangle{\pgfqpoint{0.000000in}{0.000000in}}{\pgfqpoint{3.000000in}{3.000000in}}%
\pgfusepath{clip}%
\pgfsetbuttcap%
\pgfsetroundjoin%
\definecolor{currentfill}{rgb}{0.000000,0.849020,1.000000}%
\pgfsetfillcolor{currentfill}%
\pgfsetlinewidth{0.000000pt}%
\definecolor{currentstroke}{rgb}{0.000000,0.000000,0.000000}%
\pgfsetstrokecolor{currentstroke}%
\pgfsetdash{}{0pt}%
\pgfpathmoveto{\pgfqpoint{1.273051in}{1.390285in}}%
\pgfpathlineto{\pgfqpoint{1.255960in}{1.428630in}}%
\pgfpathlineto{\pgfqpoint{1.256453in}{1.405958in}}%
\pgfpathlineto{\pgfqpoint{1.273552in}{1.368891in}}%
\pgfpathlineto{\pgfqpoint{1.273051in}{1.390285in}}%
\pgfpathclose%
\pgfusepath{fill}%
\end{pgfscope}%
\begin{pgfscope}%
\pgfpathrectangle{\pgfqpoint{0.000000in}{0.000000in}}{\pgfqpoint{3.000000in}{3.000000in}}%
\pgfusepath{clip}%
\pgfsetbuttcap%
\pgfsetroundjoin%
\definecolor{currentfill}{rgb}{1.000000,0.480029,0.000000}%
\pgfsetfillcolor{currentfill}%
\pgfsetlinewidth{0.000000pt}%
\definecolor{currentstroke}{rgb}{0.000000,0.000000,0.000000}%
\pgfsetstrokecolor{currentstroke}%
\pgfsetdash{}{0pt}%
\pgfpathmoveto{\pgfqpoint{2.062554in}{1.935832in}}%
\pgfpathlineto{\pgfqpoint{2.078025in}{1.955747in}}%
\pgfpathlineto{\pgfqpoint{2.031843in}{1.994348in}}%
\pgfpathlineto{\pgfqpoint{2.017761in}{1.973410in}}%
\pgfpathlineto{\pgfqpoint{2.062554in}{1.935832in}}%
\pgfpathclose%
\pgfusepath{fill}%
\end{pgfscope}%
\begin{pgfscope}%
\pgfpathrectangle{\pgfqpoint{0.000000in}{0.000000in}}{\pgfqpoint{3.000000in}{3.000000in}}%
\pgfusepath{clip}%
\pgfsetbuttcap%
\pgfsetroundjoin%
\definecolor{currentfill}{rgb}{0.300443,1.000000,0.667299}%
\pgfsetfillcolor{currentfill}%
\pgfsetlinewidth{0.000000pt}%
\definecolor{currentstroke}{rgb}{0.000000,0.000000,0.000000}%
\pgfsetstrokecolor{currentstroke}%
\pgfsetdash{}{0pt}%
\pgfpathmoveto{\pgfqpoint{1.857744in}{1.505332in}}%
\pgfpathlineto{\pgfqpoint{1.874769in}{1.536565in}}%
\pgfpathlineto{\pgfqpoint{1.864263in}{1.562430in}}%
\pgfpathlineto{\pgfqpoint{1.847816in}{1.529965in}}%
\pgfpathlineto{\pgfqpoint{1.857744in}{1.505332in}}%
\pgfpathclose%
\pgfusepath{fill}%
\end{pgfscope}%
\begin{pgfscope}%
\pgfpathrectangle{\pgfqpoint{0.000000in}{0.000000in}}{\pgfqpoint{3.000000in}{3.000000in}}%
\pgfusepath{clip}%
\pgfsetbuttcap%
\pgfsetroundjoin%
\definecolor{currentfill}{rgb}{0.000000,0.300000,1.000000}%
\pgfsetfillcolor{currentfill}%
\pgfsetlinewidth{0.000000pt}%
\definecolor{currentstroke}{rgb}{0.000000,0.000000,0.000000}%
\pgfsetstrokecolor{currentstroke}%
\pgfsetdash{}{0pt}%
\pgfpathmoveto{\pgfqpoint{1.729318in}{1.188809in}}%
\pgfpathlineto{\pgfqpoint{1.744306in}{1.235652in}}%
\pgfpathlineto{\pgfqpoint{1.759810in}{1.252826in}}%
\pgfpathlineto{\pgfqpoint{1.743709in}{1.204800in}}%
\pgfpathlineto{\pgfqpoint{1.729318in}{1.188809in}}%
\pgfpathclose%
\pgfusepath{fill}%
\end{pgfscope}%
\begin{pgfscope}%
\pgfpathrectangle{\pgfqpoint{0.000000in}{0.000000in}}{\pgfqpoint{3.000000in}{3.000000in}}%
\pgfusepath{clip}%
\pgfsetbuttcap%
\pgfsetroundjoin%
\definecolor{currentfill}{rgb}{0.667299,1.000000,0.300443}%
\pgfsetfillcolor{currentfill}%
\pgfsetlinewidth{0.000000pt}%
\definecolor{currentstroke}{rgb}{0.000000,0.000000,0.000000}%
\pgfsetstrokecolor{currentstroke}%
\pgfsetdash{}{0pt}%
\pgfpathmoveto{\pgfqpoint{1.913632in}{1.649764in}}%
\pgfpathlineto{\pgfqpoint{1.930098in}{1.676186in}}%
\pgfpathlineto{\pgfqpoint{1.907803in}{1.705489in}}%
\pgfpathlineto{\pgfqpoint{1.892327in}{1.677912in}}%
\pgfpathlineto{\pgfqpoint{1.913632in}{1.649764in}}%
\pgfpathclose%
\pgfusepath{fill}%
\end{pgfscope}%
\begin{pgfscope}%
\pgfpathrectangle{\pgfqpoint{0.000000in}{0.000000in}}{\pgfqpoint{3.000000in}{3.000000in}}%
\pgfusepath{clip}%
\pgfsetbuttcap%
\pgfsetroundjoin%
\definecolor{currentfill}{rgb}{1.000000,0.407407,0.000000}%
\pgfsetfillcolor{currentfill}%
\pgfsetlinewidth{0.000000pt}%
\definecolor{currentstroke}{rgb}{0.000000,0.000000,0.000000}%
\pgfsetstrokecolor{currentstroke}%
\pgfsetdash{}{0pt}%
\pgfpathmoveto{\pgfqpoint{2.078025in}{1.955747in}}%
\pgfpathlineto{\pgfqpoint{2.093495in}{1.975263in}}%
\pgfpathlineto{\pgfqpoint{2.045921in}{2.014883in}}%
\pgfpathlineto{\pgfqpoint{2.031843in}{1.994348in}}%
\pgfpathlineto{\pgfqpoint{2.078025in}{1.955747in}}%
\pgfpathclose%
\pgfusepath{fill}%
\end{pgfscope}%
\begin{pgfscope}%
\pgfpathrectangle{\pgfqpoint{0.000000in}{0.000000in}}{\pgfqpoint{3.000000in}{3.000000in}}%
\pgfusepath{clip}%
\pgfsetbuttcap%
\pgfsetroundjoin%
\definecolor{currentfill}{rgb}{1.000000,0.814089,0.000000}%
\pgfsetfillcolor{currentfill}%
\pgfsetlinewidth{0.000000pt}%
\definecolor{currentstroke}{rgb}{0.000000,0.000000,0.000000}%
\pgfsetstrokecolor{currentstroke}%
\pgfsetdash{}{0pt}%
\pgfpathmoveto{\pgfqpoint{1.120045in}{1.850987in}}%
\pgfpathlineto{\pgfqpoint{1.105447in}{1.874152in}}%
\pgfpathlineto{\pgfqpoint{1.069718in}{1.839710in}}%
\pgfpathlineto{\pgfqpoint{1.085573in}{1.817620in}}%
\pgfpathlineto{\pgfqpoint{1.120045in}{1.850987in}}%
\pgfpathclose%
\pgfusepath{fill}%
\end{pgfscope}%
\begin{pgfscope}%
\pgfpathrectangle{\pgfqpoint{0.000000in}{0.000000in}}{\pgfqpoint{3.000000in}{3.000000in}}%
\pgfusepath{clip}%
\pgfsetbuttcap%
\pgfsetroundjoin%
\definecolor{currentfill}{rgb}{0.000000,0.503922,1.000000}%
\pgfsetfillcolor{currentfill}%
\pgfsetlinewidth{0.000000pt}%
\definecolor{currentstroke}{rgb}{0.000000,0.000000,0.000000}%
\pgfsetstrokecolor{currentstroke}%
\pgfsetdash{}{0pt}%
\pgfpathmoveto{\pgfqpoint{1.759810in}{1.252826in}}%
\pgfpathlineto{\pgfqpoint{1.775936in}{1.295581in}}%
\pgfpathlineto{\pgfqpoint{1.786674in}{1.315006in}}%
\pgfpathlineto{\pgfqpoint{1.769845in}{1.271004in}}%
\pgfpathlineto{\pgfqpoint{1.759810in}{1.252826in}}%
\pgfpathclose%
\pgfusepath{fill}%
\end{pgfscope}%
\begin{pgfscope}%
\pgfpathrectangle{\pgfqpoint{0.000000in}{0.000000in}}{\pgfqpoint{3.000000in}{3.000000in}}%
\pgfusepath{clip}%
\pgfsetbuttcap%
\pgfsetroundjoin%
\definecolor{currentfill}{rgb}{1.000000,0.349310,0.000000}%
\pgfsetfillcolor{currentfill}%
\pgfsetlinewidth{0.000000pt}%
\definecolor{currentstroke}{rgb}{0.000000,0.000000,0.000000}%
\pgfsetstrokecolor{currentstroke}%
\pgfsetdash{}{0pt}%
\pgfpathmoveto{\pgfqpoint{2.093495in}{1.975263in}}%
\pgfpathlineto{\pgfqpoint{2.108963in}{1.994407in}}%
\pgfpathlineto{\pgfqpoint{2.059995in}{2.035044in}}%
\pgfpathlineto{\pgfqpoint{2.045921in}{2.014883in}}%
\pgfpathlineto{\pgfqpoint{2.093495in}{1.975263in}}%
\pgfpathclose%
\pgfusepath{fill}%
\end{pgfscope}%
\begin{pgfscope}%
\pgfpathrectangle{\pgfqpoint{0.000000in}{0.000000in}}{\pgfqpoint{3.000000in}{3.000000in}}%
\pgfusepath{clip}%
\pgfsetbuttcap%
\pgfsetroundjoin%
\definecolor{currentfill}{rgb}{0.000000,0.000000,0.838681}%
\pgfsetfillcolor{currentfill}%
\pgfsetlinewidth{0.000000pt}%
\definecolor{currentstroke}{rgb}{0.000000,0.000000,0.000000}%
\pgfsetstrokecolor{currentstroke}%
\pgfsetdash{}{0pt}%
\pgfpathmoveto{\pgfqpoint{1.588240in}{1.028276in}}%
\pgfpathlineto{\pgfqpoint{1.592747in}{1.086314in}}%
\pgfpathlineto{\pgfqpoint{1.622538in}{1.091850in}}%
\pgfpathlineto{\pgfqpoint{1.615462in}{1.033364in}}%
\pgfpathlineto{\pgfqpoint{1.588240in}{1.028276in}}%
\pgfpathclose%
\pgfusepath{fill}%
\end{pgfscope}%
\begin{pgfscope}%
\pgfpathrectangle{\pgfqpoint{0.000000in}{0.000000in}}{\pgfqpoint{3.000000in}{3.000000in}}%
\pgfusepath{clip}%
\pgfsetbuttcap%
\pgfsetroundjoin%
\definecolor{currentfill}{rgb}{1.000000,0.291213,0.000000}%
\pgfsetfillcolor{currentfill}%
\pgfsetlinewidth{0.000000pt}%
\definecolor{currentstroke}{rgb}{0.000000,0.000000,0.000000}%
\pgfsetstrokecolor{currentstroke}%
\pgfsetdash{}{0pt}%
\pgfpathmoveto{\pgfqpoint{2.108963in}{1.994407in}}%
\pgfpathlineto{\pgfqpoint{2.124431in}{2.013206in}}%
\pgfpathlineto{\pgfqpoint{2.074063in}{2.054855in}}%
\pgfpathlineto{\pgfqpoint{2.059995in}{2.035044in}}%
\pgfpathlineto{\pgfqpoint{2.108963in}{1.994407in}}%
\pgfpathclose%
\pgfusepath{fill}%
\end{pgfscope}%
\begin{pgfscope}%
\pgfpathrectangle{\pgfqpoint{0.000000in}{0.000000in}}{\pgfqpoint{3.000000in}{3.000000in}}%
\pgfusepath{clip}%
\pgfsetbuttcap%
\pgfsetroundjoin%
\definecolor{currentfill}{rgb}{0.578748,1.000000,0.388994}%
\pgfsetfillcolor{currentfill}%
\pgfsetlinewidth{0.000000pt}%
\definecolor{currentstroke}{rgb}{0.000000,0.000000,0.000000}%
\pgfsetstrokecolor{currentstroke}%
\pgfsetdash{}{0pt}%
\pgfpathmoveto{\pgfqpoint{1.196528in}{1.640293in}}%
\pgfpathlineto{\pgfqpoint{1.180681in}{1.668726in}}%
\pgfpathlineto{\pgfqpoint{1.162331in}{1.640037in}}%
\pgfpathlineto{\pgfqpoint{1.179035in}{1.612787in}}%
\pgfpathlineto{\pgfqpoint{1.196528in}{1.640293in}}%
\pgfpathclose%
\pgfusepath{fill}%
\end{pgfscope}%
\begin{pgfscope}%
\pgfpathrectangle{\pgfqpoint{0.000000in}{0.000000in}}{\pgfqpoint{3.000000in}{3.000000in}}%
\pgfusepath{clip}%
\pgfsetbuttcap%
\pgfsetroundjoin%
\definecolor{currentfill}{rgb}{0.085389,1.000000,0.882353}%
\pgfsetfillcolor{currentfill}%
\pgfsetlinewidth{0.000000pt}%
\definecolor{currentstroke}{rgb}{0.000000,0.000000,0.000000}%
\pgfsetstrokecolor{currentstroke}%
\pgfsetdash{}{0pt}%
\pgfpathmoveto{\pgfqpoint{1.825575in}{1.413514in}}%
\pgfpathlineto{\pgfqpoint{1.842734in}{1.448073in}}%
\pgfpathlineto{\pgfqpoint{1.840730in}{1.471981in}}%
\pgfpathlineto{\pgfqpoint{1.823725in}{1.436150in}}%
\pgfpathlineto{\pgfqpoint{1.825575in}{1.413514in}}%
\pgfpathclose%
\pgfusepath{fill}%
\end{pgfscope}%
\begin{pgfscope}%
\pgfpathrectangle{\pgfqpoint{0.000000in}{0.000000in}}{\pgfqpoint{3.000000in}{3.000000in}}%
\pgfusepath{clip}%
\pgfsetbuttcap%
\pgfsetroundjoin%
\definecolor{currentfill}{rgb}{0.000000,0.000000,0.838681}%
\pgfsetfillcolor{currentfill}%
\pgfsetlinewidth{0.000000pt}%
\definecolor{currentstroke}{rgb}{0.000000,0.000000,0.000000}%
\pgfsetstrokecolor{currentstroke}%
\pgfsetdash{}{0pt}%
\pgfpathmoveto{\pgfqpoint{1.474515in}{1.031419in}}%
\pgfpathlineto{\pgfqpoint{1.468279in}{1.089733in}}%
\pgfpathlineto{\pgfqpoint{1.498597in}{1.085021in}}%
\pgfpathlineto{\pgfqpoint{1.502218in}{1.027088in}}%
\pgfpathlineto{\pgfqpoint{1.474515in}{1.031419in}}%
\pgfpathclose%
\pgfusepath{fill}%
\end{pgfscope}%
\begin{pgfscope}%
\pgfpathrectangle{\pgfqpoint{0.000000in}{0.000000in}}{\pgfqpoint{3.000000in}{3.000000in}}%
\pgfusepath{clip}%
\pgfsetbuttcap%
\pgfsetroundjoin%
\definecolor{currentfill}{rgb}{1.000000,0.741467,0.000000}%
\pgfsetfillcolor{currentfill}%
\pgfsetlinewidth{0.000000pt}%
\definecolor{currentstroke}{rgb}{0.000000,0.000000,0.000000}%
\pgfsetstrokecolor{currentstroke}%
\pgfsetdash{}{0pt}%
\pgfpathmoveto{\pgfqpoint{1.105447in}{1.874152in}}%
\pgfpathlineto{\pgfqpoint{1.090852in}{1.896716in}}%
\pgfpathlineto{\pgfqpoint{1.053862in}{1.861204in}}%
\pgfpathlineto{\pgfqpoint{1.069718in}{1.839710in}}%
\pgfpathlineto{\pgfqpoint{1.105447in}{1.874152in}}%
\pgfpathclose%
\pgfusepath{fill}%
\end{pgfscope}%
\begin{pgfscope}%
\pgfpathrectangle{\pgfqpoint{0.000000in}{0.000000in}}{\pgfqpoint{3.000000in}{3.000000in}}%
\pgfusepath{clip}%
\pgfsetbuttcap%
\pgfsetroundjoin%
\definecolor{currentfill}{rgb}{1.000000,0.233115,0.000000}%
\pgfsetfillcolor{currentfill}%
\pgfsetlinewidth{0.000000pt}%
\definecolor{currentstroke}{rgb}{0.000000,0.000000,0.000000}%
\pgfsetstrokecolor{currentstroke}%
\pgfsetdash{}{0pt}%
\pgfpathmoveto{\pgfqpoint{2.124431in}{2.013206in}}%
\pgfpathlineto{\pgfqpoint{2.139897in}{2.031682in}}%
\pgfpathlineto{\pgfqpoint{2.088127in}{2.074340in}}%
\pgfpathlineto{\pgfqpoint{2.074063in}{2.054855in}}%
\pgfpathlineto{\pgfqpoint{2.124431in}{2.013206in}}%
\pgfpathclose%
\pgfusepath{fill}%
\end{pgfscope}%
\begin{pgfscope}%
\pgfpathrectangle{\pgfqpoint{0.000000in}{0.000000in}}{\pgfqpoint{3.000000in}{3.000000in}}%
\pgfusepath{clip}%
\pgfsetbuttcap%
\pgfsetroundjoin%
\definecolor{currentfill}{rgb}{0.000000,0.064706,1.000000}%
\pgfsetfillcolor{currentfill}%
\pgfsetlinewidth{0.000000pt}%
\definecolor{currentstroke}{rgb}{0.000000,0.000000,0.000000}%
\pgfsetstrokecolor{currentstroke}%
\pgfsetdash{}{0pt}%
\pgfpathmoveto{\pgfqpoint{1.675189in}{1.109816in}}%
\pgfpathlineto{\pgfqpoint{1.686812in}{1.161235in}}%
\pgfpathlineto{\pgfqpoint{1.710199in}{1.174163in}}%
\pgfpathlineto{\pgfqpoint{1.696734in}{1.121781in}}%
\pgfpathlineto{\pgfqpoint{1.675189in}{1.109816in}}%
\pgfpathclose%
\pgfusepath{fill}%
\end{pgfscope}%
\begin{pgfscope}%
\pgfpathrectangle{\pgfqpoint{0.000000in}{0.000000in}}{\pgfqpoint{3.000000in}{3.000000in}}%
\pgfusepath{clip}%
\pgfsetbuttcap%
\pgfsetroundjoin%
\definecolor{currentfill}{rgb}{0.743201,1.000000,0.224541}%
\pgfsetfillcolor{currentfill}%
\pgfsetlinewidth{0.000000pt}%
\definecolor{currentstroke}{rgb}{0.000000,0.000000,0.000000}%
\pgfsetstrokecolor{currentstroke}%
\pgfsetdash{}{0pt}%
\pgfpathmoveto{\pgfqpoint{1.930098in}{1.676186in}}%
\pgfpathlineto{\pgfqpoint{1.946569in}{1.701517in}}%
\pgfpathlineto{\pgfqpoint{1.923280in}{1.731970in}}%
\pgfpathlineto{\pgfqpoint{1.907803in}{1.705489in}}%
\pgfpathlineto{\pgfqpoint{1.930098in}{1.676186in}}%
\pgfpathclose%
\pgfusepath{fill}%
\end{pgfscope}%
\begin{pgfscope}%
\pgfpathrectangle{\pgfqpoint{0.000000in}{0.000000in}}{\pgfqpoint{3.000000in}{3.000000in}}%
\pgfusepath{clip}%
\pgfsetbuttcap%
\pgfsetroundjoin%
\definecolor{currentfill}{rgb}{0.000000,0.300000,1.000000}%
\pgfsetfillcolor{currentfill}%
\pgfsetlinewidth{0.000000pt}%
\definecolor{currentstroke}{rgb}{0.000000,0.000000,0.000000}%
\pgfsetstrokecolor{currentstroke}%
\pgfsetdash{}{0pt}%
\pgfpathmoveto{\pgfqpoint{1.341621in}{1.199345in}}%
\pgfpathlineto{\pgfqpoint{1.325846in}{1.246968in}}%
\pgfpathlineto{\pgfqpoint{1.343101in}{1.230226in}}%
\pgfpathlineto{\pgfqpoint{1.357631in}{1.183759in}}%
\pgfpathlineto{\pgfqpoint{1.341621in}{1.199345in}}%
\pgfpathclose%
\pgfusepath{fill}%
\end{pgfscope}%
\begin{pgfscope}%
\pgfpathrectangle{\pgfqpoint{0.000000in}{0.000000in}}{\pgfqpoint{3.000000in}{3.000000in}}%
\pgfusepath{clip}%
\pgfsetbuttcap%
\pgfsetroundjoin%
\definecolor{currentfill}{rgb}{1.000000,0.175018,0.000000}%
\pgfsetfillcolor{currentfill}%
\pgfsetlinewidth{0.000000pt}%
\definecolor{currentstroke}{rgb}{0.000000,0.000000,0.000000}%
\pgfsetstrokecolor{currentstroke}%
\pgfsetdash{}{0pt}%
\pgfpathmoveto{\pgfqpoint{2.139897in}{2.031682in}}%
\pgfpathlineto{\pgfqpoint{2.155361in}{2.049855in}}%
\pgfpathlineto{\pgfqpoint{2.102186in}{2.093518in}}%
\pgfpathlineto{\pgfqpoint{2.088127in}{2.074340in}}%
\pgfpathlineto{\pgfqpoint{2.139897in}{2.031682in}}%
\pgfpathclose%
\pgfusepath{fill}%
\end{pgfscope}%
\begin{pgfscope}%
\pgfpathrectangle{\pgfqpoint{0.000000in}{0.000000in}}{\pgfqpoint{3.000000in}{3.000000in}}%
\pgfusepath{clip}%
\pgfsetbuttcap%
\pgfsetroundjoin%
\definecolor{currentfill}{rgb}{0.000000,0.676471,1.000000}%
\pgfsetfillcolor{currentfill}%
\pgfsetlinewidth{0.000000pt}%
\definecolor{currentstroke}{rgb}{0.000000,0.000000,0.000000}%
\pgfsetstrokecolor{currentstroke}%
\pgfsetdash{}{0pt}%
\pgfpathmoveto{\pgfqpoint{1.290634in}{1.328309in}}%
\pgfpathlineto{\pgfqpoint{1.273552in}{1.368891in}}%
\pgfpathlineto{\pgfqpoint{1.280654in}{1.347754in}}%
\pgfpathlineto{\pgfqpoint{1.297316in}{1.308443in}}%
\pgfpathlineto{\pgfqpoint{1.290634in}{1.328309in}}%
\pgfpathclose%
\pgfusepath{fill}%
\end{pgfscope}%
\begin{pgfscope}%
\pgfpathrectangle{\pgfqpoint{0.000000in}{0.000000in}}{\pgfqpoint{3.000000in}{3.000000in}}%
\pgfusepath{clip}%
\pgfsetbuttcap%
\pgfsetroundjoin%
\definecolor{currentfill}{rgb}{0.000000,0.000000,0.838681}%
\pgfsetfillcolor{currentfill}%
\pgfsetlinewidth{0.000000pt}%
\definecolor{currentstroke}{rgb}{0.000000,0.000000,0.000000}%
\pgfsetstrokecolor{currentstroke}%
\pgfsetdash{}{0pt}%
\pgfpathmoveto{\pgfqpoint{1.559810in}{1.025496in}}%
\pgfpathlineto{\pgfqpoint{1.561632in}{1.083289in}}%
\pgfpathlineto{\pgfqpoint{1.592747in}{1.086314in}}%
\pgfpathlineto{\pgfqpoint{1.588240in}{1.028276in}}%
\pgfpathlineto{\pgfqpoint{1.559810in}{1.025496in}}%
\pgfpathclose%
\pgfusepath{fill}%
\end{pgfscope}%
\begin{pgfscope}%
\pgfpathrectangle{\pgfqpoint{0.000000in}{0.000000in}}{\pgfqpoint{3.000000in}{3.000000in}}%
\pgfusepath{clip}%
\pgfsetbuttcap%
\pgfsetroundjoin%
\definecolor{currentfill}{rgb}{1.000000,0.116921,0.000000}%
\pgfsetfillcolor{currentfill}%
\pgfsetlinewidth{0.000000pt}%
\definecolor{currentstroke}{rgb}{0.000000,0.000000,0.000000}%
\pgfsetstrokecolor{currentstroke}%
\pgfsetdash{}{0pt}%
\pgfpathmoveto{\pgfqpoint{2.155361in}{2.049855in}}%
\pgfpathlineto{\pgfqpoint{2.170824in}{2.067744in}}%
\pgfpathlineto{\pgfqpoint{2.116241in}{2.112410in}}%
\pgfpathlineto{\pgfqpoint{2.102186in}{2.093518in}}%
\pgfpathlineto{\pgfqpoint{2.155361in}{2.049855in}}%
\pgfpathclose%
\pgfusepath{fill}%
\end{pgfscope}%
\begin{pgfscope}%
\pgfpathrectangle{\pgfqpoint{0.000000in}{0.000000in}}{\pgfqpoint{3.000000in}{3.000000in}}%
\pgfusepath{clip}%
\pgfsetbuttcap%
\pgfsetroundjoin%
\definecolor{currentfill}{rgb}{1.000000,0.668845,0.000000}%
\pgfsetfillcolor{currentfill}%
\pgfsetlinewidth{0.000000pt}%
\definecolor{currentstroke}{rgb}{0.000000,0.000000,0.000000}%
\pgfsetstrokecolor{currentstroke}%
\pgfsetdash{}{0pt}%
\pgfpathmoveto{\pgfqpoint{1.090852in}{1.896716in}}%
\pgfpathlineto{\pgfqpoint{1.076260in}{1.918731in}}%
\pgfpathlineto{\pgfqpoint{1.038005in}{1.882151in}}%
\pgfpathlineto{\pgfqpoint{1.053862in}{1.861204in}}%
\pgfpathlineto{\pgfqpoint{1.090852in}{1.896716in}}%
\pgfpathclose%
\pgfusepath{fill}%
\end{pgfscope}%
\begin{pgfscope}%
\pgfpathrectangle{\pgfqpoint{0.000000in}{0.000000in}}{\pgfqpoint{3.000000in}{3.000000in}}%
\pgfusepath{clip}%
\pgfsetbuttcap%
\pgfsetroundjoin%
\definecolor{currentfill}{rgb}{0.000000,0.000000,0.838681}%
\pgfsetfillcolor{currentfill}%
\pgfsetlinewidth{0.000000pt}%
\definecolor{currentstroke}{rgb}{0.000000,0.000000,0.000000}%
\pgfsetstrokecolor{currentstroke}%
\pgfsetdash{}{0pt}%
\pgfpathmoveto{\pgfqpoint{1.502218in}{1.027088in}}%
\pgfpathlineto{\pgfqpoint{1.498597in}{1.085021in}}%
\pgfpathlineto{\pgfqpoint{1.529980in}{1.082854in}}%
\pgfpathlineto{\pgfqpoint{1.530892in}{1.025096in}}%
\pgfpathlineto{\pgfqpoint{1.502218in}{1.027088in}}%
\pgfpathclose%
\pgfusepath{fill}%
\end{pgfscope}%
\begin{pgfscope}%
\pgfpathrectangle{\pgfqpoint{0.000000in}{0.000000in}}{\pgfqpoint{3.000000in}{3.000000in}}%
\pgfusepath{clip}%
\pgfsetbuttcap%
\pgfsetroundjoin%
\definecolor{currentfill}{rgb}{0.999109,0.073348,0.000000}%
\pgfsetfillcolor{currentfill}%
\pgfsetlinewidth{0.000000pt}%
\definecolor{currentstroke}{rgb}{0.000000,0.000000,0.000000}%
\pgfsetstrokecolor{currentstroke}%
\pgfsetdash{}{0pt}%
\pgfpathmoveto{\pgfqpoint{2.170824in}{2.067744in}}%
\pgfpathlineto{\pgfqpoint{2.186286in}{2.085366in}}%
\pgfpathlineto{\pgfqpoint{2.130290in}{2.131031in}}%
\pgfpathlineto{\pgfqpoint{2.116241in}{2.112410in}}%
\pgfpathlineto{\pgfqpoint{2.170824in}{2.067744in}}%
\pgfpathclose%
\pgfusepath{fill}%
\end{pgfscope}%
\begin{pgfscope}%
\pgfpathrectangle{\pgfqpoint{0.000000in}{0.000000in}}{\pgfqpoint{3.000000in}{3.000000in}}%
\pgfusepath{clip}%
\pgfsetbuttcap%
\pgfsetroundjoin%
\definecolor{currentfill}{rgb}{0.927807,0.015251,0.000000}%
\pgfsetfillcolor{currentfill}%
\pgfsetlinewidth{0.000000pt}%
\definecolor{currentstroke}{rgb}{0.000000,0.000000,0.000000}%
\pgfsetstrokecolor{currentstroke}%
\pgfsetdash{}{0pt}%
\pgfpathmoveto{\pgfqpoint{2.186286in}{2.085366in}}%
\pgfpathlineto{\pgfqpoint{2.201746in}{2.102737in}}%
\pgfpathlineto{\pgfqpoint{2.144334in}{2.149397in}}%
\pgfpathlineto{\pgfqpoint{2.130290in}{2.131031in}}%
\pgfpathlineto{\pgfqpoint{2.186286in}{2.085366in}}%
\pgfpathclose%
\pgfusepath{fill}%
\end{pgfscope}%
\begin{pgfscope}%
\pgfpathrectangle{\pgfqpoint{0.000000in}{0.000000in}}{\pgfqpoint{3.000000in}{3.000000in}}%
\pgfusepath{clip}%
\pgfsetbuttcap%
\pgfsetroundjoin%
\definecolor{currentfill}{rgb}{0.000000,0.000000,0.838681}%
\pgfsetfillcolor{currentfill}%
\pgfsetlinewidth{0.000000pt}%
\definecolor{currentstroke}{rgb}{0.000000,0.000000,0.000000}%
\pgfsetstrokecolor{currentstroke}%
\pgfsetdash{}{0pt}%
\pgfpathmoveto{\pgfqpoint{1.530892in}{1.025096in}}%
\pgfpathlineto{\pgfqpoint{1.529980in}{1.082854in}}%
\pgfpathlineto{\pgfqpoint{1.561632in}{1.083289in}}%
\pgfpathlineto{\pgfqpoint{1.559810in}{1.025496in}}%
\pgfpathlineto{\pgfqpoint{1.530892in}{1.025096in}}%
\pgfpathclose%
\pgfusepath{fill}%
\end{pgfscope}%
\begin{pgfscope}%
\pgfpathrectangle{\pgfqpoint{0.000000in}{0.000000in}}{\pgfqpoint{3.000000in}{3.000000in}}%
\pgfusepath{clip}%
\pgfsetbuttcap%
\pgfsetroundjoin%
\definecolor{currentfill}{rgb}{0.000000,0.503922,1.000000}%
\pgfsetfillcolor{currentfill}%
\pgfsetlinewidth{0.000000pt}%
\definecolor{currentstroke}{rgb}{0.000000,0.000000,0.000000}%
\pgfsetstrokecolor{currentstroke}%
\pgfsetdash{}{0pt}%
\pgfpathmoveto{\pgfqpoint{1.313957in}{1.264861in}}%
\pgfpathlineto{\pgfqpoint{1.297316in}{1.308443in}}%
\pgfpathlineto{\pgfqpoint{1.310047in}{1.289320in}}%
\pgfpathlineto{\pgfqpoint{1.325846in}{1.246968in}}%
\pgfpathlineto{\pgfqpoint{1.313957in}{1.264861in}}%
\pgfpathclose%
\pgfusepath{fill}%
\end{pgfscope}%
\begin{pgfscope}%
\pgfpathrectangle{\pgfqpoint{0.000000in}{0.000000in}}{\pgfqpoint{3.000000in}{3.000000in}}%
\pgfusepath{clip}%
\pgfsetbuttcap%
\pgfsetroundjoin%
\definecolor{currentfill}{rgb}{0.401645,1.000000,0.566097}%
\pgfsetfillcolor{currentfill}%
\pgfsetlinewidth{0.000000pt}%
\definecolor{currentstroke}{rgb}{0.000000,0.000000,0.000000}%
\pgfsetstrokecolor{currentstroke}%
\pgfsetdash{}{0pt}%
\pgfpathmoveto{\pgfqpoint{1.874769in}{1.536565in}}%
\pgfpathlineto{\pgfqpoint{1.891804in}{1.565972in}}%
\pgfpathlineto{\pgfqpoint{1.880714in}{1.593066in}}%
\pgfpathlineto{\pgfqpoint{1.864263in}{1.562430in}}%
\pgfpathlineto{\pgfqpoint{1.874769in}{1.536565in}}%
\pgfpathclose%
\pgfusepath{fill}%
\end{pgfscope}%
\begin{pgfscope}%
\pgfpathrectangle{\pgfqpoint{0.000000in}{0.000000in}}{\pgfqpoint{3.000000in}{3.000000in}}%
\pgfusepath{clip}%
\pgfsetbuttcap%
\pgfsetroundjoin%
\definecolor{currentfill}{rgb}{0.300443,1.000000,0.667299}%
\pgfsetfillcolor{currentfill}%
\pgfsetlinewidth{0.000000pt}%
\definecolor{currentstroke}{rgb}{0.000000,0.000000,0.000000}%
\pgfsetstrokecolor{currentstroke}%
\pgfsetdash{}{0pt}%
\pgfpathmoveto{\pgfqpoint{1.229108in}{1.521852in}}%
\pgfpathlineto{\pgfqpoint{1.212424in}{1.553912in}}%
\pgfpathlineto{\pgfqpoint{1.204617in}{1.527783in}}%
\pgfpathlineto{\pgfqpoint{1.221743in}{1.496970in}}%
\pgfpathlineto{\pgfqpoint{1.229108in}{1.521852in}}%
\pgfpathclose%
\pgfusepath{fill}%
\end{pgfscope}%
\begin{pgfscope}%
\pgfpathrectangle{\pgfqpoint{0.000000in}{0.000000in}}{\pgfqpoint{3.000000in}{3.000000in}}%
\pgfusepath{clip}%
\pgfsetbuttcap%
\pgfsetroundjoin%
\definecolor{currentfill}{rgb}{0.856506,0.000000,0.000000}%
\pgfsetfillcolor{currentfill}%
\pgfsetlinewidth{0.000000pt}%
\definecolor{currentstroke}{rgb}{0.000000,0.000000,0.000000}%
\pgfsetstrokecolor{currentstroke}%
\pgfsetdash{}{0pt}%
\pgfpathmoveto{\pgfqpoint{2.201746in}{2.102737in}}%
\pgfpathlineto{\pgfqpoint{2.217204in}{2.119871in}}%
\pgfpathlineto{\pgfqpoint{2.158374in}{2.167523in}}%
\pgfpathlineto{\pgfqpoint{2.144334in}{2.149397in}}%
\pgfpathlineto{\pgfqpoint{2.201746in}{2.102737in}}%
\pgfpathclose%
\pgfusepath{fill}%
\end{pgfscope}%
\begin{pgfscope}%
\pgfpathrectangle{\pgfqpoint{0.000000in}{0.000000in}}{\pgfqpoint{3.000000in}{3.000000in}}%
\pgfusepath{clip}%
\pgfsetbuttcap%
\pgfsetroundjoin%
\definecolor{currentfill}{rgb}{0.000000,0.064706,1.000000}%
\pgfsetfillcolor{currentfill}%
\pgfsetlinewidth{0.000000pt}%
\definecolor{currentstroke}{rgb}{0.000000,0.000000,0.000000}%
\pgfsetstrokecolor{currentstroke}%
\pgfsetdash{}{0pt}%
\pgfpathmoveto{\pgfqpoint{1.391117in}{1.117600in}}%
\pgfpathlineto{\pgfqpoint{1.378229in}{1.169645in}}%
\pgfpathlineto{\pgfqpoint{1.402916in}{1.157367in}}%
\pgfpathlineto{\pgfqpoint{1.413857in}{1.106237in}}%
\pgfpathlineto{\pgfqpoint{1.391117in}{1.117600in}}%
\pgfpathclose%
\pgfusepath{fill}%
\end{pgfscope}%
\begin{pgfscope}%
\pgfpathrectangle{\pgfqpoint{0.000000in}{0.000000in}}{\pgfqpoint{3.000000in}{3.000000in}}%
\pgfusepath{clip}%
\pgfsetbuttcap%
\pgfsetroundjoin%
\definecolor{currentfill}{rgb}{0.803030,0.000000,0.000000}%
\pgfsetfillcolor{currentfill}%
\pgfsetlinewidth{0.000000pt}%
\definecolor{currentstroke}{rgb}{0.000000,0.000000,0.000000}%
\pgfsetstrokecolor{currentstroke}%
\pgfsetdash{}{0pt}%
\pgfpathmoveto{\pgfqpoint{2.217204in}{2.119871in}}%
\pgfpathlineto{\pgfqpoint{2.232660in}{2.136781in}}%
\pgfpathlineto{\pgfqpoint{2.172408in}{2.185422in}}%
\pgfpathlineto{\pgfqpoint{2.158374in}{2.167523in}}%
\pgfpathlineto{\pgfqpoint{2.217204in}{2.119871in}}%
\pgfpathclose%
\pgfusepath{fill}%
\end{pgfscope}%
\begin{pgfscope}%
\pgfpathrectangle{\pgfqpoint{0.000000in}{0.000000in}}{\pgfqpoint{3.000000in}{3.000000in}}%
\pgfusepath{clip}%
\pgfsetbuttcap%
\pgfsetroundjoin%
\definecolor{currentfill}{rgb}{1.000000,0.610748,0.000000}%
\pgfsetfillcolor{currentfill}%
\pgfsetlinewidth{0.000000pt}%
\definecolor{currentstroke}{rgb}{0.000000,0.000000,0.000000}%
\pgfsetstrokecolor{currentstroke}%
\pgfsetdash{}{0pt}%
\pgfpathmoveto{\pgfqpoint{1.076260in}{1.918731in}}%
\pgfpathlineto{\pgfqpoint{1.061671in}{1.940239in}}%
\pgfpathlineto{\pgfqpoint{1.022148in}{1.902596in}}%
\pgfpathlineto{\pgfqpoint{1.038005in}{1.882151in}}%
\pgfpathlineto{\pgfqpoint{1.076260in}{1.918731in}}%
\pgfpathclose%
\pgfusepath{fill}%
\end{pgfscope}%
\begin{pgfscope}%
\pgfpathrectangle{\pgfqpoint{0.000000in}{0.000000in}}{\pgfqpoint{3.000000in}{3.000000in}}%
\pgfusepath{clip}%
\pgfsetbuttcap%
\pgfsetroundjoin%
\definecolor{currentfill}{rgb}{0.731729,0.000000,0.000000}%
\pgfsetfillcolor{currentfill}%
\pgfsetlinewidth{0.000000pt}%
\definecolor{currentstroke}{rgb}{0.000000,0.000000,0.000000}%
\pgfsetstrokecolor{currentstroke}%
\pgfsetdash{}{0pt}%
\pgfpathmoveto{\pgfqpoint{2.232660in}{2.136781in}}%
\pgfpathlineto{\pgfqpoint{2.248115in}{2.153479in}}%
\pgfpathlineto{\pgfqpoint{2.186437in}{2.203105in}}%
\pgfpathlineto{\pgfqpoint{2.172408in}{2.185422in}}%
\pgfpathlineto{\pgfqpoint{2.232660in}{2.136781in}}%
\pgfpathclose%
\pgfusepath{fill}%
\end{pgfscope}%
\begin{pgfscope}%
\pgfpathrectangle{\pgfqpoint{0.000000in}{0.000000in}}{\pgfqpoint{3.000000in}{3.000000in}}%
\pgfusepath{clip}%
\pgfsetbuttcap%
\pgfsetroundjoin%
\definecolor{currentfill}{rgb}{0.678253,0.000000,0.000000}%
\pgfsetfillcolor{currentfill}%
\pgfsetlinewidth{0.000000pt}%
\definecolor{currentstroke}{rgb}{0.000000,0.000000,0.000000}%
\pgfsetstrokecolor{currentstroke}%
\pgfsetdash{}{0pt}%
\pgfpathmoveto{\pgfqpoint{2.248115in}{2.153479in}}%
\pgfpathlineto{\pgfqpoint{2.263567in}{2.169977in}}%
\pgfpathlineto{\pgfqpoint{2.200461in}{2.220585in}}%
\pgfpathlineto{\pgfqpoint{2.186437in}{2.203105in}}%
\pgfpathlineto{\pgfqpoint{2.248115in}{2.153479in}}%
\pgfpathclose%
\pgfusepath{fill}%
\end{pgfscope}%
\begin{pgfscope}%
\pgfpathrectangle{\pgfqpoint{0.000000in}{0.000000in}}{\pgfqpoint{3.000000in}{3.000000in}}%
\pgfusepath{clip}%
\pgfsetbuttcap%
\pgfsetroundjoin%
\definecolor{currentfill}{rgb}{0.667299,1.000000,0.300443}%
\pgfsetfillcolor{currentfill}%
\pgfsetlinewidth{0.000000pt}%
\definecolor{currentstroke}{rgb}{0.000000,0.000000,0.000000}%
\pgfsetstrokecolor{currentstroke}%
\pgfsetdash{}{0pt}%
\pgfpathmoveto{\pgfqpoint{1.180681in}{1.668726in}}%
\pgfpathlineto{\pgfqpoint{1.164833in}{1.695927in}}%
\pgfpathlineto{\pgfqpoint{1.145621in}{1.666059in}}%
\pgfpathlineto{\pgfqpoint{1.162331in}{1.640037in}}%
\pgfpathlineto{\pgfqpoint{1.180681in}{1.668726in}}%
\pgfpathclose%
\pgfusepath{fill}%
\end{pgfscope}%
\begin{pgfscope}%
\pgfpathrectangle{\pgfqpoint{0.000000in}{0.000000in}}{\pgfqpoint{3.000000in}{3.000000in}}%
\pgfusepath{clip}%
\pgfsetbuttcap%
\pgfsetroundjoin%
\definecolor{currentfill}{rgb}{0.819102,1.000000,0.148640}%
\pgfsetfillcolor{currentfill}%
\pgfsetlinewidth{0.000000pt}%
\definecolor{currentstroke}{rgb}{0.000000,0.000000,0.000000}%
\pgfsetstrokecolor{currentstroke}%
\pgfsetdash{}{0pt}%
\pgfpathmoveto{\pgfqpoint{1.946569in}{1.701517in}}%
\pgfpathlineto{\pgfqpoint{1.963044in}{1.725873in}}%
\pgfpathlineto{\pgfqpoint{1.938756in}{1.757473in}}%
\pgfpathlineto{\pgfqpoint{1.923280in}{1.731970in}}%
\pgfpathlineto{\pgfqpoint{1.946569in}{1.701517in}}%
\pgfpathclose%
\pgfusepath{fill}%
\end{pgfscope}%
\begin{pgfscope}%
\pgfpathrectangle{\pgfqpoint{0.000000in}{0.000000in}}{\pgfqpoint{3.000000in}{3.000000in}}%
\pgfusepath{clip}%
\pgfsetbuttcap%
\pgfsetroundjoin%
\definecolor{currentfill}{rgb}{0.606952,0.000000,0.000000}%
\pgfsetfillcolor{currentfill}%
\pgfsetlinewidth{0.000000pt}%
\definecolor{currentstroke}{rgb}{0.000000,0.000000,0.000000}%
\pgfsetstrokecolor{currentstroke}%
\pgfsetdash{}{0pt}%
\pgfpathmoveto{\pgfqpoint{2.263567in}{2.169977in}}%
\pgfpathlineto{\pgfqpoint{2.279018in}{2.186284in}}%
\pgfpathlineto{\pgfqpoint{2.214479in}{2.237871in}}%
\pgfpathlineto{\pgfqpoint{2.200461in}{2.220585in}}%
\pgfpathlineto{\pgfqpoint{2.263567in}{2.169977in}}%
\pgfpathclose%
\pgfusepath{fill}%
\end{pgfscope}%
\begin{pgfscope}%
\pgfpathrectangle{\pgfqpoint{0.000000in}{0.000000in}}{\pgfqpoint{3.000000in}{3.000000in}}%
\pgfusepath{clip}%
\pgfsetbuttcap%
\pgfsetroundjoin%
\definecolor{currentfill}{rgb}{0.553476,0.000000,0.000000}%
\pgfsetfillcolor{currentfill}%
\pgfsetlinewidth{0.000000pt}%
\definecolor{currentstroke}{rgb}{0.000000,0.000000,0.000000}%
\pgfsetstrokecolor{currentstroke}%
\pgfsetdash{}{0pt}%
\pgfpathmoveto{\pgfqpoint{2.279018in}{2.186284in}}%
\pgfpathlineto{\pgfqpoint{2.294466in}{2.202411in}}%
\pgfpathlineto{\pgfqpoint{2.228493in}{2.254972in}}%
\pgfpathlineto{\pgfqpoint{2.214479in}{2.237871in}}%
\pgfpathlineto{\pgfqpoint{2.279018in}{2.186284in}}%
\pgfpathclose%
\pgfusepath{fill}%
\end{pgfscope}%
\begin{pgfscope}%
\pgfpathrectangle{\pgfqpoint{0.000000in}{0.000000in}}{\pgfqpoint{3.000000in}{3.000000in}}%
\pgfusepath{clip}%
\pgfsetbuttcap%
\pgfsetroundjoin%
\definecolor{currentfill}{rgb}{0.500000,0.000000,0.000000}%
\pgfsetfillcolor{currentfill}%
\pgfsetlinewidth{0.000000pt}%
\definecolor{currentstroke}{rgb}{0.000000,0.000000,0.000000}%
\pgfsetstrokecolor{currentstroke}%
\pgfsetdash{}{0pt}%
\pgfpathmoveto{\pgfqpoint{2.294466in}{2.202411in}}%
\pgfpathlineto{\pgfqpoint{2.309913in}{2.218365in}}%
\pgfpathlineto{\pgfqpoint{2.242501in}{2.271899in}}%
\pgfpathlineto{\pgfqpoint{2.228493in}{2.254972in}}%
\pgfpathlineto{\pgfqpoint{2.294466in}{2.202411in}}%
\pgfpathclose%
\pgfusepath{fill}%
\end{pgfscope}%
\begin{pgfscope}%
\pgfpathrectangle{\pgfqpoint{0.000000in}{0.000000in}}{\pgfqpoint{3.000000in}{3.000000in}}%
\pgfusepath{clip}%
\pgfsetbuttcap%
\pgfsetroundjoin%
\definecolor{currentfill}{rgb}{1.000000,0.538126,0.000000}%
\pgfsetfillcolor{currentfill}%
\pgfsetlinewidth{0.000000pt}%
\definecolor{currentstroke}{rgb}{0.000000,0.000000,0.000000}%
\pgfsetstrokecolor{currentstroke}%
\pgfsetdash{}{0pt}%
\pgfpathmoveto{\pgfqpoint{1.061671in}{1.940239in}}%
\pgfpathlineto{\pgfqpoint{1.047086in}{1.961280in}}%
\pgfpathlineto{\pgfqpoint{1.006291in}{1.922576in}}%
\pgfpathlineto{\pgfqpoint{1.022148in}{1.902596in}}%
\pgfpathlineto{\pgfqpoint{1.061671in}{1.940239in}}%
\pgfpathclose%
\pgfusepath{fill}%
\end{pgfscope}%
\begin{pgfscope}%
\pgfpathrectangle{\pgfqpoint{0.000000in}{0.000000in}}{\pgfqpoint{3.000000in}{3.000000in}}%
\pgfusepath{clip}%
\pgfsetbuttcap%
\pgfsetroundjoin%
\definecolor{currentfill}{rgb}{0.085389,1.000000,0.882353}%
\pgfsetfillcolor{currentfill}%
\pgfsetlinewidth{0.000000pt}%
\definecolor{currentstroke}{rgb}{0.000000,0.000000,0.000000}%
\pgfsetstrokecolor{currentstroke}%
\pgfsetdash{}{0pt}%
\pgfpathmoveto{\pgfqpoint{1.255960in}{1.428630in}}%
\pgfpathlineto{\pgfqpoint{1.238858in}{1.464040in}}%
\pgfpathlineto{\pgfqpoint{1.239336in}{1.440092in}}%
\pgfpathlineto{\pgfqpoint{1.256453in}{1.405958in}}%
\pgfpathlineto{\pgfqpoint{1.255960in}{1.428630in}}%
\pgfpathclose%
\pgfusepath{fill}%
\end{pgfscope}%
\begin{pgfscope}%
\pgfpathrectangle{\pgfqpoint{0.000000in}{0.000000in}}{\pgfqpoint{3.000000in}{3.000000in}}%
\pgfusepath{clip}%
\pgfsetbuttcap%
\pgfsetroundjoin%
\definecolor{currentfill}{rgb}{1.000000,0.480029,0.000000}%
\pgfsetfillcolor{currentfill}%
\pgfsetlinewidth{0.000000pt}%
\definecolor{currentstroke}{rgb}{0.000000,0.000000,0.000000}%
\pgfsetstrokecolor{currentstroke}%
\pgfsetdash{}{0pt}%
\pgfpathmoveto{\pgfqpoint{1.047086in}{1.961280in}}%
\pgfpathlineto{\pgfqpoint{1.032504in}{1.981889in}}%
\pgfpathlineto{\pgfqpoint{0.990433in}{1.942128in}}%
\pgfpathlineto{\pgfqpoint{1.006291in}{1.922576in}}%
\pgfpathlineto{\pgfqpoint{1.047086in}{1.961280in}}%
\pgfpathclose%
\pgfusepath{fill}%
\end{pgfscope}%
\begin{pgfscope}%
\pgfpathrectangle{\pgfqpoint{0.000000in}{0.000000in}}{\pgfqpoint{3.000000in}{3.000000in}}%
\pgfusepath{clip}%
\pgfsetbuttcap%
\pgfsetroundjoin%
\definecolor{currentfill}{rgb}{0.000000,0.849020,1.000000}%
\pgfsetfillcolor{currentfill}%
\pgfsetlinewidth{0.000000pt}%
\definecolor{currentstroke}{rgb}{0.000000,0.000000,0.000000}%
\pgfsetstrokecolor{currentstroke}%
\pgfsetdash{}{0pt}%
\pgfpathmoveto{\pgfqpoint{1.803523in}{1.354739in}}%
\pgfpathlineto{\pgfqpoint{1.820393in}{1.390956in}}%
\pgfpathlineto{\pgfqpoint{1.825575in}{1.413514in}}%
\pgfpathlineto{\pgfqpoint{1.808432in}{1.376020in}}%
\pgfpathlineto{\pgfqpoint{1.803523in}{1.354739in}}%
\pgfpathclose%
\pgfusepath{fill}%
\end{pgfscope}%
\begin{pgfscope}%
\pgfpathrectangle{\pgfqpoint{0.000000in}{0.000000in}}{\pgfqpoint{3.000000in}{3.000000in}}%
\pgfusepath{clip}%
\pgfsetbuttcap%
\pgfsetroundjoin%
\definecolor{currentfill}{rgb}{0.895003,1.000000,0.072739}%
\pgfsetfillcolor{currentfill}%
\pgfsetlinewidth{0.000000pt}%
\definecolor{currentstroke}{rgb}{0.000000,0.000000,0.000000}%
\pgfsetstrokecolor{currentstroke}%
\pgfsetdash{}{0pt}%
\pgfpathmoveto{\pgfqpoint{1.963044in}{1.725873in}}%
\pgfpathlineto{\pgfqpoint{1.979523in}{1.749353in}}%
\pgfpathlineto{\pgfqpoint{1.954233in}{1.782098in}}%
\pgfpathlineto{\pgfqpoint{1.938756in}{1.757473in}}%
\pgfpathlineto{\pgfqpoint{1.963044in}{1.725873in}}%
\pgfpathclose%
\pgfusepath{fill}%
\end{pgfscope}%
\begin{pgfscope}%
\pgfpathrectangle{\pgfqpoint{0.000000in}{0.000000in}}{\pgfqpoint{3.000000in}{3.000000in}}%
\pgfusepath{clip}%
\pgfsetbuttcap%
\pgfsetroundjoin%
\definecolor{currentfill}{rgb}{1.000000,0.407407,0.000000}%
\pgfsetfillcolor{currentfill}%
\pgfsetlinewidth{0.000000pt}%
\definecolor{currentstroke}{rgb}{0.000000,0.000000,0.000000}%
\pgfsetstrokecolor{currentstroke}%
\pgfsetdash{}{0pt}%
\pgfpathmoveto{\pgfqpoint{1.032504in}{1.981889in}}%
\pgfpathlineto{\pgfqpoint{1.017925in}{2.002097in}}%
\pgfpathlineto{\pgfqpoint{0.974575in}{1.961282in}}%
\pgfpathlineto{\pgfqpoint{0.990433in}{1.942128in}}%
\pgfpathlineto{\pgfqpoint{1.032504in}{1.981889in}}%
\pgfpathclose%
\pgfusepath{fill}%
\end{pgfscope}%
\begin{pgfscope}%
\pgfpathrectangle{\pgfqpoint{0.000000in}{0.000000in}}{\pgfqpoint{3.000000in}{3.000000in}}%
\pgfusepath{clip}%
\pgfsetbuttcap%
\pgfsetroundjoin%
\definecolor{currentfill}{rgb}{0.199241,1.000000,0.768501}%
\pgfsetfillcolor{currentfill}%
\pgfsetlinewidth{0.000000pt}%
\definecolor{currentstroke}{rgb}{0.000000,0.000000,0.000000}%
\pgfsetstrokecolor{currentstroke}%
\pgfsetdash{}{0pt}%
\pgfpathmoveto{\pgfqpoint{1.842734in}{1.448073in}}%
\pgfpathlineto{\pgfqpoint{1.859908in}{1.480155in}}%
\pgfpathlineto{\pgfqpoint{1.857744in}{1.505332in}}%
\pgfpathlineto{\pgfqpoint{1.840730in}{1.471981in}}%
\pgfpathlineto{\pgfqpoint{1.842734in}{1.448073in}}%
\pgfpathclose%
\pgfusepath{fill}%
\end{pgfscope}%
\begin{pgfscope}%
\pgfpathrectangle{\pgfqpoint{0.000000in}{0.000000in}}{\pgfqpoint{3.000000in}{3.000000in}}%
\pgfusepath{clip}%
\pgfsetbuttcap%
\pgfsetroundjoin%
\definecolor{currentfill}{rgb}{0.743201,1.000000,0.224541}%
\pgfsetfillcolor{currentfill}%
\pgfsetlinewidth{0.000000pt}%
\definecolor{currentstroke}{rgb}{0.000000,0.000000,0.000000}%
\pgfsetstrokecolor{currentstroke}%
\pgfsetdash{}{0pt}%
\pgfpathmoveto{\pgfqpoint{1.164833in}{1.695927in}}%
\pgfpathlineto{\pgfqpoint{1.148984in}{1.722034in}}%
\pgfpathlineto{\pgfqpoint{1.128904in}{1.690990in}}%
\pgfpathlineto{\pgfqpoint{1.145621in}{1.666059in}}%
\pgfpathlineto{\pgfqpoint{1.164833in}{1.695927in}}%
\pgfpathclose%
\pgfusepath{fill}%
\end{pgfscope}%
\begin{pgfscope}%
\pgfpathrectangle{\pgfqpoint{0.000000in}{0.000000in}}{\pgfqpoint{3.000000in}{3.000000in}}%
\pgfusepath{clip}%
\pgfsetbuttcap%
\pgfsetroundjoin%
\definecolor{currentfill}{rgb}{0.000000,0.300000,1.000000}%
\pgfsetfillcolor{currentfill}%
\pgfsetlinewidth{0.000000pt}%
\definecolor{currentstroke}{rgb}{0.000000,0.000000,0.000000}%
\pgfsetstrokecolor{currentstroke}%
\pgfsetdash{}{0pt}%
\pgfpathmoveto{\pgfqpoint{1.710199in}{1.174163in}}%
\pgfpathlineto{\pgfqpoint{1.723691in}{1.219917in}}%
\pgfpathlineto{\pgfqpoint{1.744306in}{1.235652in}}%
\pgfpathlineto{\pgfqpoint{1.729318in}{1.188809in}}%
\pgfpathlineto{\pgfqpoint{1.710199in}{1.174163in}}%
\pgfpathclose%
\pgfusepath{fill}%
\end{pgfscope}%
\begin{pgfscope}%
\pgfpathrectangle{\pgfqpoint{0.000000in}{0.000000in}}{\pgfqpoint{3.000000in}{3.000000in}}%
\pgfusepath{clip}%
\pgfsetbuttcap%
\pgfsetroundjoin%
\definecolor{currentfill}{rgb}{0.000000,0.064706,1.000000}%
\pgfsetfillcolor{currentfill}%
\pgfsetlinewidth{0.000000pt}%
\definecolor{currentstroke}{rgb}{0.000000,0.000000,0.000000}%
\pgfsetstrokecolor{currentstroke}%
\pgfsetdash{}{0pt}%
\pgfpathmoveto{\pgfqpoint{1.650250in}{1.099753in}}%
\pgfpathlineto{\pgfqpoint{1.659732in}{1.150359in}}%
\pgfpathlineto{\pgfqpoint{1.686812in}{1.161235in}}%
\pgfpathlineto{\pgfqpoint{1.675189in}{1.109816in}}%
\pgfpathlineto{\pgfqpoint{1.650250in}{1.099753in}}%
\pgfpathclose%
\pgfusepath{fill}%
\end{pgfscope}%
\begin{pgfscope}%
\pgfpathrectangle{\pgfqpoint{0.000000in}{0.000000in}}{\pgfqpoint{3.000000in}{3.000000in}}%
\pgfusepath{clip}%
\pgfsetbuttcap%
\pgfsetroundjoin%
\definecolor{currentfill}{rgb}{0.490196,1.000000,0.477546}%
\pgfsetfillcolor{currentfill}%
\pgfsetlinewidth{0.000000pt}%
\definecolor{currentstroke}{rgb}{0.000000,0.000000,0.000000}%
\pgfsetstrokecolor{currentstroke}%
\pgfsetdash{}{0pt}%
\pgfpathmoveto{\pgfqpoint{1.891804in}{1.565972in}}%
\pgfpathlineto{\pgfqpoint{1.908849in}{1.593791in}}%
\pgfpathlineto{\pgfqpoint{1.897171in}{1.622112in}}%
\pgfpathlineto{\pgfqpoint{1.880714in}{1.593066in}}%
\pgfpathlineto{\pgfqpoint{1.891804in}{1.565972in}}%
\pgfpathclose%
\pgfusepath{fill}%
\end{pgfscope}%
\begin{pgfscope}%
\pgfpathrectangle{\pgfqpoint{0.000000in}{0.000000in}}{\pgfqpoint{3.000000in}{3.000000in}}%
\pgfusepath{clip}%
\pgfsetbuttcap%
\pgfsetroundjoin%
\definecolor{currentfill}{rgb}{1.000000,0.349310,0.000000}%
\pgfsetfillcolor{currentfill}%
\pgfsetlinewidth{0.000000pt}%
\definecolor{currentstroke}{rgb}{0.000000,0.000000,0.000000}%
\pgfsetstrokecolor{currentstroke}%
\pgfsetdash{}{0pt}%
\pgfpathmoveto{\pgfqpoint{1.017925in}{2.002097in}}%
\pgfpathlineto{\pgfqpoint{1.003350in}{2.021930in}}%
\pgfpathlineto{\pgfqpoint{0.958717in}{1.980066in}}%
\pgfpathlineto{\pgfqpoint{0.974575in}{1.961282in}}%
\pgfpathlineto{\pgfqpoint{1.017925in}{2.002097in}}%
\pgfpathclose%
\pgfusepath{fill}%
\end{pgfscope}%
\begin{pgfscope}%
\pgfpathrectangle{\pgfqpoint{0.000000in}{0.000000in}}{\pgfqpoint{3.000000in}{3.000000in}}%
\pgfusepath{clip}%
\pgfsetbuttcap%
\pgfsetroundjoin%
\definecolor{currentfill}{rgb}{0.401645,1.000000,0.566097}%
\pgfsetfillcolor{currentfill}%
\pgfsetlinewidth{0.000000pt}%
\definecolor{currentstroke}{rgb}{0.000000,0.000000,0.000000}%
\pgfsetstrokecolor{currentstroke}%
\pgfsetdash{}{0pt}%
\pgfpathmoveto{\pgfqpoint{1.212424in}{1.553912in}}%
\pgfpathlineto{\pgfqpoint{1.195733in}{1.584144in}}%
\pgfpathlineto{\pgfqpoint{1.187478in}{1.556770in}}%
\pgfpathlineto{\pgfqpoint{1.204617in}{1.527783in}}%
\pgfpathlineto{\pgfqpoint{1.212424in}{1.553912in}}%
\pgfpathclose%
\pgfusepath{fill}%
\end{pgfscope}%
\begin{pgfscope}%
\pgfpathrectangle{\pgfqpoint{0.000000in}{0.000000in}}{\pgfqpoint{3.000000in}{3.000000in}}%
\pgfusepath{clip}%
\pgfsetbuttcap%
\pgfsetroundjoin%
\definecolor{currentfill}{rgb}{1.000000,0.291213,0.000000}%
\pgfsetfillcolor{currentfill}%
\pgfsetlinewidth{0.000000pt}%
\definecolor{currentstroke}{rgb}{0.000000,0.000000,0.000000}%
\pgfsetstrokecolor{currentstroke}%
\pgfsetdash{}{0pt}%
\pgfpathmoveto{\pgfqpoint{1.003350in}{2.021930in}}%
\pgfpathlineto{\pgfqpoint{0.988779in}{2.041416in}}%
\pgfpathlineto{\pgfqpoint{0.942858in}{1.998505in}}%
\pgfpathlineto{\pgfqpoint{0.958717in}{1.980066in}}%
\pgfpathlineto{\pgfqpoint{1.003350in}{2.021930in}}%
\pgfpathclose%
\pgfusepath{fill}%
\end{pgfscope}%
\begin{pgfscope}%
\pgfpathrectangle{\pgfqpoint{0.000000in}{0.000000in}}{\pgfqpoint{3.000000in}{3.000000in}}%
\pgfusepath{clip}%
\pgfsetbuttcap%
\pgfsetroundjoin%
\definecolor{currentfill}{rgb}{0.958254,0.973856,0.009488}%
\pgfsetfillcolor{currentfill}%
\pgfsetlinewidth{0.000000pt}%
\definecolor{currentstroke}{rgb}{0.000000,0.000000,0.000000}%
\pgfsetstrokecolor{currentstroke}%
\pgfsetdash{}{0pt}%
\pgfpathmoveto{\pgfqpoint{1.979523in}{1.749353in}}%
\pgfpathlineto{\pgfqpoint{1.996007in}{1.772044in}}%
\pgfpathlineto{\pgfqpoint{1.969709in}{1.805930in}}%
\pgfpathlineto{\pgfqpoint{1.954233in}{1.782098in}}%
\pgfpathlineto{\pgfqpoint{1.979523in}{1.749353in}}%
\pgfpathclose%
\pgfusepath{fill}%
\end{pgfscope}%
\begin{pgfscope}%
\pgfpathrectangle{\pgfqpoint{0.000000in}{0.000000in}}{\pgfqpoint{3.000000in}{3.000000in}}%
\pgfusepath{clip}%
\pgfsetbuttcap%
\pgfsetroundjoin%
\definecolor{currentfill}{rgb}{0.000000,0.064706,1.000000}%
\pgfsetfillcolor{currentfill}%
\pgfsetlinewidth{0.000000pt}%
\definecolor{currentstroke}{rgb}{0.000000,0.000000,0.000000}%
\pgfsetstrokecolor{currentstroke}%
\pgfsetdash{}{0pt}%
\pgfpathmoveto{\pgfqpoint{1.413857in}{1.106237in}}%
\pgfpathlineto{\pgfqpoint{1.402916in}{1.157367in}}%
\pgfpathlineto{\pgfqpoint{1.431084in}{1.147240in}}%
\pgfpathlineto{\pgfqpoint{1.439793in}{1.096867in}}%
\pgfpathlineto{\pgfqpoint{1.413857in}{1.106237in}}%
\pgfpathclose%
\pgfusepath{fill}%
\end{pgfscope}%
\begin{pgfscope}%
\pgfpathrectangle{\pgfqpoint{0.000000in}{0.000000in}}{\pgfqpoint{3.000000in}{3.000000in}}%
\pgfusepath{clip}%
\pgfsetbuttcap%
\pgfsetroundjoin%
\definecolor{currentfill}{rgb}{0.000000,0.676471,1.000000}%
\pgfsetfillcolor{currentfill}%
\pgfsetlinewidth{0.000000pt}%
\definecolor{currentstroke}{rgb}{0.000000,0.000000,0.000000}%
\pgfsetstrokecolor{currentstroke}%
\pgfsetdash{}{0pt}%
\pgfpathmoveto{\pgfqpoint{1.775936in}{1.295581in}}%
\pgfpathlineto{\pgfqpoint{1.792085in}{1.334067in}}%
\pgfpathlineto{\pgfqpoint{1.803523in}{1.354739in}}%
\pgfpathlineto{\pgfqpoint{1.786674in}{1.315006in}}%
\pgfpathlineto{\pgfqpoint{1.775936in}{1.295581in}}%
\pgfpathclose%
\pgfusepath{fill}%
\end{pgfscope}%
\begin{pgfscope}%
\pgfpathrectangle{\pgfqpoint{0.000000in}{0.000000in}}{\pgfqpoint{3.000000in}{3.000000in}}%
\pgfusepath{clip}%
\pgfsetbuttcap%
\pgfsetroundjoin%
\definecolor{currentfill}{rgb}{1.000000,0.233115,0.000000}%
\pgfsetfillcolor{currentfill}%
\pgfsetlinewidth{0.000000pt}%
\definecolor{currentstroke}{rgb}{0.000000,0.000000,0.000000}%
\pgfsetstrokecolor{currentstroke}%
\pgfsetdash{}{0pt}%
\pgfpathmoveto{\pgfqpoint{0.988779in}{2.041416in}}%
\pgfpathlineto{\pgfqpoint{0.974212in}{2.060576in}}%
\pgfpathlineto{\pgfqpoint{0.927000in}{2.016622in}}%
\pgfpathlineto{\pgfqpoint{0.942858in}{1.998505in}}%
\pgfpathlineto{\pgfqpoint{0.988779in}{2.041416in}}%
\pgfpathclose%
\pgfusepath{fill}%
\end{pgfscope}%
\begin{pgfscope}%
\pgfpathrectangle{\pgfqpoint{0.000000in}{0.000000in}}{\pgfqpoint{3.000000in}{3.000000in}}%
\pgfusepath{clip}%
\pgfsetbuttcap%
\pgfsetroundjoin%
\definecolor{currentfill}{rgb}{0.000000,0.503922,1.000000}%
\pgfsetfillcolor{currentfill}%
\pgfsetlinewidth{0.000000pt}%
\definecolor{currentstroke}{rgb}{0.000000,0.000000,0.000000}%
\pgfsetstrokecolor{currentstroke}%
\pgfsetdash{}{0pt}%
\pgfpathmoveto{\pgfqpoint{1.744306in}{1.235652in}}%
\pgfpathlineto{\pgfqpoint{1.759319in}{1.277223in}}%
\pgfpathlineto{\pgfqpoint{1.775936in}{1.295581in}}%
\pgfpathlineto{\pgfqpoint{1.759810in}{1.252826in}}%
\pgfpathlineto{\pgfqpoint{1.744306in}{1.235652in}}%
\pgfpathclose%
\pgfusepath{fill}%
\end{pgfscope}%
\begin{pgfscope}%
\pgfpathrectangle{\pgfqpoint{0.000000in}{0.000000in}}{\pgfqpoint{3.000000in}{3.000000in}}%
\pgfusepath{clip}%
\pgfsetbuttcap%
\pgfsetroundjoin%
\definecolor{currentfill}{rgb}{0.000000,0.300000,1.000000}%
\pgfsetfillcolor{currentfill}%
\pgfsetlinewidth{0.000000pt}%
\definecolor{currentstroke}{rgb}{0.000000,0.000000,0.000000}%
\pgfsetstrokecolor{currentstroke}%
\pgfsetdash{}{0pt}%
\pgfpathmoveto{\pgfqpoint{1.357631in}{1.183759in}}%
\pgfpathlineto{\pgfqpoint{1.343101in}{1.230226in}}%
\pgfpathlineto{\pgfqpoint{1.365315in}{1.215063in}}%
\pgfpathlineto{\pgfqpoint{1.378229in}{1.169645in}}%
\pgfpathlineto{\pgfqpoint{1.357631in}{1.183759in}}%
\pgfpathclose%
\pgfusepath{fill}%
\end{pgfscope}%
\begin{pgfscope}%
\pgfpathrectangle{\pgfqpoint{0.000000in}{0.000000in}}{\pgfqpoint{3.000000in}{3.000000in}}%
\pgfusepath{clip}%
\pgfsetbuttcap%
\pgfsetroundjoin%
\definecolor{currentfill}{rgb}{0.819102,1.000000,0.148640}%
\pgfsetfillcolor{currentfill}%
\pgfsetlinewidth{0.000000pt}%
\definecolor{currentstroke}{rgb}{0.000000,0.000000,0.000000}%
\pgfsetstrokecolor{currentstroke}%
\pgfsetdash{}{0pt}%
\pgfpathmoveto{\pgfqpoint{1.148984in}{1.722034in}}%
\pgfpathlineto{\pgfqpoint{1.133133in}{1.747164in}}%
\pgfpathlineto{\pgfqpoint{1.112182in}{1.714947in}}%
\pgfpathlineto{\pgfqpoint{1.128904in}{1.690990in}}%
\pgfpathlineto{\pgfqpoint{1.148984in}{1.722034in}}%
\pgfpathclose%
\pgfusepath{fill}%
\end{pgfscope}%
\begin{pgfscope}%
\pgfpathrectangle{\pgfqpoint{0.000000in}{0.000000in}}{\pgfqpoint{3.000000in}{3.000000in}}%
\pgfusepath{clip}%
\pgfsetbuttcap%
\pgfsetroundjoin%
\definecolor{currentfill}{rgb}{0.000000,0.849020,1.000000}%
\pgfsetfillcolor{currentfill}%
\pgfsetlinewidth{0.000000pt}%
\definecolor{currentstroke}{rgb}{0.000000,0.000000,0.000000}%
\pgfsetstrokecolor{currentstroke}%
\pgfsetdash{}{0pt}%
\pgfpathmoveto{\pgfqpoint{1.273552in}{1.368891in}}%
\pgfpathlineto{\pgfqpoint{1.256453in}{1.405958in}}%
\pgfpathlineto{\pgfqpoint{1.263970in}{1.383552in}}%
\pgfpathlineto{\pgfqpoint{1.280654in}{1.347754in}}%
\pgfpathlineto{\pgfqpoint{1.273552in}{1.368891in}}%
\pgfpathclose%
\pgfusepath{fill}%
\end{pgfscope}%
\begin{pgfscope}%
\pgfpathrectangle{\pgfqpoint{0.000000in}{0.000000in}}{\pgfqpoint{3.000000in}{3.000000in}}%
\pgfusepath{clip}%
\pgfsetbuttcap%
\pgfsetroundjoin%
\definecolor{currentfill}{rgb}{1.000000,0.175018,0.000000}%
\pgfsetfillcolor{currentfill}%
\pgfsetlinewidth{0.000000pt}%
\definecolor{currentstroke}{rgb}{0.000000,0.000000,0.000000}%
\pgfsetstrokecolor{currentstroke}%
\pgfsetdash{}{0pt}%
\pgfpathmoveto{\pgfqpoint{0.974212in}{2.060576in}}%
\pgfpathlineto{\pgfqpoint{0.959648in}{2.079432in}}%
\pgfpathlineto{\pgfqpoint{0.911142in}{2.034437in}}%
\pgfpathlineto{\pgfqpoint{0.927000in}{2.016622in}}%
\pgfpathlineto{\pgfqpoint{0.974212in}{2.060576in}}%
\pgfpathclose%
\pgfusepath{fill}%
\end{pgfscope}%
\begin{pgfscope}%
\pgfpathrectangle{\pgfqpoint{0.000000in}{0.000000in}}{\pgfqpoint{3.000000in}{3.000000in}}%
\pgfusepath{clip}%
\pgfsetbuttcap%
\pgfsetroundjoin%
\definecolor{currentfill}{rgb}{1.000000,0.886710,0.000000}%
\pgfsetfillcolor{currentfill}%
\pgfsetlinewidth{0.000000pt}%
\definecolor{currentstroke}{rgb}{0.000000,0.000000,0.000000}%
\pgfsetstrokecolor{currentstroke}%
\pgfsetdash{}{0pt}%
\pgfpathmoveto{\pgfqpoint{1.996007in}{1.772044in}}%
\pgfpathlineto{\pgfqpoint{2.012494in}{1.794020in}}%
\pgfpathlineto{\pgfqpoint{1.985185in}{1.829043in}}%
\pgfpathlineto{\pgfqpoint{1.969709in}{1.805930in}}%
\pgfpathlineto{\pgfqpoint{1.996007in}{1.772044in}}%
\pgfpathclose%
\pgfusepath{fill}%
\end{pgfscope}%
\begin{pgfscope}%
\pgfpathrectangle{\pgfqpoint{0.000000in}{0.000000in}}{\pgfqpoint{3.000000in}{3.000000in}}%
\pgfusepath{clip}%
\pgfsetbuttcap%
\pgfsetroundjoin%
\definecolor{currentfill}{rgb}{1.000000,0.116921,0.000000}%
\pgfsetfillcolor{currentfill}%
\pgfsetlinewidth{0.000000pt}%
\definecolor{currentstroke}{rgb}{0.000000,0.000000,0.000000}%
\pgfsetstrokecolor{currentstroke}%
\pgfsetdash{}{0pt}%
\pgfpathmoveto{\pgfqpoint{0.959648in}{2.079432in}}%
\pgfpathlineto{\pgfqpoint{0.945088in}{2.098001in}}%
\pgfpathlineto{\pgfqpoint{0.895284in}{2.051970in}}%
\pgfpathlineto{\pgfqpoint{0.911142in}{2.034437in}}%
\pgfpathlineto{\pgfqpoint{0.959648in}{2.079432in}}%
\pgfpathclose%
\pgfusepath{fill}%
\end{pgfscope}%
\begin{pgfscope}%
\pgfpathrectangle{\pgfqpoint{0.000000in}{0.000000in}}{\pgfqpoint{3.000000in}{3.000000in}}%
\pgfusepath{clip}%
\pgfsetbuttcap%
\pgfsetroundjoin%
\definecolor{currentfill}{rgb}{0.578748,1.000000,0.388994}%
\pgfsetfillcolor{currentfill}%
\pgfsetlinewidth{0.000000pt}%
\definecolor{currentstroke}{rgb}{0.000000,0.000000,0.000000}%
\pgfsetstrokecolor{currentstroke}%
\pgfsetdash{}{0pt}%
\pgfpathmoveto{\pgfqpoint{1.908849in}{1.593791in}}%
\pgfpathlineto{\pgfqpoint{1.925904in}{1.620220in}}%
\pgfpathlineto{\pgfqpoint{1.913632in}{1.649764in}}%
\pgfpathlineto{\pgfqpoint{1.897171in}{1.622112in}}%
\pgfpathlineto{\pgfqpoint{1.908849in}{1.593791in}}%
\pgfpathclose%
\pgfusepath{fill}%
\end{pgfscope}%
\begin{pgfscope}%
\pgfpathrectangle{\pgfqpoint{0.000000in}{0.000000in}}{\pgfqpoint{3.000000in}{3.000000in}}%
\pgfusepath{clip}%
\pgfsetbuttcap%
\pgfsetroundjoin%
\definecolor{currentfill}{rgb}{0.199241,1.000000,0.768501}%
\pgfsetfillcolor{currentfill}%
\pgfsetlinewidth{0.000000pt}%
\definecolor{currentstroke}{rgb}{0.000000,0.000000,0.000000}%
\pgfsetstrokecolor{currentstroke}%
\pgfsetdash{}{0pt}%
\pgfpathmoveto{\pgfqpoint{1.238858in}{1.464040in}}%
\pgfpathlineto{\pgfqpoint{1.221743in}{1.496970in}}%
\pgfpathlineto{\pgfqpoint{1.222203in}{1.471748in}}%
\pgfpathlineto{\pgfqpoint{1.239336in}{1.440092in}}%
\pgfpathlineto{\pgfqpoint{1.238858in}{1.464040in}}%
\pgfpathclose%
\pgfusepath{fill}%
\end{pgfscope}%
\begin{pgfscope}%
\pgfpathrectangle{\pgfqpoint{0.000000in}{0.000000in}}{\pgfqpoint{3.000000in}{3.000000in}}%
\pgfusepath{clip}%
\pgfsetbuttcap%
\pgfsetroundjoin%
\definecolor{currentfill}{rgb}{0.999109,0.073348,0.000000}%
\pgfsetfillcolor{currentfill}%
\pgfsetlinewidth{0.000000pt}%
\definecolor{currentstroke}{rgb}{0.000000,0.000000,0.000000}%
\pgfsetstrokecolor{currentstroke}%
\pgfsetdash{}{0pt}%
\pgfpathmoveto{\pgfqpoint{0.945088in}{2.098001in}}%
\pgfpathlineto{\pgfqpoint{0.930533in}{2.116301in}}%
\pgfpathlineto{\pgfqpoint{0.879426in}{2.069236in}}%
\pgfpathlineto{\pgfqpoint{0.895284in}{2.051970in}}%
\pgfpathlineto{\pgfqpoint{0.945088in}{2.098001in}}%
\pgfpathclose%
\pgfusepath{fill}%
\end{pgfscope}%
\begin{pgfscope}%
\pgfpathrectangle{\pgfqpoint{0.000000in}{0.000000in}}{\pgfqpoint{3.000000in}{3.000000in}}%
\pgfusepath{clip}%
\pgfsetbuttcap%
\pgfsetroundjoin%
\definecolor{currentfill}{rgb}{0.927807,0.015251,0.000000}%
\pgfsetfillcolor{currentfill}%
\pgfsetlinewidth{0.000000pt}%
\definecolor{currentstroke}{rgb}{0.000000,0.000000,0.000000}%
\pgfsetstrokecolor{currentstroke}%
\pgfsetdash{}{0pt}%
\pgfpathmoveto{\pgfqpoint{0.930533in}{2.116301in}}%
\pgfpathlineto{\pgfqpoint{0.915981in}{2.134347in}}%
\pgfpathlineto{\pgfqpoint{0.863569in}{2.086253in}}%
\pgfpathlineto{\pgfqpoint{0.879426in}{2.069236in}}%
\pgfpathlineto{\pgfqpoint{0.930533in}{2.116301in}}%
\pgfpathclose%
\pgfusepath{fill}%
\end{pgfscope}%
\begin{pgfscope}%
\pgfpathrectangle{\pgfqpoint{0.000000in}{0.000000in}}{\pgfqpoint{3.000000in}{3.000000in}}%
\pgfusepath{clip}%
\pgfsetbuttcap%
\pgfsetroundjoin%
\definecolor{currentfill}{rgb}{0.000000,0.064706,1.000000}%
\pgfsetfillcolor{currentfill}%
\pgfsetlinewidth{0.000000pt}%
\definecolor{currentstroke}{rgb}{0.000000,0.000000,0.000000}%
\pgfsetstrokecolor{currentstroke}%
\pgfsetdash{}{0pt}%
\pgfpathmoveto{\pgfqpoint{1.622538in}{1.091850in}}%
\pgfpathlineto{\pgfqpoint{1.629630in}{1.141817in}}%
\pgfpathlineto{\pgfqpoint{1.659732in}{1.150359in}}%
\pgfpathlineto{\pgfqpoint{1.650250in}{1.099753in}}%
\pgfpathlineto{\pgfqpoint{1.622538in}{1.091850in}}%
\pgfpathclose%
\pgfusepath{fill}%
\end{pgfscope}%
\begin{pgfscope}%
\pgfpathrectangle{\pgfqpoint{0.000000in}{0.000000in}}{\pgfqpoint{3.000000in}{3.000000in}}%
\pgfusepath{clip}%
\pgfsetbuttcap%
\pgfsetroundjoin%
\definecolor{currentfill}{rgb}{0.490196,1.000000,0.477546}%
\pgfsetfillcolor{currentfill}%
\pgfsetlinewidth{0.000000pt}%
\definecolor{currentstroke}{rgb}{0.000000,0.000000,0.000000}%
\pgfsetstrokecolor{currentstroke}%
\pgfsetdash{}{0pt}%
\pgfpathmoveto{\pgfqpoint{1.195733in}{1.584144in}}%
\pgfpathlineto{\pgfqpoint{1.179035in}{1.612787in}}%
\pgfpathlineto{\pgfqpoint{1.170329in}{1.584172in}}%
\pgfpathlineto{\pgfqpoint{1.187478in}{1.556770in}}%
\pgfpathlineto{\pgfqpoint{1.195733in}{1.584144in}}%
\pgfpathclose%
\pgfusepath{fill}%
\end{pgfscope}%
\begin{pgfscope}%
\pgfpathrectangle{\pgfqpoint{0.000000in}{0.000000in}}{\pgfqpoint{3.000000in}{3.000000in}}%
\pgfusepath{clip}%
\pgfsetbuttcap%
\pgfsetroundjoin%
\definecolor{currentfill}{rgb}{0.895003,1.000000,0.072739}%
\pgfsetfillcolor{currentfill}%
\pgfsetlinewidth{0.000000pt}%
\definecolor{currentstroke}{rgb}{0.000000,0.000000,0.000000}%
\pgfsetstrokecolor{currentstroke}%
\pgfsetdash{}{0pt}%
\pgfpathmoveto{\pgfqpoint{1.133133in}{1.747164in}}%
\pgfpathlineto{\pgfqpoint{1.117281in}{1.771416in}}%
\pgfpathlineto{\pgfqpoint{1.095454in}{1.738029in}}%
\pgfpathlineto{\pgfqpoint{1.112182in}{1.714947in}}%
\pgfpathlineto{\pgfqpoint{1.133133in}{1.747164in}}%
\pgfpathclose%
\pgfusepath{fill}%
\end{pgfscope}%
\begin{pgfscope}%
\pgfpathrectangle{\pgfqpoint{0.000000in}{0.000000in}}{\pgfqpoint{3.000000in}{3.000000in}}%
\pgfusepath{clip}%
\pgfsetbuttcap%
\pgfsetroundjoin%
\definecolor{currentfill}{rgb}{0.300443,1.000000,0.667299}%
\pgfsetfillcolor{currentfill}%
\pgfsetlinewidth{0.000000pt}%
\definecolor{currentstroke}{rgb}{0.000000,0.000000,0.000000}%
\pgfsetstrokecolor{currentstroke}%
\pgfsetdash{}{0pt}%
\pgfpathmoveto{\pgfqpoint{1.859908in}{1.480155in}}%
\pgfpathlineto{\pgfqpoint{1.877097in}{1.510120in}}%
\pgfpathlineto{\pgfqpoint{1.874769in}{1.536565in}}%
\pgfpathlineto{\pgfqpoint{1.857744in}{1.505332in}}%
\pgfpathlineto{\pgfqpoint{1.859908in}{1.480155in}}%
\pgfpathclose%
\pgfusepath{fill}%
\end{pgfscope}%
\begin{pgfscope}%
\pgfpathrectangle{\pgfqpoint{0.000000in}{0.000000in}}{\pgfqpoint{3.000000in}{3.000000in}}%
\pgfusepath{clip}%
\pgfsetbuttcap%
\pgfsetroundjoin%
\definecolor{currentfill}{rgb}{0.856506,0.000000,0.000000}%
\pgfsetfillcolor{currentfill}%
\pgfsetlinewidth{0.000000pt}%
\definecolor{currentstroke}{rgb}{0.000000,0.000000,0.000000}%
\pgfsetstrokecolor{currentstroke}%
\pgfsetdash{}{0pt}%
\pgfpathmoveto{\pgfqpoint{0.915981in}{2.134347in}}%
\pgfpathlineto{\pgfqpoint{0.901433in}{2.152155in}}%
\pgfpathlineto{\pgfqpoint{0.847712in}{2.103034in}}%
\pgfpathlineto{\pgfqpoint{0.863569in}{2.086253in}}%
\pgfpathlineto{\pgfqpoint{0.915981in}{2.134347in}}%
\pgfpathclose%
\pgfusepath{fill}%
\end{pgfscope}%
\begin{pgfscope}%
\pgfpathrectangle{\pgfqpoint{0.000000in}{0.000000in}}{\pgfqpoint{3.000000in}{3.000000in}}%
\pgfusepath{clip}%
\pgfsetbuttcap%
\pgfsetroundjoin%
\definecolor{currentfill}{rgb}{0.000000,0.503922,1.000000}%
\pgfsetfillcolor{currentfill}%
\pgfsetlinewidth{0.000000pt}%
\definecolor{currentstroke}{rgb}{0.000000,0.000000,0.000000}%
\pgfsetstrokecolor{currentstroke}%
\pgfsetdash{}{0pt}%
\pgfpathmoveto{\pgfqpoint{1.325846in}{1.246968in}}%
\pgfpathlineto{\pgfqpoint{1.310047in}{1.289320in}}%
\pgfpathlineto{\pgfqpoint{1.328545in}{1.271423in}}%
\pgfpathlineto{\pgfqpoint{1.343101in}{1.230226in}}%
\pgfpathlineto{\pgfqpoint{1.325846in}{1.246968in}}%
\pgfpathclose%
\pgfusepath{fill}%
\end{pgfscope}%
\begin{pgfscope}%
\pgfpathrectangle{\pgfqpoint{0.000000in}{0.000000in}}{\pgfqpoint{3.000000in}{3.000000in}}%
\pgfusepath{clip}%
\pgfsetbuttcap%
\pgfsetroundjoin%
\definecolor{currentfill}{rgb}{0.000000,0.676471,1.000000}%
\pgfsetfillcolor{currentfill}%
\pgfsetlinewidth{0.000000pt}%
\definecolor{currentstroke}{rgb}{0.000000,0.000000,0.000000}%
\pgfsetstrokecolor{currentstroke}%
\pgfsetdash{}{0pt}%
\pgfpathmoveto{\pgfqpoint{1.297316in}{1.308443in}}%
\pgfpathlineto{\pgfqpoint{1.280654in}{1.347754in}}%
\pgfpathlineto{\pgfqpoint{1.294223in}{1.327402in}}%
\pgfpathlineto{\pgfqpoint{1.310047in}{1.289320in}}%
\pgfpathlineto{\pgfqpoint{1.297316in}{1.308443in}}%
\pgfpathclose%
\pgfusepath{fill}%
\end{pgfscope}%
\begin{pgfscope}%
\pgfpathrectangle{\pgfqpoint{0.000000in}{0.000000in}}{\pgfqpoint{3.000000in}{3.000000in}}%
\pgfusepath{clip}%
\pgfsetbuttcap%
\pgfsetroundjoin%
\definecolor{currentfill}{rgb}{1.000000,0.814089,0.000000}%
\pgfsetfillcolor{currentfill}%
\pgfsetlinewidth{0.000000pt}%
\definecolor{currentstroke}{rgb}{0.000000,0.000000,0.000000}%
\pgfsetstrokecolor{currentstroke}%
\pgfsetdash{}{0pt}%
\pgfpathmoveto{\pgfqpoint{2.012494in}{1.794020in}}%
\pgfpathlineto{\pgfqpoint{2.028985in}{1.815345in}}%
\pgfpathlineto{\pgfqpoint{2.000660in}{1.851502in}}%
\pgfpathlineto{\pgfqpoint{1.985185in}{1.829043in}}%
\pgfpathlineto{\pgfqpoint{2.012494in}{1.794020in}}%
\pgfpathclose%
\pgfusepath{fill}%
\end{pgfscope}%
\begin{pgfscope}%
\pgfpathrectangle{\pgfqpoint{0.000000in}{0.000000in}}{\pgfqpoint{3.000000in}{3.000000in}}%
\pgfusepath{clip}%
\pgfsetbuttcap%
\pgfsetroundjoin%
\definecolor{currentfill}{rgb}{0.803030,0.000000,0.000000}%
\pgfsetfillcolor{currentfill}%
\pgfsetlinewidth{0.000000pt}%
\definecolor{currentstroke}{rgb}{0.000000,0.000000,0.000000}%
\pgfsetstrokecolor{currentstroke}%
\pgfsetdash{}{0pt}%
\pgfpathmoveto{\pgfqpoint{0.901433in}{2.152155in}}%
\pgfpathlineto{\pgfqpoint{0.886890in}{2.169736in}}%
\pgfpathlineto{\pgfqpoint{0.831855in}{2.119592in}}%
\pgfpathlineto{\pgfqpoint{0.847712in}{2.103034in}}%
\pgfpathlineto{\pgfqpoint{0.901433in}{2.152155in}}%
\pgfpathclose%
\pgfusepath{fill}%
\end{pgfscope}%
\begin{pgfscope}%
\pgfpathrectangle{\pgfqpoint{0.000000in}{0.000000in}}{\pgfqpoint{3.000000in}{3.000000in}}%
\pgfusepath{clip}%
\pgfsetbuttcap%
\pgfsetroundjoin%
\definecolor{currentfill}{rgb}{0.085389,1.000000,0.882353}%
\pgfsetfillcolor{currentfill}%
\pgfsetlinewidth{0.000000pt}%
\definecolor{currentstroke}{rgb}{0.000000,0.000000,0.000000}%
\pgfsetstrokecolor{currentstroke}%
\pgfsetdash{}{0pt}%
\pgfpathmoveto{\pgfqpoint{1.820393in}{1.390956in}}%
\pgfpathlineto{\pgfqpoint{1.837282in}{1.424242in}}%
\pgfpathlineto{\pgfqpoint{1.842734in}{1.448073in}}%
\pgfpathlineto{\pgfqpoint{1.825575in}{1.413514in}}%
\pgfpathlineto{\pgfqpoint{1.820393in}{1.390956in}}%
\pgfpathclose%
\pgfusepath{fill}%
\end{pgfscope}%
\begin{pgfscope}%
\pgfpathrectangle{\pgfqpoint{0.000000in}{0.000000in}}{\pgfqpoint{3.000000in}{3.000000in}}%
\pgfusepath{clip}%
\pgfsetbuttcap%
\pgfsetroundjoin%
\definecolor{currentfill}{rgb}{0.731729,0.000000,0.000000}%
\pgfsetfillcolor{currentfill}%
\pgfsetlinewidth{0.000000pt}%
\definecolor{currentstroke}{rgb}{0.000000,0.000000,0.000000}%
\pgfsetstrokecolor{currentstroke}%
\pgfsetdash{}{0pt}%
\pgfpathmoveto{\pgfqpoint{0.886890in}{2.169736in}}%
\pgfpathlineto{\pgfqpoint{0.872350in}{2.187103in}}%
\pgfpathlineto{\pgfqpoint{0.815999in}{2.135939in}}%
\pgfpathlineto{\pgfqpoint{0.831855in}{2.119592in}}%
\pgfpathlineto{\pgfqpoint{0.886890in}{2.169736in}}%
\pgfpathclose%
\pgfusepath{fill}%
\end{pgfscope}%
\begin{pgfscope}%
\pgfpathrectangle{\pgfqpoint{0.000000in}{0.000000in}}{\pgfqpoint{3.000000in}{3.000000in}}%
\pgfusepath{clip}%
\pgfsetbuttcap%
\pgfsetroundjoin%
\definecolor{currentfill}{rgb}{0.000000,0.064706,1.000000}%
\pgfsetfillcolor{currentfill}%
\pgfsetlinewidth{0.000000pt}%
\definecolor{currentstroke}{rgb}{0.000000,0.000000,0.000000}%
\pgfsetstrokecolor{currentstroke}%
\pgfsetdash{}{0pt}%
\pgfpathmoveto{\pgfqpoint{1.439793in}{1.096867in}}%
\pgfpathlineto{\pgfqpoint{1.431084in}{1.147240in}}%
\pgfpathlineto{\pgfqpoint{1.462027in}{1.139529in}}%
\pgfpathlineto{\pgfqpoint{1.468279in}{1.089733in}}%
\pgfpathlineto{\pgfqpoint{1.439793in}{1.096867in}}%
\pgfpathclose%
\pgfusepath{fill}%
\end{pgfscope}%
\begin{pgfscope}%
\pgfpathrectangle{\pgfqpoint{0.000000in}{0.000000in}}{\pgfqpoint{3.000000in}{3.000000in}}%
\pgfusepath{clip}%
\pgfsetbuttcap%
\pgfsetroundjoin%
\definecolor{currentfill}{rgb}{0.678253,0.000000,0.000000}%
\pgfsetfillcolor{currentfill}%
\pgfsetlinewidth{0.000000pt}%
\definecolor{currentstroke}{rgb}{0.000000,0.000000,0.000000}%
\pgfsetstrokecolor{currentstroke}%
\pgfsetdash{}{0pt}%
\pgfpathmoveto{\pgfqpoint{0.872350in}{2.187103in}}%
\pgfpathlineto{\pgfqpoint{0.857815in}{2.204267in}}%
\pgfpathlineto{\pgfqpoint{0.800144in}{2.152087in}}%
\pgfpathlineto{\pgfqpoint{0.815999in}{2.135939in}}%
\pgfpathlineto{\pgfqpoint{0.872350in}{2.187103in}}%
\pgfpathclose%
\pgfusepath{fill}%
\end{pgfscope}%
\begin{pgfscope}%
\pgfpathrectangle{\pgfqpoint{0.000000in}{0.000000in}}{\pgfqpoint{3.000000in}{3.000000in}}%
\pgfusepath{clip}%
\pgfsetbuttcap%
\pgfsetroundjoin%
\definecolor{currentfill}{rgb}{0.606952,0.000000,0.000000}%
\pgfsetfillcolor{currentfill}%
\pgfsetlinewidth{0.000000pt}%
\definecolor{currentstroke}{rgb}{0.000000,0.000000,0.000000}%
\pgfsetstrokecolor{currentstroke}%
\pgfsetdash{}{0pt}%
\pgfpathmoveto{\pgfqpoint{0.857815in}{2.204267in}}%
\pgfpathlineto{\pgfqpoint{0.843284in}{2.221239in}}%
\pgfpathlineto{\pgfqpoint{0.784289in}{2.168045in}}%
\pgfpathlineto{\pgfqpoint{0.800144in}{2.152087in}}%
\pgfpathlineto{\pgfqpoint{0.857815in}{2.204267in}}%
\pgfpathclose%
\pgfusepath{fill}%
\end{pgfscope}%
\begin{pgfscope}%
\pgfpathrectangle{\pgfqpoint{0.000000in}{0.000000in}}{\pgfqpoint{3.000000in}{3.000000in}}%
\pgfusepath{clip}%
\pgfsetbuttcap%
\pgfsetroundjoin%
\definecolor{currentfill}{rgb}{0.000000,0.300000,1.000000}%
\pgfsetfillcolor{currentfill}%
\pgfsetlinewidth{0.000000pt}%
\definecolor{currentstroke}{rgb}{0.000000,0.000000,0.000000}%
\pgfsetstrokecolor{currentstroke}%
\pgfsetdash{}{0pt}%
\pgfpathmoveto{\pgfqpoint{1.686812in}{1.161235in}}%
\pgfpathlineto{\pgfqpoint{1.698461in}{1.206025in}}%
\pgfpathlineto{\pgfqpoint{1.723691in}{1.219917in}}%
\pgfpathlineto{\pgfqpoint{1.710199in}{1.174163in}}%
\pgfpathlineto{\pgfqpoint{1.686812in}{1.161235in}}%
\pgfpathclose%
\pgfusepath{fill}%
\end{pgfscope}%
\begin{pgfscope}%
\pgfpathrectangle{\pgfqpoint{0.000000in}{0.000000in}}{\pgfqpoint{3.000000in}{3.000000in}}%
\pgfusepath{clip}%
\pgfsetbuttcap%
\pgfsetroundjoin%
\definecolor{currentfill}{rgb}{1.000000,0.741467,0.000000}%
\pgfsetfillcolor{currentfill}%
\pgfsetlinewidth{0.000000pt}%
\definecolor{currentstroke}{rgb}{0.000000,0.000000,0.000000}%
\pgfsetstrokecolor{currentstroke}%
\pgfsetdash{}{0pt}%
\pgfpathmoveto{\pgfqpoint{2.028985in}{1.815345in}}%
\pgfpathlineto{\pgfqpoint{2.045480in}{1.836076in}}%
\pgfpathlineto{\pgfqpoint{2.016135in}{1.873363in}}%
\pgfpathlineto{\pgfqpoint{2.000660in}{1.851502in}}%
\pgfpathlineto{\pgfqpoint{2.028985in}{1.815345in}}%
\pgfpathclose%
\pgfusepath{fill}%
\end{pgfscope}%
\begin{pgfscope}%
\pgfpathrectangle{\pgfqpoint{0.000000in}{0.000000in}}{\pgfqpoint{3.000000in}{3.000000in}}%
\pgfusepath{clip}%
\pgfsetbuttcap%
\pgfsetroundjoin%
\definecolor{currentfill}{rgb}{0.553476,0.000000,0.000000}%
\pgfsetfillcolor{currentfill}%
\pgfsetlinewidth{0.000000pt}%
\definecolor{currentstroke}{rgb}{0.000000,0.000000,0.000000}%
\pgfsetstrokecolor{currentstroke}%
\pgfsetdash{}{0pt}%
\pgfpathmoveto{\pgfqpoint{0.843284in}{2.221239in}}%
\pgfpathlineto{\pgfqpoint{0.828758in}{2.238028in}}%
\pgfpathlineto{\pgfqpoint{0.768434in}{2.183824in}}%
\pgfpathlineto{\pgfqpoint{0.784289in}{2.168045in}}%
\pgfpathlineto{\pgfqpoint{0.843284in}{2.221239in}}%
\pgfpathclose%
\pgfusepath{fill}%
\end{pgfscope}%
\begin{pgfscope}%
\pgfpathrectangle{\pgfqpoint{0.000000in}{0.000000in}}{\pgfqpoint{3.000000in}{3.000000in}}%
\pgfusepath{clip}%
\pgfsetbuttcap%
\pgfsetroundjoin%
\definecolor{currentfill}{rgb}{0.958254,0.973856,0.009488}%
\pgfsetfillcolor{currentfill}%
\pgfsetlinewidth{0.000000pt}%
\definecolor{currentstroke}{rgb}{0.000000,0.000000,0.000000}%
\pgfsetstrokecolor{currentstroke}%
\pgfsetdash{}{0pt}%
\pgfpathmoveto{\pgfqpoint{1.117281in}{1.771416in}}%
\pgfpathlineto{\pgfqpoint{1.101427in}{1.794877in}}%
\pgfpathlineto{\pgfqpoint{1.078719in}{1.760324in}}%
\pgfpathlineto{\pgfqpoint{1.095454in}{1.738029in}}%
\pgfpathlineto{\pgfqpoint{1.117281in}{1.771416in}}%
\pgfpathclose%
\pgfusepath{fill}%
\end{pgfscope}%
\begin{pgfscope}%
\pgfpathrectangle{\pgfqpoint{0.000000in}{0.000000in}}{\pgfqpoint{3.000000in}{3.000000in}}%
\pgfusepath{clip}%
\pgfsetbuttcap%
\pgfsetroundjoin%
\definecolor{currentfill}{rgb}{0.667299,1.000000,0.300443}%
\pgfsetfillcolor{currentfill}%
\pgfsetlinewidth{0.000000pt}%
\definecolor{currentstroke}{rgb}{0.000000,0.000000,0.000000}%
\pgfsetstrokecolor{currentstroke}%
\pgfsetdash{}{0pt}%
\pgfpathmoveto{\pgfqpoint{1.925904in}{1.620220in}}%
\pgfpathlineto{\pgfqpoint{1.942968in}{1.645422in}}%
\pgfpathlineto{\pgfqpoint{1.930098in}{1.676186in}}%
\pgfpathlineto{\pgfqpoint{1.913632in}{1.649764in}}%
\pgfpathlineto{\pgfqpoint{1.925904in}{1.620220in}}%
\pgfpathclose%
\pgfusepath{fill}%
\end{pgfscope}%
\begin{pgfscope}%
\pgfpathrectangle{\pgfqpoint{0.000000in}{0.000000in}}{\pgfqpoint{3.000000in}{3.000000in}}%
\pgfusepath{clip}%
\pgfsetbuttcap%
\pgfsetroundjoin%
\definecolor{currentfill}{rgb}{0.500000,0.000000,0.000000}%
\pgfsetfillcolor{currentfill}%
\pgfsetlinewidth{0.000000pt}%
\definecolor{currentstroke}{rgb}{0.000000,0.000000,0.000000}%
\pgfsetstrokecolor{currentstroke}%
\pgfsetdash{}{0pt}%
\pgfpathmoveto{\pgfqpoint{0.828758in}{2.238028in}}%
\pgfpathlineto{\pgfqpoint{0.814235in}{2.254642in}}%
\pgfpathlineto{\pgfqpoint{0.752581in}{2.199431in}}%
\pgfpathlineto{\pgfqpoint{0.768434in}{2.183824in}}%
\pgfpathlineto{\pgfqpoint{0.828758in}{2.238028in}}%
\pgfpathclose%
\pgfusepath{fill}%
\end{pgfscope}%
\begin{pgfscope}%
\pgfpathrectangle{\pgfqpoint{0.000000in}{0.000000in}}{\pgfqpoint{3.000000in}{3.000000in}}%
\pgfusepath{clip}%
\pgfsetbuttcap%
\pgfsetroundjoin%
\definecolor{currentfill}{rgb}{0.000000,0.064706,1.000000}%
\pgfsetfillcolor{currentfill}%
\pgfsetlinewidth{0.000000pt}%
\definecolor{currentstroke}{rgb}{0.000000,0.000000,0.000000}%
\pgfsetstrokecolor{currentstroke}%
\pgfsetdash{}{0pt}%
\pgfpathmoveto{\pgfqpoint{1.592747in}{1.086314in}}%
\pgfpathlineto{\pgfqpoint{1.597266in}{1.135833in}}%
\pgfpathlineto{\pgfqpoint{1.629630in}{1.141817in}}%
\pgfpathlineto{\pgfqpoint{1.622538in}{1.091850in}}%
\pgfpathlineto{\pgfqpoint{1.592747in}{1.086314in}}%
\pgfpathclose%
\pgfusepath{fill}%
\end{pgfscope}%
\begin{pgfscope}%
\pgfpathrectangle{\pgfqpoint{0.000000in}{0.000000in}}{\pgfqpoint{3.000000in}{3.000000in}}%
\pgfusepath{clip}%
\pgfsetbuttcap%
\pgfsetroundjoin%
\definecolor{currentfill}{rgb}{1.000000,0.668845,0.000000}%
\pgfsetfillcolor{currentfill}%
\pgfsetlinewidth{0.000000pt}%
\definecolor{currentstroke}{rgb}{0.000000,0.000000,0.000000}%
\pgfsetstrokecolor{currentstroke}%
\pgfsetdash{}{0pt}%
\pgfpathmoveto{\pgfqpoint{2.045480in}{1.836076in}}%
\pgfpathlineto{\pgfqpoint{2.061979in}{1.856262in}}%
\pgfpathlineto{\pgfqpoint{2.031608in}{1.894677in}}%
\pgfpathlineto{\pgfqpoint{2.016135in}{1.873363in}}%
\pgfpathlineto{\pgfqpoint{2.045480in}{1.836076in}}%
\pgfpathclose%
\pgfusepath{fill}%
\end{pgfscope}%
\begin{pgfscope}%
\pgfpathrectangle{\pgfqpoint{0.000000in}{0.000000in}}{\pgfqpoint{3.000000in}{3.000000in}}%
\pgfusepath{clip}%
\pgfsetbuttcap%
\pgfsetroundjoin%
\definecolor{currentfill}{rgb}{0.578748,1.000000,0.388994}%
\pgfsetfillcolor{currentfill}%
\pgfsetlinewidth{0.000000pt}%
\definecolor{currentstroke}{rgb}{0.000000,0.000000,0.000000}%
\pgfsetstrokecolor{currentstroke}%
\pgfsetdash{}{0pt}%
\pgfpathmoveto{\pgfqpoint{1.179035in}{1.612787in}}%
\pgfpathlineto{\pgfqpoint{1.162331in}{1.640037in}}%
\pgfpathlineto{\pgfqpoint{1.153167in}{1.610183in}}%
\pgfpathlineto{\pgfqpoint{1.170329in}{1.584172in}}%
\pgfpathlineto{\pgfqpoint{1.179035in}{1.612787in}}%
\pgfpathclose%
\pgfusepath{fill}%
\end{pgfscope}%
\begin{pgfscope}%
\pgfpathrectangle{\pgfqpoint{0.000000in}{0.000000in}}{\pgfqpoint{3.000000in}{3.000000in}}%
\pgfusepath{clip}%
\pgfsetbuttcap%
\pgfsetroundjoin%
\definecolor{currentfill}{rgb}{0.000000,0.064706,1.000000}%
\pgfsetfillcolor{currentfill}%
\pgfsetlinewidth{0.000000pt}%
\definecolor{currentstroke}{rgb}{0.000000,0.000000,0.000000}%
\pgfsetstrokecolor{currentstroke}%
\pgfsetdash{}{0pt}%
\pgfpathmoveto{\pgfqpoint{1.468279in}{1.089733in}}%
\pgfpathlineto{\pgfqpoint{1.462027in}{1.139529in}}%
\pgfpathlineto{\pgfqpoint{1.494966in}{1.134435in}}%
\pgfpathlineto{\pgfqpoint{1.498597in}{1.085021in}}%
\pgfpathlineto{\pgfqpoint{1.468279in}{1.089733in}}%
\pgfpathclose%
\pgfusepath{fill}%
\end{pgfscope}%
\begin{pgfscope}%
\pgfpathrectangle{\pgfqpoint{0.000000in}{0.000000in}}{\pgfqpoint{3.000000in}{3.000000in}}%
\pgfusepath{clip}%
\pgfsetbuttcap%
\pgfsetroundjoin%
\definecolor{currentfill}{rgb}{0.000000,0.300000,1.000000}%
\pgfsetfillcolor{currentfill}%
\pgfsetlinewidth{0.000000pt}%
\definecolor{currentstroke}{rgb}{0.000000,0.000000,0.000000}%
\pgfsetstrokecolor{currentstroke}%
\pgfsetdash{}{0pt}%
\pgfpathmoveto{\pgfqpoint{1.378229in}{1.169645in}}%
\pgfpathlineto{\pgfqpoint{1.365315in}{1.215063in}}%
\pgfpathlineto{\pgfqpoint{1.391952in}{1.201868in}}%
\pgfpathlineto{\pgfqpoint{1.402916in}{1.157367in}}%
\pgfpathlineto{\pgfqpoint{1.378229in}{1.169645in}}%
\pgfpathclose%
\pgfusepath{fill}%
\end{pgfscope}%
\begin{pgfscope}%
\pgfpathrectangle{\pgfqpoint{0.000000in}{0.000000in}}{\pgfqpoint{3.000000in}{3.000000in}}%
\pgfusepath{clip}%
\pgfsetbuttcap%
\pgfsetroundjoin%
\definecolor{currentfill}{rgb}{0.300443,1.000000,0.667299}%
\pgfsetfillcolor{currentfill}%
\pgfsetlinewidth{0.000000pt}%
\definecolor{currentstroke}{rgb}{0.000000,0.000000,0.000000}%
\pgfsetstrokecolor{currentstroke}%
\pgfsetdash{}{0pt}%
\pgfpathmoveto{\pgfqpoint{1.221743in}{1.496970in}}%
\pgfpathlineto{\pgfqpoint{1.204617in}{1.527783in}}%
\pgfpathlineto{\pgfqpoint{1.205053in}{1.501289in}}%
\pgfpathlineto{\pgfqpoint{1.222203in}{1.471748in}}%
\pgfpathlineto{\pgfqpoint{1.221743in}{1.496970in}}%
\pgfpathclose%
\pgfusepath{fill}%
\end{pgfscope}%
\begin{pgfscope}%
\pgfpathrectangle{\pgfqpoint{0.000000in}{0.000000in}}{\pgfqpoint{3.000000in}{3.000000in}}%
\pgfusepath{clip}%
\pgfsetbuttcap%
\pgfsetroundjoin%
\definecolor{currentfill}{rgb}{1.000000,0.886710,0.000000}%
\pgfsetfillcolor{currentfill}%
\pgfsetlinewidth{0.000000pt}%
\definecolor{currentstroke}{rgb}{0.000000,0.000000,0.000000}%
\pgfsetstrokecolor{currentstroke}%
\pgfsetdash{}{0pt}%
\pgfpathmoveto{\pgfqpoint{1.101427in}{1.794877in}}%
\pgfpathlineto{\pgfqpoint{1.085573in}{1.817620in}}%
\pgfpathlineto{\pgfqpoint{1.061980in}{1.781904in}}%
\pgfpathlineto{\pgfqpoint{1.078719in}{1.760324in}}%
\pgfpathlineto{\pgfqpoint{1.101427in}{1.794877in}}%
\pgfpathclose%
\pgfusepath{fill}%
\end{pgfscope}%
\begin{pgfscope}%
\pgfpathrectangle{\pgfqpoint{0.000000in}{0.000000in}}{\pgfqpoint{3.000000in}{3.000000in}}%
\pgfusepath{clip}%
\pgfsetbuttcap%
\pgfsetroundjoin%
\definecolor{currentfill}{rgb}{0.085389,1.000000,0.882353}%
\pgfsetfillcolor{currentfill}%
\pgfsetlinewidth{0.000000pt}%
\definecolor{currentstroke}{rgb}{0.000000,0.000000,0.000000}%
\pgfsetstrokecolor{currentstroke}%
\pgfsetdash{}{0pt}%
\pgfpathmoveto{\pgfqpoint{1.256453in}{1.405958in}}%
\pgfpathlineto{\pgfqpoint{1.239336in}{1.440092in}}%
\pgfpathlineto{\pgfqpoint{1.247265in}{1.416417in}}%
\pgfpathlineto{\pgfqpoint{1.263970in}{1.383552in}}%
\pgfpathlineto{\pgfqpoint{1.256453in}{1.405958in}}%
\pgfpathclose%
\pgfusepath{fill}%
\end{pgfscope}%
\begin{pgfscope}%
\pgfpathrectangle{\pgfqpoint{0.000000in}{0.000000in}}{\pgfqpoint{3.000000in}{3.000000in}}%
\pgfusepath{clip}%
\pgfsetbuttcap%
\pgfsetroundjoin%
\definecolor{currentfill}{rgb}{0.000000,0.849020,1.000000}%
\pgfsetfillcolor{currentfill}%
\pgfsetlinewidth{0.000000pt}%
\definecolor{currentstroke}{rgb}{0.000000,0.000000,0.000000}%
\pgfsetstrokecolor{currentstroke}%
\pgfsetdash{}{0pt}%
\pgfpathmoveto{\pgfqpoint{1.792085in}{1.334067in}}%
\pgfpathlineto{\pgfqpoint{1.808257in}{1.369039in}}%
\pgfpathlineto{\pgfqpoint{1.820393in}{1.390956in}}%
\pgfpathlineto{\pgfqpoint{1.803523in}{1.354739in}}%
\pgfpathlineto{\pgfqpoint{1.792085in}{1.334067in}}%
\pgfpathclose%
\pgfusepath{fill}%
\end{pgfscope}%
\begin{pgfscope}%
\pgfpathrectangle{\pgfqpoint{0.000000in}{0.000000in}}{\pgfqpoint{3.000000in}{3.000000in}}%
\pgfusepath{clip}%
\pgfsetbuttcap%
\pgfsetroundjoin%
\definecolor{currentfill}{rgb}{0.401645,1.000000,0.566097}%
\pgfsetfillcolor{currentfill}%
\pgfsetlinewidth{0.000000pt}%
\definecolor{currentstroke}{rgb}{0.000000,0.000000,0.000000}%
\pgfsetstrokecolor{currentstroke}%
\pgfsetdash{}{0pt}%
\pgfpathmoveto{\pgfqpoint{1.877097in}{1.510120in}}%
\pgfpathlineto{\pgfqpoint{1.894302in}{1.538263in}}%
\pgfpathlineto{\pgfqpoint{1.891804in}{1.565972in}}%
\pgfpathlineto{\pgfqpoint{1.874769in}{1.536565in}}%
\pgfpathlineto{\pgfqpoint{1.877097in}{1.510120in}}%
\pgfpathclose%
\pgfusepath{fill}%
\end{pgfscope}%
\begin{pgfscope}%
\pgfpathrectangle{\pgfqpoint{0.000000in}{0.000000in}}{\pgfqpoint{3.000000in}{3.000000in}}%
\pgfusepath{clip}%
\pgfsetbuttcap%
\pgfsetroundjoin%
\definecolor{currentfill}{rgb}{1.000000,0.610748,0.000000}%
\pgfsetfillcolor{currentfill}%
\pgfsetlinewidth{0.000000pt}%
\definecolor{currentstroke}{rgb}{0.000000,0.000000,0.000000}%
\pgfsetstrokecolor{currentstroke}%
\pgfsetdash{}{0pt}%
\pgfpathmoveto{\pgfqpoint{2.061979in}{1.856262in}}%
\pgfpathlineto{\pgfqpoint{2.078481in}{1.875947in}}%
\pgfpathlineto{\pgfqpoint{2.047081in}{1.915487in}}%
\pgfpathlineto{\pgfqpoint{2.031608in}{1.894677in}}%
\pgfpathlineto{\pgfqpoint{2.061979in}{1.856262in}}%
\pgfpathclose%
\pgfusepath{fill}%
\end{pgfscope}%
\begin{pgfscope}%
\pgfpathrectangle{\pgfqpoint{0.000000in}{0.000000in}}{\pgfqpoint{3.000000in}{3.000000in}}%
\pgfusepath{clip}%
\pgfsetbuttcap%
\pgfsetroundjoin%
\definecolor{currentfill}{rgb}{0.000000,0.064706,1.000000}%
\pgfsetfillcolor{currentfill}%
\pgfsetlinewidth{0.000000pt}%
\definecolor{currentstroke}{rgb}{0.000000,0.000000,0.000000}%
\pgfsetstrokecolor{currentstroke}%
\pgfsetdash{}{0pt}%
\pgfpathmoveto{\pgfqpoint{1.561632in}{1.083289in}}%
\pgfpathlineto{\pgfqpoint{1.563458in}{1.132562in}}%
\pgfpathlineto{\pgfqpoint{1.597266in}{1.135833in}}%
\pgfpathlineto{\pgfqpoint{1.592747in}{1.086314in}}%
\pgfpathlineto{\pgfqpoint{1.561632in}{1.083289in}}%
\pgfpathclose%
\pgfusepath{fill}%
\end{pgfscope}%
\begin{pgfscope}%
\pgfpathrectangle{\pgfqpoint{0.000000in}{0.000000in}}{\pgfqpoint{3.000000in}{3.000000in}}%
\pgfusepath{clip}%
\pgfsetbuttcap%
\pgfsetroundjoin%
\definecolor{currentfill}{rgb}{0.743201,1.000000,0.224541}%
\pgfsetfillcolor{currentfill}%
\pgfsetlinewidth{0.000000pt}%
\definecolor{currentstroke}{rgb}{0.000000,0.000000,0.000000}%
\pgfsetstrokecolor{currentstroke}%
\pgfsetdash{}{0pt}%
\pgfpathmoveto{\pgfqpoint{1.942968in}{1.645422in}}%
\pgfpathlineto{\pgfqpoint{1.960042in}{1.669536in}}%
\pgfpathlineto{\pgfqpoint{1.946569in}{1.701517in}}%
\pgfpathlineto{\pgfqpoint{1.930098in}{1.676186in}}%
\pgfpathlineto{\pgfqpoint{1.942968in}{1.645422in}}%
\pgfpathclose%
\pgfusepath{fill}%
\end{pgfscope}%
\begin{pgfscope}%
\pgfpathrectangle{\pgfqpoint{0.000000in}{0.000000in}}{\pgfqpoint{3.000000in}{3.000000in}}%
\pgfusepath{clip}%
\pgfsetbuttcap%
\pgfsetroundjoin%
\definecolor{currentfill}{rgb}{0.000000,0.503922,1.000000}%
\pgfsetfillcolor{currentfill}%
\pgfsetlinewidth{0.000000pt}%
\definecolor{currentstroke}{rgb}{0.000000,0.000000,0.000000}%
\pgfsetstrokecolor{currentstroke}%
\pgfsetdash{}{0pt}%
\pgfpathmoveto{\pgfqpoint{1.723691in}{1.219917in}}%
\pgfpathlineto{\pgfqpoint{1.737209in}{1.260400in}}%
\pgfpathlineto{\pgfqpoint{1.759319in}{1.277223in}}%
\pgfpathlineto{\pgfqpoint{1.744306in}{1.235652in}}%
\pgfpathlineto{\pgfqpoint{1.723691in}{1.219917in}}%
\pgfpathclose%
\pgfusepath{fill}%
\end{pgfscope}%
\begin{pgfscope}%
\pgfpathrectangle{\pgfqpoint{0.000000in}{0.000000in}}{\pgfqpoint{3.000000in}{3.000000in}}%
\pgfusepath{clip}%
\pgfsetbuttcap%
\pgfsetroundjoin%
\definecolor{currentfill}{rgb}{0.000000,0.064706,1.000000}%
\pgfsetfillcolor{currentfill}%
\pgfsetlinewidth{0.000000pt}%
\definecolor{currentstroke}{rgb}{0.000000,0.000000,0.000000}%
\pgfsetstrokecolor{currentstroke}%
\pgfsetdash{}{0pt}%
\pgfpathmoveto{\pgfqpoint{1.498597in}{1.085021in}}%
\pgfpathlineto{\pgfqpoint{1.494966in}{1.134435in}}%
\pgfpathlineto{\pgfqpoint{1.529066in}{1.132093in}}%
\pgfpathlineto{\pgfqpoint{1.529980in}{1.082854in}}%
\pgfpathlineto{\pgfqpoint{1.498597in}{1.085021in}}%
\pgfpathclose%
\pgfusepath{fill}%
\end{pgfscope}%
\begin{pgfscope}%
\pgfpathrectangle{\pgfqpoint{0.000000in}{0.000000in}}{\pgfqpoint{3.000000in}{3.000000in}}%
\pgfusepath{clip}%
\pgfsetbuttcap%
\pgfsetroundjoin%
\definecolor{currentfill}{rgb}{0.000000,0.064706,1.000000}%
\pgfsetfillcolor{currentfill}%
\pgfsetlinewidth{0.000000pt}%
\definecolor{currentstroke}{rgb}{0.000000,0.000000,0.000000}%
\pgfsetstrokecolor{currentstroke}%
\pgfsetdash{}{0pt}%
\pgfpathmoveto{\pgfqpoint{1.529980in}{1.082854in}}%
\pgfpathlineto{\pgfqpoint{1.529066in}{1.132093in}}%
\pgfpathlineto{\pgfqpoint{1.563458in}{1.132562in}}%
\pgfpathlineto{\pgfqpoint{1.561632in}{1.083289in}}%
\pgfpathlineto{\pgfqpoint{1.529980in}{1.082854in}}%
\pgfpathclose%
\pgfusepath{fill}%
\end{pgfscope}%
\begin{pgfscope}%
\pgfpathrectangle{\pgfqpoint{0.000000in}{0.000000in}}{\pgfqpoint{3.000000in}{3.000000in}}%
\pgfusepath{clip}%
\pgfsetbuttcap%
\pgfsetroundjoin%
\definecolor{currentfill}{rgb}{1.000000,0.814089,0.000000}%
\pgfsetfillcolor{currentfill}%
\pgfsetlinewidth{0.000000pt}%
\definecolor{currentstroke}{rgb}{0.000000,0.000000,0.000000}%
\pgfsetstrokecolor{currentstroke}%
\pgfsetdash{}{0pt}%
\pgfpathmoveto{\pgfqpoint{1.085573in}{1.817620in}}%
\pgfpathlineto{\pgfqpoint{1.069718in}{1.839710in}}%
\pgfpathlineto{\pgfqpoint{1.045235in}{1.802834in}}%
\pgfpathlineto{\pgfqpoint{1.061980in}{1.781904in}}%
\pgfpathlineto{\pgfqpoint{1.085573in}{1.817620in}}%
\pgfpathclose%
\pgfusepath{fill}%
\end{pgfscope}%
\begin{pgfscope}%
\pgfpathrectangle{\pgfqpoint{0.000000in}{0.000000in}}{\pgfqpoint{3.000000in}{3.000000in}}%
\pgfusepath{clip}%
\pgfsetbuttcap%
\pgfsetroundjoin%
\definecolor{currentfill}{rgb}{0.000000,0.676471,1.000000}%
\pgfsetfillcolor{currentfill}%
\pgfsetlinewidth{0.000000pt}%
\definecolor{currentstroke}{rgb}{0.000000,0.000000,0.000000}%
\pgfsetstrokecolor{currentstroke}%
\pgfsetdash{}{0pt}%
\pgfpathmoveto{\pgfqpoint{1.759319in}{1.277223in}}%
\pgfpathlineto{\pgfqpoint{1.774359in}{1.314526in}}%
\pgfpathlineto{\pgfqpoint{1.792085in}{1.334067in}}%
\pgfpathlineto{\pgfqpoint{1.775936in}{1.295581in}}%
\pgfpathlineto{\pgfqpoint{1.759319in}{1.277223in}}%
\pgfpathclose%
\pgfusepath{fill}%
\end{pgfscope}%
\begin{pgfscope}%
\pgfpathrectangle{\pgfqpoint{0.000000in}{0.000000in}}{\pgfqpoint{3.000000in}{3.000000in}}%
\pgfusepath{clip}%
\pgfsetbuttcap%
\pgfsetroundjoin%
\definecolor{currentfill}{rgb}{1.000000,0.538126,0.000000}%
\pgfsetfillcolor{currentfill}%
\pgfsetlinewidth{0.000000pt}%
\definecolor{currentstroke}{rgb}{0.000000,0.000000,0.000000}%
\pgfsetstrokecolor{currentstroke}%
\pgfsetdash{}{0pt}%
\pgfpathmoveto{\pgfqpoint{2.078481in}{1.875947in}}%
\pgfpathlineto{\pgfqpoint{2.094987in}{1.895171in}}%
\pgfpathlineto{\pgfqpoint{2.062554in}{1.935832in}}%
\pgfpathlineto{\pgfqpoint{2.047081in}{1.915487in}}%
\pgfpathlineto{\pgfqpoint{2.078481in}{1.875947in}}%
\pgfpathclose%
\pgfusepath{fill}%
\end{pgfscope}%
\begin{pgfscope}%
\pgfpathrectangle{\pgfqpoint{0.000000in}{0.000000in}}{\pgfqpoint{3.000000in}{3.000000in}}%
\pgfusepath{clip}%
\pgfsetbuttcap%
\pgfsetroundjoin%
\definecolor{currentfill}{rgb}{0.199241,1.000000,0.768501}%
\pgfsetfillcolor{currentfill}%
\pgfsetlinewidth{0.000000pt}%
\definecolor{currentstroke}{rgb}{0.000000,0.000000,0.000000}%
\pgfsetstrokecolor{currentstroke}%
\pgfsetdash{}{0pt}%
\pgfpathmoveto{\pgfqpoint{1.837282in}{1.424242in}}%
\pgfpathlineto{\pgfqpoint{1.854191in}{1.455051in}}%
\pgfpathlineto{\pgfqpoint{1.859908in}{1.480155in}}%
\pgfpathlineto{\pgfqpoint{1.842734in}{1.448073in}}%
\pgfpathlineto{\pgfqpoint{1.837282in}{1.424242in}}%
\pgfpathclose%
\pgfusepath{fill}%
\end{pgfscope}%
\begin{pgfscope}%
\pgfpathrectangle{\pgfqpoint{0.000000in}{0.000000in}}{\pgfqpoint{3.000000in}{3.000000in}}%
\pgfusepath{clip}%
\pgfsetbuttcap%
\pgfsetroundjoin%
\definecolor{currentfill}{rgb}{0.667299,1.000000,0.300443}%
\pgfsetfillcolor{currentfill}%
\pgfsetlinewidth{0.000000pt}%
\definecolor{currentstroke}{rgb}{0.000000,0.000000,0.000000}%
\pgfsetstrokecolor{currentstroke}%
\pgfsetdash{}{0pt}%
\pgfpathmoveto{\pgfqpoint{1.162331in}{1.640037in}}%
\pgfpathlineto{\pgfqpoint{1.145621in}{1.666059in}}%
\pgfpathlineto{\pgfqpoint{1.135994in}{1.634969in}}%
\pgfpathlineto{\pgfqpoint{1.153167in}{1.610183in}}%
\pgfpathlineto{\pgfqpoint{1.162331in}{1.640037in}}%
\pgfpathclose%
\pgfusepath{fill}%
\end{pgfscope}%
\begin{pgfscope}%
\pgfpathrectangle{\pgfqpoint{0.000000in}{0.000000in}}{\pgfqpoint{3.000000in}{3.000000in}}%
\pgfusepath{clip}%
\pgfsetbuttcap%
\pgfsetroundjoin%
\definecolor{currentfill}{rgb}{0.000000,0.300000,1.000000}%
\pgfsetfillcolor{currentfill}%
\pgfsetlinewidth{0.000000pt}%
\definecolor{currentstroke}{rgb}{0.000000,0.000000,0.000000}%
\pgfsetstrokecolor{currentstroke}%
\pgfsetdash{}{0pt}%
\pgfpathmoveto{\pgfqpoint{1.659732in}{1.150359in}}%
\pgfpathlineto{\pgfqpoint{1.669235in}{1.194336in}}%
\pgfpathlineto{\pgfqpoint{1.698461in}{1.206025in}}%
\pgfpathlineto{\pgfqpoint{1.686812in}{1.161235in}}%
\pgfpathlineto{\pgfqpoint{1.659732in}{1.150359in}}%
\pgfpathclose%
\pgfusepath{fill}%
\end{pgfscope}%
\begin{pgfscope}%
\pgfpathrectangle{\pgfqpoint{0.000000in}{0.000000in}}{\pgfqpoint{3.000000in}{3.000000in}}%
\pgfusepath{clip}%
\pgfsetbuttcap%
\pgfsetroundjoin%
\definecolor{currentfill}{rgb}{1.000000,0.480029,0.000000}%
\pgfsetfillcolor{currentfill}%
\pgfsetlinewidth{0.000000pt}%
\definecolor{currentstroke}{rgb}{0.000000,0.000000,0.000000}%
\pgfsetstrokecolor{currentstroke}%
\pgfsetdash{}{0pt}%
\pgfpathmoveto{\pgfqpoint{2.094987in}{1.895171in}}%
\pgfpathlineto{\pgfqpoint{2.111496in}{1.913969in}}%
\pgfpathlineto{\pgfqpoint{2.078025in}{1.955747in}}%
\pgfpathlineto{\pgfqpoint{2.062554in}{1.935832in}}%
\pgfpathlineto{\pgfqpoint{2.094987in}{1.895171in}}%
\pgfpathclose%
\pgfusepath{fill}%
\end{pgfscope}%
\begin{pgfscope}%
\pgfpathrectangle{\pgfqpoint{0.000000in}{0.000000in}}{\pgfqpoint{3.000000in}{3.000000in}}%
\pgfusepath{clip}%
\pgfsetbuttcap%
\pgfsetroundjoin%
\definecolor{currentfill}{rgb}{0.000000,0.503922,1.000000}%
\pgfsetfillcolor{currentfill}%
\pgfsetlinewidth{0.000000pt}%
\definecolor{currentstroke}{rgb}{0.000000,0.000000,0.000000}%
\pgfsetstrokecolor{currentstroke}%
\pgfsetdash{}{0pt}%
\pgfpathmoveto{\pgfqpoint{1.343101in}{1.230226in}}%
\pgfpathlineto{\pgfqpoint{1.328545in}{1.271423in}}%
\pgfpathlineto{\pgfqpoint{1.352375in}{1.255209in}}%
\pgfpathlineto{\pgfqpoint{1.365315in}{1.215063in}}%
\pgfpathlineto{\pgfqpoint{1.343101in}{1.230226in}}%
\pgfpathclose%
\pgfusepath{fill}%
\end{pgfscope}%
\begin{pgfscope}%
\pgfpathrectangle{\pgfqpoint{0.000000in}{0.000000in}}{\pgfqpoint{3.000000in}{3.000000in}}%
\pgfusepath{clip}%
\pgfsetbuttcap%
\pgfsetroundjoin%
\definecolor{currentfill}{rgb}{0.000000,0.849020,1.000000}%
\pgfsetfillcolor{currentfill}%
\pgfsetlinewidth{0.000000pt}%
\definecolor{currentstroke}{rgb}{0.000000,0.000000,0.000000}%
\pgfsetstrokecolor{currentstroke}%
\pgfsetdash{}{0pt}%
\pgfpathmoveto{\pgfqpoint{1.280654in}{1.347754in}}%
\pgfpathlineto{\pgfqpoint{1.263970in}{1.383552in}}%
\pgfpathlineto{\pgfqpoint{1.278375in}{1.361972in}}%
\pgfpathlineto{\pgfqpoint{1.294223in}{1.327402in}}%
\pgfpathlineto{\pgfqpoint{1.280654in}{1.347754in}}%
\pgfpathclose%
\pgfusepath{fill}%
\end{pgfscope}%
\begin{pgfscope}%
\pgfpathrectangle{\pgfqpoint{0.000000in}{0.000000in}}{\pgfqpoint{3.000000in}{3.000000in}}%
\pgfusepath{clip}%
\pgfsetbuttcap%
\pgfsetroundjoin%
\definecolor{currentfill}{rgb}{1.000000,0.741467,0.000000}%
\pgfsetfillcolor{currentfill}%
\pgfsetlinewidth{0.000000pt}%
\definecolor{currentstroke}{rgb}{0.000000,0.000000,0.000000}%
\pgfsetstrokecolor{currentstroke}%
\pgfsetdash{}{0pt}%
\pgfpathmoveto{\pgfqpoint{1.069718in}{1.839710in}}%
\pgfpathlineto{\pgfqpoint{1.053862in}{1.861204in}}%
\pgfpathlineto{\pgfqpoint{1.028484in}{1.823171in}}%
\pgfpathlineto{\pgfqpoint{1.045235in}{1.802834in}}%
\pgfpathlineto{\pgfqpoint{1.069718in}{1.839710in}}%
\pgfpathclose%
\pgfusepath{fill}%
\end{pgfscope}%
\begin{pgfscope}%
\pgfpathrectangle{\pgfqpoint{0.000000in}{0.000000in}}{\pgfqpoint{3.000000in}{3.000000in}}%
\pgfusepath{clip}%
\pgfsetbuttcap%
\pgfsetroundjoin%
\definecolor{currentfill}{rgb}{0.819102,1.000000,0.148640}%
\pgfsetfillcolor{currentfill}%
\pgfsetlinewidth{0.000000pt}%
\definecolor{currentstroke}{rgb}{0.000000,0.000000,0.000000}%
\pgfsetstrokecolor{currentstroke}%
\pgfsetdash{}{0pt}%
\pgfpathmoveto{\pgfqpoint{1.960042in}{1.669536in}}%
\pgfpathlineto{\pgfqpoint{1.977126in}{1.692677in}}%
\pgfpathlineto{\pgfqpoint{1.963044in}{1.725873in}}%
\pgfpathlineto{\pgfqpoint{1.946569in}{1.701517in}}%
\pgfpathlineto{\pgfqpoint{1.960042in}{1.669536in}}%
\pgfpathclose%
\pgfusepath{fill}%
\end{pgfscope}%
\begin{pgfscope}%
\pgfpathrectangle{\pgfqpoint{0.000000in}{0.000000in}}{\pgfqpoint{3.000000in}{3.000000in}}%
\pgfusepath{clip}%
\pgfsetbuttcap%
\pgfsetroundjoin%
\definecolor{currentfill}{rgb}{1.000000,0.407407,0.000000}%
\pgfsetfillcolor{currentfill}%
\pgfsetlinewidth{0.000000pt}%
\definecolor{currentstroke}{rgb}{0.000000,0.000000,0.000000}%
\pgfsetstrokecolor{currentstroke}%
\pgfsetdash{}{0pt}%
\pgfpathmoveto{\pgfqpoint{2.111496in}{1.913969in}}%
\pgfpathlineto{\pgfqpoint{2.128008in}{1.932370in}}%
\pgfpathlineto{\pgfqpoint{2.093495in}{1.975263in}}%
\pgfpathlineto{\pgfqpoint{2.078025in}{1.955747in}}%
\pgfpathlineto{\pgfqpoint{2.111496in}{1.913969in}}%
\pgfpathclose%
\pgfusepath{fill}%
\end{pgfscope}%
\begin{pgfscope}%
\pgfpathrectangle{\pgfqpoint{0.000000in}{0.000000in}}{\pgfqpoint{3.000000in}{3.000000in}}%
\pgfusepath{clip}%
\pgfsetbuttcap%
\pgfsetroundjoin%
\definecolor{currentfill}{rgb}{0.401645,1.000000,0.566097}%
\pgfsetfillcolor{currentfill}%
\pgfsetlinewidth{0.000000pt}%
\definecolor{currentstroke}{rgb}{0.000000,0.000000,0.000000}%
\pgfsetstrokecolor{currentstroke}%
\pgfsetdash{}{0pt}%
\pgfpathmoveto{\pgfqpoint{1.204617in}{1.527783in}}%
\pgfpathlineto{\pgfqpoint{1.187478in}{1.556770in}}%
\pgfpathlineto{\pgfqpoint{1.187886in}{1.529007in}}%
\pgfpathlineto{\pgfqpoint{1.205053in}{1.501289in}}%
\pgfpathlineto{\pgfqpoint{1.204617in}{1.527783in}}%
\pgfpathclose%
\pgfusepath{fill}%
\end{pgfscope}%
\begin{pgfscope}%
\pgfpathrectangle{\pgfqpoint{0.000000in}{0.000000in}}{\pgfqpoint{3.000000in}{3.000000in}}%
\pgfusepath{clip}%
\pgfsetbuttcap%
\pgfsetroundjoin%
\definecolor{currentfill}{rgb}{0.490196,1.000000,0.477546}%
\pgfsetfillcolor{currentfill}%
\pgfsetlinewidth{0.000000pt}%
\definecolor{currentstroke}{rgb}{0.000000,0.000000,0.000000}%
\pgfsetstrokecolor{currentstroke}%
\pgfsetdash{}{0pt}%
\pgfpathmoveto{\pgfqpoint{1.894302in}{1.538263in}}%
\pgfpathlineto{\pgfqpoint{1.911521in}{1.564820in}}%
\pgfpathlineto{\pgfqpoint{1.908849in}{1.593791in}}%
\pgfpathlineto{\pgfqpoint{1.891804in}{1.565972in}}%
\pgfpathlineto{\pgfqpoint{1.894302in}{1.538263in}}%
\pgfpathclose%
\pgfusepath{fill}%
\end{pgfscope}%
\begin{pgfscope}%
\pgfpathrectangle{\pgfqpoint{0.000000in}{0.000000in}}{\pgfqpoint{3.000000in}{3.000000in}}%
\pgfusepath{clip}%
\pgfsetbuttcap%
\pgfsetroundjoin%
\definecolor{currentfill}{rgb}{0.000000,0.676471,1.000000}%
\pgfsetfillcolor{currentfill}%
\pgfsetlinewidth{0.000000pt}%
\definecolor{currentstroke}{rgb}{0.000000,0.000000,0.000000}%
\pgfsetstrokecolor{currentstroke}%
\pgfsetdash{}{0pt}%
\pgfpathmoveto{\pgfqpoint{1.310047in}{1.289320in}}%
\pgfpathlineto{\pgfqpoint{1.294223in}{1.327402in}}%
\pgfpathlineto{\pgfqpoint{1.313962in}{1.308350in}}%
\pgfpathlineto{\pgfqpoint{1.328545in}{1.271423in}}%
\pgfpathlineto{\pgfqpoint{1.310047in}{1.289320in}}%
\pgfpathclose%
\pgfusepath{fill}%
\end{pgfscope}%
\begin{pgfscope}%
\pgfpathrectangle{\pgfqpoint{0.000000in}{0.000000in}}{\pgfqpoint{3.000000in}{3.000000in}}%
\pgfusepath{clip}%
\pgfsetbuttcap%
\pgfsetroundjoin%
\definecolor{currentfill}{rgb}{0.000000,0.300000,1.000000}%
\pgfsetfillcolor{currentfill}%
\pgfsetlinewidth{0.000000pt}%
\definecolor{currentstroke}{rgb}{0.000000,0.000000,0.000000}%
\pgfsetstrokecolor{currentstroke}%
\pgfsetdash{}{0pt}%
\pgfpathmoveto{\pgfqpoint{1.402916in}{1.157367in}}%
\pgfpathlineto{\pgfqpoint{1.391952in}{1.201868in}}%
\pgfpathlineto{\pgfqpoint{1.422353in}{1.190983in}}%
\pgfpathlineto{\pgfqpoint{1.431084in}{1.147240in}}%
\pgfpathlineto{\pgfqpoint{1.402916in}{1.157367in}}%
\pgfpathclose%
\pgfusepath{fill}%
\end{pgfscope}%
\begin{pgfscope}%
\pgfpathrectangle{\pgfqpoint{0.000000in}{0.000000in}}{\pgfqpoint{3.000000in}{3.000000in}}%
\pgfusepath{clip}%
\pgfsetbuttcap%
\pgfsetroundjoin%
\definecolor{currentfill}{rgb}{1.000000,0.668845,0.000000}%
\pgfsetfillcolor{currentfill}%
\pgfsetlinewidth{0.000000pt}%
\definecolor{currentstroke}{rgb}{0.000000,0.000000,0.000000}%
\pgfsetstrokecolor{currentstroke}%
\pgfsetdash{}{0pt}%
\pgfpathmoveto{\pgfqpoint{1.053862in}{1.861204in}}%
\pgfpathlineto{\pgfqpoint{1.038005in}{1.882151in}}%
\pgfpathlineto{\pgfqpoint{1.011728in}{1.842965in}}%
\pgfpathlineto{\pgfqpoint{1.028484in}{1.823171in}}%
\pgfpathlineto{\pgfqpoint{1.053862in}{1.861204in}}%
\pgfpathclose%
\pgfusepath{fill}%
\end{pgfscope}%
\begin{pgfscope}%
\pgfpathrectangle{\pgfqpoint{0.000000in}{0.000000in}}{\pgfqpoint{3.000000in}{3.000000in}}%
\pgfusepath{clip}%
\pgfsetbuttcap%
\pgfsetroundjoin%
\definecolor{currentfill}{rgb}{1.000000,0.349310,0.000000}%
\pgfsetfillcolor{currentfill}%
\pgfsetlinewidth{0.000000pt}%
\definecolor{currentstroke}{rgb}{0.000000,0.000000,0.000000}%
\pgfsetstrokecolor{currentstroke}%
\pgfsetdash{}{0pt}%
\pgfpathmoveto{\pgfqpoint{2.128008in}{1.932370in}}%
\pgfpathlineto{\pgfqpoint{2.144524in}{1.950404in}}%
\pgfpathlineto{\pgfqpoint{2.108963in}{1.994407in}}%
\pgfpathlineto{\pgfqpoint{2.093495in}{1.975263in}}%
\pgfpathlineto{\pgfqpoint{2.128008in}{1.932370in}}%
\pgfpathclose%
\pgfusepath{fill}%
\end{pgfscope}%
\begin{pgfscope}%
\pgfpathrectangle{\pgfqpoint{0.000000in}{0.000000in}}{\pgfqpoint{3.000000in}{3.000000in}}%
\pgfusepath{clip}%
\pgfsetbuttcap%
\pgfsetroundjoin%
\definecolor{currentfill}{rgb}{0.743201,1.000000,0.224541}%
\pgfsetfillcolor{currentfill}%
\pgfsetlinewidth{0.000000pt}%
\definecolor{currentstroke}{rgb}{0.000000,0.000000,0.000000}%
\pgfsetstrokecolor{currentstroke}%
\pgfsetdash{}{0pt}%
\pgfpathmoveto{\pgfqpoint{1.145621in}{1.666059in}}%
\pgfpathlineto{\pgfqpoint{1.128904in}{1.690990in}}%
\pgfpathlineto{\pgfqpoint{1.118810in}{1.658667in}}%
\pgfpathlineto{\pgfqpoint{1.135994in}{1.634969in}}%
\pgfpathlineto{\pgfqpoint{1.145621in}{1.666059in}}%
\pgfpathclose%
\pgfusepath{fill}%
\end{pgfscope}%
\begin{pgfscope}%
\pgfpathrectangle{\pgfqpoint{0.000000in}{0.000000in}}{\pgfqpoint{3.000000in}{3.000000in}}%
\pgfusepath{clip}%
\pgfsetbuttcap%
\pgfsetroundjoin%
\definecolor{currentfill}{rgb}{0.199241,1.000000,0.768501}%
\pgfsetfillcolor{currentfill}%
\pgfsetlinewidth{0.000000pt}%
\definecolor{currentstroke}{rgb}{0.000000,0.000000,0.000000}%
\pgfsetstrokecolor{currentstroke}%
\pgfsetdash{}{0pt}%
\pgfpathmoveto{\pgfqpoint{1.239336in}{1.440092in}}%
\pgfpathlineto{\pgfqpoint{1.222203in}{1.471748in}}%
\pgfpathlineto{\pgfqpoint{1.230539in}{1.446807in}}%
\pgfpathlineto{\pgfqpoint{1.247265in}{1.416417in}}%
\pgfpathlineto{\pgfqpoint{1.239336in}{1.440092in}}%
\pgfpathclose%
\pgfusepath{fill}%
\end{pgfscope}%
\begin{pgfscope}%
\pgfpathrectangle{\pgfqpoint{0.000000in}{0.000000in}}{\pgfqpoint{3.000000in}{3.000000in}}%
\pgfusepath{clip}%
\pgfsetbuttcap%
\pgfsetroundjoin%
\definecolor{currentfill}{rgb}{0.895003,1.000000,0.072739}%
\pgfsetfillcolor{currentfill}%
\pgfsetlinewidth{0.000000pt}%
\definecolor{currentstroke}{rgb}{0.000000,0.000000,0.000000}%
\pgfsetstrokecolor{currentstroke}%
\pgfsetdash{}{0pt}%
\pgfpathmoveto{\pgfqpoint{1.977126in}{1.692677in}}%
\pgfpathlineto{\pgfqpoint{1.994218in}{1.714946in}}%
\pgfpathlineto{\pgfqpoint{1.979523in}{1.749353in}}%
\pgfpathlineto{\pgfqpoint{1.963044in}{1.725873in}}%
\pgfpathlineto{\pgfqpoint{1.977126in}{1.692677in}}%
\pgfpathclose%
\pgfusepath{fill}%
\end{pgfscope}%
\begin{pgfscope}%
\pgfpathrectangle{\pgfqpoint{0.000000in}{0.000000in}}{\pgfqpoint{3.000000in}{3.000000in}}%
\pgfusepath{clip}%
\pgfsetbuttcap%
\pgfsetroundjoin%
\definecolor{currentfill}{rgb}{1.000000,0.291213,0.000000}%
\pgfsetfillcolor{currentfill}%
\pgfsetlinewidth{0.000000pt}%
\definecolor{currentstroke}{rgb}{0.000000,0.000000,0.000000}%
\pgfsetstrokecolor{currentstroke}%
\pgfsetdash{}{0pt}%
\pgfpathmoveto{\pgfqpoint{2.144524in}{1.950404in}}%
\pgfpathlineto{\pgfqpoint{2.161042in}{1.968094in}}%
\pgfpathlineto{\pgfqpoint{2.124431in}{2.013206in}}%
\pgfpathlineto{\pgfqpoint{2.108963in}{1.994407in}}%
\pgfpathlineto{\pgfqpoint{2.144524in}{1.950404in}}%
\pgfpathclose%
\pgfusepath{fill}%
\end{pgfscope}%
\begin{pgfscope}%
\pgfpathrectangle{\pgfqpoint{0.000000in}{0.000000in}}{\pgfqpoint{3.000000in}{3.000000in}}%
\pgfusepath{clip}%
\pgfsetbuttcap%
\pgfsetroundjoin%
\definecolor{currentfill}{rgb}{1.000000,0.610748,0.000000}%
\pgfsetfillcolor{currentfill}%
\pgfsetlinewidth{0.000000pt}%
\definecolor{currentstroke}{rgb}{0.000000,0.000000,0.000000}%
\pgfsetstrokecolor{currentstroke}%
\pgfsetdash{}{0pt}%
\pgfpathmoveto{\pgfqpoint{1.038005in}{1.882151in}}%
\pgfpathlineto{\pgfqpoint{1.022148in}{1.902596in}}%
\pgfpathlineto{\pgfqpoint{0.994967in}{1.862259in}}%
\pgfpathlineto{\pgfqpoint{1.011728in}{1.842965in}}%
\pgfpathlineto{\pgfqpoint{1.038005in}{1.882151in}}%
\pgfpathclose%
\pgfusepath{fill}%
\end{pgfscope}%
\begin{pgfscope}%
\pgfpathrectangle{\pgfqpoint{0.000000in}{0.000000in}}{\pgfqpoint{3.000000in}{3.000000in}}%
\pgfusepath{clip}%
\pgfsetbuttcap%
\pgfsetroundjoin%
\definecolor{currentfill}{rgb}{0.085389,1.000000,0.882353}%
\pgfsetfillcolor{currentfill}%
\pgfsetlinewidth{0.000000pt}%
\definecolor{currentstroke}{rgb}{0.000000,0.000000,0.000000}%
\pgfsetstrokecolor{currentstroke}%
\pgfsetdash{}{0pt}%
\pgfpathmoveto{\pgfqpoint{1.808257in}{1.369039in}}%
\pgfpathlineto{\pgfqpoint{1.824454in}{1.401080in}}%
\pgfpathlineto{\pgfqpoint{1.837282in}{1.424242in}}%
\pgfpathlineto{\pgfqpoint{1.820393in}{1.390956in}}%
\pgfpathlineto{\pgfqpoint{1.808257in}{1.369039in}}%
\pgfpathclose%
\pgfusepath{fill}%
\end{pgfscope}%
\begin{pgfscope}%
\pgfpathrectangle{\pgfqpoint{0.000000in}{0.000000in}}{\pgfqpoint{3.000000in}{3.000000in}}%
\pgfusepath{clip}%
\pgfsetbuttcap%
\pgfsetroundjoin%
\definecolor{currentfill}{rgb}{0.000000,0.300000,1.000000}%
\pgfsetfillcolor{currentfill}%
\pgfsetlinewidth{0.000000pt}%
\definecolor{currentstroke}{rgb}{0.000000,0.000000,0.000000}%
\pgfsetstrokecolor{currentstroke}%
\pgfsetdash{}{0pt}%
\pgfpathmoveto{\pgfqpoint{1.629630in}{1.141817in}}%
\pgfpathlineto{\pgfqpoint{1.636741in}{1.185154in}}%
\pgfpathlineto{\pgfqpoint{1.669235in}{1.194336in}}%
\pgfpathlineto{\pgfqpoint{1.659732in}{1.150359in}}%
\pgfpathlineto{\pgfqpoint{1.629630in}{1.141817in}}%
\pgfpathclose%
\pgfusepath{fill}%
\end{pgfscope}%
\begin{pgfscope}%
\pgfpathrectangle{\pgfqpoint{0.000000in}{0.000000in}}{\pgfqpoint{3.000000in}{3.000000in}}%
\pgfusepath{clip}%
\pgfsetbuttcap%
\pgfsetroundjoin%
\definecolor{currentfill}{rgb}{1.000000,0.233115,0.000000}%
\pgfsetfillcolor{currentfill}%
\pgfsetlinewidth{0.000000pt}%
\definecolor{currentstroke}{rgb}{0.000000,0.000000,0.000000}%
\pgfsetstrokecolor{currentstroke}%
\pgfsetdash{}{0pt}%
\pgfpathmoveto{\pgfqpoint{2.161042in}{1.968094in}}%
\pgfpathlineto{\pgfqpoint{2.177564in}{1.985465in}}%
\pgfpathlineto{\pgfqpoint{2.139897in}{2.031682in}}%
\pgfpathlineto{\pgfqpoint{2.124431in}{2.013206in}}%
\pgfpathlineto{\pgfqpoint{2.161042in}{1.968094in}}%
\pgfpathclose%
\pgfusepath{fill}%
\end{pgfscope}%
\begin{pgfscope}%
\pgfpathrectangle{\pgfqpoint{0.000000in}{0.000000in}}{\pgfqpoint{3.000000in}{3.000000in}}%
\pgfusepath{clip}%
\pgfsetbuttcap%
\pgfsetroundjoin%
\definecolor{currentfill}{rgb}{0.300443,1.000000,0.667299}%
\pgfsetfillcolor{currentfill}%
\pgfsetlinewidth{0.000000pt}%
\definecolor{currentstroke}{rgb}{0.000000,0.000000,0.000000}%
\pgfsetstrokecolor{currentstroke}%
\pgfsetdash{}{0pt}%
\pgfpathmoveto{\pgfqpoint{1.854191in}{1.455051in}}%
\pgfpathlineto{\pgfqpoint{1.871120in}{1.483745in}}%
\pgfpathlineto{\pgfqpoint{1.877097in}{1.510120in}}%
\pgfpathlineto{\pgfqpoint{1.859908in}{1.480155in}}%
\pgfpathlineto{\pgfqpoint{1.854191in}{1.455051in}}%
\pgfpathclose%
\pgfusepath{fill}%
\end{pgfscope}%
\begin{pgfscope}%
\pgfpathrectangle{\pgfqpoint{0.000000in}{0.000000in}}{\pgfqpoint{3.000000in}{3.000000in}}%
\pgfusepath{clip}%
\pgfsetbuttcap%
\pgfsetroundjoin%
\definecolor{currentfill}{rgb}{0.000000,0.503922,1.000000}%
\pgfsetfillcolor{currentfill}%
\pgfsetlinewidth{0.000000pt}%
\definecolor{currentstroke}{rgb}{0.000000,0.000000,0.000000}%
\pgfsetstrokecolor{currentstroke}%
\pgfsetdash{}{0pt}%
\pgfpathmoveto{\pgfqpoint{1.698461in}{1.206025in}}%
\pgfpathlineto{\pgfqpoint{1.710135in}{1.245544in}}%
\pgfpathlineto{\pgfqpoint{1.737209in}{1.260400in}}%
\pgfpathlineto{\pgfqpoint{1.723691in}{1.219917in}}%
\pgfpathlineto{\pgfqpoint{1.698461in}{1.206025in}}%
\pgfpathclose%
\pgfusepath{fill}%
\end{pgfscope}%
\begin{pgfscope}%
\pgfpathrectangle{\pgfqpoint{0.000000in}{0.000000in}}{\pgfqpoint{3.000000in}{3.000000in}}%
\pgfusepath{clip}%
\pgfsetbuttcap%
\pgfsetroundjoin%
\definecolor{currentfill}{rgb}{1.000000,0.538126,0.000000}%
\pgfsetfillcolor{currentfill}%
\pgfsetlinewidth{0.000000pt}%
\definecolor{currentstroke}{rgb}{0.000000,0.000000,0.000000}%
\pgfsetstrokecolor{currentstroke}%
\pgfsetdash{}{0pt}%
\pgfpathmoveto{\pgfqpoint{1.022148in}{1.902596in}}%
\pgfpathlineto{\pgfqpoint{1.006291in}{1.922576in}}%
\pgfpathlineto{\pgfqpoint{0.978201in}{1.881092in}}%
\pgfpathlineto{\pgfqpoint{0.994967in}{1.862259in}}%
\pgfpathlineto{\pgfqpoint{1.022148in}{1.902596in}}%
\pgfpathclose%
\pgfusepath{fill}%
\end{pgfscope}%
\begin{pgfscope}%
\pgfpathrectangle{\pgfqpoint{0.000000in}{0.000000in}}{\pgfqpoint{3.000000in}{3.000000in}}%
\pgfusepath{clip}%
\pgfsetbuttcap%
\pgfsetroundjoin%
\definecolor{currentfill}{rgb}{0.578748,1.000000,0.388994}%
\pgfsetfillcolor{currentfill}%
\pgfsetlinewidth{0.000000pt}%
\definecolor{currentstroke}{rgb}{0.000000,0.000000,0.000000}%
\pgfsetstrokecolor{currentstroke}%
\pgfsetdash{}{0pt}%
\pgfpathmoveto{\pgfqpoint{1.911521in}{1.564820in}}%
\pgfpathlineto{\pgfqpoint{1.928756in}{1.589989in}}%
\pgfpathlineto{\pgfqpoint{1.925904in}{1.620220in}}%
\pgfpathlineto{\pgfqpoint{1.908849in}{1.593791in}}%
\pgfpathlineto{\pgfqpoint{1.911521in}{1.564820in}}%
\pgfpathclose%
\pgfusepath{fill}%
\end{pgfscope}%
\begin{pgfscope}%
\pgfpathrectangle{\pgfqpoint{0.000000in}{0.000000in}}{\pgfqpoint{3.000000in}{3.000000in}}%
\pgfusepath{clip}%
\pgfsetbuttcap%
\pgfsetroundjoin%
\definecolor{currentfill}{rgb}{1.000000,0.175018,0.000000}%
\pgfsetfillcolor{currentfill}%
\pgfsetlinewidth{0.000000pt}%
\definecolor{currentstroke}{rgb}{0.000000,0.000000,0.000000}%
\pgfsetstrokecolor{currentstroke}%
\pgfsetdash{}{0pt}%
\pgfpathmoveto{\pgfqpoint{2.177564in}{1.985465in}}%
\pgfpathlineto{\pgfqpoint{2.194088in}{2.002536in}}%
\pgfpathlineto{\pgfqpoint{2.155361in}{2.049855in}}%
\pgfpathlineto{\pgfqpoint{2.139897in}{2.031682in}}%
\pgfpathlineto{\pgfqpoint{2.177564in}{1.985465in}}%
\pgfpathclose%
\pgfusepath{fill}%
\end{pgfscope}%
\begin{pgfscope}%
\pgfpathrectangle{\pgfqpoint{0.000000in}{0.000000in}}{\pgfqpoint{3.000000in}{3.000000in}}%
\pgfusepath{clip}%
\pgfsetbuttcap%
\pgfsetroundjoin%
\definecolor{currentfill}{rgb}{0.819102,1.000000,0.148640}%
\pgfsetfillcolor{currentfill}%
\pgfsetlinewidth{0.000000pt}%
\definecolor{currentstroke}{rgb}{0.000000,0.000000,0.000000}%
\pgfsetstrokecolor{currentstroke}%
\pgfsetdash{}{0pt}%
\pgfpathmoveto{\pgfqpoint{1.128904in}{1.690990in}}%
\pgfpathlineto{\pgfqpoint{1.112182in}{1.714947in}}%
\pgfpathlineto{\pgfqpoint{1.101615in}{1.681394in}}%
\pgfpathlineto{\pgfqpoint{1.118810in}{1.658667in}}%
\pgfpathlineto{\pgfqpoint{1.128904in}{1.690990in}}%
\pgfpathclose%
\pgfusepath{fill}%
\end{pgfscope}%
\begin{pgfscope}%
\pgfpathrectangle{\pgfqpoint{0.000000in}{0.000000in}}{\pgfqpoint{3.000000in}{3.000000in}}%
\pgfusepath{clip}%
\pgfsetbuttcap%
\pgfsetroundjoin%
\definecolor{currentfill}{rgb}{0.958254,0.973856,0.009488}%
\pgfsetfillcolor{currentfill}%
\pgfsetlinewidth{0.000000pt}%
\definecolor{currentstroke}{rgb}{0.000000,0.000000,0.000000}%
\pgfsetstrokecolor{currentstroke}%
\pgfsetdash{}{0pt}%
\pgfpathmoveto{\pgfqpoint{1.994218in}{1.714946in}}%
\pgfpathlineto{\pgfqpoint{2.011320in}{1.736429in}}%
\pgfpathlineto{\pgfqpoint{1.996007in}{1.772044in}}%
\pgfpathlineto{\pgfqpoint{1.979523in}{1.749353in}}%
\pgfpathlineto{\pgfqpoint{1.994218in}{1.714946in}}%
\pgfpathclose%
\pgfusepath{fill}%
\end{pgfscope}%
\begin{pgfscope}%
\pgfpathrectangle{\pgfqpoint{0.000000in}{0.000000in}}{\pgfqpoint{3.000000in}{3.000000in}}%
\pgfusepath{clip}%
\pgfsetbuttcap%
\pgfsetroundjoin%
\definecolor{currentfill}{rgb}{0.000000,0.300000,1.000000}%
\pgfsetfillcolor{currentfill}%
\pgfsetlinewidth{0.000000pt}%
\definecolor{currentstroke}{rgb}{0.000000,0.000000,0.000000}%
\pgfsetstrokecolor{currentstroke}%
\pgfsetdash{}{0pt}%
\pgfpathmoveto{\pgfqpoint{1.431084in}{1.147240in}}%
\pgfpathlineto{\pgfqpoint{1.422353in}{1.190983in}}%
\pgfpathlineto{\pgfqpoint{1.455759in}{1.182694in}}%
\pgfpathlineto{\pgfqpoint{1.462027in}{1.139529in}}%
\pgfpathlineto{\pgfqpoint{1.431084in}{1.147240in}}%
\pgfpathclose%
\pgfusepath{fill}%
\end{pgfscope}%
\begin{pgfscope}%
\pgfpathrectangle{\pgfqpoint{0.000000in}{0.000000in}}{\pgfqpoint{3.000000in}{3.000000in}}%
\pgfusepath{clip}%
\pgfsetbuttcap%
\pgfsetroundjoin%
\definecolor{currentfill}{rgb}{0.490196,1.000000,0.477546}%
\pgfsetfillcolor{currentfill}%
\pgfsetlinewidth{0.000000pt}%
\definecolor{currentstroke}{rgb}{0.000000,0.000000,0.000000}%
\pgfsetstrokecolor{currentstroke}%
\pgfsetdash{}{0pt}%
\pgfpathmoveto{\pgfqpoint{1.187478in}{1.556770in}}%
\pgfpathlineto{\pgfqpoint{1.170329in}{1.584172in}}%
\pgfpathlineto{\pgfqpoint{1.170703in}{1.555141in}}%
\pgfpathlineto{\pgfqpoint{1.187886in}{1.529007in}}%
\pgfpathlineto{\pgfqpoint{1.187478in}{1.556770in}}%
\pgfpathclose%
\pgfusepath{fill}%
\end{pgfscope}%
\begin{pgfscope}%
\pgfpathrectangle{\pgfqpoint{0.000000in}{0.000000in}}{\pgfqpoint{3.000000in}{3.000000in}}%
\pgfusepath{clip}%
\pgfsetbuttcap%
\pgfsetroundjoin%
\definecolor{currentfill}{rgb}{1.000000,0.116921,0.000000}%
\pgfsetfillcolor{currentfill}%
\pgfsetlinewidth{0.000000pt}%
\definecolor{currentstroke}{rgb}{0.000000,0.000000,0.000000}%
\pgfsetstrokecolor{currentstroke}%
\pgfsetdash{}{0pt}%
\pgfpathmoveto{\pgfqpoint{2.194088in}{2.002536in}}%
\pgfpathlineto{\pgfqpoint{2.210615in}{2.019327in}}%
\pgfpathlineto{\pgfqpoint{2.170824in}{2.067744in}}%
\pgfpathlineto{\pgfqpoint{2.155361in}{2.049855in}}%
\pgfpathlineto{\pgfqpoint{2.194088in}{2.002536in}}%
\pgfpathclose%
\pgfusepath{fill}%
\end{pgfscope}%
\begin{pgfscope}%
\pgfpathrectangle{\pgfqpoint{0.000000in}{0.000000in}}{\pgfqpoint{3.000000in}{3.000000in}}%
\pgfusepath{clip}%
\pgfsetbuttcap%
\pgfsetroundjoin%
\definecolor{currentfill}{rgb}{0.000000,0.849020,1.000000}%
\pgfsetfillcolor{currentfill}%
\pgfsetlinewidth{0.000000pt}%
\definecolor{currentstroke}{rgb}{0.000000,0.000000,0.000000}%
\pgfsetstrokecolor{currentstroke}%
\pgfsetdash{}{0pt}%
\pgfpathmoveto{\pgfqpoint{1.774359in}{1.314526in}}%
\pgfpathlineto{\pgfqpoint{1.789425in}{1.348315in}}%
\pgfpathlineto{\pgfqpoint{1.808257in}{1.369039in}}%
\pgfpathlineto{\pgfqpoint{1.792085in}{1.334067in}}%
\pgfpathlineto{\pgfqpoint{1.774359in}{1.314526in}}%
\pgfpathclose%
\pgfusepath{fill}%
\end{pgfscope}%
\begin{pgfscope}%
\pgfpathrectangle{\pgfqpoint{0.000000in}{0.000000in}}{\pgfqpoint{3.000000in}{3.000000in}}%
\pgfusepath{clip}%
\pgfsetbuttcap%
\pgfsetroundjoin%
\definecolor{currentfill}{rgb}{1.000000,0.480029,0.000000}%
\pgfsetfillcolor{currentfill}%
\pgfsetlinewidth{0.000000pt}%
\definecolor{currentstroke}{rgb}{0.000000,0.000000,0.000000}%
\pgfsetstrokecolor{currentstroke}%
\pgfsetdash{}{0pt}%
\pgfpathmoveto{\pgfqpoint{1.006291in}{1.922576in}}%
\pgfpathlineto{\pgfqpoint{0.990433in}{1.942128in}}%
\pgfpathlineto{\pgfqpoint{0.961430in}{1.899500in}}%
\pgfpathlineto{\pgfqpoint{0.978201in}{1.881092in}}%
\pgfpathlineto{\pgfqpoint{1.006291in}{1.922576in}}%
\pgfpathclose%
\pgfusepath{fill}%
\end{pgfscope}%
\begin{pgfscope}%
\pgfpathrectangle{\pgfqpoint{0.000000in}{0.000000in}}{\pgfqpoint{3.000000in}{3.000000in}}%
\pgfusepath{clip}%
\pgfsetbuttcap%
\pgfsetroundjoin%
\definecolor{currentfill}{rgb}{0.000000,0.503922,1.000000}%
\pgfsetfillcolor{currentfill}%
\pgfsetlinewidth{0.000000pt}%
\definecolor{currentstroke}{rgb}{0.000000,0.000000,0.000000}%
\pgfsetstrokecolor{currentstroke}%
\pgfsetdash{}{0pt}%
\pgfpathmoveto{\pgfqpoint{1.365315in}{1.215063in}}%
\pgfpathlineto{\pgfqpoint{1.352375in}{1.255209in}}%
\pgfpathlineto{\pgfqpoint{1.380962in}{1.241097in}}%
\pgfpathlineto{\pgfqpoint{1.391952in}{1.201868in}}%
\pgfpathlineto{\pgfqpoint{1.365315in}{1.215063in}}%
\pgfpathclose%
\pgfusepath{fill}%
\end{pgfscope}%
\begin{pgfscope}%
\pgfpathrectangle{\pgfqpoint{0.000000in}{0.000000in}}{\pgfqpoint{3.000000in}{3.000000in}}%
\pgfusepath{clip}%
\pgfsetbuttcap%
\pgfsetroundjoin%
\definecolor{currentfill}{rgb}{0.000000,0.676471,1.000000}%
\pgfsetfillcolor{currentfill}%
\pgfsetlinewidth{0.000000pt}%
\definecolor{currentstroke}{rgb}{0.000000,0.000000,0.000000}%
\pgfsetstrokecolor{currentstroke}%
\pgfsetdash{}{0pt}%
\pgfpathmoveto{\pgfqpoint{1.737209in}{1.260400in}}%
\pgfpathlineto{\pgfqpoint{1.750754in}{1.296613in}}%
\pgfpathlineto{\pgfqpoint{1.774359in}{1.314526in}}%
\pgfpathlineto{\pgfqpoint{1.759319in}{1.277223in}}%
\pgfpathlineto{\pgfqpoint{1.737209in}{1.260400in}}%
\pgfpathclose%
\pgfusepath{fill}%
\end{pgfscope}%
\begin{pgfscope}%
\pgfpathrectangle{\pgfqpoint{0.000000in}{0.000000in}}{\pgfqpoint{3.000000in}{3.000000in}}%
\pgfusepath{clip}%
\pgfsetbuttcap%
\pgfsetroundjoin%
\definecolor{currentfill}{rgb}{0.085389,1.000000,0.882353}%
\pgfsetfillcolor{currentfill}%
\pgfsetlinewidth{0.000000pt}%
\definecolor{currentstroke}{rgb}{0.000000,0.000000,0.000000}%
\pgfsetstrokecolor{currentstroke}%
\pgfsetdash{}{0pt}%
\pgfpathmoveto{\pgfqpoint{1.263970in}{1.383552in}}%
\pgfpathlineto{\pgfqpoint{1.247265in}{1.416417in}}%
\pgfpathlineto{\pgfqpoint{1.262502in}{1.393610in}}%
\pgfpathlineto{\pgfqpoint{1.278375in}{1.361972in}}%
\pgfpathlineto{\pgfqpoint{1.263970in}{1.383552in}}%
\pgfpathclose%
\pgfusepath{fill}%
\end{pgfscope}%
\begin{pgfscope}%
\pgfpathrectangle{\pgfqpoint{0.000000in}{0.000000in}}{\pgfqpoint{3.000000in}{3.000000in}}%
\pgfusepath{clip}%
\pgfsetbuttcap%
\pgfsetroundjoin%
\definecolor{currentfill}{rgb}{0.999109,0.073348,0.000000}%
\pgfsetfillcolor{currentfill}%
\pgfsetlinewidth{0.000000pt}%
\definecolor{currentstroke}{rgb}{0.000000,0.000000,0.000000}%
\pgfsetstrokecolor{currentstroke}%
\pgfsetdash{}{0pt}%
\pgfpathmoveto{\pgfqpoint{2.210615in}{2.019327in}}%
\pgfpathlineto{\pgfqpoint{2.227145in}{2.035853in}}%
\pgfpathlineto{\pgfqpoint{2.186286in}{2.085366in}}%
\pgfpathlineto{\pgfqpoint{2.170824in}{2.067744in}}%
\pgfpathlineto{\pgfqpoint{2.210615in}{2.019327in}}%
\pgfpathclose%
\pgfusepath{fill}%
\end{pgfscope}%
\begin{pgfscope}%
\pgfpathrectangle{\pgfqpoint{0.000000in}{0.000000in}}{\pgfqpoint{3.000000in}{3.000000in}}%
\pgfusepath{clip}%
\pgfsetbuttcap%
\pgfsetroundjoin%
\definecolor{currentfill}{rgb}{0.000000,0.300000,1.000000}%
\pgfsetfillcolor{currentfill}%
\pgfsetlinewidth{0.000000pt}%
\definecolor{currentstroke}{rgb}{0.000000,0.000000,0.000000}%
\pgfsetstrokecolor{currentstroke}%
\pgfsetdash{}{0pt}%
\pgfpathmoveto{\pgfqpoint{1.597266in}{1.135833in}}%
\pgfpathlineto{\pgfqpoint{1.601796in}{1.178720in}}%
\pgfpathlineto{\pgfqpoint{1.636741in}{1.185154in}}%
\pgfpathlineto{\pgfqpoint{1.629630in}{1.141817in}}%
\pgfpathlineto{\pgfqpoint{1.597266in}{1.135833in}}%
\pgfpathclose%
\pgfusepath{fill}%
\end{pgfscope}%
\begin{pgfscope}%
\pgfpathrectangle{\pgfqpoint{0.000000in}{0.000000in}}{\pgfqpoint{3.000000in}{3.000000in}}%
\pgfusepath{clip}%
\pgfsetbuttcap%
\pgfsetroundjoin%
\definecolor{currentfill}{rgb}{1.000000,0.886710,0.000000}%
\pgfsetfillcolor{currentfill}%
\pgfsetlinewidth{0.000000pt}%
\definecolor{currentstroke}{rgb}{0.000000,0.000000,0.000000}%
\pgfsetstrokecolor{currentstroke}%
\pgfsetdash{}{0pt}%
\pgfpathmoveto{\pgfqpoint{2.011320in}{1.736429in}}%
\pgfpathlineto{\pgfqpoint{2.028430in}{1.757199in}}%
\pgfpathlineto{\pgfqpoint{2.012494in}{1.794020in}}%
\pgfpathlineto{\pgfqpoint{1.996007in}{1.772044in}}%
\pgfpathlineto{\pgfqpoint{2.011320in}{1.736429in}}%
\pgfpathclose%
\pgfusepath{fill}%
\end{pgfscope}%
\begin{pgfscope}%
\pgfpathrectangle{\pgfqpoint{0.000000in}{0.000000in}}{\pgfqpoint{3.000000in}{3.000000in}}%
\pgfusepath{clip}%
\pgfsetbuttcap%
\pgfsetroundjoin%
\definecolor{currentfill}{rgb}{1.000000,0.407407,0.000000}%
\pgfsetfillcolor{currentfill}%
\pgfsetlinewidth{0.000000pt}%
\definecolor{currentstroke}{rgb}{0.000000,0.000000,0.000000}%
\pgfsetstrokecolor{currentstroke}%
\pgfsetdash{}{0pt}%
\pgfpathmoveto{\pgfqpoint{0.990433in}{1.942128in}}%
\pgfpathlineto{\pgfqpoint{0.974575in}{1.961282in}}%
\pgfpathlineto{\pgfqpoint{0.944654in}{1.917513in}}%
\pgfpathlineto{\pgfqpoint{0.961430in}{1.899500in}}%
\pgfpathlineto{\pgfqpoint{0.990433in}{1.942128in}}%
\pgfpathclose%
\pgfusepath{fill}%
\end{pgfscope}%
\begin{pgfscope}%
\pgfpathrectangle{\pgfqpoint{0.000000in}{0.000000in}}{\pgfqpoint{3.000000in}{3.000000in}}%
\pgfusepath{clip}%
\pgfsetbuttcap%
\pgfsetroundjoin%
\definecolor{currentfill}{rgb}{0.927807,0.015251,0.000000}%
\pgfsetfillcolor{currentfill}%
\pgfsetlinewidth{0.000000pt}%
\definecolor{currentstroke}{rgb}{0.000000,0.000000,0.000000}%
\pgfsetstrokecolor{currentstroke}%
\pgfsetdash{}{0pt}%
\pgfpathmoveto{\pgfqpoint{2.227145in}{2.035853in}}%
\pgfpathlineto{\pgfqpoint{2.243678in}{2.052132in}}%
\pgfpathlineto{\pgfqpoint{2.201746in}{2.102737in}}%
\pgfpathlineto{\pgfqpoint{2.186286in}{2.085366in}}%
\pgfpathlineto{\pgfqpoint{2.227145in}{2.035853in}}%
\pgfpathclose%
\pgfusepath{fill}%
\end{pgfscope}%
\begin{pgfscope}%
\pgfpathrectangle{\pgfqpoint{0.000000in}{0.000000in}}{\pgfqpoint{3.000000in}{3.000000in}}%
\pgfusepath{clip}%
\pgfsetbuttcap%
\pgfsetroundjoin%
\definecolor{currentfill}{rgb}{0.895003,1.000000,0.072739}%
\pgfsetfillcolor{currentfill}%
\pgfsetlinewidth{0.000000pt}%
\definecolor{currentstroke}{rgb}{0.000000,0.000000,0.000000}%
\pgfsetstrokecolor{currentstroke}%
\pgfsetdash{}{0pt}%
\pgfpathmoveto{\pgfqpoint{1.112182in}{1.714947in}}%
\pgfpathlineto{\pgfqpoint{1.095454in}{1.738029in}}%
\pgfpathlineto{\pgfqpoint{1.084409in}{1.703249in}}%
\pgfpathlineto{\pgfqpoint{1.101615in}{1.681394in}}%
\pgfpathlineto{\pgfqpoint{1.112182in}{1.714947in}}%
\pgfpathclose%
\pgfusepath{fill}%
\end{pgfscope}%
\begin{pgfscope}%
\pgfpathrectangle{\pgfqpoint{0.000000in}{0.000000in}}{\pgfqpoint{3.000000in}{3.000000in}}%
\pgfusepath{clip}%
\pgfsetbuttcap%
\pgfsetroundjoin%
\definecolor{currentfill}{rgb}{0.300443,1.000000,0.667299}%
\pgfsetfillcolor{currentfill}%
\pgfsetlinewidth{0.000000pt}%
\definecolor{currentstroke}{rgb}{0.000000,0.000000,0.000000}%
\pgfsetstrokecolor{currentstroke}%
\pgfsetdash{}{0pt}%
\pgfpathmoveto{\pgfqpoint{1.222203in}{1.471748in}}%
\pgfpathlineto{\pgfqpoint{1.205053in}{1.501289in}}%
\pgfpathlineto{\pgfqpoint{1.213791in}{1.475083in}}%
\pgfpathlineto{\pgfqpoint{1.230539in}{1.446807in}}%
\pgfpathlineto{\pgfqpoint{1.222203in}{1.471748in}}%
\pgfpathclose%
\pgfusepath{fill}%
\end{pgfscope}%
\begin{pgfscope}%
\pgfpathrectangle{\pgfqpoint{0.000000in}{0.000000in}}{\pgfqpoint{3.000000in}{3.000000in}}%
\pgfusepath{clip}%
\pgfsetbuttcap%
\pgfsetroundjoin%
\definecolor{currentfill}{rgb}{0.000000,0.300000,1.000000}%
\pgfsetfillcolor{currentfill}%
\pgfsetlinewidth{0.000000pt}%
\definecolor{currentstroke}{rgb}{0.000000,0.000000,0.000000}%
\pgfsetstrokecolor{currentstroke}%
\pgfsetdash{}{0pt}%
\pgfpathmoveto{\pgfqpoint{1.462027in}{1.139529in}}%
\pgfpathlineto{\pgfqpoint{1.455759in}{1.182694in}}%
\pgfpathlineto{\pgfqpoint{1.491326in}{1.177218in}}%
\pgfpathlineto{\pgfqpoint{1.494966in}{1.134435in}}%
\pgfpathlineto{\pgfqpoint{1.462027in}{1.139529in}}%
\pgfpathclose%
\pgfusepath{fill}%
\end{pgfscope}%
\begin{pgfscope}%
\pgfpathrectangle{\pgfqpoint{0.000000in}{0.000000in}}{\pgfqpoint{3.000000in}{3.000000in}}%
\pgfusepath{clip}%
\pgfsetbuttcap%
\pgfsetroundjoin%
\definecolor{currentfill}{rgb}{0.667299,1.000000,0.300443}%
\pgfsetfillcolor{currentfill}%
\pgfsetlinewidth{0.000000pt}%
\definecolor{currentstroke}{rgb}{0.000000,0.000000,0.000000}%
\pgfsetstrokecolor{currentstroke}%
\pgfsetdash{}{0pt}%
\pgfpathmoveto{\pgfqpoint{1.928756in}{1.589989in}}%
\pgfpathlineto{\pgfqpoint{1.946005in}{1.613934in}}%
\pgfpathlineto{\pgfqpoint{1.942968in}{1.645422in}}%
\pgfpathlineto{\pgfqpoint{1.925904in}{1.620220in}}%
\pgfpathlineto{\pgfqpoint{1.928756in}{1.589989in}}%
\pgfpathclose%
\pgfusepath{fill}%
\end{pgfscope}%
\begin{pgfscope}%
\pgfpathrectangle{\pgfqpoint{0.000000in}{0.000000in}}{\pgfqpoint{3.000000in}{3.000000in}}%
\pgfusepath{clip}%
\pgfsetbuttcap%
\pgfsetroundjoin%
\definecolor{currentfill}{rgb}{0.856506,0.000000,0.000000}%
\pgfsetfillcolor{currentfill}%
\pgfsetlinewidth{0.000000pt}%
\definecolor{currentstroke}{rgb}{0.000000,0.000000,0.000000}%
\pgfsetstrokecolor{currentstroke}%
\pgfsetdash{}{0pt}%
\pgfpathmoveto{\pgfqpoint{2.243678in}{2.052132in}}%
\pgfpathlineto{\pgfqpoint{2.260213in}{2.068177in}}%
\pgfpathlineto{\pgfqpoint{2.217204in}{2.119871in}}%
\pgfpathlineto{\pgfqpoint{2.201746in}{2.102737in}}%
\pgfpathlineto{\pgfqpoint{2.243678in}{2.052132in}}%
\pgfpathclose%
\pgfusepath{fill}%
\end{pgfscope}%
\begin{pgfscope}%
\pgfpathrectangle{\pgfqpoint{0.000000in}{0.000000in}}{\pgfqpoint{3.000000in}{3.000000in}}%
\pgfusepath{clip}%
\pgfsetbuttcap%
\pgfsetroundjoin%
\definecolor{currentfill}{rgb}{0.000000,0.676471,1.000000}%
\pgfsetfillcolor{currentfill}%
\pgfsetlinewidth{0.000000pt}%
\definecolor{currentstroke}{rgb}{0.000000,0.000000,0.000000}%
\pgfsetstrokecolor{currentstroke}%
\pgfsetdash{}{0pt}%
\pgfpathmoveto{\pgfqpoint{1.328545in}{1.271423in}}%
\pgfpathlineto{\pgfqpoint{1.313962in}{1.308350in}}%
\pgfpathlineto{\pgfqpoint{1.339408in}{1.291085in}}%
\pgfpathlineto{\pgfqpoint{1.352375in}{1.255209in}}%
\pgfpathlineto{\pgfqpoint{1.328545in}{1.271423in}}%
\pgfpathclose%
\pgfusepath{fill}%
\end{pgfscope}%
\begin{pgfscope}%
\pgfpathrectangle{\pgfqpoint{0.000000in}{0.000000in}}{\pgfqpoint{3.000000in}{3.000000in}}%
\pgfusepath{clip}%
\pgfsetbuttcap%
\pgfsetroundjoin%
\definecolor{currentfill}{rgb}{1.000000,0.349310,0.000000}%
\pgfsetfillcolor{currentfill}%
\pgfsetlinewidth{0.000000pt}%
\definecolor{currentstroke}{rgb}{0.000000,0.000000,0.000000}%
\pgfsetstrokecolor{currentstroke}%
\pgfsetdash{}{0pt}%
\pgfpathmoveto{\pgfqpoint{0.974575in}{1.961282in}}%
\pgfpathlineto{\pgfqpoint{0.958717in}{1.980066in}}%
\pgfpathlineto{\pgfqpoint{0.927874in}{1.935159in}}%
\pgfpathlineto{\pgfqpoint{0.944654in}{1.917513in}}%
\pgfpathlineto{\pgfqpoint{0.974575in}{1.961282in}}%
\pgfpathclose%
\pgfusepath{fill}%
\end{pgfscope}%
\begin{pgfscope}%
\pgfpathrectangle{\pgfqpoint{0.000000in}{0.000000in}}{\pgfqpoint{3.000000in}{3.000000in}}%
\pgfusepath{clip}%
\pgfsetbuttcap%
\pgfsetroundjoin%
\definecolor{currentfill}{rgb}{0.000000,0.849020,1.000000}%
\pgfsetfillcolor{currentfill}%
\pgfsetlinewidth{0.000000pt}%
\definecolor{currentstroke}{rgb}{0.000000,0.000000,0.000000}%
\pgfsetstrokecolor{currentstroke}%
\pgfsetdash{}{0pt}%
\pgfpathmoveto{\pgfqpoint{1.294223in}{1.327402in}}%
\pgfpathlineto{\pgfqpoint{1.278375in}{1.361972in}}%
\pgfpathlineto{\pgfqpoint{1.299353in}{1.341764in}}%
\pgfpathlineto{\pgfqpoint{1.313962in}{1.308350in}}%
\pgfpathlineto{\pgfqpoint{1.294223in}{1.327402in}}%
\pgfpathclose%
\pgfusepath{fill}%
\end{pgfscope}%
\begin{pgfscope}%
\pgfpathrectangle{\pgfqpoint{0.000000in}{0.000000in}}{\pgfqpoint{3.000000in}{3.000000in}}%
\pgfusepath{clip}%
\pgfsetbuttcap%
\pgfsetroundjoin%
\definecolor{currentfill}{rgb}{0.401645,1.000000,0.566097}%
\pgfsetfillcolor{currentfill}%
\pgfsetlinewidth{0.000000pt}%
\definecolor{currentstroke}{rgb}{0.000000,0.000000,0.000000}%
\pgfsetstrokecolor{currentstroke}%
\pgfsetdash{}{0pt}%
\pgfpathmoveto{\pgfqpoint{1.871120in}{1.483745in}}%
\pgfpathlineto{\pgfqpoint{1.888069in}{1.510618in}}%
\pgfpathlineto{\pgfqpoint{1.894302in}{1.538263in}}%
\pgfpathlineto{\pgfqpoint{1.877097in}{1.510120in}}%
\pgfpathlineto{\pgfqpoint{1.871120in}{1.483745in}}%
\pgfpathclose%
\pgfusepath{fill}%
\end{pgfscope}%
\begin{pgfscope}%
\pgfpathrectangle{\pgfqpoint{0.000000in}{0.000000in}}{\pgfqpoint{3.000000in}{3.000000in}}%
\pgfusepath{clip}%
\pgfsetbuttcap%
\pgfsetroundjoin%
\definecolor{currentfill}{rgb}{0.578748,1.000000,0.388994}%
\pgfsetfillcolor{currentfill}%
\pgfsetlinewidth{0.000000pt}%
\definecolor{currentstroke}{rgb}{0.000000,0.000000,0.000000}%
\pgfsetstrokecolor{currentstroke}%
\pgfsetdash{}{0pt}%
\pgfpathmoveto{\pgfqpoint{1.170329in}{1.584172in}}%
\pgfpathlineto{\pgfqpoint{1.153167in}{1.610183in}}%
\pgfpathlineto{\pgfqpoint{1.153503in}{1.579887in}}%
\pgfpathlineto{\pgfqpoint{1.170703in}{1.555141in}}%
\pgfpathlineto{\pgfqpoint{1.170329in}{1.584172in}}%
\pgfpathclose%
\pgfusepath{fill}%
\end{pgfscope}%
\begin{pgfscope}%
\pgfpathrectangle{\pgfqpoint{0.000000in}{0.000000in}}{\pgfqpoint{3.000000in}{3.000000in}}%
\pgfusepath{clip}%
\pgfsetbuttcap%
\pgfsetroundjoin%
\definecolor{currentfill}{rgb}{0.803030,0.000000,0.000000}%
\pgfsetfillcolor{currentfill}%
\pgfsetlinewidth{0.000000pt}%
\definecolor{currentstroke}{rgb}{0.000000,0.000000,0.000000}%
\pgfsetstrokecolor{currentstroke}%
\pgfsetdash{}{0pt}%
\pgfpathmoveto{\pgfqpoint{2.260213in}{2.068177in}}%
\pgfpathlineto{\pgfqpoint{2.276751in}{2.084001in}}%
\pgfpathlineto{\pgfqpoint{2.232660in}{2.136781in}}%
\pgfpathlineto{\pgfqpoint{2.217204in}{2.119871in}}%
\pgfpathlineto{\pgfqpoint{2.260213in}{2.068177in}}%
\pgfpathclose%
\pgfusepath{fill}%
\end{pgfscope}%
\begin{pgfscope}%
\pgfpathrectangle{\pgfqpoint{0.000000in}{0.000000in}}{\pgfqpoint{3.000000in}{3.000000in}}%
\pgfusepath{clip}%
\pgfsetbuttcap%
\pgfsetroundjoin%
\definecolor{currentfill}{rgb}{0.000000,0.300000,1.000000}%
\pgfsetfillcolor{currentfill}%
\pgfsetlinewidth{0.000000pt}%
\definecolor{currentstroke}{rgb}{0.000000,0.000000,0.000000}%
\pgfsetstrokecolor{currentstroke}%
\pgfsetdash{}{0pt}%
\pgfpathmoveto{\pgfqpoint{1.563458in}{1.132562in}}%
\pgfpathlineto{\pgfqpoint{1.565289in}{1.175204in}}%
\pgfpathlineto{\pgfqpoint{1.601796in}{1.178720in}}%
\pgfpathlineto{\pgfqpoint{1.597266in}{1.135833in}}%
\pgfpathlineto{\pgfqpoint{1.563458in}{1.132562in}}%
\pgfpathclose%
\pgfusepath{fill}%
\end{pgfscope}%
\begin{pgfscope}%
\pgfpathrectangle{\pgfqpoint{0.000000in}{0.000000in}}{\pgfqpoint{3.000000in}{3.000000in}}%
\pgfusepath{clip}%
\pgfsetbuttcap%
\pgfsetroundjoin%
\definecolor{currentfill}{rgb}{0.199241,1.000000,0.768501}%
\pgfsetfillcolor{currentfill}%
\pgfsetlinewidth{0.000000pt}%
\definecolor{currentstroke}{rgb}{0.000000,0.000000,0.000000}%
\pgfsetstrokecolor{currentstroke}%
\pgfsetdash{}{0pt}%
\pgfpathmoveto{\pgfqpoint{1.824454in}{1.401080in}}%
\pgfpathlineto{\pgfqpoint{1.840674in}{1.430645in}}%
\pgfpathlineto{\pgfqpoint{1.854191in}{1.455051in}}%
\pgfpathlineto{\pgfqpoint{1.837282in}{1.424242in}}%
\pgfpathlineto{\pgfqpoint{1.824454in}{1.401080in}}%
\pgfpathclose%
\pgfusepath{fill}%
\end{pgfscope}%
\begin{pgfscope}%
\pgfpathrectangle{\pgfqpoint{0.000000in}{0.000000in}}{\pgfqpoint{3.000000in}{3.000000in}}%
\pgfusepath{clip}%
\pgfsetbuttcap%
\pgfsetroundjoin%
\definecolor{currentfill}{rgb}{1.000000,0.814089,0.000000}%
\pgfsetfillcolor{currentfill}%
\pgfsetlinewidth{0.000000pt}%
\definecolor{currentstroke}{rgb}{0.000000,0.000000,0.000000}%
\pgfsetstrokecolor{currentstroke}%
\pgfsetdash{}{0pt}%
\pgfpathmoveto{\pgfqpoint{2.028430in}{1.757199in}}%
\pgfpathlineto{\pgfqpoint{2.045550in}{1.777321in}}%
\pgfpathlineto{\pgfqpoint{2.028985in}{1.815345in}}%
\pgfpathlineto{\pgfqpoint{2.012494in}{1.794020in}}%
\pgfpathlineto{\pgfqpoint{2.028430in}{1.757199in}}%
\pgfpathclose%
\pgfusepath{fill}%
\end{pgfscope}%
\begin{pgfscope}%
\pgfpathrectangle{\pgfqpoint{0.000000in}{0.000000in}}{\pgfqpoint{3.000000in}{3.000000in}}%
\pgfusepath{clip}%
\pgfsetbuttcap%
\pgfsetroundjoin%
\definecolor{currentfill}{rgb}{0.000000,0.300000,1.000000}%
\pgfsetfillcolor{currentfill}%
\pgfsetlinewidth{0.000000pt}%
\definecolor{currentstroke}{rgb}{0.000000,0.000000,0.000000}%
\pgfsetstrokecolor{currentstroke}%
\pgfsetdash{}{0pt}%
\pgfpathmoveto{\pgfqpoint{1.494966in}{1.134435in}}%
\pgfpathlineto{\pgfqpoint{1.491326in}{1.177218in}}%
\pgfpathlineto{\pgfqpoint{1.528149in}{1.174699in}}%
\pgfpathlineto{\pgfqpoint{1.529066in}{1.132093in}}%
\pgfpathlineto{\pgfqpoint{1.494966in}{1.134435in}}%
\pgfpathclose%
\pgfusepath{fill}%
\end{pgfscope}%
\begin{pgfscope}%
\pgfpathrectangle{\pgfqpoint{0.000000in}{0.000000in}}{\pgfqpoint{3.000000in}{3.000000in}}%
\pgfusepath{clip}%
\pgfsetbuttcap%
\pgfsetroundjoin%
\definecolor{currentfill}{rgb}{0.000000,0.503922,1.000000}%
\pgfsetfillcolor{currentfill}%
\pgfsetlinewidth{0.000000pt}%
\definecolor{currentstroke}{rgb}{0.000000,0.000000,0.000000}%
\pgfsetstrokecolor{currentstroke}%
\pgfsetdash{}{0pt}%
\pgfpathmoveto{\pgfqpoint{1.669235in}{1.194336in}}%
\pgfpathlineto{\pgfqpoint{1.678761in}{1.233040in}}%
\pgfpathlineto{\pgfqpoint{1.710135in}{1.245544in}}%
\pgfpathlineto{\pgfqpoint{1.698461in}{1.206025in}}%
\pgfpathlineto{\pgfqpoint{1.669235in}{1.194336in}}%
\pgfpathclose%
\pgfusepath{fill}%
\end{pgfscope}%
\begin{pgfscope}%
\pgfpathrectangle{\pgfqpoint{0.000000in}{0.000000in}}{\pgfqpoint{3.000000in}{3.000000in}}%
\pgfusepath{clip}%
\pgfsetbuttcap%
\pgfsetroundjoin%
\definecolor{currentfill}{rgb}{0.731729,0.000000,0.000000}%
\pgfsetfillcolor{currentfill}%
\pgfsetlinewidth{0.000000pt}%
\definecolor{currentstroke}{rgb}{0.000000,0.000000,0.000000}%
\pgfsetstrokecolor{currentstroke}%
\pgfsetdash{}{0pt}%
\pgfpathmoveto{\pgfqpoint{2.276751in}{2.084001in}}%
\pgfpathlineto{\pgfqpoint{2.293291in}{2.099616in}}%
\pgfpathlineto{\pgfqpoint{2.248115in}{2.153479in}}%
\pgfpathlineto{\pgfqpoint{2.232660in}{2.136781in}}%
\pgfpathlineto{\pgfqpoint{2.276751in}{2.084001in}}%
\pgfpathclose%
\pgfusepath{fill}%
\end{pgfscope}%
\begin{pgfscope}%
\pgfpathrectangle{\pgfqpoint{0.000000in}{0.000000in}}{\pgfqpoint{3.000000in}{3.000000in}}%
\pgfusepath{clip}%
\pgfsetbuttcap%
\pgfsetroundjoin%
\definecolor{currentfill}{rgb}{0.000000,0.300000,1.000000}%
\pgfsetfillcolor{currentfill}%
\pgfsetlinewidth{0.000000pt}%
\definecolor{currentstroke}{rgb}{0.000000,0.000000,0.000000}%
\pgfsetstrokecolor{currentstroke}%
\pgfsetdash{}{0pt}%
\pgfpathmoveto{\pgfqpoint{1.529066in}{1.132093in}}%
\pgfpathlineto{\pgfqpoint{1.528149in}{1.174699in}}%
\pgfpathlineto{\pgfqpoint{1.565289in}{1.175204in}}%
\pgfpathlineto{\pgfqpoint{1.563458in}{1.132562in}}%
\pgfpathlineto{\pgfqpoint{1.529066in}{1.132093in}}%
\pgfpathclose%
\pgfusepath{fill}%
\end{pgfscope}%
\begin{pgfscope}%
\pgfpathrectangle{\pgfqpoint{0.000000in}{0.000000in}}{\pgfqpoint{3.000000in}{3.000000in}}%
\pgfusepath{clip}%
\pgfsetbuttcap%
\pgfsetroundjoin%
\definecolor{currentfill}{rgb}{1.000000,0.291213,0.000000}%
\pgfsetfillcolor{currentfill}%
\pgfsetlinewidth{0.000000pt}%
\definecolor{currentstroke}{rgb}{0.000000,0.000000,0.000000}%
\pgfsetstrokecolor{currentstroke}%
\pgfsetdash{}{0pt}%
\pgfpathmoveto{\pgfqpoint{0.958717in}{1.980066in}}%
\pgfpathlineto{\pgfqpoint{0.942858in}{1.998505in}}%
\pgfpathlineto{\pgfqpoint{0.911088in}{1.952463in}}%
\pgfpathlineto{\pgfqpoint{0.927874in}{1.935159in}}%
\pgfpathlineto{\pgfqpoint{0.958717in}{1.980066in}}%
\pgfpathclose%
\pgfusepath{fill}%
\end{pgfscope}%
\begin{pgfscope}%
\pgfpathrectangle{\pgfqpoint{0.000000in}{0.000000in}}{\pgfqpoint{3.000000in}{3.000000in}}%
\pgfusepath{clip}%
\pgfsetbuttcap%
\pgfsetroundjoin%
\definecolor{currentfill}{rgb}{0.958254,0.973856,0.009488}%
\pgfsetfillcolor{currentfill}%
\pgfsetlinewidth{0.000000pt}%
\definecolor{currentstroke}{rgb}{0.000000,0.000000,0.000000}%
\pgfsetstrokecolor{currentstroke}%
\pgfsetdash{}{0pt}%
\pgfpathmoveto{\pgfqpoint{1.095454in}{1.738029in}}%
\pgfpathlineto{\pgfqpoint{1.078719in}{1.760324in}}%
\pgfpathlineto{\pgfqpoint{1.067192in}{1.724318in}}%
\pgfpathlineto{\pgfqpoint{1.084409in}{1.703249in}}%
\pgfpathlineto{\pgfqpoint{1.095454in}{1.738029in}}%
\pgfpathclose%
\pgfusepath{fill}%
\end{pgfscope}%
\begin{pgfscope}%
\pgfpathrectangle{\pgfqpoint{0.000000in}{0.000000in}}{\pgfqpoint{3.000000in}{3.000000in}}%
\pgfusepath{clip}%
\pgfsetbuttcap%
\pgfsetroundjoin%
\definecolor{currentfill}{rgb}{0.678253,0.000000,0.000000}%
\pgfsetfillcolor{currentfill}%
\pgfsetlinewidth{0.000000pt}%
\definecolor{currentstroke}{rgb}{0.000000,0.000000,0.000000}%
\pgfsetstrokecolor{currentstroke}%
\pgfsetdash{}{0pt}%
\pgfpathmoveto{\pgfqpoint{2.293291in}{2.099616in}}%
\pgfpathlineto{\pgfqpoint{2.309833in}{2.115034in}}%
\pgfpathlineto{\pgfqpoint{2.263567in}{2.169977in}}%
\pgfpathlineto{\pgfqpoint{2.248115in}{2.153479in}}%
\pgfpathlineto{\pgfqpoint{2.293291in}{2.099616in}}%
\pgfpathclose%
\pgfusepath{fill}%
\end{pgfscope}%
\begin{pgfscope}%
\pgfpathrectangle{\pgfqpoint{0.000000in}{0.000000in}}{\pgfqpoint{3.000000in}{3.000000in}}%
\pgfusepath{clip}%
\pgfsetbuttcap%
\pgfsetroundjoin%
\definecolor{currentfill}{rgb}{1.000000,0.233115,0.000000}%
\pgfsetfillcolor{currentfill}%
\pgfsetlinewidth{0.000000pt}%
\definecolor{currentstroke}{rgb}{0.000000,0.000000,0.000000}%
\pgfsetstrokecolor{currentstroke}%
\pgfsetdash{}{0pt}%
\pgfpathmoveto{\pgfqpoint{0.942858in}{1.998505in}}%
\pgfpathlineto{\pgfqpoint{0.927000in}{2.016622in}}%
\pgfpathlineto{\pgfqpoint{0.894298in}{1.969447in}}%
\pgfpathlineto{\pgfqpoint{0.911088in}{1.952463in}}%
\pgfpathlineto{\pgfqpoint{0.942858in}{1.998505in}}%
\pgfpathclose%
\pgfusepath{fill}%
\end{pgfscope}%
\begin{pgfscope}%
\pgfpathrectangle{\pgfqpoint{0.000000in}{0.000000in}}{\pgfqpoint{3.000000in}{3.000000in}}%
\pgfusepath{clip}%
\pgfsetbuttcap%
\pgfsetroundjoin%
\definecolor{currentfill}{rgb}{0.606952,0.000000,0.000000}%
\pgfsetfillcolor{currentfill}%
\pgfsetlinewidth{0.000000pt}%
\definecolor{currentstroke}{rgb}{0.000000,0.000000,0.000000}%
\pgfsetstrokecolor{currentstroke}%
\pgfsetdash{}{0pt}%
\pgfpathmoveto{\pgfqpoint{2.309833in}{2.115034in}}%
\pgfpathlineto{\pgfqpoint{2.326378in}{2.130265in}}%
\pgfpathlineto{\pgfqpoint{2.279018in}{2.186284in}}%
\pgfpathlineto{\pgfqpoint{2.263567in}{2.169977in}}%
\pgfpathlineto{\pgfqpoint{2.309833in}{2.115034in}}%
\pgfpathclose%
\pgfusepath{fill}%
\end{pgfscope}%
\begin{pgfscope}%
\pgfpathrectangle{\pgfqpoint{0.000000in}{0.000000in}}{\pgfqpoint{3.000000in}{3.000000in}}%
\pgfusepath{clip}%
\pgfsetbuttcap%
\pgfsetroundjoin%
\definecolor{currentfill}{rgb}{0.000000,0.503922,1.000000}%
\pgfsetfillcolor{currentfill}%
\pgfsetlinewidth{0.000000pt}%
\definecolor{currentstroke}{rgb}{0.000000,0.000000,0.000000}%
\pgfsetstrokecolor{currentstroke}%
\pgfsetdash{}{0pt}%
\pgfpathmoveto{\pgfqpoint{1.391952in}{1.201868in}}%
\pgfpathlineto{\pgfqpoint{1.380962in}{1.241097in}}%
\pgfpathlineto{\pgfqpoint{1.413602in}{1.229454in}}%
\pgfpathlineto{\pgfqpoint{1.422353in}{1.190983in}}%
\pgfpathlineto{\pgfqpoint{1.391952in}{1.201868in}}%
\pgfpathclose%
\pgfusepath{fill}%
\end{pgfscope}%
\begin{pgfscope}%
\pgfpathrectangle{\pgfqpoint{0.000000in}{0.000000in}}{\pgfqpoint{3.000000in}{3.000000in}}%
\pgfusepath{clip}%
\pgfsetbuttcap%
\pgfsetroundjoin%
\definecolor{currentfill}{rgb}{1.000000,0.741467,0.000000}%
\pgfsetfillcolor{currentfill}%
\pgfsetlinewidth{0.000000pt}%
\definecolor{currentstroke}{rgb}{0.000000,0.000000,0.000000}%
\pgfsetstrokecolor{currentstroke}%
\pgfsetdash{}{0pt}%
\pgfpathmoveto{\pgfqpoint{2.045550in}{1.777321in}}%
\pgfpathlineto{\pgfqpoint{2.062678in}{1.796852in}}%
\pgfpathlineto{\pgfqpoint{2.045480in}{1.836076in}}%
\pgfpathlineto{\pgfqpoint{2.028985in}{1.815345in}}%
\pgfpathlineto{\pgfqpoint{2.045550in}{1.777321in}}%
\pgfpathclose%
\pgfusepath{fill}%
\end{pgfscope}%
\begin{pgfscope}%
\pgfpathrectangle{\pgfqpoint{0.000000in}{0.000000in}}{\pgfqpoint{3.000000in}{3.000000in}}%
\pgfusepath{clip}%
\pgfsetbuttcap%
\pgfsetroundjoin%
\definecolor{currentfill}{rgb}{0.743201,1.000000,0.224541}%
\pgfsetfillcolor{currentfill}%
\pgfsetlinewidth{0.000000pt}%
\definecolor{currentstroke}{rgb}{0.000000,0.000000,0.000000}%
\pgfsetstrokecolor{currentstroke}%
\pgfsetdash{}{0pt}%
\pgfpathmoveto{\pgfqpoint{1.946005in}{1.613934in}}%
\pgfpathlineto{\pgfqpoint{1.963269in}{1.636792in}}%
\pgfpathlineto{\pgfqpoint{1.960042in}{1.669536in}}%
\pgfpathlineto{\pgfqpoint{1.942968in}{1.645422in}}%
\pgfpathlineto{\pgfqpoint{1.946005in}{1.613934in}}%
\pgfpathclose%
\pgfusepath{fill}%
\end{pgfscope}%
\begin{pgfscope}%
\pgfpathrectangle{\pgfqpoint{0.000000in}{0.000000in}}{\pgfqpoint{3.000000in}{3.000000in}}%
\pgfusepath{clip}%
\pgfsetbuttcap%
\pgfsetroundjoin%
\definecolor{currentfill}{rgb}{0.553476,0.000000,0.000000}%
\pgfsetfillcolor{currentfill}%
\pgfsetlinewidth{0.000000pt}%
\definecolor{currentstroke}{rgb}{0.000000,0.000000,0.000000}%
\pgfsetstrokecolor{currentstroke}%
\pgfsetdash{}{0pt}%
\pgfpathmoveto{\pgfqpoint{2.326378in}{2.130265in}}%
\pgfpathlineto{\pgfqpoint{2.342925in}{2.145317in}}%
\pgfpathlineto{\pgfqpoint{2.294466in}{2.202411in}}%
\pgfpathlineto{\pgfqpoint{2.279018in}{2.186284in}}%
\pgfpathlineto{\pgfqpoint{2.326378in}{2.130265in}}%
\pgfpathclose%
\pgfusepath{fill}%
\end{pgfscope}%
\begin{pgfscope}%
\pgfpathrectangle{\pgfqpoint{0.000000in}{0.000000in}}{\pgfqpoint{3.000000in}{3.000000in}}%
\pgfusepath{clip}%
\pgfsetbuttcap%
\pgfsetroundjoin%
\definecolor{currentfill}{rgb}{1.000000,0.175018,0.000000}%
\pgfsetfillcolor{currentfill}%
\pgfsetlinewidth{0.000000pt}%
\definecolor{currentstroke}{rgb}{0.000000,0.000000,0.000000}%
\pgfsetstrokecolor{currentstroke}%
\pgfsetdash{}{0pt}%
\pgfpathmoveto{\pgfqpoint{0.927000in}{2.016622in}}%
\pgfpathlineto{\pgfqpoint{0.911142in}{2.034437in}}%
\pgfpathlineto{\pgfqpoint{0.877504in}{1.986134in}}%
\pgfpathlineto{\pgfqpoint{0.894298in}{1.969447in}}%
\pgfpathlineto{\pgfqpoint{0.927000in}{2.016622in}}%
\pgfpathclose%
\pgfusepath{fill}%
\end{pgfscope}%
\begin{pgfscope}%
\pgfpathrectangle{\pgfqpoint{0.000000in}{0.000000in}}{\pgfqpoint{3.000000in}{3.000000in}}%
\pgfusepath{clip}%
\pgfsetbuttcap%
\pgfsetroundjoin%
\definecolor{currentfill}{rgb}{0.500000,0.000000,0.000000}%
\pgfsetfillcolor{currentfill}%
\pgfsetlinewidth{0.000000pt}%
\definecolor{currentstroke}{rgb}{0.000000,0.000000,0.000000}%
\pgfsetstrokecolor{currentstroke}%
\pgfsetdash{}{0pt}%
\pgfpathmoveto{\pgfqpoint{2.342925in}{2.145317in}}%
\pgfpathlineto{\pgfqpoint{2.359475in}{2.160201in}}%
\pgfpathlineto{\pgfqpoint{2.309913in}{2.218365in}}%
\pgfpathlineto{\pgfqpoint{2.294466in}{2.202411in}}%
\pgfpathlineto{\pgfqpoint{2.342925in}{2.145317in}}%
\pgfpathclose%
\pgfusepath{fill}%
\end{pgfscope}%
\begin{pgfscope}%
\pgfpathrectangle{\pgfqpoint{0.000000in}{0.000000in}}{\pgfqpoint{3.000000in}{3.000000in}}%
\pgfusepath{clip}%
\pgfsetbuttcap%
\pgfsetroundjoin%
\definecolor{currentfill}{rgb}{1.000000,0.886710,0.000000}%
\pgfsetfillcolor{currentfill}%
\pgfsetlinewidth{0.000000pt}%
\definecolor{currentstroke}{rgb}{0.000000,0.000000,0.000000}%
\pgfsetstrokecolor{currentstroke}%
\pgfsetdash{}{0pt}%
\pgfpathmoveto{\pgfqpoint{1.078719in}{1.760324in}}%
\pgfpathlineto{\pgfqpoint{1.061980in}{1.781904in}}%
\pgfpathlineto{\pgfqpoint{1.049965in}{1.744676in}}%
\pgfpathlineto{\pgfqpoint{1.067192in}{1.724318in}}%
\pgfpathlineto{\pgfqpoint{1.078719in}{1.760324in}}%
\pgfpathclose%
\pgfusepath{fill}%
\end{pgfscope}%
\begin{pgfscope}%
\pgfpathrectangle{\pgfqpoint{0.000000in}{0.000000in}}{\pgfqpoint{3.000000in}{3.000000in}}%
\pgfusepath{clip}%
\pgfsetbuttcap%
\pgfsetroundjoin%
\definecolor{currentfill}{rgb}{0.401645,1.000000,0.566097}%
\pgfsetfillcolor{currentfill}%
\pgfsetlinewidth{0.000000pt}%
\definecolor{currentstroke}{rgb}{0.000000,0.000000,0.000000}%
\pgfsetstrokecolor{currentstroke}%
\pgfsetdash{}{0pt}%
\pgfpathmoveto{\pgfqpoint{1.205053in}{1.501289in}}%
\pgfpathlineto{\pgfqpoint{1.187886in}{1.529007in}}%
\pgfpathlineto{\pgfqpoint{1.197023in}{1.501537in}}%
\pgfpathlineto{\pgfqpoint{1.213791in}{1.475083in}}%
\pgfpathlineto{\pgfqpoint{1.205053in}{1.501289in}}%
\pgfpathclose%
\pgfusepath{fill}%
\end{pgfscope}%
\begin{pgfscope}%
\pgfpathrectangle{\pgfqpoint{0.000000in}{0.000000in}}{\pgfqpoint{3.000000in}{3.000000in}}%
\pgfusepath{clip}%
\pgfsetbuttcap%
\pgfsetroundjoin%
\definecolor{currentfill}{rgb}{0.667299,1.000000,0.300443}%
\pgfsetfillcolor{currentfill}%
\pgfsetlinewidth{0.000000pt}%
\definecolor{currentstroke}{rgb}{0.000000,0.000000,0.000000}%
\pgfsetstrokecolor{currentstroke}%
\pgfsetdash{}{0pt}%
\pgfpathmoveto{\pgfqpoint{1.153167in}{1.610183in}}%
\pgfpathlineto{\pgfqpoint{1.135994in}{1.634969in}}%
\pgfpathlineto{\pgfqpoint{1.136286in}{1.603410in}}%
\pgfpathlineto{\pgfqpoint{1.153503in}{1.579887in}}%
\pgfpathlineto{\pgfqpoint{1.153167in}{1.610183in}}%
\pgfpathclose%
\pgfusepath{fill}%
\end{pgfscope}%
\begin{pgfscope}%
\pgfpathrectangle{\pgfqpoint{0.000000in}{0.000000in}}{\pgfqpoint{3.000000in}{3.000000in}}%
\pgfusepath{clip}%
\pgfsetbuttcap%
\pgfsetroundjoin%
\definecolor{currentfill}{rgb}{0.199241,1.000000,0.768501}%
\pgfsetfillcolor{currentfill}%
\pgfsetlinewidth{0.000000pt}%
\definecolor{currentstroke}{rgb}{0.000000,0.000000,0.000000}%
\pgfsetstrokecolor{currentstroke}%
\pgfsetdash{}{0pt}%
\pgfpathmoveto{\pgfqpoint{1.247265in}{1.416417in}}%
\pgfpathlineto{\pgfqpoint{1.230539in}{1.446807in}}%
\pgfpathlineto{\pgfqpoint{1.246604in}{1.422772in}}%
\pgfpathlineto{\pgfqpoint{1.262502in}{1.393610in}}%
\pgfpathlineto{\pgfqpoint{1.247265in}{1.416417in}}%
\pgfpathclose%
\pgfusepath{fill}%
\end{pgfscope}%
\begin{pgfscope}%
\pgfpathrectangle{\pgfqpoint{0.000000in}{0.000000in}}{\pgfqpoint{3.000000in}{3.000000in}}%
\pgfusepath{clip}%
\pgfsetbuttcap%
\pgfsetroundjoin%
\definecolor{currentfill}{rgb}{0.085389,1.000000,0.882353}%
\pgfsetfillcolor{currentfill}%
\pgfsetlinewidth{0.000000pt}%
\definecolor{currentstroke}{rgb}{0.000000,0.000000,0.000000}%
\pgfsetstrokecolor{currentstroke}%
\pgfsetdash{}{0pt}%
\pgfpathmoveto{\pgfqpoint{1.789425in}{1.348315in}}%
\pgfpathlineto{\pgfqpoint{1.804516in}{1.379173in}}%
\pgfpathlineto{\pgfqpoint{1.824454in}{1.401080in}}%
\pgfpathlineto{\pgfqpoint{1.808257in}{1.369039in}}%
\pgfpathlineto{\pgfqpoint{1.789425in}{1.348315in}}%
\pgfpathclose%
\pgfusepath{fill}%
\end{pgfscope}%
\begin{pgfscope}%
\pgfpathrectangle{\pgfqpoint{0.000000in}{0.000000in}}{\pgfqpoint{3.000000in}{3.000000in}}%
\pgfusepath{clip}%
\pgfsetbuttcap%
\pgfsetroundjoin%
\definecolor{currentfill}{rgb}{0.000000,0.676471,1.000000}%
\pgfsetfillcolor{currentfill}%
\pgfsetlinewidth{0.000000pt}%
\definecolor{currentstroke}{rgb}{0.000000,0.000000,0.000000}%
\pgfsetstrokecolor{currentstroke}%
\pgfsetdash{}{0pt}%
\pgfpathmoveto{\pgfqpoint{1.710135in}{1.245544in}}%
\pgfpathlineto{\pgfqpoint{1.721835in}{1.280792in}}%
\pgfpathlineto{\pgfqpoint{1.750754in}{1.296613in}}%
\pgfpathlineto{\pgfqpoint{1.737209in}{1.260400in}}%
\pgfpathlineto{\pgfqpoint{1.710135in}{1.245544in}}%
\pgfpathclose%
\pgfusepath{fill}%
\end{pgfscope}%
\begin{pgfscope}%
\pgfpathrectangle{\pgfqpoint{0.000000in}{0.000000in}}{\pgfqpoint{3.000000in}{3.000000in}}%
\pgfusepath{clip}%
\pgfsetbuttcap%
\pgfsetroundjoin%
\definecolor{currentfill}{rgb}{1.000000,0.116921,0.000000}%
\pgfsetfillcolor{currentfill}%
\pgfsetlinewidth{0.000000pt}%
\definecolor{currentstroke}{rgb}{0.000000,0.000000,0.000000}%
\pgfsetstrokecolor{currentstroke}%
\pgfsetdash{}{0pt}%
\pgfpathmoveto{\pgfqpoint{0.911142in}{2.034437in}}%
\pgfpathlineto{\pgfqpoint{0.895284in}{2.051970in}}%
\pgfpathlineto{\pgfqpoint{0.860705in}{2.002541in}}%
\pgfpathlineto{\pgfqpoint{0.877504in}{1.986134in}}%
\pgfpathlineto{\pgfqpoint{0.911142in}{2.034437in}}%
\pgfpathclose%
\pgfusepath{fill}%
\end{pgfscope}%
\begin{pgfscope}%
\pgfpathrectangle{\pgfqpoint{0.000000in}{0.000000in}}{\pgfqpoint{3.000000in}{3.000000in}}%
\pgfusepath{clip}%
\pgfsetbuttcap%
\pgfsetroundjoin%
\definecolor{currentfill}{rgb}{0.490196,1.000000,0.477546}%
\pgfsetfillcolor{currentfill}%
\pgfsetlinewidth{0.000000pt}%
\definecolor{currentstroke}{rgb}{0.000000,0.000000,0.000000}%
\pgfsetstrokecolor{currentstroke}%
\pgfsetdash{}{0pt}%
\pgfpathmoveto{\pgfqpoint{1.888069in}{1.510618in}}%
\pgfpathlineto{\pgfqpoint{1.905038in}{1.535908in}}%
\pgfpathlineto{\pgfqpoint{1.911521in}{1.564820in}}%
\pgfpathlineto{\pgfqpoint{1.894302in}{1.538263in}}%
\pgfpathlineto{\pgfqpoint{1.888069in}{1.510618in}}%
\pgfpathclose%
\pgfusepath{fill}%
\end{pgfscope}%
\begin{pgfscope}%
\pgfpathrectangle{\pgfqpoint{0.000000in}{0.000000in}}{\pgfqpoint{3.000000in}{3.000000in}}%
\pgfusepath{clip}%
\pgfsetbuttcap%
\pgfsetroundjoin%
\definecolor{currentfill}{rgb}{1.000000,0.668845,0.000000}%
\pgfsetfillcolor{currentfill}%
\pgfsetlinewidth{0.000000pt}%
\definecolor{currentstroke}{rgb}{0.000000,0.000000,0.000000}%
\pgfsetstrokecolor{currentstroke}%
\pgfsetdash{}{0pt}%
\pgfpathmoveto{\pgfqpoint{2.062678in}{1.796852in}}%
\pgfpathlineto{\pgfqpoint{2.079815in}{1.815841in}}%
\pgfpathlineto{\pgfqpoint{2.061979in}{1.856262in}}%
\pgfpathlineto{\pgfqpoint{2.045480in}{1.836076in}}%
\pgfpathlineto{\pgfqpoint{2.062678in}{1.796852in}}%
\pgfpathclose%
\pgfusepath{fill}%
\end{pgfscope}%
\begin{pgfscope}%
\pgfpathrectangle{\pgfqpoint{0.000000in}{0.000000in}}{\pgfqpoint{3.000000in}{3.000000in}}%
\pgfusepath{clip}%
\pgfsetbuttcap%
\pgfsetroundjoin%
\definecolor{currentfill}{rgb}{0.000000,0.849020,1.000000}%
\pgfsetfillcolor{currentfill}%
\pgfsetlinewidth{0.000000pt}%
\definecolor{currentstroke}{rgb}{0.000000,0.000000,0.000000}%
\pgfsetstrokecolor{currentstroke}%
\pgfsetdash{}{0pt}%
\pgfpathmoveto{\pgfqpoint{1.750754in}{1.296613in}}%
\pgfpathlineto{\pgfqpoint{1.764326in}{1.329313in}}%
\pgfpathlineto{\pgfqpoint{1.789425in}{1.348315in}}%
\pgfpathlineto{\pgfqpoint{1.774359in}{1.314526in}}%
\pgfpathlineto{\pgfqpoint{1.750754in}{1.296613in}}%
\pgfpathclose%
\pgfusepath{fill}%
\end{pgfscope}%
\begin{pgfscope}%
\pgfpathrectangle{\pgfqpoint{0.000000in}{0.000000in}}{\pgfqpoint{3.000000in}{3.000000in}}%
\pgfusepath{clip}%
\pgfsetbuttcap%
\pgfsetroundjoin%
\definecolor{currentfill}{rgb}{0.000000,0.503922,1.000000}%
\pgfsetfillcolor{currentfill}%
\pgfsetlinewidth{0.000000pt}%
\definecolor{currentstroke}{rgb}{0.000000,0.000000,0.000000}%
\pgfsetstrokecolor{currentstroke}%
\pgfsetdash{}{0pt}%
\pgfpathmoveto{\pgfqpoint{1.636741in}{1.185154in}}%
\pgfpathlineto{\pgfqpoint{1.643868in}{1.223217in}}%
\pgfpathlineto{\pgfqpoint{1.678761in}{1.233040in}}%
\pgfpathlineto{\pgfqpoint{1.669235in}{1.194336in}}%
\pgfpathlineto{\pgfqpoint{1.636741in}{1.185154in}}%
\pgfpathclose%
\pgfusepath{fill}%
\end{pgfscope}%
\begin{pgfscope}%
\pgfpathrectangle{\pgfqpoint{0.000000in}{0.000000in}}{\pgfqpoint{3.000000in}{3.000000in}}%
\pgfusepath{clip}%
\pgfsetbuttcap%
\pgfsetroundjoin%
\definecolor{currentfill}{rgb}{0.999109,0.073348,0.000000}%
\pgfsetfillcolor{currentfill}%
\pgfsetlinewidth{0.000000pt}%
\definecolor{currentstroke}{rgb}{0.000000,0.000000,0.000000}%
\pgfsetstrokecolor{currentstroke}%
\pgfsetdash{}{0pt}%
\pgfpathmoveto{\pgfqpoint{0.895284in}{2.051970in}}%
\pgfpathlineto{\pgfqpoint{0.879426in}{2.069236in}}%
\pgfpathlineto{\pgfqpoint{0.843902in}{2.018684in}}%
\pgfpathlineto{\pgfqpoint{0.860705in}{2.002541in}}%
\pgfpathlineto{\pgfqpoint{0.895284in}{2.051970in}}%
\pgfpathclose%
\pgfusepath{fill}%
\end{pgfscope}%
\begin{pgfscope}%
\pgfpathrectangle{\pgfqpoint{0.000000in}{0.000000in}}{\pgfqpoint{3.000000in}{3.000000in}}%
\pgfusepath{clip}%
\pgfsetbuttcap%
\pgfsetroundjoin%
\definecolor{currentfill}{rgb}{0.819102,1.000000,0.148640}%
\pgfsetfillcolor{currentfill}%
\pgfsetlinewidth{0.000000pt}%
\definecolor{currentstroke}{rgb}{0.000000,0.000000,0.000000}%
\pgfsetstrokecolor{currentstroke}%
\pgfsetdash{}{0pt}%
\pgfpathmoveto{\pgfqpoint{1.963269in}{1.636792in}}%
\pgfpathlineto{\pgfqpoint{1.980547in}{1.658681in}}%
\pgfpathlineto{\pgfqpoint{1.977126in}{1.692677in}}%
\pgfpathlineto{\pgfqpoint{1.960042in}{1.669536in}}%
\pgfpathlineto{\pgfqpoint{1.963269in}{1.636792in}}%
\pgfpathclose%
\pgfusepath{fill}%
\end{pgfscope}%
\begin{pgfscope}%
\pgfpathrectangle{\pgfqpoint{0.000000in}{0.000000in}}{\pgfqpoint{3.000000in}{3.000000in}}%
\pgfusepath{clip}%
\pgfsetbuttcap%
\pgfsetroundjoin%
\definecolor{currentfill}{rgb}{0.300443,1.000000,0.667299}%
\pgfsetfillcolor{currentfill}%
\pgfsetlinewidth{0.000000pt}%
\definecolor{currentstroke}{rgb}{0.000000,0.000000,0.000000}%
\pgfsetstrokecolor{currentstroke}%
\pgfsetdash{}{0pt}%
\pgfpathmoveto{\pgfqpoint{1.840674in}{1.430645in}}%
\pgfpathlineto{\pgfqpoint{1.856917in}{1.458097in}}%
\pgfpathlineto{\pgfqpoint{1.871120in}{1.483745in}}%
\pgfpathlineto{\pgfqpoint{1.854191in}{1.455051in}}%
\pgfpathlineto{\pgfqpoint{1.840674in}{1.430645in}}%
\pgfpathclose%
\pgfusepath{fill}%
\end{pgfscope}%
\begin{pgfscope}%
\pgfpathrectangle{\pgfqpoint{0.000000in}{0.000000in}}{\pgfqpoint{3.000000in}{3.000000in}}%
\pgfusepath{clip}%
\pgfsetbuttcap%
\pgfsetroundjoin%
\definecolor{currentfill}{rgb}{1.000000,0.814089,0.000000}%
\pgfsetfillcolor{currentfill}%
\pgfsetlinewidth{0.000000pt}%
\definecolor{currentstroke}{rgb}{0.000000,0.000000,0.000000}%
\pgfsetstrokecolor{currentstroke}%
\pgfsetdash{}{0pt}%
\pgfpathmoveto{\pgfqpoint{1.061980in}{1.781904in}}%
\pgfpathlineto{\pgfqpoint{1.045235in}{1.802834in}}%
\pgfpathlineto{\pgfqpoint{1.032727in}{1.764386in}}%
\pgfpathlineto{\pgfqpoint{1.049965in}{1.744676in}}%
\pgfpathlineto{\pgfqpoint{1.061980in}{1.781904in}}%
\pgfpathclose%
\pgfusepath{fill}%
\end{pgfscope}%
\begin{pgfscope}%
\pgfpathrectangle{\pgfqpoint{0.000000in}{0.000000in}}{\pgfqpoint{3.000000in}{3.000000in}}%
\pgfusepath{clip}%
\pgfsetbuttcap%
\pgfsetroundjoin%
\definecolor{currentfill}{rgb}{0.000000,0.676471,1.000000}%
\pgfsetfillcolor{currentfill}%
\pgfsetlinewidth{0.000000pt}%
\definecolor{currentstroke}{rgb}{0.000000,0.000000,0.000000}%
\pgfsetstrokecolor{currentstroke}%
\pgfsetdash{}{0pt}%
\pgfpathmoveto{\pgfqpoint{1.352375in}{1.255209in}}%
\pgfpathlineto{\pgfqpoint{1.339408in}{1.291085in}}%
\pgfpathlineto{\pgfqpoint{1.369948in}{1.276055in}}%
\pgfpathlineto{\pgfqpoint{1.380962in}{1.241097in}}%
\pgfpathlineto{\pgfqpoint{1.352375in}{1.255209in}}%
\pgfpathclose%
\pgfusepath{fill}%
\end{pgfscope}%
\begin{pgfscope}%
\pgfpathrectangle{\pgfqpoint{0.000000in}{0.000000in}}{\pgfqpoint{3.000000in}{3.000000in}}%
\pgfusepath{clip}%
\pgfsetbuttcap%
\pgfsetroundjoin%
\definecolor{currentfill}{rgb}{0.927807,0.015251,0.000000}%
\pgfsetfillcolor{currentfill}%
\pgfsetlinewidth{0.000000pt}%
\definecolor{currentstroke}{rgb}{0.000000,0.000000,0.000000}%
\pgfsetstrokecolor{currentstroke}%
\pgfsetdash{}{0pt}%
\pgfpathmoveto{\pgfqpoint{0.879426in}{2.069236in}}%
\pgfpathlineto{\pgfqpoint{0.863569in}{2.086253in}}%
\pgfpathlineto{\pgfqpoint{0.827095in}{2.034581in}}%
\pgfpathlineto{\pgfqpoint{0.843902in}{2.018684in}}%
\pgfpathlineto{\pgfqpoint{0.879426in}{2.069236in}}%
\pgfpathclose%
\pgfusepath{fill}%
\end{pgfscope}%
\begin{pgfscope}%
\pgfpathrectangle{\pgfqpoint{0.000000in}{0.000000in}}{\pgfqpoint{3.000000in}{3.000000in}}%
\pgfusepath{clip}%
\pgfsetbuttcap%
\pgfsetroundjoin%
\definecolor{currentfill}{rgb}{1.000000,0.610748,0.000000}%
\pgfsetfillcolor{currentfill}%
\pgfsetlinewidth{0.000000pt}%
\definecolor{currentstroke}{rgb}{0.000000,0.000000,0.000000}%
\pgfsetstrokecolor{currentstroke}%
\pgfsetdash{}{0pt}%
\pgfpathmoveto{\pgfqpoint{2.079815in}{1.815841in}}%
\pgfpathlineto{\pgfqpoint{2.096960in}{1.834332in}}%
\pgfpathlineto{\pgfqpoint{2.078481in}{1.875947in}}%
\pgfpathlineto{\pgfqpoint{2.061979in}{1.856262in}}%
\pgfpathlineto{\pgfqpoint{2.079815in}{1.815841in}}%
\pgfpathclose%
\pgfusepath{fill}%
\end{pgfscope}%
\begin{pgfscope}%
\pgfpathrectangle{\pgfqpoint{0.000000in}{0.000000in}}{\pgfqpoint{3.000000in}{3.000000in}}%
\pgfusepath{clip}%
\pgfsetbuttcap%
\pgfsetroundjoin%
\definecolor{currentfill}{rgb}{0.000000,0.503922,1.000000}%
\pgfsetfillcolor{currentfill}%
\pgfsetlinewidth{0.000000pt}%
\definecolor{currentstroke}{rgb}{0.000000,0.000000,0.000000}%
\pgfsetstrokecolor{currentstroke}%
\pgfsetdash{}{0pt}%
\pgfpathmoveto{\pgfqpoint{1.422353in}{1.190983in}}%
\pgfpathlineto{\pgfqpoint{1.413602in}{1.229454in}}%
\pgfpathlineto{\pgfqpoint{1.449475in}{1.220585in}}%
\pgfpathlineto{\pgfqpoint{1.455759in}{1.182694in}}%
\pgfpathlineto{\pgfqpoint{1.422353in}{1.190983in}}%
\pgfpathclose%
\pgfusepath{fill}%
\end{pgfscope}%
\begin{pgfscope}%
\pgfpathrectangle{\pgfqpoint{0.000000in}{0.000000in}}{\pgfqpoint{3.000000in}{3.000000in}}%
\pgfusepath{clip}%
\pgfsetbuttcap%
\pgfsetroundjoin%
\definecolor{currentfill}{rgb}{0.085389,1.000000,0.882353}%
\pgfsetfillcolor{currentfill}%
\pgfsetlinewidth{0.000000pt}%
\definecolor{currentstroke}{rgb}{0.000000,0.000000,0.000000}%
\pgfsetstrokecolor{currentstroke}%
\pgfsetdash{}{0pt}%
\pgfpathmoveto{\pgfqpoint{1.278375in}{1.361972in}}%
\pgfpathlineto{\pgfqpoint{1.262502in}{1.393610in}}%
\pgfpathlineto{\pgfqpoint{1.284717in}{1.372247in}}%
\pgfpathlineto{\pgfqpoint{1.299353in}{1.341764in}}%
\pgfpathlineto{\pgfqpoint{1.278375in}{1.361972in}}%
\pgfpathclose%
\pgfusepath{fill}%
\end{pgfscope}%
\begin{pgfscope}%
\pgfpathrectangle{\pgfqpoint{0.000000in}{0.000000in}}{\pgfqpoint{3.000000in}{3.000000in}}%
\pgfusepath{clip}%
\pgfsetbuttcap%
\pgfsetroundjoin%
\definecolor{currentfill}{rgb}{0.856506,0.000000,0.000000}%
\pgfsetfillcolor{currentfill}%
\pgfsetlinewidth{0.000000pt}%
\definecolor{currentstroke}{rgb}{0.000000,0.000000,0.000000}%
\pgfsetstrokecolor{currentstroke}%
\pgfsetdash{}{0pt}%
\pgfpathmoveto{\pgfqpoint{0.863569in}{2.086253in}}%
\pgfpathlineto{\pgfqpoint{0.847712in}{2.103034in}}%
\pgfpathlineto{\pgfqpoint{0.810283in}{2.050245in}}%
\pgfpathlineto{\pgfqpoint{0.827095in}{2.034581in}}%
\pgfpathlineto{\pgfqpoint{0.863569in}{2.086253in}}%
\pgfpathclose%
\pgfusepath{fill}%
\end{pgfscope}%
\begin{pgfscope}%
\pgfpathrectangle{\pgfqpoint{0.000000in}{0.000000in}}{\pgfqpoint{3.000000in}{3.000000in}}%
\pgfusepath{clip}%
\pgfsetbuttcap%
\pgfsetroundjoin%
\definecolor{currentfill}{rgb}{0.743201,1.000000,0.224541}%
\pgfsetfillcolor{currentfill}%
\pgfsetlinewidth{0.000000pt}%
\definecolor{currentstroke}{rgb}{0.000000,0.000000,0.000000}%
\pgfsetstrokecolor{currentstroke}%
\pgfsetdash{}{0pt}%
\pgfpathmoveto{\pgfqpoint{1.135994in}{1.634969in}}%
\pgfpathlineto{\pgfqpoint{1.118810in}{1.658667in}}%
\pgfpathlineto{\pgfqpoint{1.119054in}{1.625846in}}%
\pgfpathlineto{\pgfqpoint{1.136286in}{1.603410in}}%
\pgfpathlineto{\pgfqpoint{1.135994in}{1.634969in}}%
\pgfpathclose%
\pgfusepath{fill}%
\end{pgfscope}%
\begin{pgfscope}%
\pgfpathrectangle{\pgfqpoint{0.000000in}{0.000000in}}{\pgfqpoint{3.000000in}{3.000000in}}%
\pgfusepath{clip}%
\pgfsetbuttcap%
\pgfsetroundjoin%
\definecolor{currentfill}{rgb}{0.000000,0.849020,1.000000}%
\pgfsetfillcolor{currentfill}%
\pgfsetlinewidth{0.000000pt}%
\definecolor{currentstroke}{rgb}{0.000000,0.000000,0.000000}%
\pgfsetstrokecolor{currentstroke}%
\pgfsetdash{}{0pt}%
\pgfpathmoveto{\pgfqpoint{1.313962in}{1.308350in}}%
\pgfpathlineto{\pgfqpoint{1.299353in}{1.341764in}}%
\pgfpathlineto{\pgfqpoint{1.326415in}{1.323448in}}%
\pgfpathlineto{\pgfqpoint{1.339408in}{1.291085in}}%
\pgfpathlineto{\pgfqpoint{1.313962in}{1.308350in}}%
\pgfpathclose%
\pgfusepath{fill}%
\end{pgfscope}%
\begin{pgfscope}%
\pgfpathrectangle{\pgfqpoint{0.000000in}{0.000000in}}{\pgfqpoint{3.000000in}{3.000000in}}%
\pgfusepath{clip}%
\pgfsetbuttcap%
\pgfsetroundjoin%
\definecolor{currentfill}{rgb}{0.803030,0.000000,0.000000}%
\pgfsetfillcolor{currentfill}%
\pgfsetlinewidth{0.000000pt}%
\definecolor{currentstroke}{rgb}{0.000000,0.000000,0.000000}%
\pgfsetstrokecolor{currentstroke}%
\pgfsetdash{}{0pt}%
\pgfpathmoveto{\pgfqpoint{0.847712in}{2.103034in}}%
\pgfpathlineto{\pgfqpoint{0.831855in}{2.119592in}}%
\pgfpathlineto{\pgfqpoint{0.793467in}{2.065689in}}%
\pgfpathlineto{\pgfqpoint{0.810283in}{2.050245in}}%
\pgfpathlineto{\pgfqpoint{0.847712in}{2.103034in}}%
\pgfpathclose%
\pgfusepath{fill}%
\end{pgfscope}%
\begin{pgfscope}%
\pgfpathrectangle{\pgfqpoint{0.000000in}{0.000000in}}{\pgfqpoint{3.000000in}{3.000000in}}%
\pgfusepath{clip}%
\pgfsetbuttcap%
\pgfsetroundjoin%
\definecolor{currentfill}{rgb}{1.000000,0.538126,0.000000}%
\pgfsetfillcolor{currentfill}%
\pgfsetlinewidth{0.000000pt}%
\definecolor{currentstroke}{rgb}{0.000000,0.000000,0.000000}%
\pgfsetstrokecolor{currentstroke}%
\pgfsetdash{}{0pt}%
\pgfpathmoveto{\pgfqpoint{2.096960in}{1.834332in}}%
\pgfpathlineto{\pgfqpoint{2.114114in}{1.852365in}}%
\pgfpathlineto{\pgfqpoint{2.094987in}{1.895171in}}%
\pgfpathlineto{\pgfqpoint{2.078481in}{1.875947in}}%
\pgfpathlineto{\pgfqpoint{2.096960in}{1.834332in}}%
\pgfpathclose%
\pgfusepath{fill}%
\end{pgfscope}%
\begin{pgfscope}%
\pgfpathrectangle{\pgfqpoint{0.000000in}{0.000000in}}{\pgfqpoint{3.000000in}{3.000000in}}%
\pgfusepath{clip}%
\pgfsetbuttcap%
\pgfsetroundjoin%
\definecolor{currentfill}{rgb}{1.000000,0.741467,0.000000}%
\pgfsetfillcolor{currentfill}%
\pgfsetlinewidth{0.000000pt}%
\definecolor{currentstroke}{rgb}{0.000000,0.000000,0.000000}%
\pgfsetstrokecolor{currentstroke}%
\pgfsetdash{}{0pt}%
\pgfpathmoveto{\pgfqpoint{1.045235in}{1.802834in}}%
\pgfpathlineto{\pgfqpoint{1.028484in}{1.823171in}}%
\pgfpathlineto{\pgfqpoint{1.015478in}{1.783506in}}%
\pgfpathlineto{\pgfqpoint{1.032727in}{1.764386in}}%
\pgfpathlineto{\pgfqpoint{1.045235in}{1.802834in}}%
\pgfpathclose%
\pgfusepath{fill}%
\end{pgfscope}%
\begin{pgfscope}%
\pgfpathrectangle{\pgfqpoint{0.000000in}{0.000000in}}{\pgfqpoint{3.000000in}{3.000000in}}%
\pgfusepath{clip}%
\pgfsetbuttcap%
\pgfsetroundjoin%
\definecolor{currentfill}{rgb}{0.490196,1.000000,0.477546}%
\pgfsetfillcolor{currentfill}%
\pgfsetlinewidth{0.000000pt}%
\definecolor{currentstroke}{rgb}{0.000000,0.000000,0.000000}%
\pgfsetstrokecolor{currentstroke}%
\pgfsetdash{}{0pt}%
\pgfpathmoveto{\pgfqpoint{1.187886in}{1.529007in}}%
\pgfpathlineto{\pgfqpoint{1.170703in}{1.555141in}}%
\pgfpathlineto{\pgfqpoint{1.180233in}{1.526409in}}%
\pgfpathlineto{\pgfqpoint{1.197023in}{1.501537in}}%
\pgfpathlineto{\pgfqpoint{1.187886in}{1.529007in}}%
\pgfpathclose%
\pgfusepath{fill}%
\end{pgfscope}%
\begin{pgfscope}%
\pgfpathrectangle{\pgfqpoint{0.000000in}{0.000000in}}{\pgfqpoint{3.000000in}{3.000000in}}%
\pgfusepath{clip}%
\pgfsetbuttcap%
\pgfsetroundjoin%
\definecolor{currentfill}{rgb}{0.578748,1.000000,0.388994}%
\pgfsetfillcolor{currentfill}%
\pgfsetlinewidth{0.000000pt}%
\definecolor{currentstroke}{rgb}{0.000000,0.000000,0.000000}%
\pgfsetstrokecolor{currentstroke}%
\pgfsetdash{}{0pt}%
\pgfpathmoveto{\pgfqpoint{1.905038in}{1.535908in}}%
\pgfpathlineto{\pgfqpoint{1.922026in}{1.559812in}}%
\pgfpathlineto{\pgfqpoint{1.928756in}{1.589989in}}%
\pgfpathlineto{\pgfqpoint{1.911521in}{1.564820in}}%
\pgfpathlineto{\pgfqpoint{1.905038in}{1.535908in}}%
\pgfpathclose%
\pgfusepath{fill}%
\end{pgfscope}%
\begin{pgfscope}%
\pgfpathrectangle{\pgfqpoint{0.000000in}{0.000000in}}{\pgfqpoint{3.000000in}{3.000000in}}%
\pgfusepath{clip}%
\pgfsetbuttcap%
\pgfsetroundjoin%
\definecolor{currentfill}{rgb}{0.000000,0.503922,1.000000}%
\pgfsetfillcolor{currentfill}%
\pgfsetlinewidth{0.000000pt}%
\definecolor{currentstroke}{rgb}{0.000000,0.000000,0.000000}%
\pgfsetstrokecolor{currentstroke}%
\pgfsetdash{}{0pt}%
\pgfpathmoveto{\pgfqpoint{1.601796in}{1.178720in}}%
\pgfpathlineto{\pgfqpoint{1.606338in}{1.216333in}}%
\pgfpathlineto{\pgfqpoint{1.643868in}{1.223217in}}%
\pgfpathlineto{\pgfqpoint{1.636741in}{1.185154in}}%
\pgfpathlineto{\pgfqpoint{1.601796in}{1.178720in}}%
\pgfpathclose%
\pgfusepath{fill}%
\end{pgfscope}%
\begin{pgfscope}%
\pgfpathrectangle{\pgfqpoint{0.000000in}{0.000000in}}{\pgfqpoint{3.000000in}{3.000000in}}%
\pgfusepath{clip}%
\pgfsetbuttcap%
\pgfsetroundjoin%
\definecolor{currentfill}{rgb}{0.895003,1.000000,0.072739}%
\pgfsetfillcolor{currentfill}%
\pgfsetlinewidth{0.000000pt}%
\definecolor{currentstroke}{rgb}{0.000000,0.000000,0.000000}%
\pgfsetstrokecolor{currentstroke}%
\pgfsetdash{}{0pt}%
\pgfpathmoveto{\pgfqpoint{1.980547in}{1.658681in}}%
\pgfpathlineto{\pgfqpoint{1.997840in}{1.679699in}}%
\pgfpathlineto{\pgfqpoint{1.994218in}{1.714946in}}%
\pgfpathlineto{\pgfqpoint{1.977126in}{1.692677in}}%
\pgfpathlineto{\pgfqpoint{1.980547in}{1.658681in}}%
\pgfpathclose%
\pgfusepath{fill}%
\end{pgfscope}%
\begin{pgfscope}%
\pgfpathrectangle{\pgfqpoint{0.000000in}{0.000000in}}{\pgfqpoint{3.000000in}{3.000000in}}%
\pgfusepath{clip}%
\pgfsetbuttcap%
\pgfsetroundjoin%
\definecolor{currentfill}{rgb}{0.731729,0.000000,0.000000}%
\pgfsetfillcolor{currentfill}%
\pgfsetlinewidth{0.000000pt}%
\definecolor{currentstroke}{rgb}{0.000000,0.000000,0.000000}%
\pgfsetstrokecolor{currentstroke}%
\pgfsetdash{}{0pt}%
\pgfpathmoveto{\pgfqpoint{0.831855in}{2.119592in}}%
\pgfpathlineto{\pgfqpoint{0.815999in}{2.135939in}}%
\pgfpathlineto{\pgfqpoint{0.776647in}{2.080925in}}%
\pgfpathlineto{\pgfqpoint{0.793467in}{2.065689in}}%
\pgfpathlineto{\pgfqpoint{0.831855in}{2.119592in}}%
\pgfpathclose%
\pgfusepath{fill}%
\end{pgfscope}%
\begin{pgfscope}%
\pgfpathrectangle{\pgfqpoint{0.000000in}{0.000000in}}{\pgfqpoint{3.000000in}{3.000000in}}%
\pgfusepath{clip}%
\pgfsetbuttcap%
\pgfsetroundjoin%
\definecolor{currentfill}{rgb}{0.000000,0.503922,1.000000}%
\pgfsetfillcolor{currentfill}%
\pgfsetlinewidth{0.000000pt}%
\definecolor{currentstroke}{rgb}{0.000000,0.000000,0.000000}%
\pgfsetstrokecolor{currentstroke}%
\pgfsetdash{}{0pt}%
\pgfpathmoveto{\pgfqpoint{1.455759in}{1.182694in}}%
\pgfpathlineto{\pgfqpoint{1.449475in}{1.220585in}}%
\pgfpathlineto{\pgfqpoint{1.487676in}{1.214726in}}%
\pgfpathlineto{\pgfqpoint{1.491326in}{1.177218in}}%
\pgfpathlineto{\pgfqpoint{1.455759in}{1.182694in}}%
\pgfpathclose%
\pgfusepath{fill}%
\end{pgfscope}%
\begin{pgfscope}%
\pgfpathrectangle{\pgfqpoint{0.000000in}{0.000000in}}{\pgfqpoint{3.000000in}{3.000000in}}%
\pgfusepath{clip}%
\pgfsetbuttcap%
\pgfsetroundjoin%
\definecolor{currentfill}{rgb}{0.300443,1.000000,0.667299}%
\pgfsetfillcolor{currentfill}%
\pgfsetlinewidth{0.000000pt}%
\definecolor{currentstroke}{rgb}{0.000000,0.000000,0.000000}%
\pgfsetstrokecolor{currentstroke}%
\pgfsetdash{}{0pt}%
\pgfpathmoveto{\pgfqpoint{1.230539in}{1.446807in}}%
\pgfpathlineto{\pgfqpoint{1.213791in}{1.475083in}}%
\pgfpathlineto{\pgfqpoint{1.230682in}{1.449822in}}%
\pgfpathlineto{\pgfqpoint{1.246604in}{1.422772in}}%
\pgfpathlineto{\pgfqpoint{1.230539in}{1.446807in}}%
\pgfpathclose%
\pgfusepath{fill}%
\end{pgfscope}%
\begin{pgfscope}%
\pgfpathrectangle{\pgfqpoint{0.000000in}{0.000000in}}{\pgfqpoint{3.000000in}{3.000000in}}%
\pgfusepath{clip}%
\pgfsetbuttcap%
\pgfsetroundjoin%
\definecolor{currentfill}{rgb}{0.678253,0.000000,0.000000}%
\pgfsetfillcolor{currentfill}%
\pgfsetlinewidth{0.000000pt}%
\definecolor{currentstroke}{rgb}{0.000000,0.000000,0.000000}%
\pgfsetstrokecolor{currentstroke}%
\pgfsetdash{}{0pt}%
\pgfpathmoveto{\pgfqpoint{0.815999in}{2.135939in}}%
\pgfpathlineto{\pgfqpoint{0.800144in}{2.152087in}}%
\pgfpathlineto{\pgfqpoint{0.759823in}{2.095965in}}%
\pgfpathlineto{\pgfqpoint{0.776647in}{2.080925in}}%
\pgfpathlineto{\pgfqpoint{0.815999in}{2.135939in}}%
\pgfpathclose%
\pgfusepath{fill}%
\end{pgfscope}%
\begin{pgfscope}%
\pgfpathrectangle{\pgfqpoint{0.000000in}{0.000000in}}{\pgfqpoint{3.000000in}{3.000000in}}%
\pgfusepath{clip}%
\pgfsetbuttcap%
\pgfsetroundjoin%
\definecolor{currentfill}{rgb}{1.000000,0.480029,0.000000}%
\pgfsetfillcolor{currentfill}%
\pgfsetlinewidth{0.000000pt}%
\definecolor{currentstroke}{rgb}{0.000000,0.000000,0.000000}%
\pgfsetstrokecolor{currentstroke}%
\pgfsetdash{}{0pt}%
\pgfpathmoveto{\pgfqpoint{2.114114in}{1.852365in}}%
\pgfpathlineto{\pgfqpoint{2.131276in}{1.869973in}}%
\pgfpathlineto{\pgfqpoint{2.111496in}{1.913969in}}%
\pgfpathlineto{\pgfqpoint{2.094987in}{1.895171in}}%
\pgfpathlineto{\pgfqpoint{2.114114in}{1.852365in}}%
\pgfpathclose%
\pgfusepath{fill}%
\end{pgfscope}%
\begin{pgfscope}%
\pgfpathrectangle{\pgfqpoint{0.000000in}{0.000000in}}{\pgfqpoint{3.000000in}{3.000000in}}%
\pgfusepath{clip}%
\pgfsetbuttcap%
\pgfsetroundjoin%
\definecolor{currentfill}{rgb}{0.000000,0.676471,1.000000}%
\pgfsetfillcolor{currentfill}%
\pgfsetlinewidth{0.000000pt}%
\definecolor{currentstroke}{rgb}{0.000000,0.000000,0.000000}%
\pgfsetstrokecolor{currentstroke}%
\pgfsetdash{}{0pt}%
\pgfpathmoveto{\pgfqpoint{1.678761in}{1.233040in}}%
\pgfpathlineto{\pgfqpoint{1.688309in}{1.267473in}}%
\pgfpathlineto{\pgfqpoint{1.721835in}{1.280792in}}%
\pgfpathlineto{\pgfqpoint{1.710135in}{1.245544in}}%
\pgfpathlineto{\pgfqpoint{1.678761in}{1.233040in}}%
\pgfpathclose%
\pgfusepath{fill}%
\end{pgfscope}%
\begin{pgfscope}%
\pgfpathrectangle{\pgfqpoint{0.000000in}{0.000000in}}{\pgfqpoint{3.000000in}{3.000000in}}%
\pgfusepath{clip}%
\pgfsetbuttcap%
\pgfsetroundjoin%
\definecolor{currentfill}{rgb}{0.199241,1.000000,0.768501}%
\pgfsetfillcolor{currentfill}%
\pgfsetlinewidth{0.000000pt}%
\definecolor{currentstroke}{rgb}{0.000000,0.000000,0.000000}%
\pgfsetstrokecolor{currentstroke}%
\pgfsetdash{}{0pt}%
\pgfpathmoveto{\pgfqpoint{1.804516in}{1.379173in}}%
\pgfpathlineto{\pgfqpoint{1.819634in}{1.407555in}}%
\pgfpathlineto{\pgfqpoint{1.840674in}{1.430645in}}%
\pgfpathlineto{\pgfqpoint{1.824454in}{1.401080in}}%
\pgfpathlineto{\pgfqpoint{1.804516in}{1.379173in}}%
\pgfpathclose%
\pgfusepath{fill}%
\end{pgfscope}%
\begin{pgfscope}%
\pgfpathrectangle{\pgfqpoint{0.000000in}{0.000000in}}{\pgfqpoint{3.000000in}{3.000000in}}%
\pgfusepath{clip}%
\pgfsetbuttcap%
\pgfsetroundjoin%
\definecolor{currentfill}{rgb}{1.000000,0.668845,0.000000}%
\pgfsetfillcolor{currentfill}%
\pgfsetlinewidth{0.000000pt}%
\definecolor{currentstroke}{rgb}{0.000000,0.000000,0.000000}%
\pgfsetstrokecolor{currentstroke}%
\pgfsetdash{}{0pt}%
\pgfpathmoveto{\pgfqpoint{1.028484in}{1.823171in}}%
\pgfpathlineto{\pgfqpoint{1.011728in}{1.842965in}}%
\pgfpathlineto{\pgfqpoint{0.998219in}{1.802086in}}%
\pgfpathlineto{\pgfqpoint{1.015478in}{1.783506in}}%
\pgfpathlineto{\pgfqpoint{1.028484in}{1.823171in}}%
\pgfpathclose%
\pgfusepath{fill}%
\end{pgfscope}%
\begin{pgfscope}%
\pgfpathrectangle{\pgfqpoint{0.000000in}{0.000000in}}{\pgfqpoint{3.000000in}{3.000000in}}%
\pgfusepath{clip}%
\pgfsetbuttcap%
\pgfsetroundjoin%
\definecolor{currentfill}{rgb}{0.819102,1.000000,0.148640}%
\pgfsetfillcolor{currentfill}%
\pgfsetlinewidth{0.000000pt}%
\definecolor{currentstroke}{rgb}{0.000000,0.000000,0.000000}%
\pgfsetstrokecolor{currentstroke}%
\pgfsetdash{}{0pt}%
\pgfpathmoveto{\pgfqpoint{1.118810in}{1.658667in}}%
\pgfpathlineto{\pgfqpoint{1.101615in}{1.681394in}}%
\pgfpathlineto{\pgfqpoint{1.101805in}{1.647314in}}%
\pgfpathlineto{\pgfqpoint{1.119054in}{1.625846in}}%
\pgfpathlineto{\pgfqpoint{1.118810in}{1.658667in}}%
\pgfpathclose%
\pgfusepath{fill}%
\end{pgfscope}%
\begin{pgfscope}%
\pgfpathrectangle{\pgfqpoint{0.000000in}{0.000000in}}{\pgfqpoint{3.000000in}{3.000000in}}%
\pgfusepath{clip}%
\pgfsetbuttcap%
\pgfsetroundjoin%
\definecolor{currentfill}{rgb}{0.606952,0.000000,0.000000}%
\pgfsetfillcolor{currentfill}%
\pgfsetlinewidth{0.000000pt}%
\definecolor{currentstroke}{rgb}{0.000000,0.000000,0.000000}%
\pgfsetstrokecolor{currentstroke}%
\pgfsetdash{}{0pt}%
\pgfpathmoveto{\pgfqpoint{0.800144in}{2.152087in}}%
\pgfpathlineto{\pgfqpoint{0.784289in}{2.168045in}}%
\pgfpathlineto{\pgfqpoint{0.742996in}{2.110818in}}%
\pgfpathlineto{\pgfqpoint{0.759823in}{2.095965in}}%
\pgfpathlineto{\pgfqpoint{0.800144in}{2.152087in}}%
\pgfpathclose%
\pgfusepath{fill}%
\end{pgfscope}%
\begin{pgfscope}%
\pgfpathrectangle{\pgfqpoint{0.000000in}{0.000000in}}{\pgfqpoint{3.000000in}{3.000000in}}%
\pgfusepath{clip}%
\pgfsetbuttcap%
\pgfsetroundjoin%
\definecolor{currentfill}{rgb}{0.000000,0.503922,1.000000}%
\pgfsetfillcolor{currentfill}%
\pgfsetlinewidth{0.000000pt}%
\definecolor{currentstroke}{rgb}{0.000000,0.000000,0.000000}%
\pgfsetstrokecolor{currentstroke}%
\pgfsetdash{}{0pt}%
\pgfpathmoveto{\pgfqpoint{1.565289in}{1.175204in}}%
\pgfpathlineto{\pgfqpoint{1.567124in}{1.212571in}}%
\pgfpathlineto{\pgfqpoint{1.606338in}{1.216333in}}%
\pgfpathlineto{\pgfqpoint{1.601796in}{1.178720in}}%
\pgfpathlineto{\pgfqpoint{1.565289in}{1.175204in}}%
\pgfpathclose%
\pgfusepath{fill}%
\end{pgfscope}%
\begin{pgfscope}%
\pgfpathrectangle{\pgfqpoint{0.000000in}{0.000000in}}{\pgfqpoint{3.000000in}{3.000000in}}%
\pgfusepath{clip}%
\pgfsetbuttcap%
\pgfsetroundjoin%
\definecolor{currentfill}{rgb}{0.000000,0.503922,1.000000}%
\pgfsetfillcolor{currentfill}%
\pgfsetlinewidth{0.000000pt}%
\definecolor{currentstroke}{rgb}{0.000000,0.000000,0.000000}%
\pgfsetstrokecolor{currentstroke}%
\pgfsetdash{}{0pt}%
\pgfpathmoveto{\pgfqpoint{1.491326in}{1.177218in}}%
\pgfpathlineto{\pgfqpoint{1.487676in}{1.214726in}}%
\pgfpathlineto{\pgfqpoint{1.527230in}{1.212031in}}%
\pgfpathlineto{\pgfqpoint{1.528149in}{1.174699in}}%
\pgfpathlineto{\pgfqpoint{1.491326in}{1.177218in}}%
\pgfpathclose%
\pgfusepath{fill}%
\end{pgfscope}%
\begin{pgfscope}%
\pgfpathrectangle{\pgfqpoint{0.000000in}{0.000000in}}{\pgfqpoint{3.000000in}{3.000000in}}%
\pgfusepath{clip}%
\pgfsetbuttcap%
\pgfsetroundjoin%
\definecolor{currentfill}{rgb}{0.401645,1.000000,0.566097}%
\pgfsetfillcolor{currentfill}%
\pgfsetlinewidth{0.000000pt}%
\definecolor{currentstroke}{rgb}{0.000000,0.000000,0.000000}%
\pgfsetstrokecolor{currentstroke}%
\pgfsetdash{}{0pt}%
\pgfpathmoveto{\pgfqpoint{1.856917in}{1.458097in}}%
\pgfpathlineto{\pgfqpoint{1.873184in}{1.483728in}}%
\pgfpathlineto{\pgfqpoint{1.888069in}{1.510618in}}%
\pgfpathlineto{\pgfqpoint{1.871120in}{1.483745in}}%
\pgfpathlineto{\pgfqpoint{1.856917in}{1.458097in}}%
\pgfpathclose%
\pgfusepath{fill}%
\end{pgfscope}%
\begin{pgfscope}%
\pgfpathrectangle{\pgfqpoint{0.000000in}{0.000000in}}{\pgfqpoint{3.000000in}{3.000000in}}%
\pgfusepath{clip}%
\pgfsetbuttcap%
\pgfsetroundjoin%
\definecolor{currentfill}{rgb}{0.553476,0.000000,0.000000}%
\pgfsetfillcolor{currentfill}%
\pgfsetlinewidth{0.000000pt}%
\definecolor{currentstroke}{rgb}{0.000000,0.000000,0.000000}%
\pgfsetstrokecolor{currentstroke}%
\pgfsetdash{}{0pt}%
\pgfpathmoveto{\pgfqpoint{0.784289in}{2.168045in}}%
\pgfpathlineto{\pgfqpoint{0.768434in}{2.183824in}}%
\pgfpathlineto{\pgfqpoint{0.726164in}{2.125494in}}%
\pgfpathlineto{\pgfqpoint{0.742996in}{2.110818in}}%
\pgfpathlineto{\pgfqpoint{0.784289in}{2.168045in}}%
\pgfpathclose%
\pgfusepath{fill}%
\end{pgfscope}%
\begin{pgfscope}%
\pgfpathrectangle{\pgfqpoint{0.000000in}{0.000000in}}{\pgfqpoint{3.000000in}{3.000000in}}%
\pgfusepath{clip}%
\pgfsetbuttcap%
\pgfsetroundjoin%
\definecolor{currentfill}{rgb}{1.000000,0.407407,0.000000}%
\pgfsetfillcolor{currentfill}%
\pgfsetlinewidth{0.000000pt}%
\definecolor{currentstroke}{rgb}{0.000000,0.000000,0.000000}%
\pgfsetstrokecolor{currentstroke}%
\pgfsetdash{}{0pt}%
\pgfpathmoveto{\pgfqpoint{2.131276in}{1.869973in}}%
\pgfpathlineto{\pgfqpoint{2.148447in}{1.887188in}}%
\pgfpathlineto{\pgfqpoint{2.128008in}{1.932370in}}%
\pgfpathlineto{\pgfqpoint{2.111496in}{1.913969in}}%
\pgfpathlineto{\pgfqpoint{2.131276in}{1.869973in}}%
\pgfpathclose%
\pgfusepath{fill}%
\end{pgfscope}%
\begin{pgfscope}%
\pgfpathrectangle{\pgfqpoint{0.000000in}{0.000000in}}{\pgfqpoint{3.000000in}{3.000000in}}%
\pgfusepath{clip}%
\pgfsetbuttcap%
\pgfsetroundjoin%
\definecolor{currentfill}{rgb}{0.000000,0.503922,1.000000}%
\pgfsetfillcolor{currentfill}%
\pgfsetlinewidth{0.000000pt}%
\definecolor{currentstroke}{rgb}{0.000000,0.000000,0.000000}%
\pgfsetstrokecolor{currentstroke}%
\pgfsetdash{}{0pt}%
\pgfpathmoveto{\pgfqpoint{1.528149in}{1.174699in}}%
\pgfpathlineto{\pgfqpoint{1.527230in}{1.212031in}}%
\pgfpathlineto{\pgfqpoint{1.567124in}{1.212571in}}%
\pgfpathlineto{\pgfqpoint{1.565289in}{1.175204in}}%
\pgfpathlineto{\pgfqpoint{1.528149in}{1.174699in}}%
\pgfpathclose%
\pgfusepath{fill}%
\end{pgfscope}%
\begin{pgfscope}%
\pgfpathrectangle{\pgfqpoint{0.000000in}{0.000000in}}{\pgfqpoint{3.000000in}{3.000000in}}%
\pgfusepath{clip}%
\pgfsetbuttcap%
\pgfsetroundjoin%
\definecolor{currentfill}{rgb}{0.958254,0.973856,0.009488}%
\pgfsetfillcolor{currentfill}%
\pgfsetlinewidth{0.000000pt}%
\definecolor{currentstroke}{rgb}{0.000000,0.000000,0.000000}%
\pgfsetstrokecolor{currentstroke}%
\pgfsetdash{}{0pt}%
\pgfpathmoveto{\pgfqpoint{1.997840in}{1.679699in}}%
\pgfpathlineto{\pgfqpoint{2.015146in}{1.699933in}}%
\pgfpathlineto{\pgfqpoint{2.011320in}{1.736429in}}%
\pgfpathlineto{\pgfqpoint{1.994218in}{1.714946in}}%
\pgfpathlineto{\pgfqpoint{1.997840in}{1.679699in}}%
\pgfpathclose%
\pgfusepath{fill}%
\end{pgfscope}%
\begin{pgfscope}%
\pgfpathrectangle{\pgfqpoint{0.000000in}{0.000000in}}{\pgfqpoint{3.000000in}{3.000000in}}%
\pgfusepath{clip}%
\pgfsetbuttcap%
\pgfsetroundjoin%
\definecolor{currentfill}{rgb}{0.000000,0.676471,1.000000}%
\pgfsetfillcolor{currentfill}%
\pgfsetlinewidth{0.000000pt}%
\definecolor{currentstroke}{rgb}{0.000000,0.000000,0.000000}%
\pgfsetstrokecolor{currentstroke}%
\pgfsetdash{}{0pt}%
\pgfpathmoveto{\pgfqpoint{1.380962in}{1.241097in}}%
\pgfpathlineto{\pgfqpoint{1.369948in}{1.276055in}}%
\pgfpathlineto{\pgfqpoint{1.404829in}{1.263653in}}%
\pgfpathlineto{\pgfqpoint{1.413602in}{1.229454in}}%
\pgfpathlineto{\pgfqpoint{1.380962in}{1.241097in}}%
\pgfpathclose%
\pgfusepath{fill}%
\end{pgfscope}%
\begin{pgfscope}%
\pgfpathrectangle{\pgfqpoint{0.000000in}{0.000000in}}{\pgfqpoint{3.000000in}{3.000000in}}%
\pgfusepath{clip}%
\pgfsetbuttcap%
\pgfsetroundjoin%
\definecolor{currentfill}{rgb}{0.500000,0.000000,0.000000}%
\pgfsetfillcolor{currentfill}%
\pgfsetlinewidth{0.000000pt}%
\definecolor{currentstroke}{rgb}{0.000000,0.000000,0.000000}%
\pgfsetstrokecolor{currentstroke}%
\pgfsetdash{}{0pt}%
\pgfpathmoveto{\pgfqpoint{0.768434in}{2.183824in}}%
\pgfpathlineto{\pgfqpoint{0.752581in}{2.199431in}}%
\pgfpathlineto{\pgfqpoint{0.709328in}{2.140003in}}%
\pgfpathlineto{\pgfqpoint{0.726164in}{2.125494in}}%
\pgfpathlineto{\pgfqpoint{0.768434in}{2.183824in}}%
\pgfpathclose%
\pgfusepath{fill}%
\end{pgfscope}%
\begin{pgfscope}%
\pgfpathrectangle{\pgfqpoint{0.000000in}{0.000000in}}{\pgfqpoint{3.000000in}{3.000000in}}%
\pgfusepath{clip}%
\pgfsetbuttcap%
\pgfsetroundjoin%
\definecolor{currentfill}{rgb}{0.000000,0.849020,1.000000}%
\pgfsetfillcolor{currentfill}%
\pgfsetlinewidth{0.000000pt}%
\definecolor{currentstroke}{rgb}{0.000000,0.000000,0.000000}%
\pgfsetstrokecolor{currentstroke}%
\pgfsetdash{}{0pt}%
\pgfpathmoveto{\pgfqpoint{1.721835in}{1.280792in}}%
\pgfpathlineto{\pgfqpoint{1.733560in}{1.312526in}}%
\pgfpathlineto{\pgfqpoint{1.764326in}{1.329313in}}%
\pgfpathlineto{\pgfqpoint{1.750754in}{1.296613in}}%
\pgfpathlineto{\pgfqpoint{1.721835in}{1.280792in}}%
\pgfpathclose%
\pgfusepath{fill}%
\end{pgfscope}%
\begin{pgfscope}%
\pgfpathrectangle{\pgfqpoint{0.000000in}{0.000000in}}{\pgfqpoint{3.000000in}{3.000000in}}%
\pgfusepath{clip}%
\pgfsetbuttcap%
\pgfsetroundjoin%
\definecolor{currentfill}{rgb}{0.085389,1.000000,0.882353}%
\pgfsetfillcolor{currentfill}%
\pgfsetlinewidth{0.000000pt}%
\definecolor{currentstroke}{rgb}{0.000000,0.000000,0.000000}%
\pgfsetstrokecolor{currentstroke}%
\pgfsetdash{}{0pt}%
\pgfpathmoveto{\pgfqpoint{1.764326in}{1.329313in}}%
\pgfpathlineto{\pgfqpoint{1.777925in}{1.359081in}}%
\pgfpathlineto{\pgfqpoint{1.804516in}{1.379173in}}%
\pgfpathlineto{\pgfqpoint{1.789425in}{1.348315in}}%
\pgfpathlineto{\pgfqpoint{1.764326in}{1.329313in}}%
\pgfpathclose%
\pgfusepath{fill}%
\end{pgfscope}%
\begin{pgfscope}%
\pgfpathrectangle{\pgfqpoint{0.000000in}{0.000000in}}{\pgfqpoint{3.000000in}{3.000000in}}%
\pgfusepath{clip}%
\pgfsetbuttcap%
\pgfsetroundjoin%
\definecolor{currentfill}{rgb}{1.000000,0.610748,0.000000}%
\pgfsetfillcolor{currentfill}%
\pgfsetlinewidth{0.000000pt}%
\definecolor{currentstroke}{rgb}{0.000000,0.000000,0.000000}%
\pgfsetstrokecolor{currentstroke}%
\pgfsetdash{}{0pt}%
\pgfpathmoveto{\pgfqpoint{1.011728in}{1.842965in}}%
\pgfpathlineto{\pgfqpoint{0.994967in}{1.862259in}}%
\pgfpathlineto{\pgfqpoint{0.980950in}{1.820168in}}%
\pgfpathlineto{\pgfqpoint{0.998219in}{1.802086in}}%
\pgfpathlineto{\pgfqpoint{1.011728in}{1.842965in}}%
\pgfpathclose%
\pgfusepath{fill}%
\end{pgfscope}%
\begin{pgfscope}%
\pgfpathrectangle{\pgfqpoint{0.000000in}{0.000000in}}{\pgfqpoint{3.000000in}{3.000000in}}%
\pgfusepath{clip}%
\pgfsetbuttcap%
\pgfsetroundjoin%
\definecolor{currentfill}{rgb}{0.667299,1.000000,0.300443}%
\pgfsetfillcolor{currentfill}%
\pgfsetlinewidth{0.000000pt}%
\definecolor{currentstroke}{rgb}{0.000000,0.000000,0.000000}%
\pgfsetstrokecolor{currentstroke}%
\pgfsetdash{}{0pt}%
\pgfpathmoveto{\pgfqpoint{1.922026in}{1.559812in}}%
\pgfpathlineto{\pgfqpoint{1.939033in}{1.582492in}}%
\pgfpathlineto{\pgfqpoint{1.946005in}{1.613934in}}%
\pgfpathlineto{\pgfqpoint{1.928756in}{1.589989in}}%
\pgfpathlineto{\pgfqpoint{1.922026in}{1.559812in}}%
\pgfpathclose%
\pgfusepath{fill}%
\end{pgfscope}%
\begin{pgfscope}%
\pgfpathrectangle{\pgfqpoint{0.000000in}{0.000000in}}{\pgfqpoint{3.000000in}{3.000000in}}%
\pgfusepath{clip}%
\pgfsetbuttcap%
\pgfsetroundjoin%
\definecolor{currentfill}{rgb}{0.578748,1.000000,0.388994}%
\pgfsetfillcolor{currentfill}%
\pgfsetlinewidth{0.000000pt}%
\definecolor{currentstroke}{rgb}{0.000000,0.000000,0.000000}%
\pgfsetstrokecolor{currentstroke}%
\pgfsetdash{}{0pt}%
\pgfpathmoveto{\pgfqpoint{1.170703in}{1.555141in}}%
\pgfpathlineto{\pgfqpoint{1.153503in}{1.579887in}}%
\pgfpathlineto{\pgfqpoint{1.163422in}{1.549894in}}%
\pgfpathlineto{\pgfqpoint{1.180233in}{1.526409in}}%
\pgfpathlineto{\pgfqpoint{1.170703in}{1.555141in}}%
\pgfpathclose%
\pgfusepath{fill}%
\end{pgfscope}%
\begin{pgfscope}%
\pgfpathrectangle{\pgfqpoint{0.000000in}{0.000000in}}{\pgfqpoint{3.000000in}{3.000000in}}%
\pgfusepath{clip}%
\pgfsetbuttcap%
\pgfsetroundjoin%
\definecolor{currentfill}{rgb}{1.000000,0.349310,0.000000}%
\pgfsetfillcolor{currentfill}%
\pgfsetlinewidth{0.000000pt}%
\definecolor{currentstroke}{rgb}{0.000000,0.000000,0.000000}%
\pgfsetstrokecolor{currentstroke}%
\pgfsetdash{}{0pt}%
\pgfpathmoveto{\pgfqpoint{2.148447in}{1.887188in}}%
\pgfpathlineto{\pgfqpoint{2.165625in}{1.904038in}}%
\pgfpathlineto{\pgfqpoint{2.144524in}{1.950404in}}%
\pgfpathlineto{\pgfqpoint{2.128008in}{1.932370in}}%
\pgfpathlineto{\pgfqpoint{2.148447in}{1.887188in}}%
\pgfpathclose%
\pgfusepath{fill}%
\end{pgfscope}%
\begin{pgfscope}%
\pgfpathrectangle{\pgfqpoint{0.000000in}{0.000000in}}{\pgfqpoint{3.000000in}{3.000000in}}%
\pgfusepath{clip}%
\pgfsetbuttcap%
\pgfsetroundjoin%
\definecolor{currentfill}{rgb}{0.199241,1.000000,0.768501}%
\pgfsetfillcolor{currentfill}%
\pgfsetlinewidth{0.000000pt}%
\definecolor{currentstroke}{rgb}{0.000000,0.000000,0.000000}%
\pgfsetstrokecolor{currentstroke}%
\pgfsetdash{}{0pt}%
\pgfpathmoveto{\pgfqpoint{1.262502in}{1.393610in}}%
\pgfpathlineto{\pgfqpoint{1.246604in}{1.422772in}}%
\pgfpathlineto{\pgfqpoint{1.270055in}{1.400255in}}%
\pgfpathlineto{\pgfqpoint{1.284717in}{1.372247in}}%
\pgfpathlineto{\pgfqpoint{1.262502in}{1.393610in}}%
\pgfpathclose%
\pgfusepath{fill}%
\end{pgfscope}%
\begin{pgfscope}%
\pgfpathrectangle{\pgfqpoint{0.000000in}{0.000000in}}{\pgfqpoint{3.000000in}{3.000000in}}%
\pgfusepath{clip}%
\pgfsetbuttcap%
\pgfsetroundjoin%
\definecolor{currentfill}{rgb}{0.895003,1.000000,0.072739}%
\pgfsetfillcolor{currentfill}%
\pgfsetlinewidth{0.000000pt}%
\definecolor{currentstroke}{rgb}{0.000000,0.000000,0.000000}%
\pgfsetstrokecolor{currentstroke}%
\pgfsetdash{}{0pt}%
\pgfpathmoveto{\pgfqpoint{1.101615in}{1.681394in}}%
\pgfpathlineto{\pgfqpoint{1.084409in}{1.703249in}}%
\pgfpathlineto{\pgfqpoint{1.084541in}{1.667912in}}%
\pgfpathlineto{\pgfqpoint{1.101805in}{1.647314in}}%
\pgfpathlineto{\pgfqpoint{1.101615in}{1.681394in}}%
\pgfpathclose%
\pgfusepath{fill}%
\end{pgfscope}%
\begin{pgfscope}%
\pgfpathrectangle{\pgfqpoint{0.000000in}{0.000000in}}{\pgfqpoint{3.000000in}{3.000000in}}%
\pgfusepath{clip}%
\pgfsetbuttcap%
\pgfsetroundjoin%
\definecolor{currentfill}{rgb}{1.000000,0.538126,0.000000}%
\pgfsetfillcolor{currentfill}%
\pgfsetlinewidth{0.000000pt}%
\definecolor{currentstroke}{rgb}{0.000000,0.000000,0.000000}%
\pgfsetstrokecolor{currentstroke}%
\pgfsetdash{}{0pt}%
\pgfpathmoveto{\pgfqpoint{0.994967in}{1.862259in}}%
\pgfpathlineto{\pgfqpoint{0.978201in}{1.881092in}}%
\pgfpathlineto{\pgfqpoint{0.963670in}{1.837792in}}%
\pgfpathlineto{\pgfqpoint{0.980950in}{1.820168in}}%
\pgfpathlineto{\pgfqpoint{0.994967in}{1.862259in}}%
\pgfpathclose%
\pgfusepath{fill}%
\end{pgfscope}%
\begin{pgfscope}%
\pgfpathrectangle{\pgfqpoint{0.000000in}{0.000000in}}{\pgfqpoint{3.000000in}{3.000000in}}%
\pgfusepath{clip}%
\pgfsetbuttcap%
\pgfsetroundjoin%
\definecolor{currentfill}{rgb}{1.000000,0.886710,0.000000}%
\pgfsetfillcolor{currentfill}%
\pgfsetlinewidth{0.000000pt}%
\definecolor{currentstroke}{rgb}{0.000000,0.000000,0.000000}%
\pgfsetstrokecolor{currentstroke}%
\pgfsetdash{}{0pt}%
\pgfpathmoveto{\pgfqpoint{2.015146in}{1.699933in}}%
\pgfpathlineto{\pgfqpoint{2.032467in}{1.719457in}}%
\pgfpathlineto{\pgfqpoint{2.028430in}{1.757199in}}%
\pgfpathlineto{\pgfqpoint{2.011320in}{1.736429in}}%
\pgfpathlineto{\pgfqpoint{2.015146in}{1.699933in}}%
\pgfpathclose%
\pgfusepath{fill}%
\end{pgfscope}%
\begin{pgfscope}%
\pgfpathrectangle{\pgfqpoint{0.000000in}{0.000000in}}{\pgfqpoint{3.000000in}{3.000000in}}%
\pgfusepath{clip}%
\pgfsetbuttcap%
\pgfsetroundjoin%
\definecolor{currentfill}{rgb}{0.000000,0.849020,1.000000}%
\pgfsetfillcolor{currentfill}%
\pgfsetlinewidth{0.000000pt}%
\definecolor{currentstroke}{rgb}{0.000000,0.000000,0.000000}%
\pgfsetstrokecolor{currentstroke}%
\pgfsetdash{}{0pt}%
\pgfpathmoveto{\pgfqpoint{1.339408in}{1.291085in}}%
\pgfpathlineto{\pgfqpoint{1.326415in}{1.323448in}}%
\pgfpathlineto{\pgfqpoint{1.358910in}{1.307499in}}%
\pgfpathlineto{\pgfqpoint{1.369948in}{1.276055in}}%
\pgfpathlineto{\pgfqpoint{1.339408in}{1.291085in}}%
\pgfpathclose%
\pgfusepath{fill}%
\end{pgfscope}%
\begin{pgfscope}%
\pgfpathrectangle{\pgfqpoint{0.000000in}{0.000000in}}{\pgfqpoint{3.000000in}{3.000000in}}%
\pgfusepath{clip}%
\pgfsetbuttcap%
\pgfsetroundjoin%
\definecolor{currentfill}{rgb}{1.000000,0.291213,0.000000}%
\pgfsetfillcolor{currentfill}%
\pgfsetlinewidth{0.000000pt}%
\definecolor{currentstroke}{rgb}{0.000000,0.000000,0.000000}%
\pgfsetstrokecolor{currentstroke}%
\pgfsetdash{}{0pt}%
\pgfpathmoveto{\pgfqpoint{2.165625in}{1.904038in}}%
\pgfpathlineto{\pgfqpoint{2.182812in}{1.920548in}}%
\pgfpathlineto{\pgfqpoint{2.161042in}{1.968094in}}%
\pgfpathlineto{\pgfqpoint{2.144524in}{1.950404in}}%
\pgfpathlineto{\pgfqpoint{2.165625in}{1.904038in}}%
\pgfpathclose%
\pgfusepath{fill}%
\end{pgfscope}%
\begin{pgfscope}%
\pgfpathrectangle{\pgfqpoint{0.000000in}{0.000000in}}{\pgfqpoint{3.000000in}{3.000000in}}%
\pgfusepath{clip}%
\pgfsetbuttcap%
\pgfsetroundjoin%
\definecolor{currentfill}{rgb}{0.000000,0.676471,1.000000}%
\pgfsetfillcolor{currentfill}%
\pgfsetlinewidth{0.000000pt}%
\definecolor{currentstroke}{rgb}{0.000000,0.000000,0.000000}%
\pgfsetstrokecolor{currentstroke}%
\pgfsetdash{}{0pt}%
\pgfpathmoveto{\pgfqpoint{1.643868in}{1.223217in}}%
\pgfpathlineto{\pgfqpoint{1.651014in}{1.257008in}}%
\pgfpathlineto{\pgfqpoint{1.688309in}{1.267473in}}%
\pgfpathlineto{\pgfqpoint{1.678761in}{1.233040in}}%
\pgfpathlineto{\pgfqpoint{1.643868in}{1.223217in}}%
\pgfpathclose%
\pgfusepath{fill}%
\end{pgfscope}%
\begin{pgfscope}%
\pgfpathrectangle{\pgfqpoint{0.000000in}{0.000000in}}{\pgfqpoint{3.000000in}{3.000000in}}%
\pgfusepath{clip}%
\pgfsetbuttcap%
\pgfsetroundjoin%
\definecolor{currentfill}{rgb}{0.401645,1.000000,0.566097}%
\pgfsetfillcolor{currentfill}%
\pgfsetlinewidth{0.000000pt}%
\definecolor{currentstroke}{rgb}{0.000000,0.000000,0.000000}%
\pgfsetstrokecolor{currentstroke}%
\pgfsetdash{}{0pt}%
\pgfpathmoveto{\pgfqpoint{1.213791in}{1.475083in}}%
\pgfpathlineto{\pgfqpoint{1.197023in}{1.501537in}}%
\pgfpathlineto{\pgfqpoint{1.214736in}{1.475051in}}%
\pgfpathlineto{\pgfqpoint{1.230682in}{1.449822in}}%
\pgfpathlineto{\pgfqpoint{1.213791in}{1.475083in}}%
\pgfpathclose%
\pgfusepath{fill}%
\end{pgfscope}%
\begin{pgfscope}%
\pgfpathrectangle{\pgfqpoint{0.000000in}{0.000000in}}{\pgfqpoint{3.000000in}{3.000000in}}%
\pgfusepath{clip}%
\pgfsetbuttcap%
\pgfsetroundjoin%
\definecolor{currentfill}{rgb}{0.085389,1.000000,0.882353}%
\pgfsetfillcolor{currentfill}%
\pgfsetlinewidth{0.000000pt}%
\definecolor{currentstroke}{rgb}{0.000000,0.000000,0.000000}%
\pgfsetstrokecolor{currentstroke}%
\pgfsetdash{}{0pt}%
\pgfpathmoveto{\pgfqpoint{1.299353in}{1.341764in}}%
\pgfpathlineto{\pgfqpoint{1.284717in}{1.372247in}}%
\pgfpathlineto{\pgfqpoint{1.313395in}{1.352879in}}%
\pgfpathlineto{\pgfqpoint{1.326415in}{1.323448in}}%
\pgfpathlineto{\pgfqpoint{1.299353in}{1.341764in}}%
\pgfpathclose%
\pgfusepath{fill}%
\end{pgfscope}%
\begin{pgfscope}%
\pgfpathrectangle{\pgfqpoint{0.000000in}{0.000000in}}{\pgfqpoint{3.000000in}{3.000000in}}%
\pgfusepath{clip}%
\pgfsetbuttcap%
\pgfsetroundjoin%
\definecolor{currentfill}{rgb}{0.000000,0.676471,1.000000}%
\pgfsetfillcolor{currentfill}%
\pgfsetlinewidth{0.000000pt}%
\definecolor{currentstroke}{rgb}{0.000000,0.000000,0.000000}%
\pgfsetstrokecolor{currentstroke}%
\pgfsetdash{}{0pt}%
\pgfpathmoveto{\pgfqpoint{1.413602in}{1.229454in}}%
\pgfpathlineto{\pgfqpoint{1.404829in}{1.263653in}}%
\pgfpathlineto{\pgfqpoint{1.443176in}{1.254204in}}%
\pgfpathlineto{\pgfqpoint{1.449475in}{1.220585in}}%
\pgfpathlineto{\pgfqpoint{1.413602in}{1.229454in}}%
\pgfpathclose%
\pgfusepath{fill}%
\end{pgfscope}%
\begin{pgfscope}%
\pgfpathrectangle{\pgfqpoint{0.000000in}{0.000000in}}{\pgfqpoint{3.000000in}{3.000000in}}%
\pgfusepath{clip}%
\pgfsetbuttcap%
\pgfsetroundjoin%
\definecolor{currentfill}{rgb}{1.000000,0.233115,0.000000}%
\pgfsetfillcolor{currentfill}%
\pgfsetlinewidth{0.000000pt}%
\definecolor{currentstroke}{rgb}{0.000000,0.000000,0.000000}%
\pgfsetstrokecolor{currentstroke}%
\pgfsetdash{}{0pt}%
\pgfpathmoveto{\pgfqpoint{2.182812in}{1.920548in}}%
\pgfpathlineto{\pgfqpoint{2.200006in}{1.936741in}}%
\pgfpathlineto{\pgfqpoint{2.177564in}{1.985465in}}%
\pgfpathlineto{\pgfqpoint{2.161042in}{1.968094in}}%
\pgfpathlineto{\pgfqpoint{2.182812in}{1.920548in}}%
\pgfpathclose%
\pgfusepath{fill}%
\end{pgfscope}%
\begin{pgfscope}%
\pgfpathrectangle{\pgfqpoint{0.000000in}{0.000000in}}{\pgfqpoint{3.000000in}{3.000000in}}%
\pgfusepath{clip}%
\pgfsetbuttcap%
\pgfsetroundjoin%
\definecolor{currentfill}{rgb}{0.490196,1.000000,0.477546}%
\pgfsetfillcolor{currentfill}%
\pgfsetlinewidth{0.000000pt}%
\definecolor{currentstroke}{rgb}{0.000000,0.000000,0.000000}%
\pgfsetstrokecolor{currentstroke}%
\pgfsetdash{}{0pt}%
\pgfpathmoveto{\pgfqpoint{1.873184in}{1.483728in}}%
\pgfpathlineto{\pgfqpoint{1.889475in}{1.507777in}}%
\pgfpathlineto{\pgfqpoint{1.905038in}{1.535908in}}%
\pgfpathlineto{\pgfqpoint{1.888069in}{1.510618in}}%
\pgfpathlineto{\pgfqpoint{1.873184in}{1.483728in}}%
\pgfpathclose%
\pgfusepath{fill}%
\end{pgfscope}%
\begin{pgfscope}%
\pgfpathrectangle{\pgfqpoint{0.000000in}{0.000000in}}{\pgfqpoint{3.000000in}{3.000000in}}%
\pgfusepath{clip}%
\pgfsetbuttcap%
\pgfsetroundjoin%
\definecolor{currentfill}{rgb}{1.000000,0.480029,0.000000}%
\pgfsetfillcolor{currentfill}%
\pgfsetlinewidth{0.000000pt}%
\definecolor{currentstroke}{rgb}{0.000000,0.000000,0.000000}%
\pgfsetstrokecolor{currentstroke}%
\pgfsetdash{}{0pt}%
\pgfpathmoveto{\pgfqpoint{0.978201in}{1.881092in}}%
\pgfpathlineto{\pgfqpoint{0.961430in}{1.899500in}}%
\pgfpathlineto{\pgfqpoint{0.946381in}{1.854993in}}%
\pgfpathlineto{\pgfqpoint{0.963670in}{1.837792in}}%
\pgfpathlineto{\pgfqpoint{0.978201in}{1.881092in}}%
\pgfpathclose%
\pgfusepath{fill}%
\end{pgfscope}%
\begin{pgfscope}%
\pgfpathrectangle{\pgfqpoint{0.000000in}{0.000000in}}{\pgfqpoint{3.000000in}{3.000000in}}%
\pgfusepath{clip}%
\pgfsetbuttcap%
\pgfsetroundjoin%
\definecolor{currentfill}{rgb}{0.300443,1.000000,0.667299}%
\pgfsetfillcolor{currentfill}%
\pgfsetlinewidth{0.000000pt}%
\definecolor{currentstroke}{rgb}{0.000000,0.000000,0.000000}%
\pgfsetstrokecolor{currentstroke}%
\pgfsetdash{}{0pt}%
\pgfpathmoveto{\pgfqpoint{1.819634in}{1.407555in}}%
\pgfpathlineto{\pgfqpoint{1.834778in}{1.433825in}}%
\pgfpathlineto{\pgfqpoint{1.856917in}{1.458097in}}%
\pgfpathlineto{\pgfqpoint{1.840674in}{1.430645in}}%
\pgfpathlineto{\pgfqpoint{1.819634in}{1.407555in}}%
\pgfpathclose%
\pgfusepath{fill}%
\end{pgfscope}%
\begin{pgfscope}%
\pgfpathrectangle{\pgfqpoint{0.000000in}{0.000000in}}{\pgfqpoint{3.000000in}{3.000000in}}%
\pgfusepath{clip}%
\pgfsetbuttcap%
\pgfsetroundjoin%
\definecolor{currentfill}{rgb}{0.743201,1.000000,0.224541}%
\pgfsetfillcolor{currentfill}%
\pgfsetlinewidth{0.000000pt}%
\definecolor{currentstroke}{rgb}{0.000000,0.000000,0.000000}%
\pgfsetstrokecolor{currentstroke}%
\pgfsetdash{}{0pt}%
\pgfpathmoveto{\pgfqpoint{1.939033in}{1.582492in}}%
\pgfpathlineto{\pgfqpoint{1.956060in}{1.604088in}}%
\pgfpathlineto{\pgfqpoint{1.963269in}{1.636792in}}%
\pgfpathlineto{\pgfqpoint{1.946005in}{1.613934in}}%
\pgfpathlineto{\pgfqpoint{1.939033in}{1.582492in}}%
\pgfpathclose%
\pgfusepath{fill}%
\end{pgfscope}%
\begin{pgfscope}%
\pgfpathrectangle{\pgfqpoint{0.000000in}{0.000000in}}{\pgfqpoint{3.000000in}{3.000000in}}%
\pgfusepath{clip}%
\pgfsetbuttcap%
\pgfsetroundjoin%
\definecolor{currentfill}{rgb}{0.958254,0.973856,0.009488}%
\pgfsetfillcolor{currentfill}%
\pgfsetlinewidth{0.000000pt}%
\definecolor{currentstroke}{rgb}{0.000000,0.000000,0.000000}%
\pgfsetstrokecolor{currentstroke}%
\pgfsetdash{}{0pt}%
\pgfpathmoveto{\pgfqpoint{1.084409in}{1.703249in}}%
\pgfpathlineto{\pgfqpoint{1.067192in}{1.724318in}}%
\pgfpathlineto{\pgfqpoint{1.067260in}{1.687726in}}%
\pgfpathlineto{\pgfqpoint{1.084541in}{1.667912in}}%
\pgfpathlineto{\pgfqpoint{1.084409in}{1.703249in}}%
\pgfpathclose%
\pgfusepath{fill}%
\end{pgfscope}%
\begin{pgfscope}%
\pgfpathrectangle{\pgfqpoint{0.000000in}{0.000000in}}{\pgfqpoint{3.000000in}{3.000000in}}%
\pgfusepath{clip}%
\pgfsetbuttcap%
\pgfsetroundjoin%
\definecolor{currentfill}{rgb}{0.667299,1.000000,0.300443}%
\pgfsetfillcolor{currentfill}%
\pgfsetlinewidth{0.000000pt}%
\definecolor{currentstroke}{rgb}{0.000000,0.000000,0.000000}%
\pgfsetstrokecolor{currentstroke}%
\pgfsetdash{}{0pt}%
\pgfpathmoveto{\pgfqpoint{1.153503in}{1.579887in}}%
\pgfpathlineto{\pgfqpoint{1.136286in}{1.603410in}}%
\pgfpathlineto{\pgfqpoint{1.146591in}{1.572157in}}%
\pgfpathlineto{\pgfqpoint{1.163422in}{1.549894in}}%
\pgfpathlineto{\pgfqpoint{1.153503in}{1.579887in}}%
\pgfpathclose%
\pgfusepath{fill}%
\end{pgfscope}%
\begin{pgfscope}%
\pgfpathrectangle{\pgfqpoint{0.000000in}{0.000000in}}{\pgfqpoint{3.000000in}{3.000000in}}%
\pgfusepath{clip}%
\pgfsetbuttcap%
\pgfsetroundjoin%
\definecolor{currentfill}{rgb}{1.000000,0.814089,0.000000}%
\pgfsetfillcolor{currentfill}%
\pgfsetlinewidth{0.000000pt}%
\definecolor{currentstroke}{rgb}{0.000000,0.000000,0.000000}%
\pgfsetstrokecolor{currentstroke}%
\pgfsetdash{}{0pt}%
\pgfpathmoveto{\pgfqpoint{2.032467in}{1.719457in}}%
\pgfpathlineto{\pgfqpoint{2.049802in}{1.738335in}}%
\pgfpathlineto{\pgfqpoint{2.045550in}{1.777321in}}%
\pgfpathlineto{\pgfqpoint{2.028430in}{1.757199in}}%
\pgfpathlineto{\pgfqpoint{2.032467in}{1.719457in}}%
\pgfpathclose%
\pgfusepath{fill}%
\end{pgfscope}%
\begin{pgfscope}%
\pgfpathrectangle{\pgfqpoint{0.000000in}{0.000000in}}{\pgfqpoint{3.000000in}{3.000000in}}%
\pgfusepath{clip}%
\pgfsetbuttcap%
\pgfsetroundjoin%
\definecolor{currentfill}{rgb}{1.000000,0.175018,0.000000}%
\pgfsetfillcolor{currentfill}%
\pgfsetlinewidth{0.000000pt}%
\definecolor{currentstroke}{rgb}{0.000000,0.000000,0.000000}%
\pgfsetstrokecolor{currentstroke}%
\pgfsetdash{}{0pt}%
\pgfpathmoveto{\pgfqpoint{2.200006in}{1.936741in}}%
\pgfpathlineto{\pgfqpoint{2.217209in}{1.952637in}}%
\pgfpathlineto{\pgfqpoint{2.194088in}{2.002536in}}%
\pgfpathlineto{\pgfqpoint{2.177564in}{1.985465in}}%
\pgfpathlineto{\pgfqpoint{2.200006in}{1.936741in}}%
\pgfpathclose%
\pgfusepath{fill}%
\end{pgfscope}%
\begin{pgfscope}%
\pgfpathrectangle{\pgfqpoint{0.000000in}{0.000000in}}{\pgfqpoint{3.000000in}{3.000000in}}%
\pgfusepath{clip}%
\pgfsetbuttcap%
\pgfsetroundjoin%
\definecolor{currentfill}{rgb}{1.000000,0.407407,0.000000}%
\pgfsetfillcolor{currentfill}%
\pgfsetlinewidth{0.000000pt}%
\definecolor{currentstroke}{rgb}{0.000000,0.000000,0.000000}%
\pgfsetstrokecolor{currentstroke}%
\pgfsetdash{}{0pt}%
\pgfpathmoveto{\pgfqpoint{0.961430in}{1.899500in}}%
\pgfpathlineto{\pgfqpoint{0.944654in}{1.917513in}}%
\pgfpathlineto{\pgfqpoint{0.929082in}{1.871801in}}%
\pgfpathlineto{\pgfqpoint{0.946381in}{1.854993in}}%
\pgfpathlineto{\pgfqpoint{0.961430in}{1.899500in}}%
\pgfpathclose%
\pgfusepath{fill}%
\end{pgfscope}%
\begin{pgfscope}%
\pgfpathrectangle{\pgfqpoint{0.000000in}{0.000000in}}{\pgfqpoint{3.000000in}{3.000000in}}%
\pgfusepath{clip}%
\pgfsetbuttcap%
\pgfsetroundjoin%
\definecolor{currentfill}{rgb}{0.000000,0.676471,1.000000}%
\pgfsetfillcolor{currentfill}%
\pgfsetlinewidth{0.000000pt}%
\definecolor{currentstroke}{rgb}{0.000000,0.000000,0.000000}%
\pgfsetstrokecolor{currentstroke}%
\pgfsetdash{}{0pt}%
\pgfpathmoveto{\pgfqpoint{1.606338in}{1.216333in}}%
\pgfpathlineto{\pgfqpoint{1.610892in}{1.249674in}}%
\pgfpathlineto{\pgfqpoint{1.651014in}{1.257008in}}%
\pgfpathlineto{\pgfqpoint{1.643868in}{1.223217in}}%
\pgfpathlineto{\pgfqpoint{1.606338in}{1.216333in}}%
\pgfpathclose%
\pgfusepath{fill}%
\end{pgfscope}%
\begin{pgfscope}%
\pgfpathrectangle{\pgfqpoint{0.000000in}{0.000000in}}{\pgfqpoint{3.000000in}{3.000000in}}%
\pgfusepath{clip}%
\pgfsetbuttcap%
\pgfsetroundjoin%
\definecolor{currentfill}{rgb}{0.000000,0.849020,1.000000}%
\pgfsetfillcolor{currentfill}%
\pgfsetlinewidth{0.000000pt}%
\definecolor{currentstroke}{rgb}{0.000000,0.000000,0.000000}%
\pgfsetstrokecolor{currentstroke}%
\pgfsetdash{}{0pt}%
\pgfpathmoveto{\pgfqpoint{1.688309in}{1.267473in}}%
\pgfpathlineto{\pgfqpoint{1.697880in}{1.298391in}}%
\pgfpathlineto{\pgfqpoint{1.733560in}{1.312526in}}%
\pgfpathlineto{\pgfqpoint{1.721835in}{1.280792in}}%
\pgfpathlineto{\pgfqpoint{1.688309in}{1.267473in}}%
\pgfpathclose%
\pgfusepath{fill}%
\end{pgfscope}%
\begin{pgfscope}%
\pgfpathrectangle{\pgfqpoint{0.000000in}{0.000000in}}{\pgfqpoint{3.000000in}{3.000000in}}%
\pgfusepath{clip}%
\pgfsetbuttcap%
\pgfsetroundjoin%
\definecolor{currentfill}{rgb}{1.000000,0.116921,0.000000}%
\pgfsetfillcolor{currentfill}%
\pgfsetlinewidth{0.000000pt}%
\definecolor{currentstroke}{rgb}{0.000000,0.000000,0.000000}%
\pgfsetstrokecolor{currentstroke}%
\pgfsetdash{}{0pt}%
\pgfpathmoveto{\pgfqpoint{2.217209in}{1.952637in}}%
\pgfpathlineto{\pgfqpoint{2.234419in}{1.968255in}}%
\pgfpathlineto{\pgfqpoint{2.210615in}{2.019327in}}%
\pgfpathlineto{\pgfqpoint{2.194088in}{2.002536in}}%
\pgfpathlineto{\pgfqpoint{2.217209in}{1.952637in}}%
\pgfpathclose%
\pgfusepath{fill}%
\end{pgfscope}%
\begin{pgfscope}%
\pgfpathrectangle{\pgfqpoint{0.000000in}{0.000000in}}{\pgfqpoint{3.000000in}{3.000000in}}%
\pgfusepath{clip}%
\pgfsetbuttcap%
\pgfsetroundjoin%
\definecolor{currentfill}{rgb}{0.000000,0.676471,1.000000}%
\pgfsetfillcolor{currentfill}%
\pgfsetlinewidth{0.000000pt}%
\definecolor{currentstroke}{rgb}{0.000000,0.000000,0.000000}%
\pgfsetstrokecolor{currentstroke}%
\pgfsetdash{}{0pt}%
\pgfpathmoveto{\pgfqpoint{1.449475in}{1.220585in}}%
\pgfpathlineto{\pgfqpoint{1.443176in}{1.254204in}}%
\pgfpathlineto{\pgfqpoint{1.484017in}{1.247961in}}%
\pgfpathlineto{\pgfqpoint{1.487676in}{1.214726in}}%
\pgfpathlineto{\pgfqpoint{1.449475in}{1.220585in}}%
\pgfpathclose%
\pgfusepath{fill}%
\end{pgfscope}%
\begin{pgfscope}%
\pgfpathrectangle{\pgfqpoint{0.000000in}{0.000000in}}{\pgfqpoint{3.000000in}{3.000000in}}%
\pgfusepath{clip}%
\pgfsetbuttcap%
\pgfsetroundjoin%
\definecolor{currentfill}{rgb}{0.199241,1.000000,0.768501}%
\pgfsetfillcolor{currentfill}%
\pgfsetlinewidth{0.000000pt}%
\definecolor{currentstroke}{rgb}{0.000000,0.000000,0.000000}%
\pgfsetstrokecolor{currentstroke}%
\pgfsetdash{}{0pt}%
\pgfpathmoveto{\pgfqpoint{1.777925in}{1.359081in}}%
\pgfpathlineto{\pgfqpoint{1.791550in}{1.386374in}}%
\pgfpathlineto{\pgfqpoint{1.819634in}{1.407555in}}%
\pgfpathlineto{\pgfqpoint{1.804516in}{1.379173in}}%
\pgfpathlineto{\pgfqpoint{1.777925in}{1.359081in}}%
\pgfpathclose%
\pgfusepath{fill}%
\end{pgfscope}%
\begin{pgfscope}%
\pgfpathrectangle{\pgfqpoint{0.000000in}{0.000000in}}{\pgfqpoint{3.000000in}{3.000000in}}%
\pgfusepath{clip}%
\pgfsetbuttcap%
\pgfsetroundjoin%
\definecolor{currentfill}{rgb}{0.300443,1.000000,0.667299}%
\pgfsetfillcolor{currentfill}%
\pgfsetlinewidth{0.000000pt}%
\definecolor{currentstroke}{rgb}{0.000000,0.000000,0.000000}%
\pgfsetstrokecolor{currentstroke}%
\pgfsetdash{}{0pt}%
\pgfpathmoveto{\pgfqpoint{1.246604in}{1.422772in}}%
\pgfpathlineto{\pgfqpoint{1.230682in}{1.449822in}}%
\pgfpathlineto{\pgfqpoint{1.255366in}{1.426149in}}%
\pgfpathlineto{\pgfqpoint{1.270055in}{1.400255in}}%
\pgfpathlineto{\pgfqpoint{1.246604in}{1.422772in}}%
\pgfpathclose%
\pgfusepath{fill}%
\end{pgfscope}%
\begin{pgfscope}%
\pgfpathrectangle{\pgfqpoint{0.000000in}{0.000000in}}{\pgfqpoint{3.000000in}{3.000000in}}%
\pgfusepath{clip}%
\pgfsetbuttcap%
\pgfsetroundjoin%
\definecolor{currentfill}{rgb}{1.000000,0.741467,0.000000}%
\pgfsetfillcolor{currentfill}%
\pgfsetlinewidth{0.000000pt}%
\definecolor{currentstroke}{rgb}{0.000000,0.000000,0.000000}%
\pgfsetstrokecolor{currentstroke}%
\pgfsetdash{}{0pt}%
\pgfpathmoveto{\pgfqpoint{2.049802in}{1.738335in}}%
\pgfpathlineto{\pgfqpoint{2.067151in}{1.756624in}}%
\pgfpathlineto{\pgfqpoint{2.062678in}{1.796852in}}%
\pgfpathlineto{\pgfqpoint{2.045550in}{1.777321in}}%
\pgfpathlineto{\pgfqpoint{2.049802in}{1.738335in}}%
\pgfpathclose%
\pgfusepath{fill}%
\end{pgfscope}%
\begin{pgfscope}%
\pgfpathrectangle{\pgfqpoint{0.000000in}{0.000000in}}{\pgfqpoint{3.000000in}{3.000000in}}%
\pgfusepath{clip}%
\pgfsetbuttcap%
\pgfsetroundjoin%
\definecolor{currentfill}{rgb}{1.000000,0.886710,0.000000}%
\pgfsetfillcolor{currentfill}%
\pgfsetlinewidth{0.000000pt}%
\definecolor{currentstroke}{rgb}{0.000000,0.000000,0.000000}%
\pgfsetstrokecolor{currentstroke}%
\pgfsetdash{}{0pt}%
\pgfpathmoveto{\pgfqpoint{1.067192in}{1.724318in}}%
\pgfpathlineto{\pgfqpoint{1.049965in}{1.744676in}}%
\pgfpathlineto{\pgfqpoint{1.049964in}{1.706830in}}%
\pgfpathlineto{\pgfqpoint{1.067260in}{1.687726in}}%
\pgfpathlineto{\pgfqpoint{1.067192in}{1.724318in}}%
\pgfpathclose%
\pgfusepath{fill}%
\end{pgfscope}%
\begin{pgfscope}%
\pgfpathrectangle{\pgfqpoint{0.000000in}{0.000000in}}{\pgfqpoint{3.000000in}{3.000000in}}%
\pgfusepath{clip}%
\pgfsetbuttcap%
\pgfsetroundjoin%
\definecolor{currentfill}{rgb}{1.000000,0.349310,0.000000}%
\pgfsetfillcolor{currentfill}%
\pgfsetlinewidth{0.000000pt}%
\definecolor{currentstroke}{rgb}{0.000000,0.000000,0.000000}%
\pgfsetstrokecolor{currentstroke}%
\pgfsetdash{}{0pt}%
\pgfpathmoveto{\pgfqpoint{0.944654in}{1.917513in}}%
\pgfpathlineto{\pgfqpoint{0.927874in}{1.935159in}}%
\pgfpathlineto{\pgfqpoint{0.911772in}{1.888245in}}%
\pgfpathlineto{\pgfqpoint{0.929082in}{1.871801in}}%
\pgfpathlineto{\pgfqpoint{0.944654in}{1.917513in}}%
\pgfpathclose%
\pgfusepath{fill}%
\end{pgfscope}%
\begin{pgfscope}%
\pgfpathrectangle{\pgfqpoint{0.000000in}{0.000000in}}{\pgfqpoint{3.000000in}{3.000000in}}%
\pgfusepath{clip}%
\pgfsetbuttcap%
\pgfsetroundjoin%
\definecolor{currentfill}{rgb}{0.085389,1.000000,0.882353}%
\pgfsetfillcolor{currentfill}%
\pgfsetlinewidth{0.000000pt}%
\definecolor{currentstroke}{rgb}{0.000000,0.000000,0.000000}%
\pgfsetstrokecolor{currentstroke}%
\pgfsetdash{}{0pt}%
\pgfpathmoveto{\pgfqpoint{1.733560in}{1.312526in}}%
\pgfpathlineto{\pgfqpoint{1.745311in}{1.341327in}}%
\pgfpathlineto{\pgfqpoint{1.777925in}{1.359081in}}%
\pgfpathlineto{\pgfqpoint{1.764326in}{1.329313in}}%
\pgfpathlineto{\pgfqpoint{1.733560in}{1.312526in}}%
\pgfpathclose%
\pgfusepath{fill}%
\end{pgfscope}%
\begin{pgfscope}%
\pgfpathrectangle{\pgfqpoint{0.000000in}{0.000000in}}{\pgfqpoint{3.000000in}{3.000000in}}%
\pgfusepath{clip}%
\pgfsetbuttcap%
\pgfsetroundjoin%
\definecolor{currentfill}{rgb}{0.490196,1.000000,0.477546}%
\pgfsetfillcolor{currentfill}%
\pgfsetlinewidth{0.000000pt}%
\definecolor{currentstroke}{rgb}{0.000000,0.000000,0.000000}%
\pgfsetstrokecolor{currentstroke}%
\pgfsetdash{}{0pt}%
\pgfpathmoveto{\pgfqpoint{1.197023in}{1.501537in}}%
\pgfpathlineto{\pgfqpoint{1.180233in}{1.526409in}}%
\pgfpathlineto{\pgfqpoint{1.198765in}{1.498698in}}%
\pgfpathlineto{\pgfqpoint{1.214736in}{1.475051in}}%
\pgfpathlineto{\pgfqpoint{1.197023in}{1.501537in}}%
\pgfpathclose%
\pgfusepath{fill}%
\end{pgfscope}%
\begin{pgfscope}%
\pgfpathrectangle{\pgfqpoint{0.000000in}{0.000000in}}{\pgfqpoint{3.000000in}{3.000000in}}%
\pgfusepath{clip}%
\pgfsetbuttcap%
\pgfsetroundjoin%
\definecolor{currentfill}{rgb}{0.819102,1.000000,0.148640}%
\pgfsetfillcolor{currentfill}%
\pgfsetlinewidth{0.000000pt}%
\definecolor{currentstroke}{rgb}{0.000000,0.000000,0.000000}%
\pgfsetstrokecolor{currentstroke}%
\pgfsetdash{}{0pt}%
\pgfpathmoveto{\pgfqpoint{1.956060in}{1.604088in}}%
\pgfpathlineto{\pgfqpoint{1.973106in}{1.624715in}}%
\pgfpathlineto{\pgfqpoint{1.980547in}{1.658681in}}%
\pgfpathlineto{\pgfqpoint{1.963269in}{1.636792in}}%
\pgfpathlineto{\pgfqpoint{1.956060in}{1.604088in}}%
\pgfpathclose%
\pgfusepath{fill}%
\end{pgfscope}%
\begin{pgfscope}%
\pgfpathrectangle{\pgfqpoint{0.000000in}{0.000000in}}{\pgfqpoint{3.000000in}{3.000000in}}%
\pgfusepath{clip}%
\pgfsetbuttcap%
\pgfsetroundjoin%
\definecolor{currentfill}{rgb}{0.000000,0.849020,1.000000}%
\pgfsetfillcolor{currentfill}%
\pgfsetlinewidth{0.000000pt}%
\definecolor{currentstroke}{rgb}{0.000000,0.000000,0.000000}%
\pgfsetstrokecolor{currentstroke}%
\pgfsetdash{}{0pt}%
\pgfpathmoveto{\pgfqpoint{1.369948in}{1.276055in}}%
\pgfpathlineto{\pgfqpoint{1.358910in}{1.307499in}}%
\pgfpathlineto{\pgfqpoint{1.396035in}{1.294336in}}%
\pgfpathlineto{\pgfqpoint{1.404829in}{1.263653in}}%
\pgfpathlineto{\pgfqpoint{1.369948in}{1.276055in}}%
\pgfpathclose%
\pgfusepath{fill}%
\end{pgfscope}%
\begin{pgfscope}%
\pgfpathrectangle{\pgfqpoint{0.000000in}{0.000000in}}{\pgfqpoint{3.000000in}{3.000000in}}%
\pgfusepath{clip}%
\pgfsetbuttcap%
\pgfsetroundjoin%
\definecolor{currentfill}{rgb}{0.999109,0.073348,0.000000}%
\pgfsetfillcolor{currentfill}%
\pgfsetlinewidth{0.000000pt}%
\definecolor{currentstroke}{rgb}{0.000000,0.000000,0.000000}%
\pgfsetstrokecolor{currentstroke}%
\pgfsetdash{}{0pt}%
\pgfpathmoveto{\pgfqpoint{2.234419in}{1.968255in}}%
\pgfpathlineto{\pgfqpoint{2.251637in}{1.983612in}}%
\pgfpathlineto{\pgfqpoint{2.227145in}{2.035853in}}%
\pgfpathlineto{\pgfqpoint{2.210615in}{2.019327in}}%
\pgfpathlineto{\pgfqpoint{2.234419in}{1.968255in}}%
\pgfpathclose%
\pgfusepath{fill}%
\end{pgfscope}%
\begin{pgfscope}%
\pgfpathrectangle{\pgfqpoint{0.000000in}{0.000000in}}{\pgfqpoint{3.000000in}{3.000000in}}%
\pgfusepath{clip}%
\pgfsetbuttcap%
\pgfsetroundjoin%
\definecolor{currentfill}{rgb}{0.000000,0.676471,1.000000}%
\pgfsetfillcolor{currentfill}%
\pgfsetlinewidth{0.000000pt}%
\definecolor{currentstroke}{rgb}{0.000000,0.000000,0.000000}%
\pgfsetstrokecolor{currentstroke}%
\pgfsetdash{}{0pt}%
\pgfpathmoveto{\pgfqpoint{1.567124in}{1.212571in}}%
\pgfpathlineto{\pgfqpoint{1.568965in}{1.245665in}}%
\pgfpathlineto{\pgfqpoint{1.610892in}{1.249674in}}%
\pgfpathlineto{\pgfqpoint{1.606338in}{1.216333in}}%
\pgfpathlineto{\pgfqpoint{1.567124in}{1.212571in}}%
\pgfpathclose%
\pgfusepath{fill}%
\end{pgfscope}%
\begin{pgfscope}%
\pgfpathrectangle{\pgfqpoint{0.000000in}{0.000000in}}{\pgfqpoint{3.000000in}{3.000000in}}%
\pgfusepath{clip}%
\pgfsetbuttcap%
\pgfsetroundjoin%
\definecolor{currentfill}{rgb}{0.578748,1.000000,0.388994}%
\pgfsetfillcolor{currentfill}%
\pgfsetlinewidth{0.000000pt}%
\definecolor{currentstroke}{rgb}{0.000000,0.000000,0.000000}%
\pgfsetstrokecolor{currentstroke}%
\pgfsetdash{}{0pt}%
\pgfpathmoveto{\pgfqpoint{1.889475in}{1.507777in}}%
\pgfpathlineto{\pgfqpoint{1.905789in}{1.530440in}}%
\pgfpathlineto{\pgfqpoint{1.922026in}{1.559812in}}%
\pgfpathlineto{\pgfqpoint{1.905038in}{1.535908in}}%
\pgfpathlineto{\pgfqpoint{1.889475in}{1.507777in}}%
\pgfpathclose%
\pgfusepath{fill}%
\end{pgfscope}%
\begin{pgfscope}%
\pgfpathrectangle{\pgfqpoint{0.000000in}{0.000000in}}{\pgfqpoint{3.000000in}{3.000000in}}%
\pgfusepath{clip}%
\pgfsetbuttcap%
\pgfsetroundjoin%
\definecolor{currentfill}{rgb}{0.743201,1.000000,0.224541}%
\pgfsetfillcolor{currentfill}%
\pgfsetlinewidth{0.000000pt}%
\definecolor{currentstroke}{rgb}{0.000000,0.000000,0.000000}%
\pgfsetstrokecolor{currentstroke}%
\pgfsetdash{}{0pt}%
\pgfpathmoveto{\pgfqpoint{1.136286in}{1.603410in}}%
\pgfpathlineto{\pgfqpoint{1.119054in}{1.625846in}}%
\pgfpathlineto{\pgfqpoint{1.129739in}{1.593336in}}%
\pgfpathlineto{\pgfqpoint{1.146591in}{1.572157in}}%
\pgfpathlineto{\pgfqpoint{1.136286in}{1.603410in}}%
\pgfpathclose%
\pgfusepath{fill}%
\end{pgfscope}%
\begin{pgfscope}%
\pgfpathrectangle{\pgfqpoint{0.000000in}{0.000000in}}{\pgfqpoint{3.000000in}{3.000000in}}%
\pgfusepath{clip}%
\pgfsetbuttcap%
\pgfsetroundjoin%
\definecolor{currentfill}{rgb}{0.000000,0.676471,1.000000}%
\pgfsetfillcolor{currentfill}%
\pgfsetlinewidth{0.000000pt}%
\definecolor{currentstroke}{rgb}{0.000000,0.000000,0.000000}%
\pgfsetstrokecolor{currentstroke}%
\pgfsetdash{}{0pt}%
\pgfpathmoveto{\pgfqpoint{1.487676in}{1.214726in}}%
\pgfpathlineto{\pgfqpoint{1.484017in}{1.247961in}}%
\pgfpathlineto{\pgfqpoint{1.526308in}{1.245089in}}%
\pgfpathlineto{\pgfqpoint{1.527230in}{1.212031in}}%
\pgfpathlineto{\pgfqpoint{1.487676in}{1.214726in}}%
\pgfpathclose%
\pgfusepath{fill}%
\end{pgfscope}%
\begin{pgfscope}%
\pgfpathrectangle{\pgfqpoint{0.000000in}{0.000000in}}{\pgfqpoint{3.000000in}{3.000000in}}%
\pgfusepath{clip}%
\pgfsetbuttcap%
\pgfsetroundjoin%
\definecolor{currentfill}{rgb}{0.000000,0.676471,1.000000}%
\pgfsetfillcolor{currentfill}%
\pgfsetlinewidth{0.000000pt}%
\definecolor{currentstroke}{rgb}{0.000000,0.000000,0.000000}%
\pgfsetstrokecolor{currentstroke}%
\pgfsetdash{}{0pt}%
\pgfpathmoveto{\pgfqpoint{1.527230in}{1.212031in}}%
\pgfpathlineto{\pgfqpoint{1.526308in}{1.245089in}}%
\pgfpathlineto{\pgfqpoint{1.568965in}{1.245665in}}%
\pgfpathlineto{\pgfqpoint{1.567124in}{1.212571in}}%
\pgfpathlineto{\pgfqpoint{1.527230in}{1.212031in}}%
\pgfpathclose%
\pgfusepath{fill}%
\end{pgfscope}%
\begin{pgfscope}%
\pgfpathrectangle{\pgfqpoint{0.000000in}{0.000000in}}{\pgfqpoint{3.000000in}{3.000000in}}%
\pgfusepath{clip}%
\pgfsetbuttcap%
\pgfsetroundjoin%
\definecolor{currentfill}{rgb}{1.000000,0.291213,0.000000}%
\pgfsetfillcolor{currentfill}%
\pgfsetlinewidth{0.000000pt}%
\definecolor{currentstroke}{rgb}{0.000000,0.000000,0.000000}%
\pgfsetstrokecolor{currentstroke}%
\pgfsetdash{}{0pt}%
\pgfpathmoveto{\pgfqpoint{0.927874in}{1.935159in}}%
\pgfpathlineto{\pgfqpoint{0.911088in}{1.952463in}}%
\pgfpathlineto{\pgfqpoint{0.894453in}{1.904350in}}%
\pgfpathlineto{\pgfqpoint{0.911772in}{1.888245in}}%
\pgfpathlineto{\pgfqpoint{0.927874in}{1.935159in}}%
\pgfpathclose%
\pgfusepath{fill}%
\end{pgfscope}%
\begin{pgfscope}%
\pgfpathrectangle{\pgfqpoint{0.000000in}{0.000000in}}{\pgfqpoint{3.000000in}{3.000000in}}%
\pgfusepath{clip}%
\pgfsetbuttcap%
\pgfsetroundjoin%
\definecolor{currentfill}{rgb}{0.927807,0.015251,0.000000}%
\pgfsetfillcolor{currentfill}%
\pgfsetlinewidth{0.000000pt}%
\definecolor{currentstroke}{rgb}{0.000000,0.000000,0.000000}%
\pgfsetstrokecolor{currentstroke}%
\pgfsetdash{}{0pt}%
\pgfpathmoveto{\pgfqpoint{2.251637in}{1.983612in}}%
\pgfpathlineto{\pgfqpoint{2.268863in}{1.998723in}}%
\pgfpathlineto{\pgfqpoint{2.243678in}{2.052132in}}%
\pgfpathlineto{\pgfqpoint{2.227145in}{2.035853in}}%
\pgfpathlineto{\pgfqpoint{2.251637in}{1.983612in}}%
\pgfpathclose%
\pgfusepath{fill}%
\end{pgfscope}%
\begin{pgfscope}%
\pgfpathrectangle{\pgfqpoint{0.000000in}{0.000000in}}{\pgfqpoint{3.000000in}{3.000000in}}%
\pgfusepath{clip}%
\pgfsetbuttcap%
\pgfsetroundjoin%
\definecolor{currentfill}{rgb}{0.401645,1.000000,0.566097}%
\pgfsetfillcolor{currentfill}%
\pgfsetlinewidth{0.000000pt}%
\definecolor{currentstroke}{rgb}{0.000000,0.000000,0.000000}%
\pgfsetstrokecolor{currentstroke}%
\pgfsetdash{}{0pt}%
\pgfpathmoveto{\pgfqpoint{1.834778in}{1.433825in}}%
\pgfpathlineto{\pgfqpoint{1.849948in}{1.458274in}}%
\pgfpathlineto{\pgfqpoint{1.873184in}{1.483728in}}%
\pgfpathlineto{\pgfqpoint{1.856917in}{1.458097in}}%
\pgfpathlineto{\pgfqpoint{1.834778in}{1.433825in}}%
\pgfpathclose%
\pgfusepath{fill}%
\end{pgfscope}%
\begin{pgfscope}%
\pgfpathrectangle{\pgfqpoint{0.000000in}{0.000000in}}{\pgfqpoint{3.000000in}{3.000000in}}%
\pgfusepath{clip}%
\pgfsetbuttcap%
\pgfsetroundjoin%
\definecolor{currentfill}{rgb}{0.199241,1.000000,0.768501}%
\pgfsetfillcolor{currentfill}%
\pgfsetlinewidth{0.000000pt}%
\definecolor{currentstroke}{rgb}{0.000000,0.000000,0.000000}%
\pgfsetstrokecolor{currentstroke}%
\pgfsetdash{}{0pt}%
\pgfpathmoveto{\pgfqpoint{1.284717in}{1.372247in}}%
\pgfpathlineto{\pgfqpoint{1.270055in}{1.400255in}}%
\pgfpathlineto{\pgfqpoint{1.300348in}{1.379835in}}%
\pgfpathlineto{\pgfqpoint{1.313395in}{1.352879in}}%
\pgfpathlineto{\pgfqpoint{1.284717in}{1.372247in}}%
\pgfpathclose%
\pgfusepath{fill}%
\end{pgfscope}%
\begin{pgfscope}%
\pgfpathrectangle{\pgfqpoint{0.000000in}{0.000000in}}{\pgfqpoint{3.000000in}{3.000000in}}%
\pgfusepath{clip}%
\pgfsetbuttcap%
\pgfsetroundjoin%
\definecolor{currentfill}{rgb}{1.000000,0.668845,0.000000}%
\pgfsetfillcolor{currentfill}%
\pgfsetlinewidth{0.000000pt}%
\definecolor{currentstroke}{rgb}{0.000000,0.000000,0.000000}%
\pgfsetstrokecolor{currentstroke}%
\pgfsetdash{}{0pt}%
\pgfpathmoveto{\pgfqpoint{2.067151in}{1.756624in}}%
\pgfpathlineto{\pgfqpoint{2.084513in}{1.774373in}}%
\pgfpathlineto{\pgfqpoint{2.079815in}{1.815841in}}%
\pgfpathlineto{\pgfqpoint{2.062678in}{1.796852in}}%
\pgfpathlineto{\pgfqpoint{2.067151in}{1.756624in}}%
\pgfpathclose%
\pgfusepath{fill}%
\end{pgfscope}%
\begin{pgfscope}%
\pgfpathrectangle{\pgfqpoint{0.000000in}{0.000000in}}{\pgfqpoint{3.000000in}{3.000000in}}%
\pgfusepath{clip}%
\pgfsetbuttcap%
\pgfsetroundjoin%
\definecolor{currentfill}{rgb}{0.085389,1.000000,0.882353}%
\pgfsetfillcolor{currentfill}%
\pgfsetlinewidth{0.000000pt}%
\definecolor{currentstroke}{rgb}{0.000000,0.000000,0.000000}%
\pgfsetstrokecolor{currentstroke}%
\pgfsetdash{}{0pt}%
\pgfpathmoveto{\pgfqpoint{1.326415in}{1.323448in}}%
\pgfpathlineto{\pgfqpoint{1.313395in}{1.352879in}}%
\pgfpathlineto{\pgfqpoint{1.347846in}{1.336011in}}%
\pgfpathlineto{\pgfqpoint{1.358910in}{1.307499in}}%
\pgfpathlineto{\pgfqpoint{1.326415in}{1.323448in}}%
\pgfpathclose%
\pgfusepath{fill}%
\end{pgfscope}%
\begin{pgfscope}%
\pgfpathrectangle{\pgfqpoint{0.000000in}{0.000000in}}{\pgfqpoint{3.000000in}{3.000000in}}%
\pgfusepath{clip}%
\pgfsetbuttcap%
\pgfsetroundjoin%
\definecolor{currentfill}{rgb}{1.000000,0.814089,0.000000}%
\pgfsetfillcolor{currentfill}%
\pgfsetlinewidth{0.000000pt}%
\definecolor{currentstroke}{rgb}{0.000000,0.000000,0.000000}%
\pgfsetstrokecolor{currentstroke}%
\pgfsetdash{}{0pt}%
\pgfpathmoveto{\pgfqpoint{1.049965in}{1.744676in}}%
\pgfpathlineto{\pgfqpoint{1.032727in}{1.764386in}}%
\pgfpathlineto{\pgfqpoint{1.032652in}{1.725290in}}%
\pgfpathlineto{\pgfqpoint{1.049964in}{1.706830in}}%
\pgfpathlineto{\pgfqpoint{1.049965in}{1.744676in}}%
\pgfpathclose%
\pgfusepath{fill}%
\end{pgfscope}%
\begin{pgfscope}%
\pgfpathrectangle{\pgfqpoint{0.000000in}{0.000000in}}{\pgfqpoint{3.000000in}{3.000000in}}%
\pgfusepath{clip}%
\pgfsetbuttcap%
\pgfsetroundjoin%
\definecolor{currentfill}{rgb}{0.856506,0.000000,0.000000}%
\pgfsetfillcolor{currentfill}%
\pgfsetlinewidth{0.000000pt}%
\definecolor{currentstroke}{rgb}{0.000000,0.000000,0.000000}%
\pgfsetstrokecolor{currentstroke}%
\pgfsetdash{}{0pt}%
\pgfpathmoveto{\pgfqpoint{2.268863in}{1.998723in}}%
\pgfpathlineto{\pgfqpoint{2.286096in}{2.013603in}}%
\pgfpathlineto{\pgfqpoint{2.260213in}{2.068177in}}%
\pgfpathlineto{\pgfqpoint{2.243678in}{2.052132in}}%
\pgfpathlineto{\pgfqpoint{2.268863in}{1.998723in}}%
\pgfpathclose%
\pgfusepath{fill}%
\end{pgfscope}%
\begin{pgfscope}%
\pgfpathrectangle{\pgfqpoint{0.000000in}{0.000000in}}{\pgfqpoint{3.000000in}{3.000000in}}%
\pgfusepath{clip}%
\pgfsetbuttcap%
\pgfsetroundjoin%
\definecolor{currentfill}{rgb}{1.000000,0.233115,0.000000}%
\pgfsetfillcolor{currentfill}%
\pgfsetlinewidth{0.000000pt}%
\definecolor{currentstroke}{rgb}{0.000000,0.000000,0.000000}%
\pgfsetstrokecolor{currentstroke}%
\pgfsetdash{}{0pt}%
\pgfpathmoveto{\pgfqpoint{0.911088in}{1.952463in}}%
\pgfpathlineto{\pgfqpoint{0.894298in}{1.969447in}}%
\pgfpathlineto{\pgfqpoint{0.877125in}{1.920138in}}%
\pgfpathlineto{\pgfqpoint{0.894453in}{1.904350in}}%
\pgfpathlineto{\pgfqpoint{0.911088in}{1.952463in}}%
\pgfpathclose%
\pgfusepath{fill}%
\end{pgfscope}%
\begin{pgfscope}%
\pgfpathrectangle{\pgfqpoint{0.000000in}{0.000000in}}{\pgfqpoint{3.000000in}{3.000000in}}%
\pgfusepath{clip}%
\pgfsetbuttcap%
\pgfsetroundjoin%
\definecolor{currentfill}{rgb}{0.000000,0.849020,1.000000}%
\pgfsetfillcolor{currentfill}%
\pgfsetlinewidth{0.000000pt}%
\definecolor{currentstroke}{rgb}{0.000000,0.000000,0.000000}%
\pgfsetstrokecolor{currentstroke}%
\pgfsetdash{}{0pt}%
\pgfpathmoveto{\pgfqpoint{1.651014in}{1.257008in}}%
\pgfpathlineto{\pgfqpoint{1.658178in}{1.287284in}}%
\pgfpathlineto{\pgfqpoint{1.697880in}{1.298391in}}%
\pgfpathlineto{\pgfqpoint{1.688309in}{1.267473in}}%
\pgfpathlineto{\pgfqpoint{1.651014in}{1.257008in}}%
\pgfpathclose%
\pgfusepath{fill}%
\end{pgfscope}%
\begin{pgfscope}%
\pgfpathrectangle{\pgfqpoint{0.000000in}{0.000000in}}{\pgfqpoint{3.000000in}{3.000000in}}%
\pgfusepath{clip}%
\pgfsetbuttcap%
\pgfsetroundjoin%
\definecolor{currentfill}{rgb}{0.895003,1.000000,0.072739}%
\pgfsetfillcolor{currentfill}%
\pgfsetlinewidth{0.000000pt}%
\definecolor{currentstroke}{rgb}{0.000000,0.000000,0.000000}%
\pgfsetstrokecolor{currentstroke}%
\pgfsetdash{}{0pt}%
\pgfpathmoveto{\pgfqpoint{1.973106in}{1.624715in}}%
\pgfpathlineto{\pgfqpoint{1.990171in}{1.644473in}}%
\pgfpathlineto{\pgfqpoint{1.997840in}{1.679699in}}%
\pgfpathlineto{\pgfqpoint{1.980547in}{1.658681in}}%
\pgfpathlineto{\pgfqpoint{1.973106in}{1.624715in}}%
\pgfpathclose%
\pgfusepath{fill}%
\end{pgfscope}%
\begin{pgfscope}%
\pgfpathrectangle{\pgfqpoint{0.000000in}{0.000000in}}{\pgfqpoint{3.000000in}{3.000000in}}%
\pgfusepath{clip}%
\pgfsetbuttcap%
\pgfsetroundjoin%
\definecolor{currentfill}{rgb}{0.803030,0.000000,0.000000}%
\pgfsetfillcolor{currentfill}%
\pgfsetlinewidth{0.000000pt}%
\definecolor{currentstroke}{rgb}{0.000000,0.000000,0.000000}%
\pgfsetstrokecolor{currentstroke}%
\pgfsetdash{}{0pt}%
\pgfpathmoveto{\pgfqpoint{2.286096in}{2.013603in}}%
\pgfpathlineto{\pgfqpoint{2.303337in}{2.028265in}}%
\pgfpathlineto{\pgfqpoint{2.276751in}{2.084001in}}%
\pgfpathlineto{\pgfqpoint{2.260213in}{2.068177in}}%
\pgfpathlineto{\pgfqpoint{2.286096in}{2.013603in}}%
\pgfpathclose%
\pgfusepath{fill}%
\end{pgfscope}%
\begin{pgfscope}%
\pgfpathrectangle{\pgfqpoint{0.000000in}{0.000000in}}{\pgfqpoint{3.000000in}{3.000000in}}%
\pgfusepath{clip}%
\pgfsetbuttcap%
\pgfsetroundjoin%
\definecolor{currentfill}{rgb}{1.000000,0.610748,0.000000}%
\pgfsetfillcolor{currentfill}%
\pgfsetlinewidth{0.000000pt}%
\definecolor{currentstroke}{rgb}{0.000000,0.000000,0.000000}%
\pgfsetstrokecolor{currentstroke}%
\pgfsetdash{}{0pt}%
\pgfpathmoveto{\pgfqpoint{2.084513in}{1.774373in}}%
\pgfpathlineto{\pgfqpoint{2.101890in}{1.791627in}}%
\pgfpathlineto{\pgfqpoint{2.096960in}{1.834332in}}%
\pgfpathlineto{\pgfqpoint{2.079815in}{1.815841in}}%
\pgfpathlineto{\pgfqpoint{2.084513in}{1.774373in}}%
\pgfpathclose%
\pgfusepath{fill}%
\end{pgfscope}%
\begin{pgfscope}%
\pgfpathrectangle{\pgfqpoint{0.000000in}{0.000000in}}{\pgfqpoint{3.000000in}{3.000000in}}%
\pgfusepath{clip}%
\pgfsetbuttcap%
\pgfsetroundjoin%
\definecolor{currentfill}{rgb}{0.819102,1.000000,0.148640}%
\pgfsetfillcolor{currentfill}%
\pgfsetlinewidth{0.000000pt}%
\definecolor{currentstroke}{rgb}{0.000000,0.000000,0.000000}%
\pgfsetstrokecolor{currentstroke}%
\pgfsetdash{}{0pt}%
\pgfpathmoveto{\pgfqpoint{1.119054in}{1.625846in}}%
\pgfpathlineto{\pgfqpoint{1.101805in}{1.647314in}}%
\pgfpathlineto{\pgfqpoint{1.112866in}{1.613546in}}%
\pgfpathlineto{\pgfqpoint{1.129739in}{1.593336in}}%
\pgfpathlineto{\pgfqpoint{1.119054in}{1.625846in}}%
\pgfpathclose%
\pgfusepath{fill}%
\end{pgfscope}%
\begin{pgfscope}%
\pgfpathrectangle{\pgfqpoint{0.000000in}{0.000000in}}{\pgfqpoint{3.000000in}{3.000000in}}%
\pgfusepath{clip}%
\pgfsetbuttcap%
\pgfsetroundjoin%
\definecolor{currentfill}{rgb}{1.000000,0.175018,0.000000}%
\pgfsetfillcolor{currentfill}%
\pgfsetlinewidth{0.000000pt}%
\definecolor{currentstroke}{rgb}{0.000000,0.000000,0.000000}%
\pgfsetstrokecolor{currentstroke}%
\pgfsetdash{}{0pt}%
\pgfpathmoveto{\pgfqpoint{0.894298in}{1.969447in}}%
\pgfpathlineto{\pgfqpoint{0.877504in}{1.986134in}}%
\pgfpathlineto{\pgfqpoint{0.859786in}{1.935631in}}%
\pgfpathlineto{\pgfqpoint{0.877125in}{1.920138in}}%
\pgfpathlineto{\pgfqpoint{0.894298in}{1.969447in}}%
\pgfpathclose%
\pgfusepath{fill}%
\end{pgfscope}%
\begin{pgfscope}%
\pgfpathrectangle{\pgfqpoint{0.000000in}{0.000000in}}{\pgfqpoint{3.000000in}{3.000000in}}%
\pgfusepath{clip}%
\pgfsetbuttcap%
\pgfsetroundjoin%
\definecolor{currentfill}{rgb}{0.578748,1.000000,0.388994}%
\pgfsetfillcolor{currentfill}%
\pgfsetlinewidth{0.000000pt}%
\definecolor{currentstroke}{rgb}{0.000000,0.000000,0.000000}%
\pgfsetstrokecolor{currentstroke}%
\pgfsetdash{}{0pt}%
\pgfpathmoveto{\pgfqpoint{1.180233in}{1.526409in}}%
\pgfpathlineto{\pgfqpoint{1.163422in}{1.549894in}}%
\pgfpathlineto{\pgfqpoint{1.182769in}{1.520959in}}%
\pgfpathlineto{\pgfqpoint{1.198765in}{1.498698in}}%
\pgfpathlineto{\pgfqpoint{1.180233in}{1.526409in}}%
\pgfpathclose%
\pgfusepath{fill}%
\end{pgfscope}%
\begin{pgfscope}%
\pgfpathrectangle{\pgfqpoint{0.000000in}{0.000000in}}{\pgfqpoint{3.000000in}{3.000000in}}%
\pgfusepath{clip}%
\pgfsetbuttcap%
\pgfsetroundjoin%
\definecolor{currentfill}{rgb}{0.000000,0.849020,1.000000}%
\pgfsetfillcolor{currentfill}%
\pgfsetlinewidth{0.000000pt}%
\definecolor{currentstroke}{rgb}{0.000000,0.000000,0.000000}%
\pgfsetstrokecolor{currentstroke}%
\pgfsetdash{}{0pt}%
\pgfpathmoveto{\pgfqpoint{1.404829in}{1.263653in}}%
\pgfpathlineto{\pgfqpoint{1.396035in}{1.294336in}}%
\pgfpathlineto{\pgfqpoint{1.436860in}{1.284307in}}%
\pgfpathlineto{\pgfqpoint{1.443176in}{1.254204in}}%
\pgfpathlineto{\pgfqpoint{1.404829in}{1.263653in}}%
\pgfpathclose%
\pgfusepath{fill}%
\end{pgfscope}%
\begin{pgfscope}%
\pgfpathrectangle{\pgfqpoint{0.000000in}{0.000000in}}{\pgfqpoint{3.000000in}{3.000000in}}%
\pgfusepath{clip}%
\pgfsetbuttcap%
\pgfsetroundjoin%
\definecolor{currentfill}{rgb}{1.000000,0.741467,0.000000}%
\pgfsetfillcolor{currentfill}%
\pgfsetlinewidth{0.000000pt}%
\definecolor{currentstroke}{rgb}{0.000000,0.000000,0.000000}%
\pgfsetstrokecolor{currentstroke}%
\pgfsetdash{}{0pt}%
\pgfpathmoveto{\pgfqpoint{1.032727in}{1.764386in}}%
\pgfpathlineto{\pgfqpoint{1.015478in}{1.783506in}}%
\pgfpathlineto{\pgfqpoint{1.015324in}{1.743160in}}%
\pgfpathlineto{\pgfqpoint{1.032652in}{1.725290in}}%
\pgfpathlineto{\pgfqpoint{1.032727in}{1.764386in}}%
\pgfpathclose%
\pgfusepath{fill}%
\end{pgfscope}%
\begin{pgfscope}%
\pgfpathrectangle{\pgfqpoint{0.000000in}{0.000000in}}{\pgfqpoint{3.000000in}{3.000000in}}%
\pgfusepath{clip}%
\pgfsetbuttcap%
\pgfsetroundjoin%
\definecolor{currentfill}{rgb}{0.401645,1.000000,0.566097}%
\pgfsetfillcolor{currentfill}%
\pgfsetlinewidth{0.000000pt}%
\definecolor{currentstroke}{rgb}{0.000000,0.000000,0.000000}%
\pgfsetstrokecolor{currentstroke}%
\pgfsetdash{}{0pt}%
\pgfpathmoveto{\pgfqpoint{1.230682in}{1.449822in}}%
\pgfpathlineto{\pgfqpoint{1.214736in}{1.475051in}}%
\pgfpathlineto{\pgfqpoint{1.240651in}{1.450223in}}%
\pgfpathlineto{\pgfqpoint{1.255366in}{1.426149in}}%
\pgfpathlineto{\pgfqpoint{1.230682in}{1.449822in}}%
\pgfpathclose%
\pgfusepath{fill}%
\end{pgfscope}%
\begin{pgfscope}%
\pgfpathrectangle{\pgfqpoint{0.000000in}{0.000000in}}{\pgfqpoint{3.000000in}{3.000000in}}%
\pgfusepath{clip}%
\pgfsetbuttcap%
\pgfsetroundjoin%
\definecolor{currentfill}{rgb}{0.667299,1.000000,0.300443}%
\pgfsetfillcolor{currentfill}%
\pgfsetlinewidth{0.000000pt}%
\definecolor{currentstroke}{rgb}{0.000000,0.000000,0.000000}%
\pgfsetstrokecolor{currentstroke}%
\pgfsetdash{}{0pt}%
\pgfpathmoveto{\pgfqpoint{1.905789in}{1.530440in}}%
\pgfpathlineto{\pgfqpoint{1.922126in}{1.551882in}}%
\pgfpathlineto{\pgfqpoint{1.939033in}{1.582492in}}%
\pgfpathlineto{\pgfqpoint{1.922026in}{1.559812in}}%
\pgfpathlineto{\pgfqpoint{1.905789in}{1.530440in}}%
\pgfpathclose%
\pgfusepath{fill}%
\end{pgfscope}%
\begin{pgfscope}%
\pgfpathrectangle{\pgfqpoint{0.000000in}{0.000000in}}{\pgfqpoint{3.000000in}{3.000000in}}%
\pgfusepath{clip}%
\pgfsetbuttcap%
\pgfsetroundjoin%
\definecolor{currentfill}{rgb}{0.731729,0.000000,0.000000}%
\pgfsetfillcolor{currentfill}%
\pgfsetlinewidth{0.000000pt}%
\definecolor{currentstroke}{rgb}{0.000000,0.000000,0.000000}%
\pgfsetstrokecolor{currentstroke}%
\pgfsetdash{}{0pt}%
\pgfpathmoveto{\pgfqpoint{2.303337in}{2.028265in}}%
\pgfpathlineto{\pgfqpoint{2.320585in}{2.042721in}}%
\pgfpathlineto{\pgfqpoint{2.293291in}{2.099616in}}%
\pgfpathlineto{\pgfqpoint{2.276751in}{2.084001in}}%
\pgfpathlineto{\pgfqpoint{2.303337in}{2.028265in}}%
\pgfpathclose%
\pgfusepath{fill}%
\end{pgfscope}%
\begin{pgfscope}%
\pgfpathrectangle{\pgfqpoint{0.000000in}{0.000000in}}{\pgfqpoint{3.000000in}{3.000000in}}%
\pgfusepath{clip}%
\pgfsetbuttcap%
\pgfsetroundjoin%
\definecolor{currentfill}{rgb}{0.300443,1.000000,0.667299}%
\pgfsetfillcolor{currentfill}%
\pgfsetlinewidth{0.000000pt}%
\definecolor{currentstroke}{rgb}{0.000000,0.000000,0.000000}%
\pgfsetstrokecolor{currentstroke}%
\pgfsetdash{}{0pt}%
\pgfpathmoveto{\pgfqpoint{1.791550in}{1.386374in}}%
\pgfpathlineto{\pgfqpoint{1.805202in}{1.411554in}}%
\pgfpathlineto{\pgfqpoint{1.834778in}{1.433825in}}%
\pgfpathlineto{\pgfqpoint{1.819634in}{1.407555in}}%
\pgfpathlineto{\pgfqpoint{1.791550in}{1.386374in}}%
\pgfpathclose%
\pgfusepath{fill}%
\end{pgfscope}%
\begin{pgfscope}%
\pgfpathrectangle{\pgfqpoint{0.000000in}{0.000000in}}{\pgfqpoint{3.000000in}{3.000000in}}%
\pgfusepath{clip}%
\pgfsetbuttcap%
\pgfsetroundjoin%
\definecolor{currentfill}{rgb}{1.000000,0.116921,0.000000}%
\pgfsetfillcolor{currentfill}%
\pgfsetlinewidth{0.000000pt}%
\definecolor{currentstroke}{rgb}{0.000000,0.000000,0.000000}%
\pgfsetstrokecolor{currentstroke}%
\pgfsetdash{}{0pt}%
\pgfpathmoveto{\pgfqpoint{0.877504in}{1.986134in}}%
\pgfpathlineto{\pgfqpoint{0.860705in}{2.002541in}}%
\pgfpathlineto{\pgfqpoint{0.842439in}{1.950845in}}%
\pgfpathlineto{\pgfqpoint{0.859786in}{1.935631in}}%
\pgfpathlineto{\pgfqpoint{0.877504in}{1.986134in}}%
\pgfpathclose%
\pgfusepath{fill}%
\end{pgfscope}%
\begin{pgfscope}%
\pgfpathrectangle{\pgfqpoint{0.000000in}{0.000000in}}{\pgfqpoint{3.000000in}{3.000000in}}%
\pgfusepath{clip}%
\pgfsetbuttcap%
\pgfsetroundjoin%
\definecolor{currentfill}{rgb}{0.085389,1.000000,0.882353}%
\pgfsetfillcolor{currentfill}%
\pgfsetlinewidth{0.000000pt}%
\definecolor{currentstroke}{rgb}{0.000000,0.000000,0.000000}%
\pgfsetstrokecolor{currentstroke}%
\pgfsetdash{}{0pt}%
\pgfpathmoveto{\pgfqpoint{1.697880in}{1.298391in}}%
\pgfpathlineto{\pgfqpoint{1.707474in}{1.326376in}}%
\pgfpathlineto{\pgfqpoint{1.745311in}{1.341327in}}%
\pgfpathlineto{\pgfqpoint{1.733560in}{1.312526in}}%
\pgfpathlineto{\pgfqpoint{1.697880in}{1.298391in}}%
\pgfpathclose%
\pgfusepath{fill}%
\end{pgfscope}%
\begin{pgfscope}%
\pgfpathrectangle{\pgfqpoint{0.000000in}{0.000000in}}{\pgfqpoint{3.000000in}{3.000000in}}%
\pgfusepath{clip}%
\pgfsetbuttcap%
\pgfsetroundjoin%
\definecolor{currentfill}{rgb}{0.678253,0.000000,0.000000}%
\pgfsetfillcolor{currentfill}%
\pgfsetlinewidth{0.000000pt}%
\definecolor{currentstroke}{rgb}{0.000000,0.000000,0.000000}%
\pgfsetstrokecolor{currentstroke}%
\pgfsetdash{}{0pt}%
\pgfpathmoveto{\pgfqpoint{2.320585in}{2.042721in}}%
\pgfpathlineto{\pgfqpoint{2.337840in}{2.056982in}}%
\pgfpathlineto{\pgfqpoint{2.309833in}{2.115034in}}%
\pgfpathlineto{\pgfqpoint{2.293291in}{2.099616in}}%
\pgfpathlineto{\pgfqpoint{2.320585in}{2.042721in}}%
\pgfpathclose%
\pgfusepath{fill}%
\end{pgfscope}%
\begin{pgfscope}%
\pgfpathrectangle{\pgfqpoint{0.000000in}{0.000000in}}{\pgfqpoint{3.000000in}{3.000000in}}%
\pgfusepath{clip}%
\pgfsetbuttcap%
\pgfsetroundjoin%
\definecolor{currentfill}{rgb}{1.000000,0.538126,0.000000}%
\pgfsetfillcolor{currentfill}%
\pgfsetlinewidth{0.000000pt}%
\definecolor{currentstroke}{rgb}{0.000000,0.000000,0.000000}%
\pgfsetstrokecolor{currentstroke}%
\pgfsetdash{}{0pt}%
\pgfpathmoveto{\pgfqpoint{2.101890in}{1.791627in}}%
\pgfpathlineto{\pgfqpoint{2.119279in}{1.808424in}}%
\pgfpathlineto{\pgfqpoint{2.114114in}{1.852365in}}%
\pgfpathlineto{\pgfqpoint{2.096960in}{1.834332in}}%
\pgfpathlineto{\pgfqpoint{2.101890in}{1.791627in}}%
\pgfpathclose%
\pgfusepath{fill}%
\end{pgfscope}%
\begin{pgfscope}%
\pgfpathrectangle{\pgfqpoint{0.000000in}{0.000000in}}{\pgfqpoint{3.000000in}{3.000000in}}%
\pgfusepath{clip}%
\pgfsetbuttcap%
\pgfsetroundjoin%
\definecolor{currentfill}{rgb}{0.958254,0.973856,0.009488}%
\pgfsetfillcolor{currentfill}%
\pgfsetlinewidth{0.000000pt}%
\definecolor{currentstroke}{rgb}{0.000000,0.000000,0.000000}%
\pgfsetstrokecolor{currentstroke}%
\pgfsetdash{}{0pt}%
\pgfpathmoveto{\pgfqpoint{1.990171in}{1.644473in}}%
\pgfpathlineto{\pgfqpoint{2.007255in}{1.663449in}}%
\pgfpathlineto{\pgfqpoint{2.015146in}{1.699933in}}%
\pgfpathlineto{\pgfqpoint{1.997840in}{1.679699in}}%
\pgfpathlineto{\pgfqpoint{1.990171in}{1.644473in}}%
\pgfpathclose%
\pgfusepath{fill}%
\end{pgfscope}%
\begin{pgfscope}%
\pgfpathrectangle{\pgfqpoint{0.000000in}{0.000000in}}{\pgfqpoint{3.000000in}{3.000000in}}%
\pgfusepath{clip}%
\pgfsetbuttcap%
\pgfsetroundjoin%
\definecolor{currentfill}{rgb}{0.199241,1.000000,0.768501}%
\pgfsetfillcolor{currentfill}%
\pgfsetlinewidth{0.000000pt}%
\definecolor{currentstroke}{rgb}{0.000000,0.000000,0.000000}%
\pgfsetstrokecolor{currentstroke}%
\pgfsetdash{}{0pt}%
\pgfpathmoveto{\pgfqpoint{1.745311in}{1.341327in}}%
\pgfpathlineto{\pgfqpoint{1.757087in}{1.367653in}}%
\pgfpathlineto{\pgfqpoint{1.791550in}{1.386374in}}%
\pgfpathlineto{\pgfqpoint{1.777925in}{1.359081in}}%
\pgfpathlineto{\pgfqpoint{1.745311in}{1.341327in}}%
\pgfpathclose%
\pgfusepath{fill}%
\end{pgfscope}%
\begin{pgfscope}%
\pgfpathrectangle{\pgfqpoint{0.000000in}{0.000000in}}{\pgfqpoint{3.000000in}{3.000000in}}%
\pgfusepath{clip}%
\pgfsetbuttcap%
\pgfsetroundjoin%
\definecolor{currentfill}{rgb}{0.000000,0.849020,1.000000}%
\pgfsetfillcolor{currentfill}%
\pgfsetlinewidth{0.000000pt}%
\definecolor{currentstroke}{rgb}{0.000000,0.000000,0.000000}%
\pgfsetstrokecolor{currentstroke}%
\pgfsetdash{}{0pt}%
\pgfpathmoveto{\pgfqpoint{1.610892in}{1.249674in}}%
\pgfpathlineto{\pgfqpoint{1.615457in}{1.279498in}}%
\pgfpathlineto{\pgfqpoint{1.658178in}{1.287284in}}%
\pgfpathlineto{\pgfqpoint{1.651014in}{1.257008in}}%
\pgfpathlineto{\pgfqpoint{1.610892in}{1.249674in}}%
\pgfpathclose%
\pgfusepath{fill}%
\end{pgfscope}%
\begin{pgfscope}%
\pgfpathrectangle{\pgfqpoint{0.000000in}{0.000000in}}{\pgfqpoint{3.000000in}{3.000000in}}%
\pgfusepath{clip}%
\pgfsetbuttcap%
\pgfsetroundjoin%
\definecolor{currentfill}{rgb}{0.490196,1.000000,0.477546}%
\pgfsetfillcolor{currentfill}%
\pgfsetlinewidth{0.000000pt}%
\definecolor{currentstroke}{rgb}{0.000000,0.000000,0.000000}%
\pgfsetstrokecolor{currentstroke}%
\pgfsetdash{}{0pt}%
\pgfpathmoveto{\pgfqpoint{1.849948in}{1.458274in}}%
\pgfpathlineto{\pgfqpoint{1.865143in}{1.481141in}}%
\pgfpathlineto{\pgfqpoint{1.889475in}{1.507777in}}%
\pgfpathlineto{\pgfqpoint{1.873184in}{1.483728in}}%
\pgfpathlineto{\pgfqpoint{1.849948in}{1.458274in}}%
\pgfpathclose%
\pgfusepath{fill}%
\end{pgfscope}%
\begin{pgfscope}%
\pgfpathrectangle{\pgfqpoint{0.000000in}{0.000000in}}{\pgfqpoint{3.000000in}{3.000000in}}%
\pgfusepath{clip}%
\pgfsetbuttcap%
\pgfsetroundjoin%
\definecolor{currentfill}{rgb}{0.606952,0.000000,0.000000}%
\pgfsetfillcolor{currentfill}%
\pgfsetlinewidth{0.000000pt}%
\definecolor{currentstroke}{rgb}{0.000000,0.000000,0.000000}%
\pgfsetstrokecolor{currentstroke}%
\pgfsetdash{}{0pt}%
\pgfpathmoveto{\pgfqpoint{2.337840in}{2.056982in}}%
\pgfpathlineto{\pgfqpoint{2.355103in}{2.071059in}}%
\pgfpathlineto{\pgfqpoint{2.326378in}{2.130265in}}%
\pgfpathlineto{\pgfqpoint{2.309833in}{2.115034in}}%
\pgfpathlineto{\pgfqpoint{2.337840in}{2.056982in}}%
\pgfpathclose%
\pgfusepath{fill}%
\end{pgfscope}%
\begin{pgfscope}%
\pgfpathrectangle{\pgfqpoint{0.000000in}{0.000000in}}{\pgfqpoint{3.000000in}{3.000000in}}%
\pgfusepath{clip}%
\pgfsetbuttcap%
\pgfsetroundjoin%
\definecolor{currentfill}{rgb}{1.000000,0.668845,0.000000}%
\pgfsetfillcolor{currentfill}%
\pgfsetlinewidth{0.000000pt}%
\definecolor{currentstroke}{rgb}{0.000000,0.000000,0.000000}%
\pgfsetstrokecolor{currentstroke}%
\pgfsetdash{}{0pt}%
\pgfpathmoveto{\pgfqpoint{1.015478in}{1.783506in}}%
\pgfpathlineto{\pgfqpoint{0.998219in}{1.802086in}}%
\pgfpathlineto{\pgfqpoint{0.997981in}{1.760492in}}%
\pgfpathlineto{\pgfqpoint{1.015324in}{1.743160in}}%
\pgfpathlineto{\pgfqpoint{1.015478in}{1.783506in}}%
\pgfpathclose%
\pgfusepath{fill}%
\end{pgfscope}%
\begin{pgfscope}%
\pgfpathrectangle{\pgfqpoint{0.000000in}{0.000000in}}{\pgfqpoint{3.000000in}{3.000000in}}%
\pgfusepath{clip}%
\pgfsetbuttcap%
\pgfsetroundjoin%
\definecolor{currentfill}{rgb}{0.999109,0.073348,0.000000}%
\pgfsetfillcolor{currentfill}%
\pgfsetlinewidth{0.000000pt}%
\definecolor{currentstroke}{rgb}{0.000000,0.000000,0.000000}%
\pgfsetstrokecolor{currentstroke}%
\pgfsetdash{}{0pt}%
\pgfpathmoveto{\pgfqpoint{0.860705in}{2.002541in}}%
\pgfpathlineto{\pgfqpoint{0.843902in}{2.018684in}}%
\pgfpathlineto{\pgfqpoint{0.825081in}{1.965800in}}%
\pgfpathlineto{\pgfqpoint{0.842439in}{1.950845in}}%
\pgfpathlineto{\pgfqpoint{0.860705in}{2.002541in}}%
\pgfpathclose%
\pgfusepath{fill}%
\end{pgfscope}%
\begin{pgfscope}%
\pgfpathrectangle{\pgfqpoint{0.000000in}{0.000000in}}{\pgfqpoint{3.000000in}{3.000000in}}%
\pgfusepath{clip}%
\pgfsetbuttcap%
\pgfsetroundjoin%
\definecolor{currentfill}{rgb}{0.000000,0.849020,1.000000}%
\pgfsetfillcolor{currentfill}%
\pgfsetlinewidth{0.000000pt}%
\definecolor{currentstroke}{rgb}{0.000000,0.000000,0.000000}%
\pgfsetstrokecolor{currentstroke}%
\pgfsetdash{}{0pt}%
\pgfpathmoveto{\pgfqpoint{1.443176in}{1.254204in}}%
\pgfpathlineto{\pgfqpoint{1.436860in}{1.284307in}}%
\pgfpathlineto{\pgfqpoint{1.480348in}{1.277679in}}%
\pgfpathlineto{\pgfqpoint{1.484017in}{1.247961in}}%
\pgfpathlineto{\pgfqpoint{1.443176in}{1.254204in}}%
\pgfpathclose%
\pgfusepath{fill}%
\end{pgfscope}%
\begin{pgfscope}%
\pgfpathrectangle{\pgfqpoint{0.000000in}{0.000000in}}{\pgfqpoint{3.000000in}{3.000000in}}%
\pgfusepath{clip}%
\pgfsetbuttcap%
\pgfsetroundjoin%
\definecolor{currentfill}{rgb}{0.085389,1.000000,0.882353}%
\pgfsetfillcolor{currentfill}%
\pgfsetlinewidth{0.000000pt}%
\definecolor{currentstroke}{rgb}{0.000000,0.000000,0.000000}%
\pgfsetstrokecolor{currentstroke}%
\pgfsetdash{}{0pt}%
\pgfpathmoveto{\pgfqpoint{1.358910in}{1.307499in}}%
\pgfpathlineto{\pgfqpoint{1.347846in}{1.336011in}}%
\pgfpathlineto{\pgfqpoint{1.387220in}{1.322086in}}%
\pgfpathlineto{\pgfqpoint{1.396035in}{1.294336in}}%
\pgfpathlineto{\pgfqpoint{1.358910in}{1.307499in}}%
\pgfpathclose%
\pgfusepath{fill}%
\end{pgfscope}%
\begin{pgfscope}%
\pgfpathrectangle{\pgfqpoint{0.000000in}{0.000000in}}{\pgfqpoint{3.000000in}{3.000000in}}%
\pgfusepath{clip}%
\pgfsetbuttcap%
\pgfsetroundjoin%
\definecolor{currentfill}{rgb}{0.895003,1.000000,0.072739}%
\pgfsetfillcolor{currentfill}%
\pgfsetlinewidth{0.000000pt}%
\definecolor{currentstroke}{rgb}{0.000000,0.000000,0.000000}%
\pgfsetstrokecolor{currentstroke}%
\pgfsetdash{}{0pt}%
\pgfpathmoveto{\pgfqpoint{1.101805in}{1.647314in}}%
\pgfpathlineto{\pgfqpoint{1.084541in}{1.667912in}}%
\pgfpathlineto{\pgfqpoint{1.095973in}{1.632888in}}%
\pgfpathlineto{\pgfqpoint{1.112866in}{1.613546in}}%
\pgfpathlineto{\pgfqpoint{1.101805in}{1.647314in}}%
\pgfpathclose%
\pgfusepath{fill}%
\end{pgfscope}%
\begin{pgfscope}%
\pgfpathrectangle{\pgfqpoint{0.000000in}{0.000000in}}{\pgfqpoint{3.000000in}{3.000000in}}%
\pgfusepath{clip}%
\pgfsetbuttcap%
\pgfsetroundjoin%
\definecolor{currentfill}{rgb}{0.300443,1.000000,0.667299}%
\pgfsetfillcolor{currentfill}%
\pgfsetlinewidth{0.000000pt}%
\definecolor{currentstroke}{rgb}{0.000000,0.000000,0.000000}%
\pgfsetstrokecolor{currentstroke}%
\pgfsetdash{}{0pt}%
\pgfpathmoveto{\pgfqpoint{1.270055in}{1.400255in}}%
\pgfpathlineto{\pgfqpoint{1.255366in}{1.426149in}}%
\pgfpathlineto{\pgfqpoint{1.287275in}{1.404676in}}%
\pgfpathlineto{\pgfqpoint{1.300348in}{1.379835in}}%
\pgfpathlineto{\pgfqpoint{1.270055in}{1.400255in}}%
\pgfpathclose%
\pgfusepath{fill}%
\end{pgfscope}%
\begin{pgfscope}%
\pgfpathrectangle{\pgfqpoint{0.000000in}{0.000000in}}{\pgfqpoint{3.000000in}{3.000000in}}%
\pgfusepath{clip}%
\pgfsetbuttcap%
\pgfsetroundjoin%
\definecolor{currentfill}{rgb}{1.000000,0.480029,0.000000}%
\pgfsetfillcolor{currentfill}%
\pgfsetlinewidth{0.000000pt}%
\definecolor{currentstroke}{rgb}{0.000000,0.000000,0.000000}%
\pgfsetstrokecolor{currentstroke}%
\pgfsetdash{}{0pt}%
\pgfpathmoveto{\pgfqpoint{2.119279in}{1.808424in}}%
\pgfpathlineto{\pgfqpoint{2.136683in}{1.824799in}}%
\pgfpathlineto{\pgfqpoint{2.131276in}{1.869973in}}%
\pgfpathlineto{\pgfqpoint{2.114114in}{1.852365in}}%
\pgfpathlineto{\pgfqpoint{2.119279in}{1.808424in}}%
\pgfpathclose%
\pgfusepath{fill}%
\end{pgfscope}%
\begin{pgfscope}%
\pgfpathrectangle{\pgfqpoint{0.000000in}{0.000000in}}{\pgfqpoint{3.000000in}{3.000000in}}%
\pgfusepath{clip}%
\pgfsetbuttcap%
\pgfsetroundjoin%
\definecolor{currentfill}{rgb}{0.553476,0.000000,0.000000}%
\pgfsetfillcolor{currentfill}%
\pgfsetlinewidth{0.000000pt}%
\definecolor{currentstroke}{rgb}{0.000000,0.000000,0.000000}%
\pgfsetstrokecolor{currentstroke}%
\pgfsetdash{}{0pt}%
\pgfpathmoveto{\pgfqpoint{2.355103in}{2.071059in}}%
\pgfpathlineto{\pgfqpoint{2.372374in}{2.084960in}}%
\pgfpathlineto{\pgfqpoint{2.342925in}{2.145317in}}%
\pgfpathlineto{\pgfqpoint{2.326378in}{2.130265in}}%
\pgfpathlineto{\pgfqpoint{2.355103in}{2.071059in}}%
\pgfpathclose%
\pgfusepath{fill}%
\end{pgfscope}%
\begin{pgfscope}%
\pgfpathrectangle{\pgfqpoint{0.000000in}{0.000000in}}{\pgfqpoint{3.000000in}{3.000000in}}%
\pgfusepath{clip}%
\pgfsetbuttcap%
\pgfsetroundjoin%
\definecolor{currentfill}{rgb}{0.927807,0.015251,0.000000}%
\pgfsetfillcolor{currentfill}%
\pgfsetlinewidth{0.000000pt}%
\definecolor{currentstroke}{rgb}{0.000000,0.000000,0.000000}%
\pgfsetstrokecolor{currentstroke}%
\pgfsetdash{}{0pt}%
\pgfpathmoveto{\pgfqpoint{0.843902in}{2.018684in}}%
\pgfpathlineto{\pgfqpoint{0.827095in}{2.034581in}}%
\pgfpathlineto{\pgfqpoint{0.807714in}{1.980510in}}%
\pgfpathlineto{\pgfqpoint{0.825081in}{1.965800in}}%
\pgfpathlineto{\pgfqpoint{0.843902in}{2.018684in}}%
\pgfpathclose%
\pgfusepath{fill}%
\end{pgfscope}%
\begin{pgfscope}%
\pgfpathrectangle{\pgfqpoint{0.000000in}{0.000000in}}{\pgfqpoint{3.000000in}{3.000000in}}%
\pgfusepath{clip}%
\pgfsetbuttcap%
\pgfsetroundjoin%
\definecolor{currentfill}{rgb}{0.667299,1.000000,0.300443}%
\pgfsetfillcolor{currentfill}%
\pgfsetlinewidth{0.000000pt}%
\definecolor{currentstroke}{rgb}{0.000000,0.000000,0.000000}%
\pgfsetstrokecolor{currentstroke}%
\pgfsetdash{}{0pt}%
\pgfpathmoveto{\pgfqpoint{1.163422in}{1.549894in}}%
\pgfpathlineto{\pgfqpoint{1.146591in}{1.572157in}}%
\pgfpathlineto{\pgfqpoint{1.166750in}{1.541998in}}%
\pgfpathlineto{\pgfqpoint{1.182769in}{1.520959in}}%
\pgfpathlineto{\pgfqpoint{1.163422in}{1.549894in}}%
\pgfpathclose%
\pgfusepath{fill}%
\end{pgfscope}%
\begin{pgfscope}%
\pgfpathrectangle{\pgfqpoint{0.000000in}{0.000000in}}{\pgfqpoint{3.000000in}{3.000000in}}%
\pgfusepath{clip}%
\pgfsetbuttcap%
\pgfsetroundjoin%
\definecolor{currentfill}{rgb}{0.000000,0.849020,1.000000}%
\pgfsetfillcolor{currentfill}%
\pgfsetlinewidth{0.000000pt}%
\definecolor{currentstroke}{rgb}{0.000000,0.000000,0.000000}%
\pgfsetstrokecolor{currentstroke}%
\pgfsetdash{}{0pt}%
\pgfpathmoveto{\pgfqpoint{1.568965in}{1.245665in}}%
\pgfpathlineto{\pgfqpoint{1.570810in}{1.275242in}}%
\pgfpathlineto{\pgfqpoint{1.615457in}{1.279498in}}%
\pgfpathlineto{\pgfqpoint{1.610892in}{1.249674in}}%
\pgfpathlineto{\pgfqpoint{1.568965in}{1.245665in}}%
\pgfpathclose%
\pgfusepath{fill}%
\end{pgfscope}%
\begin{pgfscope}%
\pgfpathrectangle{\pgfqpoint{0.000000in}{0.000000in}}{\pgfqpoint{3.000000in}{3.000000in}}%
\pgfusepath{clip}%
\pgfsetbuttcap%
\pgfsetroundjoin%
\definecolor{currentfill}{rgb}{0.199241,1.000000,0.768501}%
\pgfsetfillcolor{currentfill}%
\pgfsetlinewidth{0.000000pt}%
\definecolor{currentstroke}{rgb}{0.000000,0.000000,0.000000}%
\pgfsetstrokecolor{currentstroke}%
\pgfsetdash{}{0pt}%
\pgfpathmoveto{\pgfqpoint{1.313395in}{1.352879in}}%
\pgfpathlineto{\pgfqpoint{1.300348in}{1.379835in}}%
\pgfpathlineto{\pgfqpoint{1.336758in}{1.362046in}}%
\pgfpathlineto{\pgfqpoint{1.347846in}{1.336011in}}%
\pgfpathlineto{\pgfqpoint{1.313395in}{1.352879in}}%
\pgfpathclose%
\pgfusepath{fill}%
\end{pgfscope}%
\begin{pgfscope}%
\pgfpathrectangle{\pgfqpoint{0.000000in}{0.000000in}}{\pgfqpoint{3.000000in}{3.000000in}}%
\pgfusepath{clip}%
\pgfsetbuttcap%
\pgfsetroundjoin%
\definecolor{currentfill}{rgb}{0.743201,1.000000,0.224541}%
\pgfsetfillcolor{currentfill}%
\pgfsetlinewidth{0.000000pt}%
\definecolor{currentstroke}{rgb}{0.000000,0.000000,0.000000}%
\pgfsetstrokecolor{currentstroke}%
\pgfsetdash{}{0pt}%
\pgfpathmoveto{\pgfqpoint{1.922126in}{1.551882in}}%
\pgfpathlineto{\pgfqpoint{1.938486in}{1.572239in}}%
\pgfpathlineto{\pgfqpoint{1.956060in}{1.604088in}}%
\pgfpathlineto{\pgfqpoint{1.939033in}{1.582492in}}%
\pgfpathlineto{\pgfqpoint{1.922126in}{1.551882in}}%
\pgfpathclose%
\pgfusepath{fill}%
\end{pgfscope}%
\begin{pgfscope}%
\pgfpathrectangle{\pgfqpoint{0.000000in}{0.000000in}}{\pgfqpoint{3.000000in}{3.000000in}}%
\pgfusepath{clip}%
\pgfsetbuttcap%
\pgfsetroundjoin%
\definecolor{currentfill}{rgb}{0.000000,0.849020,1.000000}%
\pgfsetfillcolor{currentfill}%
\pgfsetlinewidth{0.000000pt}%
\definecolor{currentstroke}{rgb}{0.000000,0.000000,0.000000}%
\pgfsetstrokecolor{currentstroke}%
\pgfsetdash{}{0pt}%
\pgfpathmoveto{\pgfqpoint{1.484017in}{1.247961in}}%
\pgfpathlineto{\pgfqpoint{1.480348in}{1.277679in}}%
\pgfpathlineto{\pgfqpoint{1.525384in}{1.274630in}}%
\pgfpathlineto{\pgfqpoint{1.526308in}{1.245089in}}%
\pgfpathlineto{\pgfqpoint{1.484017in}{1.247961in}}%
\pgfpathclose%
\pgfusepath{fill}%
\end{pgfscope}%
\begin{pgfscope}%
\pgfpathrectangle{\pgfqpoint{0.000000in}{0.000000in}}{\pgfqpoint{3.000000in}{3.000000in}}%
\pgfusepath{clip}%
\pgfsetbuttcap%
\pgfsetroundjoin%
\definecolor{currentfill}{rgb}{0.500000,0.000000,0.000000}%
\pgfsetfillcolor{currentfill}%
\pgfsetlinewidth{0.000000pt}%
\definecolor{currentstroke}{rgb}{0.000000,0.000000,0.000000}%
\pgfsetstrokecolor{currentstroke}%
\pgfsetdash{}{0pt}%
\pgfpathmoveto{\pgfqpoint{2.372374in}{2.084960in}}%
\pgfpathlineto{\pgfqpoint{2.389651in}{2.098695in}}%
\pgfpathlineto{\pgfqpoint{2.359475in}{2.160201in}}%
\pgfpathlineto{\pgfqpoint{2.342925in}{2.145317in}}%
\pgfpathlineto{\pgfqpoint{2.372374in}{2.084960in}}%
\pgfpathclose%
\pgfusepath{fill}%
\end{pgfscope}%
\begin{pgfscope}%
\pgfpathrectangle{\pgfqpoint{0.000000in}{0.000000in}}{\pgfqpoint{3.000000in}{3.000000in}}%
\pgfusepath{clip}%
\pgfsetbuttcap%
\pgfsetroundjoin%
\definecolor{currentfill}{rgb}{1.000000,0.886710,0.000000}%
\pgfsetfillcolor{currentfill}%
\pgfsetlinewidth{0.000000pt}%
\definecolor{currentstroke}{rgb}{0.000000,0.000000,0.000000}%
\pgfsetstrokecolor{currentstroke}%
\pgfsetdash{}{0pt}%
\pgfpathmoveto{\pgfqpoint{2.007255in}{1.663449in}}%
\pgfpathlineto{\pgfqpoint{2.024358in}{1.681716in}}%
\pgfpathlineto{\pgfqpoint{2.032467in}{1.719457in}}%
\pgfpathlineto{\pgfqpoint{2.015146in}{1.699933in}}%
\pgfpathlineto{\pgfqpoint{2.007255in}{1.663449in}}%
\pgfpathclose%
\pgfusepath{fill}%
\end{pgfscope}%
\begin{pgfscope}%
\pgfpathrectangle{\pgfqpoint{0.000000in}{0.000000in}}{\pgfqpoint{3.000000in}{3.000000in}}%
\pgfusepath{clip}%
\pgfsetbuttcap%
\pgfsetroundjoin%
\definecolor{currentfill}{rgb}{1.000000,0.610748,0.000000}%
\pgfsetfillcolor{currentfill}%
\pgfsetlinewidth{0.000000pt}%
\definecolor{currentstroke}{rgb}{0.000000,0.000000,0.000000}%
\pgfsetstrokecolor{currentstroke}%
\pgfsetdash{}{0pt}%
\pgfpathmoveto{\pgfqpoint{0.998219in}{1.802086in}}%
\pgfpathlineto{\pgfqpoint{0.980950in}{1.820168in}}%
\pgfpathlineto{\pgfqpoint{0.980623in}{1.777329in}}%
\pgfpathlineto{\pgfqpoint{0.997981in}{1.760492in}}%
\pgfpathlineto{\pgfqpoint{0.998219in}{1.802086in}}%
\pgfpathclose%
\pgfusepath{fill}%
\end{pgfscope}%
\begin{pgfscope}%
\pgfpathrectangle{\pgfqpoint{0.000000in}{0.000000in}}{\pgfqpoint{3.000000in}{3.000000in}}%
\pgfusepath{clip}%
\pgfsetbuttcap%
\pgfsetroundjoin%
\definecolor{currentfill}{rgb}{0.000000,0.849020,1.000000}%
\pgfsetfillcolor{currentfill}%
\pgfsetlinewidth{0.000000pt}%
\definecolor{currentstroke}{rgb}{0.000000,0.000000,0.000000}%
\pgfsetstrokecolor{currentstroke}%
\pgfsetdash{}{0pt}%
\pgfpathmoveto{\pgfqpoint{1.526308in}{1.245089in}}%
\pgfpathlineto{\pgfqpoint{1.525384in}{1.274630in}}%
\pgfpathlineto{\pgfqpoint{1.570810in}{1.275242in}}%
\pgfpathlineto{\pgfqpoint{1.568965in}{1.245665in}}%
\pgfpathlineto{\pgfqpoint{1.526308in}{1.245089in}}%
\pgfpathclose%
\pgfusepath{fill}%
\end{pgfscope}%
\begin{pgfscope}%
\pgfpathrectangle{\pgfqpoint{0.000000in}{0.000000in}}{\pgfqpoint{3.000000in}{3.000000in}}%
\pgfusepath{clip}%
\pgfsetbuttcap%
\pgfsetroundjoin%
\definecolor{currentfill}{rgb}{0.490196,1.000000,0.477546}%
\pgfsetfillcolor{currentfill}%
\pgfsetlinewidth{0.000000pt}%
\definecolor{currentstroke}{rgb}{0.000000,0.000000,0.000000}%
\pgfsetstrokecolor{currentstroke}%
\pgfsetdash{}{0pt}%
\pgfpathmoveto{\pgfqpoint{1.214736in}{1.475051in}}%
\pgfpathlineto{\pgfqpoint{1.198765in}{1.498698in}}%
\pgfpathlineto{\pgfqpoint{1.225910in}{1.472715in}}%
\pgfpathlineto{\pgfqpoint{1.240651in}{1.450223in}}%
\pgfpathlineto{\pgfqpoint{1.214736in}{1.475051in}}%
\pgfpathclose%
\pgfusepath{fill}%
\end{pgfscope}%
\begin{pgfscope}%
\pgfpathrectangle{\pgfqpoint{0.000000in}{0.000000in}}{\pgfqpoint{3.000000in}{3.000000in}}%
\pgfusepath{clip}%
\pgfsetbuttcap%
\pgfsetroundjoin%
\definecolor{currentfill}{rgb}{0.856506,0.000000,0.000000}%
\pgfsetfillcolor{currentfill}%
\pgfsetlinewidth{0.000000pt}%
\definecolor{currentstroke}{rgb}{0.000000,0.000000,0.000000}%
\pgfsetstrokecolor{currentstroke}%
\pgfsetdash{}{0pt}%
\pgfpathmoveto{\pgfqpoint{0.827095in}{2.034581in}}%
\pgfpathlineto{\pgfqpoint{0.810283in}{2.050245in}}%
\pgfpathlineto{\pgfqpoint{0.790338in}{1.994990in}}%
\pgfpathlineto{\pgfqpoint{0.807714in}{1.980510in}}%
\pgfpathlineto{\pgfqpoint{0.827095in}{2.034581in}}%
\pgfpathclose%
\pgfusepath{fill}%
\end{pgfscope}%
\begin{pgfscope}%
\pgfpathrectangle{\pgfqpoint{0.000000in}{0.000000in}}{\pgfqpoint{3.000000in}{3.000000in}}%
\pgfusepath{clip}%
\pgfsetbuttcap%
\pgfsetroundjoin%
\definecolor{currentfill}{rgb}{1.000000,0.407407,0.000000}%
\pgfsetfillcolor{currentfill}%
\pgfsetlinewidth{0.000000pt}%
\definecolor{currentstroke}{rgb}{0.000000,0.000000,0.000000}%
\pgfsetstrokecolor{currentstroke}%
\pgfsetdash{}{0pt}%
\pgfpathmoveto{\pgfqpoint{2.136683in}{1.824799in}}%
\pgfpathlineto{\pgfqpoint{2.154099in}{1.840784in}}%
\pgfpathlineto{\pgfqpoint{2.148447in}{1.887188in}}%
\pgfpathlineto{\pgfqpoint{2.131276in}{1.869973in}}%
\pgfpathlineto{\pgfqpoint{2.136683in}{1.824799in}}%
\pgfpathclose%
\pgfusepath{fill}%
\end{pgfscope}%
\begin{pgfscope}%
\pgfpathrectangle{\pgfqpoint{0.000000in}{0.000000in}}{\pgfqpoint{3.000000in}{3.000000in}}%
\pgfusepath{clip}%
\pgfsetbuttcap%
\pgfsetroundjoin%
\definecolor{currentfill}{rgb}{0.085389,1.000000,0.882353}%
\pgfsetfillcolor{currentfill}%
\pgfsetlinewidth{0.000000pt}%
\definecolor{currentstroke}{rgb}{0.000000,0.000000,0.000000}%
\pgfsetstrokecolor{currentstroke}%
\pgfsetdash{}{0pt}%
\pgfpathmoveto{\pgfqpoint{1.658178in}{1.287284in}}%
\pgfpathlineto{\pgfqpoint{1.665359in}{1.314625in}}%
\pgfpathlineto{\pgfqpoint{1.707474in}{1.326376in}}%
\pgfpathlineto{\pgfqpoint{1.697880in}{1.298391in}}%
\pgfpathlineto{\pgfqpoint{1.658178in}{1.287284in}}%
\pgfpathclose%
\pgfusepath{fill}%
\end{pgfscope}%
\begin{pgfscope}%
\pgfpathrectangle{\pgfqpoint{0.000000in}{0.000000in}}{\pgfqpoint{3.000000in}{3.000000in}}%
\pgfusepath{clip}%
\pgfsetbuttcap%
\pgfsetroundjoin%
\definecolor{currentfill}{rgb}{0.401645,1.000000,0.566097}%
\pgfsetfillcolor{currentfill}%
\pgfsetlinewidth{0.000000pt}%
\definecolor{currentstroke}{rgb}{0.000000,0.000000,0.000000}%
\pgfsetstrokecolor{currentstroke}%
\pgfsetdash{}{0pt}%
\pgfpathmoveto{\pgfqpoint{1.805202in}{1.411554in}}%
\pgfpathlineto{\pgfqpoint{1.818880in}{1.434912in}}%
\pgfpathlineto{\pgfqpoint{1.849948in}{1.458274in}}%
\pgfpathlineto{\pgfqpoint{1.834778in}{1.433825in}}%
\pgfpathlineto{\pgfqpoint{1.805202in}{1.411554in}}%
\pgfpathclose%
\pgfusepath{fill}%
\end{pgfscope}%
\begin{pgfscope}%
\pgfpathrectangle{\pgfqpoint{0.000000in}{0.000000in}}{\pgfqpoint{3.000000in}{3.000000in}}%
\pgfusepath{clip}%
\pgfsetbuttcap%
\pgfsetroundjoin%
\definecolor{currentfill}{rgb}{0.958254,0.973856,0.009488}%
\pgfsetfillcolor{currentfill}%
\pgfsetlinewidth{0.000000pt}%
\definecolor{currentstroke}{rgb}{0.000000,0.000000,0.000000}%
\pgfsetstrokecolor{currentstroke}%
\pgfsetdash{}{0pt}%
\pgfpathmoveto{\pgfqpoint{1.084541in}{1.667912in}}%
\pgfpathlineto{\pgfqpoint{1.067260in}{1.687726in}}%
\pgfpathlineto{\pgfqpoint{1.079059in}{1.651448in}}%
\pgfpathlineto{\pgfqpoint{1.095973in}{1.632888in}}%
\pgfpathlineto{\pgfqpoint{1.084541in}{1.667912in}}%
\pgfpathclose%
\pgfusepath{fill}%
\end{pgfscope}%
\begin{pgfscope}%
\pgfpathrectangle{\pgfqpoint{0.000000in}{0.000000in}}{\pgfqpoint{3.000000in}{3.000000in}}%
\pgfusepath{clip}%
\pgfsetbuttcap%
\pgfsetroundjoin%
\definecolor{currentfill}{rgb}{0.803030,0.000000,0.000000}%
\pgfsetfillcolor{currentfill}%
\pgfsetlinewidth{0.000000pt}%
\definecolor{currentstroke}{rgb}{0.000000,0.000000,0.000000}%
\pgfsetstrokecolor{currentstroke}%
\pgfsetdash{}{0pt}%
\pgfpathmoveto{\pgfqpoint{0.810283in}{2.050245in}}%
\pgfpathlineto{\pgfqpoint{0.793467in}{2.065689in}}%
\pgfpathlineto{\pgfqpoint{0.772953in}{2.009252in}}%
\pgfpathlineto{\pgfqpoint{0.790338in}{1.994990in}}%
\pgfpathlineto{\pgfqpoint{0.810283in}{2.050245in}}%
\pgfpathclose%
\pgfusepath{fill}%
\end{pgfscope}%
\begin{pgfscope}%
\pgfpathrectangle{\pgfqpoint{0.000000in}{0.000000in}}{\pgfqpoint{3.000000in}{3.000000in}}%
\pgfusepath{clip}%
\pgfsetbuttcap%
\pgfsetroundjoin%
\definecolor{currentfill}{rgb}{0.578748,1.000000,0.388994}%
\pgfsetfillcolor{currentfill}%
\pgfsetlinewidth{0.000000pt}%
\definecolor{currentstroke}{rgb}{0.000000,0.000000,0.000000}%
\pgfsetstrokecolor{currentstroke}%
\pgfsetdash{}{0pt}%
\pgfpathmoveto{\pgfqpoint{1.865143in}{1.481141in}}%
\pgfpathlineto{\pgfqpoint{1.880364in}{1.502622in}}%
\pgfpathlineto{\pgfqpoint{1.905789in}{1.530440in}}%
\pgfpathlineto{\pgfqpoint{1.889475in}{1.507777in}}%
\pgfpathlineto{\pgfqpoint{1.865143in}{1.481141in}}%
\pgfpathclose%
\pgfusepath{fill}%
\end{pgfscope}%
\begin{pgfscope}%
\pgfpathrectangle{\pgfqpoint{0.000000in}{0.000000in}}{\pgfqpoint{3.000000in}{3.000000in}}%
\pgfusepath{clip}%
\pgfsetbuttcap%
\pgfsetroundjoin%
\definecolor{currentfill}{rgb}{1.000000,0.538126,0.000000}%
\pgfsetfillcolor{currentfill}%
\pgfsetlinewidth{0.000000pt}%
\definecolor{currentstroke}{rgb}{0.000000,0.000000,0.000000}%
\pgfsetstrokecolor{currentstroke}%
\pgfsetdash{}{0pt}%
\pgfpathmoveto{\pgfqpoint{0.980950in}{1.820168in}}%
\pgfpathlineto{\pgfqpoint{0.963670in}{1.837792in}}%
\pgfpathlineto{\pgfqpoint{0.963250in}{1.793710in}}%
\pgfpathlineto{\pgfqpoint{0.980623in}{1.777329in}}%
\pgfpathlineto{\pgfqpoint{0.980950in}{1.820168in}}%
\pgfpathclose%
\pgfusepath{fill}%
\end{pgfscope}%
\begin{pgfscope}%
\pgfpathrectangle{\pgfqpoint{0.000000in}{0.000000in}}{\pgfqpoint{3.000000in}{3.000000in}}%
\pgfusepath{clip}%
\pgfsetbuttcap%
\pgfsetroundjoin%
\definecolor{currentfill}{rgb}{0.085389,1.000000,0.882353}%
\pgfsetfillcolor{currentfill}%
\pgfsetlinewidth{0.000000pt}%
\definecolor{currentstroke}{rgb}{0.000000,0.000000,0.000000}%
\pgfsetstrokecolor{currentstroke}%
\pgfsetdash{}{0pt}%
\pgfpathmoveto{\pgfqpoint{1.396035in}{1.294336in}}%
\pgfpathlineto{\pgfqpoint{1.387220in}{1.322086in}}%
\pgfpathlineto{\pgfqpoint{1.430529in}{1.311475in}}%
\pgfpathlineto{\pgfqpoint{1.436860in}{1.284307in}}%
\pgfpathlineto{\pgfqpoint{1.396035in}{1.294336in}}%
\pgfpathclose%
\pgfusepath{fill}%
\end{pgfscope}%
\begin{pgfscope}%
\pgfpathrectangle{\pgfqpoint{0.000000in}{0.000000in}}{\pgfqpoint{3.000000in}{3.000000in}}%
\pgfusepath{clip}%
\pgfsetbuttcap%
\pgfsetroundjoin%
\definecolor{currentfill}{rgb}{1.000000,0.349310,0.000000}%
\pgfsetfillcolor{currentfill}%
\pgfsetlinewidth{0.000000pt}%
\definecolor{currentstroke}{rgb}{0.000000,0.000000,0.000000}%
\pgfsetstrokecolor{currentstroke}%
\pgfsetdash{}{0pt}%
\pgfpathmoveto{\pgfqpoint{2.154099in}{1.840784in}}%
\pgfpathlineto{\pgfqpoint{2.171529in}{1.856405in}}%
\pgfpathlineto{\pgfqpoint{2.165625in}{1.904038in}}%
\pgfpathlineto{\pgfqpoint{2.148447in}{1.887188in}}%
\pgfpathlineto{\pgfqpoint{2.154099in}{1.840784in}}%
\pgfpathclose%
\pgfusepath{fill}%
\end{pgfscope}%
\begin{pgfscope}%
\pgfpathrectangle{\pgfqpoint{0.000000in}{0.000000in}}{\pgfqpoint{3.000000in}{3.000000in}}%
\pgfusepath{clip}%
\pgfsetbuttcap%
\pgfsetroundjoin%
\definecolor{currentfill}{rgb}{1.000000,0.814089,0.000000}%
\pgfsetfillcolor{currentfill}%
\pgfsetlinewidth{0.000000pt}%
\definecolor{currentstroke}{rgb}{0.000000,0.000000,0.000000}%
\pgfsetstrokecolor{currentstroke}%
\pgfsetdash{}{0pt}%
\pgfpathmoveto{\pgfqpoint{2.024358in}{1.681716in}}%
\pgfpathlineto{\pgfqpoint{2.041480in}{1.699339in}}%
\pgfpathlineto{\pgfqpoint{2.049802in}{1.738335in}}%
\pgfpathlineto{\pgfqpoint{2.032467in}{1.719457in}}%
\pgfpathlineto{\pgfqpoint{2.024358in}{1.681716in}}%
\pgfpathclose%
\pgfusepath{fill}%
\end{pgfscope}%
\begin{pgfscope}%
\pgfpathrectangle{\pgfqpoint{0.000000in}{0.000000in}}{\pgfqpoint{3.000000in}{3.000000in}}%
\pgfusepath{clip}%
\pgfsetbuttcap%
\pgfsetroundjoin%
\definecolor{currentfill}{rgb}{0.731729,0.000000,0.000000}%
\pgfsetfillcolor{currentfill}%
\pgfsetlinewidth{0.000000pt}%
\definecolor{currentstroke}{rgb}{0.000000,0.000000,0.000000}%
\pgfsetstrokecolor{currentstroke}%
\pgfsetdash{}{0pt}%
\pgfpathmoveto{\pgfqpoint{0.793467in}{2.065689in}}%
\pgfpathlineto{\pgfqpoint{0.776647in}{2.080925in}}%
\pgfpathlineto{\pgfqpoint{0.755558in}{2.023308in}}%
\pgfpathlineto{\pgfqpoint{0.772953in}{2.009252in}}%
\pgfpathlineto{\pgfqpoint{0.793467in}{2.065689in}}%
\pgfpathclose%
\pgfusepath{fill}%
\end{pgfscope}%
\begin{pgfscope}%
\pgfpathrectangle{\pgfqpoint{0.000000in}{0.000000in}}{\pgfqpoint{3.000000in}{3.000000in}}%
\pgfusepath{clip}%
\pgfsetbuttcap%
\pgfsetroundjoin%
\definecolor{currentfill}{rgb}{0.819102,1.000000,0.148640}%
\pgfsetfillcolor{currentfill}%
\pgfsetlinewidth{0.000000pt}%
\definecolor{currentstroke}{rgb}{0.000000,0.000000,0.000000}%
\pgfsetstrokecolor{currentstroke}%
\pgfsetdash{}{0pt}%
\pgfpathmoveto{\pgfqpoint{1.938486in}{1.572239in}}%
\pgfpathlineto{\pgfqpoint{1.954869in}{1.591628in}}%
\pgfpathlineto{\pgfqpoint{1.973106in}{1.624715in}}%
\pgfpathlineto{\pgfqpoint{1.956060in}{1.604088in}}%
\pgfpathlineto{\pgfqpoint{1.938486in}{1.572239in}}%
\pgfpathclose%
\pgfusepath{fill}%
\end{pgfscope}%
\begin{pgfscope}%
\pgfpathrectangle{\pgfqpoint{0.000000in}{0.000000in}}{\pgfqpoint{3.000000in}{3.000000in}}%
\pgfusepath{clip}%
\pgfsetbuttcap%
\pgfsetroundjoin%
\definecolor{currentfill}{rgb}{0.743201,1.000000,0.224541}%
\pgfsetfillcolor{currentfill}%
\pgfsetlinewidth{0.000000pt}%
\definecolor{currentstroke}{rgb}{0.000000,0.000000,0.000000}%
\pgfsetstrokecolor{currentstroke}%
\pgfsetdash{}{0pt}%
\pgfpathmoveto{\pgfqpoint{1.146591in}{1.572157in}}%
\pgfpathlineto{\pgfqpoint{1.129739in}{1.593336in}}%
\pgfpathlineto{\pgfqpoint{1.150706in}{1.561953in}}%
\pgfpathlineto{\pgfqpoint{1.166750in}{1.541998in}}%
\pgfpathlineto{\pgfqpoint{1.146591in}{1.572157in}}%
\pgfpathclose%
\pgfusepath{fill}%
\end{pgfscope}%
\begin{pgfscope}%
\pgfpathrectangle{\pgfqpoint{0.000000in}{0.000000in}}{\pgfqpoint{3.000000in}{3.000000in}}%
\pgfusepath{clip}%
\pgfsetbuttcap%
\pgfsetroundjoin%
\definecolor{currentfill}{rgb}{0.300443,1.000000,0.667299}%
\pgfsetfillcolor{currentfill}%
\pgfsetlinewidth{0.000000pt}%
\definecolor{currentstroke}{rgb}{0.000000,0.000000,0.000000}%
\pgfsetstrokecolor{currentstroke}%
\pgfsetdash{}{0pt}%
\pgfpathmoveto{\pgfqpoint{1.757087in}{1.367653in}}%
\pgfpathlineto{\pgfqpoint{1.768889in}{1.391864in}}%
\pgfpathlineto{\pgfqpoint{1.805202in}{1.411554in}}%
\pgfpathlineto{\pgfqpoint{1.791550in}{1.386374in}}%
\pgfpathlineto{\pgfqpoint{1.757087in}{1.367653in}}%
\pgfpathclose%
\pgfusepath{fill}%
\end{pgfscope}%
\begin{pgfscope}%
\pgfpathrectangle{\pgfqpoint{0.000000in}{0.000000in}}{\pgfqpoint{3.000000in}{3.000000in}}%
\pgfusepath{clip}%
\pgfsetbuttcap%
\pgfsetroundjoin%
\definecolor{currentfill}{rgb}{0.199241,1.000000,0.768501}%
\pgfsetfillcolor{currentfill}%
\pgfsetlinewidth{0.000000pt}%
\definecolor{currentstroke}{rgb}{0.000000,0.000000,0.000000}%
\pgfsetstrokecolor{currentstroke}%
\pgfsetdash{}{0pt}%
\pgfpathmoveto{\pgfqpoint{1.707474in}{1.326376in}}%
\pgfpathlineto{\pgfqpoint{1.717090in}{1.351884in}}%
\pgfpathlineto{\pgfqpoint{1.757087in}{1.367653in}}%
\pgfpathlineto{\pgfqpoint{1.745311in}{1.341327in}}%
\pgfpathlineto{\pgfqpoint{1.707474in}{1.326376in}}%
\pgfpathclose%
\pgfusepath{fill}%
\end{pgfscope}%
\begin{pgfscope}%
\pgfpathrectangle{\pgfqpoint{0.000000in}{0.000000in}}{\pgfqpoint{3.000000in}{3.000000in}}%
\pgfusepath{clip}%
\pgfsetbuttcap%
\pgfsetroundjoin%
\definecolor{currentfill}{rgb}{0.401645,1.000000,0.566097}%
\pgfsetfillcolor{currentfill}%
\pgfsetlinewidth{0.000000pt}%
\definecolor{currentstroke}{rgb}{0.000000,0.000000,0.000000}%
\pgfsetstrokecolor{currentstroke}%
\pgfsetdash{}{0pt}%
\pgfpathmoveto{\pgfqpoint{1.255366in}{1.426149in}}%
\pgfpathlineto{\pgfqpoint{1.240651in}{1.450223in}}%
\pgfpathlineto{\pgfqpoint{1.274176in}{1.427697in}}%
\pgfpathlineto{\pgfqpoint{1.287275in}{1.404676in}}%
\pgfpathlineto{\pgfqpoint{1.255366in}{1.426149in}}%
\pgfpathclose%
\pgfusepath{fill}%
\end{pgfscope}%
\begin{pgfscope}%
\pgfpathrectangle{\pgfqpoint{0.000000in}{0.000000in}}{\pgfqpoint{3.000000in}{3.000000in}}%
\pgfusepath{clip}%
\pgfsetbuttcap%
\pgfsetroundjoin%
\definecolor{currentfill}{rgb}{0.678253,0.000000,0.000000}%
\pgfsetfillcolor{currentfill}%
\pgfsetlinewidth{0.000000pt}%
\definecolor{currentstroke}{rgb}{0.000000,0.000000,0.000000}%
\pgfsetstrokecolor{currentstroke}%
\pgfsetdash{}{0pt}%
\pgfpathmoveto{\pgfqpoint{0.776647in}{2.080925in}}%
\pgfpathlineto{\pgfqpoint{0.759823in}{2.095965in}}%
\pgfpathlineto{\pgfqpoint{0.738155in}{2.037171in}}%
\pgfpathlineto{\pgfqpoint{0.755558in}{2.023308in}}%
\pgfpathlineto{\pgfqpoint{0.776647in}{2.080925in}}%
\pgfpathclose%
\pgfusepath{fill}%
\end{pgfscope}%
\begin{pgfscope}%
\pgfpathrectangle{\pgfqpoint{0.000000in}{0.000000in}}{\pgfqpoint{3.000000in}{3.000000in}}%
\pgfusepath{clip}%
\pgfsetbuttcap%
\pgfsetroundjoin%
\definecolor{currentfill}{rgb}{1.000000,0.480029,0.000000}%
\pgfsetfillcolor{currentfill}%
\pgfsetlinewidth{0.000000pt}%
\definecolor{currentstroke}{rgb}{0.000000,0.000000,0.000000}%
\pgfsetstrokecolor{currentstroke}%
\pgfsetdash{}{0pt}%
\pgfpathmoveto{\pgfqpoint{0.963670in}{1.837792in}}%
\pgfpathlineto{\pgfqpoint{0.946381in}{1.854993in}}%
\pgfpathlineto{\pgfqpoint{0.945861in}{1.809669in}}%
\pgfpathlineto{\pgfqpoint{0.963250in}{1.793710in}}%
\pgfpathlineto{\pgfqpoint{0.963670in}{1.837792in}}%
\pgfpathclose%
\pgfusepath{fill}%
\end{pgfscope}%
\begin{pgfscope}%
\pgfpathrectangle{\pgfqpoint{0.000000in}{0.000000in}}{\pgfqpoint{3.000000in}{3.000000in}}%
\pgfusepath{clip}%
\pgfsetbuttcap%
\pgfsetroundjoin%
\definecolor{currentfill}{rgb}{1.000000,0.291213,0.000000}%
\pgfsetfillcolor{currentfill}%
\pgfsetlinewidth{0.000000pt}%
\definecolor{currentstroke}{rgb}{0.000000,0.000000,0.000000}%
\pgfsetstrokecolor{currentstroke}%
\pgfsetdash{}{0pt}%
\pgfpathmoveto{\pgfqpoint{2.171529in}{1.856405in}}%
\pgfpathlineto{\pgfqpoint{2.188973in}{1.871688in}}%
\pgfpathlineto{\pgfqpoint{2.182812in}{1.920548in}}%
\pgfpathlineto{\pgfqpoint{2.165625in}{1.904038in}}%
\pgfpathlineto{\pgfqpoint{2.171529in}{1.856405in}}%
\pgfpathclose%
\pgfusepath{fill}%
\end{pgfscope}%
\begin{pgfscope}%
\pgfpathrectangle{\pgfqpoint{0.000000in}{0.000000in}}{\pgfqpoint{3.000000in}{3.000000in}}%
\pgfusepath{clip}%
\pgfsetbuttcap%
\pgfsetroundjoin%
\definecolor{currentfill}{rgb}{1.000000,0.886710,0.000000}%
\pgfsetfillcolor{currentfill}%
\pgfsetlinewidth{0.000000pt}%
\definecolor{currentstroke}{rgb}{0.000000,0.000000,0.000000}%
\pgfsetstrokecolor{currentstroke}%
\pgfsetdash{}{0pt}%
\pgfpathmoveto{\pgfqpoint{1.067260in}{1.687726in}}%
\pgfpathlineto{\pgfqpoint{1.049964in}{1.706830in}}%
\pgfpathlineto{\pgfqpoint{1.062125in}{1.669299in}}%
\pgfpathlineto{\pgfqpoint{1.079059in}{1.651448in}}%
\pgfpathlineto{\pgfqpoint{1.067260in}{1.687726in}}%
\pgfpathclose%
\pgfusepath{fill}%
\end{pgfscope}%
\begin{pgfscope}%
\pgfpathrectangle{\pgfqpoint{0.000000in}{0.000000in}}{\pgfqpoint{3.000000in}{3.000000in}}%
\pgfusepath{clip}%
\pgfsetbuttcap%
\pgfsetroundjoin%
\definecolor{currentfill}{rgb}{0.085389,1.000000,0.882353}%
\pgfsetfillcolor{currentfill}%
\pgfsetlinewidth{0.000000pt}%
\definecolor{currentstroke}{rgb}{0.000000,0.000000,0.000000}%
\pgfsetstrokecolor{currentstroke}%
\pgfsetdash{}{0pt}%
\pgfpathmoveto{\pgfqpoint{1.615457in}{1.279498in}}%
\pgfpathlineto{\pgfqpoint{1.620035in}{1.306386in}}%
\pgfpathlineto{\pgfqpoint{1.665359in}{1.314625in}}%
\pgfpathlineto{\pgfqpoint{1.658178in}{1.287284in}}%
\pgfpathlineto{\pgfqpoint{1.615457in}{1.279498in}}%
\pgfpathclose%
\pgfusepath{fill}%
\end{pgfscope}%
\begin{pgfscope}%
\pgfpathrectangle{\pgfqpoint{0.000000in}{0.000000in}}{\pgfqpoint{3.000000in}{3.000000in}}%
\pgfusepath{clip}%
\pgfsetbuttcap%
\pgfsetroundjoin%
\definecolor{currentfill}{rgb}{0.199241,1.000000,0.768501}%
\pgfsetfillcolor{currentfill}%
\pgfsetlinewidth{0.000000pt}%
\definecolor{currentstroke}{rgb}{0.000000,0.000000,0.000000}%
\pgfsetstrokecolor{currentstroke}%
\pgfsetdash{}{0pt}%
\pgfpathmoveto{\pgfqpoint{1.347846in}{1.336011in}}%
\pgfpathlineto{\pgfqpoint{1.336758in}{1.362046in}}%
\pgfpathlineto{\pgfqpoint{1.378384in}{1.347359in}}%
\pgfpathlineto{\pgfqpoint{1.387220in}{1.322086in}}%
\pgfpathlineto{\pgfqpoint{1.347846in}{1.336011in}}%
\pgfpathclose%
\pgfusepath{fill}%
\end{pgfscope}%
\begin{pgfscope}%
\pgfpathrectangle{\pgfqpoint{0.000000in}{0.000000in}}{\pgfqpoint{3.000000in}{3.000000in}}%
\pgfusepath{clip}%
\pgfsetbuttcap%
\pgfsetroundjoin%
\definecolor{currentfill}{rgb}{0.578748,1.000000,0.388994}%
\pgfsetfillcolor{currentfill}%
\pgfsetlinewidth{0.000000pt}%
\definecolor{currentstroke}{rgb}{0.000000,0.000000,0.000000}%
\pgfsetstrokecolor{currentstroke}%
\pgfsetdash{}{0pt}%
\pgfpathmoveto{\pgfqpoint{1.198765in}{1.498698in}}%
\pgfpathlineto{\pgfqpoint{1.182769in}{1.520959in}}%
\pgfpathlineto{\pgfqpoint{1.211142in}{1.493821in}}%
\pgfpathlineto{\pgfqpoint{1.225910in}{1.472715in}}%
\pgfpathlineto{\pgfqpoint{1.198765in}{1.498698in}}%
\pgfpathclose%
\pgfusepath{fill}%
\end{pgfscope}%
\begin{pgfscope}%
\pgfpathrectangle{\pgfqpoint{0.000000in}{0.000000in}}{\pgfqpoint{3.000000in}{3.000000in}}%
\pgfusepath{clip}%
\pgfsetbuttcap%
\pgfsetroundjoin%
\definecolor{currentfill}{rgb}{1.000000,0.741467,0.000000}%
\pgfsetfillcolor{currentfill}%
\pgfsetlinewidth{0.000000pt}%
\definecolor{currentstroke}{rgb}{0.000000,0.000000,0.000000}%
\pgfsetstrokecolor{currentstroke}%
\pgfsetdash{}{0pt}%
\pgfpathmoveto{\pgfqpoint{2.041480in}{1.699339in}}%
\pgfpathlineto{\pgfqpoint{2.058620in}{1.716374in}}%
\pgfpathlineto{\pgfqpoint{2.067151in}{1.756624in}}%
\pgfpathlineto{\pgfqpoint{2.049802in}{1.738335in}}%
\pgfpathlineto{\pgfqpoint{2.041480in}{1.699339in}}%
\pgfpathclose%
\pgfusepath{fill}%
\end{pgfscope}%
\begin{pgfscope}%
\pgfpathrectangle{\pgfqpoint{0.000000in}{0.000000in}}{\pgfqpoint{3.000000in}{3.000000in}}%
\pgfusepath{clip}%
\pgfsetbuttcap%
\pgfsetroundjoin%
\definecolor{currentfill}{rgb}{0.606952,0.000000,0.000000}%
\pgfsetfillcolor{currentfill}%
\pgfsetlinewidth{0.000000pt}%
\definecolor{currentstroke}{rgb}{0.000000,0.000000,0.000000}%
\pgfsetstrokecolor{currentstroke}%
\pgfsetdash{}{0pt}%
\pgfpathmoveto{\pgfqpoint{0.759823in}{2.095965in}}%
\pgfpathlineto{\pgfqpoint{0.742996in}{2.110818in}}%
\pgfpathlineto{\pgfqpoint{0.720742in}{2.050850in}}%
\pgfpathlineto{\pgfqpoint{0.738155in}{2.037171in}}%
\pgfpathlineto{\pgfqpoint{0.759823in}{2.095965in}}%
\pgfpathclose%
\pgfusepath{fill}%
\end{pgfscope}%
\begin{pgfscope}%
\pgfpathrectangle{\pgfqpoint{0.000000in}{0.000000in}}{\pgfqpoint{3.000000in}{3.000000in}}%
\pgfusepath{clip}%
\pgfsetbuttcap%
\pgfsetroundjoin%
\definecolor{currentfill}{rgb}{0.300443,1.000000,0.667299}%
\pgfsetfillcolor{currentfill}%
\pgfsetlinewidth{0.000000pt}%
\definecolor{currentstroke}{rgb}{0.000000,0.000000,0.000000}%
\pgfsetstrokecolor{currentstroke}%
\pgfsetdash{}{0pt}%
\pgfpathmoveto{\pgfqpoint{1.300348in}{1.379835in}}%
\pgfpathlineto{\pgfqpoint{1.287275in}{1.404676in}}%
\pgfpathlineto{\pgfqpoint{1.325644in}{1.385967in}}%
\pgfpathlineto{\pgfqpoint{1.336758in}{1.362046in}}%
\pgfpathlineto{\pgfqpoint{1.300348in}{1.379835in}}%
\pgfpathclose%
\pgfusepath{fill}%
\end{pgfscope}%
\begin{pgfscope}%
\pgfpathrectangle{\pgfqpoint{0.000000in}{0.000000in}}{\pgfqpoint{3.000000in}{3.000000in}}%
\pgfusepath{clip}%
\pgfsetbuttcap%
\pgfsetroundjoin%
\definecolor{currentfill}{rgb}{0.085389,1.000000,0.882353}%
\pgfsetfillcolor{currentfill}%
\pgfsetlinewidth{0.000000pt}%
\definecolor{currentstroke}{rgb}{0.000000,0.000000,0.000000}%
\pgfsetstrokecolor{currentstroke}%
\pgfsetdash{}{0pt}%
\pgfpathmoveto{\pgfqpoint{1.436860in}{1.284307in}}%
\pgfpathlineto{\pgfqpoint{1.430529in}{1.311475in}}%
\pgfpathlineto{\pgfqpoint{1.476669in}{1.304462in}}%
\pgfpathlineto{\pgfqpoint{1.480348in}{1.277679in}}%
\pgfpathlineto{\pgfqpoint{1.436860in}{1.284307in}}%
\pgfpathclose%
\pgfusepath{fill}%
\end{pgfscope}%
\begin{pgfscope}%
\pgfpathrectangle{\pgfqpoint{0.000000in}{0.000000in}}{\pgfqpoint{3.000000in}{3.000000in}}%
\pgfusepath{clip}%
\pgfsetbuttcap%
\pgfsetroundjoin%
\definecolor{currentfill}{rgb}{1.000000,0.233115,0.000000}%
\pgfsetfillcolor{currentfill}%
\pgfsetlinewidth{0.000000pt}%
\definecolor{currentstroke}{rgb}{0.000000,0.000000,0.000000}%
\pgfsetstrokecolor{currentstroke}%
\pgfsetdash{}{0pt}%
\pgfpathmoveto{\pgfqpoint{2.188973in}{1.871688in}}%
\pgfpathlineto{\pgfqpoint{2.206429in}{1.886656in}}%
\pgfpathlineto{\pgfqpoint{2.200006in}{1.936741in}}%
\pgfpathlineto{\pgfqpoint{2.182812in}{1.920548in}}%
\pgfpathlineto{\pgfqpoint{2.188973in}{1.871688in}}%
\pgfpathclose%
\pgfusepath{fill}%
\end{pgfscope}%
\begin{pgfscope}%
\pgfpathrectangle{\pgfqpoint{0.000000in}{0.000000in}}{\pgfqpoint{3.000000in}{3.000000in}}%
\pgfusepath{clip}%
\pgfsetbuttcap%
\pgfsetroundjoin%
\definecolor{currentfill}{rgb}{0.667299,1.000000,0.300443}%
\pgfsetfillcolor{currentfill}%
\pgfsetlinewidth{0.000000pt}%
\definecolor{currentstroke}{rgb}{0.000000,0.000000,0.000000}%
\pgfsetstrokecolor{currentstroke}%
\pgfsetdash{}{0pt}%
\pgfpathmoveto{\pgfqpoint{1.880364in}{1.502622in}}%
\pgfpathlineto{\pgfqpoint{1.895612in}{1.522882in}}%
\pgfpathlineto{\pgfqpoint{1.922126in}{1.551882in}}%
\pgfpathlineto{\pgfqpoint{1.905789in}{1.530440in}}%
\pgfpathlineto{\pgfqpoint{1.880364in}{1.502622in}}%
\pgfpathclose%
\pgfusepath{fill}%
\end{pgfscope}%
\begin{pgfscope}%
\pgfpathrectangle{\pgfqpoint{0.000000in}{0.000000in}}{\pgfqpoint{3.000000in}{3.000000in}}%
\pgfusepath{clip}%
\pgfsetbuttcap%
\pgfsetroundjoin%
\definecolor{currentfill}{rgb}{0.490196,1.000000,0.477546}%
\pgfsetfillcolor{currentfill}%
\pgfsetlinewidth{0.000000pt}%
\definecolor{currentstroke}{rgb}{0.000000,0.000000,0.000000}%
\pgfsetstrokecolor{currentstroke}%
\pgfsetdash{}{0pt}%
\pgfpathmoveto{\pgfqpoint{1.818880in}{1.434912in}}%
\pgfpathlineto{\pgfqpoint{1.832585in}{1.456688in}}%
\pgfpathlineto{\pgfqpoint{1.865143in}{1.481141in}}%
\pgfpathlineto{\pgfqpoint{1.849948in}{1.458274in}}%
\pgfpathlineto{\pgfqpoint{1.818880in}{1.434912in}}%
\pgfpathclose%
\pgfusepath{fill}%
\end{pgfscope}%
\begin{pgfscope}%
\pgfpathrectangle{\pgfqpoint{0.000000in}{0.000000in}}{\pgfqpoint{3.000000in}{3.000000in}}%
\pgfusepath{clip}%
\pgfsetbuttcap%
\pgfsetroundjoin%
\definecolor{currentfill}{rgb}{1.000000,0.407407,0.000000}%
\pgfsetfillcolor{currentfill}%
\pgfsetlinewidth{0.000000pt}%
\definecolor{currentstroke}{rgb}{0.000000,0.000000,0.000000}%
\pgfsetstrokecolor{currentstroke}%
\pgfsetdash{}{0pt}%
\pgfpathmoveto{\pgfqpoint{0.946381in}{1.854993in}}%
\pgfpathlineto{\pgfqpoint{0.929082in}{1.871801in}}%
\pgfpathlineto{\pgfqpoint{0.928457in}{1.825238in}}%
\pgfpathlineto{\pgfqpoint{0.945861in}{1.809669in}}%
\pgfpathlineto{\pgfqpoint{0.946381in}{1.854993in}}%
\pgfpathclose%
\pgfusepath{fill}%
\end{pgfscope}%
\begin{pgfscope}%
\pgfpathrectangle{\pgfqpoint{0.000000in}{0.000000in}}{\pgfqpoint{3.000000in}{3.000000in}}%
\pgfusepath{clip}%
\pgfsetbuttcap%
\pgfsetroundjoin%
\definecolor{currentfill}{rgb}{0.553476,0.000000,0.000000}%
\pgfsetfillcolor{currentfill}%
\pgfsetlinewidth{0.000000pt}%
\definecolor{currentstroke}{rgb}{0.000000,0.000000,0.000000}%
\pgfsetstrokecolor{currentstroke}%
\pgfsetdash{}{0pt}%
\pgfpathmoveto{\pgfqpoint{0.742996in}{2.110818in}}%
\pgfpathlineto{\pgfqpoint{0.726164in}{2.125494in}}%
\pgfpathlineto{\pgfqpoint{0.703320in}{2.064354in}}%
\pgfpathlineto{\pgfqpoint{0.720742in}{2.050850in}}%
\pgfpathlineto{\pgfqpoint{0.742996in}{2.110818in}}%
\pgfpathclose%
\pgfusepath{fill}%
\end{pgfscope}%
\begin{pgfscope}%
\pgfpathrectangle{\pgfqpoint{0.000000in}{0.000000in}}{\pgfqpoint{3.000000in}{3.000000in}}%
\pgfusepath{clip}%
\pgfsetbuttcap%
\pgfsetroundjoin%
\definecolor{currentfill}{rgb}{0.895003,1.000000,0.072739}%
\pgfsetfillcolor{currentfill}%
\pgfsetlinewidth{0.000000pt}%
\definecolor{currentstroke}{rgb}{0.000000,0.000000,0.000000}%
\pgfsetstrokecolor{currentstroke}%
\pgfsetdash{}{0pt}%
\pgfpathmoveto{\pgfqpoint{1.954869in}{1.591628in}}%
\pgfpathlineto{\pgfqpoint{1.971275in}{1.610150in}}%
\pgfpathlineto{\pgfqpoint{1.990171in}{1.644473in}}%
\pgfpathlineto{\pgfqpoint{1.973106in}{1.624715in}}%
\pgfpathlineto{\pgfqpoint{1.954869in}{1.591628in}}%
\pgfpathclose%
\pgfusepath{fill}%
\end{pgfscope}%
\begin{pgfscope}%
\pgfpathrectangle{\pgfqpoint{0.000000in}{0.000000in}}{\pgfqpoint{3.000000in}{3.000000in}}%
\pgfusepath{clip}%
\pgfsetbuttcap%
\pgfsetroundjoin%
\definecolor{currentfill}{rgb}{0.085389,1.000000,0.882353}%
\pgfsetfillcolor{currentfill}%
\pgfsetlinewidth{0.000000pt}%
\definecolor{currentstroke}{rgb}{0.000000,0.000000,0.000000}%
\pgfsetstrokecolor{currentstroke}%
\pgfsetdash{}{0pt}%
\pgfpathmoveto{\pgfqpoint{1.570810in}{1.275242in}}%
\pgfpathlineto{\pgfqpoint{1.572661in}{1.301883in}}%
\pgfpathlineto{\pgfqpoint{1.620035in}{1.306386in}}%
\pgfpathlineto{\pgfqpoint{1.615457in}{1.279498in}}%
\pgfpathlineto{\pgfqpoint{1.570810in}{1.275242in}}%
\pgfpathclose%
\pgfusepath{fill}%
\end{pgfscope}%
\begin{pgfscope}%
\pgfpathrectangle{\pgfqpoint{0.000000in}{0.000000in}}{\pgfqpoint{3.000000in}{3.000000in}}%
\pgfusepath{clip}%
\pgfsetbuttcap%
\pgfsetroundjoin%
\definecolor{currentfill}{rgb}{0.819102,1.000000,0.148640}%
\pgfsetfillcolor{currentfill}%
\pgfsetlinewidth{0.000000pt}%
\definecolor{currentstroke}{rgb}{0.000000,0.000000,0.000000}%
\pgfsetstrokecolor{currentstroke}%
\pgfsetdash{}{0pt}%
\pgfpathmoveto{\pgfqpoint{1.129739in}{1.593336in}}%
\pgfpathlineto{\pgfqpoint{1.112866in}{1.613546in}}%
\pgfpathlineto{\pgfqpoint{1.134638in}{1.580941in}}%
\pgfpathlineto{\pgfqpoint{1.150706in}{1.561953in}}%
\pgfpathlineto{\pgfqpoint{1.129739in}{1.593336in}}%
\pgfpathclose%
\pgfusepath{fill}%
\end{pgfscope}%
\begin{pgfscope}%
\pgfpathrectangle{\pgfqpoint{0.000000in}{0.000000in}}{\pgfqpoint{3.000000in}{3.000000in}}%
\pgfusepath{clip}%
\pgfsetbuttcap%
\pgfsetroundjoin%
\definecolor{currentfill}{rgb}{0.085389,1.000000,0.882353}%
\pgfsetfillcolor{currentfill}%
\pgfsetlinewidth{0.000000pt}%
\definecolor{currentstroke}{rgb}{0.000000,0.000000,0.000000}%
\pgfsetstrokecolor{currentstroke}%
\pgfsetdash{}{0pt}%
\pgfpathmoveto{\pgfqpoint{1.480348in}{1.277679in}}%
\pgfpathlineto{\pgfqpoint{1.476669in}{1.304462in}}%
\pgfpathlineto{\pgfqpoint{1.524457in}{1.301236in}}%
\pgfpathlineto{\pgfqpoint{1.525384in}{1.274630in}}%
\pgfpathlineto{\pgfqpoint{1.480348in}{1.277679in}}%
\pgfpathclose%
\pgfusepath{fill}%
\end{pgfscope}%
\begin{pgfscope}%
\pgfpathrectangle{\pgfqpoint{0.000000in}{0.000000in}}{\pgfqpoint{3.000000in}{3.000000in}}%
\pgfusepath{clip}%
\pgfsetbuttcap%
\pgfsetroundjoin%
\definecolor{currentfill}{rgb}{1.000000,0.814089,0.000000}%
\pgfsetfillcolor{currentfill}%
\pgfsetlinewidth{0.000000pt}%
\definecolor{currentstroke}{rgb}{0.000000,0.000000,0.000000}%
\pgfsetstrokecolor{currentstroke}%
\pgfsetdash{}{0pt}%
\pgfpathmoveto{\pgfqpoint{1.049964in}{1.706830in}}%
\pgfpathlineto{\pgfqpoint{1.032652in}{1.725290in}}%
\pgfpathlineto{\pgfqpoint{1.045171in}{1.686507in}}%
\pgfpathlineto{\pgfqpoint{1.062125in}{1.669299in}}%
\pgfpathlineto{\pgfqpoint{1.049964in}{1.706830in}}%
\pgfpathclose%
\pgfusepath{fill}%
\end{pgfscope}%
\begin{pgfscope}%
\pgfpathrectangle{\pgfqpoint{0.000000in}{0.000000in}}{\pgfqpoint{3.000000in}{3.000000in}}%
\pgfusepath{clip}%
\pgfsetbuttcap%
\pgfsetroundjoin%
\definecolor{currentfill}{rgb}{0.500000,0.000000,0.000000}%
\pgfsetfillcolor{currentfill}%
\pgfsetlinewidth{0.000000pt}%
\definecolor{currentstroke}{rgb}{0.000000,0.000000,0.000000}%
\pgfsetstrokecolor{currentstroke}%
\pgfsetdash{}{0pt}%
\pgfpathmoveto{\pgfqpoint{0.726164in}{2.125494in}}%
\pgfpathlineto{\pgfqpoint{0.709328in}{2.140003in}}%
\pgfpathlineto{\pgfqpoint{0.685889in}{2.077692in}}%
\pgfpathlineto{\pgfqpoint{0.703320in}{2.064354in}}%
\pgfpathlineto{\pgfqpoint{0.726164in}{2.125494in}}%
\pgfpathclose%
\pgfusepath{fill}%
\end{pgfscope}%
\begin{pgfscope}%
\pgfpathrectangle{\pgfqpoint{0.000000in}{0.000000in}}{\pgfqpoint{3.000000in}{3.000000in}}%
\pgfusepath{clip}%
\pgfsetbuttcap%
\pgfsetroundjoin%
\definecolor{currentfill}{rgb}{0.085389,1.000000,0.882353}%
\pgfsetfillcolor{currentfill}%
\pgfsetlinewidth{0.000000pt}%
\definecolor{currentstroke}{rgb}{0.000000,0.000000,0.000000}%
\pgfsetstrokecolor{currentstroke}%
\pgfsetdash{}{0pt}%
\pgfpathmoveto{\pgfqpoint{1.525384in}{1.274630in}}%
\pgfpathlineto{\pgfqpoint{1.524457in}{1.301236in}}%
\pgfpathlineto{\pgfqpoint{1.572661in}{1.301883in}}%
\pgfpathlineto{\pgfqpoint{1.570810in}{1.275242in}}%
\pgfpathlineto{\pgfqpoint{1.525384in}{1.274630in}}%
\pgfpathclose%
\pgfusepath{fill}%
\end{pgfscope}%
\begin{pgfscope}%
\pgfpathrectangle{\pgfqpoint{0.000000in}{0.000000in}}{\pgfqpoint{3.000000in}{3.000000in}}%
\pgfusepath{clip}%
\pgfsetbuttcap%
\pgfsetroundjoin%
\definecolor{currentfill}{rgb}{1.000000,0.175018,0.000000}%
\pgfsetfillcolor{currentfill}%
\pgfsetlinewidth{0.000000pt}%
\definecolor{currentstroke}{rgb}{0.000000,0.000000,0.000000}%
\pgfsetstrokecolor{currentstroke}%
\pgfsetdash{}{0pt}%
\pgfpathmoveto{\pgfqpoint{2.206429in}{1.886656in}}%
\pgfpathlineto{\pgfqpoint{2.223899in}{1.901329in}}%
\pgfpathlineto{\pgfqpoint{2.217209in}{1.952637in}}%
\pgfpathlineto{\pgfqpoint{2.200006in}{1.936741in}}%
\pgfpathlineto{\pgfqpoint{2.206429in}{1.886656in}}%
\pgfpathclose%
\pgfusepath{fill}%
\end{pgfscope}%
\begin{pgfscope}%
\pgfpathrectangle{\pgfqpoint{0.000000in}{0.000000in}}{\pgfqpoint{3.000000in}{3.000000in}}%
\pgfusepath{clip}%
\pgfsetbuttcap%
\pgfsetroundjoin%
\definecolor{currentfill}{rgb}{1.000000,0.668845,0.000000}%
\pgfsetfillcolor{currentfill}%
\pgfsetlinewidth{0.000000pt}%
\definecolor{currentstroke}{rgb}{0.000000,0.000000,0.000000}%
\pgfsetstrokecolor{currentstroke}%
\pgfsetdash{}{0pt}%
\pgfpathmoveto{\pgfqpoint{2.058620in}{1.716374in}}%
\pgfpathlineto{\pgfqpoint{2.075779in}{1.732871in}}%
\pgfpathlineto{\pgfqpoint{2.084513in}{1.774373in}}%
\pgfpathlineto{\pgfqpoint{2.067151in}{1.756624in}}%
\pgfpathlineto{\pgfqpoint{2.058620in}{1.716374in}}%
\pgfpathclose%
\pgfusepath{fill}%
\end{pgfscope}%
\begin{pgfscope}%
\pgfpathrectangle{\pgfqpoint{0.000000in}{0.000000in}}{\pgfqpoint{3.000000in}{3.000000in}}%
\pgfusepath{clip}%
\pgfsetbuttcap%
\pgfsetroundjoin%
\definecolor{currentfill}{rgb}{0.199241,1.000000,0.768501}%
\pgfsetfillcolor{currentfill}%
\pgfsetlinewidth{0.000000pt}%
\definecolor{currentstroke}{rgb}{0.000000,0.000000,0.000000}%
\pgfsetstrokecolor{currentstroke}%
\pgfsetdash{}{0pt}%
\pgfpathmoveto{\pgfqpoint{1.665359in}{1.314625in}}%
\pgfpathlineto{\pgfqpoint{1.672559in}{1.339488in}}%
\pgfpathlineto{\pgfqpoint{1.717090in}{1.351884in}}%
\pgfpathlineto{\pgfqpoint{1.707474in}{1.326376in}}%
\pgfpathlineto{\pgfqpoint{1.665359in}{1.314625in}}%
\pgfpathclose%
\pgfusepath{fill}%
\end{pgfscope}%
\begin{pgfscope}%
\pgfpathrectangle{\pgfqpoint{0.000000in}{0.000000in}}{\pgfqpoint{3.000000in}{3.000000in}}%
\pgfusepath{clip}%
\pgfsetbuttcap%
\pgfsetroundjoin%
\definecolor{currentfill}{rgb}{1.000000,0.349310,0.000000}%
\pgfsetfillcolor{currentfill}%
\pgfsetlinewidth{0.000000pt}%
\definecolor{currentstroke}{rgb}{0.000000,0.000000,0.000000}%
\pgfsetstrokecolor{currentstroke}%
\pgfsetdash{}{0pt}%
\pgfpathmoveto{\pgfqpoint{0.929082in}{1.871801in}}%
\pgfpathlineto{\pgfqpoint{0.911772in}{1.888245in}}%
\pgfpathlineto{\pgfqpoint{0.911038in}{1.840444in}}%
\pgfpathlineto{\pgfqpoint{0.928457in}{1.825238in}}%
\pgfpathlineto{\pgfqpoint{0.929082in}{1.871801in}}%
\pgfpathclose%
\pgfusepath{fill}%
\end{pgfscope}%
\begin{pgfscope}%
\pgfpathrectangle{\pgfqpoint{0.000000in}{0.000000in}}{\pgfqpoint{3.000000in}{3.000000in}}%
\pgfusepath{clip}%
\pgfsetbuttcap%
\pgfsetroundjoin%
\definecolor{currentfill}{rgb}{0.401645,1.000000,0.566097}%
\pgfsetfillcolor{currentfill}%
\pgfsetlinewidth{0.000000pt}%
\definecolor{currentstroke}{rgb}{0.000000,0.000000,0.000000}%
\pgfsetstrokecolor{currentstroke}%
\pgfsetdash{}{0pt}%
\pgfpathmoveto{\pgfqpoint{1.768889in}{1.391864in}}%
\pgfpathlineto{\pgfqpoint{1.780717in}{1.414254in}}%
\pgfpathlineto{\pgfqpoint{1.818880in}{1.434912in}}%
\pgfpathlineto{\pgfqpoint{1.805202in}{1.411554in}}%
\pgfpathlineto{\pgfqpoint{1.768889in}{1.391864in}}%
\pgfpathclose%
\pgfusepath{fill}%
\end{pgfscope}%
\begin{pgfscope}%
\pgfpathrectangle{\pgfqpoint{0.000000in}{0.000000in}}{\pgfqpoint{3.000000in}{3.000000in}}%
\pgfusepath{clip}%
\pgfsetbuttcap%
\pgfsetroundjoin%
\definecolor{currentfill}{rgb}{0.490196,1.000000,0.477546}%
\pgfsetfillcolor{currentfill}%
\pgfsetlinewidth{0.000000pt}%
\definecolor{currentstroke}{rgb}{0.000000,0.000000,0.000000}%
\pgfsetstrokecolor{currentstroke}%
\pgfsetdash{}{0pt}%
\pgfpathmoveto{\pgfqpoint{1.240651in}{1.450223in}}%
\pgfpathlineto{\pgfqpoint{1.225910in}{1.472715in}}%
\pgfpathlineto{\pgfqpoint{1.261050in}{1.449135in}}%
\pgfpathlineto{\pgfqpoint{1.274176in}{1.427697in}}%
\pgfpathlineto{\pgfqpoint{1.240651in}{1.450223in}}%
\pgfpathclose%
\pgfusepath{fill}%
\end{pgfscope}%
\begin{pgfscope}%
\pgfpathrectangle{\pgfqpoint{0.000000in}{0.000000in}}{\pgfqpoint{3.000000in}{3.000000in}}%
\pgfusepath{clip}%
\pgfsetbuttcap%
\pgfsetroundjoin%
\definecolor{currentfill}{rgb}{0.199241,1.000000,0.768501}%
\pgfsetfillcolor{currentfill}%
\pgfsetlinewidth{0.000000pt}%
\definecolor{currentstroke}{rgb}{0.000000,0.000000,0.000000}%
\pgfsetstrokecolor{currentstroke}%
\pgfsetdash{}{0pt}%
\pgfpathmoveto{\pgfqpoint{1.387220in}{1.322086in}}%
\pgfpathlineto{\pgfqpoint{1.378384in}{1.347359in}}%
\pgfpathlineto{\pgfqpoint{1.424181in}{1.336165in}}%
\pgfpathlineto{\pgfqpoint{1.430529in}{1.311475in}}%
\pgfpathlineto{\pgfqpoint{1.387220in}{1.322086in}}%
\pgfpathclose%
\pgfusepath{fill}%
\end{pgfscope}%
\begin{pgfscope}%
\pgfpathrectangle{\pgfqpoint{0.000000in}{0.000000in}}{\pgfqpoint{3.000000in}{3.000000in}}%
\pgfusepath{clip}%
\pgfsetbuttcap%
\pgfsetroundjoin%
\definecolor{currentfill}{rgb}{0.300443,1.000000,0.667299}%
\pgfsetfillcolor{currentfill}%
\pgfsetlinewidth{0.000000pt}%
\definecolor{currentstroke}{rgb}{0.000000,0.000000,0.000000}%
\pgfsetstrokecolor{currentstroke}%
\pgfsetdash{}{0pt}%
\pgfpathmoveto{\pgfqpoint{1.717090in}{1.351884in}}%
\pgfpathlineto{\pgfqpoint{1.726729in}{1.375277in}}%
\pgfpathlineto{\pgfqpoint{1.768889in}{1.391864in}}%
\pgfpathlineto{\pgfqpoint{1.757087in}{1.367653in}}%
\pgfpathlineto{\pgfqpoint{1.717090in}{1.351884in}}%
\pgfpathclose%
\pgfusepath{fill}%
\end{pgfscope}%
\begin{pgfscope}%
\pgfpathrectangle{\pgfqpoint{0.000000in}{0.000000in}}{\pgfqpoint{3.000000in}{3.000000in}}%
\pgfusepath{clip}%
\pgfsetbuttcap%
\pgfsetroundjoin%
\definecolor{currentfill}{rgb}{1.000000,0.116921,0.000000}%
\pgfsetfillcolor{currentfill}%
\pgfsetlinewidth{0.000000pt}%
\definecolor{currentstroke}{rgb}{0.000000,0.000000,0.000000}%
\pgfsetstrokecolor{currentstroke}%
\pgfsetdash{}{0pt}%
\pgfpathmoveto{\pgfqpoint{2.223899in}{1.901329in}}%
\pgfpathlineto{\pgfqpoint{2.241381in}{1.915726in}}%
\pgfpathlineto{\pgfqpoint{2.234419in}{1.968255in}}%
\pgfpathlineto{\pgfqpoint{2.217209in}{1.952637in}}%
\pgfpathlineto{\pgfqpoint{2.223899in}{1.901329in}}%
\pgfpathclose%
\pgfusepath{fill}%
\end{pgfscope}%
\begin{pgfscope}%
\pgfpathrectangle{\pgfqpoint{0.000000in}{0.000000in}}{\pgfqpoint{3.000000in}{3.000000in}}%
\pgfusepath{clip}%
\pgfsetbuttcap%
\pgfsetroundjoin%
\definecolor{currentfill}{rgb}{0.667299,1.000000,0.300443}%
\pgfsetfillcolor{currentfill}%
\pgfsetlinewidth{0.000000pt}%
\definecolor{currentstroke}{rgb}{0.000000,0.000000,0.000000}%
\pgfsetstrokecolor{currentstroke}%
\pgfsetdash{}{0pt}%
\pgfpathmoveto{\pgfqpoint{1.182769in}{1.520959in}}%
\pgfpathlineto{\pgfqpoint{1.166750in}{1.541998in}}%
\pgfpathlineto{\pgfqpoint{1.196349in}{1.513705in}}%
\pgfpathlineto{\pgfqpoint{1.211142in}{1.493821in}}%
\pgfpathlineto{\pgfqpoint{1.182769in}{1.520959in}}%
\pgfpathclose%
\pgfusepath{fill}%
\end{pgfscope}%
\begin{pgfscope}%
\pgfpathrectangle{\pgfqpoint{0.000000in}{0.000000in}}{\pgfqpoint{3.000000in}{3.000000in}}%
\pgfusepath{clip}%
\pgfsetbuttcap%
\pgfsetroundjoin%
\definecolor{currentfill}{rgb}{0.958254,0.973856,0.009488}%
\pgfsetfillcolor{currentfill}%
\pgfsetlinewidth{0.000000pt}%
\definecolor{currentstroke}{rgb}{0.000000,0.000000,0.000000}%
\pgfsetstrokecolor{currentstroke}%
\pgfsetdash{}{0pt}%
\pgfpathmoveto{\pgfqpoint{1.971275in}{1.610150in}}%
\pgfpathlineto{\pgfqpoint{1.987705in}{1.627890in}}%
\pgfpathlineto{\pgfqpoint{2.007255in}{1.663449in}}%
\pgfpathlineto{\pgfqpoint{1.990171in}{1.644473in}}%
\pgfpathlineto{\pgfqpoint{1.971275in}{1.610150in}}%
\pgfpathclose%
\pgfusepath{fill}%
\end{pgfscope}%
\begin{pgfscope}%
\pgfpathrectangle{\pgfqpoint{0.000000in}{0.000000in}}{\pgfqpoint{3.000000in}{3.000000in}}%
\pgfusepath{clip}%
\pgfsetbuttcap%
\pgfsetroundjoin%
\definecolor{currentfill}{rgb}{1.000000,0.741467,0.000000}%
\pgfsetfillcolor{currentfill}%
\pgfsetlinewidth{0.000000pt}%
\definecolor{currentstroke}{rgb}{0.000000,0.000000,0.000000}%
\pgfsetstrokecolor{currentstroke}%
\pgfsetdash{}{0pt}%
\pgfpathmoveto{\pgfqpoint{1.032652in}{1.725290in}}%
\pgfpathlineto{\pgfqpoint{1.015324in}{1.743160in}}%
\pgfpathlineto{\pgfqpoint{1.028197in}{1.703127in}}%
\pgfpathlineto{\pgfqpoint{1.045171in}{1.686507in}}%
\pgfpathlineto{\pgfqpoint{1.032652in}{1.725290in}}%
\pgfpathclose%
\pgfusepath{fill}%
\end{pgfscope}%
\begin{pgfscope}%
\pgfpathrectangle{\pgfqpoint{0.000000in}{0.000000in}}{\pgfqpoint{3.000000in}{3.000000in}}%
\pgfusepath{clip}%
\pgfsetbuttcap%
\pgfsetroundjoin%
\definecolor{currentfill}{rgb}{1.000000,0.291213,0.000000}%
\pgfsetfillcolor{currentfill}%
\pgfsetlinewidth{0.000000pt}%
\definecolor{currentstroke}{rgb}{0.000000,0.000000,0.000000}%
\pgfsetstrokecolor{currentstroke}%
\pgfsetdash{}{0pt}%
\pgfpathmoveto{\pgfqpoint{0.911772in}{1.888245in}}%
\pgfpathlineto{\pgfqpoint{0.894453in}{1.904350in}}%
\pgfpathlineto{\pgfqpoint{0.893605in}{1.855313in}}%
\pgfpathlineto{\pgfqpoint{0.911038in}{1.840444in}}%
\pgfpathlineto{\pgfqpoint{0.911772in}{1.888245in}}%
\pgfpathclose%
\pgfusepath{fill}%
\end{pgfscope}%
\begin{pgfscope}%
\pgfpathrectangle{\pgfqpoint{0.000000in}{0.000000in}}{\pgfqpoint{3.000000in}{3.000000in}}%
\pgfusepath{clip}%
\pgfsetbuttcap%
\pgfsetroundjoin%
\definecolor{currentfill}{rgb}{0.743201,1.000000,0.224541}%
\pgfsetfillcolor{currentfill}%
\pgfsetlinewidth{0.000000pt}%
\definecolor{currentstroke}{rgb}{0.000000,0.000000,0.000000}%
\pgfsetstrokecolor{currentstroke}%
\pgfsetdash{}{0pt}%
\pgfpathmoveto{\pgfqpoint{1.895612in}{1.522882in}}%
\pgfpathlineto{\pgfqpoint{1.910884in}{1.542058in}}%
\pgfpathlineto{\pgfqpoint{1.938486in}{1.572239in}}%
\pgfpathlineto{\pgfqpoint{1.922126in}{1.551882in}}%
\pgfpathlineto{\pgfqpoint{1.895612in}{1.522882in}}%
\pgfpathclose%
\pgfusepath{fill}%
\end{pgfscope}%
\begin{pgfscope}%
\pgfpathrectangle{\pgfqpoint{0.000000in}{0.000000in}}{\pgfqpoint{3.000000in}{3.000000in}}%
\pgfusepath{clip}%
\pgfsetbuttcap%
\pgfsetroundjoin%
\definecolor{currentfill}{rgb}{1.000000,0.610748,0.000000}%
\pgfsetfillcolor{currentfill}%
\pgfsetlinewidth{0.000000pt}%
\definecolor{currentstroke}{rgb}{0.000000,0.000000,0.000000}%
\pgfsetstrokecolor{currentstroke}%
\pgfsetdash{}{0pt}%
\pgfpathmoveto{\pgfqpoint{2.075779in}{1.732871in}}%
\pgfpathlineto{\pgfqpoint{2.092956in}{1.748874in}}%
\pgfpathlineto{\pgfqpoint{2.101890in}{1.791627in}}%
\pgfpathlineto{\pgfqpoint{2.084513in}{1.774373in}}%
\pgfpathlineto{\pgfqpoint{2.075779in}{1.732871in}}%
\pgfpathclose%
\pgfusepath{fill}%
\end{pgfscope}%
\begin{pgfscope}%
\pgfpathrectangle{\pgfqpoint{0.000000in}{0.000000in}}{\pgfqpoint{3.000000in}{3.000000in}}%
\pgfusepath{clip}%
\pgfsetbuttcap%
\pgfsetroundjoin%
\definecolor{currentfill}{rgb}{0.895003,1.000000,0.072739}%
\pgfsetfillcolor{currentfill}%
\pgfsetlinewidth{0.000000pt}%
\definecolor{currentstroke}{rgb}{0.000000,0.000000,0.000000}%
\pgfsetstrokecolor{currentstroke}%
\pgfsetdash{}{0pt}%
\pgfpathmoveto{\pgfqpoint{1.112866in}{1.613546in}}%
\pgfpathlineto{\pgfqpoint{1.095973in}{1.632888in}}%
\pgfpathlineto{\pgfqpoint{1.118546in}{1.599062in}}%
\pgfpathlineto{\pgfqpoint{1.134638in}{1.580941in}}%
\pgfpathlineto{\pgfqpoint{1.112866in}{1.613546in}}%
\pgfpathclose%
\pgfusepath{fill}%
\end{pgfscope}%
\begin{pgfscope}%
\pgfpathrectangle{\pgfqpoint{0.000000in}{0.000000in}}{\pgfqpoint{3.000000in}{3.000000in}}%
\pgfusepath{clip}%
\pgfsetbuttcap%
\pgfsetroundjoin%
\definecolor{currentfill}{rgb}{0.999109,0.073348,0.000000}%
\pgfsetfillcolor{currentfill}%
\pgfsetlinewidth{0.000000pt}%
\definecolor{currentstroke}{rgb}{0.000000,0.000000,0.000000}%
\pgfsetstrokecolor{currentstroke}%
\pgfsetdash{}{0pt}%
\pgfpathmoveto{\pgfqpoint{2.241381in}{1.915726in}}%
\pgfpathlineto{\pgfqpoint{2.258877in}{1.929864in}}%
\pgfpathlineto{\pgfqpoint{2.251637in}{1.983612in}}%
\pgfpathlineto{\pgfqpoint{2.234419in}{1.968255in}}%
\pgfpathlineto{\pgfqpoint{2.241381in}{1.915726in}}%
\pgfpathclose%
\pgfusepath{fill}%
\end{pgfscope}%
\begin{pgfscope}%
\pgfpathrectangle{\pgfqpoint{0.000000in}{0.000000in}}{\pgfqpoint{3.000000in}{3.000000in}}%
\pgfusepath{clip}%
\pgfsetbuttcap%
\pgfsetroundjoin%
\definecolor{currentfill}{rgb}{0.401645,1.000000,0.566097}%
\pgfsetfillcolor{currentfill}%
\pgfsetlinewidth{0.000000pt}%
\definecolor{currentstroke}{rgb}{0.000000,0.000000,0.000000}%
\pgfsetstrokecolor{currentstroke}%
\pgfsetdash{}{0pt}%
\pgfpathmoveto{\pgfqpoint{1.287275in}{1.404676in}}%
\pgfpathlineto{\pgfqpoint{1.274176in}{1.427697in}}%
\pgfpathlineto{\pgfqpoint{1.314506in}{1.408065in}}%
\pgfpathlineto{\pgfqpoint{1.325644in}{1.385967in}}%
\pgfpathlineto{\pgfqpoint{1.287275in}{1.404676in}}%
\pgfpathclose%
\pgfusepath{fill}%
\end{pgfscope}%
\begin{pgfscope}%
\pgfpathrectangle{\pgfqpoint{0.000000in}{0.000000in}}{\pgfqpoint{3.000000in}{3.000000in}}%
\pgfusepath{clip}%
\pgfsetbuttcap%
\pgfsetroundjoin%
\definecolor{currentfill}{rgb}{0.578748,1.000000,0.388994}%
\pgfsetfillcolor{currentfill}%
\pgfsetlinewidth{0.000000pt}%
\definecolor{currentstroke}{rgb}{0.000000,0.000000,0.000000}%
\pgfsetstrokecolor{currentstroke}%
\pgfsetdash{}{0pt}%
\pgfpathmoveto{\pgfqpoint{1.832585in}{1.456688in}}%
\pgfpathlineto{\pgfqpoint{1.846317in}{1.477078in}}%
\pgfpathlineto{\pgfqpoint{1.880364in}{1.502622in}}%
\pgfpathlineto{\pgfqpoint{1.865143in}{1.481141in}}%
\pgfpathlineto{\pgfqpoint{1.832585in}{1.456688in}}%
\pgfpathclose%
\pgfusepath{fill}%
\end{pgfscope}%
\begin{pgfscope}%
\pgfpathrectangle{\pgfqpoint{0.000000in}{0.000000in}}{\pgfqpoint{3.000000in}{3.000000in}}%
\pgfusepath{clip}%
\pgfsetbuttcap%
\pgfsetroundjoin%
\definecolor{currentfill}{rgb}{0.300443,1.000000,0.667299}%
\pgfsetfillcolor{currentfill}%
\pgfsetlinewidth{0.000000pt}%
\definecolor{currentstroke}{rgb}{0.000000,0.000000,0.000000}%
\pgfsetstrokecolor{currentstroke}%
\pgfsetdash{}{0pt}%
\pgfpathmoveto{\pgfqpoint{1.336758in}{1.362046in}}%
\pgfpathlineto{\pgfqpoint{1.325644in}{1.385967in}}%
\pgfpathlineto{\pgfqpoint{1.369527in}{1.370516in}}%
\pgfpathlineto{\pgfqpoint{1.378384in}{1.347359in}}%
\pgfpathlineto{\pgfqpoint{1.336758in}{1.362046in}}%
\pgfpathclose%
\pgfusepath{fill}%
\end{pgfscope}%
\begin{pgfscope}%
\pgfpathrectangle{\pgfqpoint{0.000000in}{0.000000in}}{\pgfqpoint{3.000000in}{3.000000in}}%
\pgfusepath{clip}%
\pgfsetbuttcap%
\pgfsetroundjoin%
\definecolor{currentfill}{rgb}{0.199241,1.000000,0.768501}%
\pgfsetfillcolor{currentfill}%
\pgfsetlinewidth{0.000000pt}%
\definecolor{currentstroke}{rgb}{0.000000,0.000000,0.000000}%
\pgfsetstrokecolor{currentstroke}%
\pgfsetdash{}{0pt}%
\pgfpathmoveto{\pgfqpoint{1.620035in}{1.306386in}}%
\pgfpathlineto{\pgfqpoint{1.624624in}{1.330796in}}%
\pgfpathlineto{\pgfqpoint{1.672559in}{1.339488in}}%
\pgfpathlineto{\pgfqpoint{1.665359in}{1.314625in}}%
\pgfpathlineto{\pgfqpoint{1.620035in}{1.306386in}}%
\pgfpathclose%
\pgfusepath{fill}%
\end{pgfscope}%
\begin{pgfscope}%
\pgfpathrectangle{\pgfqpoint{0.000000in}{0.000000in}}{\pgfqpoint{3.000000in}{3.000000in}}%
\pgfusepath{clip}%
\pgfsetbuttcap%
\pgfsetroundjoin%
\definecolor{currentfill}{rgb}{1.000000,0.233115,0.000000}%
\pgfsetfillcolor{currentfill}%
\pgfsetlinewidth{0.000000pt}%
\definecolor{currentstroke}{rgb}{0.000000,0.000000,0.000000}%
\pgfsetstrokecolor{currentstroke}%
\pgfsetdash{}{0pt}%
\pgfpathmoveto{\pgfqpoint{0.894453in}{1.904350in}}%
\pgfpathlineto{\pgfqpoint{0.877125in}{1.920138in}}%
\pgfpathlineto{\pgfqpoint{0.876156in}{1.869867in}}%
\pgfpathlineto{\pgfqpoint{0.893605in}{1.855313in}}%
\pgfpathlineto{\pgfqpoint{0.894453in}{1.904350in}}%
\pgfpathclose%
\pgfusepath{fill}%
\end{pgfscope}%
\begin{pgfscope}%
\pgfpathrectangle{\pgfqpoint{0.000000in}{0.000000in}}{\pgfqpoint{3.000000in}{3.000000in}}%
\pgfusepath{clip}%
\pgfsetbuttcap%
\pgfsetroundjoin%
\definecolor{currentfill}{rgb}{0.199241,1.000000,0.768501}%
\pgfsetfillcolor{currentfill}%
\pgfsetlinewidth{0.000000pt}%
\definecolor{currentstroke}{rgb}{0.000000,0.000000,0.000000}%
\pgfsetstrokecolor{currentstroke}%
\pgfsetdash{}{0pt}%
\pgfpathmoveto{\pgfqpoint{1.430529in}{1.311475in}}%
\pgfpathlineto{\pgfqpoint{1.424181in}{1.336165in}}%
\pgfpathlineto{\pgfqpoint{1.472981in}{1.328766in}}%
\pgfpathlineto{\pgfqpoint{1.476669in}{1.304462in}}%
\pgfpathlineto{\pgfqpoint{1.430529in}{1.311475in}}%
\pgfpathclose%
\pgfusepath{fill}%
\end{pgfscope}%
\begin{pgfscope}%
\pgfpathrectangle{\pgfqpoint{0.000000in}{0.000000in}}{\pgfqpoint{3.000000in}{3.000000in}}%
\pgfusepath{clip}%
\pgfsetbuttcap%
\pgfsetroundjoin%
\definecolor{currentfill}{rgb}{0.927807,0.015251,0.000000}%
\pgfsetfillcolor{currentfill}%
\pgfsetlinewidth{0.000000pt}%
\definecolor{currentstroke}{rgb}{0.000000,0.000000,0.000000}%
\pgfsetstrokecolor{currentstroke}%
\pgfsetdash{}{0pt}%
\pgfpathmoveto{\pgfqpoint{2.258877in}{1.929864in}}%
\pgfpathlineto{\pgfqpoint{2.276385in}{1.943758in}}%
\pgfpathlineto{\pgfqpoint{2.268863in}{1.998723in}}%
\pgfpathlineto{\pgfqpoint{2.251637in}{1.983612in}}%
\pgfpathlineto{\pgfqpoint{2.258877in}{1.929864in}}%
\pgfpathclose%
\pgfusepath{fill}%
\end{pgfscope}%
\begin{pgfscope}%
\pgfpathrectangle{\pgfqpoint{0.000000in}{0.000000in}}{\pgfqpoint{3.000000in}{3.000000in}}%
\pgfusepath{clip}%
\pgfsetbuttcap%
\pgfsetroundjoin%
\definecolor{currentfill}{rgb}{1.000000,0.668845,0.000000}%
\pgfsetfillcolor{currentfill}%
\pgfsetlinewidth{0.000000pt}%
\definecolor{currentstroke}{rgb}{0.000000,0.000000,0.000000}%
\pgfsetstrokecolor{currentstroke}%
\pgfsetdash{}{0pt}%
\pgfpathmoveto{\pgfqpoint{1.015324in}{1.743160in}}%
\pgfpathlineto{\pgfqpoint{0.997981in}{1.760492in}}%
\pgfpathlineto{\pgfqpoint{1.011202in}{1.719209in}}%
\pgfpathlineto{\pgfqpoint{1.028197in}{1.703127in}}%
\pgfpathlineto{\pgfqpoint{1.015324in}{1.743160in}}%
\pgfpathclose%
\pgfusepath{fill}%
\end{pgfscope}%
\begin{pgfscope}%
\pgfpathrectangle{\pgfqpoint{0.000000in}{0.000000in}}{\pgfqpoint{3.000000in}{3.000000in}}%
\pgfusepath{clip}%
\pgfsetbuttcap%
\pgfsetroundjoin%
\definecolor{currentfill}{rgb}{1.000000,0.538126,0.000000}%
\pgfsetfillcolor{currentfill}%
\pgfsetlinewidth{0.000000pt}%
\definecolor{currentstroke}{rgb}{0.000000,0.000000,0.000000}%
\pgfsetstrokecolor{currentstroke}%
\pgfsetdash{}{0pt}%
\pgfpathmoveto{\pgfqpoint{2.092956in}{1.748874in}}%
\pgfpathlineto{\pgfqpoint{2.110152in}{1.764422in}}%
\pgfpathlineto{\pgfqpoint{2.119279in}{1.808424in}}%
\pgfpathlineto{\pgfqpoint{2.101890in}{1.791627in}}%
\pgfpathlineto{\pgfqpoint{2.092956in}{1.748874in}}%
\pgfpathclose%
\pgfusepath{fill}%
\end{pgfscope}%
\begin{pgfscope}%
\pgfpathrectangle{\pgfqpoint{0.000000in}{0.000000in}}{\pgfqpoint{3.000000in}{3.000000in}}%
\pgfusepath{clip}%
\pgfsetbuttcap%
\pgfsetroundjoin%
\definecolor{currentfill}{rgb}{1.000000,0.886710,0.000000}%
\pgfsetfillcolor{currentfill}%
\pgfsetlinewidth{0.000000pt}%
\definecolor{currentstroke}{rgb}{0.000000,0.000000,0.000000}%
\pgfsetstrokecolor{currentstroke}%
\pgfsetdash{}{0pt}%
\pgfpathmoveto{\pgfqpoint{1.987705in}{1.627890in}}%
\pgfpathlineto{\pgfqpoint{2.004156in}{1.644921in}}%
\pgfpathlineto{\pgfqpoint{2.024358in}{1.681716in}}%
\pgfpathlineto{\pgfqpoint{2.007255in}{1.663449in}}%
\pgfpathlineto{\pgfqpoint{1.987705in}{1.627890in}}%
\pgfpathclose%
\pgfusepath{fill}%
\end{pgfscope}%
\begin{pgfscope}%
\pgfpathrectangle{\pgfqpoint{0.000000in}{0.000000in}}{\pgfqpoint{3.000000in}{3.000000in}}%
\pgfusepath{clip}%
\pgfsetbuttcap%
\pgfsetroundjoin%
\definecolor{currentfill}{rgb}{0.743201,1.000000,0.224541}%
\pgfsetfillcolor{currentfill}%
\pgfsetlinewidth{0.000000pt}%
\definecolor{currentstroke}{rgb}{0.000000,0.000000,0.000000}%
\pgfsetstrokecolor{currentstroke}%
\pgfsetdash{}{0pt}%
\pgfpathmoveto{\pgfqpoint{1.166750in}{1.541998in}}%
\pgfpathlineto{\pgfqpoint{1.150706in}{1.561953in}}%
\pgfpathlineto{\pgfqpoint{1.181529in}{1.532505in}}%
\pgfpathlineto{\pgfqpoint{1.196349in}{1.513705in}}%
\pgfpathlineto{\pgfqpoint{1.166750in}{1.541998in}}%
\pgfpathclose%
\pgfusepath{fill}%
\end{pgfscope}%
\begin{pgfscope}%
\pgfpathrectangle{\pgfqpoint{0.000000in}{0.000000in}}{\pgfqpoint{3.000000in}{3.000000in}}%
\pgfusepath{clip}%
\pgfsetbuttcap%
\pgfsetroundjoin%
\definecolor{currentfill}{rgb}{0.199241,1.000000,0.768501}%
\pgfsetfillcolor{currentfill}%
\pgfsetlinewidth{0.000000pt}%
\definecolor{currentstroke}{rgb}{0.000000,0.000000,0.000000}%
\pgfsetstrokecolor{currentstroke}%
\pgfsetdash{}{0pt}%
\pgfpathmoveto{\pgfqpoint{1.572661in}{1.301883in}}%
\pgfpathlineto{\pgfqpoint{1.574516in}{1.326045in}}%
\pgfpathlineto{\pgfqpoint{1.624624in}{1.330796in}}%
\pgfpathlineto{\pgfqpoint{1.620035in}{1.306386in}}%
\pgfpathlineto{\pgfqpoint{1.572661in}{1.301883in}}%
\pgfpathclose%
\pgfusepath{fill}%
\end{pgfscope}%
\begin{pgfscope}%
\pgfpathrectangle{\pgfqpoint{0.000000in}{0.000000in}}{\pgfqpoint{3.000000in}{3.000000in}}%
\pgfusepath{clip}%
\pgfsetbuttcap%
\pgfsetroundjoin%
\definecolor{currentfill}{rgb}{0.958254,0.973856,0.009488}%
\pgfsetfillcolor{currentfill}%
\pgfsetlinewidth{0.000000pt}%
\definecolor{currentstroke}{rgb}{0.000000,0.000000,0.000000}%
\pgfsetstrokecolor{currentstroke}%
\pgfsetdash{}{0pt}%
\pgfpathmoveto{\pgfqpoint{1.095973in}{1.632888in}}%
\pgfpathlineto{\pgfqpoint{1.079059in}{1.651448in}}%
\pgfpathlineto{\pgfqpoint{1.102431in}{1.616400in}}%
\pgfpathlineto{\pgfqpoint{1.118546in}{1.599062in}}%
\pgfpathlineto{\pgfqpoint{1.095973in}{1.632888in}}%
\pgfpathclose%
\pgfusepath{fill}%
\end{pgfscope}%
\begin{pgfscope}%
\pgfpathrectangle{\pgfqpoint{0.000000in}{0.000000in}}{\pgfqpoint{3.000000in}{3.000000in}}%
\pgfusepath{clip}%
\pgfsetbuttcap%
\pgfsetroundjoin%
\definecolor{currentfill}{rgb}{0.578748,1.000000,0.388994}%
\pgfsetfillcolor{currentfill}%
\pgfsetlinewidth{0.000000pt}%
\definecolor{currentstroke}{rgb}{0.000000,0.000000,0.000000}%
\pgfsetstrokecolor{currentstroke}%
\pgfsetdash{}{0pt}%
\pgfpathmoveto{\pgfqpoint{1.225910in}{1.472715in}}%
\pgfpathlineto{\pgfqpoint{1.211142in}{1.493821in}}%
\pgfpathlineto{\pgfqpoint{1.247897in}{1.469187in}}%
\pgfpathlineto{\pgfqpoint{1.261050in}{1.449135in}}%
\pgfpathlineto{\pgfqpoint{1.225910in}{1.472715in}}%
\pgfpathclose%
\pgfusepath{fill}%
\end{pgfscope}%
\begin{pgfscope}%
\pgfpathrectangle{\pgfqpoint{0.000000in}{0.000000in}}{\pgfqpoint{3.000000in}{3.000000in}}%
\pgfusepath{clip}%
\pgfsetbuttcap%
\pgfsetroundjoin%
\definecolor{currentfill}{rgb}{1.000000,0.175018,0.000000}%
\pgfsetfillcolor{currentfill}%
\pgfsetlinewidth{0.000000pt}%
\definecolor{currentstroke}{rgb}{0.000000,0.000000,0.000000}%
\pgfsetstrokecolor{currentstroke}%
\pgfsetdash{}{0pt}%
\pgfpathmoveto{\pgfqpoint{0.877125in}{1.920138in}}%
\pgfpathlineto{\pgfqpoint{0.859786in}{1.935631in}}%
\pgfpathlineto{\pgfqpoint{0.858693in}{1.884126in}}%
\pgfpathlineto{\pgfqpoint{0.876156in}{1.869867in}}%
\pgfpathlineto{\pgfqpoint{0.877125in}{1.920138in}}%
\pgfpathclose%
\pgfusepath{fill}%
\end{pgfscope}%
\begin{pgfscope}%
\pgfpathrectangle{\pgfqpoint{0.000000in}{0.000000in}}{\pgfqpoint{3.000000in}{3.000000in}}%
\pgfusepath{clip}%
\pgfsetbuttcap%
\pgfsetroundjoin%
\definecolor{currentfill}{rgb}{0.490196,1.000000,0.477546}%
\pgfsetfillcolor{currentfill}%
\pgfsetlinewidth{0.000000pt}%
\definecolor{currentstroke}{rgb}{0.000000,0.000000,0.000000}%
\pgfsetstrokecolor{currentstroke}%
\pgfsetdash{}{0pt}%
\pgfpathmoveto{\pgfqpoint{1.780717in}{1.414254in}}%
\pgfpathlineto{\pgfqpoint{1.792570in}{1.435060in}}%
\pgfpathlineto{\pgfqpoint{1.832585in}{1.456688in}}%
\pgfpathlineto{\pgfqpoint{1.818880in}{1.434912in}}%
\pgfpathlineto{\pgfqpoint{1.780717in}{1.414254in}}%
\pgfpathclose%
\pgfusepath{fill}%
\end{pgfscope}%
\begin{pgfscope}%
\pgfpathrectangle{\pgfqpoint{0.000000in}{0.000000in}}{\pgfqpoint{3.000000in}{3.000000in}}%
\pgfusepath{clip}%
\pgfsetbuttcap%
\pgfsetroundjoin%
\definecolor{currentfill}{rgb}{0.199241,1.000000,0.768501}%
\pgfsetfillcolor{currentfill}%
\pgfsetlinewidth{0.000000pt}%
\definecolor{currentstroke}{rgb}{0.000000,0.000000,0.000000}%
\pgfsetstrokecolor{currentstroke}%
\pgfsetdash{}{0pt}%
\pgfpathmoveto{\pgfqpoint{1.476669in}{1.304462in}}%
\pgfpathlineto{\pgfqpoint{1.472981in}{1.328766in}}%
\pgfpathlineto{\pgfqpoint{1.523528in}{1.325362in}}%
\pgfpathlineto{\pgfqpoint{1.524457in}{1.301236in}}%
\pgfpathlineto{\pgfqpoint{1.476669in}{1.304462in}}%
\pgfpathclose%
\pgfusepath{fill}%
\end{pgfscope}%
\begin{pgfscope}%
\pgfpathrectangle{\pgfqpoint{0.000000in}{0.000000in}}{\pgfqpoint{3.000000in}{3.000000in}}%
\pgfusepath{clip}%
\pgfsetbuttcap%
\pgfsetroundjoin%
\definecolor{currentfill}{rgb}{0.856506,0.000000,0.000000}%
\pgfsetfillcolor{currentfill}%
\pgfsetlinewidth{0.000000pt}%
\definecolor{currentstroke}{rgb}{0.000000,0.000000,0.000000}%
\pgfsetstrokecolor{currentstroke}%
\pgfsetdash{}{0pt}%
\pgfpathmoveto{\pgfqpoint{2.276385in}{1.943758in}}%
\pgfpathlineto{\pgfqpoint{2.293906in}{1.957424in}}%
\pgfpathlineto{\pgfqpoint{2.286096in}{2.013603in}}%
\pgfpathlineto{\pgfqpoint{2.268863in}{1.998723in}}%
\pgfpathlineto{\pgfqpoint{2.276385in}{1.943758in}}%
\pgfpathclose%
\pgfusepath{fill}%
\end{pgfscope}%
\begin{pgfscope}%
\pgfpathrectangle{\pgfqpoint{0.000000in}{0.000000in}}{\pgfqpoint{3.000000in}{3.000000in}}%
\pgfusepath{clip}%
\pgfsetbuttcap%
\pgfsetroundjoin%
\definecolor{currentfill}{rgb}{0.819102,1.000000,0.148640}%
\pgfsetfillcolor{currentfill}%
\pgfsetlinewidth{0.000000pt}%
\definecolor{currentstroke}{rgb}{0.000000,0.000000,0.000000}%
\pgfsetstrokecolor{currentstroke}%
\pgfsetdash{}{0pt}%
\pgfpathmoveto{\pgfqpoint{1.910884in}{1.542058in}}%
\pgfpathlineto{\pgfqpoint{1.926182in}{1.560266in}}%
\pgfpathlineto{\pgfqpoint{1.954869in}{1.591628in}}%
\pgfpathlineto{\pgfqpoint{1.938486in}{1.572239in}}%
\pgfpathlineto{\pgfqpoint{1.910884in}{1.542058in}}%
\pgfpathclose%
\pgfusepath{fill}%
\end{pgfscope}%
\begin{pgfscope}%
\pgfpathrectangle{\pgfqpoint{0.000000in}{0.000000in}}{\pgfqpoint{3.000000in}{3.000000in}}%
\pgfusepath{clip}%
\pgfsetbuttcap%
\pgfsetroundjoin%
\definecolor{currentfill}{rgb}{0.300443,1.000000,0.667299}%
\pgfsetfillcolor{currentfill}%
\pgfsetlinewidth{0.000000pt}%
\definecolor{currentstroke}{rgb}{0.000000,0.000000,0.000000}%
\pgfsetstrokecolor{currentstroke}%
\pgfsetdash{}{0pt}%
\pgfpathmoveto{\pgfqpoint{1.672559in}{1.339488in}}%
\pgfpathlineto{\pgfqpoint{1.679777in}{1.362235in}}%
\pgfpathlineto{\pgfqpoint{1.726729in}{1.375277in}}%
\pgfpathlineto{\pgfqpoint{1.717090in}{1.351884in}}%
\pgfpathlineto{\pgfqpoint{1.672559in}{1.339488in}}%
\pgfpathclose%
\pgfusepath{fill}%
\end{pgfscope}%
\begin{pgfscope}%
\pgfpathrectangle{\pgfqpoint{0.000000in}{0.000000in}}{\pgfqpoint{3.000000in}{3.000000in}}%
\pgfusepath{clip}%
\pgfsetbuttcap%
\pgfsetroundjoin%
\definecolor{currentfill}{rgb}{0.199241,1.000000,0.768501}%
\pgfsetfillcolor{currentfill}%
\pgfsetlinewidth{0.000000pt}%
\definecolor{currentstroke}{rgb}{0.000000,0.000000,0.000000}%
\pgfsetstrokecolor{currentstroke}%
\pgfsetdash{}{0pt}%
\pgfpathmoveto{\pgfqpoint{1.524457in}{1.301236in}}%
\pgfpathlineto{\pgfqpoint{1.523528in}{1.325362in}}%
\pgfpathlineto{\pgfqpoint{1.574516in}{1.326045in}}%
\pgfpathlineto{\pgfqpoint{1.572661in}{1.301883in}}%
\pgfpathlineto{\pgfqpoint{1.524457in}{1.301236in}}%
\pgfpathclose%
\pgfusepath{fill}%
\end{pgfscope}%
\begin{pgfscope}%
\pgfpathrectangle{\pgfqpoint{0.000000in}{0.000000in}}{\pgfqpoint{3.000000in}{3.000000in}}%
\pgfusepath{clip}%
\pgfsetbuttcap%
\pgfsetroundjoin%
\definecolor{currentfill}{rgb}{1.000000,0.480029,0.000000}%
\pgfsetfillcolor{currentfill}%
\pgfsetlinewidth{0.000000pt}%
\definecolor{currentstroke}{rgb}{0.000000,0.000000,0.000000}%
\pgfsetstrokecolor{currentstroke}%
\pgfsetdash{}{0pt}%
\pgfpathmoveto{\pgfqpoint{2.110152in}{1.764422in}}%
\pgfpathlineto{\pgfqpoint{2.127366in}{1.779549in}}%
\pgfpathlineto{\pgfqpoint{2.136683in}{1.824799in}}%
\pgfpathlineto{\pgfqpoint{2.119279in}{1.808424in}}%
\pgfpathlineto{\pgfqpoint{2.110152in}{1.764422in}}%
\pgfpathclose%
\pgfusepath{fill}%
\end{pgfscope}%
\begin{pgfscope}%
\pgfpathrectangle{\pgfqpoint{0.000000in}{0.000000in}}{\pgfqpoint{3.000000in}{3.000000in}}%
\pgfusepath{clip}%
\pgfsetbuttcap%
\pgfsetroundjoin%
\definecolor{currentfill}{rgb}{0.401645,1.000000,0.566097}%
\pgfsetfillcolor{currentfill}%
\pgfsetlinewidth{0.000000pt}%
\definecolor{currentstroke}{rgb}{0.000000,0.000000,0.000000}%
\pgfsetstrokecolor{currentstroke}%
\pgfsetdash{}{0pt}%
\pgfpathmoveto{\pgfqpoint{1.726729in}{1.375277in}}%
\pgfpathlineto{\pgfqpoint{1.736391in}{1.396847in}}%
\pgfpathlineto{\pgfqpoint{1.780717in}{1.414254in}}%
\pgfpathlineto{\pgfqpoint{1.768889in}{1.391864in}}%
\pgfpathlineto{\pgfqpoint{1.726729in}{1.375277in}}%
\pgfpathclose%
\pgfusepath{fill}%
\end{pgfscope}%
\begin{pgfscope}%
\pgfpathrectangle{\pgfqpoint{0.000000in}{0.000000in}}{\pgfqpoint{3.000000in}{3.000000in}}%
\pgfusepath{clip}%
\pgfsetbuttcap%
\pgfsetroundjoin%
\definecolor{currentfill}{rgb}{1.000000,0.610748,0.000000}%
\pgfsetfillcolor{currentfill}%
\pgfsetlinewidth{0.000000pt}%
\definecolor{currentstroke}{rgb}{0.000000,0.000000,0.000000}%
\pgfsetstrokecolor{currentstroke}%
\pgfsetdash{}{0pt}%
\pgfpathmoveto{\pgfqpoint{0.997981in}{1.760492in}}%
\pgfpathlineto{\pgfqpoint{0.980623in}{1.777329in}}%
\pgfpathlineto{\pgfqpoint{0.994188in}{1.734797in}}%
\pgfpathlineto{\pgfqpoint{1.011202in}{1.719209in}}%
\pgfpathlineto{\pgfqpoint{0.997981in}{1.760492in}}%
\pgfpathclose%
\pgfusepath{fill}%
\end{pgfscope}%
\begin{pgfscope}%
\pgfpathrectangle{\pgfqpoint{0.000000in}{0.000000in}}{\pgfqpoint{3.000000in}{3.000000in}}%
\pgfusepath{clip}%
\pgfsetbuttcap%
\pgfsetroundjoin%
\definecolor{currentfill}{rgb}{0.300443,1.000000,0.667299}%
\pgfsetfillcolor{currentfill}%
\pgfsetlinewidth{0.000000pt}%
\definecolor{currentstroke}{rgb}{0.000000,0.000000,0.000000}%
\pgfsetstrokecolor{currentstroke}%
\pgfsetdash{}{0pt}%
\pgfpathmoveto{\pgfqpoint{1.378384in}{1.347359in}}%
\pgfpathlineto{\pgfqpoint{1.369527in}{1.370516in}}%
\pgfpathlineto{\pgfqpoint{1.417817in}{1.358739in}}%
\pgfpathlineto{\pgfqpoint{1.424181in}{1.336165in}}%
\pgfpathlineto{\pgfqpoint{1.378384in}{1.347359in}}%
\pgfpathclose%
\pgfusepath{fill}%
\end{pgfscope}%
\begin{pgfscope}%
\pgfpathrectangle{\pgfqpoint{0.000000in}{0.000000in}}{\pgfqpoint{3.000000in}{3.000000in}}%
\pgfusepath{clip}%
\pgfsetbuttcap%
\pgfsetroundjoin%
\definecolor{currentfill}{rgb}{0.803030,0.000000,0.000000}%
\pgfsetfillcolor{currentfill}%
\pgfsetlinewidth{0.000000pt}%
\definecolor{currentstroke}{rgb}{0.000000,0.000000,0.000000}%
\pgfsetstrokecolor{currentstroke}%
\pgfsetdash{}{0pt}%
\pgfpathmoveto{\pgfqpoint{2.293906in}{1.957424in}}%
\pgfpathlineto{\pgfqpoint{2.311440in}{1.970873in}}%
\pgfpathlineto{\pgfqpoint{2.303337in}{2.028265in}}%
\pgfpathlineto{\pgfqpoint{2.286096in}{2.013603in}}%
\pgfpathlineto{\pgfqpoint{2.293906in}{1.957424in}}%
\pgfpathclose%
\pgfusepath{fill}%
\end{pgfscope}%
\begin{pgfscope}%
\pgfpathrectangle{\pgfqpoint{0.000000in}{0.000000in}}{\pgfqpoint{3.000000in}{3.000000in}}%
\pgfusepath{clip}%
\pgfsetbuttcap%
\pgfsetroundjoin%
\definecolor{currentfill}{rgb}{0.667299,1.000000,0.300443}%
\pgfsetfillcolor{currentfill}%
\pgfsetlinewidth{0.000000pt}%
\definecolor{currentstroke}{rgb}{0.000000,0.000000,0.000000}%
\pgfsetstrokecolor{currentstroke}%
\pgfsetdash{}{0pt}%
\pgfpathmoveto{\pgfqpoint{1.846317in}{1.477078in}}%
\pgfpathlineto{\pgfqpoint{1.860076in}{1.496246in}}%
\pgfpathlineto{\pgfqpoint{1.895612in}{1.522882in}}%
\pgfpathlineto{\pgfqpoint{1.880364in}{1.502622in}}%
\pgfpathlineto{\pgfqpoint{1.846317in}{1.477078in}}%
\pgfpathclose%
\pgfusepath{fill}%
\end{pgfscope}%
\begin{pgfscope}%
\pgfpathrectangle{\pgfqpoint{0.000000in}{0.000000in}}{\pgfqpoint{3.000000in}{3.000000in}}%
\pgfusepath{clip}%
\pgfsetbuttcap%
\pgfsetroundjoin%
\definecolor{currentfill}{rgb}{1.000000,0.116921,0.000000}%
\pgfsetfillcolor{currentfill}%
\pgfsetlinewidth{0.000000pt}%
\definecolor{currentstroke}{rgb}{0.000000,0.000000,0.000000}%
\pgfsetstrokecolor{currentstroke}%
\pgfsetdash{}{0pt}%
\pgfpathmoveto{\pgfqpoint{0.859786in}{1.935631in}}%
\pgfpathlineto{\pgfqpoint{0.842439in}{1.950845in}}%
\pgfpathlineto{\pgfqpoint{0.841215in}{1.898111in}}%
\pgfpathlineto{\pgfqpoint{0.858693in}{1.884126in}}%
\pgfpathlineto{\pgfqpoint{0.859786in}{1.935631in}}%
\pgfpathclose%
\pgfusepath{fill}%
\end{pgfscope}%
\begin{pgfscope}%
\pgfpathrectangle{\pgfqpoint{0.000000in}{0.000000in}}{\pgfqpoint{3.000000in}{3.000000in}}%
\pgfusepath{clip}%
\pgfsetbuttcap%
\pgfsetroundjoin%
\definecolor{currentfill}{rgb}{1.000000,0.814089,0.000000}%
\pgfsetfillcolor{currentfill}%
\pgfsetlinewidth{0.000000pt}%
\definecolor{currentstroke}{rgb}{0.000000,0.000000,0.000000}%
\pgfsetstrokecolor{currentstroke}%
\pgfsetdash{}{0pt}%
\pgfpathmoveto{\pgfqpoint{2.004156in}{1.644921in}}%
\pgfpathlineto{\pgfqpoint{2.020631in}{1.661310in}}%
\pgfpathlineto{\pgfqpoint{2.041480in}{1.699339in}}%
\pgfpathlineto{\pgfqpoint{2.024358in}{1.681716in}}%
\pgfpathlineto{\pgfqpoint{2.004156in}{1.644921in}}%
\pgfpathclose%
\pgfusepath{fill}%
\end{pgfscope}%
\begin{pgfscope}%
\pgfpathrectangle{\pgfqpoint{0.000000in}{0.000000in}}{\pgfqpoint{3.000000in}{3.000000in}}%
\pgfusepath{clip}%
\pgfsetbuttcap%
\pgfsetroundjoin%
\definecolor{currentfill}{rgb}{0.490196,1.000000,0.477546}%
\pgfsetfillcolor{currentfill}%
\pgfsetlinewidth{0.000000pt}%
\definecolor{currentstroke}{rgb}{0.000000,0.000000,0.000000}%
\pgfsetstrokecolor{currentstroke}%
\pgfsetdash{}{0pt}%
\pgfpathmoveto{\pgfqpoint{1.274176in}{1.427697in}}%
\pgfpathlineto{\pgfqpoint{1.261050in}{1.449135in}}%
\pgfpathlineto{\pgfqpoint{1.303344in}{1.428580in}}%
\pgfpathlineto{\pgfqpoint{1.314506in}{1.408065in}}%
\pgfpathlineto{\pgfqpoint{1.274176in}{1.427697in}}%
\pgfpathclose%
\pgfusepath{fill}%
\end{pgfscope}%
\begin{pgfscope}%
\pgfpathrectangle{\pgfqpoint{0.000000in}{0.000000in}}{\pgfqpoint{3.000000in}{3.000000in}}%
\pgfusepath{clip}%
\pgfsetbuttcap%
\pgfsetroundjoin%
\definecolor{currentfill}{rgb}{0.401645,1.000000,0.566097}%
\pgfsetfillcolor{currentfill}%
\pgfsetlinewidth{0.000000pt}%
\definecolor{currentstroke}{rgb}{0.000000,0.000000,0.000000}%
\pgfsetstrokecolor{currentstroke}%
\pgfsetdash{}{0pt}%
\pgfpathmoveto{\pgfqpoint{1.325644in}{1.385967in}}%
\pgfpathlineto{\pgfqpoint{1.314506in}{1.408065in}}%
\pgfpathlineto{\pgfqpoint{1.360648in}{1.391850in}}%
\pgfpathlineto{\pgfqpoint{1.369527in}{1.370516in}}%
\pgfpathlineto{\pgfqpoint{1.325644in}{1.385967in}}%
\pgfpathclose%
\pgfusepath{fill}%
\end{pgfscope}%
\begin{pgfscope}%
\pgfpathrectangle{\pgfqpoint{0.000000in}{0.000000in}}{\pgfqpoint{3.000000in}{3.000000in}}%
\pgfusepath{clip}%
\pgfsetbuttcap%
\pgfsetroundjoin%
\definecolor{currentfill}{rgb}{0.731729,0.000000,0.000000}%
\pgfsetfillcolor{currentfill}%
\pgfsetlinewidth{0.000000pt}%
\definecolor{currentstroke}{rgb}{0.000000,0.000000,0.000000}%
\pgfsetstrokecolor{currentstroke}%
\pgfsetdash{}{0pt}%
\pgfpathmoveto{\pgfqpoint{2.311440in}{1.970873in}}%
\pgfpathlineto{\pgfqpoint{2.328987in}{1.984118in}}%
\pgfpathlineto{\pgfqpoint{2.320585in}{2.042721in}}%
\pgfpathlineto{\pgfqpoint{2.303337in}{2.028265in}}%
\pgfpathlineto{\pgfqpoint{2.311440in}{1.970873in}}%
\pgfpathclose%
\pgfusepath{fill}%
\end{pgfscope}%
\begin{pgfscope}%
\pgfpathrectangle{\pgfqpoint{0.000000in}{0.000000in}}{\pgfqpoint{3.000000in}{3.000000in}}%
\pgfusepath{clip}%
\pgfsetbuttcap%
\pgfsetroundjoin%
\definecolor{currentfill}{rgb}{1.000000,0.886710,0.000000}%
\pgfsetfillcolor{currentfill}%
\pgfsetlinewidth{0.000000pt}%
\definecolor{currentstroke}{rgb}{0.000000,0.000000,0.000000}%
\pgfsetstrokecolor{currentstroke}%
\pgfsetdash{}{0pt}%
\pgfpathmoveto{\pgfqpoint{1.079059in}{1.651448in}}%
\pgfpathlineto{\pgfqpoint{1.062125in}{1.669299in}}%
\pgfpathlineto{\pgfqpoint{1.086291in}{1.633030in}}%
\pgfpathlineto{\pgfqpoint{1.102431in}{1.616400in}}%
\pgfpathlineto{\pgfqpoint{1.079059in}{1.651448in}}%
\pgfpathclose%
\pgfusepath{fill}%
\end{pgfscope}%
\begin{pgfscope}%
\pgfpathrectangle{\pgfqpoint{0.000000in}{0.000000in}}{\pgfqpoint{3.000000in}{3.000000in}}%
\pgfusepath{clip}%
\pgfsetbuttcap%
\pgfsetroundjoin%
\definecolor{currentfill}{rgb}{1.000000,0.407407,0.000000}%
\pgfsetfillcolor{currentfill}%
\pgfsetlinewidth{0.000000pt}%
\definecolor{currentstroke}{rgb}{0.000000,0.000000,0.000000}%
\pgfsetstrokecolor{currentstroke}%
\pgfsetdash{}{0pt}%
\pgfpathmoveto{\pgfqpoint{2.127366in}{1.779549in}}%
\pgfpathlineto{\pgfqpoint{2.144599in}{1.794286in}}%
\pgfpathlineto{\pgfqpoint{2.154099in}{1.840784in}}%
\pgfpathlineto{\pgfqpoint{2.136683in}{1.824799in}}%
\pgfpathlineto{\pgfqpoint{2.127366in}{1.779549in}}%
\pgfpathclose%
\pgfusepath{fill}%
\end{pgfscope}%
\begin{pgfscope}%
\pgfpathrectangle{\pgfqpoint{0.000000in}{0.000000in}}{\pgfqpoint{3.000000in}{3.000000in}}%
\pgfusepath{clip}%
\pgfsetbuttcap%
\pgfsetroundjoin%
\definecolor{currentfill}{rgb}{0.999109,0.073348,0.000000}%
\pgfsetfillcolor{currentfill}%
\pgfsetlinewidth{0.000000pt}%
\definecolor{currentstroke}{rgb}{0.000000,0.000000,0.000000}%
\pgfsetstrokecolor{currentstroke}%
\pgfsetdash{}{0pt}%
\pgfpathmoveto{\pgfqpoint{0.842439in}{1.950845in}}%
\pgfpathlineto{\pgfqpoint{0.825081in}{1.965800in}}%
\pgfpathlineto{\pgfqpoint{0.823722in}{1.911837in}}%
\pgfpathlineto{\pgfqpoint{0.841215in}{1.898111in}}%
\pgfpathlineto{\pgfqpoint{0.842439in}{1.950845in}}%
\pgfpathclose%
\pgfusepath{fill}%
\end{pgfscope}%
\begin{pgfscope}%
\pgfpathrectangle{\pgfqpoint{0.000000in}{0.000000in}}{\pgfqpoint{3.000000in}{3.000000in}}%
\pgfusepath{clip}%
\pgfsetbuttcap%
\pgfsetroundjoin%
\definecolor{currentfill}{rgb}{0.819102,1.000000,0.148640}%
\pgfsetfillcolor{currentfill}%
\pgfsetlinewidth{0.000000pt}%
\definecolor{currentstroke}{rgb}{0.000000,0.000000,0.000000}%
\pgfsetstrokecolor{currentstroke}%
\pgfsetdash{}{0pt}%
\pgfpathmoveto{\pgfqpoint{1.150706in}{1.561953in}}%
\pgfpathlineto{\pgfqpoint{1.134638in}{1.580941in}}%
\pgfpathlineto{\pgfqpoint{1.166683in}{1.550337in}}%
\pgfpathlineto{\pgfqpoint{1.181529in}{1.532505in}}%
\pgfpathlineto{\pgfqpoint{1.150706in}{1.561953in}}%
\pgfpathclose%
\pgfusepath{fill}%
\end{pgfscope}%
\begin{pgfscope}%
\pgfpathrectangle{\pgfqpoint{0.000000in}{0.000000in}}{\pgfqpoint{3.000000in}{3.000000in}}%
\pgfusepath{clip}%
\pgfsetbuttcap%
\pgfsetroundjoin%
\definecolor{currentfill}{rgb}{1.000000,0.538126,0.000000}%
\pgfsetfillcolor{currentfill}%
\pgfsetlinewidth{0.000000pt}%
\definecolor{currentstroke}{rgb}{0.000000,0.000000,0.000000}%
\pgfsetstrokecolor{currentstroke}%
\pgfsetdash{}{0pt}%
\pgfpathmoveto{\pgfqpoint{0.980623in}{1.777329in}}%
\pgfpathlineto{\pgfqpoint{0.963250in}{1.793710in}}%
\pgfpathlineto{\pgfqpoint{0.977154in}{1.749931in}}%
\pgfpathlineto{\pgfqpoint{0.994188in}{1.734797in}}%
\pgfpathlineto{\pgfqpoint{0.980623in}{1.777329in}}%
\pgfpathclose%
\pgfusepath{fill}%
\end{pgfscope}%
\begin{pgfscope}%
\pgfpathrectangle{\pgfqpoint{0.000000in}{0.000000in}}{\pgfqpoint{3.000000in}{3.000000in}}%
\pgfusepath{clip}%
\pgfsetbuttcap%
\pgfsetroundjoin%
\definecolor{currentfill}{rgb}{0.300443,1.000000,0.667299}%
\pgfsetfillcolor{currentfill}%
\pgfsetlinewidth{0.000000pt}%
\definecolor{currentstroke}{rgb}{0.000000,0.000000,0.000000}%
\pgfsetstrokecolor{currentstroke}%
\pgfsetdash{}{0pt}%
\pgfpathmoveto{\pgfqpoint{1.624624in}{1.330796in}}%
\pgfpathlineto{\pgfqpoint{1.629226in}{1.353090in}}%
\pgfpathlineto{\pgfqpoint{1.679777in}{1.362235in}}%
\pgfpathlineto{\pgfqpoint{1.672559in}{1.339488in}}%
\pgfpathlineto{\pgfqpoint{1.624624in}{1.330796in}}%
\pgfpathclose%
\pgfusepath{fill}%
\end{pgfscope}%
\begin{pgfscope}%
\pgfpathrectangle{\pgfqpoint{0.000000in}{0.000000in}}{\pgfqpoint{3.000000in}{3.000000in}}%
\pgfusepath{clip}%
\pgfsetbuttcap%
\pgfsetroundjoin%
\definecolor{currentfill}{rgb}{0.895003,1.000000,0.072739}%
\pgfsetfillcolor{currentfill}%
\pgfsetlinewidth{0.000000pt}%
\definecolor{currentstroke}{rgb}{0.000000,0.000000,0.000000}%
\pgfsetstrokecolor{currentstroke}%
\pgfsetdash{}{0pt}%
\pgfpathmoveto{\pgfqpoint{1.926182in}{1.560266in}}%
\pgfpathlineto{\pgfqpoint{1.941506in}{1.577606in}}%
\pgfpathlineto{\pgfqpoint{1.971275in}{1.610150in}}%
\pgfpathlineto{\pgfqpoint{1.954869in}{1.591628in}}%
\pgfpathlineto{\pgfqpoint{1.926182in}{1.560266in}}%
\pgfpathclose%
\pgfusepath{fill}%
\end{pgfscope}%
\begin{pgfscope}%
\pgfpathrectangle{\pgfqpoint{0.000000in}{0.000000in}}{\pgfqpoint{3.000000in}{3.000000in}}%
\pgfusepath{clip}%
\pgfsetbuttcap%
\pgfsetroundjoin%
\definecolor{currentfill}{rgb}{0.678253,0.000000,0.000000}%
\pgfsetfillcolor{currentfill}%
\pgfsetlinewidth{0.000000pt}%
\definecolor{currentstroke}{rgb}{0.000000,0.000000,0.000000}%
\pgfsetstrokecolor{currentstroke}%
\pgfsetdash{}{0pt}%
\pgfpathmoveto{\pgfqpoint{2.328987in}{1.984118in}}%
\pgfpathlineto{\pgfqpoint{2.346547in}{1.997171in}}%
\pgfpathlineto{\pgfqpoint{2.337840in}{2.056982in}}%
\pgfpathlineto{\pgfqpoint{2.320585in}{2.042721in}}%
\pgfpathlineto{\pgfqpoint{2.328987in}{1.984118in}}%
\pgfpathclose%
\pgfusepath{fill}%
\end{pgfscope}%
\begin{pgfscope}%
\pgfpathrectangle{\pgfqpoint{0.000000in}{0.000000in}}{\pgfqpoint{3.000000in}{3.000000in}}%
\pgfusepath{clip}%
\pgfsetbuttcap%
\pgfsetroundjoin%
\definecolor{currentfill}{rgb}{0.667299,1.000000,0.300443}%
\pgfsetfillcolor{currentfill}%
\pgfsetlinewidth{0.000000pt}%
\definecolor{currentstroke}{rgb}{0.000000,0.000000,0.000000}%
\pgfsetstrokecolor{currentstroke}%
\pgfsetdash{}{0pt}%
\pgfpathmoveto{\pgfqpoint{1.211142in}{1.493821in}}%
\pgfpathlineto{\pgfqpoint{1.196349in}{1.513705in}}%
\pgfpathlineto{\pgfqpoint{1.234718in}{1.488016in}}%
\pgfpathlineto{\pgfqpoint{1.247897in}{1.469187in}}%
\pgfpathlineto{\pgfqpoint{1.211142in}{1.493821in}}%
\pgfpathclose%
\pgfusepath{fill}%
\end{pgfscope}%
\begin{pgfscope}%
\pgfpathrectangle{\pgfqpoint{0.000000in}{0.000000in}}{\pgfqpoint{3.000000in}{3.000000in}}%
\pgfusepath{clip}%
\pgfsetbuttcap%
\pgfsetroundjoin%
\definecolor{currentfill}{rgb}{0.300443,1.000000,0.667299}%
\pgfsetfillcolor{currentfill}%
\pgfsetlinewidth{0.000000pt}%
\definecolor{currentstroke}{rgb}{0.000000,0.000000,0.000000}%
\pgfsetstrokecolor{currentstroke}%
\pgfsetdash{}{0pt}%
\pgfpathmoveto{\pgfqpoint{1.424181in}{1.336165in}}%
\pgfpathlineto{\pgfqpoint{1.417817in}{1.358739in}}%
\pgfpathlineto{\pgfqpoint{1.469283in}{1.350953in}}%
\pgfpathlineto{\pgfqpoint{1.472981in}{1.328766in}}%
\pgfpathlineto{\pgfqpoint{1.424181in}{1.336165in}}%
\pgfpathclose%
\pgfusepath{fill}%
\end{pgfscope}%
\begin{pgfscope}%
\pgfpathrectangle{\pgfqpoint{0.000000in}{0.000000in}}{\pgfqpoint{3.000000in}{3.000000in}}%
\pgfusepath{clip}%
\pgfsetbuttcap%
\pgfsetroundjoin%
\definecolor{currentfill}{rgb}{0.927807,0.015251,0.000000}%
\pgfsetfillcolor{currentfill}%
\pgfsetlinewidth{0.000000pt}%
\definecolor{currentstroke}{rgb}{0.000000,0.000000,0.000000}%
\pgfsetstrokecolor{currentstroke}%
\pgfsetdash{}{0pt}%
\pgfpathmoveto{\pgfqpoint{0.825081in}{1.965800in}}%
\pgfpathlineto{\pgfqpoint{0.807714in}{1.980510in}}%
\pgfpathlineto{\pgfqpoint{0.806214in}{1.925319in}}%
\pgfpathlineto{\pgfqpoint{0.823722in}{1.911837in}}%
\pgfpathlineto{\pgfqpoint{0.825081in}{1.965800in}}%
\pgfpathclose%
\pgfusepath{fill}%
\end{pgfscope}%
\begin{pgfscope}%
\pgfpathrectangle{\pgfqpoint{0.000000in}{0.000000in}}{\pgfqpoint{3.000000in}{3.000000in}}%
\pgfusepath{clip}%
\pgfsetbuttcap%
\pgfsetroundjoin%
\definecolor{currentfill}{rgb}{1.000000,0.741467,0.000000}%
\pgfsetfillcolor{currentfill}%
\pgfsetlinewidth{0.000000pt}%
\definecolor{currentstroke}{rgb}{0.000000,0.000000,0.000000}%
\pgfsetstrokecolor{currentstroke}%
\pgfsetdash{}{0pt}%
\pgfpathmoveto{\pgfqpoint{2.020631in}{1.661310in}}%
\pgfpathlineto{\pgfqpoint{2.037128in}{1.677111in}}%
\pgfpathlineto{\pgfqpoint{2.058620in}{1.716374in}}%
\pgfpathlineto{\pgfqpoint{2.041480in}{1.699339in}}%
\pgfpathlineto{\pgfqpoint{2.020631in}{1.661310in}}%
\pgfpathclose%
\pgfusepath{fill}%
\end{pgfscope}%
\begin{pgfscope}%
\pgfpathrectangle{\pgfqpoint{0.000000in}{0.000000in}}{\pgfqpoint{3.000000in}{3.000000in}}%
\pgfusepath{clip}%
\pgfsetbuttcap%
\pgfsetroundjoin%
\definecolor{currentfill}{rgb}{0.578748,1.000000,0.388994}%
\pgfsetfillcolor{currentfill}%
\pgfsetlinewidth{0.000000pt}%
\definecolor{currentstroke}{rgb}{0.000000,0.000000,0.000000}%
\pgfsetstrokecolor{currentstroke}%
\pgfsetdash{}{0pt}%
\pgfpathmoveto{\pgfqpoint{1.792570in}{1.435060in}}%
\pgfpathlineto{\pgfqpoint{1.804449in}{1.454480in}}%
\pgfpathlineto{\pgfqpoint{1.846317in}{1.477078in}}%
\pgfpathlineto{\pgfqpoint{1.832585in}{1.456688in}}%
\pgfpathlineto{\pgfqpoint{1.792570in}{1.435060in}}%
\pgfpathclose%
\pgfusepath{fill}%
\end{pgfscope}%
\begin{pgfscope}%
\pgfpathrectangle{\pgfqpoint{0.000000in}{0.000000in}}{\pgfqpoint{3.000000in}{3.000000in}}%
\pgfusepath{clip}%
\pgfsetbuttcap%
\pgfsetroundjoin%
\definecolor{currentfill}{rgb}{1.000000,0.349310,0.000000}%
\pgfsetfillcolor{currentfill}%
\pgfsetlinewidth{0.000000pt}%
\definecolor{currentstroke}{rgb}{0.000000,0.000000,0.000000}%
\pgfsetstrokecolor{currentstroke}%
\pgfsetdash{}{0pt}%
\pgfpathmoveto{\pgfqpoint{2.144599in}{1.794286in}}%
\pgfpathlineto{\pgfqpoint{2.161850in}{1.808662in}}%
\pgfpathlineto{\pgfqpoint{2.171529in}{1.856405in}}%
\pgfpathlineto{\pgfqpoint{2.154099in}{1.840784in}}%
\pgfpathlineto{\pgfqpoint{2.144599in}{1.794286in}}%
\pgfpathclose%
\pgfusepath{fill}%
\end{pgfscope}%
\begin{pgfscope}%
\pgfpathrectangle{\pgfqpoint{0.000000in}{0.000000in}}{\pgfqpoint{3.000000in}{3.000000in}}%
\pgfusepath{clip}%
\pgfsetbuttcap%
\pgfsetroundjoin%
\definecolor{currentfill}{rgb}{0.606952,0.000000,0.000000}%
\pgfsetfillcolor{currentfill}%
\pgfsetlinewidth{0.000000pt}%
\definecolor{currentstroke}{rgb}{0.000000,0.000000,0.000000}%
\pgfsetstrokecolor{currentstroke}%
\pgfsetdash{}{0pt}%
\pgfpathmoveto{\pgfqpoint{2.346547in}{1.997171in}}%
\pgfpathlineto{\pgfqpoint{2.364119in}{2.010040in}}%
\pgfpathlineto{\pgfqpoint{2.355103in}{2.071059in}}%
\pgfpathlineto{\pgfqpoint{2.337840in}{2.056982in}}%
\pgfpathlineto{\pgfqpoint{2.346547in}{1.997171in}}%
\pgfpathclose%
\pgfusepath{fill}%
\end{pgfscope}%
\begin{pgfscope}%
\pgfpathrectangle{\pgfqpoint{0.000000in}{0.000000in}}{\pgfqpoint{3.000000in}{3.000000in}}%
\pgfusepath{clip}%
\pgfsetbuttcap%
\pgfsetroundjoin%
\definecolor{currentfill}{rgb}{0.743201,1.000000,0.224541}%
\pgfsetfillcolor{currentfill}%
\pgfsetlinewidth{0.000000pt}%
\definecolor{currentstroke}{rgb}{0.000000,0.000000,0.000000}%
\pgfsetstrokecolor{currentstroke}%
\pgfsetdash{}{0pt}%
\pgfpathmoveto{\pgfqpoint{1.860076in}{1.496246in}}%
\pgfpathlineto{\pgfqpoint{1.873861in}{1.514329in}}%
\pgfpathlineto{\pgfqpoint{1.910884in}{1.542058in}}%
\pgfpathlineto{\pgfqpoint{1.895612in}{1.522882in}}%
\pgfpathlineto{\pgfqpoint{1.860076in}{1.496246in}}%
\pgfpathclose%
\pgfusepath{fill}%
\end{pgfscope}%
\begin{pgfscope}%
\pgfpathrectangle{\pgfqpoint{0.000000in}{0.000000in}}{\pgfqpoint{3.000000in}{3.000000in}}%
\pgfusepath{clip}%
\pgfsetbuttcap%
\pgfsetroundjoin%
\definecolor{currentfill}{rgb}{0.300443,1.000000,0.667299}%
\pgfsetfillcolor{currentfill}%
\pgfsetlinewidth{0.000000pt}%
\definecolor{currentstroke}{rgb}{0.000000,0.000000,0.000000}%
\pgfsetstrokecolor{currentstroke}%
\pgfsetdash{}{0pt}%
\pgfpathmoveto{\pgfqpoint{1.574516in}{1.326045in}}%
\pgfpathlineto{\pgfqpoint{1.576376in}{1.348090in}}%
\pgfpathlineto{\pgfqpoint{1.629226in}{1.353090in}}%
\pgfpathlineto{\pgfqpoint{1.624624in}{1.330796in}}%
\pgfpathlineto{\pgfqpoint{1.574516in}{1.326045in}}%
\pgfpathclose%
\pgfusepath{fill}%
\end{pgfscope}%
\begin{pgfscope}%
\pgfpathrectangle{\pgfqpoint{0.000000in}{0.000000in}}{\pgfqpoint{3.000000in}{3.000000in}}%
\pgfusepath{clip}%
\pgfsetbuttcap%
\pgfsetroundjoin%
\definecolor{currentfill}{rgb}{1.000000,0.480029,0.000000}%
\pgfsetfillcolor{currentfill}%
\pgfsetlinewidth{0.000000pt}%
\definecolor{currentstroke}{rgb}{0.000000,0.000000,0.000000}%
\pgfsetstrokecolor{currentstroke}%
\pgfsetdash{}{0pt}%
\pgfpathmoveto{\pgfqpoint{0.963250in}{1.793710in}}%
\pgfpathlineto{\pgfqpoint{0.945861in}{1.809669in}}%
\pgfpathlineto{\pgfqpoint{0.960100in}{1.764644in}}%
\pgfpathlineto{\pgfqpoint{0.977154in}{1.749931in}}%
\pgfpathlineto{\pgfqpoint{0.963250in}{1.793710in}}%
\pgfpathclose%
\pgfusepath{fill}%
\end{pgfscope}%
\begin{pgfscope}%
\pgfpathrectangle{\pgfqpoint{0.000000in}{0.000000in}}{\pgfqpoint{3.000000in}{3.000000in}}%
\pgfusepath{clip}%
\pgfsetbuttcap%
\pgfsetroundjoin%
\definecolor{currentfill}{rgb}{0.401645,1.000000,0.566097}%
\pgfsetfillcolor{currentfill}%
\pgfsetlinewidth{0.000000pt}%
\definecolor{currentstroke}{rgb}{0.000000,0.000000,0.000000}%
\pgfsetstrokecolor{currentstroke}%
\pgfsetdash{}{0pt}%
\pgfpathmoveto{\pgfqpoint{1.679777in}{1.362235in}}%
\pgfpathlineto{\pgfqpoint{1.687012in}{1.383158in}}%
\pgfpathlineto{\pgfqpoint{1.736391in}{1.396847in}}%
\pgfpathlineto{\pgfqpoint{1.726729in}{1.375277in}}%
\pgfpathlineto{\pgfqpoint{1.679777in}{1.362235in}}%
\pgfpathclose%
\pgfusepath{fill}%
\end{pgfscope}%
\begin{pgfscope}%
\pgfpathrectangle{\pgfqpoint{0.000000in}{0.000000in}}{\pgfqpoint{3.000000in}{3.000000in}}%
\pgfusepath{clip}%
\pgfsetbuttcap%
\pgfsetroundjoin%
\definecolor{currentfill}{rgb}{1.000000,0.814089,0.000000}%
\pgfsetfillcolor{currentfill}%
\pgfsetlinewidth{0.000000pt}%
\definecolor{currentstroke}{rgb}{0.000000,0.000000,0.000000}%
\pgfsetstrokecolor{currentstroke}%
\pgfsetdash{}{0pt}%
\pgfpathmoveto{\pgfqpoint{1.062125in}{1.669299in}}%
\pgfpathlineto{\pgfqpoint{1.045171in}{1.686507in}}%
\pgfpathlineto{\pgfqpoint{1.070128in}{1.649017in}}%
\pgfpathlineto{\pgfqpoint{1.086291in}{1.633030in}}%
\pgfpathlineto{\pgfqpoint{1.062125in}{1.669299in}}%
\pgfpathclose%
\pgfusepath{fill}%
\end{pgfscope}%
\begin{pgfscope}%
\pgfpathrectangle{\pgfqpoint{0.000000in}{0.000000in}}{\pgfqpoint{3.000000in}{3.000000in}}%
\pgfusepath{clip}%
\pgfsetbuttcap%
\pgfsetroundjoin%
\definecolor{currentfill}{rgb}{0.490196,1.000000,0.477546}%
\pgfsetfillcolor{currentfill}%
\pgfsetlinewidth{0.000000pt}%
\definecolor{currentstroke}{rgb}{0.000000,0.000000,0.000000}%
\pgfsetstrokecolor{currentstroke}%
\pgfsetdash{}{0pt}%
\pgfpathmoveto{\pgfqpoint{1.736391in}{1.396847in}}%
\pgfpathlineto{\pgfqpoint{1.746075in}{1.416832in}}%
\pgfpathlineto{\pgfqpoint{1.792570in}{1.435060in}}%
\pgfpathlineto{\pgfqpoint{1.780717in}{1.414254in}}%
\pgfpathlineto{\pgfqpoint{1.736391in}{1.396847in}}%
\pgfpathclose%
\pgfusepath{fill}%
\end{pgfscope}%
\begin{pgfscope}%
\pgfpathrectangle{\pgfqpoint{0.000000in}{0.000000in}}{\pgfqpoint{3.000000in}{3.000000in}}%
\pgfusepath{clip}%
\pgfsetbuttcap%
\pgfsetroundjoin%
\definecolor{currentfill}{rgb}{0.300443,1.000000,0.667299}%
\pgfsetfillcolor{currentfill}%
\pgfsetlinewidth{0.000000pt}%
\definecolor{currentstroke}{rgb}{0.000000,0.000000,0.000000}%
\pgfsetstrokecolor{currentstroke}%
\pgfsetdash{}{0pt}%
\pgfpathmoveto{\pgfqpoint{1.472981in}{1.328766in}}%
\pgfpathlineto{\pgfqpoint{1.469283in}{1.350953in}}%
\pgfpathlineto{\pgfqpoint{1.522597in}{1.347371in}}%
\pgfpathlineto{\pgfqpoint{1.523528in}{1.325362in}}%
\pgfpathlineto{\pgfqpoint{1.472981in}{1.328766in}}%
\pgfpathclose%
\pgfusepath{fill}%
\end{pgfscope}%
\begin{pgfscope}%
\pgfpathrectangle{\pgfqpoint{0.000000in}{0.000000in}}{\pgfqpoint{3.000000in}{3.000000in}}%
\pgfusepath{clip}%
\pgfsetbuttcap%
\pgfsetroundjoin%
\definecolor{currentfill}{rgb}{0.856506,0.000000,0.000000}%
\pgfsetfillcolor{currentfill}%
\pgfsetlinewidth{0.000000pt}%
\definecolor{currentstroke}{rgb}{0.000000,0.000000,0.000000}%
\pgfsetstrokecolor{currentstroke}%
\pgfsetdash{}{0pt}%
\pgfpathmoveto{\pgfqpoint{0.807714in}{1.980510in}}%
\pgfpathlineto{\pgfqpoint{0.790338in}{1.994990in}}%
\pgfpathlineto{\pgfqpoint{0.788693in}{1.938574in}}%
\pgfpathlineto{\pgfqpoint{0.806214in}{1.925319in}}%
\pgfpathlineto{\pgfqpoint{0.807714in}{1.980510in}}%
\pgfpathclose%
\pgfusepath{fill}%
\end{pgfscope}%
\begin{pgfscope}%
\pgfpathrectangle{\pgfqpoint{0.000000in}{0.000000in}}{\pgfqpoint{3.000000in}{3.000000in}}%
\pgfusepath{clip}%
\pgfsetbuttcap%
\pgfsetroundjoin%
\definecolor{currentfill}{rgb}{0.300443,1.000000,0.667299}%
\pgfsetfillcolor{currentfill}%
\pgfsetlinewidth{0.000000pt}%
\definecolor{currentstroke}{rgb}{0.000000,0.000000,0.000000}%
\pgfsetstrokecolor{currentstroke}%
\pgfsetdash{}{0pt}%
\pgfpathmoveto{\pgfqpoint{1.523528in}{1.325362in}}%
\pgfpathlineto{\pgfqpoint{1.522597in}{1.347371in}}%
\pgfpathlineto{\pgfqpoint{1.576376in}{1.348090in}}%
\pgfpathlineto{\pgfqpoint{1.574516in}{1.326045in}}%
\pgfpathlineto{\pgfqpoint{1.523528in}{1.325362in}}%
\pgfpathclose%
\pgfusepath{fill}%
\end{pgfscope}%
\begin{pgfscope}%
\pgfpathrectangle{\pgfqpoint{0.000000in}{0.000000in}}{\pgfqpoint{3.000000in}{3.000000in}}%
\pgfusepath{clip}%
\pgfsetbuttcap%
\pgfsetroundjoin%
\definecolor{currentfill}{rgb}{0.553476,0.000000,0.000000}%
\pgfsetfillcolor{currentfill}%
\pgfsetlinewidth{0.000000pt}%
\definecolor{currentstroke}{rgb}{0.000000,0.000000,0.000000}%
\pgfsetstrokecolor{currentstroke}%
\pgfsetdash{}{0pt}%
\pgfpathmoveto{\pgfqpoint{2.364119in}{2.010040in}}%
\pgfpathlineto{\pgfqpoint{2.381703in}{2.022737in}}%
\pgfpathlineto{\pgfqpoint{2.372374in}{2.084960in}}%
\pgfpathlineto{\pgfqpoint{2.355103in}{2.071059in}}%
\pgfpathlineto{\pgfqpoint{2.364119in}{2.010040in}}%
\pgfpathclose%
\pgfusepath{fill}%
\end{pgfscope}%
\begin{pgfscope}%
\pgfpathrectangle{\pgfqpoint{0.000000in}{0.000000in}}{\pgfqpoint{3.000000in}{3.000000in}}%
\pgfusepath{clip}%
\pgfsetbuttcap%
\pgfsetroundjoin%
\definecolor{currentfill}{rgb}{0.578748,1.000000,0.388994}%
\pgfsetfillcolor{currentfill}%
\pgfsetlinewidth{0.000000pt}%
\definecolor{currentstroke}{rgb}{0.000000,0.000000,0.000000}%
\pgfsetstrokecolor{currentstroke}%
\pgfsetdash{}{0pt}%
\pgfpathmoveto{\pgfqpoint{1.261050in}{1.449135in}}%
\pgfpathlineto{\pgfqpoint{1.247897in}{1.469187in}}%
\pgfpathlineto{\pgfqpoint{1.292156in}{1.447708in}}%
\pgfpathlineto{\pgfqpoint{1.303344in}{1.428580in}}%
\pgfpathlineto{\pgfqpoint{1.261050in}{1.449135in}}%
\pgfpathclose%
\pgfusepath{fill}%
\end{pgfscope}%
\begin{pgfscope}%
\pgfpathrectangle{\pgfqpoint{0.000000in}{0.000000in}}{\pgfqpoint{3.000000in}{3.000000in}}%
\pgfusepath{clip}%
\pgfsetbuttcap%
\pgfsetroundjoin%
\definecolor{currentfill}{rgb}{0.895003,1.000000,0.072739}%
\pgfsetfillcolor{currentfill}%
\pgfsetlinewidth{0.000000pt}%
\definecolor{currentstroke}{rgb}{0.000000,0.000000,0.000000}%
\pgfsetstrokecolor{currentstroke}%
\pgfsetdash{}{0pt}%
\pgfpathmoveto{\pgfqpoint{1.134638in}{1.580941in}}%
\pgfpathlineto{\pgfqpoint{1.118546in}{1.599062in}}%
\pgfpathlineto{\pgfqpoint{1.151811in}{1.567302in}}%
\pgfpathlineto{\pgfqpoint{1.166683in}{1.550337in}}%
\pgfpathlineto{\pgfqpoint{1.134638in}{1.580941in}}%
\pgfpathclose%
\pgfusepath{fill}%
\end{pgfscope}%
\begin{pgfscope}%
\pgfpathrectangle{\pgfqpoint{0.000000in}{0.000000in}}{\pgfqpoint{3.000000in}{3.000000in}}%
\pgfusepath{clip}%
\pgfsetbuttcap%
\pgfsetroundjoin%
\definecolor{currentfill}{rgb}{0.958254,0.973856,0.009488}%
\pgfsetfillcolor{currentfill}%
\pgfsetlinewidth{0.000000pt}%
\definecolor{currentstroke}{rgb}{0.000000,0.000000,0.000000}%
\pgfsetstrokecolor{currentstroke}%
\pgfsetdash{}{0pt}%
\pgfpathmoveto{\pgfqpoint{1.941506in}{1.577606in}}%
\pgfpathlineto{\pgfqpoint{1.956856in}{1.594165in}}%
\pgfpathlineto{\pgfqpoint{1.987705in}{1.627890in}}%
\pgfpathlineto{\pgfqpoint{1.971275in}{1.610150in}}%
\pgfpathlineto{\pgfqpoint{1.941506in}{1.577606in}}%
\pgfpathclose%
\pgfusepath{fill}%
\end{pgfscope}%
\begin{pgfscope}%
\pgfpathrectangle{\pgfqpoint{0.000000in}{0.000000in}}{\pgfqpoint{3.000000in}{3.000000in}}%
\pgfusepath{clip}%
\pgfsetbuttcap%
\pgfsetroundjoin%
\definecolor{currentfill}{rgb}{0.401645,1.000000,0.566097}%
\pgfsetfillcolor{currentfill}%
\pgfsetlinewidth{0.000000pt}%
\definecolor{currentstroke}{rgb}{0.000000,0.000000,0.000000}%
\pgfsetstrokecolor{currentstroke}%
\pgfsetdash{}{0pt}%
\pgfpathmoveto{\pgfqpoint{1.369527in}{1.370516in}}%
\pgfpathlineto{\pgfqpoint{1.360648in}{1.391850in}}%
\pgfpathlineto{\pgfqpoint{1.411437in}{1.379489in}}%
\pgfpathlineto{\pgfqpoint{1.417817in}{1.358739in}}%
\pgfpathlineto{\pgfqpoint{1.369527in}{1.370516in}}%
\pgfpathclose%
\pgfusepath{fill}%
\end{pgfscope}%
\begin{pgfscope}%
\pgfpathrectangle{\pgfqpoint{0.000000in}{0.000000in}}{\pgfqpoint{3.000000in}{3.000000in}}%
\pgfusepath{clip}%
\pgfsetbuttcap%
\pgfsetroundjoin%
\definecolor{currentfill}{rgb}{1.000000,0.291213,0.000000}%
\pgfsetfillcolor{currentfill}%
\pgfsetlinewidth{0.000000pt}%
\definecolor{currentstroke}{rgb}{0.000000,0.000000,0.000000}%
\pgfsetstrokecolor{currentstroke}%
\pgfsetdash{}{0pt}%
\pgfpathmoveto{\pgfqpoint{2.161850in}{1.808662in}}%
\pgfpathlineto{\pgfqpoint{2.179118in}{1.822701in}}%
\pgfpathlineto{\pgfqpoint{2.188973in}{1.871688in}}%
\pgfpathlineto{\pgfqpoint{2.171529in}{1.856405in}}%
\pgfpathlineto{\pgfqpoint{2.161850in}{1.808662in}}%
\pgfpathclose%
\pgfusepath{fill}%
\end{pgfscope}%
\begin{pgfscope}%
\pgfpathrectangle{\pgfqpoint{0.000000in}{0.000000in}}{\pgfqpoint{3.000000in}{3.000000in}}%
\pgfusepath{clip}%
\pgfsetbuttcap%
\pgfsetroundjoin%
\definecolor{currentfill}{rgb}{1.000000,0.668845,0.000000}%
\pgfsetfillcolor{currentfill}%
\pgfsetlinewidth{0.000000pt}%
\definecolor{currentstroke}{rgb}{0.000000,0.000000,0.000000}%
\pgfsetstrokecolor{currentstroke}%
\pgfsetdash{}{0pt}%
\pgfpathmoveto{\pgfqpoint{2.037128in}{1.677111in}}%
\pgfpathlineto{\pgfqpoint{2.053648in}{1.692374in}}%
\pgfpathlineto{\pgfqpoint{2.075779in}{1.732871in}}%
\pgfpathlineto{\pgfqpoint{2.058620in}{1.716374in}}%
\pgfpathlineto{\pgfqpoint{2.037128in}{1.677111in}}%
\pgfpathclose%
\pgfusepath{fill}%
\end{pgfscope}%
\begin{pgfscope}%
\pgfpathrectangle{\pgfqpoint{0.000000in}{0.000000in}}{\pgfqpoint{3.000000in}{3.000000in}}%
\pgfusepath{clip}%
\pgfsetbuttcap%
\pgfsetroundjoin%
\definecolor{currentfill}{rgb}{0.803030,0.000000,0.000000}%
\pgfsetfillcolor{currentfill}%
\pgfsetlinewidth{0.000000pt}%
\definecolor{currentstroke}{rgb}{0.000000,0.000000,0.000000}%
\pgfsetstrokecolor{currentstroke}%
\pgfsetdash{}{0pt}%
\pgfpathmoveto{\pgfqpoint{0.790338in}{1.994990in}}%
\pgfpathlineto{\pgfqpoint{0.772953in}{2.009252in}}%
\pgfpathlineto{\pgfqpoint{0.771156in}{1.951612in}}%
\pgfpathlineto{\pgfqpoint{0.788693in}{1.938574in}}%
\pgfpathlineto{\pgfqpoint{0.790338in}{1.994990in}}%
\pgfpathclose%
\pgfusepath{fill}%
\end{pgfscope}%
\begin{pgfscope}%
\pgfpathrectangle{\pgfqpoint{0.000000in}{0.000000in}}{\pgfqpoint{3.000000in}{3.000000in}}%
\pgfusepath{clip}%
\pgfsetbuttcap%
\pgfsetroundjoin%
\definecolor{currentfill}{rgb}{0.490196,1.000000,0.477546}%
\pgfsetfillcolor{currentfill}%
\pgfsetlinewidth{0.000000pt}%
\definecolor{currentstroke}{rgb}{0.000000,0.000000,0.000000}%
\pgfsetstrokecolor{currentstroke}%
\pgfsetdash{}{0pt}%
\pgfpathmoveto{\pgfqpoint{1.314506in}{1.408065in}}%
\pgfpathlineto{\pgfqpoint{1.303344in}{1.428580in}}%
\pgfpathlineto{\pgfqpoint{1.351748in}{1.411600in}}%
\pgfpathlineto{\pgfqpoint{1.360648in}{1.391850in}}%
\pgfpathlineto{\pgfqpoint{1.314506in}{1.408065in}}%
\pgfpathclose%
\pgfusepath{fill}%
\end{pgfscope}%
\begin{pgfscope}%
\pgfpathrectangle{\pgfqpoint{0.000000in}{0.000000in}}{\pgfqpoint{3.000000in}{3.000000in}}%
\pgfusepath{clip}%
\pgfsetbuttcap%
\pgfsetroundjoin%
\definecolor{currentfill}{rgb}{1.000000,0.407407,0.000000}%
\pgfsetfillcolor{currentfill}%
\pgfsetlinewidth{0.000000pt}%
\definecolor{currentstroke}{rgb}{0.000000,0.000000,0.000000}%
\pgfsetstrokecolor{currentstroke}%
\pgfsetdash{}{0pt}%
\pgfpathmoveto{\pgfqpoint{0.945861in}{1.809669in}}%
\pgfpathlineto{\pgfqpoint{0.928457in}{1.825238in}}%
\pgfpathlineto{\pgfqpoint{0.943027in}{1.778967in}}%
\pgfpathlineto{\pgfqpoint{0.960100in}{1.764644in}}%
\pgfpathlineto{\pgfqpoint{0.945861in}{1.809669in}}%
\pgfpathclose%
\pgfusepath{fill}%
\end{pgfscope}%
\begin{pgfscope}%
\pgfpathrectangle{\pgfqpoint{0.000000in}{0.000000in}}{\pgfqpoint{3.000000in}{3.000000in}}%
\pgfusepath{clip}%
\pgfsetbuttcap%
\pgfsetroundjoin%
\definecolor{currentfill}{rgb}{0.500000,0.000000,0.000000}%
\pgfsetfillcolor{currentfill}%
\pgfsetlinewidth{0.000000pt}%
\definecolor{currentstroke}{rgb}{0.000000,0.000000,0.000000}%
\pgfsetstrokecolor{currentstroke}%
\pgfsetdash{}{0pt}%
\pgfpathmoveto{\pgfqpoint{2.381703in}{2.022737in}}%
\pgfpathlineto{\pgfqpoint{2.399301in}{2.035269in}}%
\pgfpathlineto{\pgfqpoint{2.389651in}{2.098695in}}%
\pgfpathlineto{\pgfqpoint{2.372374in}{2.084960in}}%
\pgfpathlineto{\pgfqpoint{2.381703in}{2.022737in}}%
\pgfpathclose%
\pgfusepath{fill}%
\end{pgfscope}%
\begin{pgfscope}%
\pgfpathrectangle{\pgfqpoint{0.000000in}{0.000000in}}{\pgfqpoint{3.000000in}{3.000000in}}%
\pgfusepath{clip}%
\pgfsetbuttcap%
\pgfsetroundjoin%
\definecolor{currentfill}{rgb}{0.743201,1.000000,0.224541}%
\pgfsetfillcolor{currentfill}%
\pgfsetlinewidth{0.000000pt}%
\definecolor{currentstroke}{rgb}{0.000000,0.000000,0.000000}%
\pgfsetstrokecolor{currentstroke}%
\pgfsetdash{}{0pt}%
\pgfpathmoveto{\pgfqpoint{1.196349in}{1.513705in}}%
\pgfpathlineto{\pgfqpoint{1.181529in}{1.532505in}}%
\pgfpathlineto{\pgfqpoint{1.221513in}{1.505760in}}%
\pgfpathlineto{\pgfqpoint{1.234718in}{1.488016in}}%
\pgfpathlineto{\pgfqpoint{1.196349in}{1.513705in}}%
\pgfpathclose%
\pgfusepath{fill}%
\end{pgfscope}%
\begin{pgfscope}%
\pgfpathrectangle{\pgfqpoint{0.000000in}{0.000000in}}{\pgfqpoint{3.000000in}{3.000000in}}%
\pgfusepath{clip}%
\pgfsetbuttcap%
\pgfsetroundjoin%
\definecolor{currentfill}{rgb}{1.000000,0.741467,0.000000}%
\pgfsetfillcolor{currentfill}%
\pgfsetlinewidth{0.000000pt}%
\definecolor{currentstroke}{rgb}{0.000000,0.000000,0.000000}%
\pgfsetstrokecolor{currentstroke}%
\pgfsetdash{}{0pt}%
\pgfpathmoveto{\pgfqpoint{1.045171in}{1.686507in}}%
\pgfpathlineto{\pgfqpoint{1.028197in}{1.703127in}}%
\pgfpathlineto{\pgfqpoint{1.053941in}{1.664417in}}%
\pgfpathlineto{\pgfqpoint{1.070128in}{1.649017in}}%
\pgfpathlineto{\pgfqpoint{1.045171in}{1.686507in}}%
\pgfpathclose%
\pgfusepath{fill}%
\end{pgfscope}%
\begin{pgfscope}%
\pgfpathrectangle{\pgfqpoint{0.000000in}{0.000000in}}{\pgfqpoint{3.000000in}{3.000000in}}%
\pgfusepath{clip}%
\pgfsetbuttcap%
\pgfsetroundjoin%
\definecolor{currentfill}{rgb}{1.000000,0.233115,0.000000}%
\pgfsetfillcolor{currentfill}%
\pgfsetlinewidth{0.000000pt}%
\definecolor{currentstroke}{rgb}{0.000000,0.000000,0.000000}%
\pgfsetstrokecolor{currentstroke}%
\pgfsetdash{}{0pt}%
\pgfpathmoveto{\pgfqpoint{2.179118in}{1.822701in}}%
\pgfpathlineto{\pgfqpoint{2.196405in}{1.836426in}}%
\pgfpathlineto{\pgfqpoint{2.206429in}{1.886656in}}%
\pgfpathlineto{\pgfqpoint{2.188973in}{1.871688in}}%
\pgfpathlineto{\pgfqpoint{2.179118in}{1.822701in}}%
\pgfpathclose%
\pgfusepath{fill}%
\end{pgfscope}%
\begin{pgfscope}%
\pgfpathrectangle{\pgfqpoint{0.000000in}{0.000000in}}{\pgfqpoint{3.000000in}{3.000000in}}%
\pgfusepath{clip}%
\pgfsetbuttcap%
\pgfsetroundjoin%
\definecolor{currentfill}{rgb}{0.667299,1.000000,0.300443}%
\pgfsetfillcolor{currentfill}%
\pgfsetlinewidth{0.000000pt}%
\definecolor{currentstroke}{rgb}{0.000000,0.000000,0.000000}%
\pgfsetstrokecolor{currentstroke}%
\pgfsetdash{}{0pt}%
\pgfpathmoveto{\pgfqpoint{1.804449in}{1.454480in}}%
\pgfpathlineto{\pgfqpoint{1.816353in}{1.472676in}}%
\pgfpathlineto{\pgfqpoint{1.860076in}{1.496246in}}%
\pgfpathlineto{\pgfqpoint{1.846317in}{1.477078in}}%
\pgfpathlineto{\pgfqpoint{1.804449in}{1.454480in}}%
\pgfpathclose%
\pgfusepath{fill}%
\end{pgfscope}%
\begin{pgfscope}%
\pgfpathrectangle{\pgfqpoint{0.000000in}{0.000000in}}{\pgfqpoint{3.000000in}{3.000000in}}%
\pgfusepath{clip}%
\pgfsetbuttcap%
\pgfsetroundjoin%
\definecolor{currentfill}{rgb}{0.731729,0.000000,0.000000}%
\pgfsetfillcolor{currentfill}%
\pgfsetlinewidth{0.000000pt}%
\definecolor{currentstroke}{rgb}{0.000000,0.000000,0.000000}%
\pgfsetstrokecolor{currentstroke}%
\pgfsetdash{}{0pt}%
\pgfpathmoveto{\pgfqpoint{0.772953in}{2.009252in}}%
\pgfpathlineto{\pgfqpoint{0.755558in}{2.023308in}}%
\pgfpathlineto{\pgfqpoint{0.753605in}{1.964448in}}%
\pgfpathlineto{\pgfqpoint{0.771156in}{1.951612in}}%
\pgfpathlineto{\pgfqpoint{0.772953in}{2.009252in}}%
\pgfpathclose%
\pgfusepath{fill}%
\end{pgfscope}%
\begin{pgfscope}%
\pgfpathrectangle{\pgfqpoint{0.000000in}{0.000000in}}{\pgfqpoint{3.000000in}{3.000000in}}%
\pgfusepath{clip}%
\pgfsetbuttcap%
\pgfsetroundjoin%
\definecolor{currentfill}{rgb}{0.401645,1.000000,0.566097}%
\pgfsetfillcolor{currentfill}%
\pgfsetlinewidth{0.000000pt}%
\definecolor{currentstroke}{rgb}{0.000000,0.000000,0.000000}%
\pgfsetstrokecolor{currentstroke}%
\pgfsetdash{}{0pt}%
\pgfpathmoveto{\pgfqpoint{1.629226in}{1.353090in}}%
\pgfpathlineto{\pgfqpoint{1.633839in}{1.373558in}}%
\pgfpathlineto{\pgfqpoint{1.687012in}{1.383158in}}%
\pgfpathlineto{\pgfqpoint{1.679777in}{1.362235in}}%
\pgfpathlineto{\pgfqpoint{1.629226in}{1.353090in}}%
\pgfpathclose%
\pgfusepath{fill}%
\end{pgfscope}%
\begin{pgfscope}%
\pgfpathrectangle{\pgfqpoint{0.000000in}{0.000000in}}{\pgfqpoint{3.000000in}{3.000000in}}%
\pgfusepath{clip}%
\pgfsetbuttcap%
\pgfsetroundjoin%
\definecolor{currentfill}{rgb}{0.819102,1.000000,0.148640}%
\pgfsetfillcolor{currentfill}%
\pgfsetlinewidth{0.000000pt}%
\definecolor{currentstroke}{rgb}{0.000000,0.000000,0.000000}%
\pgfsetstrokecolor{currentstroke}%
\pgfsetdash{}{0pt}%
\pgfpathmoveto{\pgfqpoint{1.873861in}{1.514329in}}%
\pgfpathlineto{\pgfqpoint{1.887672in}{1.531445in}}%
\pgfpathlineto{\pgfqpoint{1.926182in}{1.560266in}}%
\pgfpathlineto{\pgfqpoint{1.910884in}{1.542058in}}%
\pgfpathlineto{\pgfqpoint{1.873861in}{1.514329in}}%
\pgfpathclose%
\pgfusepath{fill}%
\end{pgfscope}%
\begin{pgfscope}%
\pgfpathrectangle{\pgfqpoint{0.000000in}{0.000000in}}{\pgfqpoint{3.000000in}{3.000000in}}%
\pgfusepath{clip}%
\pgfsetbuttcap%
\pgfsetroundjoin%
\definecolor{currentfill}{rgb}{1.000000,0.610748,0.000000}%
\pgfsetfillcolor{currentfill}%
\pgfsetlinewidth{0.000000pt}%
\definecolor{currentstroke}{rgb}{0.000000,0.000000,0.000000}%
\pgfsetstrokecolor{currentstroke}%
\pgfsetdash{}{0pt}%
\pgfpathmoveto{\pgfqpoint{2.053648in}{1.692374in}}%
\pgfpathlineto{\pgfqpoint{2.070190in}{1.707145in}}%
\pgfpathlineto{\pgfqpoint{2.092956in}{1.748874in}}%
\pgfpathlineto{\pgfqpoint{2.075779in}{1.732871in}}%
\pgfpathlineto{\pgfqpoint{2.053648in}{1.692374in}}%
\pgfpathclose%
\pgfusepath{fill}%
\end{pgfscope}%
\begin{pgfscope}%
\pgfpathrectangle{\pgfqpoint{0.000000in}{0.000000in}}{\pgfqpoint{3.000000in}{3.000000in}}%
\pgfusepath{clip}%
\pgfsetbuttcap%
\pgfsetroundjoin%
\definecolor{currentfill}{rgb}{1.000000,0.349310,0.000000}%
\pgfsetfillcolor{currentfill}%
\pgfsetlinewidth{0.000000pt}%
\definecolor{currentstroke}{rgb}{0.000000,0.000000,0.000000}%
\pgfsetstrokecolor{currentstroke}%
\pgfsetdash{}{0pt}%
\pgfpathmoveto{\pgfqpoint{0.928457in}{1.825238in}}%
\pgfpathlineto{\pgfqpoint{0.911038in}{1.840444in}}%
\pgfpathlineto{\pgfqpoint{0.925933in}{1.792929in}}%
\pgfpathlineto{\pgfqpoint{0.943027in}{1.778967in}}%
\pgfpathlineto{\pgfqpoint{0.928457in}{1.825238in}}%
\pgfpathclose%
\pgfusepath{fill}%
\end{pgfscope}%
\begin{pgfscope}%
\pgfpathrectangle{\pgfqpoint{0.000000in}{0.000000in}}{\pgfqpoint{3.000000in}{3.000000in}}%
\pgfusepath{clip}%
\pgfsetbuttcap%
\pgfsetroundjoin%
\definecolor{currentfill}{rgb}{0.401645,1.000000,0.566097}%
\pgfsetfillcolor{currentfill}%
\pgfsetlinewidth{0.000000pt}%
\definecolor{currentstroke}{rgb}{0.000000,0.000000,0.000000}%
\pgfsetstrokecolor{currentstroke}%
\pgfsetdash{}{0pt}%
\pgfpathmoveto{\pgfqpoint{1.417817in}{1.358739in}}%
\pgfpathlineto{\pgfqpoint{1.411437in}{1.379489in}}%
\pgfpathlineto{\pgfqpoint{1.465575in}{1.371316in}}%
\pgfpathlineto{\pgfqpoint{1.469283in}{1.350953in}}%
\pgfpathlineto{\pgfqpoint{1.417817in}{1.358739in}}%
\pgfpathclose%
\pgfusepath{fill}%
\end{pgfscope}%
\begin{pgfscope}%
\pgfpathrectangle{\pgfqpoint{0.000000in}{0.000000in}}{\pgfqpoint{3.000000in}{3.000000in}}%
\pgfusepath{clip}%
\pgfsetbuttcap%
\pgfsetroundjoin%
\definecolor{currentfill}{rgb}{1.000000,0.886710,0.000000}%
\pgfsetfillcolor{currentfill}%
\pgfsetlinewidth{0.000000pt}%
\definecolor{currentstroke}{rgb}{0.000000,0.000000,0.000000}%
\pgfsetstrokecolor{currentstroke}%
\pgfsetdash{}{0pt}%
\pgfpathmoveto{\pgfqpoint{1.956856in}{1.594165in}}%
\pgfpathlineto{\pgfqpoint{1.972230in}{1.610015in}}%
\pgfpathlineto{\pgfqpoint{2.004156in}{1.644921in}}%
\pgfpathlineto{\pgfqpoint{1.987705in}{1.627890in}}%
\pgfpathlineto{\pgfqpoint{1.956856in}{1.594165in}}%
\pgfpathclose%
\pgfusepath{fill}%
\end{pgfscope}%
\begin{pgfscope}%
\pgfpathrectangle{\pgfqpoint{0.000000in}{0.000000in}}{\pgfqpoint{3.000000in}{3.000000in}}%
\pgfusepath{clip}%
\pgfsetbuttcap%
\pgfsetroundjoin%
\definecolor{currentfill}{rgb}{0.958254,0.973856,0.009488}%
\pgfsetfillcolor{currentfill}%
\pgfsetlinewidth{0.000000pt}%
\definecolor{currentstroke}{rgb}{0.000000,0.000000,0.000000}%
\pgfsetstrokecolor{currentstroke}%
\pgfsetdash{}{0pt}%
\pgfpathmoveto{\pgfqpoint{1.118546in}{1.599062in}}%
\pgfpathlineto{\pgfqpoint{1.102431in}{1.616400in}}%
\pgfpathlineto{\pgfqpoint{1.136913in}{1.583485in}}%
\pgfpathlineto{\pgfqpoint{1.151811in}{1.567302in}}%
\pgfpathlineto{\pgfqpoint{1.118546in}{1.599062in}}%
\pgfpathclose%
\pgfusepath{fill}%
\end{pgfscope}%
\begin{pgfscope}%
\pgfpathrectangle{\pgfqpoint{0.000000in}{0.000000in}}{\pgfqpoint{3.000000in}{3.000000in}}%
\pgfusepath{clip}%
\pgfsetbuttcap%
\pgfsetroundjoin%
\definecolor{currentfill}{rgb}{0.678253,0.000000,0.000000}%
\pgfsetfillcolor{currentfill}%
\pgfsetlinewidth{0.000000pt}%
\definecolor{currentstroke}{rgb}{0.000000,0.000000,0.000000}%
\pgfsetstrokecolor{currentstroke}%
\pgfsetdash{}{0pt}%
\pgfpathmoveto{\pgfqpoint{0.755558in}{2.023308in}}%
\pgfpathlineto{\pgfqpoint{0.738155in}{2.037171in}}%
\pgfpathlineto{\pgfqpoint{0.736040in}{1.977090in}}%
\pgfpathlineto{\pgfqpoint{0.753605in}{1.964448in}}%
\pgfpathlineto{\pgfqpoint{0.755558in}{2.023308in}}%
\pgfpathclose%
\pgfusepath{fill}%
\end{pgfscope}%
\begin{pgfscope}%
\pgfpathrectangle{\pgfqpoint{0.000000in}{0.000000in}}{\pgfqpoint{3.000000in}{3.000000in}}%
\pgfusepath{clip}%
\pgfsetbuttcap%
\pgfsetroundjoin%
\definecolor{currentfill}{rgb}{1.000000,0.175018,0.000000}%
\pgfsetfillcolor{currentfill}%
\pgfsetlinewidth{0.000000pt}%
\definecolor{currentstroke}{rgb}{0.000000,0.000000,0.000000}%
\pgfsetstrokecolor{currentstroke}%
\pgfsetdash{}{0pt}%
\pgfpathmoveto{\pgfqpoint{2.196405in}{1.836426in}}%
\pgfpathlineto{\pgfqpoint{2.213710in}{1.849857in}}%
\pgfpathlineto{\pgfqpoint{2.223899in}{1.901329in}}%
\pgfpathlineto{\pgfqpoint{2.206429in}{1.886656in}}%
\pgfpathlineto{\pgfqpoint{2.196405in}{1.836426in}}%
\pgfpathclose%
\pgfusepath{fill}%
\end{pgfscope}%
\begin{pgfscope}%
\pgfpathrectangle{\pgfqpoint{0.000000in}{0.000000in}}{\pgfqpoint{3.000000in}{3.000000in}}%
\pgfusepath{clip}%
\pgfsetbuttcap%
\pgfsetroundjoin%
\definecolor{currentfill}{rgb}{0.578748,1.000000,0.388994}%
\pgfsetfillcolor{currentfill}%
\pgfsetlinewidth{0.000000pt}%
\definecolor{currentstroke}{rgb}{0.000000,0.000000,0.000000}%
\pgfsetstrokecolor{currentstroke}%
\pgfsetdash{}{0pt}%
\pgfpathmoveto{\pgfqpoint{1.746075in}{1.416832in}}%
\pgfpathlineto{\pgfqpoint{1.755782in}{1.435430in}}%
\pgfpathlineto{\pgfqpoint{1.804449in}{1.454480in}}%
\pgfpathlineto{\pgfqpoint{1.792570in}{1.435060in}}%
\pgfpathlineto{\pgfqpoint{1.746075in}{1.416832in}}%
\pgfpathclose%
\pgfusepath{fill}%
\end{pgfscope}%
\begin{pgfscope}%
\pgfpathrectangle{\pgfqpoint{0.000000in}{0.000000in}}{\pgfqpoint{3.000000in}{3.000000in}}%
\pgfusepath{clip}%
\pgfsetbuttcap%
\pgfsetroundjoin%
\definecolor{currentfill}{rgb}{0.490196,1.000000,0.477546}%
\pgfsetfillcolor{currentfill}%
\pgfsetlinewidth{0.000000pt}%
\definecolor{currentstroke}{rgb}{0.000000,0.000000,0.000000}%
\pgfsetstrokecolor{currentstroke}%
\pgfsetdash{}{0pt}%
\pgfpathmoveto{\pgfqpoint{1.687012in}{1.383158in}}%
\pgfpathlineto{\pgfqpoint{1.694266in}{1.402496in}}%
\pgfpathlineto{\pgfqpoint{1.746075in}{1.416832in}}%
\pgfpathlineto{\pgfqpoint{1.736391in}{1.396847in}}%
\pgfpathlineto{\pgfqpoint{1.687012in}{1.383158in}}%
\pgfpathclose%
\pgfusepath{fill}%
\end{pgfscope}%
\begin{pgfscope}%
\pgfpathrectangle{\pgfqpoint{0.000000in}{0.000000in}}{\pgfqpoint{3.000000in}{3.000000in}}%
\pgfusepath{clip}%
\pgfsetbuttcap%
\pgfsetroundjoin%
\definecolor{currentfill}{rgb}{0.401645,1.000000,0.566097}%
\pgfsetfillcolor{currentfill}%
\pgfsetlinewidth{0.000000pt}%
\definecolor{currentstroke}{rgb}{0.000000,0.000000,0.000000}%
\pgfsetstrokecolor{currentstroke}%
\pgfsetdash{}{0pt}%
\pgfpathmoveto{\pgfqpoint{1.576376in}{1.348090in}}%
\pgfpathlineto{\pgfqpoint{1.578241in}{1.368310in}}%
\pgfpathlineto{\pgfqpoint{1.633839in}{1.373558in}}%
\pgfpathlineto{\pgfqpoint{1.629226in}{1.353090in}}%
\pgfpathlineto{\pgfqpoint{1.576376in}{1.348090in}}%
\pgfpathclose%
\pgfusepath{fill}%
\end{pgfscope}%
\begin{pgfscope}%
\pgfpathrectangle{\pgfqpoint{0.000000in}{0.000000in}}{\pgfqpoint{3.000000in}{3.000000in}}%
\pgfusepath{clip}%
\pgfsetbuttcap%
\pgfsetroundjoin%
\definecolor{currentfill}{rgb}{0.667299,1.000000,0.300443}%
\pgfsetfillcolor{currentfill}%
\pgfsetlinewidth{0.000000pt}%
\definecolor{currentstroke}{rgb}{0.000000,0.000000,0.000000}%
\pgfsetstrokecolor{currentstroke}%
\pgfsetdash{}{0pt}%
\pgfpathmoveto{\pgfqpoint{1.247897in}{1.469187in}}%
\pgfpathlineto{\pgfqpoint{1.234718in}{1.488016in}}%
\pgfpathlineto{\pgfqpoint{1.280943in}{1.465612in}}%
\pgfpathlineto{\pgfqpoint{1.292156in}{1.447708in}}%
\pgfpathlineto{\pgfqpoint{1.247897in}{1.469187in}}%
\pgfpathclose%
\pgfusepath{fill}%
\end{pgfscope}%
\begin{pgfscope}%
\pgfpathrectangle{\pgfqpoint{0.000000in}{0.000000in}}{\pgfqpoint{3.000000in}{3.000000in}}%
\pgfusepath{clip}%
\pgfsetbuttcap%
\pgfsetroundjoin%
\definecolor{currentfill}{rgb}{1.000000,0.668845,0.000000}%
\pgfsetfillcolor{currentfill}%
\pgfsetlinewidth{0.000000pt}%
\definecolor{currentstroke}{rgb}{0.000000,0.000000,0.000000}%
\pgfsetstrokecolor{currentstroke}%
\pgfsetdash{}{0pt}%
\pgfpathmoveto{\pgfqpoint{1.028197in}{1.703127in}}%
\pgfpathlineto{\pgfqpoint{1.011202in}{1.719209in}}%
\pgfpathlineto{\pgfqpoint{1.037731in}{1.679279in}}%
\pgfpathlineto{\pgfqpoint{1.053941in}{1.664417in}}%
\pgfpathlineto{\pgfqpoint{1.028197in}{1.703127in}}%
\pgfpathclose%
\pgfusepath{fill}%
\end{pgfscope}%
\begin{pgfscope}%
\pgfpathrectangle{\pgfqpoint{0.000000in}{0.000000in}}{\pgfqpoint{3.000000in}{3.000000in}}%
\pgfusepath{clip}%
\pgfsetbuttcap%
\pgfsetroundjoin%
\definecolor{currentfill}{rgb}{0.401645,1.000000,0.566097}%
\pgfsetfillcolor{currentfill}%
\pgfsetlinewidth{0.000000pt}%
\definecolor{currentstroke}{rgb}{0.000000,0.000000,0.000000}%
\pgfsetstrokecolor{currentstroke}%
\pgfsetdash{}{0pt}%
\pgfpathmoveto{\pgfqpoint{1.469283in}{1.350953in}}%
\pgfpathlineto{\pgfqpoint{1.465575in}{1.371316in}}%
\pgfpathlineto{\pgfqpoint{1.521663in}{1.367555in}}%
\pgfpathlineto{\pgfqpoint{1.522597in}{1.347371in}}%
\pgfpathlineto{\pgfqpoint{1.469283in}{1.350953in}}%
\pgfpathclose%
\pgfusepath{fill}%
\end{pgfscope}%
\begin{pgfscope}%
\pgfpathrectangle{\pgfqpoint{0.000000in}{0.000000in}}{\pgfqpoint{3.000000in}{3.000000in}}%
\pgfusepath{clip}%
\pgfsetbuttcap%
\pgfsetroundjoin%
\definecolor{currentfill}{rgb}{0.606952,0.000000,0.000000}%
\pgfsetfillcolor{currentfill}%
\pgfsetlinewidth{0.000000pt}%
\definecolor{currentstroke}{rgb}{0.000000,0.000000,0.000000}%
\pgfsetstrokecolor{currentstroke}%
\pgfsetdash{}{0pt}%
\pgfpathmoveto{\pgfqpoint{0.738155in}{2.037171in}}%
\pgfpathlineto{\pgfqpoint{0.720742in}{2.050850in}}%
\pgfpathlineto{\pgfqpoint{0.718460in}{1.989551in}}%
\pgfpathlineto{\pgfqpoint{0.736040in}{1.977090in}}%
\pgfpathlineto{\pgfqpoint{0.738155in}{2.037171in}}%
\pgfpathclose%
\pgfusepath{fill}%
\end{pgfscope}%
\begin{pgfscope}%
\pgfpathrectangle{\pgfqpoint{0.000000in}{0.000000in}}{\pgfqpoint{3.000000in}{3.000000in}}%
\pgfusepath{clip}%
\pgfsetbuttcap%
\pgfsetroundjoin%
\definecolor{currentfill}{rgb}{1.000000,0.291213,0.000000}%
\pgfsetfillcolor{currentfill}%
\pgfsetlinewidth{0.000000pt}%
\definecolor{currentstroke}{rgb}{0.000000,0.000000,0.000000}%
\pgfsetstrokecolor{currentstroke}%
\pgfsetdash{}{0pt}%
\pgfpathmoveto{\pgfqpoint{0.911038in}{1.840444in}}%
\pgfpathlineto{\pgfqpoint{0.893605in}{1.855313in}}%
\pgfpathlineto{\pgfqpoint{0.908820in}{1.806555in}}%
\pgfpathlineto{\pgfqpoint{0.925933in}{1.792929in}}%
\pgfpathlineto{\pgfqpoint{0.911038in}{1.840444in}}%
\pgfpathclose%
\pgfusepath{fill}%
\end{pgfscope}%
\begin{pgfscope}%
\pgfpathrectangle{\pgfqpoint{0.000000in}{0.000000in}}{\pgfqpoint{3.000000in}{3.000000in}}%
\pgfusepath{clip}%
\pgfsetbuttcap%
\pgfsetroundjoin%
\definecolor{currentfill}{rgb}{0.401645,1.000000,0.566097}%
\pgfsetfillcolor{currentfill}%
\pgfsetlinewidth{0.000000pt}%
\definecolor{currentstroke}{rgb}{0.000000,0.000000,0.000000}%
\pgfsetstrokecolor{currentstroke}%
\pgfsetdash{}{0pt}%
\pgfpathmoveto{\pgfqpoint{1.522597in}{1.347371in}}%
\pgfpathlineto{\pgfqpoint{1.521663in}{1.367555in}}%
\pgfpathlineto{\pgfqpoint{1.578241in}{1.368310in}}%
\pgfpathlineto{\pgfqpoint{1.576376in}{1.348090in}}%
\pgfpathlineto{\pgfqpoint{1.522597in}{1.347371in}}%
\pgfpathclose%
\pgfusepath{fill}%
\end{pgfscope}%
\begin{pgfscope}%
\pgfpathrectangle{\pgfqpoint{0.000000in}{0.000000in}}{\pgfqpoint{3.000000in}{3.000000in}}%
\pgfusepath{clip}%
\pgfsetbuttcap%
\pgfsetroundjoin%
\definecolor{currentfill}{rgb}{0.819102,1.000000,0.148640}%
\pgfsetfillcolor{currentfill}%
\pgfsetlinewidth{0.000000pt}%
\definecolor{currentstroke}{rgb}{0.000000,0.000000,0.000000}%
\pgfsetstrokecolor{currentstroke}%
\pgfsetdash{}{0pt}%
\pgfpathmoveto{\pgfqpoint{1.181529in}{1.532505in}}%
\pgfpathlineto{\pgfqpoint{1.166683in}{1.550337in}}%
\pgfpathlineto{\pgfqpoint{1.208281in}{1.522536in}}%
\pgfpathlineto{\pgfqpoint{1.221513in}{1.505760in}}%
\pgfpathlineto{\pgfqpoint{1.181529in}{1.532505in}}%
\pgfpathclose%
\pgfusepath{fill}%
\end{pgfscope}%
\begin{pgfscope}%
\pgfpathrectangle{\pgfqpoint{0.000000in}{0.000000in}}{\pgfqpoint{3.000000in}{3.000000in}}%
\pgfusepath{clip}%
\pgfsetbuttcap%
\pgfsetroundjoin%
\definecolor{currentfill}{rgb}{0.490196,1.000000,0.477546}%
\pgfsetfillcolor{currentfill}%
\pgfsetlinewidth{0.000000pt}%
\definecolor{currentstroke}{rgb}{0.000000,0.000000,0.000000}%
\pgfsetstrokecolor{currentstroke}%
\pgfsetdash{}{0pt}%
\pgfpathmoveto{\pgfqpoint{1.360648in}{1.391850in}}%
\pgfpathlineto{\pgfqpoint{1.351748in}{1.411600in}}%
\pgfpathlineto{\pgfqpoint{1.405041in}{1.398653in}}%
\pgfpathlineto{\pgfqpoint{1.411437in}{1.379489in}}%
\pgfpathlineto{\pgfqpoint{1.360648in}{1.391850in}}%
\pgfpathclose%
\pgfusepath{fill}%
\end{pgfscope}%
\begin{pgfscope}%
\pgfpathrectangle{\pgfqpoint{0.000000in}{0.000000in}}{\pgfqpoint{3.000000in}{3.000000in}}%
\pgfusepath{clip}%
\pgfsetbuttcap%
\pgfsetroundjoin%
\definecolor{currentfill}{rgb}{1.000000,0.538126,0.000000}%
\pgfsetfillcolor{currentfill}%
\pgfsetlinewidth{0.000000pt}%
\definecolor{currentstroke}{rgb}{0.000000,0.000000,0.000000}%
\pgfsetstrokecolor{currentstroke}%
\pgfsetdash{}{0pt}%
\pgfpathmoveto{\pgfqpoint{2.070190in}{1.707145in}}%
\pgfpathlineto{\pgfqpoint{2.086755in}{1.721460in}}%
\pgfpathlineto{\pgfqpoint{2.110152in}{1.764422in}}%
\pgfpathlineto{\pgfqpoint{2.092956in}{1.748874in}}%
\pgfpathlineto{\pgfqpoint{2.070190in}{1.707145in}}%
\pgfpathclose%
\pgfusepath{fill}%
\end{pgfscope}%
\begin{pgfscope}%
\pgfpathrectangle{\pgfqpoint{0.000000in}{0.000000in}}{\pgfqpoint{3.000000in}{3.000000in}}%
\pgfusepath{clip}%
\pgfsetbuttcap%
\pgfsetroundjoin%
\definecolor{currentfill}{rgb}{1.000000,0.116921,0.000000}%
\pgfsetfillcolor{currentfill}%
\pgfsetlinewidth{0.000000pt}%
\definecolor{currentstroke}{rgb}{0.000000,0.000000,0.000000}%
\pgfsetstrokecolor{currentstroke}%
\pgfsetdash{}{0pt}%
\pgfpathmoveto{\pgfqpoint{2.213710in}{1.849857in}}%
\pgfpathlineto{\pgfqpoint{2.231033in}{1.863014in}}%
\pgfpathlineto{\pgfqpoint{2.241381in}{1.915726in}}%
\pgfpathlineto{\pgfqpoint{2.223899in}{1.901329in}}%
\pgfpathlineto{\pgfqpoint{2.213710in}{1.849857in}}%
\pgfpathclose%
\pgfusepath{fill}%
\end{pgfscope}%
\begin{pgfscope}%
\pgfpathrectangle{\pgfqpoint{0.000000in}{0.000000in}}{\pgfqpoint{3.000000in}{3.000000in}}%
\pgfusepath{clip}%
\pgfsetbuttcap%
\pgfsetroundjoin%
\definecolor{currentfill}{rgb}{0.578748,1.000000,0.388994}%
\pgfsetfillcolor{currentfill}%
\pgfsetlinewidth{0.000000pt}%
\definecolor{currentstroke}{rgb}{0.000000,0.000000,0.000000}%
\pgfsetstrokecolor{currentstroke}%
\pgfsetdash{}{0pt}%
\pgfpathmoveto{\pgfqpoint{1.303344in}{1.428580in}}%
\pgfpathlineto{\pgfqpoint{1.292156in}{1.447708in}}%
\pgfpathlineto{\pgfqpoint{1.342827in}{1.429961in}}%
\pgfpathlineto{\pgfqpoint{1.351748in}{1.411600in}}%
\pgfpathlineto{\pgfqpoint{1.303344in}{1.428580in}}%
\pgfpathclose%
\pgfusepath{fill}%
\end{pgfscope}%
\begin{pgfscope}%
\pgfpathrectangle{\pgfqpoint{0.000000in}{0.000000in}}{\pgfqpoint{3.000000in}{3.000000in}}%
\pgfusepath{clip}%
\pgfsetbuttcap%
\pgfsetroundjoin%
\definecolor{currentfill}{rgb}{0.895003,1.000000,0.072739}%
\pgfsetfillcolor{currentfill}%
\pgfsetlinewidth{0.000000pt}%
\definecolor{currentstroke}{rgb}{0.000000,0.000000,0.000000}%
\pgfsetstrokecolor{currentstroke}%
\pgfsetdash{}{0pt}%
\pgfpathmoveto{\pgfqpoint{1.887672in}{1.531445in}}%
\pgfpathlineto{\pgfqpoint{1.901510in}{1.547692in}}%
\pgfpathlineto{\pgfqpoint{1.941506in}{1.577606in}}%
\pgfpathlineto{\pgfqpoint{1.926182in}{1.560266in}}%
\pgfpathlineto{\pgfqpoint{1.887672in}{1.531445in}}%
\pgfpathclose%
\pgfusepath{fill}%
\end{pgfscope}%
\begin{pgfscope}%
\pgfpathrectangle{\pgfqpoint{0.000000in}{0.000000in}}{\pgfqpoint{3.000000in}{3.000000in}}%
\pgfusepath{clip}%
\pgfsetbuttcap%
\pgfsetroundjoin%
\definecolor{currentfill}{rgb}{1.000000,0.814089,0.000000}%
\pgfsetfillcolor{currentfill}%
\pgfsetlinewidth{0.000000pt}%
\definecolor{currentstroke}{rgb}{0.000000,0.000000,0.000000}%
\pgfsetstrokecolor{currentstroke}%
\pgfsetdash{}{0pt}%
\pgfpathmoveto{\pgfqpoint{1.972230in}{1.610015in}}%
\pgfpathlineto{\pgfqpoint{1.987630in}{1.625222in}}%
\pgfpathlineto{\pgfqpoint{2.020631in}{1.661310in}}%
\pgfpathlineto{\pgfqpoint{2.004156in}{1.644921in}}%
\pgfpathlineto{\pgfqpoint{1.972230in}{1.610015in}}%
\pgfpathclose%
\pgfusepath{fill}%
\end{pgfscope}%
\begin{pgfscope}%
\pgfpathrectangle{\pgfqpoint{0.000000in}{0.000000in}}{\pgfqpoint{3.000000in}{3.000000in}}%
\pgfusepath{clip}%
\pgfsetbuttcap%
\pgfsetroundjoin%
\definecolor{currentfill}{rgb}{0.743201,1.000000,0.224541}%
\pgfsetfillcolor{currentfill}%
\pgfsetlinewidth{0.000000pt}%
\definecolor{currentstroke}{rgb}{0.000000,0.000000,0.000000}%
\pgfsetstrokecolor{currentstroke}%
\pgfsetdash{}{0pt}%
\pgfpathmoveto{\pgfqpoint{1.816353in}{1.472676in}}%
\pgfpathlineto{\pgfqpoint{1.828283in}{1.489787in}}%
\pgfpathlineto{\pgfqpoint{1.873861in}{1.514329in}}%
\pgfpathlineto{\pgfqpoint{1.860076in}{1.496246in}}%
\pgfpathlineto{\pgfqpoint{1.816353in}{1.472676in}}%
\pgfpathclose%
\pgfusepath{fill}%
\end{pgfscope}%
\begin{pgfscope}%
\pgfpathrectangle{\pgfqpoint{0.000000in}{0.000000in}}{\pgfqpoint{3.000000in}{3.000000in}}%
\pgfusepath{clip}%
\pgfsetbuttcap%
\pgfsetroundjoin%
\definecolor{currentfill}{rgb}{0.553476,0.000000,0.000000}%
\pgfsetfillcolor{currentfill}%
\pgfsetlinewidth{0.000000pt}%
\definecolor{currentstroke}{rgb}{0.000000,0.000000,0.000000}%
\pgfsetstrokecolor{currentstroke}%
\pgfsetdash{}{0pt}%
\pgfpathmoveto{\pgfqpoint{0.720742in}{2.050850in}}%
\pgfpathlineto{\pgfqpoint{0.703320in}{2.064354in}}%
\pgfpathlineto{\pgfqpoint{0.700866in}{2.001839in}}%
\pgfpathlineto{\pgfqpoint{0.718460in}{1.989551in}}%
\pgfpathlineto{\pgfqpoint{0.720742in}{2.050850in}}%
\pgfpathclose%
\pgfusepath{fill}%
\end{pgfscope}%
\begin{pgfscope}%
\pgfpathrectangle{\pgfqpoint{0.000000in}{0.000000in}}{\pgfqpoint{3.000000in}{3.000000in}}%
\pgfusepath{clip}%
\pgfsetbuttcap%
\pgfsetroundjoin%
\definecolor{currentfill}{rgb}{1.000000,0.886710,0.000000}%
\pgfsetfillcolor{currentfill}%
\pgfsetlinewidth{0.000000pt}%
\definecolor{currentstroke}{rgb}{0.000000,0.000000,0.000000}%
\pgfsetstrokecolor{currentstroke}%
\pgfsetdash{}{0pt}%
\pgfpathmoveto{\pgfqpoint{1.102431in}{1.616400in}}%
\pgfpathlineto{\pgfqpoint{1.086291in}{1.633030in}}%
\pgfpathlineto{\pgfqpoint{1.121989in}{1.598959in}}%
\pgfpathlineto{\pgfqpoint{1.136913in}{1.583485in}}%
\pgfpathlineto{\pgfqpoint{1.102431in}{1.616400in}}%
\pgfpathclose%
\pgfusepath{fill}%
\end{pgfscope}%
\begin{pgfscope}%
\pgfpathrectangle{\pgfqpoint{0.000000in}{0.000000in}}{\pgfqpoint{3.000000in}{3.000000in}}%
\pgfusepath{clip}%
\pgfsetbuttcap%
\pgfsetroundjoin%
\definecolor{currentfill}{rgb}{1.000000,0.610748,0.000000}%
\pgfsetfillcolor{currentfill}%
\pgfsetlinewidth{0.000000pt}%
\definecolor{currentstroke}{rgb}{0.000000,0.000000,0.000000}%
\pgfsetstrokecolor{currentstroke}%
\pgfsetdash{}{0pt}%
\pgfpathmoveto{\pgfqpoint{1.011202in}{1.719209in}}%
\pgfpathlineto{\pgfqpoint{0.994188in}{1.734797in}}%
\pgfpathlineto{\pgfqpoint{1.021497in}{1.693648in}}%
\pgfpathlineto{\pgfqpoint{1.037731in}{1.679279in}}%
\pgfpathlineto{\pgfqpoint{1.011202in}{1.719209in}}%
\pgfpathclose%
\pgfusepath{fill}%
\end{pgfscope}%
\begin{pgfscope}%
\pgfpathrectangle{\pgfqpoint{0.000000in}{0.000000in}}{\pgfqpoint{3.000000in}{3.000000in}}%
\pgfusepath{clip}%
\pgfsetbuttcap%
\pgfsetroundjoin%
\definecolor{currentfill}{rgb}{1.000000,0.233115,0.000000}%
\pgfsetfillcolor{currentfill}%
\pgfsetlinewidth{0.000000pt}%
\definecolor{currentstroke}{rgb}{0.000000,0.000000,0.000000}%
\pgfsetstrokecolor{currentstroke}%
\pgfsetdash{}{0pt}%
\pgfpathmoveto{\pgfqpoint{0.893605in}{1.855313in}}%
\pgfpathlineto{\pgfqpoint{0.876156in}{1.869867in}}%
\pgfpathlineto{\pgfqpoint{0.891688in}{1.819868in}}%
\pgfpathlineto{\pgfqpoint{0.908820in}{1.806555in}}%
\pgfpathlineto{\pgfqpoint{0.893605in}{1.855313in}}%
\pgfpathclose%
\pgfusepath{fill}%
\end{pgfscope}%
\begin{pgfscope}%
\pgfpathrectangle{\pgfqpoint{0.000000in}{0.000000in}}{\pgfqpoint{3.000000in}{3.000000in}}%
\pgfusepath{clip}%
\pgfsetbuttcap%
\pgfsetroundjoin%
\definecolor{currentfill}{rgb}{0.490196,1.000000,0.477546}%
\pgfsetfillcolor{currentfill}%
\pgfsetlinewidth{0.000000pt}%
\definecolor{currentstroke}{rgb}{0.000000,0.000000,0.000000}%
\pgfsetstrokecolor{currentstroke}%
\pgfsetdash{}{0pt}%
\pgfpathmoveto{\pgfqpoint{1.633839in}{1.373558in}}%
\pgfpathlineto{\pgfqpoint{1.638464in}{1.392441in}}%
\pgfpathlineto{\pgfqpoint{1.694266in}{1.402496in}}%
\pgfpathlineto{\pgfqpoint{1.687012in}{1.383158in}}%
\pgfpathlineto{\pgfqpoint{1.633839in}{1.373558in}}%
\pgfpathclose%
\pgfusepath{fill}%
\end{pgfscope}%
\begin{pgfscope}%
\pgfpathrectangle{\pgfqpoint{0.000000in}{0.000000in}}{\pgfqpoint{3.000000in}{3.000000in}}%
\pgfusepath{clip}%
\pgfsetbuttcap%
\pgfsetroundjoin%
\definecolor{currentfill}{rgb}{0.999109,0.073348,0.000000}%
\pgfsetfillcolor{currentfill}%
\pgfsetlinewidth{0.000000pt}%
\definecolor{currentstroke}{rgb}{0.000000,0.000000,0.000000}%
\pgfsetstrokecolor{currentstroke}%
\pgfsetdash{}{0pt}%
\pgfpathmoveto{\pgfqpoint{2.231033in}{1.863014in}}%
\pgfpathlineto{\pgfqpoint{2.248374in}{1.875914in}}%
\pgfpathlineto{\pgfqpoint{2.258877in}{1.929864in}}%
\pgfpathlineto{\pgfqpoint{2.241381in}{1.915726in}}%
\pgfpathlineto{\pgfqpoint{2.231033in}{1.863014in}}%
\pgfpathclose%
\pgfusepath{fill}%
\end{pgfscope}%
\begin{pgfscope}%
\pgfpathrectangle{\pgfqpoint{0.000000in}{0.000000in}}{\pgfqpoint{3.000000in}{3.000000in}}%
\pgfusepath{clip}%
\pgfsetbuttcap%
\pgfsetroundjoin%
\definecolor{currentfill}{rgb}{0.500000,0.000000,0.000000}%
\pgfsetfillcolor{currentfill}%
\pgfsetlinewidth{0.000000pt}%
\definecolor{currentstroke}{rgb}{0.000000,0.000000,0.000000}%
\pgfsetstrokecolor{currentstroke}%
\pgfsetdash{}{0pt}%
\pgfpathmoveto{\pgfqpoint{0.703320in}{2.064354in}}%
\pgfpathlineto{\pgfqpoint{0.685889in}{2.077692in}}%
\pgfpathlineto{\pgfqpoint{0.683258in}{2.013962in}}%
\pgfpathlineto{\pgfqpoint{0.700866in}{2.001839in}}%
\pgfpathlineto{\pgfqpoint{0.703320in}{2.064354in}}%
\pgfpathclose%
\pgfusepath{fill}%
\end{pgfscope}%
\begin{pgfscope}%
\pgfpathrectangle{\pgfqpoint{0.000000in}{0.000000in}}{\pgfqpoint{3.000000in}{3.000000in}}%
\pgfusepath{clip}%
\pgfsetbuttcap%
\pgfsetroundjoin%
\definecolor{currentfill}{rgb}{1.000000,0.480029,0.000000}%
\pgfsetfillcolor{currentfill}%
\pgfsetlinewidth{0.000000pt}%
\definecolor{currentstroke}{rgb}{0.000000,0.000000,0.000000}%
\pgfsetstrokecolor{currentstroke}%
\pgfsetdash{}{0pt}%
\pgfpathmoveto{\pgfqpoint{2.086755in}{1.721460in}}%
\pgfpathlineto{\pgfqpoint{2.103342in}{1.735356in}}%
\pgfpathlineto{\pgfqpoint{2.127366in}{1.779549in}}%
\pgfpathlineto{\pgfqpoint{2.110152in}{1.764422in}}%
\pgfpathlineto{\pgfqpoint{2.086755in}{1.721460in}}%
\pgfpathclose%
\pgfusepath{fill}%
\end{pgfscope}%
\begin{pgfscope}%
\pgfpathrectangle{\pgfqpoint{0.000000in}{0.000000in}}{\pgfqpoint{3.000000in}{3.000000in}}%
\pgfusepath{clip}%
\pgfsetbuttcap%
\pgfsetroundjoin%
\definecolor{currentfill}{rgb}{0.490196,1.000000,0.477546}%
\pgfsetfillcolor{currentfill}%
\pgfsetlinewidth{0.000000pt}%
\definecolor{currentstroke}{rgb}{0.000000,0.000000,0.000000}%
\pgfsetstrokecolor{currentstroke}%
\pgfsetdash{}{0pt}%
\pgfpathmoveto{\pgfqpoint{1.411437in}{1.379489in}}%
\pgfpathlineto{\pgfqpoint{1.405041in}{1.398653in}}%
\pgfpathlineto{\pgfqpoint{1.461858in}{1.390092in}}%
\pgfpathlineto{\pgfqpoint{1.465575in}{1.371316in}}%
\pgfpathlineto{\pgfqpoint{1.411437in}{1.379489in}}%
\pgfpathclose%
\pgfusepath{fill}%
\end{pgfscope}%
\begin{pgfscope}%
\pgfpathrectangle{\pgfqpoint{0.000000in}{0.000000in}}{\pgfqpoint{3.000000in}{3.000000in}}%
\pgfusepath{clip}%
\pgfsetbuttcap%
\pgfsetroundjoin%
\definecolor{currentfill}{rgb}{0.667299,1.000000,0.300443}%
\pgfsetfillcolor{currentfill}%
\pgfsetlinewidth{0.000000pt}%
\definecolor{currentstroke}{rgb}{0.000000,0.000000,0.000000}%
\pgfsetstrokecolor{currentstroke}%
\pgfsetdash{}{0pt}%
\pgfpathmoveto{\pgfqpoint{1.755782in}{1.435430in}}%
\pgfpathlineto{\pgfqpoint{1.765511in}{1.452804in}}%
\pgfpathlineto{\pgfqpoint{1.816353in}{1.472676in}}%
\pgfpathlineto{\pgfqpoint{1.804449in}{1.454480in}}%
\pgfpathlineto{\pgfqpoint{1.755782in}{1.435430in}}%
\pgfpathclose%
\pgfusepath{fill}%
\end{pgfscope}%
\begin{pgfscope}%
\pgfpathrectangle{\pgfqpoint{0.000000in}{0.000000in}}{\pgfqpoint{3.000000in}{3.000000in}}%
\pgfusepath{clip}%
\pgfsetbuttcap%
\pgfsetroundjoin%
\definecolor{currentfill}{rgb}{0.743201,1.000000,0.224541}%
\pgfsetfillcolor{currentfill}%
\pgfsetlinewidth{0.000000pt}%
\definecolor{currentstroke}{rgb}{0.000000,0.000000,0.000000}%
\pgfsetstrokecolor{currentstroke}%
\pgfsetdash{}{0pt}%
\pgfpathmoveto{\pgfqpoint{1.234718in}{1.488016in}}%
\pgfpathlineto{\pgfqpoint{1.221513in}{1.505760in}}%
\pgfpathlineto{\pgfqpoint{1.269706in}{1.482431in}}%
\pgfpathlineto{\pgfqpoint{1.280943in}{1.465612in}}%
\pgfpathlineto{\pgfqpoint{1.234718in}{1.488016in}}%
\pgfpathclose%
\pgfusepath{fill}%
\end{pgfscope}%
\begin{pgfscope}%
\pgfpathrectangle{\pgfqpoint{0.000000in}{0.000000in}}{\pgfqpoint{3.000000in}{3.000000in}}%
\pgfusepath{clip}%
\pgfsetbuttcap%
\pgfsetroundjoin%
\definecolor{currentfill}{rgb}{0.578748,1.000000,0.388994}%
\pgfsetfillcolor{currentfill}%
\pgfsetlinewidth{0.000000pt}%
\definecolor{currentstroke}{rgb}{0.000000,0.000000,0.000000}%
\pgfsetstrokecolor{currentstroke}%
\pgfsetdash{}{0pt}%
\pgfpathmoveto{\pgfqpoint{1.694266in}{1.402496in}}%
\pgfpathlineto{\pgfqpoint{1.701537in}{1.420445in}}%
\pgfpathlineto{\pgfqpoint{1.755782in}{1.435430in}}%
\pgfpathlineto{\pgfqpoint{1.746075in}{1.416832in}}%
\pgfpathlineto{\pgfqpoint{1.694266in}{1.402496in}}%
\pgfpathclose%
\pgfusepath{fill}%
\end{pgfscope}%
\begin{pgfscope}%
\pgfpathrectangle{\pgfqpoint{0.000000in}{0.000000in}}{\pgfqpoint{3.000000in}{3.000000in}}%
\pgfusepath{clip}%
\pgfsetbuttcap%
\pgfsetroundjoin%
\definecolor{currentfill}{rgb}{0.895003,1.000000,0.072739}%
\pgfsetfillcolor{currentfill}%
\pgfsetlinewidth{0.000000pt}%
\definecolor{currentstroke}{rgb}{0.000000,0.000000,0.000000}%
\pgfsetstrokecolor{currentstroke}%
\pgfsetdash{}{0pt}%
\pgfpathmoveto{\pgfqpoint{1.166683in}{1.550337in}}%
\pgfpathlineto{\pgfqpoint{1.151811in}{1.567302in}}%
\pgfpathlineto{\pgfqpoint{1.195023in}{1.538444in}}%
\pgfpathlineto{\pgfqpoint{1.208281in}{1.522536in}}%
\pgfpathlineto{\pgfqpoint{1.166683in}{1.550337in}}%
\pgfpathclose%
\pgfusepath{fill}%
\end{pgfscope}%
\begin{pgfscope}%
\pgfpathrectangle{\pgfqpoint{0.000000in}{0.000000in}}{\pgfqpoint{3.000000in}{3.000000in}}%
\pgfusepath{clip}%
\pgfsetbuttcap%
\pgfsetroundjoin%
\definecolor{currentfill}{rgb}{1.000000,0.175018,0.000000}%
\pgfsetfillcolor{currentfill}%
\pgfsetlinewidth{0.000000pt}%
\definecolor{currentstroke}{rgb}{0.000000,0.000000,0.000000}%
\pgfsetstrokecolor{currentstroke}%
\pgfsetdash{}{0pt}%
\pgfpathmoveto{\pgfqpoint{0.876156in}{1.869867in}}%
\pgfpathlineto{\pgfqpoint{0.858693in}{1.884126in}}%
\pgfpathlineto{\pgfqpoint{0.874536in}{1.832887in}}%
\pgfpathlineto{\pgfqpoint{0.891688in}{1.819868in}}%
\pgfpathlineto{\pgfqpoint{0.876156in}{1.869867in}}%
\pgfpathclose%
\pgfusepath{fill}%
\end{pgfscope}%
\begin{pgfscope}%
\pgfpathrectangle{\pgfqpoint{0.000000in}{0.000000in}}{\pgfqpoint{3.000000in}{3.000000in}}%
\pgfusepath{clip}%
\pgfsetbuttcap%
\pgfsetroundjoin%
\definecolor{currentfill}{rgb}{0.927807,0.015251,0.000000}%
\pgfsetfillcolor{currentfill}%
\pgfsetlinewidth{0.000000pt}%
\definecolor{currentstroke}{rgb}{0.000000,0.000000,0.000000}%
\pgfsetstrokecolor{currentstroke}%
\pgfsetdash{}{0pt}%
\pgfpathmoveto{\pgfqpoint{2.248374in}{1.875914in}}%
\pgfpathlineto{\pgfqpoint{2.265733in}{1.888571in}}%
\pgfpathlineto{\pgfqpoint{2.276385in}{1.943758in}}%
\pgfpathlineto{\pgfqpoint{2.258877in}{1.929864in}}%
\pgfpathlineto{\pgfqpoint{2.248374in}{1.875914in}}%
\pgfpathclose%
\pgfusepath{fill}%
\end{pgfscope}%
\begin{pgfscope}%
\pgfpathrectangle{\pgfqpoint{0.000000in}{0.000000in}}{\pgfqpoint{3.000000in}{3.000000in}}%
\pgfusepath{clip}%
\pgfsetbuttcap%
\pgfsetroundjoin%
\definecolor{currentfill}{rgb}{1.000000,0.741467,0.000000}%
\pgfsetfillcolor{currentfill}%
\pgfsetlinewidth{0.000000pt}%
\definecolor{currentstroke}{rgb}{0.000000,0.000000,0.000000}%
\pgfsetstrokecolor{currentstroke}%
\pgfsetdash{}{0pt}%
\pgfpathmoveto{\pgfqpoint{1.987630in}{1.625222in}}%
\pgfpathlineto{\pgfqpoint{2.003055in}{1.639842in}}%
\pgfpathlineto{\pgfqpoint{2.037128in}{1.677111in}}%
\pgfpathlineto{\pgfqpoint{2.020631in}{1.661310in}}%
\pgfpathlineto{\pgfqpoint{1.987630in}{1.625222in}}%
\pgfpathclose%
\pgfusepath{fill}%
\end{pgfscope}%
\begin{pgfscope}%
\pgfpathrectangle{\pgfqpoint{0.000000in}{0.000000in}}{\pgfqpoint{3.000000in}{3.000000in}}%
\pgfusepath{clip}%
\pgfsetbuttcap%
\pgfsetroundjoin%
\definecolor{currentfill}{rgb}{0.958254,0.973856,0.009488}%
\pgfsetfillcolor{currentfill}%
\pgfsetlinewidth{0.000000pt}%
\definecolor{currentstroke}{rgb}{0.000000,0.000000,0.000000}%
\pgfsetstrokecolor{currentstroke}%
\pgfsetdash{}{0pt}%
\pgfpathmoveto{\pgfqpoint{1.901510in}{1.547692in}}%
\pgfpathlineto{\pgfqpoint{1.915374in}{1.563156in}}%
\pgfpathlineto{\pgfqpoint{1.956856in}{1.594165in}}%
\pgfpathlineto{\pgfqpoint{1.941506in}{1.577606in}}%
\pgfpathlineto{\pgfqpoint{1.901510in}{1.547692in}}%
\pgfpathclose%
\pgfusepath{fill}%
\end{pgfscope}%
\begin{pgfscope}%
\pgfpathrectangle{\pgfqpoint{0.000000in}{0.000000in}}{\pgfqpoint{3.000000in}{3.000000in}}%
\pgfusepath{clip}%
\pgfsetbuttcap%
\pgfsetroundjoin%
\definecolor{currentfill}{rgb}{0.490196,1.000000,0.477546}%
\pgfsetfillcolor{currentfill}%
\pgfsetlinewidth{0.000000pt}%
\definecolor{currentstroke}{rgb}{0.000000,0.000000,0.000000}%
\pgfsetstrokecolor{currentstroke}%
\pgfsetdash{}{0pt}%
\pgfpathmoveto{\pgfqpoint{1.578241in}{1.368310in}}%
\pgfpathlineto{\pgfqpoint{1.580111in}{1.386943in}}%
\pgfpathlineto{\pgfqpoint{1.638464in}{1.392441in}}%
\pgfpathlineto{\pgfqpoint{1.633839in}{1.373558in}}%
\pgfpathlineto{\pgfqpoint{1.578241in}{1.368310in}}%
\pgfpathclose%
\pgfusepath{fill}%
\end{pgfscope}%
\begin{pgfscope}%
\pgfpathrectangle{\pgfqpoint{0.000000in}{0.000000in}}{\pgfqpoint{3.000000in}{3.000000in}}%
\pgfusepath{clip}%
\pgfsetbuttcap%
\pgfsetroundjoin%
\definecolor{currentfill}{rgb}{1.000000,0.538126,0.000000}%
\pgfsetfillcolor{currentfill}%
\pgfsetlinewidth{0.000000pt}%
\definecolor{currentstroke}{rgb}{0.000000,0.000000,0.000000}%
\pgfsetstrokecolor{currentstroke}%
\pgfsetdash{}{0pt}%
\pgfpathmoveto{\pgfqpoint{0.994188in}{1.734797in}}%
\pgfpathlineto{\pgfqpoint{0.977154in}{1.749931in}}%
\pgfpathlineto{\pgfqpoint{1.005239in}{1.707563in}}%
\pgfpathlineto{\pgfqpoint{1.021497in}{1.693648in}}%
\pgfpathlineto{\pgfqpoint{0.994188in}{1.734797in}}%
\pgfpathclose%
\pgfusepath{fill}%
\end{pgfscope}%
\begin{pgfscope}%
\pgfpathrectangle{\pgfqpoint{0.000000in}{0.000000in}}{\pgfqpoint{3.000000in}{3.000000in}}%
\pgfusepath{clip}%
\pgfsetbuttcap%
\pgfsetroundjoin%
\definecolor{currentfill}{rgb}{1.000000,0.814089,0.000000}%
\pgfsetfillcolor{currentfill}%
\pgfsetlinewidth{0.000000pt}%
\definecolor{currentstroke}{rgb}{0.000000,0.000000,0.000000}%
\pgfsetstrokecolor{currentstroke}%
\pgfsetdash{}{0pt}%
\pgfpathmoveto{\pgfqpoint{1.086291in}{1.633030in}}%
\pgfpathlineto{\pgfqpoint{1.070128in}{1.649017in}}%
\pgfpathlineto{\pgfqpoint{1.107040in}{1.613790in}}%
\pgfpathlineto{\pgfqpoint{1.121989in}{1.598959in}}%
\pgfpathlineto{\pgfqpoint{1.086291in}{1.633030in}}%
\pgfpathclose%
\pgfusepath{fill}%
\end{pgfscope}%
\begin{pgfscope}%
\pgfpathrectangle{\pgfqpoint{0.000000in}{0.000000in}}{\pgfqpoint{3.000000in}{3.000000in}}%
\pgfusepath{clip}%
\pgfsetbuttcap%
\pgfsetroundjoin%
\definecolor{currentfill}{rgb}{0.490196,1.000000,0.477546}%
\pgfsetfillcolor{currentfill}%
\pgfsetlinewidth{0.000000pt}%
\definecolor{currentstroke}{rgb}{0.000000,0.000000,0.000000}%
\pgfsetstrokecolor{currentstroke}%
\pgfsetdash{}{0pt}%
\pgfpathmoveto{\pgfqpoint{1.465575in}{1.371316in}}%
\pgfpathlineto{\pgfqpoint{1.461858in}{1.390092in}}%
\pgfpathlineto{\pgfqpoint{1.520726in}{1.386152in}}%
\pgfpathlineto{\pgfqpoint{1.521663in}{1.367555in}}%
\pgfpathlineto{\pgfqpoint{1.465575in}{1.371316in}}%
\pgfpathclose%
\pgfusepath{fill}%
\end{pgfscope}%
\begin{pgfscope}%
\pgfpathrectangle{\pgfqpoint{0.000000in}{0.000000in}}{\pgfqpoint{3.000000in}{3.000000in}}%
\pgfusepath{clip}%
\pgfsetbuttcap%
\pgfsetroundjoin%
\definecolor{currentfill}{rgb}{0.578748,1.000000,0.388994}%
\pgfsetfillcolor{currentfill}%
\pgfsetlinewidth{0.000000pt}%
\definecolor{currentstroke}{rgb}{0.000000,0.000000,0.000000}%
\pgfsetstrokecolor{currentstroke}%
\pgfsetdash{}{0pt}%
\pgfpathmoveto{\pgfqpoint{1.351748in}{1.411600in}}%
\pgfpathlineto{\pgfqpoint{1.342827in}{1.429961in}}%
\pgfpathlineto{\pgfqpoint{1.398629in}{1.416427in}}%
\pgfpathlineto{\pgfqpoint{1.405041in}{1.398653in}}%
\pgfpathlineto{\pgfqpoint{1.351748in}{1.411600in}}%
\pgfpathclose%
\pgfusepath{fill}%
\end{pgfscope}%
\begin{pgfscope}%
\pgfpathrectangle{\pgfqpoint{0.000000in}{0.000000in}}{\pgfqpoint{3.000000in}{3.000000in}}%
\pgfusepath{clip}%
\pgfsetbuttcap%
\pgfsetroundjoin%
\definecolor{currentfill}{rgb}{1.000000,0.407407,0.000000}%
\pgfsetfillcolor{currentfill}%
\pgfsetlinewidth{0.000000pt}%
\definecolor{currentstroke}{rgb}{0.000000,0.000000,0.000000}%
\pgfsetstrokecolor{currentstroke}%
\pgfsetdash{}{0pt}%
\pgfpathmoveto{\pgfqpoint{2.103342in}{1.735356in}}%
\pgfpathlineto{\pgfqpoint{2.119951in}{1.748862in}}%
\pgfpathlineto{\pgfqpoint{2.144599in}{1.794286in}}%
\pgfpathlineto{\pgfqpoint{2.127366in}{1.779549in}}%
\pgfpathlineto{\pgfqpoint{2.103342in}{1.735356in}}%
\pgfpathclose%
\pgfusepath{fill}%
\end{pgfscope}%
\begin{pgfscope}%
\pgfpathrectangle{\pgfqpoint{0.000000in}{0.000000in}}{\pgfqpoint{3.000000in}{3.000000in}}%
\pgfusepath{clip}%
\pgfsetbuttcap%
\pgfsetroundjoin%
\definecolor{currentfill}{rgb}{0.667299,1.000000,0.300443}%
\pgfsetfillcolor{currentfill}%
\pgfsetlinewidth{0.000000pt}%
\definecolor{currentstroke}{rgb}{0.000000,0.000000,0.000000}%
\pgfsetstrokecolor{currentstroke}%
\pgfsetdash{}{0pt}%
\pgfpathmoveto{\pgfqpoint{1.292156in}{1.447708in}}%
\pgfpathlineto{\pgfqpoint{1.280943in}{1.465612in}}%
\pgfpathlineto{\pgfqpoint{1.333884in}{1.447098in}}%
\pgfpathlineto{\pgfqpoint{1.342827in}{1.429961in}}%
\pgfpathlineto{\pgfqpoint{1.292156in}{1.447708in}}%
\pgfpathclose%
\pgfusepath{fill}%
\end{pgfscope}%
\begin{pgfscope}%
\pgfpathrectangle{\pgfqpoint{0.000000in}{0.000000in}}{\pgfqpoint{3.000000in}{3.000000in}}%
\pgfusepath{clip}%
\pgfsetbuttcap%
\pgfsetroundjoin%
\definecolor{currentfill}{rgb}{0.819102,1.000000,0.148640}%
\pgfsetfillcolor{currentfill}%
\pgfsetlinewidth{0.000000pt}%
\definecolor{currentstroke}{rgb}{0.000000,0.000000,0.000000}%
\pgfsetstrokecolor{currentstroke}%
\pgfsetdash{}{0pt}%
\pgfpathmoveto{\pgfqpoint{1.828283in}{1.489787in}}%
\pgfpathlineto{\pgfqpoint{1.840238in}{1.505929in}}%
\pgfpathlineto{\pgfqpoint{1.887672in}{1.531445in}}%
\pgfpathlineto{\pgfqpoint{1.873861in}{1.514329in}}%
\pgfpathlineto{\pgfqpoint{1.828283in}{1.489787in}}%
\pgfpathclose%
\pgfusepath{fill}%
\end{pgfscope}%
\begin{pgfscope}%
\pgfpathrectangle{\pgfqpoint{0.000000in}{0.000000in}}{\pgfqpoint{3.000000in}{3.000000in}}%
\pgfusepath{clip}%
\pgfsetbuttcap%
\pgfsetroundjoin%
\definecolor{currentfill}{rgb}{0.490196,1.000000,0.477546}%
\pgfsetfillcolor{currentfill}%
\pgfsetlinewidth{0.000000pt}%
\definecolor{currentstroke}{rgb}{0.000000,0.000000,0.000000}%
\pgfsetstrokecolor{currentstroke}%
\pgfsetdash{}{0pt}%
\pgfpathmoveto{\pgfqpoint{1.521663in}{1.367555in}}%
\pgfpathlineto{\pgfqpoint{1.520726in}{1.386152in}}%
\pgfpathlineto{\pgfqpoint{1.580111in}{1.386943in}}%
\pgfpathlineto{\pgfqpoint{1.578241in}{1.368310in}}%
\pgfpathlineto{\pgfqpoint{1.521663in}{1.367555in}}%
\pgfpathclose%
\pgfusepath{fill}%
\end{pgfscope}%
\begin{pgfscope}%
\pgfpathrectangle{\pgfqpoint{0.000000in}{0.000000in}}{\pgfqpoint{3.000000in}{3.000000in}}%
\pgfusepath{clip}%
\pgfsetbuttcap%
\pgfsetroundjoin%
\definecolor{currentfill}{rgb}{0.856506,0.000000,0.000000}%
\pgfsetfillcolor{currentfill}%
\pgfsetlinewidth{0.000000pt}%
\definecolor{currentstroke}{rgb}{0.000000,0.000000,0.000000}%
\pgfsetstrokecolor{currentstroke}%
\pgfsetdash{}{0pt}%
\pgfpathmoveto{\pgfqpoint{2.265733in}{1.888571in}}%
\pgfpathlineto{\pgfqpoint{2.283109in}{1.901000in}}%
\pgfpathlineto{\pgfqpoint{2.293906in}{1.957424in}}%
\pgfpathlineto{\pgfqpoint{2.276385in}{1.943758in}}%
\pgfpathlineto{\pgfqpoint{2.265733in}{1.888571in}}%
\pgfpathclose%
\pgfusepath{fill}%
\end{pgfscope}%
\begin{pgfscope}%
\pgfpathrectangle{\pgfqpoint{0.000000in}{0.000000in}}{\pgfqpoint{3.000000in}{3.000000in}}%
\pgfusepath{clip}%
\pgfsetbuttcap%
\pgfsetroundjoin%
\definecolor{currentfill}{rgb}{1.000000,0.116921,0.000000}%
\pgfsetfillcolor{currentfill}%
\pgfsetlinewidth{0.000000pt}%
\definecolor{currentstroke}{rgb}{0.000000,0.000000,0.000000}%
\pgfsetstrokecolor{currentstroke}%
\pgfsetdash{}{0pt}%
\pgfpathmoveto{\pgfqpoint{0.858693in}{1.884126in}}%
\pgfpathlineto{\pgfqpoint{0.841215in}{1.898111in}}%
\pgfpathlineto{\pgfqpoint{0.857364in}{1.845632in}}%
\pgfpathlineto{\pgfqpoint{0.874536in}{1.832887in}}%
\pgfpathlineto{\pgfqpoint{0.858693in}{1.884126in}}%
\pgfpathclose%
\pgfusepath{fill}%
\end{pgfscope}%
\begin{pgfscope}%
\pgfpathrectangle{\pgfqpoint{0.000000in}{0.000000in}}{\pgfqpoint{3.000000in}{3.000000in}}%
\pgfusepath{clip}%
\pgfsetbuttcap%
\pgfsetroundjoin%
\definecolor{currentfill}{rgb}{1.000000,0.480029,0.000000}%
\pgfsetfillcolor{currentfill}%
\pgfsetlinewidth{0.000000pt}%
\definecolor{currentstroke}{rgb}{0.000000,0.000000,0.000000}%
\pgfsetstrokecolor{currentstroke}%
\pgfsetdash{}{0pt}%
\pgfpathmoveto{\pgfqpoint{0.977154in}{1.749931in}}%
\pgfpathlineto{\pgfqpoint{0.960100in}{1.764644in}}%
\pgfpathlineto{\pgfqpoint{0.988958in}{1.721057in}}%
\pgfpathlineto{\pgfqpoint{1.005239in}{1.707563in}}%
\pgfpathlineto{\pgfqpoint{0.977154in}{1.749931in}}%
\pgfpathclose%
\pgfusepath{fill}%
\end{pgfscope}%
\begin{pgfscope}%
\pgfpathrectangle{\pgfqpoint{0.000000in}{0.000000in}}{\pgfqpoint{3.000000in}{3.000000in}}%
\pgfusepath{clip}%
\pgfsetbuttcap%
\pgfsetroundjoin%
\definecolor{currentfill}{rgb}{1.000000,0.668845,0.000000}%
\pgfsetfillcolor{currentfill}%
\pgfsetlinewidth{0.000000pt}%
\definecolor{currentstroke}{rgb}{0.000000,0.000000,0.000000}%
\pgfsetstrokecolor{currentstroke}%
\pgfsetdash{}{0pt}%
\pgfpathmoveto{\pgfqpoint{2.003055in}{1.639842in}}%
\pgfpathlineto{\pgfqpoint{2.018506in}{1.653924in}}%
\pgfpathlineto{\pgfqpoint{2.053648in}{1.692374in}}%
\pgfpathlineto{\pgfqpoint{2.037128in}{1.677111in}}%
\pgfpathlineto{\pgfqpoint{2.003055in}{1.639842in}}%
\pgfpathclose%
\pgfusepath{fill}%
\end{pgfscope}%
\begin{pgfscope}%
\pgfpathrectangle{\pgfqpoint{0.000000in}{0.000000in}}{\pgfqpoint{3.000000in}{3.000000in}}%
\pgfusepath{clip}%
\pgfsetbuttcap%
\pgfsetroundjoin%
\definecolor{currentfill}{rgb}{0.578748,1.000000,0.388994}%
\pgfsetfillcolor{currentfill}%
\pgfsetlinewidth{0.000000pt}%
\definecolor{currentstroke}{rgb}{0.000000,0.000000,0.000000}%
\pgfsetstrokecolor{currentstroke}%
\pgfsetdash{}{0pt}%
\pgfpathmoveto{\pgfqpoint{1.638464in}{1.392441in}}%
\pgfpathlineto{\pgfqpoint{1.643101in}{1.409934in}}%
\pgfpathlineto{\pgfqpoint{1.701537in}{1.420445in}}%
\pgfpathlineto{\pgfqpoint{1.694266in}{1.402496in}}%
\pgfpathlineto{\pgfqpoint{1.638464in}{1.392441in}}%
\pgfpathclose%
\pgfusepath{fill}%
\end{pgfscope}%
\begin{pgfscope}%
\pgfpathrectangle{\pgfqpoint{0.000000in}{0.000000in}}{\pgfqpoint{3.000000in}{3.000000in}}%
\pgfusepath{clip}%
\pgfsetbuttcap%
\pgfsetroundjoin%
\definecolor{currentfill}{rgb}{1.000000,0.349310,0.000000}%
\pgfsetfillcolor{currentfill}%
\pgfsetlinewidth{0.000000pt}%
\definecolor{currentstroke}{rgb}{0.000000,0.000000,0.000000}%
\pgfsetstrokecolor{currentstroke}%
\pgfsetdash{}{0pt}%
\pgfpathmoveto{\pgfqpoint{2.119951in}{1.748862in}}%
\pgfpathlineto{\pgfqpoint{2.136583in}{1.762008in}}%
\pgfpathlineto{\pgfqpoint{2.161850in}{1.808662in}}%
\pgfpathlineto{\pgfqpoint{2.144599in}{1.794286in}}%
\pgfpathlineto{\pgfqpoint{2.119951in}{1.748862in}}%
\pgfpathclose%
\pgfusepath{fill}%
\end{pgfscope}%
\begin{pgfscope}%
\pgfpathrectangle{\pgfqpoint{0.000000in}{0.000000in}}{\pgfqpoint{3.000000in}{3.000000in}}%
\pgfusepath{clip}%
\pgfsetbuttcap%
\pgfsetroundjoin%
\definecolor{currentfill}{rgb}{0.958254,0.973856,0.009488}%
\pgfsetfillcolor{currentfill}%
\pgfsetlinewidth{0.000000pt}%
\definecolor{currentstroke}{rgb}{0.000000,0.000000,0.000000}%
\pgfsetstrokecolor{currentstroke}%
\pgfsetdash{}{0pt}%
\pgfpathmoveto{\pgfqpoint{1.151811in}{1.567302in}}%
\pgfpathlineto{\pgfqpoint{1.136913in}{1.583485in}}%
\pgfpathlineto{\pgfqpoint{1.181739in}{1.553569in}}%
\pgfpathlineto{\pgfqpoint{1.195023in}{1.538444in}}%
\pgfpathlineto{\pgfqpoint{1.151811in}{1.567302in}}%
\pgfpathclose%
\pgfusepath{fill}%
\end{pgfscope}%
\begin{pgfscope}%
\pgfpathrectangle{\pgfqpoint{0.000000in}{0.000000in}}{\pgfqpoint{3.000000in}{3.000000in}}%
\pgfusepath{clip}%
\pgfsetbuttcap%
\pgfsetroundjoin%
\definecolor{currentfill}{rgb}{0.803030,0.000000,0.000000}%
\pgfsetfillcolor{currentfill}%
\pgfsetlinewidth{0.000000pt}%
\definecolor{currentstroke}{rgb}{0.000000,0.000000,0.000000}%
\pgfsetstrokecolor{currentstroke}%
\pgfsetdash{}{0pt}%
\pgfpathmoveto{\pgfqpoint{2.283109in}{1.901000in}}%
\pgfpathlineto{\pgfqpoint{2.300503in}{1.913214in}}%
\pgfpathlineto{\pgfqpoint{2.311440in}{1.970873in}}%
\pgfpathlineto{\pgfqpoint{2.293906in}{1.957424in}}%
\pgfpathlineto{\pgfqpoint{2.283109in}{1.901000in}}%
\pgfpathclose%
\pgfusepath{fill}%
\end{pgfscope}%
\begin{pgfscope}%
\pgfpathrectangle{\pgfqpoint{0.000000in}{0.000000in}}{\pgfqpoint{3.000000in}{3.000000in}}%
\pgfusepath{clip}%
\pgfsetbuttcap%
\pgfsetroundjoin%
\definecolor{currentfill}{rgb}{0.819102,1.000000,0.148640}%
\pgfsetfillcolor{currentfill}%
\pgfsetlinewidth{0.000000pt}%
\definecolor{currentstroke}{rgb}{0.000000,0.000000,0.000000}%
\pgfsetstrokecolor{currentstroke}%
\pgfsetdash{}{0pt}%
\pgfpathmoveto{\pgfqpoint{1.221513in}{1.505760in}}%
\pgfpathlineto{\pgfqpoint{1.208281in}{1.522536in}}%
\pgfpathlineto{\pgfqpoint{1.258444in}{1.498281in}}%
\pgfpathlineto{\pgfqpoint{1.269706in}{1.482431in}}%
\pgfpathlineto{\pgfqpoint{1.221513in}{1.505760in}}%
\pgfpathclose%
\pgfusepath{fill}%
\end{pgfscope}%
\begin{pgfscope}%
\pgfpathrectangle{\pgfqpoint{0.000000in}{0.000000in}}{\pgfqpoint{3.000000in}{3.000000in}}%
\pgfusepath{clip}%
\pgfsetbuttcap%
\pgfsetroundjoin%
\definecolor{currentfill}{rgb}{0.743201,1.000000,0.224541}%
\pgfsetfillcolor{currentfill}%
\pgfsetlinewidth{0.000000pt}%
\definecolor{currentstroke}{rgb}{0.000000,0.000000,0.000000}%
\pgfsetstrokecolor{currentstroke}%
\pgfsetdash{}{0pt}%
\pgfpathmoveto{\pgfqpoint{1.765511in}{1.452804in}}%
\pgfpathlineto{\pgfqpoint{1.775263in}{1.469091in}}%
\pgfpathlineto{\pgfqpoint{1.828283in}{1.489787in}}%
\pgfpathlineto{\pgfqpoint{1.816353in}{1.472676in}}%
\pgfpathlineto{\pgfqpoint{1.765511in}{1.452804in}}%
\pgfpathclose%
\pgfusepath{fill}%
\end{pgfscope}%
\begin{pgfscope}%
\pgfpathrectangle{\pgfqpoint{0.000000in}{0.000000in}}{\pgfqpoint{3.000000in}{3.000000in}}%
\pgfusepath{clip}%
\pgfsetbuttcap%
\pgfsetroundjoin%
\definecolor{currentfill}{rgb}{0.999109,0.073348,0.000000}%
\pgfsetfillcolor{currentfill}%
\pgfsetlinewidth{0.000000pt}%
\definecolor{currentstroke}{rgb}{0.000000,0.000000,0.000000}%
\pgfsetstrokecolor{currentstroke}%
\pgfsetdash{}{0pt}%
\pgfpathmoveto{\pgfqpoint{0.841215in}{1.898111in}}%
\pgfpathlineto{\pgfqpoint{0.823722in}{1.911837in}}%
\pgfpathlineto{\pgfqpoint{0.840173in}{1.858119in}}%
\pgfpathlineto{\pgfqpoint{0.857364in}{1.845632in}}%
\pgfpathlineto{\pgfqpoint{0.841215in}{1.898111in}}%
\pgfpathclose%
\pgfusepath{fill}%
\end{pgfscope}%
\begin{pgfscope}%
\pgfpathrectangle{\pgfqpoint{0.000000in}{0.000000in}}{\pgfqpoint{3.000000in}{3.000000in}}%
\pgfusepath{clip}%
\pgfsetbuttcap%
\pgfsetroundjoin%
\definecolor{currentfill}{rgb}{1.000000,0.741467,0.000000}%
\pgfsetfillcolor{currentfill}%
\pgfsetlinewidth{0.000000pt}%
\definecolor{currentstroke}{rgb}{0.000000,0.000000,0.000000}%
\pgfsetstrokecolor{currentstroke}%
\pgfsetdash{}{0pt}%
\pgfpathmoveto{\pgfqpoint{1.070128in}{1.649017in}}%
\pgfpathlineto{\pgfqpoint{1.053941in}{1.664417in}}%
\pgfpathlineto{\pgfqpoint{1.092064in}{1.628033in}}%
\pgfpathlineto{\pgfqpoint{1.107040in}{1.613790in}}%
\pgfpathlineto{\pgfqpoint{1.070128in}{1.649017in}}%
\pgfpathclose%
\pgfusepath{fill}%
\end{pgfscope}%
\begin{pgfscope}%
\pgfpathrectangle{\pgfqpoint{0.000000in}{0.000000in}}{\pgfqpoint{3.000000in}{3.000000in}}%
\pgfusepath{clip}%
\pgfsetbuttcap%
\pgfsetroundjoin%
\definecolor{currentfill}{rgb}{1.000000,0.886710,0.000000}%
\pgfsetfillcolor{currentfill}%
\pgfsetlinewidth{0.000000pt}%
\definecolor{currentstroke}{rgb}{0.000000,0.000000,0.000000}%
\pgfsetstrokecolor{currentstroke}%
\pgfsetdash{}{0pt}%
\pgfpathmoveto{\pgfqpoint{1.915374in}{1.563156in}}%
\pgfpathlineto{\pgfqpoint{1.929264in}{1.577913in}}%
\pgfpathlineto{\pgfqpoint{1.972230in}{1.610015in}}%
\pgfpathlineto{\pgfqpoint{1.956856in}{1.594165in}}%
\pgfpathlineto{\pgfqpoint{1.915374in}{1.563156in}}%
\pgfpathclose%
\pgfusepath{fill}%
\end{pgfscope}%
\begin{pgfscope}%
\pgfpathrectangle{\pgfqpoint{0.000000in}{0.000000in}}{\pgfqpoint{3.000000in}{3.000000in}}%
\pgfusepath{clip}%
\pgfsetbuttcap%
\pgfsetroundjoin%
\definecolor{currentfill}{rgb}{0.578748,1.000000,0.388994}%
\pgfsetfillcolor{currentfill}%
\pgfsetlinewidth{0.000000pt}%
\definecolor{currentstroke}{rgb}{0.000000,0.000000,0.000000}%
\pgfsetstrokecolor{currentstroke}%
\pgfsetdash{}{0pt}%
\pgfpathmoveto{\pgfqpoint{1.405041in}{1.398653in}}%
\pgfpathlineto{\pgfqpoint{1.398629in}{1.416427in}}%
\pgfpathlineto{\pgfqpoint{1.458131in}{1.407478in}}%
\pgfpathlineto{\pgfqpoint{1.461858in}{1.390092in}}%
\pgfpathlineto{\pgfqpoint{1.405041in}{1.398653in}}%
\pgfpathclose%
\pgfusepath{fill}%
\end{pgfscope}%
\begin{pgfscope}%
\pgfpathrectangle{\pgfqpoint{0.000000in}{0.000000in}}{\pgfqpoint{3.000000in}{3.000000in}}%
\pgfusepath{clip}%
\pgfsetbuttcap%
\pgfsetroundjoin%
\definecolor{currentfill}{rgb}{0.667299,1.000000,0.300443}%
\pgfsetfillcolor{currentfill}%
\pgfsetlinewidth{0.000000pt}%
\definecolor{currentstroke}{rgb}{0.000000,0.000000,0.000000}%
\pgfsetstrokecolor{currentstroke}%
\pgfsetdash{}{0pt}%
\pgfpathmoveto{\pgfqpoint{1.701537in}{1.420445in}}%
\pgfpathlineto{\pgfqpoint{1.708827in}{1.437169in}}%
\pgfpathlineto{\pgfqpoint{1.765511in}{1.452804in}}%
\pgfpathlineto{\pgfqpoint{1.755782in}{1.435430in}}%
\pgfpathlineto{\pgfqpoint{1.701537in}{1.420445in}}%
\pgfpathclose%
\pgfusepath{fill}%
\end{pgfscope}%
\begin{pgfscope}%
\pgfpathrectangle{\pgfqpoint{0.000000in}{0.000000in}}{\pgfqpoint{3.000000in}{3.000000in}}%
\pgfusepath{clip}%
\pgfsetbuttcap%
\pgfsetroundjoin%
\definecolor{currentfill}{rgb}{0.731729,0.000000,0.000000}%
\pgfsetfillcolor{currentfill}%
\pgfsetlinewidth{0.000000pt}%
\definecolor{currentstroke}{rgb}{0.000000,0.000000,0.000000}%
\pgfsetstrokecolor{currentstroke}%
\pgfsetdash{}{0pt}%
\pgfpathmoveto{\pgfqpoint{2.300503in}{1.913214in}}%
\pgfpathlineto{\pgfqpoint{2.317916in}{1.925226in}}%
\pgfpathlineto{\pgfqpoint{2.328987in}{1.984118in}}%
\pgfpathlineto{\pgfqpoint{2.311440in}{1.970873in}}%
\pgfpathlineto{\pgfqpoint{2.300503in}{1.913214in}}%
\pgfpathclose%
\pgfusepath{fill}%
\end{pgfscope}%
\begin{pgfscope}%
\pgfpathrectangle{\pgfqpoint{0.000000in}{0.000000in}}{\pgfqpoint{3.000000in}{3.000000in}}%
\pgfusepath{clip}%
\pgfsetbuttcap%
\pgfsetroundjoin%
\definecolor{currentfill}{rgb}{0.895003,1.000000,0.072739}%
\pgfsetfillcolor{currentfill}%
\pgfsetlinewidth{0.000000pt}%
\definecolor{currentstroke}{rgb}{0.000000,0.000000,0.000000}%
\pgfsetstrokecolor{currentstroke}%
\pgfsetdash{}{0pt}%
\pgfpathmoveto{\pgfqpoint{1.840238in}{1.505929in}}%
\pgfpathlineto{\pgfqpoint{1.852219in}{1.521203in}}%
\pgfpathlineto{\pgfqpoint{1.901510in}{1.547692in}}%
\pgfpathlineto{\pgfqpoint{1.887672in}{1.531445in}}%
\pgfpathlineto{\pgfqpoint{1.840238in}{1.505929in}}%
\pgfpathclose%
\pgfusepath{fill}%
\end{pgfscope}%
\begin{pgfscope}%
\pgfpathrectangle{\pgfqpoint{0.000000in}{0.000000in}}{\pgfqpoint{3.000000in}{3.000000in}}%
\pgfusepath{clip}%
\pgfsetbuttcap%
\pgfsetroundjoin%
\definecolor{currentfill}{rgb}{1.000000,0.407407,0.000000}%
\pgfsetfillcolor{currentfill}%
\pgfsetlinewidth{0.000000pt}%
\definecolor{currentstroke}{rgb}{0.000000,0.000000,0.000000}%
\pgfsetstrokecolor{currentstroke}%
\pgfsetdash{}{0pt}%
\pgfpathmoveto{\pgfqpoint{0.960100in}{1.764644in}}%
\pgfpathlineto{\pgfqpoint{0.943027in}{1.778967in}}%
\pgfpathlineto{\pgfqpoint{0.972654in}{1.734163in}}%
\pgfpathlineto{\pgfqpoint{0.988958in}{1.721057in}}%
\pgfpathlineto{\pgfqpoint{0.960100in}{1.764644in}}%
\pgfpathclose%
\pgfusepath{fill}%
\end{pgfscope}%
\begin{pgfscope}%
\pgfpathrectangle{\pgfqpoint{0.000000in}{0.000000in}}{\pgfqpoint{3.000000in}{3.000000in}}%
\pgfusepath{clip}%
\pgfsetbuttcap%
\pgfsetroundjoin%
\definecolor{currentfill}{rgb}{0.927807,0.015251,0.000000}%
\pgfsetfillcolor{currentfill}%
\pgfsetlinewidth{0.000000pt}%
\definecolor{currentstroke}{rgb}{0.000000,0.000000,0.000000}%
\pgfsetstrokecolor{currentstroke}%
\pgfsetdash{}{0pt}%
\pgfpathmoveto{\pgfqpoint{0.823722in}{1.911837in}}%
\pgfpathlineto{\pgfqpoint{0.806214in}{1.925319in}}%
\pgfpathlineto{\pgfqpoint{0.822963in}{1.870364in}}%
\pgfpathlineto{\pgfqpoint{0.840173in}{1.858119in}}%
\pgfpathlineto{\pgfqpoint{0.823722in}{1.911837in}}%
\pgfpathclose%
\pgfusepath{fill}%
\end{pgfscope}%
\begin{pgfscope}%
\pgfpathrectangle{\pgfqpoint{0.000000in}{0.000000in}}{\pgfqpoint{3.000000in}{3.000000in}}%
\pgfusepath{clip}%
\pgfsetbuttcap%
\pgfsetroundjoin%
\definecolor{currentfill}{rgb}{0.743201,1.000000,0.224541}%
\pgfsetfillcolor{currentfill}%
\pgfsetlinewidth{0.000000pt}%
\definecolor{currentstroke}{rgb}{0.000000,0.000000,0.000000}%
\pgfsetstrokecolor{currentstroke}%
\pgfsetdash{}{0pt}%
\pgfpathmoveto{\pgfqpoint{1.280943in}{1.465612in}}%
\pgfpathlineto{\pgfqpoint{1.269706in}{1.482431in}}%
\pgfpathlineto{\pgfqpoint{1.324920in}{1.463148in}}%
\pgfpathlineto{\pgfqpoint{1.333884in}{1.447098in}}%
\pgfpathlineto{\pgfqpoint{1.280943in}{1.465612in}}%
\pgfpathclose%
\pgfusepath{fill}%
\end{pgfscope}%
\begin{pgfscope}%
\pgfpathrectangle{\pgfqpoint{0.000000in}{0.000000in}}{\pgfqpoint{3.000000in}{3.000000in}}%
\pgfusepath{clip}%
\pgfsetbuttcap%
\pgfsetroundjoin%
\definecolor{currentfill}{rgb}{0.578748,1.000000,0.388994}%
\pgfsetfillcolor{currentfill}%
\pgfsetlinewidth{0.000000pt}%
\definecolor{currentstroke}{rgb}{0.000000,0.000000,0.000000}%
\pgfsetstrokecolor{currentstroke}%
\pgfsetdash{}{0pt}%
\pgfpathmoveto{\pgfqpoint{1.580111in}{1.386943in}}%
\pgfpathlineto{\pgfqpoint{1.581986in}{1.404185in}}%
\pgfpathlineto{\pgfqpoint{1.643101in}{1.409934in}}%
\pgfpathlineto{\pgfqpoint{1.638464in}{1.392441in}}%
\pgfpathlineto{\pgfqpoint{1.580111in}{1.386943in}}%
\pgfpathclose%
\pgfusepath{fill}%
\end{pgfscope}%
\begin{pgfscope}%
\pgfpathrectangle{\pgfqpoint{0.000000in}{0.000000in}}{\pgfqpoint{3.000000in}{3.000000in}}%
\pgfusepath{clip}%
\pgfsetbuttcap%
\pgfsetroundjoin%
\definecolor{currentfill}{rgb}{1.000000,0.291213,0.000000}%
\pgfsetfillcolor{currentfill}%
\pgfsetlinewidth{0.000000pt}%
\definecolor{currentstroke}{rgb}{0.000000,0.000000,0.000000}%
\pgfsetstrokecolor{currentstroke}%
\pgfsetdash{}{0pt}%
\pgfpathmoveto{\pgfqpoint{2.136583in}{1.762008in}}%
\pgfpathlineto{\pgfqpoint{2.153237in}{1.774817in}}%
\pgfpathlineto{\pgfqpoint{2.179118in}{1.822701in}}%
\pgfpathlineto{\pgfqpoint{2.161850in}{1.808662in}}%
\pgfpathlineto{\pgfqpoint{2.136583in}{1.762008in}}%
\pgfpathclose%
\pgfusepath{fill}%
\end{pgfscope}%
\begin{pgfscope}%
\pgfpathrectangle{\pgfqpoint{0.000000in}{0.000000in}}{\pgfqpoint{3.000000in}{3.000000in}}%
\pgfusepath{clip}%
\pgfsetbuttcap%
\pgfsetroundjoin%
\definecolor{currentfill}{rgb}{0.667299,1.000000,0.300443}%
\pgfsetfillcolor{currentfill}%
\pgfsetlinewidth{0.000000pt}%
\definecolor{currentstroke}{rgb}{0.000000,0.000000,0.000000}%
\pgfsetstrokecolor{currentstroke}%
\pgfsetdash{}{0pt}%
\pgfpathmoveto{\pgfqpoint{1.342827in}{1.429961in}}%
\pgfpathlineto{\pgfqpoint{1.333884in}{1.447098in}}%
\pgfpathlineto{\pgfqpoint{1.392201in}{1.432977in}}%
\pgfpathlineto{\pgfqpoint{1.398629in}{1.416427in}}%
\pgfpathlineto{\pgfqpoint{1.342827in}{1.429961in}}%
\pgfpathclose%
\pgfusepath{fill}%
\end{pgfscope}%
\begin{pgfscope}%
\pgfpathrectangle{\pgfqpoint{0.000000in}{0.000000in}}{\pgfqpoint{3.000000in}{3.000000in}}%
\pgfusepath{clip}%
\pgfsetbuttcap%
\pgfsetroundjoin%
\definecolor{currentfill}{rgb}{1.000000,0.610748,0.000000}%
\pgfsetfillcolor{currentfill}%
\pgfsetlinewidth{0.000000pt}%
\definecolor{currentstroke}{rgb}{0.000000,0.000000,0.000000}%
\pgfsetstrokecolor{currentstroke}%
\pgfsetdash{}{0pt}%
\pgfpathmoveto{\pgfqpoint{2.018506in}{1.653924in}}%
\pgfpathlineto{\pgfqpoint{2.033982in}{1.667513in}}%
\pgfpathlineto{\pgfqpoint{2.070190in}{1.707145in}}%
\pgfpathlineto{\pgfqpoint{2.053648in}{1.692374in}}%
\pgfpathlineto{\pgfqpoint{2.018506in}{1.653924in}}%
\pgfpathclose%
\pgfusepath{fill}%
\end{pgfscope}%
\begin{pgfscope}%
\pgfpathrectangle{\pgfqpoint{0.000000in}{0.000000in}}{\pgfqpoint{3.000000in}{3.000000in}}%
\pgfusepath{clip}%
\pgfsetbuttcap%
\pgfsetroundjoin%
\definecolor{currentfill}{rgb}{0.578748,1.000000,0.388994}%
\pgfsetfillcolor{currentfill}%
\pgfsetlinewidth{0.000000pt}%
\definecolor{currentstroke}{rgb}{0.000000,0.000000,0.000000}%
\pgfsetstrokecolor{currentstroke}%
\pgfsetdash{}{0pt}%
\pgfpathmoveto{\pgfqpoint{1.461858in}{1.390092in}}%
\pgfpathlineto{\pgfqpoint{1.458131in}{1.407478in}}%
\pgfpathlineto{\pgfqpoint{1.519787in}{1.403359in}}%
\pgfpathlineto{\pgfqpoint{1.520726in}{1.386152in}}%
\pgfpathlineto{\pgfqpoint{1.461858in}{1.390092in}}%
\pgfpathclose%
\pgfusepath{fill}%
\end{pgfscope}%
\begin{pgfscope}%
\pgfpathrectangle{\pgfqpoint{0.000000in}{0.000000in}}{\pgfqpoint{3.000000in}{3.000000in}}%
\pgfusepath{clip}%
\pgfsetbuttcap%
\pgfsetroundjoin%
\definecolor{currentfill}{rgb}{0.578748,1.000000,0.388994}%
\pgfsetfillcolor{currentfill}%
\pgfsetlinewidth{0.000000pt}%
\definecolor{currentstroke}{rgb}{0.000000,0.000000,0.000000}%
\pgfsetstrokecolor{currentstroke}%
\pgfsetdash{}{0pt}%
\pgfpathmoveto{\pgfqpoint{1.520726in}{1.386152in}}%
\pgfpathlineto{\pgfqpoint{1.519787in}{1.403359in}}%
\pgfpathlineto{\pgfqpoint{1.581986in}{1.404185in}}%
\pgfpathlineto{\pgfqpoint{1.580111in}{1.386943in}}%
\pgfpathlineto{\pgfqpoint{1.520726in}{1.386152in}}%
\pgfpathclose%
\pgfusepath{fill}%
\end{pgfscope}%
\begin{pgfscope}%
\pgfpathrectangle{\pgfqpoint{0.000000in}{0.000000in}}{\pgfqpoint{3.000000in}{3.000000in}}%
\pgfusepath{clip}%
\pgfsetbuttcap%
\pgfsetroundjoin%
\definecolor{currentfill}{rgb}{1.000000,0.668845,0.000000}%
\pgfsetfillcolor{currentfill}%
\pgfsetlinewidth{0.000000pt}%
\definecolor{currentstroke}{rgb}{0.000000,0.000000,0.000000}%
\pgfsetstrokecolor{currentstroke}%
\pgfsetdash{}{0pt}%
\pgfpathmoveto{\pgfqpoint{1.053941in}{1.664417in}}%
\pgfpathlineto{\pgfqpoint{1.037731in}{1.679279in}}%
\pgfpathlineto{\pgfqpoint{1.077063in}{1.641739in}}%
\pgfpathlineto{\pgfqpoint{1.092064in}{1.628033in}}%
\pgfpathlineto{\pgfqpoint{1.053941in}{1.664417in}}%
\pgfpathclose%
\pgfusepath{fill}%
\end{pgfscope}%
\begin{pgfscope}%
\pgfpathrectangle{\pgfqpoint{0.000000in}{0.000000in}}{\pgfqpoint{3.000000in}{3.000000in}}%
\pgfusepath{clip}%
\pgfsetbuttcap%
\pgfsetroundjoin%
\definecolor{currentfill}{rgb}{0.678253,0.000000,0.000000}%
\pgfsetfillcolor{currentfill}%
\pgfsetlinewidth{0.000000pt}%
\definecolor{currentstroke}{rgb}{0.000000,0.000000,0.000000}%
\pgfsetstrokecolor{currentstroke}%
\pgfsetdash{}{0pt}%
\pgfpathmoveto{\pgfqpoint{2.317916in}{1.925226in}}%
\pgfpathlineto{\pgfqpoint{2.335345in}{1.937046in}}%
\pgfpathlineto{\pgfqpoint{2.346547in}{1.997171in}}%
\pgfpathlineto{\pgfqpoint{2.328987in}{1.984118in}}%
\pgfpathlineto{\pgfqpoint{2.317916in}{1.925226in}}%
\pgfpathclose%
\pgfusepath{fill}%
\end{pgfscope}%
\begin{pgfscope}%
\pgfpathrectangle{\pgfqpoint{0.000000in}{0.000000in}}{\pgfqpoint{3.000000in}{3.000000in}}%
\pgfusepath{clip}%
\pgfsetbuttcap%
\pgfsetroundjoin%
\definecolor{currentfill}{rgb}{1.000000,0.886710,0.000000}%
\pgfsetfillcolor{currentfill}%
\pgfsetlinewidth{0.000000pt}%
\definecolor{currentstroke}{rgb}{0.000000,0.000000,0.000000}%
\pgfsetstrokecolor{currentstroke}%
\pgfsetdash{}{0pt}%
\pgfpathmoveto{\pgfqpoint{1.136913in}{1.583485in}}%
\pgfpathlineto{\pgfqpoint{1.121989in}{1.598959in}}%
\pgfpathlineto{\pgfqpoint{1.168428in}{1.567985in}}%
\pgfpathlineto{\pgfqpoint{1.181739in}{1.553569in}}%
\pgfpathlineto{\pgfqpoint{1.136913in}{1.583485in}}%
\pgfpathclose%
\pgfusepath{fill}%
\end{pgfscope}%
\begin{pgfscope}%
\pgfpathrectangle{\pgfqpoint{0.000000in}{0.000000in}}{\pgfqpoint{3.000000in}{3.000000in}}%
\pgfusepath{clip}%
\pgfsetbuttcap%
\pgfsetroundjoin%
\definecolor{currentfill}{rgb}{1.000000,0.814089,0.000000}%
\pgfsetfillcolor{currentfill}%
\pgfsetlinewidth{0.000000pt}%
\definecolor{currentstroke}{rgb}{0.000000,0.000000,0.000000}%
\pgfsetstrokecolor{currentstroke}%
\pgfsetdash{}{0pt}%
\pgfpathmoveto{\pgfqpoint{1.929264in}{1.577913in}}%
\pgfpathlineto{\pgfqpoint{1.943181in}{1.592025in}}%
\pgfpathlineto{\pgfqpoint{1.987630in}{1.625222in}}%
\pgfpathlineto{\pgfqpoint{1.972230in}{1.610015in}}%
\pgfpathlineto{\pgfqpoint{1.929264in}{1.577913in}}%
\pgfpathclose%
\pgfusepath{fill}%
\end{pgfscope}%
\begin{pgfscope}%
\pgfpathrectangle{\pgfqpoint{0.000000in}{0.000000in}}{\pgfqpoint{3.000000in}{3.000000in}}%
\pgfusepath{clip}%
\pgfsetbuttcap%
\pgfsetroundjoin%
\definecolor{currentfill}{rgb}{0.856506,0.000000,0.000000}%
\pgfsetfillcolor{currentfill}%
\pgfsetlinewidth{0.000000pt}%
\definecolor{currentstroke}{rgb}{0.000000,0.000000,0.000000}%
\pgfsetstrokecolor{currentstroke}%
\pgfsetdash{}{0pt}%
\pgfpathmoveto{\pgfqpoint{0.806214in}{1.925319in}}%
\pgfpathlineto{\pgfqpoint{0.788693in}{1.938574in}}%
\pgfpathlineto{\pgfqpoint{0.805734in}{1.882382in}}%
\pgfpathlineto{\pgfqpoint{0.822963in}{1.870364in}}%
\pgfpathlineto{\pgfqpoint{0.806214in}{1.925319in}}%
\pgfpathclose%
\pgfusepath{fill}%
\end{pgfscope}%
\begin{pgfscope}%
\pgfpathrectangle{\pgfqpoint{0.000000in}{0.000000in}}{\pgfqpoint{3.000000in}{3.000000in}}%
\pgfusepath{clip}%
\pgfsetbuttcap%
\pgfsetroundjoin%
\definecolor{currentfill}{rgb}{0.895003,1.000000,0.072739}%
\pgfsetfillcolor{currentfill}%
\pgfsetlinewidth{0.000000pt}%
\definecolor{currentstroke}{rgb}{0.000000,0.000000,0.000000}%
\pgfsetstrokecolor{currentstroke}%
\pgfsetdash{}{0pt}%
\pgfpathmoveto{\pgfqpoint{1.208281in}{1.522536in}}%
\pgfpathlineto{\pgfqpoint{1.195023in}{1.538444in}}%
\pgfpathlineto{\pgfqpoint{1.247158in}{1.513261in}}%
\pgfpathlineto{\pgfqpoint{1.258444in}{1.498281in}}%
\pgfpathlineto{\pgfqpoint{1.208281in}{1.522536in}}%
\pgfpathclose%
\pgfusepath{fill}%
\end{pgfscope}%
\begin{pgfscope}%
\pgfpathrectangle{\pgfqpoint{0.000000in}{0.000000in}}{\pgfqpoint{3.000000in}{3.000000in}}%
\pgfusepath{clip}%
\pgfsetbuttcap%
\pgfsetroundjoin%
\definecolor{currentfill}{rgb}{1.000000,0.233115,0.000000}%
\pgfsetfillcolor{currentfill}%
\pgfsetlinewidth{0.000000pt}%
\definecolor{currentstroke}{rgb}{0.000000,0.000000,0.000000}%
\pgfsetstrokecolor{currentstroke}%
\pgfsetdash{}{0pt}%
\pgfpathmoveto{\pgfqpoint{2.153237in}{1.774817in}}%
\pgfpathlineto{\pgfqpoint{2.169913in}{1.787313in}}%
\pgfpathlineto{\pgfqpoint{2.196405in}{1.836426in}}%
\pgfpathlineto{\pgfqpoint{2.179118in}{1.822701in}}%
\pgfpathlineto{\pgfqpoint{2.153237in}{1.774817in}}%
\pgfpathclose%
\pgfusepath{fill}%
\end{pgfscope}%
\begin{pgfscope}%
\pgfpathrectangle{\pgfqpoint{0.000000in}{0.000000in}}{\pgfqpoint{3.000000in}{3.000000in}}%
\pgfusepath{clip}%
\pgfsetbuttcap%
\pgfsetroundjoin%
\definecolor{currentfill}{rgb}{1.000000,0.349310,0.000000}%
\pgfsetfillcolor{currentfill}%
\pgfsetlinewidth{0.000000pt}%
\definecolor{currentstroke}{rgb}{0.000000,0.000000,0.000000}%
\pgfsetstrokecolor{currentstroke}%
\pgfsetdash{}{0pt}%
\pgfpathmoveto{\pgfqpoint{0.943027in}{1.778967in}}%
\pgfpathlineto{\pgfqpoint{0.925933in}{1.792929in}}%
\pgfpathlineto{\pgfqpoint{0.956326in}{1.746908in}}%
\pgfpathlineto{\pgfqpoint{0.972654in}{1.734163in}}%
\pgfpathlineto{\pgfqpoint{0.943027in}{1.778967in}}%
\pgfpathclose%
\pgfusepath{fill}%
\end{pgfscope}%
\begin{pgfscope}%
\pgfpathrectangle{\pgfqpoint{0.000000in}{0.000000in}}{\pgfqpoint{3.000000in}{3.000000in}}%
\pgfusepath{clip}%
\pgfsetbuttcap%
\pgfsetroundjoin%
\definecolor{currentfill}{rgb}{0.819102,1.000000,0.148640}%
\pgfsetfillcolor{currentfill}%
\pgfsetlinewidth{0.000000pt}%
\definecolor{currentstroke}{rgb}{0.000000,0.000000,0.000000}%
\pgfsetstrokecolor{currentstroke}%
\pgfsetdash{}{0pt}%
\pgfpathmoveto{\pgfqpoint{1.775263in}{1.469091in}}%
\pgfpathlineto{\pgfqpoint{1.785038in}{1.484409in}}%
\pgfpathlineto{\pgfqpoint{1.840238in}{1.505929in}}%
\pgfpathlineto{\pgfqpoint{1.828283in}{1.489787in}}%
\pgfpathlineto{\pgfqpoint{1.775263in}{1.469091in}}%
\pgfpathclose%
\pgfusepath{fill}%
\end{pgfscope}%
\begin{pgfscope}%
\pgfpathrectangle{\pgfqpoint{0.000000in}{0.000000in}}{\pgfqpoint{3.000000in}{3.000000in}}%
\pgfusepath{clip}%
\pgfsetbuttcap%
\pgfsetroundjoin%
\definecolor{currentfill}{rgb}{0.667299,1.000000,0.300443}%
\pgfsetfillcolor{currentfill}%
\pgfsetlinewidth{0.000000pt}%
\definecolor{currentstroke}{rgb}{0.000000,0.000000,0.000000}%
\pgfsetstrokecolor{currentstroke}%
\pgfsetdash{}{0pt}%
\pgfpathmoveto{\pgfqpoint{1.643101in}{1.409934in}}%
\pgfpathlineto{\pgfqpoint{1.647750in}{1.426201in}}%
\pgfpathlineto{\pgfqpoint{1.708827in}{1.437169in}}%
\pgfpathlineto{\pgfqpoint{1.701537in}{1.420445in}}%
\pgfpathlineto{\pgfqpoint{1.643101in}{1.409934in}}%
\pgfpathclose%
\pgfusepath{fill}%
\end{pgfscope}%
\begin{pgfscope}%
\pgfpathrectangle{\pgfqpoint{0.000000in}{0.000000in}}{\pgfqpoint{3.000000in}{3.000000in}}%
\pgfusepath{clip}%
\pgfsetbuttcap%
\pgfsetroundjoin%
\definecolor{currentfill}{rgb}{0.606952,0.000000,0.000000}%
\pgfsetfillcolor{currentfill}%
\pgfsetlinewidth{0.000000pt}%
\definecolor{currentstroke}{rgb}{0.000000,0.000000,0.000000}%
\pgfsetstrokecolor{currentstroke}%
\pgfsetdash{}{0pt}%
\pgfpathmoveto{\pgfqpoint{2.335345in}{1.937046in}}%
\pgfpathlineto{\pgfqpoint{2.352793in}{1.948685in}}%
\pgfpathlineto{\pgfqpoint{2.364119in}{2.010040in}}%
\pgfpathlineto{\pgfqpoint{2.346547in}{1.997171in}}%
\pgfpathlineto{\pgfqpoint{2.335345in}{1.937046in}}%
\pgfpathclose%
\pgfusepath{fill}%
\end{pgfscope}%
\begin{pgfscope}%
\pgfpathrectangle{\pgfqpoint{0.000000in}{0.000000in}}{\pgfqpoint{3.000000in}{3.000000in}}%
\pgfusepath{clip}%
\pgfsetbuttcap%
\pgfsetroundjoin%
\definecolor{currentfill}{rgb}{1.000000,0.538126,0.000000}%
\pgfsetfillcolor{currentfill}%
\pgfsetlinewidth{0.000000pt}%
\definecolor{currentstroke}{rgb}{0.000000,0.000000,0.000000}%
\pgfsetstrokecolor{currentstroke}%
\pgfsetdash{}{0pt}%
\pgfpathmoveto{\pgfqpoint{2.033982in}{1.667513in}}%
\pgfpathlineto{\pgfqpoint{2.049482in}{1.680647in}}%
\pgfpathlineto{\pgfqpoint{2.086755in}{1.721460in}}%
\pgfpathlineto{\pgfqpoint{2.070190in}{1.707145in}}%
\pgfpathlineto{\pgfqpoint{2.033982in}{1.667513in}}%
\pgfpathclose%
\pgfusepath{fill}%
\end{pgfscope}%
\begin{pgfscope}%
\pgfpathrectangle{\pgfqpoint{0.000000in}{0.000000in}}{\pgfqpoint{3.000000in}{3.000000in}}%
\pgfusepath{clip}%
\pgfsetbuttcap%
\pgfsetroundjoin%
\definecolor{currentfill}{rgb}{0.958254,0.973856,0.009488}%
\pgfsetfillcolor{currentfill}%
\pgfsetlinewidth{0.000000pt}%
\definecolor{currentstroke}{rgb}{0.000000,0.000000,0.000000}%
\pgfsetstrokecolor{currentstroke}%
\pgfsetdash{}{0pt}%
\pgfpathmoveto{\pgfqpoint{1.852219in}{1.521203in}}%
\pgfpathlineto{\pgfqpoint{1.864225in}{1.535692in}}%
\pgfpathlineto{\pgfqpoint{1.915374in}{1.563156in}}%
\pgfpathlineto{\pgfqpoint{1.901510in}{1.547692in}}%
\pgfpathlineto{\pgfqpoint{1.852219in}{1.521203in}}%
\pgfpathclose%
\pgfusepath{fill}%
\end{pgfscope}%
\begin{pgfscope}%
\pgfpathrectangle{\pgfqpoint{0.000000in}{0.000000in}}{\pgfqpoint{3.000000in}{3.000000in}}%
\pgfusepath{clip}%
\pgfsetbuttcap%
\pgfsetroundjoin%
\definecolor{currentfill}{rgb}{0.743201,1.000000,0.224541}%
\pgfsetfillcolor{currentfill}%
\pgfsetlinewidth{0.000000pt}%
\definecolor{currentstroke}{rgb}{0.000000,0.000000,0.000000}%
\pgfsetstrokecolor{currentstroke}%
\pgfsetdash{}{0pt}%
\pgfpathmoveto{\pgfqpoint{1.708827in}{1.437169in}}%
\pgfpathlineto{\pgfqpoint{1.716134in}{1.452806in}}%
\pgfpathlineto{\pgfqpoint{1.775263in}{1.469091in}}%
\pgfpathlineto{\pgfqpoint{1.765511in}{1.452804in}}%
\pgfpathlineto{\pgfqpoint{1.708827in}{1.437169in}}%
\pgfpathclose%
\pgfusepath{fill}%
\end{pgfscope}%
\begin{pgfscope}%
\pgfpathrectangle{\pgfqpoint{0.000000in}{0.000000in}}{\pgfqpoint{3.000000in}{3.000000in}}%
\pgfusepath{clip}%
\pgfsetbuttcap%
\pgfsetroundjoin%
\definecolor{currentfill}{rgb}{0.667299,1.000000,0.300443}%
\pgfsetfillcolor{currentfill}%
\pgfsetlinewidth{0.000000pt}%
\definecolor{currentstroke}{rgb}{0.000000,0.000000,0.000000}%
\pgfsetstrokecolor{currentstroke}%
\pgfsetdash{}{0pt}%
\pgfpathmoveto{\pgfqpoint{1.398629in}{1.416427in}}%
\pgfpathlineto{\pgfqpoint{1.392201in}{1.432977in}}%
\pgfpathlineto{\pgfqpoint{1.454395in}{1.423638in}}%
\pgfpathlineto{\pgfqpoint{1.458131in}{1.407478in}}%
\pgfpathlineto{\pgfqpoint{1.398629in}{1.416427in}}%
\pgfpathclose%
\pgfusepath{fill}%
\end{pgfscope}%
\begin{pgfscope}%
\pgfpathrectangle{\pgfqpoint{0.000000in}{0.000000in}}{\pgfqpoint{3.000000in}{3.000000in}}%
\pgfusepath{clip}%
\pgfsetbuttcap%
\pgfsetroundjoin%
\definecolor{currentfill}{rgb}{0.803030,0.000000,0.000000}%
\pgfsetfillcolor{currentfill}%
\pgfsetlinewidth{0.000000pt}%
\definecolor{currentstroke}{rgb}{0.000000,0.000000,0.000000}%
\pgfsetstrokecolor{currentstroke}%
\pgfsetdash{}{0pt}%
\pgfpathmoveto{\pgfqpoint{0.788693in}{1.938574in}}%
\pgfpathlineto{\pgfqpoint{0.771156in}{1.951612in}}%
\pgfpathlineto{\pgfqpoint{0.788485in}{1.894185in}}%
\pgfpathlineto{\pgfqpoint{0.805734in}{1.882382in}}%
\pgfpathlineto{\pgfqpoint{0.788693in}{1.938574in}}%
\pgfpathclose%
\pgfusepath{fill}%
\end{pgfscope}%
\begin{pgfscope}%
\pgfpathrectangle{\pgfqpoint{0.000000in}{0.000000in}}{\pgfqpoint{3.000000in}{3.000000in}}%
\pgfusepath{clip}%
\pgfsetbuttcap%
\pgfsetroundjoin%
\definecolor{currentfill}{rgb}{1.000000,0.610748,0.000000}%
\pgfsetfillcolor{currentfill}%
\pgfsetlinewidth{0.000000pt}%
\definecolor{currentstroke}{rgb}{0.000000,0.000000,0.000000}%
\pgfsetstrokecolor{currentstroke}%
\pgfsetdash{}{0pt}%
\pgfpathmoveto{\pgfqpoint{1.037731in}{1.679279in}}%
\pgfpathlineto{\pgfqpoint{1.021497in}{1.693648in}}%
\pgfpathlineto{\pgfqpoint{1.062036in}{1.654952in}}%
\pgfpathlineto{\pgfqpoint{1.077063in}{1.641739in}}%
\pgfpathlineto{\pgfqpoint{1.037731in}{1.679279in}}%
\pgfpathclose%
\pgfusepath{fill}%
\end{pgfscope}%
\begin{pgfscope}%
\pgfpathrectangle{\pgfqpoint{0.000000in}{0.000000in}}{\pgfqpoint{3.000000in}{3.000000in}}%
\pgfusepath{clip}%
\pgfsetbuttcap%
\pgfsetroundjoin%
\definecolor{currentfill}{rgb}{0.819102,1.000000,0.148640}%
\pgfsetfillcolor{currentfill}%
\pgfsetlinewidth{0.000000pt}%
\definecolor{currentstroke}{rgb}{0.000000,0.000000,0.000000}%
\pgfsetstrokecolor{currentstroke}%
\pgfsetdash{}{0pt}%
\pgfpathmoveto{\pgfqpoint{1.269706in}{1.482431in}}%
\pgfpathlineto{\pgfqpoint{1.258444in}{1.498281in}}%
\pgfpathlineto{\pgfqpoint{1.315935in}{1.478228in}}%
\pgfpathlineto{\pgfqpoint{1.324920in}{1.463148in}}%
\pgfpathlineto{\pgfqpoint{1.269706in}{1.482431in}}%
\pgfpathclose%
\pgfusepath{fill}%
\end{pgfscope}%
\begin{pgfscope}%
\pgfpathrectangle{\pgfqpoint{0.000000in}{0.000000in}}{\pgfqpoint{3.000000in}{3.000000in}}%
\pgfusepath{clip}%
\pgfsetbuttcap%
\pgfsetroundjoin%
\definecolor{currentfill}{rgb}{0.553476,0.000000,0.000000}%
\pgfsetfillcolor{currentfill}%
\pgfsetlinewidth{0.000000pt}%
\definecolor{currentstroke}{rgb}{0.000000,0.000000,0.000000}%
\pgfsetstrokecolor{currentstroke}%
\pgfsetdash{}{0pt}%
\pgfpathmoveto{\pgfqpoint{2.352793in}{1.948685in}}%
\pgfpathlineto{\pgfqpoint{2.370258in}{1.960151in}}%
\pgfpathlineto{\pgfqpoint{2.381703in}{2.022737in}}%
\pgfpathlineto{\pgfqpoint{2.364119in}{2.010040in}}%
\pgfpathlineto{\pgfqpoint{2.352793in}{1.948685in}}%
\pgfpathclose%
\pgfusepath{fill}%
\end{pgfscope}%
\begin{pgfscope}%
\pgfpathrectangle{\pgfqpoint{0.000000in}{0.000000in}}{\pgfqpoint{3.000000in}{3.000000in}}%
\pgfusepath{clip}%
\pgfsetbuttcap%
\pgfsetroundjoin%
\definecolor{currentfill}{rgb}{1.000000,0.175018,0.000000}%
\pgfsetfillcolor{currentfill}%
\pgfsetlinewidth{0.000000pt}%
\definecolor{currentstroke}{rgb}{0.000000,0.000000,0.000000}%
\pgfsetstrokecolor{currentstroke}%
\pgfsetdash{}{0pt}%
\pgfpathmoveto{\pgfqpoint{2.169913in}{1.787313in}}%
\pgfpathlineto{\pgfqpoint{2.186611in}{1.799517in}}%
\pgfpathlineto{\pgfqpoint{2.213710in}{1.849857in}}%
\pgfpathlineto{\pgfqpoint{2.196405in}{1.836426in}}%
\pgfpathlineto{\pgfqpoint{2.169913in}{1.787313in}}%
\pgfpathclose%
\pgfusepath{fill}%
\end{pgfscope}%
\begin{pgfscope}%
\pgfpathrectangle{\pgfqpoint{0.000000in}{0.000000in}}{\pgfqpoint{3.000000in}{3.000000in}}%
\pgfusepath{clip}%
\pgfsetbuttcap%
\pgfsetroundjoin%
\definecolor{currentfill}{rgb}{1.000000,0.291213,0.000000}%
\pgfsetfillcolor{currentfill}%
\pgfsetlinewidth{0.000000pt}%
\definecolor{currentstroke}{rgb}{0.000000,0.000000,0.000000}%
\pgfsetstrokecolor{currentstroke}%
\pgfsetdash{}{0pt}%
\pgfpathmoveto{\pgfqpoint{0.925933in}{1.792929in}}%
\pgfpathlineto{\pgfqpoint{0.908820in}{1.806555in}}%
\pgfpathlineto{\pgfqpoint{0.939975in}{1.759316in}}%
\pgfpathlineto{\pgfqpoint{0.956326in}{1.746908in}}%
\pgfpathlineto{\pgfqpoint{0.925933in}{1.792929in}}%
\pgfpathclose%
\pgfusepath{fill}%
\end{pgfscope}%
\begin{pgfscope}%
\pgfpathrectangle{\pgfqpoint{0.000000in}{0.000000in}}{\pgfqpoint{3.000000in}{3.000000in}}%
\pgfusepath{clip}%
\pgfsetbuttcap%
\pgfsetroundjoin%
\definecolor{currentfill}{rgb}{1.000000,0.814089,0.000000}%
\pgfsetfillcolor{currentfill}%
\pgfsetlinewidth{0.000000pt}%
\definecolor{currentstroke}{rgb}{0.000000,0.000000,0.000000}%
\pgfsetstrokecolor{currentstroke}%
\pgfsetdash{}{0pt}%
\pgfpathmoveto{\pgfqpoint{1.121989in}{1.598959in}}%
\pgfpathlineto{\pgfqpoint{1.107040in}{1.613790in}}%
\pgfpathlineto{\pgfqpoint{1.155091in}{1.581757in}}%
\pgfpathlineto{\pgfqpoint{1.168428in}{1.567985in}}%
\pgfpathlineto{\pgfqpoint{1.121989in}{1.598959in}}%
\pgfpathclose%
\pgfusepath{fill}%
\end{pgfscope}%
\begin{pgfscope}%
\pgfpathrectangle{\pgfqpoint{0.000000in}{0.000000in}}{\pgfqpoint{3.000000in}{3.000000in}}%
\pgfusepath{clip}%
\pgfsetbuttcap%
\pgfsetroundjoin%
\definecolor{currentfill}{rgb}{0.743201,1.000000,0.224541}%
\pgfsetfillcolor{currentfill}%
\pgfsetlinewidth{0.000000pt}%
\definecolor{currentstroke}{rgb}{0.000000,0.000000,0.000000}%
\pgfsetstrokecolor{currentstroke}%
\pgfsetdash{}{0pt}%
\pgfpathmoveto{\pgfqpoint{1.333884in}{1.447098in}}%
\pgfpathlineto{\pgfqpoint{1.324920in}{1.463148in}}%
\pgfpathlineto{\pgfqpoint{1.385757in}{1.448438in}}%
\pgfpathlineto{\pgfqpoint{1.392201in}{1.432977in}}%
\pgfpathlineto{\pgfqpoint{1.333884in}{1.447098in}}%
\pgfpathclose%
\pgfusepath{fill}%
\end{pgfscope}%
\begin{pgfscope}%
\pgfpathrectangle{\pgfqpoint{0.000000in}{0.000000in}}{\pgfqpoint{3.000000in}{3.000000in}}%
\pgfusepath{clip}%
\pgfsetbuttcap%
\pgfsetroundjoin%
\definecolor{currentfill}{rgb}{0.667299,1.000000,0.300443}%
\pgfsetfillcolor{currentfill}%
\pgfsetlinewidth{0.000000pt}%
\definecolor{currentstroke}{rgb}{0.000000,0.000000,0.000000}%
\pgfsetstrokecolor{currentstroke}%
\pgfsetdash{}{0pt}%
\pgfpathmoveto{\pgfqpoint{1.581986in}{1.404185in}}%
\pgfpathlineto{\pgfqpoint{1.583866in}{1.420202in}}%
\pgfpathlineto{\pgfqpoint{1.647750in}{1.426201in}}%
\pgfpathlineto{\pgfqpoint{1.643101in}{1.409934in}}%
\pgfpathlineto{\pgfqpoint{1.581986in}{1.404185in}}%
\pgfpathclose%
\pgfusepath{fill}%
\end{pgfscope}%
\begin{pgfscope}%
\pgfpathrectangle{\pgfqpoint{0.000000in}{0.000000in}}{\pgfqpoint{3.000000in}{3.000000in}}%
\pgfusepath{clip}%
\pgfsetbuttcap%
\pgfsetroundjoin%
\definecolor{currentfill}{rgb}{1.000000,0.741467,0.000000}%
\pgfsetfillcolor{currentfill}%
\pgfsetlinewidth{0.000000pt}%
\definecolor{currentstroke}{rgb}{0.000000,0.000000,0.000000}%
\pgfsetstrokecolor{currentstroke}%
\pgfsetdash{}{0pt}%
\pgfpathmoveto{\pgfqpoint{1.943181in}{1.592025in}}%
\pgfpathlineto{\pgfqpoint{1.957124in}{1.605549in}}%
\pgfpathlineto{\pgfqpoint{2.003055in}{1.639842in}}%
\pgfpathlineto{\pgfqpoint{1.987630in}{1.625222in}}%
\pgfpathlineto{\pgfqpoint{1.943181in}{1.592025in}}%
\pgfpathclose%
\pgfusepath{fill}%
\end{pgfscope}%
\begin{pgfscope}%
\pgfpathrectangle{\pgfqpoint{0.000000in}{0.000000in}}{\pgfqpoint{3.000000in}{3.000000in}}%
\pgfusepath{clip}%
\pgfsetbuttcap%
\pgfsetroundjoin%
\definecolor{currentfill}{rgb}{0.731729,0.000000,0.000000}%
\pgfsetfillcolor{currentfill}%
\pgfsetlinewidth{0.000000pt}%
\definecolor{currentstroke}{rgb}{0.000000,0.000000,0.000000}%
\pgfsetstrokecolor{currentstroke}%
\pgfsetdash{}{0pt}%
\pgfpathmoveto{\pgfqpoint{0.771156in}{1.951612in}}%
\pgfpathlineto{\pgfqpoint{0.753605in}{1.964448in}}%
\pgfpathlineto{\pgfqpoint{0.771216in}{1.905786in}}%
\pgfpathlineto{\pgfqpoint{0.788485in}{1.894185in}}%
\pgfpathlineto{\pgfqpoint{0.771156in}{1.951612in}}%
\pgfpathclose%
\pgfusepath{fill}%
\end{pgfscope}%
\begin{pgfscope}%
\pgfpathrectangle{\pgfqpoint{0.000000in}{0.000000in}}{\pgfqpoint{3.000000in}{3.000000in}}%
\pgfusepath{clip}%
\pgfsetbuttcap%
\pgfsetroundjoin%
\definecolor{currentfill}{rgb}{0.667299,1.000000,0.300443}%
\pgfsetfillcolor{currentfill}%
\pgfsetlinewidth{0.000000pt}%
\definecolor{currentstroke}{rgb}{0.000000,0.000000,0.000000}%
\pgfsetstrokecolor{currentstroke}%
\pgfsetdash{}{0pt}%
\pgfpathmoveto{\pgfqpoint{1.458131in}{1.407478in}}%
\pgfpathlineto{\pgfqpoint{1.454395in}{1.423638in}}%
\pgfpathlineto{\pgfqpoint{1.518846in}{1.419340in}}%
\pgfpathlineto{\pgfqpoint{1.519787in}{1.403359in}}%
\pgfpathlineto{\pgfqpoint{1.458131in}{1.407478in}}%
\pgfpathclose%
\pgfusepath{fill}%
\end{pgfscope}%
\begin{pgfscope}%
\pgfpathrectangle{\pgfqpoint{0.000000in}{0.000000in}}{\pgfqpoint{3.000000in}{3.000000in}}%
\pgfusepath{clip}%
\pgfsetbuttcap%
\pgfsetroundjoin%
\definecolor{currentfill}{rgb}{1.000000,0.480029,0.000000}%
\pgfsetfillcolor{currentfill}%
\pgfsetlinewidth{0.000000pt}%
\definecolor{currentstroke}{rgb}{0.000000,0.000000,0.000000}%
\pgfsetstrokecolor{currentstroke}%
\pgfsetdash{}{0pt}%
\pgfpathmoveto{\pgfqpoint{2.049482in}{1.680647in}}%
\pgfpathlineto{\pgfqpoint{2.065008in}{1.693362in}}%
\pgfpathlineto{\pgfqpoint{2.103342in}{1.735356in}}%
\pgfpathlineto{\pgfqpoint{2.086755in}{1.721460in}}%
\pgfpathlineto{\pgfqpoint{2.049482in}{1.680647in}}%
\pgfpathclose%
\pgfusepath{fill}%
\end{pgfscope}%
\begin{pgfscope}%
\pgfpathrectangle{\pgfqpoint{0.000000in}{0.000000in}}{\pgfqpoint{3.000000in}{3.000000in}}%
\pgfusepath{clip}%
\pgfsetbuttcap%
\pgfsetroundjoin%
\definecolor{currentfill}{rgb}{0.958254,0.973856,0.009488}%
\pgfsetfillcolor{currentfill}%
\pgfsetlinewidth{0.000000pt}%
\definecolor{currentstroke}{rgb}{0.000000,0.000000,0.000000}%
\pgfsetstrokecolor{currentstroke}%
\pgfsetdash{}{0pt}%
\pgfpathmoveto{\pgfqpoint{1.195023in}{1.538444in}}%
\pgfpathlineto{\pgfqpoint{1.181739in}{1.553569in}}%
\pgfpathlineto{\pgfqpoint{1.235846in}{1.527457in}}%
\pgfpathlineto{\pgfqpoint{1.247158in}{1.513261in}}%
\pgfpathlineto{\pgfqpoint{1.195023in}{1.538444in}}%
\pgfpathclose%
\pgfusepath{fill}%
\end{pgfscope}%
\begin{pgfscope}%
\pgfpathrectangle{\pgfqpoint{0.000000in}{0.000000in}}{\pgfqpoint{3.000000in}{3.000000in}}%
\pgfusepath{clip}%
\pgfsetbuttcap%
\pgfsetroundjoin%
\definecolor{currentfill}{rgb}{0.667299,1.000000,0.300443}%
\pgfsetfillcolor{currentfill}%
\pgfsetlinewidth{0.000000pt}%
\definecolor{currentstroke}{rgb}{0.000000,0.000000,0.000000}%
\pgfsetstrokecolor{currentstroke}%
\pgfsetdash{}{0pt}%
\pgfpathmoveto{\pgfqpoint{1.519787in}{1.403359in}}%
\pgfpathlineto{\pgfqpoint{1.518846in}{1.419340in}}%
\pgfpathlineto{\pgfqpoint{1.583866in}{1.420202in}}%
\pgfpathlineto{\pgfqpoint{1.581986in}{1.404185in}}%
\pgfpathlineto{\pgfqpoint{1.519787in}{1.403359in}}%
\pgfpathclose%
\pgfusepath{fill}%
\end{pgfscope}%
\begin{pgfscope}%
\pgfpathrectangle{\pgfqpoint{0.000000in}{0.000000in}}{\pgfqpoint{3.000000in}{3.000000in}}%
\pgfusepath{clip}%
\pgfsetbuttcap%
\pgfsetroundjoin%
\definecolor{currentfill}{rgb}{0.500000,0.000000,0.000000}%
\pgfsetfillcolor{currentfill}%
\pgfsetlinewidth{0.000000pt}%
\definecolor{currentstroke}{rgb}{0.000000,0.000000,0.000000}%
\pgfsetstrokecolor{currentstroke}%
\pgfsetdash{}{0pt}%
\pgfpathmoveto{\pgfqpoint{2.370258in}{1.960151in}}%
\pgfpathlineto{\pgfqpoint{2.387740in}{1.971455in}}%
\pgfpathlineto{\pgfqpoint{2.399301in}{2.035269in}}%
\pgfpathlineto{\pgfqpoint{2.381703in}{2.022737in}}%
\pgfpathlineto{\pgfqpoint{2.370258in}{1.960151in}}%
\pgfpathclose%
\pgfusepath{fill}%
\end{pgfscope}%
\begin{pgfscope}%
\pgfpathrectangle{\pgfqpoint{0.000000in}{0.000000in}}{\pgfqpoint{3.000000in}{3.000000in}}%
\pgfusepath{clip}%
\pgfsetbuttcap%
\pgfsetroundjoin%
\definecolor{currentfill}{rgb}{0.895003,1.000000,0.072739}%
\pgfsetfillcolor{currentfill}%
\pgfsetlinewidth{0.000000pt}%
\definecolor{currentstroke}{rgb}{0.000000,0.000000,0.000000}%
\pgfsetstrokecolor{currentstroke}%
\pgfsetdash{}{0pt}%
\pgfpathmoveto{\pgfqpoint{1.785038in}{1.484409in}}%
\pgfpathlineto{\pgfqpoint{1.794835in}{1.498856in}}%
\pgfpathlineto{\pgfqpoint{1.852219in}{1.521203in}}%
\pgfpathlineto{\pgfqpoint{1.840238in}{1.505929in}}%
\pgfpathlineto{\pgfqpoint{1.785038in}{1.484409in}}%
\pgfpathclose%
\pgfusepath{fill}%
\end{pgfscope}%
\begin{pgfscope}%
\pgfpathrectangle{\pgfqpoint{0.000000in}{0.000000in}}{\pgfqpoint{3.000000in}{3.000000in}}%
\pgfusepath{clip}%
\pgfsetbuttcap%
\pgfsetroundjoin%
\definecolor{currentfill}{rgb}{1.000000,0.116921,0.000000}%
\pgfsetfillcolor{currentfill}%
\pgfsetlinewidth{0.000000pt}%
\definecolor{currentstroke}{rgb}{0.000000,0.000000,0.000000}%
\pgfsetstrokecolor{currentstroke}%
\pgfsetdash{}{0pt}%
\pgfpathmoveto{\pgfqpoint{2.186611in}{1.799517in}}%
\pgfpathlineto{\pgfqpoint{2.203332in}{1.811446in}}%
\pgfpathlineto{\pgfqpoint{2.231033in}{1.863014in}}%
\pgfpathlineto{\pgfqpoint{2.213710in}{1.849857in}}%
\pgfpathlineto{\pgfqpoint{2.186611in}{1.799517in}}%
\pgfpathclose%
\pgfusepath{fill}%
\end{pgfscope}%
\begin{pgfscope}%
\pgfpathrectangle{\pgfqpoint{0.000000in}{0.000000in}}{\pgfqpoint{3.000000in}{3.000000in}}%
\pgfusepath{clip}%
\pgfsetbuttcap%
\pgfsetroundjoin%
\definecolor{currentfill}{rgb}{1.000000,0.538126,0.000000}%
\pgfsetfillcolor{currentfill}%
\pgfsetlinewidth{0.000000pt}%
\definecolor{currentstroke}{rgb}{0.000000,0.000000,0.000000}%
\pgfsetstrokecolor{currentstroke}%
\pgfsetdash{}{0pt}%
\pgfpathmoveto{\pgfqpoint{1.021497in}{1.693648in}}%
\pgfpathlineto{\pgfqpoint{1.005239in}{1.707563in}}%
\pgfpathlineto{\pgfqpoint{1.046983in}{1.667709in}}%
\pgfpathlineto{\pgfqpoint{1.062036in}{1.654952in}}%
\pgfpathlineto{\pgfqpoint{1.021497in}{1.693648in}}%
\pgfpathclose%
\pgfusepath{fill}%
\end{pgfscope}%
\begin{pgfscope}%
\pgfpathrectangle{\pgfqpoint{0.000000in}{0.000000in}}{\pgfqpoint{3.000000in}{3.000000in}}%
\pgfusepath{clip}%
\pgfsetbuttcap%
\pgfsetroundjoin%
\definecolor{currentfill}{rgb}{1.000000,0.886710,0.000000}%
\pgfsetfillcolor{currentfill}%
\pgfsetlinewidth{0.000000pt}%
\definecolor{currentstroke}{rgb}{0.000000,0.000000,0.000000}%
\pgfsetstrokecolor{currentstroke}%
\pgfsetdash{}{0pt}%
\pgfpathmoveto{\pgfqpoint{1.864225in}{1.535692in}}%
\pgfpathlineto{\pgfqpoint{1.876257in}{1.549473in}}%
\pgfpathlineto{\pgfqpoint{1.929264in}{1.577913in}}%
\pgfpathlineto{\pgfqpoint{1.915374in}{1.563156in}}%
\pgfpathlineto{\pgfqpoint{1.864225in}{1.535692in}}%
\pgfpathclose%
\pgfusepath{fill}%
\end{pgfscope}%
\begin{pgfscope}%
\pgfpathrectangle{\pgfqpoint{0.000000in}{0.000000in}}{\pgfqpoint{3.000000in}{3.000000in}}%
\pgfusepath{clip}%
\pgfsetbuttcap%
\pgfsetroundjoin%
\definecolor{currentfill}{rgb}{1.000000,0.233115,0.000000}%
\pgfsetfillcolor{currentfill}%
\pgfsetlinewidth{0.000000pt}%
\definecolor{currentstroke}{rgb}{0.000000,0.000000,0.000000}%
\pgfsetstrokecolor{currentstroke}%
\pgfsetdash{}{0pt}%
\pgfpathmoveto{\pgfqpoint{0.908820in}{1.806555in}}%
\pgfpathlineto{\pgfqpoint{0.891688in}{1.819868in}}%
\pgfpathlineto{\pgfqpoint{0.923600in}{1.771412in}}%
\pgfpathlineto{\pgfqpoint{0.939975in}{1.759316in}}%
\pgfpathlineto{\pgfqpoint{0.908820in}{1.806555in}}%
\pgfpathclose%
\pgfusepath{fill}%
\end{pgfscope}%
\begin{pgfscope}%
\pgfpathrectangle{\pgfqpoint{0.000000in}{0.000000in}}{\pgfqpoint{3.000000in}{3.000000in}}%
\pgfusepath{clip}%
\pgfsetbuttcap%
\pgfsetroundjoin%
\definecolor{currentfill}{rgb}{0.678253,0.000000,0.000000}%
\pgfsetfillcolor{currentfill}%
\pgfsetlinewidth{0.000000pt}%
\definecolor{currentstroke}{rgb}{0.000000,0.000000,0.000000}%
\pgfsetstrokecolor{currentstroke}%
\pgfsetdash{}{0pt}%
\pgfpathmoveto{\pgfqpoint{0.753605in}{1.964448in}}%
\pgfpathlineto{\pgfqpoint{0.736040in}{1.977090in}}%
\pgfpathlineto{\pgfqpoint{0.753929in}{1.917196in}}%
\pgfpathlineto{\pgfqpoint{0.771216in}{1.905786in}}%
\pgfpathlineto{\pgfqpoint{0.753605in}{1.964448in}}%
\pgfpathclose%
\pgfusepath{fill}%
\end{pgfscope}%
\begin{pgfscope}%
\pgfpathrectangle{\pgfqpoint{0.000000in}{0.000000in}}{\pgfqpoint{3.000000in}{3.000000in}}%
\pgfusepath{clip}%
\pgfsetbuttcap%
\pgfsetroundjoin%
\definecolor{currentfill}{rgb}{0.743201,1.000000,0.224541}%
\pgfsetfillcolor{currentfill}%
\pgfsetlinewidth{0.000000pt}%
\definecolor{currentstroke}{rgb}{0.000000,0.000000,0.000000}%
\pgfsetstrokecolor{currentstroke}%
\pgfsetdash{}{0pt}%
\pgfpathmoveto{\pgfqpoint{1.647750in}{1.426201in}}%
\pgfpathlineto{\pgfqpoint{1.652411in}{1.441379in}}%
\pgfpathlineto{\pgfqpoint{1.716134in}{1.452806in}}%
\pgfpathlineto{\pgfqpoint{1.708827in}{1.437169in}}%
\pgfpathlineto{\pgfqpoint{1.647750in}{1.426201in}}%
\pgfpathclose%
\pgfusepath{fill}%
\end{pgfscope}%
\begin{pgfscope}%
\pgfpathrectangle{\pgfqpoint{0.000000in}{0.000000in}}{\pgfqpoint{3.000000in}{3.000000in}}%
\pgfusepath{clip}%
\pgfsetbuttcap%
\pgfsetroundjoin%
\definecolor{currentfill}{rgb}{0.819102,1.000000,0.148640}%
\pgfsetfillcolor{currentfill}%
\pgfsetlinewidth{0.000000pt}%
\definecolor{currentstroke}{rgb}{0.000000,0.000000,0.000000}%
\pgfsetstrokecolor{currentstroke}%
\pgfsetdash{}{0pt}%
\pgfpathmoveto{\pgfqpoint{1.716134in}{1.452806in}}%
\pgfpathlineto{\pgfqpoint{1.723460in}{1.467471in}}%
\pgfpathlineto{\pgfqpoint{1.785038in}{1.484409in}}%
\pgfpathlineto{\pgfqpoint{1.775263in}{1.469091in}}%
\pgfpathlineto{\pgfqpoint{1.716134in}{1.452806in}}%
\pgfpathclose%
\pgfusepath{fill}%
\end{pgfscope}%
\begin{pgfscope}%
\pgfpathrectangle{\pgfqpoint{0.000000in}{0.000000in}}{\pgfqpoint{3.000000in}{3.000000in}}%
\pgfusepath{clip}%
\pgfsetbuttcap%
\pgfsetroundjoin%
\definecolor{currentfill}{rgb}{1.000000,0.668845,0.000000}%
\pgfsetfillcolor{currentfill}%
\pgfsetlinewidth{0.000000pt}%
\definecolor{currentstroke}{rgb}{0.000000,0.000000,0.000000}%
\pgfsetstrokecolor{currentstroke}%
\pgfsetdash{}{0pt}%
\pgfpathmoveto{\pgfqpoint{1.957124in}{1.605549in}}%
\pgfpathlineto{\pgfqpoint{1.971094in}{1.618535in}}%
\pgfpathlineto{\pgfqpoint{2.018506in}{1.653924in}}%
\pgfpathlineto{\pgfqpoint{2.003055in}{1.639842in}}%
\pgfpathlineto{\pgfqpoint{1.957124in}{1.605549in}}%
\pgfpathclose%
\pgfusepath{fill}%
\end{pgfscope}%
\begin{pgfscope}%
\pgfpathrectangle{\pgfqpoint{0.000000in}{0.000000in}}{\pgfqpoint{3.000000in}{3.000000in}}%
\pgfusepath{clip}%
\pgfsetbuttcap%
\pgfsetroundjoin%
\definecolor{currentfill}{rgb}{1.000000,0.741467,0.000000}%
\pgfsetfillcolor{currentfill}%
\pgfsetlinewidth{0.000000pt}%
\definecolor{currentstroke}{rgb}{0.000000,0.000000,0.000000}%
\pgfsetstrokecolor{currentstroke}%
\pgfsetdash{}{0pt}%
\pgfpathmoveto{\pgfqpoint{1.107040in}{1.613790in}}%
\pgfpathlineto{\pgfqpoint{1.092064in}{1.628033in}}%
\pgfpathlineto{\pgfqpoint{1.141728in}{1.594941in}}%
\pgfpathlineto{\pgfqpoint{1.155091in}{1.581757in}}%
\pgfpathlineto{\pgfqpoint{1.107040in}{1.613790in}}%
\pgfpathclose%
\pgfusepath{fill}%
\end{pgfscope}%
\begin{pgfscope}%
\pgfpathrectangle{\pgfqpoint{0.000000in}{0.000000in}}{\pgfqpoint{3.000000in}{3.000000in}}%
\pgfusepath{clip}%
\pgfsetbuttcap%
\pgfsetroundjoin%
\definecolor{currentfill}{rgb}{0.743201,1.000000,0.224541}%
\pgfsetfillcolor{currentfill}%
\pgfsetlinewidth{0.000000pt}%
\definecolor{currentstroke}{rgb}{0.000000,0.000000,0.000000}%
\pgfsetstrokecolor{currentstroke}%
\pgfsetdash{}{0pt}%
\pgfpathmoveto{\pgfqpoint{1.392201in}{1.432977in}}%
\pgfpathlineto{\pgfqpoint{1.385757in}{1.448438in}}%
\pgfpathlineto{\pgfqpoint{1.450648in}{1.438709in}}%
\pgfpathlineto{\pgfqpoint{1.454395in}{1.423638in}}%
\pgfpathlineto{\pgfqpoint{1.392201in}{1.432977in}}%
\pgfpathclose%
\pgfusepath{fill}%
\end{pgfscope}%
\begin{pgfscope}%
\pgfpathrectangle{\pgfqpoint{0.000000in}{0.000000in}}{\pgfqpoint{3.000000in}{3.000000in}}%
\pgfusepath{clip}%
\pgfsetbuttcap%
\pgfsetroundjoin%
\definecolor{currentfill}{rgb}{1.000000,0.407407,0.000000}%
\pgfsetfillcolor{currentfill}%
\pgfsetlinewidth{0.000000pt}%
\definecolor{currentstroke}{rgb}{0.000000,0.000000,0.000000}%
\pgfsetstrokecolor{currentstroke}%
\pgfsetdash{}{0pt}%
\pgfpathmoveto{\pgfqpoint{2.065008in}{1.693362in}}%
\pgfpathlineto{\pgfqpoint{2.080560in}{1.705687in}}%
\pgfpathlineto{\pgfqpoint{2.119951in}{1.748862in}}%
\pgfpathlineto{\pgfqpoint{2.103342in}{1.735356in}}%
\pgfpathlineto{\pgfqpoint{2.065008in}{1.693362in}}%
\pgfpathclose%
\pgfusepath{fill}%
\end{pgfscope}%
\begin{pgfscope}%
\pgfpathrectangle{\pgfqpoint{0.000000in}{0.000000in}}{\pgfqpoint{3.000000in}{3.000000in}}%
\pgfusepath{clip}%
\pgfsetbuttcap%
\pgfsetroundjoin%
\definecolor{currentfill}{rgb}{0.895003,1.000000,0.072739}%
\pgfsetfillcolor{currentfill}%
\pgfsetlinewidth{0.000000pt}%
\definecolor{currentstroke}{rgb}{0.000000,0.000000,0.000000}%
\pgfsetstrokecolor{currentstroke}%
\pgfsetdash{}{0pt}%
\pgfpathmoveto{\pgfqpoint{1.258444in}{1.498281in}}%
\pgfpathlineto{\pgfqpoint{1.247158in}{1.513261in}}%
\pgfpathlineto{\pgfqpoint{1.306929in}{1.492437in}}%
\pgfpathlineto{\pgfqpoint{1.315935in}{1.478228in}}%
\pgfpathlineto{\pgfqpoint{1.258444in}{1.498281in}}%
\pgfpathclose%
\pgfusepath{fill}%
\end{pgfscope}%
\begin{pgfscope}%
\pgfpathrectangle{\pgfqpoint{0.000000in}{0.000000in}}{\pgfqpoint{3.000000in}{3.000000in}}%
\pgfusepath{clip}%
\pgfsetbuttcap%
\pgfsetroundjoin%
\definecolor{currentfill}{rgb}{0.999109,0.073348,0.000000}%
\pgfsetfillcolor{currentfill}%
\pgfsetlinewidth{0.000000pt}%
\definecolor{currentstroke}{rgb}{0.000000,0.000000,0.000000}%
\pgfsetstrokecolor{currentstroke}%
\pgfsetdash{}{0pt}%
\pgfpathmoveto{\pgfqpoint{2.203332in}{1.811446in}}%
\pgfpathlineto{\pgfqpoint{2.220074in}{1.823118in}}%
\pgfpathlineto{\pgfqpoint{2.248374in}{1.875914in}}%
\pgfpathlineto{\pgfqpoint{2.231033in}{1.863014in}}%
\pgfpathlineto{\pgfqpoint{2.203332in}{1.811446in}}%
\pgfpathclose%
\pgfusepath{fill}%
\end{pgfscope}%
\begin{pgfscope}%
\pgfpathrectangle{\pgfqpoint{0.000000in}{0.000000in}}{\pgfqpoint{3.000000in}{3.000000in}}%
\pgfusepath{clip}%
\pgfsetbuttcap%
\pgfsetroundjoin%
\definecolor{currentfill}{rgb}{0.606952,0.000000,0.000000}%
\pgfsetfillcolor{currentfill}%
\pgfsetlinewidth{0.000000pt}%
\definecolor{currentstroke}{rgb}{0.000000,0.000000,0.000000}%
\pgfsetstrokecolor{currentstroke}%
\pgfsetdash{}{0pt}%
\pgfpathmoveto{\pgfqpoint{0.736040in}{1.977090in}}%
\pgfpathlineto{\pgfqpoint{0.718460in}{1.989551in}}%
\pgfpathlineto{\pgfqpoint{0.736622in}{1.928424in}}%
\pgfpathlineto{\pgfqpoint{0.753929in}{1.917196in}}%
\pgfpathlineto{\pgfqpoint{0.736040in}{1.977090in}}%
\pgfpathclose%
\pgfusepath{fill}%
\end{pgfscope}%
\begin{pgfscope}%
\pgfpathrectangle{\pgfqpoint{0.000000in}{0.000000in}}{\pgfqpoint{3.000000in}{3.000000in}}%
\pgfusepath{clip}%
\pgfsetbuttcap%
\pgfsetroundjoin%
\definecolor{currentfill}{rgb}{0.819102,1.000000,0.148640}%
\pgfsetfillcolor{currentfill}%
\pgfsetlinewidth{0.000000pt}%
\definecolor{currentstroke}{rgb}{0.000000,0.000000,0.000000}%
\pgfsetstrokecolor{currentstroke}%
\pgfsetdash{}{0pt}%
\pgfpathmoveto{\pgfqpoint{1.324920in}{1.463148in}}%
\pgfpathlineto{\pgfqpoint{1.315935in}{1.478228in}}%
\pgfpathlineto{\pgfqpoint{1.379296in}{1.462929in}}%
\pgfpathlineto{\pgfqpoint{1.385757in}{1.448438in}}%
\pgfpathlineto{\pgfqpoint{1.324920in}{1.463148in}}%
\pgfpathclose%
\pgfusepath{fill}%
\end{pgfscope}%
\begin{pgfscope}%
\pgfpathrectangle{\pgfqpoint{0.000000in}{0.000000in}}{\pgfqpoint{3.000000in}{3.000000in}}%
\pgfusepath{clip}%
\pgfsetbuttcap%
\pgfsetroundjoin%
\definecolor{currentfill}{rgb}{1.000000,0.175018,0.000000}%
\pgfsetfillcolor{currentfill}%
\pgfsetlinewidth{0.000000pt}%
\definecolor{currentstroke}{rgb}{0.000000,0.000000,0.000000}%
\pgfsetstrokecolor{currentstroke}%
\pgfsetdash{}{0pt}%
\pgfpathmoveto{\pgfqpoint{0.891688in}{1.819868in}}%
\pgfpathlineto{\pgfqpoint{0.874536in}{1.832887in}}%
\pgfpathlineto{\pgfqpoint{0.907202in}{1.783214in}}%
\pgfpathlineto{\pgfqpoint{0.923600in}{1.771412in}}%
\pgfpathlineto{\pgfqpoint{0.891688in}{1.819868in}}%
\pgfpathclose%
\pgfusepath{fill}%
\end{pgfscope}%
\begin{pgfscope}%
\pgfpathrectangle{\pgfqpoint{0.000000in}{0.000000in}}{\pgfqpoint{3.000000in}{3.000000in}}%
\pgfusepath{clip}%
\pgfsetbuttcap%
\pgfsetroundjoin%
\definecolor{currentfill}{rgb}{1.000000,0.886710,0.000000}%
\pgfsetfillcolor{currentfill}%
\pgfsetlinewidth{0.000000pt}%
\definecolor{currentstroke}{rgb}{0.000000,0.000000,0.000000}%
\pgfsetstrokecolor{currentstroke}%
\pgfsetdash{}{0pt}%
\pgfpathmoveto{\pgfqpoint{1.181739in}{1.553569in}}%
\pgfpathlineto{\pgfqpoint{1.168428in}{1.567985in}}%
\pgfpathlineto{\pgfqpoint{1.224510in}{1.540943in}}%
\pgfpathlineto{\pgfqpoint{1.235846in}{1.527457in}}%
\pgfpathlineto{\pgfqpoint{1.181739in}{1.553569in}}%
\pgfpathclose%
\pgfusepath{fill}%
\end{pgfscope}%
\begin{pgfscope}%
\pgfpathrectangle{\pgfqpoint{0.000000in}{0.000000in}}{\pgfqpoint{3.000000in}{3.000000in}}%
\pgfusepath{clip}%
\pgfsetbuttcap%
\pgfsetroundjoin%
\definecolor{currentfill}{rgb}{0.743201,1.000000,0.224541}%
\pgfsetfillcolor{currentfill}%
\pgfsetlinewidth{0.000000pt}%
\definecolor{currentstroke}{rgb}{0.000000,0.000000,0.000000}%
\pgfsetstrokecolor{currentstroke}%
\pgfsetdash{}{0pt}%
\pgfpathmoveto{\pgfqpoint{1.583866in}{1.420202in}}%
\pgfpathlineto{\pgfqpoint{1.585751in}{1.435129in}}%
\pgfpathlineto{\pgfqpoint{1.652411in}{1.441379in}}%
\pgfpathlineto{\pgfqpoint{1.647750in}{1.426201in}}%
\pgfpathlineto{\pgfqpoint{1.583866in}{1.420202in}}%
\pgfpathclose%
\pgfusepath{fill}%
\end{pgfscope}%
\begin{pgfscope}%
\pgfpathrectangle{\pgfqpoint{0.000000in}{0.000000in}}{\pgfqpoint{3.000000in}{3.000000in}}%
\pgfusepath{clip}%
\pgfsetbuttcap%
\pgfsetroundjoin%
\definecolor{currentfill}{rgb}{1.000000,0.480029,0.000000}%
\pgfsetfillcolor{currentfill}%
\pgfsetlinewidth{0.000000pt}%
\definecolor{currentstroke}{rgb}{0.000000,0.000000,0.000000}%
\pgfsetstrokecolor{currentstroke}%
\pgfsetdash{}{0pt}%
\pgfpathmoveto{\pgfqpoint{1.005239in}{1.707563in}}%
\pgfpathlineto{\pgfqpoint{0.988958in}{1.721057in}}%
\pgfpathlineto{\pgfqpoint{1.031904in}{1.680046in}}%
\pgfpathlineto{\pgfqpoint{1.046983in}{1.667709in}}%
\pgfpathlineto{\pgfqpoint{1.005239in}{1.707563in}}%
\pgfpathclose%
\pgfusepath{fill}%
\end{pgfscope}%
\begin{pgfscope}%
\pgfpathrectangle{\pgfqpoint{0.000000in}{0.000000in}}{\pgfqpoint{3.000000in}{3.000000in}}%
\pgfusepath{clip}%
\pgfsetbuttcap%
\pgfsetroundjoin%
\definecolor{currentfill}{rgb}{0.743201,1.000000,0.224541}%
\pgfsetfillcolor{currentfill}%
\pgfsetlinewidth{0.000000pt}%
\definecolor{currentstroke}{rgb}{0.000000,0.000000,0.000000}%
\pgfsetstrokecolor{currentstroke}%
\pgfsetdash{}{0pt}%
\pgfpathmoveto{\pgfqpoint{1.454395in}{1.423638in}}%
\pgfpathlineto{\pgfqpoint{1.450648in}{1.438709in}}%
\pgfpathlineto{\pgfqpoint{1.517902in}{1.434231in}}%
\pgfpathlineto{\pgfqpoint{1.518846in}{1.419340in}}%
\pgfpathlineto{\pgfqpoint{1.454395in}{1.423638in}}%
\pgfpathclose%
\pgfusepath{fill}%
\end{pgfscope}%
\begin{pgfscope}%
\pgfpathrectangle{\pgfqpoint{0.000000in}{0.000000in}}{\pgfqpoint{3.000000in}{3.000000in}}%
\pgfusepath{clip}%
\pgfsetbuttcap%
\pgfsetroundjoin%
\definecolor{currentfill}{rgb}{0.958254,0.973856,0.009488}%
\pgfsetfillcolor{currentfill}%
\pgfsetlinewidth{0.000000pt}%
\definecolor{currentstroke}{rgb}{0.000000,0.000000,0.000000}%
\pgfsetstrokecolor{currentstroke}%
\pgfsetdash{}{0pt}%
\pgfpathmoveto{\pgfqpoint{1.794835in}{1.498856in}}%
\pgfpathlineto{\pgfqpoint{1.804655in}{1.512518in}}%
\pgfpathlineto{\pgfqpoint{1.864225in}{1.535692in}}%
\pgfpathlineto{\pgfqpoint{1.852219in}{1.521203in}}%
\pgfpathlineto{\pgfqpoint{1.794835in}{1.498856in}}%
\pgfpathclose%
\pgfusepath{fill}%
\end{pgfscope}%
\begin{pgfscope}%
\pgfpathrectangle{\pgfqpoint{0.000000in}{0.000000in}}{\pgfqpoint{3.000000in}{3.000000in}}%
\pgfusepath{clip}%
\pgfsetbuttcap%
\pgfsetroundjoin%
\definecolor{currentfill}{rgb}{0.553476,0.000000,0.000000}%
\pgfsetfillcolor{currentfill}%
\pgfsetlinewidth{0.000000pt}%
\definecolor{currentstroke}{rgb}{0.000000,0.000000,0.000000}%
\pgfsetstrokecolor{currentstroke}%
\pgfsetdash{}{0pt}%
\pgfpathmoveto{\pgfqpoint{0.718460in}{1.989551in}}%
\pgfpathlineto{\pgfqpoint{0.700866in}{2.001839in}}%
\pgfpathlineto{\pgfqpoint{0.719297in}{1.939480in}}%
\pgfpathlineto{\pgfqpoint{0.736622in}{1.928424in}}%
\pgfpathlineto{\pgfqpoint{0.718460in}{1.989551in}}%
\pgfpathclose%
\pgfusepath{fill}%
\end{pgfscope}%
\begin{pgfscope}%
\pgfpathrectangle{\pgfqpoint{0.000000in}{0.000000in}}{\pgfqpoint{3.000000in}{3.000000in}}%
\pgfusepath{clip}%
\pgfsetbuttcap%
\pgfsetroundjoin%
\definecolor{currentfill}{rgb}{1.000000,0.814089,0.000000}%
\pgfsetfillcolor{currentfill}%
\pgfsetlinewidth{0.000000pt}%
\definecolor{currentstroke}{rgb}{0.000000,0.000000,0.000000}%
\pgfsetstrokecolor{currentstroke}%
\pgfsetdash{}{0pt}%
\pgfpathmoveto{\pgfqpoint{1.876257in}{1.549473in}}%
\pgfpathlineto{\pgfqpoint{1.888314in}{1.562608in}}%
\pgfpathlineto{\pgfqpoint{1.943181in}{1.592025in}}%
\pgfpathlineto{\pgfqpoint{1.929264in}{1.577913in}}%
\pgfpathlineto{\pgfqpoint{1.876257in}{1.549473in}}%
\pgfpathclose%
\pgfusepath{fill}%
\end{pgfscope}%
\begin{pgfscope}%
\pgfpathrectangle{\pgfqpoint{0.000000in}{0.000000in}}{\pgfqpoint{3.000000in}{3.000000in}}%
\pgfusepath{clip}%
\pgfsetbuttcap%
\pgfsetroundjoin%
\definecolor{currentfill}{rgb}{0.743201,1.000000,0.224541}%
\pgfsetfillcolor{currentfill}%
\pgfsetlinewidth{0.000000pt}%
\definecolor{currentstroke}{rgb}{0.000000,0.000000,0.000000}%
\pgfsetstrokecolor{currentstroke}%
\pgfsetdash{}{0pt}%
\pgfpathmoveto{\pgfqpoint{1.518846in}{1.419340in}}%
\pgfpathlineto{\pgfqpoint{1.517902in}{1.434231in}}%
\pgfpathlineto{\pgfqpoint{1.585751in}{1.435129in}}%
\pgfpathlineto{\pgfqpoint{1.583866in}{1.420202in}}%
\pgfpathlineto{\pgfqpoint{1.518846in}{1.419340in}}%
\pgfpathclose%
\pgfusepath{fill}%
\end{pgfscope}%
\begin{pgfscope}%
\pgfpathrectangle{\pgfqpoint{0.000000in}{0.000000in}}{\pgfqpoint{3.000000in}{3.000000in}}%
\pgfusepath{clip}%
\pgfsetbuttcap%
\pgfsetroundjoin%
\definecolor{currentfill}{rgb}{0.927807,0.015251,0.000000}%
\pgfsetfillcolor{currentfill}%
\pgfsetlinewidth{0.000000pt}%
\definecolor{currentstroke}{rgb}{0.000000,0.000000,0.000000}%
\pgfsetstrokecolor{currentstroke}%
\pgfsetdash{}{0pt}%
\pgfpathmoveto{\pgfqpoint{2.220074in}{1.823118in}}%
\pgfpathlineto{\pgfqpoint{2.236839in}{1.834548in}}%
\pgfpathlineto{\pgfqpoint{2.265733in}{1.888571in}}%
\pgfpathlineto{\pgfqpoint{2.248374in}{1.875914in}}%
\pgfpathlineto{\pgfqpoint{2.220074in}{1.823118in}}%
\pgfpathclose%
\pgfusepath{fill}%
\end{pgfscope}%
\begin{pgfscope}%
\pgfpathrectangle{\pgfqpoint{0.000000in}{0.000000in}}{\pgfqpoint{3.000000in}{3.000000in}}%
\pgfusepath{clip}%
\pgfsetbuttcap%
\pgfsetroundjoin%
\definecolor{currentfill}{rgb}{1.000000,0.349310,0.000000}%
\pgfsetfillcolor{currentfill}%
\pgfsetlinewidth{0.000000pt}%
\definecolor{currentstroke}{rgb}{0.000000,0.000000,0.000000}%
\pgfsetstrokecolor{currentstroke}%
\pgfsetdash{}{0pt}%
\pgfpathmoveto{\pgfqpoint{2.080560in}{1.705687in}}%
\pgfpathlineto{\pgfqpoint{2.096136in}{1.717651in}}%
\pgfpathlineto{\pgfqpoint{2.136583in}{1.762008in}}%
\pgfpathlineto{\pgfqpoint{2.119951in}{1.748862in}}%
\pgfpathlineto{\pgfqpoint{2.080560in}{1.705687in}}%
\pgfpathclose%
\pgfusepath{fill}%
\end{pgfscope}%
\begin{pgfscope}%
\pgfpathrectangle{\pgfqpoint{0.000000in}{0.000000in}}{\pgfqpoint{3.000000in}{3.000000in}}%
\pgfusepath{clip}%
\pgfsetbuttcap%
\pgfsetroundjoin%
\definecolor{currentfill}{rgb}{1.000000,0.610748,0.000000}%
\pgfsetfillcolor{currentfill}%
\pgfsetlinewidth{0.000000pt}%
\definecolor{currentstroke}{rgb}{0.000000,0.000000,0.000000}%
\pgfsetstrokecolor{currentstroke}%
\pgfsetdash{}{0pt}%
\pgfpathmoveto{\pgfqpoint{1.971094in}{1.618535in}}%
\pgfpathlineto{\pgfqpoint{1.985089in}{1.631027in}}%
\pgfpathlineto{\pgfqpoint{2.033982in}{1.667513in}}%
\pgfpathlineto{\pgfqpoint{2.018506in}{1.653924in}}%
\pgfpathlineto{\pgfqpoint{1.971094in}{1.618535in}}%
\pgfpathclose%
\pgfusepath{fill}%
\end{pgfscope}%
\begin{pgfscope}%
\pgfpathrectangle{\pgfqpoint{0.000000in}{0.000000in}}{\pgfqpoint{3.000000in}{3.000000in}}%
\pgfusepath{clip}%
\pgfsetbuttcap%
\pgfsetroundjoin%
\definecolor{currentfill}{rgb}{1.000000,0.668845,0.000000}%
\pgfsetfillcolor{currentfill}%
\pgfsetlinewidth{0.000000pt}%
\definecolor{currentstroke}{rgb}{0.000000,0.000000,0.000000}%
\pgfsetstrokecolor{currentstroke}%
\pgfsetdash{}{0pt}%
\pgfpathmoveto{\pgfqpoint{1.092064in}{1.628033in}}%
\pgfpathlineto{\pgfqpoint{1.077063in}{1.641739in}}%
\pgfpathlineto{\pgfqpoint{1.128339in}{1.607586in}}%
\pgfpathlineto{\pgfqpoint{1.141728in}{1.594941in}}%
\pgfpathlineto{\pgfqpoint{1.092064in}{1.628033in}}%
\pgfpathclose%
\pgfusepath{fill}%
\end{pgfscope}%
\begin{pgfscope}%
\pgfpathrectangle{\pgfqpoint{0.000000in}{0.000000in}}{\pgfqpoint{3.000000in}{3.000000in}}%
\pgfusepath{clip}%
\pgfsetbuttcap%
\pgfsetroundjoin%
\definecolor{currentfill}{rgb}{1.000000,0.116921,0.000000}%
\pgfsetfillcolor{currentfill}%
\pgfsetlinewidth{0.000000pt}%
\definecolor{currentstroke}{rgb}{0.000000,0.000000,0.000000}%
\pgfsetstrokecolor{currentstroke}%
\pgfsetdash{}{0pt}%
\pgfpathmoveto{\pgfqpoint{0.874536in}{1.832887in}}%
\pgfpathlineto{\pgfqpoint{0.857364in}{1.845632in}}%
\pgfpathlineto{\pgfqpoint{0.890781in}{1.794743in}}%
\pgfpathlineto{\pgfqpoint{0.907202in}{1.783214in}}%
\pgfpathlineto{\pgfqpoint{0.874536in}{1.832887in}}%
\pgfpathclose%
\pgfusepath{fill}%
\end{pgfscope}%
\begin{pgfscope}%
\pgfpathrectangle{\pgfqpoint{0.000000in}{0.000000in}}{\pgfqpoint{3.000000in}{3.000000in}}%
\pgfusepath{clip}%
\pgfsetbuttcap%
\pgfsetroundjoin%
\definecolor{currentfill}{rgb}{0.819102,1.000000,0.148640}%
\pgfsetfillcolor{currentfill}%
\pgfsetlinewidth{0.000000pt}%
\definecolor{currentstroke}{rgb}{0.000000,0.000000,0.000000}%
\pgfsetstrokecolor{currentstroke}%
\pgfsetdash{}{0pt}%
\pgfpathmoveto{\pgfqpoint{1.652411in}{1.441379in}}%
\pgfpathlineto{\pgfqpoint{1.657084in}{1.455586in}}%
\pgfpathlineto{\pgfqpoint{1.723460in}{1.467471in}}%
\pgfpathlineto{\pgfqpoint{1.716134in}{1.452806in}}%
\pgfpathlineto{\pgfqpoint{1.652411in}{1.441379in}}%
\pgfpathclose%
\pgfusepath{fill}%
\end{pgfscope}%
\begin{pgfscope}%
\pgfpathrectangle{\pgfqpoint{0.000000in}{0.000000in}}{\pgfqpoint{3.000000in}{3.000000in}}%
\pgfusepath{clip}%
\pgfsetbuttcap%
\pgfsetroundjoin%
\definecolor{currentfill}{rgb}{0.895003,1.000000,0.072739}%
\pgfsetfillcolor{currentfill}%
\pgfsetlinewidth{0.000000pt}%
\definecolor{currentstroke}{rgb}{0.000000,0.000000,0.000000}%
\pgfsetstrokecolor{currentstroke}%
\pgfsetdash{}{0pt}%
\pgfpathmoveto{\pgfqpoint{1.723460in}{1.467471in}}%
\pgfpathlineto{\pgfqpoint{1.730803in}{1.481265in}}%
\pgfpathlineto{\pgfqpoint{1.794835in}{1.498856in}}%
\pgfpathlineto{\pgfqpoint{1.785038in}{1.484409in}}%
\pgfpathlineto{\pgfqpoint{1.723460in}{1.467471in}}%
\pgfpathclose%
\pgfusepath{fill}%
\end{pgfscope}%
\begin{pgfscope}%
\pgfpathrectangle{\pgfqpoint{0.000000in}{0.000000in}}{\pgfqpoint{3.000000in}{3.000000in}}%
\pgfusepath{clip}%
\pgfsetbuttcap%
\pgfsetroundjoin%
\definecolor{currentfill}{rgb}{0.500000,0.000000,0.000000}%
\pgfsetfillcolor{currentfill}%
\pgfsetlinewidth{0.000000pt}%
\definecolor{currentstroke}{rgb}{0.000000,0.000000,0.000000}%
\pgfsetstrokecolor{currentstroke}%
\pgfsetdash{}{0pt}%
\pgfpathmoveto{\pgfqpoint{0.700866in}{2.001839in}}%
\pgfpathlineto{\pgfqpoint{0.683258in}{2.013962in}}%
\pgfpathlineto{\pgfqpoint{0.701952in}{1.950374in}}%
\pgfpathlineto{\pgfqpoint{0.719297in}{1.939480in}}%
\pgfpathlineto{\pgfqpoint{0.700866in}{2.001839in}}%
\pgfpathclose%
\pgfusepath{fill}%
\end{pgfscope}%
\begin{pgfscope}%
\pgfpathrectangle{\pgfqpoint{0.000000in}{0.000000in}}{\pgfqpoint{3.000000in}{3.000000in}}%
\pgfusepath{clip}%
\pgfsetbuttcap%
\pgfsetroundjoin%
\definecolor{currentfill}{rgb}{1.000000,0.407407,0.000000}%
\pgfsetfillcolor{currentfill}%
\pgfsetlinewidth{0.000000pt}%
\definecolor{currentstroke}{rgb}{0.000000,0.000000,0.000000}%
\pgfsetstrokecolor{currentstroke}%
\pgfsetdash{}{0pt}%
\pgfpathmoveto{\pgfqpoint{0.988958in}{1.721057in}}%
\pgfpathlineto{\pgfqpoint{0.972654in}{1.734163in}}%
\pgfpathlineto{\pgfqpoint{1.016800in}{1.691994in}}%
\pgfpathlineto{\pgfqpoint{1.031904in}{1.680046in}}%
\pgfpathlineto{\pgfqpoint{0.988958in}{1.721057in}}%
\pgfpathclose%
\pgfusepath{fill}%
\end{pgfscope}%
\begin{pgfscope}%
\pgfpathrectangle{\pgfqpoint{0.000000in}{0.000000in}}{\pgfqpoint{3.000000in}{3.000000in}}%
\pgfusepath{clip}%
\pgfsetbuttcap%
\pgfsetroundjoin%
\definecolor{currentfill}{rgb}{0.958254,0.973856,0.009488}%
\pgfsetfillcolor{currentfill}%
\pgfsetlinewidth{0.000000pt}%
\definecolor{currentstroke}{rgb}{0.000000,0.000000,0.000000}%
\pgfsetstrokecolor{currentstroke}%
\pgfsetdash{}{0pt}%
\pgfpathmoveto{\pgfqpoint{1.247158in}{1.513261in}}%
\pgfpathlineto{\pgfqpoint{1.235846in}{1.527457in}}%
\pgfpathlineto{\pgfqpoint{1.297901in}{1.505861in}}%
\pgfpathlineto{\pgfqpoint{1.306929in}{1.492437in}}%
\pgfpathlineto{\pgfqpoint{1.247158in}{1.513261in}}%
\pgfpathclose%
\pgfusepath{fill}%
\end{pgfscope}%
\begin{pgfscope}%
\pgfpathrectangle{\pgfqpoint{0.000000in}{0.000000in}}{\pgfqpoint{3.000000in}{3.000000in}}%
\pgfusepath{clip}%
\pgfsetbuttcap%
\pgfsetroundjoin%
\definecolor{currentfill}{rgb}{0.856506,0.000000,0.000000}%
\pgfsetfillcolor{currentfill}%
\pgfsetlinewidth{0.000000pt}%
\definecolor{currentstroke}{rgb}{0.000000,0.000000,0.000000}%
\pgfsetstrokecolor{currentstroke}%
\pgfsetdash{}{0pt}%
\pgfpathmoveto{\pgfqpoint{2.236839in}{1.834548in}}%
\pgfpathlineto{\pgfqpoint{2.253625in}{1.845752in}}%
\pgfpathlineto{\pgfqpoint{2.283109in}{1.901000in}}%
\pgfpathlineto{\pgfqpoint{2.265733in}{1.888571in}}%
\pgfpathlineto{\pgfqpoint{2.236839in}{1.834548in}}%
\pgfpathclose%
\pgfusepath{fill}%
\end{pgfscope}%
\begin{pgfscope}%
\pgfpathrectangle{\pgfqpoint{0.000000in}{0.000000in}}{\pgfqpoint{3.000000in}{3.000000in}}%
\pgfusepath{clip}%
\pgfsetbuttcap%
\pgfsetroundjoin%
\definecolor{currentfill}{rgb}{0.819102,1.000000,0.148640}%
\pgfsetfillcolor{currentfill}%
\pgfsetlinewidth{0.000000pt}%
\definecolor{currentstroke}{rgb}{0.000000,0.000000,0.000000}%
\pgfsetstrokecolor{currentstroke}%
\pgfsetdash{}{0pt}%
\pgfpathmoveto{\pgfqpoint{1.385757in}{1.448438in}}%
\pgfpathlineto{\pgfqpoint{1.379296in}{1.462929in}}%
\pgfpathlineto{\pgfqpoint{1.446892in}{1.452808in}}%
\pgfpathlineto{\pgfqpoint{1.450648in}{1.438709in}}%
\pgfpathlineto{\pgfqpoint{1.385757in}{1.448438in}}%
\pgfpathclose%
\pgfusepath{fill}%
\end{pgfscope}%
\begin{pgfscope}%
\pgfpathrectangle{\pgfqpoint{0.000000in}{0.000000in}}{\pgfqpoint{3.000000in}{3.000000in}}%
\pgfusepath{clip}%
\pgfsetbuttcap%
\pgfsetroundjoin%
\definecolor{currentfill}{rgb}{1.000000,0.814089,0.000000}%
\pgfsetfillcolor{currentfill}%
\pgfsetlinewidth{0.000000pt}%
\definecolor{currentstroke}{rgb}{0.000000,0.000000,0.000000}%
\pgfsetstrokecolor{currentstroke}%
\pgfsetdash{}{0pt}%
\pgfpathmoveto{\pgfqpoint{1.168428in}{1.567985in}}%
\pgfpathlineto{\pgfqpoint{1.155091in}{1.581757in}}%
\pgfpathlineto{\pgfqpoint{1.213149in}{1.553784in}}%
\pgfpathlineto{\pgfqpoint{1.224510in}{1.540943in}}%
\pgfpathlineto{\pgfqpoint{1.168428in}{1.567985in}}%
\pgfpathclose%
\pgfusepath{fill}%
\end{pgfscope}%
\begin{pgfscope}%
\pgfpathrectangle{\pgfqpoint{0.000000in}{0.000000in}}{\pgfqpoint{3.000000in}{3.000000in}}%
\pgfusepath{clip}%
\pgfsetbuttcap%
\pgfsetroundjoin%
\definecolor{currentfill}{rgb}{0.895003,1.000000,0.072739}%
\pgfsetfillcolor{currentfill}%
\pgfsetlinewidth{0.000000pt}%
\definecolor{currentstroke}{rgb}{0.000000,0.000000,0.000000}%
\pgfsetstrokecolor{currentstroke}%
\pgfsetdash{}{0pt}%
\pgfpathmoveto{\pgfqpoint{1.315935in}{1.478228in}}%
\pgfpathlineto{\pgfqpoint{1.306929in}{1.492437in}}%
\pgfpathlineto{\pgfqpoint{1.372820in}{1.476547in}}%
\pgfpathlineto{\pgfqpoint{1.379296in}{1.462929in}}%
\pgfpathlineto{\pgfqpoint{1.315935in}{1.478228in}}%
\pgfpathclose%
\pgfusepath{fill}%
\end{pgfscope}%
\begin{pgfscope}%
\pgfpathrectangle{\pgfqpoint{0.000000in}{0.000000in}}{\pgfqpoint{3.000000in}{3.000000in}}%
\pgfusepath{clip}%
\pgfsetbuttcap%
\pgfsetroundjoin%
\definecolor{currentfill}{rgb}{1.000000,0.291213,0.000000}%
\pgfsetfillcolor{currentfill}%
\pgfsetlinewidth{0.000000pt}%
\definecolor{currentstroke}{rgb}{0.000000,0.000000,0.000000}%
\pgfsetstrokecolor{currentstroke}%
\pgfsetdash{}{0pt}%
\pgfpathmoveto{\pgfqpoint{2.096136in}{1.717651in}}%
\pgfpathlineto{\pgfqpoint{2.111737in}{1.729279in}}%
\pgfpathlineto{\pgfqpoint{2.153237in}{1.774817in}}%
\pgfpathlineto{\pgfqpoint{2.136583in}{1.762008in}}%
\pgfpathlineto{\pgfqpoint{2.096136in}{1.717651in}}%
\pgfpathclose%
\pgfusepath{fill}%
\end{pgfscope}%
\begin{pgfscope}%
\pgfpathrectangle{\pgfqpoint{0.000000in}{0.000000in}}{\pgfqpoint{3.000000in}{3.000000in}}%
\pgfusepath{clip}%
\pgfsetbuttcap%
\pgfsetroundjoin%
\definecolor{currentfill}{rgb}{0.999109,0.073348,0.000000}%
\pgfsetfillcolor{currentfill}%
\pgfsetlinewidth{0.000000pt}%
\definecolor{currentstroke}{rgb}{0.000000,0.000000,0.000000}%
\pgfsetstrokecolor{currentstroke}%
\pgfsetdash{}{0pt}%
\pgfpathmoveto{\pgfqpoint{0.857364in}{1.845632in}}%
\pgfpathlineto{\pgfqpoint{0.840173in}{1.858119in}}%
\pgfpathlineto{\pgfqpoint{0.874337in}{1.806014in}}%
\pgfpathlineto{\pgfqpoint{0.890781in}{1.794743in}}%
\pgfpathlineto{\pgfqpoint{0.857364in}{1.845632in}}%
\pgfpathclose%
\pgfusepath{fill}%
\end{pgfscope}%
\begin{pgfscope}%
\pgfpathrectangle{\pgfqpoint{0.000000in}{0.000000in}}{\pgfqpoint{3.000000in}{3.000000in}}%
\pgfusepath{clip}%
\pgfsetbuttcap%
\pgfsetroundjoin%
\definecolor{currentfill}{rgb}{1.000000,0.741467,0.000000}%
\pgfsetfillcolor{currentfill}%
\pgfsetlinewidth{0.000000pt}%
\definecolor{currentstroke}{rgb}{0.000000,0.000000,0.000000}%
\pgfsetstrokecolor{currentstroke}%
\pgfsetdash{}{0pt}%
\pgfpathmoveto{\pgfqpoint{1.888314in}{1.562608in}}%
\pgfpathlineto{\pgfqpoint{1.900396in}{1.575154in}}%
\pgfpathlineto{\pgfqpoint{1.957124in}{1.605549in}}%
\pgfpathlineto{\pgfqpoint{1.943181in}{1.592025in}}%
\pgfpathlineto{\pgfqpoint{1.888314in}{1.562608in}}%
\pgfpathclose%
\pgfusepath{fill}%
\end{pgfscope}%
\begin{pgfscope}%
\pgfpathrectangle{\pgfqpoint{0.000000in}{0.000000in}}{\pgfqpoint{3.000000in}{3.000000in}}%
\pgfusepath{clip}%
\pgfsetbuttcap%
\pgfsetroundjoin%
\definecolor{currentfill}{rgb}{1.000000,0.538126,0.000000}%
\pgfsetfillcolor{currentfill}%
\pgfsetlinewidth{0.000000pt}%
\definecolor{currentstroke}{rgb}{0.000000,0.000000,0.000000}%
\pgfsetstrokecolor{currentstroke}%
\pgfsetdash{}{0pt}%
\pgfpathmoveto{\pgfqpoint{1.985089in}{1.631027in}}%
\pgfpathlineto{\pgfqpoint{1.999111in}{1.643064in}}%
\pgfpathlineto{\pgfqpoint{2.049482in}{1.680647in}}%
\pgfpathlineto{\pgfqpoint{2.033982in}{1.667513in}}%
\pgfpathlineto{\pgfqpoint{1.985089in}{1.631027in}}%
\pgfpathclose%
\pgfusepath{fill}%
\end{pgfscope}%
\begin{pgfscope}%
\pgfpathrectangle{\pgfqpoint{0.000000in}{0.000000in}}{\pgfqpoint{3.000000in}{3.000000in}}%
\pgfusepath{clip}%
\pgfsetbuttcap%
\pgfsetroundjoin%
\definecolor{currentfill}{rgb}{0.819102,1.000000,0.148640}%
\pgfsetfillcolor{currentfill}%
\pgfsetlinewidth{0.000000pt}%
\definecolor{currentstroke}{rgb}{0.000000,0.000000,0.000000}%
\pgfsetstrokecolor{currentstroke}%
\pgfsetdash{}{0pt}%
\pgfpathmoveto{\pgfqpoint{1.585751in}{1.435129in}}%
\pgfpathlineto{\pgfqpoint{1.587641in}{1.449085in}}%
\pgfpathlineto{\pgfqpoint{1.657084in}{1.455586in}}%
\pgfpathlineto{\pgfqpoint{1.652411in}{1.441379in}}%
\pgfpathlineto{\pgfqpoint{1.585751in}{1.435129in}}%
\pgfpathclose%
\pgfusepath{fill}%
\end{pgfscope}%
\begin{pgfscope}%
\pgfpathrectangle{\pgfqpoint{0.000000in}{0.000000in}}{\pgfqpoint{3.000000in}{3.000000in}}%
\pgfusepath{clip}%
\pgfsetbuttcap%
\pgfsetroundjoin%
\definecolor{currentfill}{rgb}{1.000000,0.886710,0.000000}%
\pgfsetfillcolor{currentfill}%
\pgfsetlinewidth{0.000000pt}%
\definecolor{currentstroke}{rgb}{0.000000,0.000000,0.000000}%
\pgfsetstrokecolor{currentstroke}%
\pgfsetdash{}{0pt}%
\pgfpathmoveto{\pgfqpoint{1.804655in}{1.512518in}}%
\pgfpathlineto{\pgfqpoint{1.814498in}{1.525470in}}%
\pgfpathlineto{\pgfqpoint{1.876257in}{1.549473in}}%
\pgfpathlineto{\pgfqpoint{1.864225in}{1.535692in}}%
\pgfpathlineto{\pgfqpoint{1.804655in}{1.512518in}}%
\pgfpathclose%
\pgfusepath{fill}%
\end{pgfscope}%
\begin{pgfscope}%
\pgfpathrectangle{\pgfqpoint{0.000000in}{0.000000in}}{\pgfqpoint{3.000000in}{3.000000in}}%
\pgfusepath{clip}%
\pgfsetbuttcap%
\pgfsetroundjoin%
\definecolor{currentfill}{rgb}{1.000000,0.610748,0.000000}%
\pgfsetfillcolor{currentfill}%
\pgfsetlinewidth{0.000000pt}%
\definecolor{currentstroke}{rgb}{0.000000,0.000000,0.000000}%
\pgfsetstrokecolor{currentstroke}%
\pgfsetdash{}{0pt}%
\pgfpathmoveto{\pgfqpoint{1.077063in}{1.641739in}}%
\pgfpathlineto{\pgfqpoint{1.062036in}{1.654952in}}%
\pgfpathlineto{\pgfqpoint{1.114924in}{1.619738in}}%
\pgfpathlineto{\pgfqpoint{1.128339in}{1.607586in}}%
\pgfpathlineto{\pgfqpoint{1.077063in}{1.641739in}}%
\pgfpathclose%
\pgfusepath{fill}%
\end{pgfscope}%
\begin{pgfscope}%
\pgfpathrectangle{\pgfqpoint{0.000000in}{0.000000in}}{\pgfqpoint{3.000000in}{3.000000in}}%
\pgfusepath{clip}%
\pgfsetbuttcap%
\pgfsetroundjoin%
\definecolor{currentfill}{rgb}{0.803030,0.000000,0.000000}%
\pgfsetfillcolor{currentfill}%
\pgfsetlinewidth{0.000000pt}%
\definecolor{currentstroke}{rgb}{0.000000,0.000000,0.000000}%
\pgfsetstrokecolor{currentstroke}%
\pgfsetdash{}{0pt}%
\pgfpathmoveto{\pgfqpoint{2.253625in}{1.845752in}}%
\pgfpathlineto{\pgfqpoint{2.270434in}{1.856741in}}%
\pgfpathlineto{\pgfqpoint{2.300503in}{1.913214in}}%
\pgfpathlineto{\pgfqpoint{2.283109in}{1.901000in}}%
\pgfpathlineto{\pgfqpoint{2.253625in}{1.845752in}}%
\pgfpathclose%
\pgfusepath{fill}%
\end{pgfscope}%
\begin{pgfscope}%
\pgfpathrectangle{\pgfqpoint{0.000000in}{0.000000in}}{\pgfqpoint{3.000000in}{3.000000in}}%
\pgfusepath{clip}%
\pgfsetbuttcap%
\pgfsetroundjoin%
\definecolor{currentfill}{rgb}{0.819102,1.000000,0.148640}%
\pgfsetfillcolor{currentfill}%
\pgfsetlinewidth{0.000000pt}%
\definecolor{currentstroke}{rgb}{0.000000,0.000000,0.000000}%
\pgfsetstrokecolor{currentstroke}%
\pgfsetdash{}{0pt}%
\pgfpathmoveto{\pgfqpoint{1.450648in}{1.438709in}}%
\pgfpathlineto{\pgfqpoint{1.446892in}{1.452808in}}%
\pgfpathlineto{\pgfqpoint{1.516956in}{1.448150in}}%
\pgfpathlineto{\pgfqpoint{1.517902in}{1.434231in}}%
\pgfpathlineto{\pgfqpoint{1.450648in}{1.438709in}}%
\pgfpathclose%
\pgfusepath{fill}%
\end{pgfscope}%
\begin{pgfscope}%
\pgfpathrectangle{\pgfqpoint{0.000000in}{0.000000in}}{\pgfqpoint{3.000000in}{3.000000in}}%
\pgfusepath{clip}%
\pgfsetbuttcap%
\pgfsetroundjoin%
\definecolor{currentfill}{rgb}{1.000000,0.349310,0.000000}%
\pgfsetfillcolor{currentfill}%
\pgfsetlinewidth{0.000000pt}%
\definecolor{currentstroke}{rgb}{0.000000,0.000000,0.000000}%
\pgfsetstrokecolor{currentstroke}%
\pgfsetdash{}{0pt}%
\pgfpathmoveto{\pgfqpoint{0.972654in}{1.734163in}}%
\pgfpathlineto{\pgfqpoint{0.956326in}{1.746908in}}%
\pgfpathlineto{\pgfqpoint{1.001671in}{1.703581in}}%
\pgfpathlineto{\pgfqpoint{1.016800in}{1.691994in}}%
\pgfpathlineto{\pgfqpoint{0.972654in}{1.734163in}}%
\pgfpathclose%
\pgfusepath{fill}%
\end{pgfscope}%
\begin{pgfscope}%
\pgfpathrectangle{\pgfqpoint{0.000000in}{0.000000in}}{\pgfqpoint{3.000000in}{3.000000in}}%
\pgfusepath{clip}%
\pgfsetbuttcap%
\pgfsetroundjoin%
\definecolor{currentfill}{rgb}{0.819102,1.000000,0.148640}%
\pgfsetfillcolor{currentfill}%
\pgfsetlinewidth{0.000000pt}%
\definecolor{currentstroke}{rgb}{0.000000,0.000000,0.000000}%
\pgfsetstrokecolor{currentstroke}%
\pgfsetdash{}{0pt}%
\pgfpathmoveto{\pgfqpoint{1.517902in}{1.434231in}}%
\pgfpathlineto{\pgfqpoint{1.516956in}{1.448150in}}%
\pgfpathlineto{\pgfqpoint{1.587641in}{1.449085in}}%
\pgfpathlineto{\pgfqpoint{1.585751in}{1.435129in}}%
\pgfpathlineto{\pgfqpoint{1.517902in}{1.434231in}}%
\pgfpathclose%
\pgfusepath{fill}%
\end{pgfscope}%
\begin{pgfscope}%
\pgfpathrectangle{\pgfqpoint{0.000000in}{0.000000in}}{\pgfqpoint{3.000000in}{3.000000in}}%
\pgfusepath{clip}%
\pgfsetbuttcap%
\pgfsetroundjoin%
\definecolor{currentfill}{rgb}{0.927807,0.015251,0.000000}%
\pgfsetfillcolor{currentfill}%
\pgfsetlinewidth{0.000000pt}%
\definecolor{currentstroke}{rgb}{0.000000,0.000000,0.000000}%
\pgfsetstrokecolor{currentstroke}%
\pgfsetdash{}{0pt}%
\pgfpathmoveto{\pgfqpoint{0.840173in}{1.858119in}}%
\pgfpathlineto{\pgfqpoint{0.822963in}{1.870364in}}%
\pgfpathlineto{\pgfqpoint{0.857870in}{1.817044in}}%
\pgfpathlineto{\pgfqpoint{0.874337in}{1.806014in}}%
\pgfpathlineto{\pgfqpoint{0.840173in}{1.858119in}}%
\pgfpathclose%
\pgfusepath{fill}%
\end{pgfscope}%
\begin{pgfscope}%
\pgfpathrectangle{\pgfqpoint{0.000000in}{0.000000in}}{\pgfqpoint{3.000000in}{3.000000in}}%
\pgfusepath{clip}%
\pgfsetbuttcap%
\pgfsetroundjoin%
\definecolor{currentfill}{rgb}{1.000000,0.233115,0.000000}%
\pgfsetfillcolor{currentfill}%
\pgfsetlinewidth{0.000000pt}%
\definecolor{currentstroke}{rgb}{0.000000,0.000000,0.000000}%
\pgfsetstrokecolor{currentstroke}%
\pgfsetdash{}{0pt}%
\pgfpathmoveto{\pgfqpoint{2.111737in}{1.729279in}}%
\pgfpathlineto{\pgfqpoint{2.127363in}{1.740593in}}%
\pgfpathlineto{\pgfqpoint{2.169913in}{1.787313in}}%
\pgfpathlineto{\pgfqpoint{2.153237in}{1.774817in}}%
\pgfpathlineto{\pgfqpoint{2.111737in}{1.729279in}}%
\pgfpathclose%
\pgfusepath{fill}%
\end{pgfscope}%
\begin{pgfscope}%
\pgfpathrectangle{\pgfqpoint{0.000000in}{0.000000in}}{\pgfqpoint{3.000000in}{3.000000in}}%
\pgfusepath{clip}%
\pgfsetbuttcap%
\pgfsetroundjoin%
\definecolor{currentfill}{rgb}{0.895003,1.000000,0.072739}%
\pgfsetfillcolor{currentfill}%
\pgfsetlinewidth{0.000000pt}%
\definecolor{currentstroke}{rgb}{0.000000,0.000000,0.000000}%
\pgfsetstrokecolor{currentstroke}%
\pgfsetdash{}{0pt}%
\pgfpathmoveto{\pgfqpoint{1.657084in}{1.455586in}}%
\pgfpathlineto{\pgfqpoint{1.661769in}{1.468921in}}%
\pgfpathlineto{\pgfqpoint{1.730803in}{1.481265in}}%
\pgfpathlineto{\pgfqpoint{1.723460in}{1.467471in}}%
\pgfpathlineto{\pgfqpoint{1.657084in}{1.455586in}}%
\pgfpathclose%
\pgfusepath{fill}%
\end{pgfscope}%
\begin{pgfscope}%
\pgfpathrectangle{\pgfqpoint{0.000000in}{0.000000in}}{\pgfqpoint{3.000000in}{3.000000in}}%
\pgfusepath{clip}%
\pgfsetbuttcap%
\pgfsetroundjoin%
\definecolor{currentfill}{rgb}{0.958254,0.973856,0.009488}%
\pgfsetfillcolor{currentfill}%
\pgfsetlinewidth{0.000000pt}%
\definecolor{currentstroke}{rgb}{0.000000,0.000000,0.000000}%
\pgfsetstrokecolor{currentstroke}%
\pgfsetdash{}{0pt}%
\pgfpathmoveto{\pgfqpoint{1.730803in}{1.481265in}}%
\pgfpathlineto{\pgfqpoint{1.738165in}{1.494274in}}%
\pgfpathlineto{\pgfqpoint{1.804655in}{1.512518in}}%
\pgfpathlineto{\pgfqpoint{1.794835in}{1.498856in}}%
\pgfpathlineto{\pgfqpoint{1.730803in}{1.481265in}}%
\pgfpathclose%
\pgfusepath{fill}%
\end{pgfscope}%
\begin{pgfscope}%
\pgfpathrectangle{\pgfqpoint{0.000000in}{0.000000in}}{\pgfqpoint{3.000000in}{3.000000in}}%
\pgfusepath{clip}%
\pgfsetbuttcap%
\pgfsetroundjoin%
\definecolor{currentfill}{rgb}{1.000000,0.886710,0.000000}%
\pgfsetfillcolor{currentfill}%
\pgfsetlinewidth{0.000000pt}%
\definecolor{currentstroke}{rgb}{0.000000,0.000000,0.000000}%
\pgfsetstrokecolor{currentstroke}%
\pgfsetdash{}{0pt}%
\pgfpathmoveto{\pgfqpoint{1.235846in}{1.527457in}}%
\pgfpathlineto{\pgfqpoint{1.224510in}{1.540943in}}%
\pgfpathlineto{\pgfqpoint{1.288852in}{1.518574in}}%
\pgfpathlineto{\pgfqpoint{1.297901in}{1.505861in}}%
\pgfpathlineto{\pgfqpoint{1.235846in}{1.527457in}}%
\pgfpathclose%
\pgfusepath{fill}%
\end{pgfscope}%
\begin{pgfscope}%
\pgfpathrectangle{\pgfqpoint{0.000000in}{0.000000in}}{\pgfqpoint{3.000000in}{3.000000in}}%
\pgfusepath{clip}%
\pgfsetbuttcap%
\pgfsetroundjoin%
\definecolor{currentfill}{rgb}{0.731729,0.000000,0.000000}%
\pgfsetfillcolor{currentfill}%
\pgfsetlinewidth{0.000000pt}%
\definecolor{currentstroke}{rgb}{0.000000,0.000000,0.000000}%
\pgfsetstrokecolor{currentstroke}%
\pgfsetdash{}{0pt}%
\pgfpathmoveto{\pgfqpoint{2.270434in}{1.856741in}}%
\pgfpathlineto{\pgfqpoint{2.287265in}{1.867527in}}%
\pgfpathlineto{\pgfqpoint{2.317916in}{1.925226in}}%
\pgfpathlineto{\pgfqpoint{2.300503in}{1.913214in}}%
\pgfpathlineto{\pgfqpoint{2.270434in}{1.856741in}}%
\pgfpathclose%
\pgfusepath{fill}%
\end{pgfscope}%
\begin{pgfscope}%
\pgfpathrectangle{\pgfqpoint{0.000000in}{0.000000in}}{\pgfqpoint{3.000000in}{3.000000in}}%
\pgfusepath{clip}%
\pgfsetbuttcap%
\pgfsetroundjoin%
\definecolor{currentfill}{rgb}{1.000000,0.741467,0.000000}%
\pgfsetfillcolor{currentfill}%
\pgfsetlinewidth{0.000000pt}%
\definecolor{currentstroke}{rgb}{0.000000,0.000000,0.000000}%
\pgfsetstrokecolor{currentstroke}%
\pgfsetdash{}{0pt}%
\pgfpathmoveto{\pgfqpoint{1.155091in}{1.581757in}}%
\pgfpathlineto{\pgfqpoint{1.141728in}{1.594941in}}%
\pgfpathlineto{\pgfqpoint{1.201764in}{1.566036in}}%
\pgfpathlineto{\pgfqpoint{1.213149in}{1.553784in}}%
\pgfpathlineto{\pgfqpoint{1.155091in}{1.581757in}}%
\pgfpathclose%
\pgfusepath{fill}%
\end{pgfscope}%
\begin{pgfscope}%
\pgfpathrectangle{\pgfqpoint{0.000000in}{0.000000in}}{\pgfqpoint{3.000000in}{3.000000in}}%
\pgfusepath{clip}%
\pgfsetbuttcap%
\pgfsetroundjoin%
\definecolor{currentfill}{rgb}{0.895003,1.000000,0.072739}%
\pgfsetfillcolor{currentfill}%
\pgfsetlinewidth{0.000000pt}%
\definecolor{currentstroke}{rgb}{0.000000,0.000000,0.000000}%
\pgfsetstrokecolor{currentstroke}%
\pgfsetdash{}{0pt}%
\pgfpathmoveto{\pgfqpoint{1.379296in}{1.462929in}}%
\pgfpathlineto{\pgfqpoint{1.372820in}{1.476547in}}%
\pgfpathlineto{\pgfqpoint{1.443127in}{1.466035in}}%
\pgfpathlineto{\pgfqpoint{1.446892in}{1.452808in}}%
\pgfpathlineto{\pgfqpoint{1.379296in}{1.462929in}}%
\pgfpathclose%
\pgfusepath{fill}%
\end{pgfscope}%
\begin{pgfscope}%
\pgfpathrectangle{\pgfqpoint{0.000000in}{0.000000in}}{\pgfqpoint{3.000000in}{3.000000in}}%
\pgfusepath{clip}%
\pgfsetbuttcap%
\pgfsetroundjoin%
\definecolor{currentfill}{rgb}{1.000000,0.480029,0.000000}%
\pgfsetfillcolor{currentfill}%
\pgfsetlinewidth{0.000000pt}%
\definecolor{currentstroke}{rgb}{0.000000,0.000000,0.000000}%
\pgfsetstrokecolor{currentstroke}%
\pgfsetdash{}{0pt}%
\pgfpathmoveto{\pgfqpoint{1.999111in}{1.643064in}}%
\pgfpathlineto{\pgfqpoint{2.013159in}{1.654681in}}%
\pgfpathlineto{\pgfqpoint{2.065008in}{1.693362in}}%
\pgfpathlineto{\pgfqpoint{2.049482in}{1.680647in}}%
\pgfpathlineto{\pgfqpoint{1.999111in}{1.643064in}}%
\pgfpathclose%
\pgfusepath{fill}%
\end{pgfscope}%
\begin{pgfscope}%
\pgfpathrectangle{\pgfqpoint{0.000000in}{0.000000in}}{\pgfqpoint{3.000000in}{3.000000in}}%
\pgfusepath{clip}%
\pgfsetbuttcap%
\pgfsetroundjoin%
\definecolor{currentfill}{rgb}{1.000000,0.668845,0.000000}%
\pgfsetfillcolor{currentfill}%
\pgfsetlinewidth{0.000000pt}%
\definecolor{currentstroke}{rgb}{0.000000,0.000000,0.000000}%
\pgfsetstrokecolor{currentstroke}%
\pgfsetdash{}{0pt}%
\pgfpathmoveto{\pgfqpoint{1.900396in}{1.575154in}}%
\pgfpathlineto{\pgfqpoint{1.912504in}{1.587162in}}%
\pgfpathlineto{\pgfqpoint{1.971094in}{1.618535in}}%
\pgfpathlineto{\pgfqpoint{1.957124in}{1.605549in}}%
\pgfpathlineto{\pgfqpoint{1.900396in}{1.575154in}}%
\pgfpathclose%
\pgfusepath{fill}%
\end{pgfscope}%
\begin{pgfscope}%
\pgfpathrectangle{\pgfqpoint{0.000000in}{0.000000in}}{\pgfqpoint{3.000000in}{3.000000in}}%
\pgfusepath{clip}%
\pgfsetbuttcap%
\pgfsetroundjoin%
\definecolor{currentfill}{rgb}{1.000000,0.538126,0.000000}%
\pgfsetfillcolor{currentfill}%
\pgfsetlinewidth{0.000000pt}%
\definecolor{currentstroke}{rgb}{0.000000,0.000000,0.000000}%
\pgfsetstrokecolor{currentstroke}%
\pgfsetdash{}{0pt}%
\pgfpathmoveto{\pgfqpoint{1.062036in}{1.654952in}}%
\pgfpathlineto{\pgfqpoint{1.046983in}{1.667709in}}%
\pgfpathlineto{\pgfqpoint{1.101482in}{1.631433in}}%
\pgfpathlineto{\pgfqpoint{1.114924in}{1.619738in}}%
\pgfpathlineto{\pgfqpoint{1.062036in}{1.654952in}}%
\pgfpathclose%
\pgfusepath{fill}%
\end{pgfscope}%
\begin{pgfscope}%
\pgfpathrectangle{\pgfqpoint{0.000000in}{0.000000in}}{\pgfqpoint{3.000000in}{3.000000in}}%
\pgfusepath{clip}%
\pgfsetbuttcap%
\pgfsetroundjoin%
\definecolor{currentfill}{rgb}{0.958254,0.973856,0.009488}%
\pgfsetfillcolor{currentfill}%
\pgfsetlinewidth{0.000000pt}%
\definecolor{currentstroke}{rgb}{0.000000,0.000000,0.000000}%
\pgfsetstrokecolor{currentstroke}%
\pgfsetdash{}{0pt}%
\pgfpathmoveto{\pgfqpoint{1.306929in}{1.492437in}}%
\pgfpathlineto{\pgfqpoint{1.297901in}{1.505861in}}%
\pgfpathlineto{\pgfqpoint{1.366327in}{1.489380in}}%
\pgfpathlineto{\pgfqpoint{1.372820in}{1.476547in}}%
\pgfpathlineto{\pgfqpoint{1.306929in}{1.492437in}}%
\pgfpathclose%
\pgfusepath{fill}%
\end{pgfscope}%
\begin{pgfscope}%
\pgfpathrectangle{\pgfqpoint{0.000000in}{0.000000in}}{\pgfqpoint{3.000000in}{3.000000in}}%
\pgfusepath{clip}%
\pgfsetbuttcap%
\pgfsetroundjoin%
\definecolor{currentfill}{rgb}{1.000000,0.291213,0.000000}%
\pgfsetfillcolor{currentfill}%
\pgfsetlinewidth{0.000000pt}%
\definecolor{currentstroke}{rgb}{0.000000,0.000000,0.000000}%
\pgfsetstrokecolor{currentstroke}%
\pgfsetdash{}{0pt}%
\pgfpathmoveto{\pgfqpoint{0.956326in}{1.746908in}}%
\pgfpathlineto{\pgfqpoint{0.939975in}{1.759316in}}%
\pgfpathlineto{\pgfqpoint{0.986515in}{1.714831in}}%
\pgfpathlineto{\pgfqpoint{1.001671in}{1.703581in}}%
\pgfpathlineto{\pgfqpoint{0.956326in}{1.746908in}}%
\pgfpathclose%
\pgfusepath{fill}%
\end{pgfscope}%
\begin{pgfscope}%
\pgfpathrectangle{\pgfqpoint{0.000000in}{0.000000in}}{\pgfqpoint{3.000000in}{3.000000in}}%
\pgfusepath{clip}%
\pgfsetbuttcap%
\pgfsetroundjoin%
\definecolor{currentfill}{rgb}{0.856506,0.000000,0.000000}%
\pgfsetfillcolor{currentfill}%
\pgfsetlinewidth{0.000000pt}%
\definecolor{currentstroke}{rgb}{0.000000,0.000000,0.000000}%
\pgfsetstrokecolor{currentstroke}%
\pgfsetdash{}{0pt}%
\pgfpathmoveto{\pgfqpoint{0.822963in}{1.870364in}}%
\pgfpathlineto{\pgfqpoint{0.805734in}{1.882382in}}%
\pgfpathlineto{\pgfqpoint{0.841379in}{1.827846in}}%
\pgfpathlineto{\pgfqpoint{0.857870in}{1.817044in}}%
\pgfpathlineto{\pgfqpoint{0.822963in}{1.870364in}}%
\pgfpathclose%
\pgfusepath{fill}%
\end{pgfscope}%
\begin{pgfscope}%
\pgfpathrectangle{\pgfqpoint{0.000000in}{0.000000in}}{\pgfqpoint{3.000000in}{3.000000in}}%
\pgfusepath{clip}%
\pgfsetbuttcap%
\pgfsetroundjoin%
\definecolor{currentfill}{rgb}{1.000000,0.814089,0.000000}%
\pgfsetfillcolor{currentfill}%
\pgfsetlinewidth{0.000000pt}%
\definecolor{currentstroke}{rgb}{0.000000,0.000000,0.000000}%
\pgfsetstrokecolor{currentstroke}%
\pgfsetdash{}{0pt}%
\pgfpathmoveto{\pgfqpoint{1.814498in}{1.525470in}}%
\pgfpathlineto{\pgfqpoint{1.824363in}{1.537776in}}%
\pgfpathlineto{\pgfqpoint{1.888314in}{1.562608in}}%
\pgfpathlineto{\pgfqpoint{1.876257in}{1.549473in}}%
\pgfpathlineto{\pgfqpoint{1.814498in}{1.525470in}}%
\pgfpathclose%
\pgfusepath{fill}%
\end{pgfscope}%
\begin{pgfscope}%
\pgfpathrectangle{\pgfqpoint{0.000000in}{0.000000in}}{\pgfqpoint{3.000000in}{3.000000in}}%
\pgfusepath{clip}%
\pgfsetbuttcap%
\pgfsetroundjoin%
\definecolor{currentfill}{rgb}{1.000000,0.175018,0.000000}%
\pgfsetfillcolor{currentfill}%
\pgfsetlinewidth{0.000000pt}%
\definecolor{currentstroke}{rgb}{0.000000,0.000000,0.000000}%
\pgfsetstrokecolor{currentstroke}%
\pgfsetdash{}{0pt}%
\pgfpathmoveto{\pgfqpoint{2.127363in}{1.740593in}}%
\pgfpathlineto{\pgfqpoint{2.143015in}{1.751614in}}%
\pgfpathlineto{\pgfqpoint{2.186611in}{1.799517in}}%
\pgfpathlineto{\pgfqpoint{2.169913in}{1.787313in}}%
\pgfpathlineto{\pgfqpoint{2.127363in}{1.740593in}}%
\pgfpathclose%
\pgfusepath{fill}%
\end{pgfscope}%
\begin{pgfscope}%
\pgfpathrectangle{\pgfqpoint{0.000000in}{0.000000in}}{\pgfqpoint{3.000000in}{3.000000in}}%
\pgfusepath{clip}%
\pgfsetbuttcap%
\pgfsetroundjoin%
\definecolor{currentfill}{rgb}{0.678253,0.000000,0.000000}%
\pgfsetfillcolor{currentfill}%
\pgfsetlinewidth{0.000000pt}%
\definecolor{currentstroke}{rgb}{0.000000,0.000000,0.000000}%
\pgfsetstrokecolor{currentstroke}%
\pgfsetdash{}{0pt}%
\pgfpathmoveto{\pgfqpoint{2.287265in}{1.867527in}}%
\pgfpathlineto{\pgfqpoint{2.304117in}{1.878123in}}%
\pgfpathlineto{\pgfqpoint{2.335345in}{1.937046in}}%
\pgfpathlineto{\pgfqpoint{2.317916in}{1.925226in}}%
\pgfpathlineto{\pgfqpoint{2.287265in}{1.867527in}}%
\pgfpathclose%
\pgfusepath{fill}%
\end{pgfscope}%
\begin{pgfscope}%
\pgfpathrectangle{\pgfqpoint{0.000000in}{0.000000in}}{\pgfqpoint{3.000000in}{3.000000in}}%
\pgfusepath{clip}%
\pgfsetbuttcap%
\pgfsetroundjoin%
\definecolor{currentfill}{rgb}{0.895003,1.000000,0.072739}%
\pgfsetfillcolor{currentfill}%
\pgfsetlinewidth{0.000000pt}%
\definecolor{currentstroke}{rgb}{0.000000,0.000000,0.000000}%
\pgfsetstrokecolor{currentstroke}%
\pgfsetdash{}{0pt}%
\pgfpathmoveto{\pgfqpoint{1.587641in}{1.449085in}}%
\pgfpathlineto{\pgfqpoint{1.589536in}{1.462167in}}%
\pgfpathlineto{\pgfqpoint{1.661769in}{1.468921in}}%
\pgfpathlineto{\pgfqpoint{1.657084in}{1.455586in}}%
\pgfpathlineto{\pgfqpoint{1.587641in}{1.449085in}}%
\pgfpathclose%
\pgfusepath{fill}%
\end{pgfscope}%
\begin{pgfscope}%
\pgfpathrectangle{\pgfqpoint{0.000000in}{0.000000in}}{\pgfqpoint{3.000000in}{3.000000in}}%
\pgfusepath{clip}%
\pgfsetbuttcap%
\pgfsetroundjoin%
\definecolor{currentfill}{rgb}{0.895003,1.000000,0.072739}%
\pgfsetfillcolor{currentfill}%
\pgfsetlinewidth{0.000000pt}%
\definecolor{currentstroke}{rgb}{0.000000,0.000000,0.000000}%
\pgfsetstrokecolor{currentstroke}%
\pgfsetdash{}{0pt}%
\pgfpathmoveto{\pgfqpoint{1.446892in}{1.452808in}}%
\pgfpathlineto{\pgfqpoint{1.443127in}{1.466035in}}%
\pgfpathlineto{\pgfqpoint{1.516007in}{1.461197in}}%
\pgfpathlineto{\pgfqpoint{1.516956in}{1.448150in}}%
\pgfpathlineto{\pgfqpoint{1.446892in}{1.452808in}}%
\pgfpathclose%
\pgfusepath{fill}%
\end{pgfscope}%
\begin{pgfscope}%
\pgfpathrectangle{\pgfqpoint{0.000000in}{0.000000in}}{\pgfqpoint{3.000000in}{3.000000in}}%
\pgfusepath{clip}%
\pgfsetbuttcap%
\pgfsetroundjoin%
\definecolor{currentfill}{rgb}{0.895003,1.000000,0.072739}%
\pgfsetfillcolor{currentfill}%
\pgfsetlinewidth{0.000000pt}%
\definecolor{currentstroke}{rgb}{0.000000,0.000000,0.000000}%
\pgfsetstrokecolor{currentstroke}%
\pgfsetdash{}{0pt}%
\pgfpathmoveto{\pgfqpoint{1.516956in}{1.448150in}}%
\pgfpathlineto{\pgfqpoint{1.516007in}{1.461197in}}%
\pgfpathlineto{\pgfqpoint{1.589536in}{1.462167in}}%
\pgfpathlineto{\pgfqpoint{1.587641in}{1.449085in}}%
\pgfpathlineto{\pgfqpoint{1.516956in}{1.448150in}}%
\pgfpathclose%
\pgfusepath{fill}%
\end{pgfscope}%
\begin{pgfscope}%
\pgfpathrectangle{\pgfqpoint{0.000000in}{0.000000in}}{\pgfqpoint{3.000000in}{3.000000in}}%
\pgfusepath{clip}%
\pgfsetbuttcap%
\pgfsetroundjoin%
\definecolor{currentfill}{rgb}{0.803030,0.000000,0.000000}%
\pgfsetfillcolor{currentfill}%
\pgfsetlinewidth{0.000000pt}%
\definecolor{currentstroke}{rgb}{0.000000,0.000000,0.000000}%
\pgfsetstrokecolor{currentstroke}%
\pgfsetdash{}{0pt}%
\pgfpathmoveto{\pgfqpoint{0.805734in}{1.882382in}}%
\pgfpathlineto{\pgfqpoint{0.788485in}{1.894185in}}%
\pgfpathlineto{\pgfqpoint{0.824866in}{1.838434in}}%
\pgfpathlineto{\pgfqpoint{0.841379in}{1.827846in}}%
\pgfpathlineto{\pgfqpoint{0.805734in}{1.882382in}}%
\pgfpathclose%
\pgfusepath{fill}%
\end{pgfscope}%
\begin{pgfscope}%
\pgfpathrectangle{\pgfqpoint{0.000000in}{0.000000in}}{\pgfqpoint{3.000000in}{3.000000in}}%
\pgfusepath{clip}%
\pgfsetbuttcap%
\pgfsetroundjoin%
\definecolor{currentfill}{rgb}{1.000000,0.407407,0.000000}%
\pgfsetfillcolor{currentfill}%
\pgfsetlinewidth{0.000000pt}%
\definecolor{currentstroke}{rgb}{0.000000,0.000000,0.000000}%
\pgfsetstrokecolor{currentstroke}%
\pgfsetdash{}{0pt}%
\pgfpathmoveto{\pgfqpoint{2.013159in}{1.654681in}}%
\pgfpathlineto{\pgfqpoint{2.027234in}{1.665907in}}%
\pgfpathlineto{\pgfqpoint{2.080560in}{1.705687in}}%
\pgfpathlineto{\pgfqpoint{2.065008in}{1.693362in}}%
\pgfpathlineto{\pgfqpoint{2.013159in}{1.654681in}}%
\pgfpathclose%
\pgfusepath{fill}%
\end{pgfscope}%
\begin{pgfscope}%
\pgfpathrectangle{\pgfqpoint{0.000000in}{0.000000in}}{\pgfqpoint{3.000000in}{3.000000in}}%
\pgfusepath{clip}%
\pgfsetbuttcap%
\pgfsetroundjoin%
\definecolor{currentfill}{rgb}{1.000000,0.886710,0.000000}%
\pgfsetfillcolor{currentfill}%
\pgfsetlinewidth{0.000000pt}%
\definecolor{currentstroke}{rgb}{0.000000,0.000000,0.000000}%
\pgfsetstrokecolor{currentstroke}%
\pgfsetdash{}{0pt}%
\pgfpathmoveto{\pgfqpoint{1.738165in}{1.494274in}}%
\pgfpathlineto{\pgfqpoint{1.745544in}{1.506571in}}%
\pgfpathlineto{\pgfqpoint{1.814498in}{1.525470in}}%
\pgfpathlineto{\pgfqpoint{1.804655in}{1.512518in}}%
\pgfpathlineto{\pgfqpoint{1.738165in}{1.494274in}}%
\pgfpathclose%
\pgfusepath{fill}%
\end{pgfscope}%
\begin{pgfscope}%
\pgfpathrectangle{\pgfqpoint{0.000000in}{0.000000in}}{\pgfqpoint{3.000000in}{3.000000in}}%
\pgfusepath{clip}%
\pgfsetbuttcap%
\pgfsetroundjoin%
\definecolor{currentfill}{rgb}{1.000000,0.814089,0.000000}%
\pgfsetfillcolor{currentfill}%
\pgfsetlinewidth{0.000000pt}%
\definecolor{currentstroke}{rgb}{0.000000,0.000000,0.000000}%
\pgfsetstrokecolor{currentstroke}%
\pgfsetdash{}{0pt}%
\pgfpathmoveto{\pgfqpoint{1.224510in}{1.540943in}}%
\pgfpathlineto{\pgfqpoint{1.213149in}{1.553784in}}%
\pgfpathlineto{\pgfqpoint{1.279782in}{1.530641in}}%
\pgfpathlineto{\pgfqpoint{1.288852in}{1.518574in}}%
\pgfpathlineto{\pgfqpoint{1.224510in}{1.540943in}}%
\pgfpathclose%
\pgfusepath{fill}%
\end{pgfscope}%
\begin{pgfscope}%
\pgfpathrectangle{\pgfqpoint{0.000000in}{0.000000in}}{\pgfqpoint{3.000000in}{3.000000in}}%
\pgfusepath{clip}%
\pgfsetbuttcap%
\pgfsetroundjoin%
\definecolor{currentfill}{rgb}{1.000000,0.668845,0.000000}%
\pgfsetfillcolor{currentfill}%
\pgfsetlinewidth{0.000000pt}%
\definecolor{currentstroke}{rgb}{0.000000,0.000000,0.000000}%
\pgfsetstrokecolor{currentstroke}%
\pgfsetdash{}{0pt}%
\pgfpathmoveto{\pgfqpoint{1.141728in}{1.594941in}}%
\pgfpathlineto{\pgfqpoint{1.128339in}{1.607586in}}%
\pgfpathlineto{\pgfqpoint{1.190354in}{1.577749in}}%
\pgfpathlineto{\pgfqpoint{1.201764in}{1.566036in}}%
\pgfpathlineto{\pgfqpoint{1.141728in}{1.594941in}}%
\pgfpathclose%
\pgfusepath{fill}%
\end{pgfscope}%
\begin{pgfscope}%
\pgfpathrectangle{\pgfqpoint{0.000000in}{0.000000in}}{\pgfqpoint{3.000000in}{3.000000in}}%
\pgfusepath{clip}%
\pgfsetbuttcap%
\pgfsetroundjoin%
\definecolor{currentfill}{rgb}{0.958254,0.973856,0.009488}%
\pgfsetfillcolor{currentfill}%
\pgfsetlinewidth{0.000000pt}%
\definecolor{currentstroke}{rgb}{0.000000,0.000000,0.000000}%
\pgfsetstrokecolor{currentstroke}%
\pgfsetdash{}{0pt}%
\pgfpathmoveto{\pgfqpoint{1.661769in}{1.468921in}}%
\pgfpathlineto{\pgfqpoint{1.666466in}{1.481468in}}%
\pgfpathlineto{\pgfqpoint{1.738165in}{1.494274in}}%
\pgfpathlineto{\pgfqpoint{1.730803in}{1.481265in}}%
\pgfpathlineto{\pgfqpoint{1.661769in}{1.468921in}}%
\pgfpathclose%
\pgfusepath{fill}%
\end{pgfscope}%
\begin{pgfscope}%
\pgfpathrectangle{\pgfqpoint{0.000000in}{0.000000in}}{\pgfqpoint{3.000000in}{3.000000in}}%
\pgfusepath{clip}%
\pgfsetbuttcap%
\pgfsetroundjoin%
\definecolor{currentfill}{rgb}{1.000000,0.233115,0.000000}%
\pgfsetfillcolor{currentfill}%
\pgfsetlinewidth{0.000000pt}%
\definecolor{currentstroke}{rgb}{0.000000,0.000000,0.000000}%
\pgfsetstrokecolor{currentstroke}%
\pgfsetdash{}{0pt}%
\pgfpathmoveto{\pgfqpoint{0.939975in}{1.759316in}}%
\pgfpathlineto{\pgfqpoint{0.923600in}{1.771412in}}%
\pgfpathlineto{\pgfqpoint{0.971334in}{1.725768in}}%
\pgfpathlineto{\pgfqpoint{0.986515in}{1.714831in}}%
\pgfpathlineto{\pgfqpoint{0.939975in}{1.759316in}}%
\pgfpathclose%
\pgfusepath{fill}%
\end{pgfscope}%
\begin{pgfscope}%
\pgfpathrectangle{\pgfqpoint{0.000000in}{0.000000in}}{\pgfqpoint{3.000000in}{3.000000in}}%
\pgfusepath{clip}%
\pgfsetbuttcap%
\pgfsetroundjoin%
\definecolor{currentfill}{rgb}{0.606952,0.000000,0.000000}%
\pgfsetfillcolor{currentfill}%
\pgfsetlinewidth{0.000000pt}%
\definecolor{currentstroke}{rgb}{0.000000,0.000000,0.000000}%
\pgfsetstrokecolor{currentstroke}%
\pgfsetdash{}{0pt}%
\pgfpathmoveto{\pgfqpoint{2.304117in}{1.878123in}}%
\pgfpathlineto{\pgfqpoint{2.320992in}{1.888537in}}%
\pgfpathlineto{\pgfqpoint{2.352793in}{1.948685in}}%
\pgfpathlineto{\pgfqpoint{2.335345in}{1.937046in}}%
\pgfpathlineto{\pgfqpoint{2.304117in}{1.878123in}}%
\pgfpathclose%
\pgfusepath{fill}%
\end{pgfscope}%
\begin{pgfscope}%
\pgfpathrectangle{\pgfqpoint{0.000000in}{0.000000in}}{\pgfqpoint{3.000000in}{3.000000in}}%
\pgfusepath{clip}%
\pgfsetbuttcap%
\pgfsetroundjoin%
\definecolor{currentfill}{rgb}{1.000000,0.480029,0.000000}%
\pgfsetfillcolor{currentfill}%
\pgfsetlinewidth{0.000000pt}%
\definecolor{currentstroke}{rgb}{0.000000,0.000000,0.000000}%
\pgfsetstrokecolor{currentstroke}%
\pgfsetdash{}{0pt}%
\pgfpathmoveto{\pgfqpoint{1.046983in}{1.667709in}}%
\pgfpathlineto{\pgfqpoint{1.031904in}{1.680046in}}%
\pgfpathlineto{\pgfqpoint{1.088014in}{1.642707in}}%
\pgfpathlineto{\pgfqpoint{1.101482in}{1.631433in}}%
\pgfpathlineto{\pgfqpoint{1.046983in}{1.667709in}}%
\pgfpathclose%
\pgfusepath{fill}%
\end{pgfscope}%
\begin{pgfscope}%
\pgfpathrectangle{\pgfqpoint{0.000000in}{0.000000in}}{\pgfqpoint{3.000000in}{3.000000in}}%
\pgfusepath{clip}%
\pgfsetbuttcap%
\pgfsetroundjoin%
\definecolor{currentfill}{rgb}{1.000000,0.610748,0.000000}%
\pgfsetfillcolor{currentfill}%
\pgfsetlinewidth{0.000000pt}%
\definecolor{currentstroke}{rgb}{0.000000,0.000000,0.000000}%
\pgfsetstrokecolor{currentstroke}%
\pgfsetdash{}{0pt}%
\pgfpathmoveto{\pgfqpoint{1.912504in}{1.587162in}}%
\pgfpathlineto{\pgfqpoint{1.924637in}{1.598674in}}%
\pgfpathlineto{\pgfqpoint{1.985089in}{1.631027in}}%
\pgfpathlineto{\pgfqpoint{1.971094in}{1.618535in}}%
\pgfpathlineto{\pgfqpoint{1.912504in}{1.587162in}}%
\pgfpathclose%
\pgfusepath{fill}%
\end{pgfscope}%
\begin{pgfscope}%
\pgfpathrectangle{\pgfqpoint{0.000000in}{0.000000in}}{\pgfqpoint{3.000000in}{3.000000in}}%
\pgfusepath{clip}%
\pgfsetbuttcap%
\pgfsetroundjoin%
\definecolor{currentfill}{rgb}{1.000000,0.116921,0.000000}%
\pgfsetfillcolor{currentfill}%
\pgfsetlinewidth{0.000000pt}%
\definecolor{currentstroke}{rgb}{0.000000,0.000000,0.000000}%
\pgfsetstrokecolor{currentstroke}%
\pgfsetdash{}{0pt}%
\pgfpathmoveto{\pgfqpoint{2.143015in}{1.751614in}}%
\pgfpathlineto{\pgfqpoint{2.158691in}{1.762362in}}%
\pgfpathlineto{\pgfqpoint{2.203332in}{1.811446in}}%
\pgfpathlineto{\pgfqpoint{2.186611in}{1.799517in}}%
\pgfpathlineto{\pgfqpoint{2.143015in}{1.751614in}}%
\pgfpathclose%
\pgfusepath{fill}%
\end{pgfscope}%
\begin{pgfscope}%
\pgfpathrectangle{\pgfqpoint{0.000000in}{0.000000in}}{\pgfqpoint{3.000000in}{3.000000in}}%
\pgfusepath{clip}%
\pgfsetbuttcap%
\pgfsetroundjoin%
\definecolor{currentfill}{rgb}{0.958254,0.973856,0.009488}%
\pgfsetfillcolor{currentfill}%
\pgfsetlinewidth{0.000000pt}%
\definecolor{currentstroke}{rgb}{0.000000,0.000000,0.000000}%
\pgfsetstrokecolor{currentstroke}%
\pgfsetdash{}{0pt}%
\pgfpathmoveto{\pgfqpoint{1.372820in}{1.476547in}}%
\pgfpathlineto{\pgfqpoint{1.366327in}{1.489380in}}%
\pgfpathlineto{\pgfqpoint{1.439351in}{1.478476in}}%
\pgfpathlineto{\pgfqpoint{1.443127in}{1.466035in}}%
\pgfpathlineto{\pgfqpoint{1.372820in}{1.476547in}}%
\pgfpathclose%
\pgfusepath{fill}%
\end{pgfscope}%
\begin{pgfscope}%
\pgfpathrectangle{\pgfqpoint{0.000000in}{0.000000in}}{\pgfqpoint{3.000000in}{3.000000in}}%
\pgfusepath{clip}%
\pgfsetbuttcap%
\pgfsetroundjoin%
\definecolor{currentfill}{rgb}{1.000000,0.741467,0.000000}%
\pgfsetfillcolor{currentfill}%
\pgfsetlinewidth{0.000000pt}%
\definecolor{currentstroke}{rgb}{0.000000,0.000000,0.000000}%
\pgfsetstrokecolor{currentstroke}%
\pgfsetdash{}{0pt}%
\pgfpathmoveto{\pgfqpoint{1.824363in}{1.537776in}}%
\pgfpathlineto{\pgfqpoint{1.834251in}{1.549492in}}%
\pgfpathlineto{\pgfqpoint{1.900396in}{1.575154in}}%
\pgfpathlineto{\pgfqpoint{1.888314in}{1.562608in}}%
\pgfpathlineto{\pgfqpoint{1.824363in}{1.537776in}}%
\pgfpathclose%
\pgfusepath{fill}%
\end{pgfscope}%
\begin{pgfscope}%
\pgfpathrectangle{\pgfqpoint{0.000000in}{0.000000in}}{\pgfqpoint{3.000000in}{3.000000in}}%
\pgfusepath{clip}%
\pgfsetbuttcap%
\pgfsetroundjoin%
\definecolor{currentfill}{rgb}{1.000000,0.886710,0.000000}%
\pgfsetfillcolor{currentfill}%
\pgfsetlinewidth{0.000000pt}%
\definecolor{currentstroke}{rgb}{0.000000,0.000000,0.000000}%
\pgfsetstrokecolor{currentstroke}%
\pgfsetdash{}{0pt}%
\pgfpathmoveto{\pgfqpoint{1.297901in}{1.505861in}}%
\pgfpathlineto{\pgfqpoint{1.288852in}{1.518574in}}%
\pgfpathlineto{\pgfqpoint{1.359818in}{1.501501in}}%
\pgfpathlineto{\pgfqpoint{1.366327in}{1.489380in}}%
\pgfpathlineto{\pgfqpoint{1.297901in}{1.505861in}}%
\pgfpathclose%
\pgfusepath{fill}%
\end{pgfscope}%
\begin{pgfscope}%
\pgfpathrectangle{\pgfqpoint{0.000000in}{0.000000in}}{\pgfqpoint{3.000000in}{3.000000in}}%
\pgfusepath{clip}%
\pgfsetbuttcap%
\pgfsetroundjoin%
\definecolor{currentfill}{rgb}{0.731729,0.000000,0.000000}%
\pgfsetfillcolor{currentfill}%
\pgfsetlinewidth{0.000000pt}%
\definecolor{currentstroke}{rgb}{0.000000,0.000000,0.000000}%
\pgfsetstrokecolor{currentstroke}%
\pgfsetdash{}{0pt}%
\pgfpathmoveto{\pgfqpoint{0.788485in}{1.894185in}}%
\pgfpathlineto{\pgfqpoint{0.771216in}{1.905786in}}%
\pgfpathlineto{\pgfqpoint{0.808329in}{1.848820in}}%
\pgfpathlineto{\pgfqpoint{0.824866in}{1.838434in}}%
\pgfpathlineto{\pgfqpoint{0.788485in}{1.894185in}}%
\pgfpathclose%
\pgfusepath{fill}%
\end{pgfscope}%
\begin{pgfscope}%
\pgfpathrectangle{\pgfqpoint{0.000000in}{0.000000in}}{\pgfqpoint{3.000000in}{3.000000in}}%
\pgfusepath{clip}%
\pgfsetbuttcap%
\pgfsetroundjoin%
\definecolor{currentfill}{rgb}{0.553476,0.000000,0.000000}%
\pgfsetfillcolor{currentfill}%
\pgfsetlinewidth{0.000000pt}%
\definecolor{currentstroke}{rgb}{0.000000,0.000000,0.000000}%
\pgfsetstrokecolor{currentstroke}%
\pgfsetdash{}{0pt}%
\pgfpathmoveto{\pgfqpoint{2.320992in}{1.888537in}}%
\pgfpathlineto{\pgfqpoint{2.337888in}{1.898781in}}%
\pgfpathlineto{\pgfqpoint{2.370258in}{1.960151in}}%
\pgfpathlineto{\pgfqpoint{2.352793in}{1.948685in}}%
\pgfpathlineto{\pgfqpoint{2.320992in}{1.888537in}}%
\pgfpathclose%
\pgfusepath{fill}%
\end{pgfscope}%
\begin{pgfscope}%
\pgfpathrectangle{\pgfqpoint{0.000000in}{0.000000in}}{\pgfqpoint{3.000000in}{3.000000in}}%
\pgfusepath{clip}%
\pgfsetbuttcap%
\pgfsetroundjoin%
\definecolor{currentfill}{rgb}{1.000000,0.175018,0.000000}%
\pgfsetfillcolor{currentfill}%
\pgfsetlinewidth{0.000000pt}%
\definecolor{currentstroke}{rgb}{0.000000,0.000000,0.000000}%
\pgfsetstrokecolor{currentstroke}%
\pgfsetdash{}{0pt}%
\pgfpathmoveto{\pgfqpoint{0.923600in}{1.771412in}}%
\pgfpathlineto{\pgfqpoint{0.907202in}{1.783214in}}%
\pgfpathlineto{\pgfqpoint{0.956127in}{1.736412in}}%
\pgfpathlineto{\pgfqpoint{0.971334in}{1.725768in}}%
\pgfpathlineto{\pgfqpoint{0.923600in}{1.771412in}}%
\pgfpathclose%
\pgfusepath{fill}%
\end{pgfscope}%
\begin{pgfscope}%
\pgfpathrectangle{\pgfqpoint{0.000000in}{0.000000in}}{\pgfqpoint{3.000000in}{3.000000in}}%
\pgfusepath{clip}%
\pgfsetbuttcap%
\pgfsetroundjoin%
\definecolor{currentfill}{rgb}{1.000000,0.349310,0.000000}%
\pgfsetfillcolor{currentfill}%
\pgfsetlinewidth{0.000000pt}%
\definecolor{currentstroke}{rgb}{0.000000,0.000000,0.000000}%
\pgfsetstrokecolor{currentstroke}%
\pgfsetdash{}{0pt}%
\pgfpathmoveto{\pgfqpoint{2.027234in}{1.665907in}}%
\pgfpathlineto{\pgfqpoint{2.041334in}{1.676772in}}%
\pgfpathlineto{\pgfqpoint{2.096136in}{1.717651in}}%
\pgfpathlineto{\pgfqpoint{2.080560in}{1.705687in}}%
\pgfpathlineto{\pgfqpoint{2.027234in}{1.665907in}}%
\pgfpathclose%
\pgfusepath{fill}%
\end{pgfscope}%
\begin{pgfscope}%
\pgfpathrectangle{\pgfqpoint{0.000000in}{0.000000in}}{\pgfqpoint{3.000000in}{3.000000in}}%
\pgfusepath{clip}%
\pgfsetbuttcap%
\pgfsetroundjoin%
\definecolor{currentfill}{rgb}{0.958254,0.973856,0.009488}%
\pgfsetfillcolor{currentfill}%
\pgfsetlinewidth{0.000000pt}%
\definecolor{currentstroke}{rgb}{0.000000,0.000000,0.000000}%
\pgfsetstrokecolor{currentstroke}%
\pgfsetdash{}{0pt}%
\pgfpathmoveto{\pgfqpoint{1.589536in}{1.462167in}}%
\pgfpathlineto{\pgfqpoint{1.591436in}{1.474463in}}%
\pgfpathlineto{\pgfqpoint{1.666466in}{1.481468in}}%
\pgfpathlineto{\pgfqpoint{1.661769in}{1.468921in}}%
\pgfpathlineto{\pgfqpoint{1.589536in}{1.462167in}}%
\pgfpathclose%
\pgfusepath{fill}%
\end{pgfscope}%
\begin{pgfscope}%
\pgfpathrectangle{\pgfqpoint{0.000000in}{0.000000in}}{\pgfqpoint{3.000000in}{3.000000in}}%
\pgfusepath{clip}%
\pgfsetbuttcap%
\pgfsetroundjoin%
\definecolor{currentfill}{rgb}{0.999109,0.073348,0.000000}%
\pgfsetfillcolor{currentfill}%
\pgfsetlinewidth{0.000000pt}%
\definecolor{currentstroke}{rgb}{0.000000,0.000000,0.000000}%
\pgfsetstrokecolor{currentstroke}%
\pgfsetdash{}{0pt}%
\pgfpathmoveto{\pgfqpoint{2.158691in}{1.762362in}}%
\pgfpathlineto{\pgfqpoint{2.174392in}{1.772851in}}%
\pgfpathlineto{\pgfqpoint{2.220074in}{1.823118in}}%
\pgfpathlineto{\pgfqpoint{2.203332in}{1.811446in}}%
\pgfpathlineto{\pgfqpoint{2.158691in}{1.762362in}}%
\pgfpathclose%
\pgfusepath{fill}%
\end{pgfscope}%
\begin{pgfscope}%
\pgfpathrectangle{\pgfqpoint{0.000000in}{0.000000in}}{\pgfqpoint{3.000000in}{3.000000in}}%
\pgfusepath{clip}%
\pgfsetbuttcap%
\pgfsetroundjoin%
\definecolor{currentfill}{rgb}{0.958254,0.973856,0.009488}%
\pgfsetfillcolor{currentfill}%
\pgfsetlinewidth{0.000000pt}%
\definecolor{currentstroke}{rgb}{0.000000,0.000000,0.000000}%
\pgfsetstrokecolor{currentstroke}%
\pgfsetdash{}{0pt}%
\pgfpathmoveto{\pgfqpoint{1.443127in}{1.466035in}}%
\pgfpathlineto{\pgfqpoint{1.439351in}{1.478476in}}%
\pgfpathlineto{\pgfqpoint{1.515055in}{1.473456in}}%
\pgfpathlineto{\pgfqpoint{1.516007in}{1.461197in}}%
\pgfpathlineto{\pgfqpoint{1.443127in}{1.466035in}}%
\pgfpathclose%
\pgfusepath{fill}%
\end{pgfscope}%
\begin{pgfscope}%
\pgfpathrectangle{\pgfqpoint{0.000000in}{0.000000in}}{\pgfqpoint{3.000000in}{3.000000in}}%
\pgfusepath{clip}%
\pgfsetbuttcap%
\pgfsetroundjoin%
\definecolor{currentfill}{rgb}{1.000000,0.610748,0.000000}%
\pgfsetfillcolor{currentfill}%
\pgfsetlinewidth{0.000000pt}%
\definecolor{currentstroke}{rgb}{0.000000,0.000000,0.000000}%
\pgfsetstrokecolor{currentstroke}%
\pgfsetdash{}{0pt}%
\pgfpathmoveto{\pgfqpoint{1.128339in}{1.607586in}}%
\pgfpathlineto{\pgfqpoint{1.114924in}{1.619738in}}%
\pgfpathlineto{\pgfqpoint{1.178919in}{1.588966in}}%
\pgfpathlineto{\pgfqpoint{1.190354in}{1.577749in}}%
\pgfpathlineto{\pgfqpoint{1.128339in}{1.607586in}}%
\pgfpathclose%
\pgfusepath{fill}%
\end{pgfscope}%
\begin{pgfscope}%
\pgfpathrectangle{\pgfqpoint{0.000000in}{0.000000in}}{\pgfqpoint{3.000000in}{3.000000in}}%
\pgfusepath{clip}%
\pgfsetbuttcap%
\pgfsetroundjoin%
\definecolor{currentfill}{rgb}{1.000000,0.407407,0.000000}%
\pgfsetfillcolor{currentfill}%
\pgfsetlinewidth{0.000000pt}%
\definecolor{currentstroke}{rgb}{0.000000,0.000000,0.000000}%
\pgfsetstrokecolor{currentstroke}%
\pgfsetdash{}{0pt}%
\pgfpathmoveto{\pgfqpoint{1.031904in}{1.680046in}}%
\pgfpathlineto{\pgfqpoint{1.016800in}{1.691994in}}%
\pgfpathlineto{\pgfqpoint{1.074520in}{1.653592in}}%
\pgfpathlineto{\pgfqpoint{1.088014in}{1.642707in}}%
\pgfpathlineto{\pgfqpoint{1.031904in}{1.680046in}}%
\pgfpathclose%
\pgfusepath{fill}%
\end{pgfscope}%
\begin{pgfscope}%
\pgfpathrectangle{\pgfqpoint{0.000000in}{0.000000in}}{\pgfqpoint{3.000000in}{3.000000in}}%
\pgfusepath{clip}%
\pgfsetbuttcap%
\pgfsetroundjoin%
\definecolor{currentfill}{rgb}{1.000000,0.741467,0.000000}%
\pgfsetfillcolor{currentfill}%
\pgfsetlinewidth{0.000000pt}%
\definecolor{currentstroke}{rgb}{0.000000,0.000000,0.000000}%
\pgfsetstrokecolor{currentstroke}%
\pgfsetdash{}{0pt}%
\pgfpathmoveto{\pgfqpoint{1.213149in}{1.553784in}}%
\pgfpathlineto{\pgfqpoint{1.201764in}{1.566036in}}%
\pgfpathlineto{\pgfqpoint{1.270690in}{1.542118in}}%
\pgfpathlineto{\pgfqpoint{1.279782in}{1.530641in}}%
\pgfpathlineto{\pgfqpoint{1.213149in}{1.553784in}}%
\pgfpathclose%
\pgfusepath{fill}%
\end{pgfscope}%
\begin{pgfscope}%
\pgfpathrectangle{\pgfqpoint{0.000000in}{0.000000in}}{\pgfqpoint{3.000000in}{3.000000in}}%
\pgfusepath{clip}%
\pgfsetbuttcap%
\pgfsetroundjoin%
\definecolor{currentfill}{rgb}{1.000000,0.538126,0.000000}%
\pgfsetfillcolor{currentfill}%
\pgfsetlinewidth{0.000000pt}%
\definecolor{currentstroke}{rgb}{0.000000,0.000000,0.000000}%
\pgfsetstrokecolor{currentstroke}%
\pgfsetdash{}{0pt}%
\pgfpathmoveto{\pgfqpoint{1.924637in}{1.598674in}}%
\pgfpathlineto{\pgfqpoint{1.936796in}{1.609730in}}%
\pgfpathlineto{\pgfqpoint{1.999111in}{1.643064in}}%
\pgfpathlineto{\pgfqpoint{1.985089in}{1.631027in}}%
\pgfpathlineto{\pgfqpoint{1.924637in}{1.598674in}}%
\pgfpathclose%
\pgfusepath{fill}%
\end{pgfscope}%
\begin{pgfscope}%
\pgfpathrectangle{\pgfqpoint{0.000000in}{0.000000in}}{\pgfqpoint{3.000000in}{3.000000in}}%
\pgfusepath{clip}%
\pgfsetbuttcap%
\pgfsetroundjoin%
\definecolor{currentfill}{rgb}{1.000000,0.814089,0.000000}%
\pgfsetfillcolor{currentfill}%
\pgfsetlinewidth{0.000000pt}%
\definecolor{currentstroke}{rgb}{0.000000,0.000000,0.000000}%
\pgfsetstrokecolor{currentstroke}%
\pgfsetdash{}{0pt}%
\pgfpathmoveto{\pgfqpoint{1.745544in}{1.506571in}}%
\pgfpathlineto{\pgfqpoint{1.752942in}{1.518220in}}%
\pgfpathlineto{\pgfqpoint{1.824363in}{1.537776in}}%
\pgfpathlineto{\pgfqpoint{1.814498in}{1.525470in}}%
\pgfpathlineto{\pgfqpoint{1.745544in}{1.506571in}}%
\pgfpathclose%
\pgfusepath{fill}%
\end{pgfscope}%
\begin{pgfscope}%
\pgfpathrectangle{\pgfqpoint{0.000000in}{0.000000in}}{\pgfqpoint{3.000000in}{3.000000in}}%
\pgfusepath{clip}%
\pgfsetbuttcap%
\pgfsetroundjoin%
\definecolor{currentfill}{rgb}{0.958254,0.973856,0.009488}%
\pgfsetfillcolor{currentfill}%
\pgfsetlinewidth{0.000000pt}%
\definecolor{currentstroke}{rgb}{0.000000,0.000000,0.000000}%
\pgfsetstrokecolor{currentstroke}%
\pgfsetdash{}{0pt}%
\pgfpathmoveto{\pgfqpoint{1.516007in}{1.461197in}}%
\pgfpathlineto{\pgfqpoint{1.515055in}{1.473456in}}%
\pgfpathlineto{\pgfqpoint{1.591436in}{1.474463in}}%
\pgfpathlineto{\pgfqpoint{1.589536in}{1.462167in}}%
\pgfpathlineto{\pgfqpoint{1.516007in}{1.461197in}}%
\pgfpathclose%
\pgfusepath{fill}%
\end{pgfscope}%
\begin{pgfscope}%
\pgfpathrectangle{\pgfqpoint{0.000000in}{0.000000in}}{\pgfqpoint{3.000000in}{3.000000in}}%
\pgfusepath{clip}%
\pgfsetbuttcap%
\pgfsetroundjoin%
\definecolor{currentfill}{rgb}{0.678253,0.000000,0.000000}%
\pgfsetfillcolor{currentfill}%
\pgfsetlinewidth{0.000000pt}%
\definecolor{currentstroke}{rgb}{0.000000,0.000000,0.000000}%
\pgfsetstrokecolor{currentstroke}%
\pgfsetdash{}{0pt}%
\pgfpathmoveto{\pgfqpoint{0.771216in}{1.905786in}}%
\pgfpathlineto{\pgfqpoint{0.753929in}{1.917196in}}%
\pgfpathlineto{\pgfqpoint{0.791769in}{1.859015in}}%
\pgfpathlineto{\pgfqpoint{0.808329in}{1.848820in}}%
\pgfpathlineto{\pgfqpoint{0.771216in}{1.905786in}}%
\pgfpathclose%
\pgfusepath{fill}%
\end{pgfscope}%
\begin{pgfscope}%
\pgfpathrectangle{\pgfqpoint{0.000000in}{0.000000in}}{\pgfqpoint{3.000000in}{3.000000in}}%
\pgfusepath{clip}%
\pgfsetbuttcap%
\pgfsetroundjoin%
\definecolor{currentfill}{rgb}{0.500000,0.000000,0.000000}%
\pgfsetfillcolor{currentfill}%
\pgfsetlinewidth{0.000000pt}%
\definecolor{currentstroke}{rgb}{0.000000,0.000000,0.000000}%
\pgfsetstrokecolor{currentstroke}%
\pgfsetdash{}{0pt}%
\pgfpathmoveto{\pgfqpoint{2.337888in}{1.898781in}}%
\pgfpathlineto{\pgfqpoint{2.354806in}{1.908861in}}%
\pgfpathlineto{\pgfqpoint{2.387740in}{1.971455in}}%
\pgfpathlineto{\pgfqpoint{2.370258in}{1.960151in}}%
\pgfpathlineto{\pgfqpoint{2.337888in}{1.898781in}}%
\pgfpathclose%
\pgfusepath{fill}%
\end{pgfscope}%
\begin{pgfscope}%
\pgfpathrectangle{\pgfqpoint{0.000000in}{0.000000in}}{\pgfqpoint{3.000000in}{3.000000in}}%
\pgfusepath{clip}%
\pgfsetbuttcap%
\pgfsetroundjoin%
\definecolor{currentfill}{rgb}{1.000000,0.886710,0.000000}%
\pgfsetfillcolor{currentfill}%
\pgfsetlinewidth{0.000000pt}%
\definecolor{currentstroke}{rgb}{0.000000,0.000000,0.000000}%
\pgfsetstrokecolor{currentstroke}%
\pgfsetdash{}{0pt}%
\pgfpathmoveto{\pgfqpoint{1.666466in}{1.481468in}}%
\pgfpathlineto{\pgfqpoint{1.671175in}{1.493304in}}%
\pgfpathlineto{\pgfqpoint{1.745544in}{1.506571in}}%
\pgfpathlineto{\pgfqpoint{1.738165in}{1.494274in}}%
\pgfpathlineto{\pgfqpoint{1.666466in}{1.481468in}}%
\pgfpathclose%
\pgfusepath{fill}%
\end{pgfscope}%
\begin{pgfscope}%
\pgfpathrectangle{\pgfqpoint{0.000000in}{0.000000in}}{\pgfqpoint{3.000000in}{3.000000in}}%
\pgfusepath{clip}%
\pgfsetbuttcap%
\pgfsetroundjoin%
\definecolor{currentfill}{rgb}{1.000000,0.668845,0.000000}%
\pgfsetfillcolor{currentfill}%
\pgfsetlinewidth{0.000000pt}%
\definecolor{currentstroke}{rgb}{0.000000,0.000000,0.000000}%
\pgfsetstrokecolor{currentstroke}%
\pgfsetdash{}{0pt}%
\pgfpathmoveto{\pgfqpoint{1.834251in}{1.549492in}}%
\pgfpathlineto{\pgfqpoint{1.844161in}{1.560668in}}%
\pgfpathlineto{\pgfqpoint{1.912504in}{1.587162in}}%
\pgfpathlineto{\pgfqpoint{1.900396in}{1.575154in}}%
\pgfpathlineto{\pgfqpoint{1.834251in}{1.549492in}}%
\pgfpathclose%
\pgfusepath{fill}%
\end{pgfscope}%
\begin{pgfscope}%
\pgfpathrectangle{\pgfqpoint{0.000000in}{0.000000in}}{\pgfqpoint{3.000000in}{3.000000in}}%
\pgfusepath{clip}%
\pgfsetbuttcap%
\pgfsetroundjoin%
\definecolor{currentfill}{rgb}{1.000000,0.116921,0.000000}%
\pgfsetfillcolor{currentfill}%
\pgfsetlinewidth{0.000000pt}%
\definecolor{currentstroke}{rgb}{0.000000,0.000000,0.000000}%
\pgfsetstrokecolor{currentstroke}%
\pgfsetdash{}{0pt}%
\pgfpathmoveto{\pgfqpoint{0.907202in}{1.783214in}}%
\pgfpathlineto{\pgfqpoint{0.890781in}{1.794743in}}%
\pgfpathlineto{\pgfqpoint{0.940895in}{1.746781in}}%
\pgfpathlineto{\pgfqpoint{0.956127in}{1.736412in}}%
\pgfpathlineto{\pgfqpoint{0.907202in}{1.783214in}}%
\pgfpathclose%
\pgfusepath{fill}%
\end{pgfscope}%
\begin{pgfscope}%
\pgfpathrectangle{\pgfqpoint{0.000000in}{0.000000in}}{\pgfqpoint{3.000000in}{3.000000in}}%
\pgfusepath{clip}%
\pgfsetbuttcap%
\pgfsetroundjoin%
\definecolor{currentfill}{rgb}{1.000000,0.886710,0.000000}%
\pgfsetfillcolor{currentfill}%
\pgfsetlinewidth{0.000000pt}%
\definecolor{currentstroke}{rgb}{0.000000,0.000000,0.000000}%
\pgfsetstrokecolor{currentstroke}%
\pgfsetdash{}{0pt}%
\pgfpathmoveto{\pgfqpoint{1.366327in}{1.489380in}}%
\pgfpathlineto{\pgfqpoint{1.359818in}{1.501501in}}%
\pgfpathlineto{\pgfqpoint{1.435566in}{1.490203in}}%
\pgfpathlineto{\pgfqpoint{1.439351in}{1.478476in}}%
\pgfpathlineto{\pgfqpoint{1.366327in}{1.489380in}}%
\pgfpathclose%
\pgfusepath{fill}%
\end{pgfscope}%
\begin{pgfscope}%
\pgfpathrectangle{\pgfqpoint{0.000000in}{0.000000in}}{\pgfqpoint{3.000000in}{3.000000in}}%
\pgfusepath{clip}%
\pgfsetbuttcap%
\pgfsetroundjoin%
\definecolor{currentfill}{rgb}{0.927807,0.015251,0.000000}%
\pgfsetfillcolor{currentfill}%
\pgfsetlinewidth{0.000000pt}%
\definecolor{currentstroke}{rgb}{0.000000,0.000000,0.000000}%
\pgfsetstrokecolor{currentstroke}%
\pgfsetdash{}{0pt}%
\pgfpathmoveto{\pgfqpoint{2.174392in}{1.772851in}}%
\pgfpathlineto{\pgfqpoint{2.190118in}{1.783100in}}%
\pgfpathlineto{\pgfqpoint{2.236839in}{1.834548in}}%
\pgfpathlineto{\pgfqpoint{2.220074in}{1.823118in}}%
\pgfpathlineto{\pgfqpoint{2.174392in}{1.772851in}}%
\pgfpathclose%
\pgfusepath{fill}%
\end{pgfscope}%
\begin{pgfscope}%
\pgfpathrectangle{\pgfqpoint{0.000000in}{0.000000in}}{\pgfqpoint{3.000000in}{3.000000in}}%
\pgfusepath{clip}%
\pgfsetbuttcap%
\pgfsetroundjoin%
\definecolor{currentfill}{rgb}{1.000000,0.814089,0.000000}%
\pgfsetfillcolor{currentfill}%
\pgfsetlinewidth{0.000000pt}%
\definecolor{currentstroke}{rgb}{0.000000,0.000000,0.000000}%
\pgfsetstrokecolor{currentstroke}%
\pgfsetdash{}{0pt}%
\pgfpathmoveto{\pgfqpoint{1.288852in}{1.518574in}}%
\pgfpathlineto{\pgfqpoint{1.279782in}{1.530641in}}%
\pgfpathlineto{\pgfqpoint{1.353293in}{1.512974in}}%
\pgfpathlineto{\pgfqpoint{1.359818in}{1.501501in}}%
\pgfpathlineto{\pgfqpoint{1.288852in}{1.518574in}}%
\pgfpathclose%
\pgfusepath{fill}%
\end{pgfscope}%
\begin{pgfscope}%
\pgfpathrectangle{\pgfqpoint{0.000000in}{0.000000in}}{\pgfqpoint{3.000000in}{3.000000in}}%
\pgfusepath{clip}%
\pgfsetbuttcap%
\pgfsetroundjoin%
\definecolor{currentfill}{rgb}{1.000000,0.291213,0.000000}%
\pgfsetfillcolor{currentfill}%
\pgfsetlinewidth{0.000000pt}%
\definecolor{currentstroke}{rgb}{0.000000,0.000000,0.000000}%
\pgfsetstrokecolor{currentstroke}%
\pgfsetdash{}{0pt}%
\pgfpathmoveto{\pgfqpoint{2.041334in}{1.676772in}}%
\pgfpathlineto{\pgfqpoint{2.055461in}{1.687300in}}%
\pgfpathlineto{\pgfqpoint{2.111737in}{1.729279in}}%
\pgfpathlineto{\pgfqpoint{2.096136in}{1.717651in}}%
\pgfpathlineto{\pgfqpoint{2.041334in}{1.676772in}}%
\pgfpathclose%
\pgfusepath{fill}%
\end{pgfscope}%
\begin{pgfscope}%
\pgfpathrectangle{\pgfqpoint{0.000000in}{0.000000in}}{\pgfqpoint{3.000000in}{3.000000in}}%
\pgfusepath{clip}%
\pgfsetbuttcap%
\pgfsetroundjoin%
\definecolor{currentfill}{rgb}{0.606952,0.000000,0.000000}%
\pgfsetfillcolor{currentfill}%
\pgfsetlinewidth{0.000000pt}%
\definecolor{currentstroke}{rgb}{0.000000,0.000000,0.000000}%
\pgfsetstrokecolor{currentstroke}%
\pgfsetdash{}{0pt}%
\pgfpathmoveto{\pgfqpoint{0.753929in}{1.917196in}}%
\pgfpathlineto{\pgfqpoint{0.736622in}{1.928424in}}%
\pgfpathlineto{\pgfqpoint{0.775185in}{1.869029in}}%
\pgfpathlineto{\pgfqpoint{0.791769in}{1.859015in}}%
\pgfpathlineto{\pgfqpoint{0.753929in}{1.917196in}}%
\pgfpathclose%
\pgfusepath{fill}%
\end{pgfscope}%
\begin{pgfscope}%
\pgfpathrectangle{\pgfqpoint{0.000000in}{0.000000in}}{\pgfqpoint{3.000000in}{3.000000in}}%
\pgfusepath{clip}%
\pgfsetbuttcap%
\pgfsetroundjoin%
\definecolor{currentfill}{rgb}{1.000000,0.349310,0.000000}%
\pgfsetfillcolor{currentfill}%
\pgfsetlinewidth{0.000000pt}%
\definecolor{currentstroke}{rgb}{0.000000,0.000000,0.000000}%
\pgfsetstrokecolor{currentstroke}%
\pgfsetdash{}{0pt}%
\pgfpathmoveto{\pgfqpoint{1.016800in}{1.691994in}}%
\pgfpathlineto{\pgfqpoint{1.001671in}{1.703581in}}%
\pgfpathlineto{\pgfqpoint{1.061000in}{1.664114in}}%
\pgfpathlineto{\pgfqpoint{1.074520in}{1.653592in}}%
\pgfpathlineto{\pgfqpoint{1.016800in}{1.691994in}}%
\pgfpathclose%
\pgfusepath{fill}%
\end{pgfscope}%
\begin{pgfscope}%
\pgfpathrectangle{\pgfqpoint{0.000000in}{0.000000in}}{\pgfqpoint{3.000000in}{3.000000in}}%
\pgfusepath{clip}%
\pgfsetbuttcap%
\pgfsetroundjoin%
\definecolor{currentfill}{rgb}{1.000000,0.538126,0.000000}%
\pgfsetfillcolor{currentfill}%
\pgfsetlinewidth{0.000000pt}%
\definecolor{currentstroke}{rgb}{0.000000,0.000000,0.000000}%
\pgfsetstrokecolor{currentstroke}%
\pgfsetdash{}{0pt}%
\pgfpathmoveto{\pgfqpoint{1.114924in}{1.619738in}}%
\pgfpathlineto{\pgfqpoint{1.101482in}{1.631433in}}%
\pgfpathlineto{\pgfqpoint{1.167459in}{1.599726in}}%
\pgfpathlineto{\pgfqpoint{1.178919in}{1.588966in}}%
\pgfpathlineto{\pgfqpoint{1.114924in}{1.619738in}}%
\pgfpathclose%
\pgfusepath{fill}%
\end{pgfscope}%
\begin{pgfscope}%
\pgfpathrectangle{\pgfqpoint{0.000000in}{0.000000in}}{\pgfqpoint{3.000000in}{3.000000in}}%
\pgfusepath{clip}%
\pgfsetbuttcap%
\pgfsetroundjoin%
\definecolor{currentfill}{rgb}{1.000000,0.480029,0.000000}%
\pgfsetfillcolor{currentfill}%
\pgfsetlinewidth{0.000000pt}%
\definecolor{currentstroke}{rgb}{0.000000,0.000000,0.000000}%
\pgfsetstrokecolor{currentstroke}%
\pgfsetdash{}{0pt}%
\pgfpathmoveto{\pgfqpoint{1.936796in}{1.609730in}}%
\pgfpathlineto{\pgfqpoint{1.948980in}{1.620365in}}%
\pgfpathlineto{\pgfqpoint{2.013159in}{1.654681in}}%
\pgfpathlineto{\pgfqpoint{1.999111in}{1.643064in}}%
\pgfpathlineto{\pgfqpoint{1.936796in}{1.609730in}}%
\pgfpathclose%
\pgfusepath{fill}%
\end{pgfscope}%
\begin{pgfscope}%
\pgfpathrectangle{\pgfqpoint{0.000000in}{0.000000in}}{\pgfqpoint{3.000000in}{3.000000in}}%
\pgfusepath{clip}%
\pgfsetbuttcap%
\pgfsetroundjoin%
\definecolor{currentfill}{rgb}{1.000000,0.886710,0.000000}%
\pgfsetfillcolor{currentfill}%
\pgfsetlinewidth{0.000000pt}%
\definecolor{currentstroke}{rgb}{0.000000,0.000000,0.000000}%
\pgfsetstrokecolor{currentstroke}%
\pgfsetdash{}{0pt}%
\pgfpathmoveto{\pgfqpoint{1.591436in}{1.474463in}}%
\pgfpathlineto{\pgfqpoint{1.593340in}{1.486045in}}%
\pgfpathlineto{\pgfqpoint{1.671175in}{1.493304in}}%
\pgfpathlineto{\pgfqpoint{1.666466in}{1.481468in}}%
\pgfpathlineto{\pgfqpoint{1.591436in}{1.474463in}}%
\pgfpathclose%
\pgfusepath{fill}%
\end{pgfscope}%
\begin{pgfscope}%
\pgfpathrectangle{\pgfqpoint{0.000000in}{0.000000in}}{\pgfqpoint{3.000000in}{3.000000in}}%
\pgfusepath{clip}%
\pgfsetbuttcap%
\pgfsetroundjoin%
\definecolor{currentfill}{rgb}{1.000000,0.668845,0.000000}%
\pgfsetfillcolor{currentfill}%
\pgfsetlinewidth{0.000000pt}%
\definecolor{currentstroke}{rgb}{0.000000,0.000000,0.000000}%
\pgfsetstrokecolor{currentstroke}%
\pgfsetdash{}{0pt}%
\pgfpathmoveto{\pgfqpoint{1.201764in}{1.566036in}}%
\pgfpathlineto{\pgfqpoint{1.190354in}{1.577749in}}%
\pgfpathlineto{\pgfqpoint{1.261578in}{1.553053in}}%
\pgfpathlineto{\pgfqpoint{1.270690in}{1.542118in}}%
\pgfpathlineto{\pgfqpoint{1.201764in}{1.566036in}}%
\pgfpathclose%
\pgfusepath{fill}%
\end{pgfscope}%
\begin{pgfscope}%
\pgfpathrectangle{\pgfqpoint{0.000000in}{0.000000in}}{\pgfqpoint{3.000000in}{3.000000in}}%
\pgfusepath{clip}%
\pgfsetbuttcap%
\pgfsetroundjoin%
\definecolor{currentfill}{rgb}{1.000000,0.886710,0.000000}%
\pgfsetfillcolor{currentfill}%
\pgfsetlinewidth{0.000000pt}%
\definecolor{currentstroke}{rgb}{0.000000,0.000000,0.000000}%
\pgfsetstrokecolor{currentstroke}%
\pgfsetdash{}{0pt}%
\pgfpathmoveto{\pgfqpoint{1.439351in}{1.478476in}}%
\pgfpathlineto{\pgfqpoint{1.435566in}{1.490203in}}%
\pgfpathlineto{\pgfqpoint{1.514102in}{1.485001in}}%
\pgfpathlineto{\pgfqpoint{1.515055in}{1.473456in}}%
\pgfpathlineto{\pgfqpoint{1.439351in}{1.478476in}}%
\pgfpathclose%
\pgfusepath{fill}%
\end{pgfscope}%
\begin{pgfscope}%
\pgfpathrectangle{\pgfqpoint{0.000000in}{0.000000in}}{\pgfqpoint{3.000000in}{3.000000in}}%
\pgfusepath{clip}%
\pgfsetbuttcap%
\pgfsetroundjoin%
\definecolor{currentfill}{rgb}{0.999109,0.073348,0.000000}%
\pgfsetfillcolor{currentfill}%
\pgfsetlinewidth{0.000000pt}%
\definecolor{currentstroke}{rgb}{0.000000,0.000000,0.000000}%
\pgfsetstrokecolor{currentstroke}%
\pgfsetdash{}{0pt}%
\pgfpathmoveto{\pgfqpoint{0.890781in}{1.794743in}}%
\pgfpathlineto{\pgfqpoint{0.874337in}{1.806014in}}%
\pgfpathlineto{\pgfqpoint{0.925637in}{1.756893in}}%
\pgfpathlineto{\pgfqpoint{0.940895in}{1.746781in}}%
\pgfpathlineto{\pgfqpoint{0.890781in}{1.794743in}}%
\pgfpathclose%
\pgfusepath{fill}%
\end{pgfscope}%
\begin{pgfscope}%
\pgfpathrectangle{\pgfqpoint{0.000000in}{0.000000in}}{\pgfqpoint{3.000000in}{3.000000in}}%
\pgfusepath{clip}%
\pgfsetbuttcap%
\pgfsetroundjoin%
\definecolor{currentfill}{rgb}{1.000000,0.741467,0.000000}%
\pgfsetfillcolor{currentfill}%
\pgfsetlinewidth{0.000000pt}%
\definecolor{currentstroke}{rgb}{0.000000,0.000000,0.000000}%
\pgfsetstrokecolor{currentstroke}%
\pgfsetdash{}{0pt}%
\pgfpathmoveto{\pgfqpoint{1.752942in}{1.518220in}}%
\pgfpathlineto{\pgfqpoint{1.760358in}{1.529279in}}%
\pgfpathlineto{\pgfqpoint{1.834251in}{1.549492in}}%
\pgfpathlineto{\pgfqpoint{1.824363in}{1.537776in}}%
\pgfpathlineto{\pgfqpoint{1.752942in}{1.518220in}}%
\pgfpathclose%
\pgfusepath{fill}%
\end{pgfscope}%
\begin{pgfscope}%
\pgfpathrectangle{\pgfqpoint{0.000000in}{0.000000in}}{\pgfqpoint{3.000000in}{3.000000in}}%
\pgfusepath{clip}%
\pgfsetbuttcap%
\pgfsetroundjoin%
\definecolor{currentfill}{rgb}{0.856506,0.000000,0.000000}%
\pgfsetfillcolor{currentfill}%
\pgfsetlinewidth{0.000000pt}%
\definecolor{currentstroke}{rgb}{0.000000,0.000000,0.000000}%
\pgfsetstrokecolor{currentstroke}%
\pgfsetdash{}{0pt}%
\pgfpathmoveto{\pgfqpoint{2.190118in}{1.783100in}}%
\pgfpathlineto{\pgfqpoint{2.205870in}{1.793120in}}%
\pgfpathlineto{\pgfqpoint{2.253625in}{1.845752in}}%
\pgfpathlineto{\pgfqpoint{2.236839in}{1.834548in}}%
\pgfpathlineto{\pgfqpoint{2.190118in}{1.783100in}}%
\pgfpathclose%
\pgfusepath{fill}%
\end{pgfscope}%
\begin{pgfscope}%
\pgfpathrectangle{\pgfqpoint{0.000000in}{0.000000in}}{\pgfqpoint{3.000000in}{3.000000in}}%
\pgfusepath{clip}%
\pgfsetbuttcap%
\pgfsetroundjoin%
\definecolor{currentfill}{rgb}{0.553476,0.000000,0.000000}%
\pgfsetfillcolor{currentfill}%
\pgfsetlinewidth{0.000000pt}%
\definecolor{currentstroke}{rgb}{0.000000,0.000000,0.000000}%
\pgfsetstrokecolor{currentstroke}%
\pgfsetdash{}{0pt}%
\pgfpathmoveto{\pgfqpoint{0.736622in}{1.928424in}}%
\pgfpathlineto{\pgfqpoint{0.719297in}{1.939480in}}%
\pgfpathlineto{\pgfqpoint{0.758579in}{1.878872in}}%
\pgfpathlineto{\pgfqpoint{0.775185in}{1.869029in}}%
\pgfpathlineto{\pgfqpoint{0.736622in}{1.928424in}}%
\pgfpathclose%
\pgfusepath{fill}%
\end{pgfscope}%
\begin{pgfscope}%
\pgfpathrectangle{\pgfqpoint{0.000000in}{0.000000in}}{\pgfqpoint{3.000000in}{3.000000in}}%
\pgfusepath{clip}%
\pgfsetbuttcap%
\pgfsetroundjoin%
\definecolor{currentfill}{rgb}{1.000000,0.886710,0.000000}%
\pgfsetfillcolor{currentfill}%
\pgfsetlinewidth{0.000000pt}%
\definecolor{currentstroke}{rgb}{0.000000,0.000000,0.000000}%
\pgfsetstrokecolor{currentstroke}%
\pgfsetdash{}{0pt}%
\pgfpathmoveto{\pgfqpoint{1.515055in}{1.473456in}}%
\pgfpathlineto{\pgfqpoint{1.514102in}{1.485001in}}%
\pgfpathlineto{\pgfqpoint{1.593340in}{1.486045in}}%
\pgfpathlineto{\pgfqpoint{1.591436in}{1.474463in}}%
\pgfpathlineto{\pgfqpoint{1.515055in}{1.473456in}}%
\pgfpathclose%
\pgfusepath{fill}%
\end{pgfscope}%
\begin{pgfscope}%
\pgfpathrectangle{\pgfqpoint{0.000000in}{0.000000in}}{\pgfqpoint{3.000000in}{3.000000in}}%
\pgfusepath{clip}%
\pgfsetbuttcap%
\pgfsetroundjoin%
\definecolor{currentfill}{rgb}{1.000000,0.233115,0.000000}%
\pgfsetfillcolor{currentfill}%
\pgfsetlinewidth{0.000000pt}%
\definecolor{currentstroke}{rgb}{0.000000,0.000000,0.000000}%
\pgfsetstrokecolor{currentstroke}%
\pgfsetdash{}{0pt}%
\pgfpathmoveto{\pgfqpoint{2.055461in}{1.687300in}}%
\pgfpathlineto{\pgfqpoint{2.069614in}{1.697513in}}%
\pgfpathlineto{\pgfqpoint{2.127363in}{1.740593in}}%
\pgfpathlineto{\pgfqpoint{2.111737in}{1.729279in}}%
\pgfpathlineto{\pgfqpoint{2.055461in}{1.687300in}}%
\pgfpathclose%
\pgfusepath{fill}%
\end{pgfscope}%
\begin{pgfscope}%
\pgfpathrectangle{\pgfqpoint{0.000000in}{0.000000in}}{\pgfqpoint{3.000000in}{3.000000in}}%
\pgfusepath{clip}%
\pgfsetbuttcap%
\pgfsetroundjoin%
\definecolor{currentfill}{rgb}{1.000000,0.814089,0.000000}%
\pgfsetfillcolor{currentfill}%
\pgfsetlinewidth{0.000000pt}%
\definecolor{currentstroke}{rgb}{0.000000,0.000000,0.000000}%
\pgfsetstrokecolor{currentstroke}%
\pgfsetdash{}{0pt}%
\pgfpathmoveto{\pgfqpoint{1.671175in}{1.493304in}}%
\pgfpathlineto{\pgfqpoint{1.675896in}{1.504491in}}%
\pgfpathlineto{\pgfqpoint{1.752942in}{1.518220in}}%
\pgfpathlineto{\pgfqpoint{1.745544in}{1.506571in}}%
\pgfpathlineto{\pgfqpoint{1.671175in}{1.493304in}}%
\pgfpathclose%
\pgfusepath{fill}%
\end{pgfscope}%
\begin{pgfscope}%
\pgfpathrectangle{\pgfqpoint{0.000000in}{0.000000in}}{\pgfqpoint{3.000000in}{3.000000in}}%
\pgfusepath{clip}%
\pgfsetbuttcap%
\pgfsetroundjoin%
\definecolor{currentfill}{rgb}{1.000000,0.610748,0.000000}%
\pgfsetfillcolor{currentfill}%
\pgfsetlinewidth{0.000000pt}%
\definecolor{currentstroke}{rgb}{0.000000,0.000000,0.000000}%
\pgfsetstrokecolor{currentstroke}%
\pgfsetdash{}{0pt}%
\pgfpathmoveto{\pgfqpoint{1.844161in}{1.560668in}}%
\pgfpathlineto{\pgfqpoint{1.854094in}{1.571347in}}%
\pgfpathlineto{\pgfqpoint{1.924637in}{1.598674in}}%
\pgfpathlineto{\pgfqpoint{1.912504in}{1.587162in}}%
\pgfpathlineto{\pgfqpoint{1.844161in}{1.560668in}}%
\pgfpathclose%
\pgfusepath{fill}%
\end{pgfscope}%
\begin{pgfscope}%
\pgfpathrectangle{\pgfqpoint{0.000000in}{0.000000in}}{\pgfqpoint{3.000000in}{3.000000in}}%
\pgfusepath{clip}%
\pgfsetbuttcap%
\pgfsetroundjoin%
\definecolor{currentfill}{rgb}{1.000000,0.291213,0.000000}%
\pgfsetfillcolor{currentfill}%
\pgfsetlinewidth{0.000000pt}%
\definecolor{currentstroke}{rgb}{0.000000,0.000000,0.000000}%
\pgfsetstrokecolor{currentstroke}%
\pgfsetdash{}{0pt}%
\pgfpathmoveto{\pgfqpoint{1.001671in}{1.703581in}}%
\pgfpathlineto{\pgfqpoint{0.986515in}{1.714831in}}%
\pgfpathlineto{\pgfqpoint{1.047454in}{1.674299in}}%
\pgfpathlineto{\pgfqpoint{1.061000in}{1.664114in}}%
\pgfpathlineto{\pgfqpoint{1.001671in}{1.703581in}}%
\pgfpathclose%
\pgfusepath{fill}%
\end{pgfscope}%
\begin{pgfscope}%
\pgfpathrectangle{\pgfqpoint{0.000000in}{0.000000in}}{\pgfqpoint{3.000000in}{3.000000in}}%
\pgfusepath{clip}%
\pgfsetbuttcap%
\pgfsetroundjoin%
\definecolor{currentfill}{rgb}{1.000000,0.741467,0.000000}%
\pgfsetfillcolor{currentfill}%
\pgfsetlinewidth{0.000000pt}%
\definecolor{currentstroke}{rgb}{0.000000,0.000000,0.000000}%
\pgfsetstrokecolor{currentstroke}%
\pgfsetdash{}{0pt}%
\pgfpathmoveto{\pgfqpoint{1.279782in}{1.530641in}}%
\pgfpathlineto{\pgfqpoint{1.270690in}{1.542118in}}%
\pgfpathlineto{\pgfqpoint{1.346752in}{1.523855in}}%
\pgfpathlineto{\pgfqpoint{1.353293in}{1.512974in}}%
\pgfpathlineto{\pgfqpoint{1.279782in}{1.530641in}}%
\pgfpathclose%
\pgfusepath{fill}%
\end{pgfscope}%
\begin{pgfscope}%
\pgfpathrectangle{\pgfqpoint{0.000000in}{0.000000in}}{\pgfqpoint{3.000000in}{3.000000in}}%
\pgfusepath{clip}%
\pgfsetbuttcap%
\pgfsetroundjoin%
\definecolor{currentfill}{rgb}{1.000000,0.814089,0.000000}%
\pgfsetfillcolor{currentfill}%
\pgfsetlinewidth{0.000000pt}%
\definecolor{currentstroke}{rgb}{0.000000,0.000000,0.000000}%
\pgfsetstrokecolor{currentstroke}%
\pgfsetdash{}{0pt}%
\pgfpathmoveto{\pgfqpoint{1.359818in}{1.501501in}}%
\pgfpathlineto{\pgfqpoint{1.353293in}{1.512974in}}%
\pgfpathlineto{\pgfqpoint{1.431771in}{1.501282in}}%
\pgfpathlineto{\pgfqpoint{1.435566in}{1.490203in}}%
\pgfpathlineto{\pgfqpoint{1.359818in}{1.501501in}}%
\pgfpathclose%
\pgfusepath{fill}%
\end{pgfscope}%
\begin{pgfscope}%
\pgfpathrectangle{\pgfqpoint{0.000000in}{0.000000in}}{\pgfqpoint{3.000000in}{3.000000in}}%
\pgfusepath{clip}%
\pgfsetbuttcap%
\pgfsetroundjoin%
\definecolor{currentfill}{rgb}{0.500000,0.000000,0.000000}%
\pgfsetfillcolor{currentfill}%
\pgfsetlinewidth{0.000000pt}%
\definecolor{currentstroke}{rgb}{0.000000,0.000000,0.000000}%
\pgfsetstrokecolor{currentstroke}%
\pgfsetdash{}{0pt}%
\pgfpathmoveto{\pgfqpoint{0.719297in}{1.939480in}}%
\pgfpathlineto{\pgfqpoint{0.701952in}{1.950374in}}%
\pgfpathlineto{\pgfqpoint{0.741950in}{1.888552in}}%
\pgfpathlineto{\pgfqpoint{0.758579in}{1.878872in}}%
\pgfpathlineto{\pgfqpoint{0.719297in}{1.939480in}}%
\pgfpathclose%
\pgfusepath{fill}%
\end{pgfscope}%
\begin{pgfscope}%
\pgfpathrectangle{\pgfqpoint{0.000000in}{0.000000in}}{\pgfqpoint{3.000000in}{3.000000in}}%
\pgfusepath{clip}%
\pgfsetbuttcap%
\pgfsetroundjoin%
\definecolor{currentfill}{rgb}{1.000000,0.480029,0.000000}%
\pgfsetfillcolor{currentfill}%
\pgfsetlinewidth{0.000000pt}%
\definecolor{currentstroke}{rgb}{0.000000,0.000000,0.000000}%
\pgfsetstrokecolor{currentstroke}%
\pgfsetdash{}{0pt}%
\pgfpathmoveto{\pgfqpoint{1.101482in}{1.631433in}}%
\pgfpathlineto{\pgfqpoint{1.088014in}{1.642707in}}%
\pgfpathlineto{\pgfqpoint{1.155975in}{1.610065in}}%
\pgfpathlineto{\pgfqpoint{1.167459in}{1.599726in}}%
\pgfpathlineto{\pgfqpoint{1.101482in}{1.631433in}}%
\pgfpathclose%
\pgfusepath{fill}%
\end{pgfscope}%
\begin{pgfscope}%
\pgfpathrectangle{\pgfqpoint{0.000000in}{0.000000in}}{\pgfqpoint{3.000000in}{3.000000in}}%
\pgfusepath{clip}%
\pgfsetbuttcap%
\pgfsetroundjoin%
\definecolor{currentfill}{rgb}{0.803030,0.000000,0.000000}%
\pgfsetfillcolor{currentfill}%
\pgfsetlinewidth{0.000000pt}%
\definecolor{currentstroke}{rgb}{0.000000,0.000000,0.000000}%
\pgfsetstrokecolor{currentstroke}%
\pgfsetdash{}{0pt}%
\pgfpathmoveto{\pgfqpoint{2.205870in}{1.793120in}}%
\pgfpathlineto{\pgfqpoint{2.221646in}{1.802927in}}%
\pgfpathlineto{\pgfqpoint{2.270434in}{1.856741in}}%
\pgfpathlineto{\pgfqpoint{2.253625in}{1.845752in}}%
\pgfpathlineto{\pgfqpoint{2.205870in}{1.793120in}}%
\pgfpathclose%
\pgfusepath{fill}%
\end{pgfscope}%
\begin{pgfscope}%
\pgfpathrectangle{\pgfqpoint{0.000000in}{0.000000in}}{\pgfqpoint{3.000000in}{3.000000in}}%
\pgfusepath{clip}%
\pgfsetbuttcap%
\pgfsetroundjoin%
\definecolor{currentfill}{rgb}{0.927807,0.015251,0.000000}%
\pgfsetfillcolor{currentfill}%
\pgfsetlinewidth{0.000000pt}%
\definecolor{currentstroke}{rgb}{0.000000,0.000000,0.000000}%
\pgfsetstrokecolor{currentstroke}%
\pgfsetdash{}{0pt}%
\pgfpathmoveto{\pgfqpoint{0.874337in}{1.806014in}}%
\pgfpathlineto{\pgfqpoint{0.857870in}{1.817044in}}%
\pgfpathlineto{\pgfqpoint{0.910354in}{1.766763in}}%
\pgfpathlineto{\pgfqpoint{0.925637in}{1.756893in}}%
\pgfpathlineto{\pgfqpoint{0.874337in}{1.806014in}}%
\pgfpathclose%
\pgfusepath{fill}%
\end{pgfscope}%
\begin{pgfscope}%
\pgfpathrectangle{\pgfqpoint{0.000000in}{0.000000in}}{\pgfqpoint{3.000000in}{3.000000in}}%
\pgfusepath{clip}%
\pgfsetbuttcap%
\pgfsetroundjoin%
\definecolor{currentfill}{rgb}{1.000000,0.407407,0.000000}%
\pgfsetfillcolor{currentfill}%
\pgfsetlinewidth{0.000000pt}%
\definecolor{currentstroke}{rgb}{0.000000,0.000000,0.000000}%
\pgfsetstrokecolor{currentstroke}%
\pgfsetdash{}{0pt}%
\pgfpathmoveto{\pgfqpoint{1.948980in}{1.620365in}}%
\pgfpathlineto{\pgfqpoint{1.961190in}{1.630608in}}%
\pgfpathlineto{\pgfqpoint{2.027234in}{1.665907in}}%
\pgfpathlineto{\pgfqpoint{2.013159in}{1.654681in}}%
\pgfpathlineto{\pgfqpoint{1.948980in}{1.620365in}}%
\pgfpathclose%
\pgfusepath{fill}%
\end{pgfscope}%
\begin{pgfscope}%
\pgfpathrectangle{\pgfqpoint{0.000000in}{0.000000in}}{\pgfqpoint{3.000000in}{3.000000in}}%
\pgfusepath{clip}%
\pgfsetbuttcap%
\pgfsetroundjoin%
\definecolor{currentfill}{rgb}{1.000000,0.610748,0.000000}%
\pgfsetfillcolor{currentfill}%
\pgfsetlinewidth{0.000000pt}%
\definecolor{currentstroke}{rgb}{0.000000,0.000000,0.000000}%
\pgfsetstrokecolor{currentstroke}%
\pgfsetdash{}{0pt}%
\pgfpathmoveto{\pgfqpoint{1.190354in}{1.577749in}}%
\pgfpathlineto{\pgfqpoint{1.178919in}{1.588966in}}%
\pgfpathlineto{\pgfqpoint{1.252443in}{1.563493in}}%
\pgfpathlineto{\pgfqpoint{1.261578in}{1.553053in}}%
\pgfpathlineto{\pgfqpoint{1.190354in}{1.577749in}}%
\pgfpathclose%
\pgfusepath{fill}%
\end{pgfscope}%
\begin{pgfscope}%
\pgfpathrectangle{\pgfqpoint{0.000000in}{0.000000in}}{\pgfqpoint{3.000000in}{3.000000in}}%
\pgfusepath{clip}%
\pgfsetbuttcap%
\pgfsetroundjoin%
\definecolor{currentfill}{rgb}{1.000000,0.175018,0.000000}%
\pgfsetfillcolor{currentfill}%
\pgfsetlinewidth{0.000000pt}%
\definecolor{currentstroke}{rgb}{0.000000,0.000000,0.000000}%
\pgfsetstrokecolor{currentstroke}%
\pgfsetdash{}{0pt}%
\pgfpathmoveto{\pgfqpoint{2.069614in}{1.697513in}}%
\pgfpathlineto{\pgfqpoint{2.083793in}{1.707434in}}%
\pgfpathlineto{\pgfqpoint{2.143015in}{1.751614in}}%
\pgfpathlineto{\pgfqpoint{2.127363in}{1.740593in}}%
\pgfpathlineto{\pgfqpoint{2.069614in}{1.697513in}}%
\pgfpathclose%
\pgfusepath{fill}%
\end{pgfscope}%
\begin{pgfscope}%
\pgfpathrectangle{\pgfqpoint{0.000000in}{0.000000in}}{\pgfqpoint{3.000000in}{3.000000in}}%
\pgfusepath{clip}%
\pgfsetbuttcap%
\pgfsetroundjoin%
\definecolor{currentfill}{rgb}{1.000000,0.814089,0.000000}%
\pgfsetfillcolor{currentfill}%
\pgfsetlinewidth{0.000000pt}%
\definecolor{currentstroke}{rgb}{0.000000,0.000000,0.000000}%
\pgfsetstrokecolor{currentstroke}%
\pgfsetdash{}{0pt}%
\pgfpathmoveto{\pgfqpoint{1.593340in}{1.486045in}}%
\pgfpathlineto{\pgfqpoint{1.595250in}{1.496979in}}%
\pgfpathlineto{\pgfqpoint{1.675896in}{1.504491in}}%
\pgfpathlineto{\pgfqpoint{1.671175in}{1.493304in}}%
\pgfpathlineto{\pgfqpoint{1.593340in}{1.486045in}}%
\pgfpathclose%
\pgfusepath{fill}%
\end{pgfscope}%
\begin{pgfscope}%
\pgfpathrectangle{\pgfqpoint{0.000000in}{0.000000in}}{\pgfqpoint{3.000000in}{3.000000in}}%
\pgfusepath{clip}%
\pgfsetbuttcap%
\pgfsetroundjoin%
\definecolor{currentfill}{rgb}{1.000000,0.668845,0.000000}%
\pgfsetfillcolor{currentfill}%
\pgfsetlinewidth{0.000000pt}%
\definecolor{currentstroke}{rgb}{0.000000,0.000000,0.000000}%
\pgfsetstrokecolor{currentstroke}%
\pgfsetdash{}{0pt}%
\pgfpathmoveto{\pgfqpoint{1.760358in}{1.529279in}}%
\pgfpathlineto{\pgfqpoint{1.767791in}{1.539796in}}%
\pgfpathlineto{\pgfqpoint{1.844161in}{1.560668in}}%
\pgfpathlineto{\pgfqpoint{1.834251in}{1.549492in}}%
\pgfpathlineto{\pgfqpoint{1.760358in}{1.529279in}}%
\pgfpathclose%
\pgfusepath{fill}%
\end{pgfscope}%
\begin{pgfscope}%
\pgfpathrectangle{\pgfqpoint{0.000000in}{0.000000in}}{\pgfqpoint{3.000000in}{3.000000in}}%
\pgfusepath{clip}%
\pgfsetbuttcap%
\pgfsetroundjoin%
\definecolor{currentfill}{rgb}{1.000000,0.814089,0.000000}%
\pgfsetfillcolor{currentfill}%
\pgfsetlinewidth{0.000000pt}%
\definecolor{currentstroke}{rgb}{0.000000,0.000000,0.000000}%
\pgfsetstrokecolor{currentstroke}%
\pgfsetdash{}{0pt}%
\pgfpathmoveto{\pgfqpoint{1.435566in}{1.490203in}}%
\pgfpathlineto{\pgfqpoint{1.431771in}{1.501282in}}%
\pgfpathlineto{\pgfqpoint{1.513145in}{1.495899in}}%
\pgfpathlineto{\pgfqpoint{1.514102in}{1.485001in}}%
\pgfpathlineto{\pgfqpoint{1.435566in}{1.490203in}}%
\pgfpathclose%
\pgfusepath{fill}%
\end{pgfscope}%
\begin{pgfscope}%
\pgfpathrectangle{\pgfqpoint{0.000000in}{0.000000in}}{\pgfqpoint{3.000000in}{3.000000in}}%
\pgfusepath{clip}%
\pgfsetbuttcap%
\pgfsetroundjoin%
\definecolor{currentfill}{rgb}{1.000000,0.538126,0.000000}%
\pgfsetfillcolor{currentfill}%
\pgfsetlinewidth{0.000000pt}%
\definecolor{currentstroke}{rgb}{0.000000,0.000000,0.000000}%
\pgfsetstrokecolor{currentstroke}%
\pgfsetdash{}{0pt}%
\pgfpathmoveto{\pgfqpoint{1.854094in}{1.571347in}}%
\pgfpathlineto{\pgfqpoint{1.864050in}{1.581569in}}%
\pgfpathlineto{\pgfqpoint{1.936796in}{1.609730in}}%
\pgfpathlineto{\pgfqpoint{1.924637in}{1.598674in}}%
\pgfpathlineto{\pgfqpoint{1.854094in}{1.571347in}}%
\pgfpathclose%
\pgfusepath{fill}%
\end{pgfscope}%
\begin{pgfscope}%
\pgfpathrectangle{\pgfqpoint{0.000000in}{0.000000in}}{\pgfqpoint{3.000000in}{3.000000in}}%
\pgfusepath{clip}%
\pgfsetbuttcap%
\pgfsetroundjoin%
\definecolor{currentfill}{rgb}{1.000000,0.233115,0.000000}%
\pgfsetfillcolor{currentfill}%
\pgfsetlinewidth{0.000000pt}%
\definecolor{currentstroke}{rgb}{0.000000,0.000000,0.000000}%
\pgfsetstrokecolor{currentstroke}%
\pgfsetdash{}{0pt}%
\pgfpathmoveto{\pgfqpoint{0.986515in}{1.714831in}}%
\pgfpathlineto{\pgfqpoint{0.971334in}{1.725768in}}%
\pgfpathlineto{\pgfqpoint{1.033881in}{1.684170in}}%
\pgfpathlineto{\pgfqpoint{1.047454in}{1.674299in}}%
\pgfpathlineto{\pgfqpoint{0.986515in}{1.714831in}}%
\pgfpathclose%
\pgfusepath{fill}%
\end{pgfscope}%
\begin{pgfscope}%
\pgfpathrectangle{\pgfqpoint{0.000000in}{0.000000in}}{\pgfqpoint{3.000000in}{3.000000in}}%
\pgfusepath{clip}%
\pgfsetbuttcap%
\pgfsetroundjoin%
\definecolor{currentfill}{rgb}{1.000000,0.814089,0.000000}%
\pgfsetfillcolor{currentfill}%
\pgfsetlinewidth{0.000000pt}%
\definecolor{currentstroke}{rgb}{0.000000,0.000000,0.000000}%
\pgfsetstrokecolor{currentstroke}%
\pgfsetdash{}{0pt}%
\pgfpathmoveto{\pgfqpoint{1.514102in}{1.485001in}}%
\pgfpathlineto{\pgfqpoint{1.513145in}{1.495899in}}%
\pgfpathlineto{\pgfqpoint{1.595250in}{1.496979in}}%
\pgfpathlineto{\pgfqpoint{1.593340in}{1.486045in}}%
\pgfpathlineto{\pgfqpoint{1.514102in}{1.485001in}}%
\pgfpathclose%
\pgfusepath{fill}%
\end{pgfscope}%
\begin{pgfscope}%
\pgfpathrectangle{\pgfqpoint{0.000000in}{0.000000in}}{\pgfqpoint{3.000000in}{3.000000in}}%
\pgfusepath{clip}%
\pgfsetbuttcap%
\pgfsetroundjoin%
\definecolor{currentfill}{rgb}{0.731729,0.000000,0.000000}%
\pgfsetfillcolor{currentfill}%
\pgfsetlinewidth{0.000000pt}%
\definecolor{currentstroke}{rgb}{0.000000,0.000000,0.000000}%
\pgfsetstrokecolor{currentstroke}%
\pgfsetdash{}{0pt}%
\pgfpathmoveto{\pgfqpoint{2.221646in}{1.802927in}}%
\pgfpathlineto{\pgfqpoint{2.237447in}{1.812531in}}%
\pgfpathlineto{\pgfqpoint{2.287265in}{1.867527in}}%
\pgfpathlineto{\pgfqpoint{2.270434in}{1.856741in}}%
\pgfpathlineto{\pgfqpoint{2.221646in}{1.802927in}}%
\pgfpathclose%
\pgfusepath{fill}%
\end{pgfscope}%
\begin{pgfscope}%
\pgfpathrectangle{\pgfqpoint{0.000000in}{0.000000in}}{\pgfqpoint{3.000000in}{3.000000in}}%
\pgfusepath{clip}%
\pgfsetbuttcap%
\pgfsetroundjoin%
\definecolor{currentfill}{rgb}{1.000000,0.741467,0.000000}%
\pgfsetfillcolor{currentfill}%
\pgfsetlinewidth{0.000000pt}%
\definecolor{currentstroke}{rgb}{0.000000,0.000000,0.000000}%
\pgfsetstrokecolor{currentstroke}%
\pgfsetdash{}{0pt}%
\pgfpathmoveto{\pgfqpoint{1.675896in}{1.504491in}}%
\pgfpathlineto{\pgfqpoint{1.680629in}{1.515086in}}%
\pgfpathlineto{\pgfqpoint{1.760358in}{1.529279in}}%
\pgfpathlineto{\pgfqpoint{1.752942in}{1.518220in}}%
\pgfpathlineto{\pgfqpoint{1.675896in}{1.504491in}}%
\pgfpathclose%
\pgfusepath{fill}%
\end{pgfscope}%
\begin{pgfscope}%
\pgfpathrectangle{\pgfqpoint{0.000000in}{0.000000in}}{\pgfqpoint{3.000000in}{3.000000in}}%
\pgfusepath{clip}%
\pgfsetbuttcap%
\pgfsetroundjoin%
\definecolor{currentfill}{rgb}{0.856506,0.000000,0.000000}%
\pgfsetfillcolor{currentfill}%
\pgfsetlinewidth{0.000000pt}%
\definecolor{currentstroke}{rgb}{0.000000,0.000000,0.000000}%
\pgfsetstrokecolor{currentstroke}%
\pgfsetdash{}{0pt}%
\pgfpathmoveto{\pgfqpoint{0.857870in}{1.817044in}}%
\pgfpathlineto{\pgfqpoint{0.841379in}{1.827846in}}%
\pgfpathlineto{\pgfqpoint{0.895045in}{1.776405in}}%
\pgfpathlineto{\pgfqpoint{0.910354in}{1.766763in}}%
\pgfpathlineto{\pgfqpoint{0.857870in}{1.817044in}}%
\pgfpathclose%
\pgfusepath{fill}%
\end{pgfscope}%
\begin{pgfscope}%
\pgfpathrectangle{\pgfqpoint{0.000000in}{0.000000in}}{\pgfqpoint{3.000000in}{3.000000in}}%
\pgfusepath{clip}%
\pgfsetbuttcap%
\pgfsetroundjoin%
\definecolor{currentfill}{rgb}{1.000000,0.668845,0.000000}%
\pgfsetfillcolor{currentfill}%
\pgfsetlinewidth{0.000000pt}%
\definecolor{currentstroke}{rgb}{0.000000,0.000000,0.000000}%
\pgfsetstrokecolor{currentstroke}%
\pgfsetdash{}{0pt}%
\pgfpathmoveto{\pgfqpoint{1.270690in}{1.542118in}}%
\pgfpathlineto{\pgfqpoint{1.261578in}{1.553053in}}%
\pgfpathlineto{\pgfqpoint{1.340194in}{1.534195in}}%
\pgfpathlineto{\pgfqpoint{1.346752in}{1.523855in}}%
\pgfpathlineto{\pgfqpoint{1.270690in}{1.542118in}}%
\pgfpathclose%
\pgfusepath{fill}%
\end{pgfscope}%
\begin{pgfscope}%
\pgfpathrectangle{\pgfqpoint{0.000000in}{0.000000in}}{\pgfqpoint{3.000000in}{3.000000in}}%
\pgfusepath{clip}%
\pgfsetbuttcap%
\pgfsetroundjoin%
\definecolor{currentfill}{rgb}{1.000000,0.407407,0.000000}%
\pgfsetfillcolor{currentfill}%
\pgfsetlinewidth{0.000000pt}%
\definecolor{currentstroke}{rgb}{0.000000,0.000000,0.000000}%
\pgfsetstrokecolor{currentstroke}%
\pgfsetdash{}{0pt}%
\pgfpathmoveto{\pgfqpoint{1.088014in}{1.642707in}}%
\pgfpathlineto{\pgfqpoint{1.074520in}{1.653592in}}%
\pgfpathlineto{\pgfqpoint{1.144466in}{1.620012in}}%
\pgfpathlineto{\pgfqpoint{1.155975in}{1.610065in}}%
\pgfpathlineto{\pgfqpoint{1.088014in}{1.642707in}}%
\pgfpathclose%
\pgfusepath{fill}%
\end{pgfscope}%
\begin{pgfscope}%
\pgfpathrectangle{\pgfqpoint{0.000000in}{0.000000in}}{\pgfqpoint{3.000000in}{3.000000in}}%
\pgfusepath{clip}%
\pgfsetbuttcap%
\pgfsetroundjoin%
\definecolor{currentfill}{rgb}{1.000000,0.349310,0.000000}%
\pgfsetfillcolor{currentfill}%
\pgfsetlinewidth{0.000000pt}%
\definecolor{currentstroke}{rgb}{0.000000,0.000000,0.000000}%
\pgfsetstrokecolor{currentstroke}%
\pgfsetdash{}{0pt}%
\pgfpathmoveto{\pgfqpoint{1.961190in}{1.630608in}}%
\pgfpathlineto{\pgfqpoint{1.973425in}{1.640489in}}%
\pgfpathlineto{\pgfqpoint{2.041334in}{1.676772in}}%
\pgfpathlineto{\pgfqpoint{2.027234in}{1.665907in}}%
\pgfpathlineto{\pgfqpoint{1.961190in}{1.630608in}}%
\pgfpathclose%
\pgfusepath{fill}%
\end{pgfscope}%
\begin{pgfscope}%
\pgfpathrectangle{\pgfqpoint{0.000000in}{0.000000in}}{\pgfqpoint{3.000000in}{3.000000in}}%
\pgfusepath{clip}%
\pgfsetbuttcap%
\pgfsetroundjoin%
\definecolor{currentfill}{rgb}{1.000000,0.741467,0.000000}%
\pgfsetfillcolor{currentfill}%
\pgfsetlinewidth{0.000000pt}%
\definecolor{currentstroke}{rgb}{0.000000,0.000000,0.000000}%
\pgfsetstrokecolor{currentstroke}%
\pgfsetdash{}{0pt}%
\pgfpathmoveto{\pgfqpoint{1.353293in}{1.512974in}}%
\pgfpathlineto{\pgfqpoint{1.346752in}{1.523855in}}%
\pgfpathlineto{\pgfqpoint{1.427967in}{1.511768in}}%
\pgfpathlineto{\pgfqpoint{1.431771in}{1.501282in}}%
\pgfpathlineto{\pgfqpoint{1.353293in}{1.512974in}}%
\pgfpathclose%
\pgfusepath{fill}%
\end{pgfscope}%
\begin{pgfscope}%
\pgfpathrectangle{\pgfqpoint{0.000000in}{0.000000in}}{\pgfqpoint{3.000000in}{3.000000in}}%
\pgfusepath{clip}%
\pgfsetbuttcap%
\pgfsetroundjoin%
\definecolor{currentfill}{rgb}{1.000000,0.116921,0.000000}%
\pgfsetfillcolor{currentfill}%
\pgfsetlinewidth{0.000000pt}%
\definecolor{currentstroke}{rgb}{0.000000,0.000000,0.000000}%
\pgfsetstrokecolor{currentstroke}%
\pgfsetdash{}{0pt}%
\pgfpathmoveto{\pgfqpoint{2.083793in}{1.707434in}}%
\pgfpathlineto{\pgfqpoint{2.097999in}{1.717079in}}%
\pgfpathlineto{\pgfqpoint{2.158691in}{1.762362in}}%
\pgfpathlineto{\pgfqpoint{2.143015in}{1.751614in}}%
\pgfpathlineto{\pgfqpoint{2.083793in}{1.707434in}}%
\pgfpathclose%
\pgfusepath{fill}%
\end{pgfscope}%
\begin{pgfscope}%
\pgfpathrectangle{\pgfqpoint{0.000000in}{0.000000in}}{\pgfqpoint{3.000000in}{3.000000in}}%
\pgfusepath{clip}%
\pgfsetbuttcap%
\pgfsetroundjoin%
\definecolor{currentfill}{rgb}{0.678253,0.000000,0.000000}%
\pgfsetfillcolor{currentfill}%
\pgfsetlinewidth{0.000000pt}%
\definecolor{currentstroke}{rgb}{0.000000,0.000000,0.000000}%
\pgfsetstrokecolor{currentstroke}%
\pgfsetdash{}{0pt}%
\pgfpathmoveto{\pgfqpoint{2.237447in}{1.812531in}}%
\pgfpathlineto{\pgfqpoint{2.253273in}{1.821943in}}%
\pgfpathlineto{\pgfqpoint{2.304117in}{1.878123in}}%
\pgfpathlineto{\pgfqpoint{2.287265in}{1.867527in}}%
\pgfpathlineto{\pgfqpoint{2.237447in}{1.812531in}}%
\pgfpathclose%
\pgfusepath{fill}%
\end{pgfscope}%
\begin{pgfscope}%
\pgfpathrectangle{\pgfqpoint{0.000000in}{0.000000in}}{\pgfqpoint{3.000000in}{3.000000in}}%
\pgfusepath{clip}%
\pgfsetbuttcap%
\pgfsetroundjoin%
\definecolor{currentfill}{rgb}{1.000000,0.538126,0.000000}%
\pgfsetfillcolor{currentfill}%
\pgfsetlinewidth{0.000000pt}%
\definecolor{currentstroke}{rgb}{0.000000,0.000000,0.000000}%
\pgfsetstrokecolor{currentstroke}%
\pgfsetdash{}{0pt}%
\pgfpathmoveto{\pgfqpoint{1.178919in}{1.588966in}}%
\pgfpathlineto{\pgfqpoint{1.167459in}{1.599726in}}%
\pgfpathlineto{\pgfqpoint{1.243288in}{1.573474in}}%
\pgfpathlineto{\pgfqpoint{1.252443in}{1.563493in}}%
\pgfpathlineto{\pgfqpoint{1.178919in}{1.588966in}}%
\pgfpathclose%
\pgfusepath{fill}%
\end{pgfscope}%
\begin{pgfscope}%
\pgfpathrectangle{\pgfqpoint{0.000000in}{0.000000in}}{\pgfqpoint{3.000000in}{3.000000in}}%
\pgfusepath{clip}%
\pgfsetbuttcap%
\pgfsetroundjoin%
\definecolor{currentfill}{rgb}{0.803030,0.000000,0.000000}%
\pgfsetfillcolor{currentfill}%
\pgfsetlinewidth{0.000000pt}%
\definecolor{currentstroke}{rgb}{0.000000,0.000000,0.000000}%
\pgfsetstrokecolor{currentstroke}%
\pgfsetdash{}{0pt}%
\pgfpathmoveto{\pgfqpoint{0.841379in}{1.827846in}}%
\pgfpathlineto{\pgfqpoint{0.824866in}{1.838434in}}%
\pgfpathlineto{\pgfqpoint{0.879710in}{1.785833in}}%
\pgfpathlineto{\pgfqpoint{0.895045in}{1.776405in}}%
\pgfpathlineto{\pgfqpoint{0.841379in}{1.827846in}}%
\pgfpathclose%
\pgfusepath{fill}%
\end{pgfscope}%
\begin{pgfscope}%
\pgfpathrectangle{\pgfqpoint{0.000000in}{0.000000in}}{\pgfqpoint{3.000000in}{3.000000in}}%
\pgfusepath{clip}%
\pgfsetbuttcap%
\pgfsetroundjoin%
\definecolor{currentfill}{rgb}{1.000000,0.175018,0.000000}%
\pgfsetfillcolor{currentfill}%
\pgfsetlinewidth{0.000000pt}%
\definecolor{currentstroke}{rgb}{0.000000,0.000000,0.000000}%
\pgfsetstrokecolor{currentstroke}%
\pgfsetdash{}{0pt}%
\pgfpathmoveto{\pgfqpoint{0.971334in}{1.725768in}}%
\pgfpathlineto{\pgfqpoint{0.956127in}{1.736412in}}%
\pgfpathlineto{\pgfqpoint{1.020283in}{1.693747in}}%
\pgfpathlineto{\pgfqpoint{1.033881in}{1.684170in}}%
\pgfpathlineto{\pgfqpoint{0.971334in}{1.725768in}}%
\pgfpathclose%
\pgfusepath{fill}%
\end{pgfscope}%
\begin{pgfscope}%
\pgfpathrectangle{\pgfqpoint{0.000000in}{0.000000in}}{\pgfqpoint{3.000000in}{3.000000in}}%
\pgfusepath{clip}%
\pgfsetbuttcap%
\pgfsetroundjoin%
\definecolor{currentfill}{rgb}{1.000000,0.610748,0.000000}%
\pgfsetfillcolor{currentfill}%
\pgfsetlinewidth{0.000000pt}%
\definecolor{currentstroke}{rgb}{0.000000,0.000000,0.000000}%
\pgfsetstrokecolor{currentstroke}%
\pgfsetdash{}{0pt}%
\pgfpathmoveto{\pgfqpoint{1.767791in}{1.539796in}}%
\pgfpathlineto{\pgfqpoint{1.775243in}{1.549816in}}%
\pgfpathlineto{\pgfqpoint{1.854094in}{1.571347in}}%
\pgfpathlineto{\pgfqpoint{1.844161in}{1.560668in}}%
\pgfpathlineto{\pgfqpoint{1.767791in}{1.539796in}}%
\pgfpathclose%
\pgfusepath{fill}%
\end{pgfscope}%
\begin{pgfscope}%
\pgfpathrectangle{\pgfqpoint{0.000000in}{0.000000in}}{\pgfqpoint{3.000000in}{3.000000in}}%
\pgfusepath{clip}%
\pgfsetbuttcap%
\pgfsetroundjoin%
\definecolor{currentfill}{rgb}{1.000000,0.741467,0.000000}%
\pgfsetfillcolor{currentfill}%
\pgfsetlinewidth{0.000000pt}%
\definecolor{currentstroke}{rgb}{0.000000,0.000000,0.000000}%
\pgfsetstrokecolor{currentstroke}%
\pgfsetdash{}{0pt}%
\pgfpathmoveto{\pgfqpoint{1.595250in}{1.496979in}}%
\pgfpathlineto{\pgfqpoint{1.597164in}{1.507320in}}%
\pgfpathlineto{\pgfqpoint{1.680629in}{1.515086in}}%
\pgfpathlineto{\pgfqpoint{1.675896in}{1.504491in}}%
\pgfpathlineto{\pgfqpoint{1.595250in}{1.496979in}}%
\pgfpathclose%
\pgfusepath{fill}%
\end{pgfscope}%
\begin{pgfscope}%
\pgfpathrectangle{\pgfqpoint{0.000000in}{0.000000in}}{\pgfqpoint{3.000000in}{3.000000in}}%
\pgfusepath{clip}%
\pgfsetbuttcap%
\pgfsetroundjoin%
\definecolor{currentfill}{rgb}{1.000000,0.480029,0.000000}%
\pgfsetfillcolor{currentfill}%
\pgfsetlinewidth{0.000000pt}%
\definecolor{currentstroke}{rgb}{0.000000,0.000000,0.000000}%
\pgfsetstrokecolor{currentstroke}%
\pgfsetdash{}{0pt}%
\pgfpathmoveto{\pgfqpoint{1.864050in}{1.581569in}}%
\pgfpathlineto{\pgfqpoint{1.874029in}{1.591368in}}%
\pgfpathlineto{\pgfqpoint{1.948980in}{1.620365in}}%
\pgfpathlineto{\pgfqpoint{1.936796in}{1.609730in}}%
\pgfpathlineto{\pgfqpoint{1.864050in}{1.581569in}}%
\pgfpathclose%
\pgfusepath{fill}%
\end{pgfscope}%
\begin{pgfscope}%
\pgfpathrectangle{\pgfqpoint{0.000000in}{0.000000in}}{\pgfqpoint{3.000000in}{3.000000in}}%
\pgfusepath{clip}%
\pgfsetbuttcap%
\pgfsetroundjoin%
\definecolor{currentfill}{rgb}{1.000000,0.741467,0.000000}%
\pgfsetfillcolor{currentfill}%
\pgfsetlinewidth{0.000000pt}%
\definecolor{currentstroke}{rgb}{0.000000,0.000000,0.000000}%
\pgfsetstrokecolor{currentstroke}%
\pgfsetdash{}{0pt}%
\pgfpathmoveto{\pgfqpoint{1.431771in}{1.501282in}}%
\pgfpathlineto{\pgfqpoint{1.427967in}{1.511768in}}%
\pgfpathlineto{\pgfqpoint{1.512186in}{1.506203in}}%
\pgfpathlineto{\pgfqpoint{1.513145in}{1.495899in}}%
\pgfpathlineto{\pgfqpoint{1.431771in}{1.501282in}}%
\pgfpathclose%
\pgfusepath{fill}%
\end{pgfscope}%
\begin{pgfscope}%
\pgfpathrectangle{\pgfqpoint{0.000000in}{0.000000in}}{\pgfqpoint{3.000000in}{3.000000in}}%
\pgfusepath{clip}%
\pgfsetbuttcap%
\pgfsetroundjoin%
\definecolor{currentfill}{rgb}{1.000000,0.668845,0.000000}%
\pgfsetfillcolor{currentfill}%
\pgfsetlinewidth{0.000000pt}%
\definecolor{currentstroke}{rgb}{0.000000,0.000000,0.000000}%
\pgfsetstrokecolor{currentstroke}%
\pgfsetdash{}{0pt}%
\pgfpathmoveto{\pgfqpoint{1.680629in}{1.515086in}}%
\pgfpathlineto{\pgfqpoint{1.685374in}{1.525139in}}%
\pgfpathlineto{\pgfqpoint{1.767791in}{1.539796in}}%
\pgfpathlineto{\pgfqpoint{1.760358in}{1.529279in}}%
\pgfpathlineto{\pgfqpoint{1.680629in}{1.515086in}}%
\pgfpathclose%
\pgfusepath{fill}%
\end{pgfscope}%
\begin{pgfscope}%
\pgfpathrectangle{\pgfqpoint{0.000000in}{0.000000in}}{\pgfqpoint{3.000000in}{3.000000in}}%
\pgfusepath{clip}%
\pgfsetbuttcap%
\pgfsetroundjoin%
\definecolor{currentfill}{rgb}{1.000000,0.291213,0.000000}%
\pgfsetfillcolor{currentfill}%
\pgfsetlinewidth{0.000000pt}%
\definecolor{currentstroke}{rgb}{0.000000,0.000000,0.000000}%
\pgfsetstrokecolor{currentstroke}%
\pgfsetdash{}{0pt}%
\pgfpathmoveto{\pgfqpoint{1.973425in}{1.640489in}}%
\pgfpathlineto{\pgfqpoint{1.985685in}{1.650032in}}%
\pgfpathlineto{\pgfqpoint{2.055461in}{1.687300in}}%
\pgfpathlineto{\pgfqpoint{2.041334in}{1.676772in}}%
\pgfpathlineto{\pgfqpoint{1.973425in}{1.640489in}}%
\pgfpathclose%
\pgfusepath{fill}%
\end{pgfscope}%
\begin{pgfscope}%
\pgfpathrectangle{\pgfqpoint{0.000000in}{0.000000in}}{\pgfqpoint{3.000000in}{3.000000in}}%
\pgfusepath{clip}%
\pgfsetbuttcap%
\pgfsetroundjoin%
\definecolor{currentfill}{rgb}{1.000000,0.741467,0.000000}%
\pgfsetfillcolor{currentfill}%
\pgfsetlinewidth{0.000000pt}%
\definecolor{currentstroke}{rgb}{0.000000,0.000000,0.000000}%
\pgfsetstrokecolor{currentstroke}%
\pgfsetdash{}{0pt}%
\pgfpathmoveto{\pgfqpoint{1.513145in}{1.495899in}}%
\pgfpathlineto{\pgfqpoint{1.512186in}{1.506203in}}%
\pgfpathlineto{\pgfqpoint{1.597164in}{1.507320in}}%
\pgfpathlineto{\pgfqpoint{1.595250in}{1.496979in}}%
\pgfpathlineto{\pgfqpoint{1.513145in}{1.495899in}}%
\pgfpathclose%
\pgfusepath{fill}%
\end{pgfscope}%
\begin{pgfscope}%
\pgfpathrectangle{\pgfqpoint{0.000000in}{0.000000in}}{\pgfqpoint{3.000000in}{3.000000in}}%
\pgfusepath{clip}%
\pgfsetbuttcap%
\pgfsetroundjoin%
\definecolor{currentfill}{rgb}{0.999109,0.073348,0.000000}%
\pgfsetfillcolor{currentfill}%
\pgfsetlinewidth{0.000000pt}%
\definecolor{currentstroke}{rgb}{0.000000,0.000000,0.000000}%
\pgfsetstrokecolor{currentstroke}%
\pgfsetdash{}{0pt}%
\pgfpathmoveto{\pgfqpoint{2.097999in}{1.717079in}}%
\pgfpathlineto{\pgfqpoint{2.112230in}{1.726466in}}%
\pgfpathlineto{\pgfqpoint{2.174392in}{1.772851in}}%
\pgfpathlineto{\pgfqpoint{2.158691in}{1.762362in}}%
\pgfpathlineto{\pgfqpoint{2.097999in}{1.717079in}}%
\pgfpathclose%
\pgfusepath{fill}%
\end{pgfscope}%
\begin{pgfscope}%
\pgfpathrectangle{\pgfqpoint{0.000000in}{0.000000in}}{\pgfqpoint{3.000000in}{3.000000in}}%
\pgfusepath{clip}%
\pgfsetbuttcap%
\pgfsetroundjoin%
\definecolor{currentfill}{rgb}{1.000000,0.349310,0.000000}%
\pgfsetfillcolor{currentfill}%
\pgfsetlinewidth{0.000000pt}%
\definecolor{currentstroke}{rgb}{0.000000,0.000000,0.000000}%
\pgfsetstrokecolor{currentstroke}%
\pgfsetdash{}{0pt}%
\pgfpathmoveto{\pgfqpoint{1.074520in}{1.653592in}}%
\pgfpathlineto{\pgfqpoint{1.061000in}{1.664114in}}%
\pgfpathlineto{\pgfqpoint{1.132932in}{1.629595in}}%
\pgfpathlineto{\pgfqpoint{1.144466in}{1.620012in}}%
\pgfpathlineto{\pgfqpoint{1.074520in}{1.653592in}}%
\pgfpathclose%
\pgfusepath{fill}%
\end{pgfscope}%
\begin{pgfscope}%
\pgfpathrectangle{\pgfqpoint{0.000000in}{0.000000in}}{\pgfqpoint{3.000000in}{3.000000in}}%
\pgfusepath{clip}%
\pgfsetbuttcap%
\pgfsetroundjoin%
\definecolor{currentfill}{rgb}{0.606952,0.000000,0.000000}%
\pgfsetfillcolor{currentfill}%
\pgfsetlinewidth{0.000000pt}%
\definecolor{currentstroke}{rgb}{0.000000,0.000000,0.000000}%
\pgfsetstrokecolor{currentstroke}%
\pgfsetdash{}{0pt}%
\pgfpathmoveto{\pgfqpoint{2.253273in}{1.821943in}}%
\pgfpathlineto{\pgfqpoint{2.269124in}{1.831175in}}%
\pgfpathlineto{\pgfqpoint{2.320992in}{1.888537in}}%
\pgfpathlineto{\pgfqpoint{2.304117in}{1.878123in}}%
\pgfpathlineto{\pgfqpoint{2.253273in}{1.821943in}}%
\pgfpathclose%
\pgfusepath{fill}%
\end{pgfscope}%
\begin{pgfscope}%
\pgfpathrectangle{\pgfqpoint{0.000000in}{0.000000in}}{\pgfqpoint{3.000000in}{3.000000in}}%
\pgfusepath{clip}%
\pgfsetbuttcap%
\pgfsetroundjoin%
\definecolor{currentfill}{rgb}{1.000000,0.610748,0.000000}%
\pgfsetfillcolor{currentfill}%
\pgfsetlinewidth{0.000000pt}%
\definecolor{currentstroke}{rgb}{0.000000,0.000000,0.000000}%
\pgfsetstrokecolor{currentstroke}%
\pgfsetdash{}{0pt}%
\pgfpathmoveto{\pgfqpoint{1.261578in}{1.553053in}}%
\pgfpathlineto{\pgfqpoint{1.252443in}{1.563493in}}%
\pgfpathlineto{\pgfqpoint{1.333621in}{1.544037in}}%
\pgfpathlineto{\pgfqpoint{1.340194in}{1.534195in}}%
\pgfpathlineto{\pgfqpoint{1.261578in}{1.553053in}}%
\pgfpathclose%
\pgfusepath{fill}%
\end{pgfscope}%
\begin{pgfscope}%
\pgfpathrectangle{\pgfqpoint{0.000000in}{0.000000in}}{\pgfqpoint{3.000000in}{3.000000in}}%
\pgfusepath{clip}%
\pgfsetbuttcap%
\pgfsetroundjoin%
\definecolor{currentfill}{rgb}{0.731729,0.000000,0.000000}%
\pgfsetfillcolor{currentfill}%
\pgfsetlinewidth{0.000000pt}%
\definecolor{currentstroke}{rgb}{0.000000,0.000000,0.000000}%
\pgfsetstrokecolor{currentstroke}%
\pgfsetdash{}{0pt}%
\pgfpathmoveto{\pgfqpoint{0.824866in}{1.838434in}}%
\pgfpathlineto{\pgfqpoint{0.808329in}{1.848820in}}%
\pgfpathlineto{\pgfqpoint{0.864350in}{1.795058in}}%
\pgfpathlineto{\pgfqpoint{0.879710in}{1.785833in}}%
\pgfpathlineto{\pgfqpoint{0.824866in}{1.838434in}}%
\pgfpathclose%
\pgfusepath{fill}%
\end{pgfscope}%
\begin{pgfscope}%
\pgfpathrectangle{\pgfqpoint{0.000000in}{0.000000in}}{\pgfqpoint{3.000000in}{3.000000in}}%
\pgfusepath{clip}%
\pgfsetbuttcap%
\pgfsetroundjoin%
\definecolor{currentfill}{rgb}{1.000000,0.668845,0.000000}%
\pgfsetfillcolor{currentfill}%
\pgfsetlinewidth{0.000000pt}%
\definecolor{currentstroke}{rgb}{0.000000,0.000000,0.000000}%
\pgfsetstrokecolor{currentstroke}%
\pgfsetdash{}{0pt}%
\pgfpathmoveto{\pgfqpoint{1.346752in}{1.523855in}}%
\pgfpathlineto{\pgfqpoint{1.340194in}{1.534195in}}%
\pgfpathlineto{\pgfqpoint{1.424152in}{1.521713in}}%
\pgfpathlineto{\pgfqpoint{1.427967in}{1.511768in}}%
\pgfpathlineto{\pgfqpoint{1.346752in}{1.523855in}}%
\pgfpathclose%
\pgfusepath{fill}%
\end{pgfscope}%
\begin{pgfscope}%
\pgfpathrectangle{\pgfqpoint{0.000000in}{0.000000in}}{\pgfqpoint{3.000000in}{3.000000in}}%
\pgfusepath{clip}%
\pgfsetbuttcap%
\pgfsetroundjoin%
\definecolor{currentfill}{rgb}{1.000000,0.116921,0.000000}%
\pgfsetfillcolor{currentfill}%
\pgfsetlinewidth{0.000000pt}%
\definecolor{currentstroke}{rgb}{0.000000,0.000000,0.000000}%
\pgfsetstrokecolor{currentstroke}%
\pgfsetdash{}{0pt}%
\pgfpathmoveto{\pgfqpoint{0.956127in}{1.736412in}}%
\pgfpathlineto{\pgfqpoint{0.940895in}{1.746781in}}%
\pgfpathlineto{\pgfqpoint{1.006658in}{1.703048in}}%
\pgfpathlineto{\pgfqpoint{1.020283in}{1.693747in}}%
\pgfpathlineto{\pgfqpoint{0.956127in}{1.736412in}}%
\pgfpathclose%
\pgfusepath{fill}%
\end{pgfscope}%
\begin{pgfscope}%
\pgfpathrectangle{\pgfqpoint{0.000000in}{0.000000in}}{\pgfqpoint{3.000000in}{3.000000in}}%
\pgfusepath{clip}%
\pgfsetbuttcap%
\pgfsetroundjoin%
\definecolor{currentfill}{rgb}{1.000000,0.480029,0.000000}%
\pgfsetfillcolor{currentfill}%
\pgfsetlinewidth{0.000000pt}%
\definecolor{currentstroke}{rgb}{0.000000,0.000000,0.000000}%
\pgfsetstrokecolor{currentstroke}%
\pgfsetdash{}{0pt}%
\pgfpathmoveto{\pgfqpoint{1.167459in}{1.599726in}}%
\pgfpathlineto{\pgfqpoint{1.155975in}{1.610065in}}%
\pgfpathlineto{\pgfqpoint{1.234111in}{1.583032in}}%
\pgfpathlineto{\pgfqpoint{1.243288in}{1.573474in}}%
\pgfpathlineto{\pgfqpoint{1.167459in}{1.599726in}}%
\pgfpathclose%
\pgfusepath{fill}%
\end{pgfscope}%
\begin{pgfscope}%
\pgfpathrectangle{\pgfqpoint{0.000000in}{0.000000in}}{\pgfqpoint{3.000000in}{3.000000in}}%
\pgfusepath{clip}%
\pgfsetbuttcap%
\pgfsetroundjoin%
\definecolor{currentfill}{rgb}{1.000000,0.538126,0.000000}%
\pgfsetfillcolor{currentfill}%
\pgfsetlinewidth{0.000000pt}%
\definecolor{currentstroke}{rgb}{0.000000,0.000000,0.000000}%
\pgfsetstrokecolor{currentstroke}%
\pgfsetdash{}{0pt}%
\pgfpathmoveto{\pgfqpoint{1.775243in}{1.549816in}}%
\pgfpathlineto{\pgfqpoint{1.782713in}{1.559377in}}%
\pgfpathlineto{\pgfqpoint{1.864050in}{1.581569in}}%
\pgfpathlineto{\pgfqpoint{1.854094in}{1.571347in}}%
\pgfpathlineto{\pgfqpoint{1.775243in}{1.549816in}}%
\pgfpathclose%
\pgfusepath{fill}%
\end{pgfscope}%
\begin{pgfscope}%
\pgfpathrectangle{\pgfqpoint{0.000000in}{0.000000in}}{\pgfqpoint{3.000000in}{3.000000in}}%
\pgfusepath{clip}%
\pgfsetbuttcap%
\pgfsetroundjoin%
\definecolor{currentfill}{rgb}{1.000000,0.407407,0.000000}%
\pgfsetfillcolor{currentfill}%
\pgfsetlinewidth{0.000000pt}%
\definecolor{currentstroke}{rgb}{0.000000,0.000000,0.000000}%
\pgfsetstrokecolor{currentstroke}%
\pgfsetdash{}{0pt}%
\pgfpathmoveto{\pgfqpoint{1.874029in}{1.591368in}}%
\pgfpathlineto{\pgfqpoint{1.884030in}{1.600775in}}%
\pgfpathlineto{\pgfqpoint{1.961190in}{1.630608in}}%
\pgfpathlineto{\pgfqpoint{1.948980in}{1.620365in}}%
\pgfpathlineto{\pgfqpoint{1.874029in}{1.591368in}}%
\pgfpathclose%
\pgfusepath{fill}%
\end{pgfscope}%
\begin{pgfscope}%
\pgfpathrectangle{\pgfqpoint{0.000000in}{0.000000in}}{\pgfqpoint{3.000000in}{3.000000in}}%
\pgfusepath{clip}%
\pgfsetbuttcap%
\pgfsetroundjoin%
\definecolor{currentfill}{rgb}{0.553476,0.000000,0.000000}%
\pgfsetfillcolor{currentfill}%
\pgfsetlinewidth{0.000000pt}%
\definecolor{currentstroke}{rgb}{0.000000,0.000000,0.000000}%
\pgfsetstrokecolor{currentstroke}%
\pgfsetdash{}{0pt}%
\pgfpathmoveto{\pgfqpoint{2.269124in}{1.831175in}}%
\pgfpathlineto{\pgfqpoint{2.285000in}{1.840234in}}%
\pgfpathlineto{\pgfqpoint{2.337888in}{1.898781in}}%
\pgfpathlineto{\pgfqpoint{2.320992in}{1.888537in}}%
\pgfpathlineto{\pgfqpoint{2.269124in}{1.831175in}}%
\pgfpathclose%
\pgfusepath{fill}%
\end{pgfscope}%
\begin{pgfscope}%
\pgfpathrectangle{\pgfqpoint{0.000000in}{0.000000in}}{\pgfqpoint{3.000000in}{3.000000in}}%
\pgfusepath{clip}%
\pgfsetbuttcap%
\pgfsetroundjoin%
\definecolor{currentfill}{rgb}{0.927807,0.015251,0.000000}%
\pgfsetfillcolor{currentfill}%
\pgfsetlinewidth{0.000000pt}%
\definecolor{currentstroke}{rgb}{0.000000,0.000000,0.000000}%
\pgfsetstrokecolor{currentstroke}%
\pgfsetdash{}{0pt}%
\pgfpathmoveto{\pgfqpoint{2.112230in}{1.726466in}}%
\pgfpathlineto{\pgfqpoint{2.126488in}{1.735611in}}%
\pgfpathlineto{\pgfqpoint{2.190118in}{1.783100in}}%
\pgfpathlineto{\pgfqpoint{2.174392in}{1.772851in}}%
\pgfpathlineto{\pgfqpoint{2.112230in}{1.726466in}}%
\pgfpathclose%
\pgfusepath{fill}%
\end{pgfscope}%
\begin{pgfscope}%
\pgfpathrectangle{\pgfqpoint{0.000000in}{0.000000in}}{\pgfqpoint{3.000000in}{3.000000in}}%
\pgfusepath{clip}%
\pgfsetbuttcap%
\pgfsetroundjoin%
\definecolor{currentfill}{rgb}{1.000000,0.668845,0.000000}%
\pgfsetfillcolor{currentfill}%
\pgfsetlinewidth{0.000000pt}%
\definecolor{currentstroke}{rgb}{0.000000,0.000000,0.000000}%
\pgfsetstrokecolor{currentstroke}%
\pgfsetdash{}{0pt}%
\pgfpathmoveto{\pgfqpoint{1.597164in}{1.507320in}}%
\pgfpathlineto{\pgfqpoint{1.599084in}{1.517118in}}%
\pgfpathlineto{\pgfqpoint{1.685374in}{1.525139in}}%
\pgfpathlineto{\pgfqpoint{1.680629in}{1.515086in}}%
\pgfpathlineto{\pgfqpoint{1.597164in}{1.507320in}}%
\pgfpathclose%
\pgfusepath{fill}%
\end{pgfscope}%
\begin{pgfscope}%
\pgfpathrectangle{\pgfqpoint{0.000000in}{0.000000in}}{\pgfqpoint{3.000000in}{3.000000in}}%
\pgfusepath{clip}%
\pgfsetbuttcap%
\pgfsetroundjoin%
\definecolor{currentfill}{rgb}{1.000000,0.233115,0.000000}%
\pgfsetfillcolor{currentfill}%
\pgfsetlinewidth{0.000000pt}%
\definecolor{currentstroke}{rgb}{0.000000,0.000000,0.000000}%
\pgfsetstrokecolor{currentstroke}%
\pgfsetdash{}{0pt}%
\pgfpathmoveto{\pgfqpoint{1.985685in}{1.650032in}}%
\pgfpathlineto{\pgfqpoint{1.997971in}{1.659259in}}%
\pgfpathlineto{\pgfqpoint{2.069614in}{1.697513in}}%
\pgfpathlineto{\pgfqpoint{2.055461in}{1.687300in}}%
\pgfpathlineto{\pgfqpoint{1.985685in}{1.650032in}}%
\pgfpathclose%
\pgfusepath{fill}%
\end{pgfscope}%
\begin{pgfscope}%
\pgfpathrectangle{\pgfqpoint{0.000000in}{0.000000in}}{\pgfqpoint{3.000000in}{3.000000in}}%
\pgfusepath{clip}%
\pgfsetbuttcap%
\pgfsetroundjoin%
\definecolor{currentfill}{rgb}{1.000000,0.291213,0.000000}%
\pgfsetfillcolor{currentfill}%
\pgfsetlinewidth{0.000000pt}%
\definecolor{currentstroke}{rgb}{0.000000,0.000000,0.000000}%
\pgfsetstrokecolor{currentstroke}%
\pgfsetdash{}{0pt}%
\pgfpathmoveto{\pgfqpoint{1.061000in}{1.664114in}}%
\pgfpathlineto{\pgfqpoint{1.047454in}{1.674299in}}%
\pgfpathlineto{\pgfqpoint{1.121373in}{1.638841in}}%
\pgfpathlineto{\pgfqpoint{1.132932in}{1.629595in}}%
\pgfpathlineto{\pgfqpoint{1.061000in}{1.664114in}}%
\pgfpathclose%
\pgfusepath{fill}%
\end{pgfscope}%
\begin{pgfscope}%
\pgfpathrectangle{\pgfqpoint{0.000000in}{0.000000in}}{\pgfqpoint{3.000000in}{3.000000in}}%
\pgfusepath{clip}%
\pgfsetbuttcap%
\pgfsetroundjoin%
\definecolor{currentfill}{rgb}{0.678253,0.000000,0.000000}%
\pgfsetfillcolor{currentfill}%
\pgfsetlinewidth{0.000000pt}%
\definecolor{currentstroke}{rgb}{0.000000,0.000000,0.000000}%
\pgfsetstrokecolor{currentstroke}%
\pgfsetdash{}{0pt}%
\pgfpathmoveto{\pgfqpoint{0.808329in}{1.848820in}}%
\pgfpathlineto{\pgfqpoint{0.791769in}{1.859015in}}%
\pgfpathlineto{\pgfqpoint{0.848964in}{1.804091in}}%
\pgfpathlineto{\pgfqpoint{0.864350in}{1.795058in}}%
\pgfpathlineto{\pgfqpoint{0.808329in}{1.848820in}}%
\pgfpathclose%
\pgfusepath{fill}%
\end{pgfscope}%
\begin{pgfscope}%
\pgfpathrectangle{\pgfqpoint{0.000000in}{0.000000in}}{\pgfqpoint{3.000000in}{3.000000in}}%
\pgfusepath{clip}%
\pgfsetbuttcap%
\pgfsetroundjoin%
\definecolor{currentfill}{rgb}{1.000000,0.668845,0.000000}%
\pgfsetfillcolor{currentfill}%
\pgfsetlinewidth{0.000000pt}%
\definecolor{currentstroke}{rgb}{0.000000,0.000000,0.000000}%
\pgfsetstrokecolor{currentstroke}%
\pgfsetdash{}{0pt}%
\pgfpathmoveto{\pgfqpoint{1.427967in}{1.511768in}}%
\pgfpathlineto{\pgfqpoint{1.424152in}{1.521713in}}%
\pgfpathlineto{\pgfqpoint{1.511225in}{1.515965in}}%
\pgfpathlineto{\pgfqpoint{1.512186in}{1.506203in}}%
\pgfpathlineto{\pgfqpoint{1.427967in}{1.511768in}}%
\pgfpathclose%
\pgfusepath{fill}%
\end{pgfscope}%
\begin{pgfscope}%
\pgfpathrectangle{\pgfqpoint{0.000000in}{0.000000in}}{\pgfqpoint{3.000000in}{3.000000in}}%
\pgfusepath{clip}%
\pgfsetbuttcap%
\pgfsetroundjoin%
\definecolor{currentfill}{rgb}{1.000000,0.610748,0.000000}%
\pgfsetfillcolor{currentfill}%
\pgfsetlinewidth{0.000000pt}%
\definecolor{currentstroke}{rgb}{0.000000,0.000000,0.000000}%
\pgfsetstrokecolor{currentstroke}%
\pgfsetdash{}{0pt}%
\pgfpathmoveto{\pgfqpoint{1.685374in}{1.525139in}}%
\pgfpathlineto{\pgfqpoint{1.690130in}{1.534694in}}%
\pgfpathlineto{\pgfqpoint{1.775243in}{1.549816in}}%
\pgfpathlineto{\pgfqpoint{1.767791in}{1.539796in}}%
\pgfpathlineto{\pgfqpoint{1.685374in}{1.525139in}}%
\pgfpathclose%
\pgfusepath{fill}%
\end{pgfscope}%
\begin{pgfscope}%
\pgfpathrectangle{\pgfqpoint{0.000000in}{0.000000in}}{\pgfqpoint{3.000000in}{3.000000in}}%
\pgfusepath{clip}%
\pgfsetbuttcap%
\pgfsetroundjoin%
\definecolor{currentfill}{rgb}{1.000000,0.668845,0.000000}%
\pgfsetfillcolor{currentfill}%
\pgfsetlinewidth{0.000000pt}%
\definecolor{currentstroke}{rgb}{0.000000,0.000000,0.000000}%
\pgfsetstrokecolor{currentstroke}%
\pgfsetdash{}{0pt}%
\pgfpathmoveto{\pgfqpoint{1.512186in}{1.506203in}}%
\pgfpathlineto{\pgfqpoint{1.511225in}{1.515965in}}%
\pgfpathlineto{\pgfqpoint{1.599084in}{1.517118in}}%
\pgfpathlineto{\pgfqpoint{1.597164in}{1.507320in}}%
\pgfpathlineto{\pgfqpoint{1.512186in}{1.506203in}}%
\pgfpathclose%
\pgfusepath{fill}%
\end{pgfscope}%
\begin{pgfscope}%
\pgfpathrectangle{\pgfqpoint{0.000000in}{0.000000in}}{\pgfqpoint{3.000000in}{3.000000in}}%
\pgfusepath{clip}%
\pgfsetbuttcap%
\pgfsetroundjoin%
\definecolor{currentfill}{rgb}{0.999109,0.073348,0.000000}%
\pgfsetfillcolor{currentfill}%
\pgfsetlinewidth{0.000000pt}%
\definecolor{currentstroke}{rgb}{0.000000,0.000000,0.000000}%
\pgfsetstrokecolor{currentstroke}%
\pgfsetdash{}{0pt}%
\pgfpathmoveto{\pgfqpoint{0.940895in}{1.746781in}}%
\pgfpathlineto{\pgfqpoint{0.925637in}{1.756893in}}%
\pgfpathlineto{\pgfqpoint{0.993007in}{1.712091in}}%
\pgfpathlineto{\pgfqpoint{1.006658in}{1.703048in}}%
\pgfpathlineto{\pgfqpoint{0.940895in}{1.746781in}}%
\pgfpathclose%
\pgfusepath{fill}%
\end{pgfscope}%
\begin{pgfscope}%
\pgfpathrectangle{\pgfqpoint{0.000000in}{0.000000in}}{\pgfqpoint{3.000000in}{3.000000in}}%
\pgfusepath{clip}%
\pgfsetbuttcap%
\pgfsetroundjoin%
\definecolor{currentfill}{rgb}{1.000000,0.538126,0.000000}%
\pgfsetfillcolor{currentfill}%
\pgfsetlinewidth{0.000000pt}%
\definecolor{currentstroke}{rgb}{0.000000,0.000000,0.000000}%
\pgfsetstrokecolor{currentstroke}%
\pgfsetdash{}{0pt}%
\pgfpathmoveto{\pgfqpoint{1.252443in}{1.563493in}}%
\pgfpathlineto{\pgfqpoint{1.243288in}{1.573474in}}%
\pgfpathlineto{\pgfqpoint{1.327031in}{1.553420in}}%
\pgfpathlineto{\pgfqpoint{1.333621in}{1.544037in}}%
\pgfpathlineto{\pgfqpoint{1.252443in}{1.563493in}}%
\pgfpathclose%
\pgfusepath{fill}%
\end{pgfscope}%
\begin{pgfscope}%
\pgfpathrectangle{\pgfqpoint{0.000000in}{0.000000in}}{\pgfqpoint{3.000000in}{3.000000in}}%
\pgfusepath{clip}%
\pgfsetbuttcap%
\pgfsetroundjoin%
\definecolor{currentfill}{rgb}{0.500000,0.000000,0.000000}%
\pgfsetfillcolor{currentfill}%
\pgfsetlinewidth{0.000000pt}%
\definecolor{currentstroke}{rgb}{0.000000,0.000000,0.000000}%
\pgfsetstrokecolor{currentstroke}%
\pgfsetdash{}{0pt}%
\pgfpathmoveto{\pgfqpoint{2.285000in}{1.840234in}}%
\pgfpathlineto{\pgfqpoint{2.300901in}{1.849131in}}%
\pgfpathlineto{\pgfqpoint{2.354806in}{1.908861in}}%
\pgfpathlineto{\pgfqpoint{2.337888in}{1.898781in}}%
\pgfpathlineto{\pgfqpoint{2.285000in}{1.840234in}}%
\pgfpathclose%
\pgfusepath{fill}%
\end{pgfscope}%
\begin{pgfscope}%
\pgfpathrectangle{\pgfqpoint{0.000000in}{0.000000in}}{\pgfqpoint{3.000000in}{3.000000in}}%
\pgfusepath{clip}%
\pgfsetbuttcap%
\pgfsetroundjoin%
\definecolor{currentfill}{rgb}{1.000000,0.610748,0.000000}%
\pgfsetfillcolor{currentfill}%
\pgfsetlinewidth{0.000000pt}%
\definecolor{currentstroke}{rgb}{0.000000,0.000000,0.000000}%
\pgfsetstrokecolor{currentstroke}%
\pgfsetdash{}{0pt}%
\pgfpathmoveto{\pgfqpoint{1.340194in}{1.534195in}}%
\pgfpathlineto{\pgfqpoint{1.333621in}{1.544037in}}%
\pgfpathlineto{\pgfqpoint{1.420328in}{1.531158in}}%
\pgfpathlineto{\pgfqpoint{1.424152in}{1.521713in}}%
\pgfpathlineto{\pgfqpoint{1.340194in}{1.534195in}}%
\pgfpathclose%
\pgfusepath{fill}%
\end{pgfscope}%
\begin{pgfscope}%
\pgfpathrectangle{\pgfqpoint{0.000000in}{0.000000in}}{\pgfqpoint{3.000000in}{3.000000in}}%
\pgfusepath{clip}%
\pgfsetbuttcap%
\pgfsetroundjoin%
\definecolor{currentfill}{rgb}{1.000000,0.407407,0.000000}%
\pgfsetfillcolor{currentfill}%
\pgfsetlinewidth{0.000000pt}%
\definecolor{currentstroke}{rgb}{0.000000,0.000000,0.000000}%
\pgfsetstrokecolor{currentstroke}%
\pgfsetdash{}{0pt}%
\pgfpathmoveto{\pgfqpoint{1.155975in}{1.610065in}}%
\pgfpathlineto{\pgfqpoint{1.144466in}{1.620012in}}%
\pgfpathlineto{\pgfqpoint{1.224913in}{1.592197in}}%
\pgfpathlineto{\pgfqpoint{1.234111in}{1.583032in}}%
\pgfpathlineto{\pgfqpoint{1.155975in}{1.610065in}}%
\pgfpathclose%
\pgfusepath{fill}%
\end{pgfscope}%
\begin{pgfscope}%
\pgfpathrectangle{\pgfqpoint{0.000000in}{0.000000in}}{\pgfqpoint{3.000000in}{3.000000in}}%
\pgfusepath{clip}%
\pgfsetbuttcap%
\pgfsetroundjoin%
\definecolor{currentfill}{rgb}{0.856506,0.000000,0.000000}%
\pgfsetfillcolor{currentfill}%
\pgfsetlinewidth{0.000000pt}%
\definecolor{currentstroke}{rgb}{0.000000,0.000000,0.000000}%
\pgfsetstrokecolor{currentstroke}%
\pgfsetdash{}{0pt}%
\pgfpathmoveto{\pgfqpoint{2.126488in}{1.735611in}}%
\pgfpathlineto{\pgfqpoint{2.140772in}{1.744527in}}%
\pgfpathlineto{\pgfqpoint{2.205870in}{1.793120in}}%
\pgfpathlineto{\pgfqpoint{2.190118in}{1.783100in}}%
\pgfpathlineto{\pgfqpoint{2.126488in}{1.735611in}}%
\pgfpathclose%
\pgfusepath{fill}%
\end{pgfscope}%
\begin{pgfscope}%
\pgfpathrectangle{\pgfqpoint{0.000000in}{0.000000in}}{\pgfqpoint{3.000000in}{3.000000in}}%
\pgfusepath{clip}%
\pgfsetbuttcap%
\pgfsetroundjoin%
\definecolor{currentfill}{rgb}{0.606952,0.000000,0.000000}%
\pgfsetfillcolor{currentfill}%
\pgfsetlinewidth{0.000000pt}%
\definecolor{currentstroke}{rgb}{0.000000,0.000000,0.000000}%
\pgfsetstrokecolor{currentstroke}%
\pgfsetdash{}{0pt}%
\pgfpathmoveto{\pgfqpoint{0.791769in}{1.859015in}}%
\pgfpathlineto{\pgfqpoint{0.775185in}{1.869029in}}%
\pgfpathlineto{\pgfqpoint{0.833552in}{1.812944in}}%
\pgfpathlineto{\pgfqpoint{0.848964in}{1.804091in}}%
\pgfpathlineto{\pgfqpoint{0.791769in}{1.859015in}}%
\pgfpathclose%
\pgfusepath{fill}%
\end{pgfscope}%
\begin{pgfscope}%
\pgfpathrectangle{\pgfqpoint{0.000000in}{0.000000in}}{\pgfqpoint{3.000000in}{3.000000in}}%
\pgfusepath{clip}%
\pgfsetbuttcap%
\pgfsetroundjoin%
\definecolor{currentfill}{rgb}{1.000000,0.349310,0.000000}%
\pgfsetfillcolor{currentfill}%
\pgfsetlinewidth{0.000000pt}%
\definecolor{currentstroke}{rgb}{0.000000,0.000000,0.000000}%
\pgfsetstrokecolor{currentstroke}%
\pgfsetdash{}{0pt}%
\pgfpathmoveto{\pgfqpoint{1.884030in}{1.600775in}}%
\pgfpathlineto{\pgfqpoint{1.894054in}{1.609818in}}%
\pgfpathlineto{\pgfqpoint{1.973425in}{1.640489in}}%
\pgfpathlineto{\pgfqpoint{1.961190in}{1.630608in}}%
\pgfpathlineto{\pgfqpoint{1.884030in}{1.600775in}}%
\pgfpathclose%
\pgfusepath{fill}%
\end{pgfscope}%
\begin{pgfscope}%
\pgfpathrectangle{\pgfqpoint{0.000000in}{0.000000in}}{\pgfqpoint{3.000000in}{3.000000in}}%
\pgfusepath{clip}%
\pgfsetbuttcap%
\pgfsetroundjoin%
\definecolor{currentfill}{rgb}{1.000000,0.175018,0.000000}%
\pgfsetfillcolor{currentfill}%
\pgfsetlinewidth{0.000000pt}%
\definecolor{currentstroke}{rgb}{0.000000,0.000000,0.000000}%
\pgfsetstrokecolor{currentstroke}%
\pgfsetdash{}{0pt}%
\pgfpathmoveto{\pgfqpoint{1.997971in}{1.659259in}}%
\pgfpathlineto{\pgfqpoint{2.010283in}{1.668192in}}%
\pgfpathlineto{\pgfqpoint{2.083793in}{1.707434in}}%
\pgfpathlineto{\pgfqpoint{2.069614in}{1.697513in}}%
\pgfpathlineto{\pgfqpoint{1.997971in}{1.659259in}}%
\pgfpathclose%
\pgfusepath{fill}%
\end{pgfscope}%
\begin{pgfscope}%
\pgfpathrectangle{\pgfqpoint{0.000000in}{0.000000in}}{\pgfqpoint{3.000000in}{3.000000in}}%
\pgfusepath{clip}%
\pgfsetbuttcap%
\pgfsetroundjoin%
\definecolor{currentfill}{rgb}{1.000000,0.480029,0.000000}%
\pgfsetfillcolor{currentfill}%
\pgfsetlinewidth{0.000000pt}%
\definecolor{currentstroke}{rgb}{0.000000,0.000000,0.000000}%
\pgfsetstrokecolor{currentstroke}%
\pgfsetdash{}{0pt}%
\pgfpathmoveto{\pgfqpoint{1.782713in}{1.559377in}}%
\pgfpathlineto{\pgfqpoint{1.790201in}{1.568513in}}%
\pgfpathlineto{\pgfqpoint{1.874029in}{1.591368in}}%
\pgfpathlineto{\pgfqpoint{1.864050in}{1.581569in}}%
\pgfpathlineto{\pgfqpoint{1.782713in}{1.559377in}}%
\pgfpathclose%
\pgfusepath{fill}%
\end{pgfscope}%
\begin{pgfscope}%
\pgfpathrectangle{\pgfqpoint{0.000000in}{0.000000in}}{\pgfqpoint{3.000000in}{3.000000in}}%
\pgfusepath{clip}%
\pgfsetbuttcap%
\pgfsetroundjoin%
\definecolor{currentfill}{rgb}{1.000000,0.233115,0.000000}%
\pgfsetfillcolor{currentfill}%
\pgfsetlinewidth{0.000000pt}%
\definecolor{currentstroke}{rgb}{0.000000,0.000000,0.000000}%
\pgfsetstrokecolor{currentstroke}%
\pgfsetdash{}{0pt}%
\pgfpathmoveto{\pgfqpoint{1.047454in}{1.674299in}}%
\pgfpathlineto{\pgfqpoint{1.033881in}{1.684170in}}%
\pgfpathlineto{\pgfqpoint{1.109790in}{1.647771in}}%
\pgfpathlineto{\pgfqpoint{1.121373in}{1.638841in}}%
\pgfpathlineto{\pgfqpoint{1.047454in}{1.674299in}}%
\pgfpathclose%
\pgfusepath{fill}%
\end{pgfscope}%
\begin{pgfscope}%
\pgfpathrectangle{\pgfqpoint{0.000000in}{0.000000in}}{\pgfqpoint{3.000000in}{3.000000in}}%
\pgfusepath{clip}%
\pgfsetbuttcap%
\pgfsetroundjoin%
\definecolor{currentfill}{rgb}{1.000000,0.610748,0.000000}%
\pgfsetfillcolor{currentfill}%
\pgfsetlinewidth{0.000000pt}%
\definecolor{currentstroke}{rgb}{0.000000,0.000000,0.000000}%
\pgfsetstrokecolor{currentstroke}%
\pgfsetdash{}{0pt}%
\pgfpathmoveto{\pgfqpoint{1.599084in}{1.517118in}}%
\pgfpathlineto{\pgfqpoint{1.601008in}{1.526417in}}%
\pgfpathlineto{\pgfqpoint{1.690130in}{1.534694in}}%
\pgfpathlineto{\pgfqpoint{1.685374in}{1.525139in}}%
\pgfpathlineto{\pgfqpoint{1.599084in}{1.517118in}}%
\pgfpathclose%
\pgfusepath{fill}%
\end{pgfscope}%
\begin{pgfscope}%
\pgfpathrectangle{\pgfqpoint{0.000000in}{0.000000in}}{\pgfqpoint{3.000000in}{3.000000in}}%
\pgfusepath{clip}%
\pgfsetbuttcap%
\pgfsetroundjoin%
\definecolor{currentfill}{rgb}{0.927807,0.015251,0.000000}%
\pgfsetfillcolor{currentfill}%
\pgfsetlinewidth{0.000000pt}%
\definecolor{currentstroke}{rgb}{0.000000,0.000000,0.000000}%
\pgfsetstrokecolor{currentstroke}%
\pgfsetdash{}{0pt}%
\pgfpathmoveto{\pgfqpoint{0.925637in}{1.756893in}}%
\pgfpathlineto{\pgfqpoint{0.910354in}{1.766763in}}%
\pgfpathlineto{\pgfqpoint{0.979329in}{1.720892in}}%
\pgfpathlineto{\pgfqpoint{0.993007in}{1.712091in}}%
\pgfpathlineto{\pgfqpoint{0.925637in}{1.756893in}}%
\pgfpathclose%
\pgfusepath{fill}%
\end{pgfscope}%
\begin{pgfscope}%
\pgfpathrectangle{\pgfqpoint{0.000000in}{0.000000in}}{\pgfqpoint{3.000000in}{3.000000in}}%
\pgfusepath{clip}%
\pgfsetbuttcap%
\pgfsetroundjoin%
\definecolor{currentfill}{rgb}{1.000000,0.610748,0.000000}%
\pgfsetfillcolor{currentfill}%
\pgfsetlinewidth{0.000000pt}%
\definecolor{currentstroke}{rgb}{0.000000,0.000000,0.000000}%
\pgfsetstrokecolor{currentstroke}%
\pgfsetdash{}{0pt}%
\pgfpathmoveto{\pgfqpoint{1.424152in}{1.521713in}}%
\pgfpathlineto{\pgfqpoint{1.420328in}{1.531158in}}%
\pgfpathlineto{\pgfqpoint{1.510261in}{1.525227in}}%
\pgfpathlineto{\pgfqpoint{1.511225in}{1.515965in}}%
\pgfpathlineto{\pgfqpoint{1.424152in}{1.521713in}}%
\pgfpathclose%
\pgfusepath{fill}%
\end{pgfscope}%
\begin{pgfscope}%
\pgfpathrectangle{\pgfqpoint{0.000000in}{0.000000in}}{\pgfqpoint{3.000000in}{3.000000in}}%
\pgfusepath{clip}%
\pgfsetbuttcap%
\pgfsetroundjoin%
\definecolor{currentfill}{rgb}{1.000000,0.538126,0.000000}%
\pgfsetfillcolor{currentfill}%
\pgfsetlinewidth{0.000000pt}%
\definecolor{currentstroke}{rgb}{0.000000,0.000000,0.000000}%
\pgfsetstrokecolor{currentstroke}%
\pgfsetdash{}{0pt}%
\pgfpathmoveto{\pgfqpoint{1.690130in}{1.534694in}}%
\pgfpathlineto{\pgfqpoint{1.694899in}{1.543788in}}%
\pgfpathlineto{\pgfqpoint{1.782713in}{1.559377in}}%
\pgfpathlineto{\pgfqpoint{1.775243in}{1.549816in}}%
\pgfpathlineto{\pgfqpoint{1.690130in}{1.534694in}}%
\pgfpathclose%
\pgfusepath{fill}%
\end{pgfscope}%
\begin{pgfscope}%
\pgfpathrectangle{\pgfqpoint{0.000000in}{0.000000in}}{\pgfqpoint{3.000000in}{3.000000in}}%
\pgfusepath{clip}%
\pgfsetbuttcap%
\pgfsetroundjoin%
\definecolor{currentfill}{rgb}{0.803030,0.000000,0.000000}%
\pgfsetfillcolor{currentfill}%
\pgfsetlinewidth{0.000000pt}%
\definecolor{currentstroke}{rgb}{0.000000,0.000000,0.000000}%
\pgfsetstrokecolor{currentstroke}%
\pgfsetdash{}{0pt}%
\pgfpathmoveto{\pgfqpoint{2.140772in}{1.744527in}}%
\pgfpathlineto{\pgfqpoint{2.155083in}{1.753228in}}%
\pgfpathlineto{\pgfqpoint{2.221646in}{1.802927in}}%
\pgfpathlineto{\pgfqpoint{2.205870in}{1.793120in}}%
\pgfpathlineto{\pgfqpoint{2.140772in}{1.744527in}}%
\pgfpathclose%
\pgfusepath{fill}%
\end{pgfscope}%
\begin{pgfscope}%
\pgfpathrectangle{\pgfqpoint{0.000000in}{0.000000in}}{\pgfqpoint{3.000000in}{3.000000in}}%
\pgfusepath{clip}%
\pgfsetbuttcap%
\pgfsetroundjoin%
\definecolor{currentfill}{rgb}{0.553476,0.000000,0.000000}%
\pgfsetfillcolor{currentfill}%
\pgfsetlinewidth{0.000000pt}%
\definecolor{currentstroke}{rgb}{0.000000,0.000000,0.000000}%
\pgfsetstrokecolor{currentstroke}%
\pgfsetdash{}{0pt}%
\pgfpathmoveto{\pgfqpoint{0.775185in}{1.869029in}}%
\pgfpathlineto{\pgfqpoint{0.758579in}{1.878872in}}%
\pgfpathlineto{\pgfqpoint{0.818115in}{1.821624in}}%
\pgfpathlineto{\pgfqpoint{0.833552in}{1.812944in}}%
\pgfpathlineto{\pgfqpoint{0.775185in}{1.869029in}}%
\pgfpathclose%
\pgfusepath{fill}%
\end{pgfscope}%
\begin{pgfscope}%
\pgfpathrectangle{\pgfqpoint{0.000000in}{0.000000in}}{\pgfqpoint{3.000000in}{3.000000in}}%
\pgfusepath{clip}%
\pgfsetbuttcap%
\pgfsetroundjoin%
\definecolor{currentfill}{rgb}{1.000000,0.480029,0.000000}%
\pgfsetfillcolor{currentfill}%
\pgfsetlinewidth{0.000000pt}%
\definecolor{currentstroke}{rgb}{0.000000,0.000000,0.000000}%
\pgfsetstrokecolor{currentstroke}%
\pgfsetdash{}{0pt}%
\pgfpathmoveto{\pgfqpoint{1.243288in}{1.573474in}}%
\pgfpathlineto{\pgfqpoint{1.234111in}{1.583032in}}%
\pgfpathlineto{\pgfqpoint{1.320425in}{1.562379in}}%
\pgfpathlineto{\pgfqpoint{1.327031in}{1.553420in}}%
\pgfpathlineto{\pgfqpoint{1.243288in}{1.573474in}}%
\pgfpathclose%
\pgfusepath{fill}%
\end{pgfscope}%
\begin{pgfscope}%
\pgfpathrectangle{\pgfqpoint{0.000000in}{0.000000in}}{\pgfqpoint{3.000000in}{3.000000in}}%
\pgfusepath{clip}%
\pgfsetbuttcap%
\pgfsetroundjoin%
\definecolor{currentfill}{rgb}{1.000000,0.610748,0.000000}%
\pgfsetfillcolor{currentfill}%
\pgfsetlinewidth{0.000000pt}%
\definecolor{currentstroke}{rgb}{0.000000,0.000000,0.000000}%
\pgfsetstrokecolor{currentstroke}%
\pgfsetdash{}{0pt}%
\pgfpathmoveto{\pgfqpoint{1.511225in}{1.515965in}}%
\pgfpathlineto{\pgfqpoint{1.510261in}{1.525227in}}%
\pgfpathlineto{\pgfqpoint{1.601008in}{1.526417in}}%
\pgfpathlineto{\pgfqpoint{1.599084in}{1.517118in}}%
\pgfpathlineto{\pgfqpoint{1.511225in}{1.515965in}}%
\pgfpathclose%
\pgfusepath{fill}%
\end{pgfscope}%
\begin{pgfscope}%
\pgfpathrectangle{\pgfqpoint{0.000000in}{0.000000in}}{\pgfqpoint{3.000000in}{3.000000in}}%
\pgfusepath{clip}%
\pgfsetbuttcap%
\pgfsetroundjoin%
\definecolor{currentfill}{rgb}{1.000000,0.349310,0.000000}%
\pgfsetfillcolor{currentfill}%
\pgfsetlinewidth{0.000000pt}%
\definecolor{currentstroke}{rgb}{0.000000,0.000000,0.000000}%
\pgfsetstrokecolor{currentstroke}%
\pgfsetdash{}{0pt}%
\pgfpathmoveto{\pgfqpoint{1.144466in}{1.620012in}}%
\pgfpathlineto{\pgfqpoint{1.132932in}{1.629595in}}%
\pgfpathlineto{\pgfqpoint{1.215694in}{1.600998in}}%
\pgfpathlineto{\pgfqpoint{1.224913in}{1.592197in}}%
\pgfpathlineto{\pgfqpoint{1.144466in}{1.620012in}}%
\pgfpathclose%
\pgfusepath{fill}%
\end{pgfscope}%
\begin{pgfscope}%
\pgfpathrectangle{\pgfqpoint{0.000000in}{0.000000in}}{\pgfqpoint{3.000000in}{3.000000in}}%
\pgfusepath{clip}%
\pgfsetbuttcap%
\pgfsetroundjoin%
\definecolor{currentfill}{rgb}{1.000000,0.538126,0.000000}%
\pgfsetfillcolor{currentfill}%
\pgfsetlinewidth{0.000000pt}%
\definecolor{currentstroke}{rgb}{0.000000,0.000000,0.000000}%
\pgfsetstrokecolor{currentstroke}%
\pgfsetdash{}{0pt}%
\pgfpathmoveto{\pgfqpoint{1.333621in}{1.544037in}}%
\pgfpathlineto{\pgfqpoint{1.327031in}{1.553420in}}%
\pgfpathlineto{\pgfqpoint{1.416494in}{1.540144in}}%
\pgfpathlineto{\pgfqpoint{1.420328in}{1.531158in}}%
\pgfpathlineto{\pgfqpoint{1.333621in}{1.544037in}}%
\pgfpathclose%
\pgfusepath{fill}%
\end{pgfscope}%
\begin{pgfscope}%
\pgfpathrectangle{\pgfqpoint{0.000000in}{0.000000in}}{\pgfqpoint{3.000000in}{3.000000in}}%
\pgfusepath{clip}%
\pgfsetbuttcap%
\pgfsetroundjoin%
\definecolor{currentfill}{rgb}{1.000000,0.116921,0.000000}%
\pgfsetfillcolor{currentfill}%
\pgfsetlinewidth{0.000000pt}%
\definecolor{currentstroke}{rgb}{0.000000,0.000000,0.000000}%
\pgfsetstrokecolor{currentstroke}%
\pgfsetdash{}{0pt}%
\pgfpathmoveto{\pgfqpoint{2.010283in}{1.668192in}}%
\pgfpathlineto{\pgfqpoint{2.022620in}{1.676849in}}%
\pgfpathlineto{\pgfqpoint{2.097999in}{1.717079in}}%
\pgfpathlineto{\pgfqpoint{2.083793in}{1.707434in}}%
\pgfpathlineto{\pgfqpoint{2.010283in}{1.668192in}}%
\pgfpathclose%
\pgfusepath{fill}%
\end{pgfscope}%
\begin{pgfscope}%
\pgfpathrectangle{\pgfqpoint{0.000000in}{0.000000in}}{\pgfqpoint{3.000000in}{3.000000in}}%
\pgfusepath{clip}%
\pgfsetbuttcap%
\pgfsetroundjoin%
\definecolor{currentfill}{rgb}{1.000000,0.291213,0.000000}%
\pgfsetfillcolor{currentfill}%
\pgfsetlinewidth{0.000000pt}%
\definecolor{currentstroke}{rgb}{0.000000,0.000000,0.000000}%
\pgfsetstrokecolor{currentstroke}%
\pgfsetdash{}{0pt}%
\pgfpathmoveto{\pgfqpoint{1.894054in}{1.609818in}}%
\pgfpathlineto{\pgfqpoint{1.904100in}{1.618522in}}%
\pgfpathlineto{\pgfqpoint{1.985685in}{1.650032in}}%
\pgfpathlineto{\pgfqpoint{1.973425in}{1.640489in}}%
\pgfpathlineto{\pgfqpoint{1.894054in}{1.609818in}}%
\pgfpathclose%
\pgfusepath{fill}%
\end{pgfscope}%
\begin{pgfscope}%
\pgfpathrectangle{\pgfqpoint{0.000000in}{0.000000in}}{\pgfqpoint{3.000000in}{3.000000in}}%
\pgfusepath{clip}%
\pgfsetbuttcap%
\pgfsetroundjoin%
\definecolor{currentfill}{rgb}{1.000000,0.175018,0.000000}%
\pgfsetfillcolor{currentfill}%
\pgfsetlinewidth{0.000000pt}%
\definecolor{currentstroke}{rgb}{0.000000,0.000000,0.000000}%
\pgfsetstrokecolor{currentstroke}%
\pgfsetdash{}{0pt}%
\pgfpathmoveto{\pgfqpoint{1.033881in}{1.684170in}}%
\pgfpathlineto{\pgfqpoint{1.020283in}{1.693747in}}%
\pgfpathlineto{\pgfqpoint{1.098182in}{1.656405in}}%
\pgfpathlineto{\pgfqpoint{1.109790in}{1.647771in}}%
\pgfpathlineto{\pgfqpoint{1.033881in}{1.684170in}}%
\pgfpathclose%
\pgfusepath{fill}%
\end{pgfscope}%
\begin{pgfscope}%
\pgfpathrectangle{\pgfqpoint{0.000000in}{0.000000in}}{\pgfqpoint{3.000000in}{3.000000in}}%
\pgfusepath{clip}%
\pgfsetbuttcap%
\pgfsetroundjoin%
\definecolor{currentfill}{rgb}{0.856506,0.000000,0.000000}%
\pgfsetfillcolor{currentfill}%
\pgfsetlinewidth{0.000000pt}%
\definecolor{currentstroke}{rgb}{0.000000,0.000000,0.000000}%
\pgfsetstrokecolor{currentstroke}%
\pgfsetdash{}{0pt}%
\pgfpathmoveto{\pgfqpoint{0.910354in}{1.766763in}}%
\pgfpathlineto{\pgfqpoint{0.895045in}{1.776405in}}%
\pgfpathlineto{\pgfqpoint{0.965626in}{1.729464in}}%
\pgfpathlineto{\pgfqpoint{0.979329in}{1.720892in}}%
\pgfpathlineto{\pgfqpoint{0.910354in}{1.766763in}}%
\pgfpathclose%
\pgfusepath{fill}%
\end{pgfscope}%
\begin{pgfscope}%
\pgfpathrectangle{\pgfqpoint{0.000000in}{0.000000in}}{\pgfqpoint{3.000000in}{3.000000in}}%
\pgfusepath{clip}%
\pgfsetbuttcap%
\pgfsetroundjoin%
\definecolor{currentfill}{rgb}{1.000000,0.407407,0.000000}%
\pgfsetfillcolor{currentfill}%
\pgfsetlinewidth{0.000000pt}%
\definecolor{currentstroke}{rgb}{0.000000,0.000000,0.000000}%
\pgfsetstrokecolor{currentstroke}%
\pgfsetdash{}{0pt}%
\pgfpathmoveto{\pgfqpoint{1.790201in}{1.568513in}}%
\pgfpathlineto{\pgfqpoint{1.797707in}{1.577257in}}%
\pgfpathlineto{\pgfqpoint{1.884030in}{1.600775in}}%
\pgfpathlineto{\pgfqpoint{1.874029in}{1.591368in}}%
\pgfpathlineto{\pgfqpoint{1.790201in}{1.568513in}}%
\pgfpathclose%
\pgfusepath{fill}%
\end{pgfscope}%
\begin{pgfscope}%
\pgfpathrectangle{\pgfqpoint{0.000000in}{0.000000in}}{\pgfqpoint{3.000000in}{3.000000in}}%
\pgfusepath{clip}%
\pgfsetbuttcap%
\pgfsetroundjoin%
\definecolor{currentfill}{rgb}{0.500000,0.000000,0.000000}%
\pgfsetfillcolor{currentfill}%
\pgfsetlinewidth{0.000000pt}%
\definecolor{currentstroke}{rgb}{0.000000,0.000000,0.000000}%
\pgfsetstrokecolor{currentstroke}%
\pgfsetdash{}{0pt}%
\pgfpathmoveto{\pgfqpoint{0.758579in}{1.878872in}}%
\pgfpathlineto{\pgfqpoint{0.741950in}{1.888552in}}%
\pgfpathlineto{\pgfqpoint{0.802653in}{1.830141in}}%
\pgfpathlineto{\pgfqpoint{0.818115in}{1.821624in}}%
\pgfpathlineto{\pgfqpoint{0.758579in}{1.878872in}}%
\pgfpathclose%
\pgfusepath{fill}%
\end{pgfscope}%
\begin{pgfscope}%
\pgfpathrectangle{\pgfqpoint{0.000000in}{0.000000in}}{\pgfqpoint{3.000000in}{3.000000in}}%
\pgfusepath{clip}%
\pgfsetbuttcap%
\pgfsetroundjoin%
\definecolor{currentfill}{rgb}{0.731729,0.000000,0.000000}%
\pgfsetfillcolor{currentfill}%
\pgfsetlinewidth{0.000000pt}%
\definecolor{currentstroke}{rgb}{0.000000,0.000000,0.000000}%
\pgfsetstrokecolor{currentstroke}%
\pgfsetdash{}{0pt}%
\pgfpathmoveto{\pgfqpoint{2.155083in}{1.753228in}}%
\pgfpathlineto{\pgfqpoint{2.169419in}{1.761727in}}%
\pgfpathlineto{\pgfqpoint{2.237447in}{1.812531in}}%
\pgfpathlineto{\pgfqpoint{2.221646in}{1.802927in}}%
\pgfpathlineto{\pgfqpoint{2.155083in}{1.753228in}}%
\pgfpathclose%
\pgfusepath{fill}%
\end{pgfscope}%
\begin{pgfscope}%
\pgfpathrectangle{\pgfqpoint{0.000000in}{0.000000in}}{\pgfqpoint{3.000000in}{3.000000in}}%
\pgfusepath{clip}%
\pgfsetbuttcap%
\pgfsetroundjoin%
\definecolor{currentfill}{rgb}{1.000000,0.538126,0.000000}%
\pgfsetfillcolor{currentfill}%
\pgfsetlinewidth{0.000000pt}%
\definecolor{currentstroke}{rgb}{0.000000,0.000000,0.000000}%
\pgfsetstrokecolor{currentstroke}%
\pgfsetdash{}{0pt}%
\pgfpathmoveto{\pgfqpoint{1.601008in}{1.526417in}}%
\pgfpathlineto{\pgfqpoint{1.602938in}{1.535256in}}%
\pgfpathlineto{\pgfqpoint{1.694899in}{1.543788in}}%
\pgfpathlineto{\pgfqpoint{1.690130in}{1.534694in}}%
\pgfpathlineto{\pgfqpoint{1.601008in}{1.526417in}}%
\pgfpathclose%
\pgfusepath{fill}%
\end{pgfscope}%
\begin{pgfscope}%
\pgfpathrectangle{\pgfqpoint{0.000000in}{0.000000in}}{\pgfqpoint{3.000000in}{3.000000in}}%
\pgfusepath{clip}%
\pgfsetbuttcap%
\pgfsetroundjoin%
\definecolor{currentfill}{rgb}{1.000000,0.480029,0.000000}%
\pgfsetfillcolor{currentfill}%
\pgfsetlinewidth{0.000000pt}%
\definecolor{currentstroke}{rgb}{0.000000,0.000000,0.000000}%
\pgfsetstrokecolor{currentstroke}%
\pgfsetdash{}{0pt}%
\pgfpathmoveto{\pgfqpoint{1.694899in}{1.543788in}}%
\pgfpathlineto{\pgfqpoint{1.699680in}{1.552458in}}%
\pgfpathlineto{\pgfqpoint{1.790201in}{1.568513in}}%
\pgfpathlineto{\pgfqpoint{1.782713in}{1.559377in}}%
\pgfpathlineto{\pgfqpoint{1.694899in}{1.543788in}}%
\pgfpathclose%
\pgfusepath{fill}%
\end{pgfscope}%
\begin{pgfscope}%
\pgfpathrectangle{\pgfqpoint{0.000000in}{0.000000in}}{\pgfqpoint{3.000000in}{3.000000in}}%
\pgfusepath{clip}%
\pgfsetbuttcap%
\pgfsetroundjoin%
\definecolor{currentfill}{rgb}{1.000000,0.538126,0.000000}%
\pgfsetfillcolor{currentfill}%
\pgfsetlinewidth{0.000000pt}%
\definecolor{currentstroke}{rgb}{0.000000,0.000000,0.000000}%
\pgfsetstrokecolor{currentstroke}%
\pgfsetdash{}{0pt}%
\pgfpathmoveto{\pgfqpoint{1.420328in}{1.531158in}}%
\pgfpathlineto{\pgfqpoint{1.416494in}{1.540144in}}%
\pgfpathlineto{\pgfqpoint{1.509295in}{1.534029in}}%
\pgfpathlineto{\pgfqpoint{1.510261in}{1.525227in}}%
\pgfpathlineto{\pgfqpoint{1.420328in}{1.531158in}}%
\pgfpathclose%
\pgfusepath{fill}%
\end{pgfscope}%
\begin{pgfscope}%
\pgfpathrectangle{\pgfqpoint{0.000000in}{0.000000in}}{\pgfqpoint{3.000000in}{3.000000in}}%
\pgfusepath{clip}%
\pgfsetbuttcap%
\pgfsetroundjoin%
\definecolor{currentfill}{rgb}{1.000000,0.407407,0.000000}%
\pgfsetfillcolor{currentfill}%
\pgfsetlinewidth{0.000000pt}%
\definecolor{currentstroke}{rgb}{0.000000,0.000000,0.000000}%
\pgfsetstrokecolor{currentstroke}%
\pgfsetdash{}{0pt}%
\pgfpathmoveto{\pgfqpoint{1.234111in}{1.583032in}}%
\pgfpathlineto{\pgfqpoint{1.224913in}{1.592197in}}%
\pgfpathlineto{\pgfqpoint{1.313803in}{1.570944in}}%
\pgfpathlineto{\pgfqpoint{1.320425in}{1.562379in}}%
\pgfpathlineto{\pgfqpoint{1.234111in}{1.583032in}}%
\pgfpathclose%
\pgfusepath{fill}%
\end{pgfscope}%
\begin{pgfscope}%
\pgfpathrectangle{\pgfqpoint{0.000000in}{0.000000in}}{\pgfqpoint{3.000000in}{3.000000in}}%
\pgfusepath{clip}%
\pgfsetbuttcap%
\pgfsetroundjoin%
\definecolor{currentfill}{rgb}{0.999109,0.073348,0.000000}%
\pgfsetfillcolor{currentfill}%
\pgfsetlinewidth{0.000000pt}%
\definecolor{currentstroke}{rgb}{0.000000,0.000000,0.000000}%
\pgfsetstrokecolor{currentstroke}%
\pgfsetdash{}{0pt}%
\pgfpathmoveto{\pgfqpoint{2.022620in}{1.676849in}}%
\pgfpathlineto{\pgfqpoint{2.034982in}{1.685247in}}%
\pgfpathlineto{\pgfqpoint{2.112230in}{1.726466in}}%
\pgfpathlineto{\pgfqpoint{2.097999in}{1.717079in}}%
\pgfpathlineto{\pgfqpoint{2.022620in}{1.676849in}}%
\pgfpathclose%
\pgfusepath{fill}%
\end{pgfscope}%
\begin{pgfscope}%
\pgfpathrectangle{\pgfqpoint{0.000000in}{0.000000in}}{\pgfqpoint{3.000000in}{3.000000in}}%
\pgfusepath{clip}%
\pgfsetbuttcap%
\pgfsetroundjoin%
\definecolor{currentfill}{rgb}{1.000000,0.291213,0.000000}%
\pgfsetfillcolor{currentfill}%
\pgfsetlinewidth{0.000000pt}%
\definecolor{currentstroke}{rgb}{0.000000,0.000000,0.000000}%
\pgfsetstrokecolor{currentstroke}%
\pgfsetdash{}{0pt}%
\pgfpathmoveto{\pgfqpoint{1.132932in}{1.629595in}}%
\pgfpathlineto{\pgfqpoint{1.121373in}{1.638841in}}%
\pgfpathlineto{\pgfqpoint{1.206453in}{1.609459in}}%
\pgfpathlineto{\pgfqpoint{1.215694in}{1.600998in}}%
\pgfpathlineto{\pgfqpoint{1.132932in}{1.629595in}}%
\pgfpathclose%
\pgfusepath{fill}%
\end{pgfscope}%
\begin{pgfscope}%
\pgfpathrectangle{\pgfqpoint{0.000000in}{0.000000in}}{\pgfqpoint{3.000000in}{3.000000in}}%
\pgfusepath{clip}%
\pgfsetbuttcap%
\pgfsetroundjoin%
\definecolor{currentfill}{rgb}{1.000000,0.538126,0.000000}%
\pgfsetfillcolor{currentfill}%
\pgfsetlinewidth{0.000000pt}%
\definecolor{currentstroke}{rgb}{0.000000,0.000000,0.000000}%
\pgfsetstrokecolor{currentstroke}%
\pgfsetdash{}{0pt}%
\pgfpathmoveto{\pgfqpoint{1.510261in}{1.525227in}}%
\pgfpathlineto{\pgfqpoint{1.509295in}{1.534029in}}%
\pgfpathlineto{\pgfqpoint{1.602938in}{1.535256in}}%
\pgfpathlineto{\pgfqpoint{1.601008in}{1.526417in}}%
\pgfpathlineto{\pgfqpoint{1.510261in}{1.525227in}}%
\pgfpathclose%
\pgfusepath{fill}%
\end{pgfscope}%
\begin{pgfscope}%
\pgfpathrectangle{\pgfqpoint{0.000000in}{0.000000in}}{\pgfqpoint{3.000000in}{3.000000in}}%
\pgfusepath{clip}%
\pgfsetbuttcap%
\pgfsetroundjoin%
\definecolor{currentfill}{rgb}{0.803030,0.000000,0.000000}%
\pgfsetfillcolor{currentfill}%
\pgfsetlinewidth{0.000000pt}%
\definecolor{currentstroke}{rgb}{0.000000,0.000000,0.000000}%
\pgfsetstrokecolor{currentstroke}%
\pgfsetdash{}{0pt}%
\pgfpathmoveto{\pgfqpoint{0.895045in}{1.776405in}}%
\pgfpathlineto{\pgfqpoint{0.879710in}{1.785833in}}%
\pgfpathlineto{\pgfqpoint{0.951896in}{1.737820in}}%
\pgfpathlineto{\pgfqpoint{0.965626in}{1.729464in}}%
\pgfpathlineto{\pgfqpoint{0.895045in}{1.776405in}}%
\pgfpathclose%
\pgfusepath{fill}%
\end{pgfscope}%
\begin{pgfscope}%
\pgfpathrectangle{\pgfqpoint{0.000000in}{0.000000in}}{\pgfqpoint{3.000000in}{3.000000in}}%
\pgfusepath{clip}%
\pgfsetbuttcap%
\pgfsetroundjoin%
\definecolor{currentfill}{rgb}{1.000000,0.233115,0.000000}%
\pgfsetfillcolor{currentfill}%
\pgfsetlinewidth{0.000000pt}%
\definecolor{currentstroke}{rgb}{0.000000,0.000000,0.000000}%
\pgfsetstrokecolor{currentstroke}%
\pgfsetdash{}{0pt}%
\pgfpathmoveto{\pgfqpoint{1.904100in}{1.618522in}}%
\pgfpathlineto{\pgfqpoint{1.914169in}{1.626909in}}%
\pgfpathlineto{\pgfqpoint{1.997971in}{1.659259in}}%
\pgfpathlineto{\pgfqpoint{1.985685in}{1.650032in}}%
\pgfpathlineto{\pgfqpoint{1.904100in}{1.618522in}}%
\pgfpathclose%
\pgfusepath{fill}%
\end{pgfscope}%
\begin{pgfscope}%
\pgfpathrectangle{\pgfqpoint{0.000000in}{0.000000in}}{\pgfqpoint{3.000000in}{3.000000in}}%
\pgfusepath{clip}%
\pgfsetbuttcap%
\pgfsetroundjoin%
\definecolor{currentfill}{rgb}{1.000000,0.116921,0.000000}%
\pgfsetfillcolor{currentfill}%
\pgfsetlinewidth{0.000000pt}%
\definecolor{currentstroke}{rgb}{0.000000,0.000000,0.000000}%
\pgfsetstrokecolor{currentstroke}%
\pgfsetdash{}{0pt}%
\pgfpathmoveto{\pgfqpoint{1.020283in}{1.693747in}}%
\pgfpathlineto{\pgfqpoint{1.006658in}{1.703048in}}%
\pgfpathlineto{\pgfqpoint{1.086549in}{1.664763in}}%
\pgfpathlineto{\pgfqpoint{1.098182in}{1.656405in}}%
\pgfpathlineto{\pgfqpoint{1.020283in}{1.693747in}}%
\pgfpathclose%
\pgfusepath{fill}%
\end{pgfscope}%
\begin{pgfscope}%
\pgfpathrectangle{\pgfqpoint{0.000000in}{0.000000in}}{\pgfqpoint{3.000000in}{3.000000in}}%
\pgfusepath{clip}%
\pgfsetbuttcap%
\pgfsetroundjoin%
\definecolor{currentfill}{rgb}{1.000000,0.480029,0.000000}%
\pgfsetfillcolor{currentfill}%
\pgfsetlinewidth{0.000000pt}%
\definecolor{currentstroke}{rgb}{0.000000,0.000000,0.000000}%
\pgfsetstrokecolor{currentstroke}%
\pgfsetdash{}{0pt}%
\pgfpathmoveto{\pgfqpoint{1.327031in}{1.553420in}}%
\pgfpathlineto{\pgfqpoint{1.320425in}{1.562379in}}%
\pgfpathlineto{\pgfqpoint{1.412651in}{1.548704in}}%
\pgfpathlineto{\pgfqpoint{1.416494in}{1.540144in}}%
\pgfpathlineto{\pgfqpoint{1.327031in}{1.553420in}}%
\pgfpathclose%
\pgfusepath{fill}%
\end{pgfscope}%
\begin{pgfscope}%
\pgfpathrectangle{\pgfqpoint{0.000000in}{0.000000in}}{\pgfqpoint{3.000000in}{3.000000in}}%
\pgfusepath{clip}%
\pgfsetbuttcap%
\pgfsetroundjoin%
\definecolor{currentfill}{rgb}{0.678253,0.000000,0.000000}%
\pgfsetfillcolor{currentfill}%
\pgfsetlinewidth{0.000000pt}%
\definecolor{currentstroke}{rgb}{0.000000,0.000000,0.000000}%
\pgfsetstrokecolor{currentstroke}%
\pgfsetdash{}{0pt}%
\pgfpathmoveto{\pgfqpoint{2.169419in}{1.761727in}}%
\pgfpathlineto{\pgfqpoint{2.183782in}{1.770033in}}%
\pgfpathlineto{\pgfqpoint{2.253273in}{1.821943in}}%
\pgfpathlineto{\pgfqpoint{2.237447in}{1.812531in}}%
\pgfpathlineto{\pgfqpoint{2.169419in}{1.761727in}}%
\pgfpathclose%
\pgfusepath{fill}%
\end{pgfscope}%
\begin{pgfscope}%
\pgfpathrectangle{\pgfqpoint{0.000000in}{0.000000in}}{\pgfqpoint{3.000000in}{3.000000in}}%
\pgfusepath{clip}%
\pgfsetbuttcap%
\pgfsetroundjoin%
\definecolor{currentfill}{rgb}{1.000000,0.349310,0.000000}%
\pgfsetfillcolor{currentfill}%
\pgfsetlinewidth{0.000000pt}%
\definecolor{currentstroke}{rgb}{0.000000,0.000000,0.000000}%
\pgfsetstrokecolor{currentstroke}%
\pgfsetdash{}{0pt}%
\pgfpathmoveto{\pgfqpoint{1.797707in}{1.577257in}}%
\pgfpathlineto{\pgfqpoint{1.805231in}{1.585635in}}%
\pgfpathlineto{\pgfqpoint{1.894054in}{1.609818in}}%
\pgfpathlineto{\pgfqpoint{1.884030in}{1.600775in}}%
\pgfpathlineto{\pgfqpoint{1.797707in}{1.577257in}}%
\pgfpathclose%
\pgfusepath{fill}%
\end{pgfscope}%
\begin{pgfscope}%
\pgfpathrectangle{\pgfqpoint{0.000000in}{0.000000in}}{\pgfqpoint{3.000000in}{3.000000in}}%
\pgfusepath{clip}%
\pgfsetbuttcap%
\pgfsetroundjoin%
\definecolor{currentfill}{rgb}{1.000000,0.480029,0.000000}%
\pgfsetfillcolor{currentfill}%
\pgfsetlinewidth{0.000000pt}%
\definecolor{currentstroke}{rgb}{0.000000,0.000000,0.000000}%
\pgfsetstrokecolor{currentstroke}%
\pgfsetdash{}{0pt}%
\pgfpathmoveto{\pgfqpoint{1.602938in}{1.535256in}}%
\pgfpathlineto{\pgfqpoint{1.604872in}{1.543670in}}%
\pgfpathlineto{\pgfqpoint{1.699680in}{1.552458in}}%
\pgfpathlineto{\pgfqpoint{1.694899in}{1.543788in}}%
\pgfpathlineto{\pgfqpoint{1.602938in}{1.535256in}}%
\pgfpathclose%
\pgfusepath{fill}%
\end{pgfscope}%
\begin{pgfscope}%
\pgfpathrectangle{\pgfqpoint{0.000000in}{0.000000in}}{\pgfqpoint{3.000000in}{3.000000in}}%
\pgfusepath{clip}%
\pgfsetbuttcap%
\pgfsetroundjoin%
\definecolor{currentfill}{rgb}{0.927807,0.015251,0.000000}%
\pgfsetfillcolor{currentfill}%
\pgfsetlinewidth{0.000000pt}%
\definecolor{currentstroke}{rgb}{0.000000,0.000000,0.000000}%
\pgfsetstrokecolor{currentstroke}%
\pgfsetdash{}{0pt}%
\pgfpathmoveto{\pgfqpoint{2.034982in}{1.685247in}}%
\pgfpathlineto{\pgfqpoint{2.047370in}{1.693401in}}%
\pgfpathlineto{\pgfqpoint{2.126488in}{1.735611in}}%
\pgfpathlineto{\pgfqpoint{2.112230in}{1.726466in}}%
\pgfpathlineto{\pgfqpoint{2.034982in}{1.685247in}}%
\pgfpathclose%
\pgfusepath{fill}%
\end{pgfscope}%
\begin{pgfscope}%
\pgfpathrectangle{\pgfqpoint{0.000000in}{0.000000in}}{\pgfqpoint{3.000000in}{3.000000in}}%
\pgfusepath{clip}%
\pgfsetbuttcap%
\pgfsetroundjoin%
\definecolor{currentfill}{rgb}{0.731729,0.000000,0.000000}%
\pgfsetfillcolor{currentfill}%
\pgfsetlinewidth{0.000000pt}%
\definecolor{currentstroke}{rgb}{0.000000,0.000000,0.000000}%
\pgfsetstrokecolor{currentstroke}%
\pgfsetdash{}{0pt}%
\pgfpathmoveto{\pgfqpoint{0.879710in}{1.785833in}}%
\pgfpathlineto{\pgfqpoint{0.864350in}{1.795058in}}%
\pgfpathlineto{\pgfqpoint{0.938140in}{1.745972in}}%
\pgfpathlineto{\pgfqpoint{0.951896in}{1.737820in}}%
\pgfpathlineto{\pgfqpoint{0.879710in}{1.785833in}}%
\pgfpathclose%
\pgfusepath{fill}%
\end{pgfscope}%
\begin{pgfscope}%
\pgfpathrectangle{\pgfqpoint{0.000000in}{0.000000in}}{\pgfqpoint{3.000000in}{3.000000in}}%
\pgfusepath{clip}%
\pgfsetbuttcap%
\pgfsetroundjoin%
\definecolor{currentfill}{rgb}{1.000000,0.407407,0.000000}%
\pgfsetfillcolor{currentfill}%
\pgfsetlinewidth{0.000000pt}%
\definecolor{currentstroke}{rgb}{0.000000,0.000000,0.000000}%
\pgfsetstrokecolor{currentstroke}%
\pgfsetdash{}{0pt}%
\pgfpathmoveto{\pgfqpoint{1.699680in}{1.552458in}}%
\pgfpathlineto{\pgfqpoint{1.704473in}{1.560734in}}%
\pgfpathlineto{\pgfqpoint{1.797707in}{1.577257in}}%
\pgfpathlineto{\pgfqpoint{1.790201in}{1.568513in}}%
\pgfpathlineto{\pgfqpoint{1.699680in}{1.552458in}}%
\pgfpathclose%
\pgfusepath{fill}%
\end{pgfscope}%
\begin{pgfscope}%
\pgfpathrectangle{\pgfqpoint{0.000000in}{0.000000in}}{\pgfqpoint{3.000000in}{3.000000in}}%
\pgfusepath{clip}%
\pgfsetbuttcap%
\pgfsetroundjoin%
\definecolor{currentfill}{rgb}{1.000000,0.480029,0.000000}%
\pgfsetfillcolor{currentfill}%
\pgfsetlinewidth{0.000000pt}%
\definecolor{currentstroke}{rgb}{0.000000,0.000000,0.000000}%
\pgfsetstrokecolor{currentstroke}%
\pgfsetdash{}{0pt}%
\pgfpathmoveto{\pgfqpoint{1.416494in}{1.540144in}}%
\pgfpathlineto{\pgfqpoint{1.412651in}{1.548704in}}%
\pgfpathlineto{\pgfqpoint{1.508326in}{1.542406in}}%
\pgfpathlineto{\pgfqpoint{1.509295in}{1.534029in}}%
\pgfpathlineto{\pgfqpoint{1.416494in}{1.540144in}}%
\pgfpathclose%
\pgfusepath{fill}%
\end{pgfscope}%
\begin{pgfscope}%
\pgfpathrectangle{\pgfqpoint{0.000000in}{0.000000in}}{\pgfqpoint{3.000000in}{3.000000in}}%
\pgfusepath{clip}%
\pgfsetbuttcap%
\pgfsetroundjoin%
\definecolor{currentfill}{rgb}{1.000000,0.349310,0.000000}%
\pgfsetfillcolor{currentfill}%
\pgfsetlinewidth{0.000000pt}%
\definecolor{currentstroke}{rgb}{0.000000,0.000000,0.000000}%
\pgfsetstrokecolor{currentstroke}%
\pgfsetdash{}{0pt}%
\pgfpathmoveto{\pgfqpoint{1.224913in}{1.592197in}}%
\pgfpathlineto{\pgfqpoint{1.215694in}{1.600998in}}%
\pgfpathlineto{\pgfqpoint{1.307164in}{1.579143in}}%
\pgfpathlineto{\pgfqpoint{1.313803in}{1.570944in}}%
\pgfpathlineto{\pgfqpoint{1.224913in}{1.592197in}}%
\pgfpathclose%
\pgfusepath{fill}%
\end{pgfscope}%
\begin{pgfscope}%
\pgfpathrectangle{\pgfqpoint{0.000000in}{0.000000in}}{\pgfqpoint{3.000000in}{3.000000in}}%
\pgfusepath{clip}%
\pgfsetbuttcap%
\pgfsetroundjoin%
\definecolor{currentfill}{rgb}{1.000000,0.233115,0.000000}%
\pgfsetfillcolor{currentfill}%
\pgfsetlinewidth{0.000000pt}%
\definecolor{currentstroke}{rgb}{0.000000,0.000000,0.000000}%
\pgfsetstrokecolor{currentstroke}%
\pgfsetdash{}{0pt}%
\pgfpathmoveto{\pgfqpoint{1.121373in}{1.638841in}}%
\pgfpathlineto{\pgfqpoint{1.109790in}{1.647771in}}%
\pgfpathlineto{\pgfqpoint{1.197190in}{1.617603in}}%
\pgfpathlineto{\pgfqpoint{1.206453in}{1.609459in}}%
\pgfpathlineto{\pgfqpoint{1.121373in}{1.638841in}}%
\pgfpathclose%
\pgfusepath{fill}%
\end{pgfscope}%
\begin{pgfscope}%
\pgfpathrectangle{\pgfqpoint{0.000000in}{0.000000in}}{\pgfqpoint{3.000000in}{3.000000in}}%
\pgfusepath{clip}%
\pgfsetbuttcap%
\pgfsetroundjoin%
\definecolor{currentfill}{rgb}{0.999109,0.073348,0.000000}%
\pgfsetfillcolor{currentfill}%
\pgfsetlinewidth{0.000000pt}%
\definecolor{currentstroke}{rgb}{0.000000,0.000000,0.000000}%
\pgfsetstrokecolor{currentstroke}%
\pgfsetdash{}{0pt}%
\pgfpathmoveto{\pgfqpoint{1.006658in}{1.703048in}}%
\pgfpathlineto{\pgfqpoint{0.993007in}{1.712091in}}%
\pgfpathlineto{\pgfqpoint{1.074891in}{1.672862in}}%
\pgfpathlineto{\pgfqpoint{1.086549in}{1.664763in}}%
\pgfpathlineto{\pgfqpoint{1.006658in}{1.703048in}}%
\pgfpathclose%
\pgfusepath{fill}%
\end{pgfscope}%
\begin{pgfscope}%
\pgfpathrectangle{\pgfqpoint{0.000000in}{0.000000in}}{\pgfqpoint{3.000000in}{3.000000in}}%
\pgfusepath{clip}%
\pgfsetbuttcap%
\pgfsetroundjoin%
\definecolor{currentfill}{rgb}{0.606952,0.000000,0.000000}%
\pgfsetfillcolor{currentfill}%
\pgfsetlinewidth{0.000000pt}%
\definecolor{currentstroke}{rgb}{0.000000,0.000000,0.000000}%
\pgfsetstrokecolor{currentstroke}%
\pgfsetdash{}{0pt}%
\pgfpathmoveto{\pgfqpoint{2.183782in}{1.770033in}}%
\pgfpathlineto{\pgfqpoint{2.198171in}{1.778157in}}%
\pgfpathlineto{\pgfqpoint{2.269124in}{1.831175in}}%
\pgfpathlineto{\pgfqpoint{2.253273in}{1.821943in}}%
\pgfpathlineto{\pgfqpoint{2.183782in}{1.770033in}}%
\pgfpathclose%
\pgfusepath{fill}%
\end{pgfscope}%
\begin{pgfscope}%
\pgfpathrectangle{\pgfqpoint{0.000000in}{0.000000in}}{\pgfqpoint{3.000000in}{3.000000in}}%
\pgfusepath{clip}%
\pgfsetbuttcap%
\pgfsetroundjoin%
\definecolor{currentfill}{rgb}{1.000000,0.175018,0.000000}%
\pgfsetfillcolor{currentfill}%
\pgfsetlinewidth{0.000000pt}%
\definecolor{currentstroke}{rgb}{0.000000,0.000000,0.000000}%
\pgfsetstrokecolor{currentstroke}%
\pgfsetdash{}{0pt}%
\pgfpathmoveto{\pgfqpoint{1.914169in}{1.626909in}}%
\pgfpathlineto{\pgfqpoint{1.924262in}{1.635000in}}%
\pgfpathlineto{\pgfqpoint{2.010283in}{1.668192in}}%
\pgfpathlineto{\pgfqpoint{1.997971in}{1.659259in}}%
\pgfpathlineto{\pgfqpoint{1.914169in}{1.626909in}}%
\pgfpathclose%
\pgfusepath{fill}%
\end{pgfscope}%
\begin{pgfscope}%
\pgfpathrectangle{\pgfqpoint{0.000000in}{0.000000in}}{\pgfqpoint{3.000000in}{3.000000in}}%
\pgfusepath{clip}%
\pgfsetbuttcap%
\pgfsetroundjoin%
\definecolor{currentfill}{rgb}{1.000000,0.480029,0.000000}%
\pgfsetfillcolor{currentfill}%
\pgfsetlinewidth{0.000000pt}%
\definecolor{currentstroke}{rgb}{0.000000,0.000000,0.000000}%
\pgfsetstrokecolor{currentstroke}%
\pgfsetdash{}{0pt}%
\pgfpathmoveto{\pgfqpoint{1.509295in}{1.534029in}}%
\pgfpathlineto{\pgfqpoint{1.508326in}{1.542406in}}%
\pgfpathlineto{\pgfqpoint{1.604872in}{1.543670in}}%
\pgfpathlineto{\pgfqpoint{1.602938in}{1.535256in}}%
\pgfpathlineto{\pgfqpoint{1.509295in}{1.534029in}}%
\pgfpathclose%
\pgfusepath{fill}%
\end{pgfscope}%
\begin{pgfscope}%
\pgfpathrectangle{\pgfqpoint{0.000000in}{0.000000in}}{\pgfqpoint{3.000000in}{3.000000in}}%
\pgfusepath{clip}%
\pgfsetbuttcap%
\pgfsetroundjoin%
\definecolor{currentfill}{rgb}{1.000000,0.407407,0.000000}%
\pgfsetfillcolor{currentfill}%
\pgfsetlinewidth{0.000000pt}%
\definecolor{currentstroke}{rgb}{0.000000,0.000000,0.000000}%
\pgfsetstrokecolor{currentstroke}%
\pgfsetdash{}{0pt}%
\pgfpathmoveto{\pgfqpoint{1.320425in}{1.562379in}}%
\pgfpathlineto{\pgfqpoint{1.313803in}{1.570944in}}%
\pgfpathlineto{\pgfqpoint{1.408797in}{1.556870in}}%
\pgfpathlineto{\pgfqpoint{1.412651in}{1.548704in}}%
\pgfpathlineto{\pgfqpoint{1.320425in}{1.562379in}}%
\pgfpathclose%
\pgfusepath{fill}%
\end{pgfscope}%
\begin{pgfscope}%
\pgfpathrectangle{\pgfqpoint{0.000000in}{0.000000in}}{\pgfqpoint{3.000000in}{3.000000in}}%
\pgfusepath{clip}%
\pgfsetbuttcap%
\pgfsetroundjoin%
\definecolor{currentfill}{rgb}{1.000000,0.291213,0.000000}%
\pgfsetfillcolor{currentfill}%
\pgfsetlinewidth{0.000000pt}%
\definecolor{currentstroke}{rgb}{0.000000,0.000000,0.000000}%
\pgfsetstrokecolor{currentstroke}%
\pgfsetdash{}{0pt}%
\pgfpathmoveto{\pgfqpoint{1.805231in}{1.585635in}}%
\pgfpathlineto{\pgfqpoint{1.812774in}{1.593673in}}%
\pgfpathlineto{\pgfqpoint{1.904100in}{1.618522in}}%
\pgfpathlineto{\pgfqpoint{1.894054in}{1.609818in}}%
\pgfpathlineto{\pgfqpoint{1.805231in}{1.585635in}}%
\pgfpathclose%
\pgfusepath{fill}%
\end{pgfscope}%
\begin{pgfscope}%
\pgfpathrectangle{\pgfqpoint{0.000000in}{0.000000in}}{\pgfqpoint{3.000000in}{3.000000in}}%
\pgfusepath{clip}%
\pgfsetbuttcap%
\pgfsetroundjoin%
\definecolor{currentfill}{rgb}{0.678253,0.000000,0.000000}%
\pgfsetfillcolor{currentfill}%
\pgfsetlinewidth{0.000000pt}%
\definecolor{currentstroke}{rgb}{0.000000,0.000000,0.000000}%
\pgfsetstrokecolor{currentstroke}%
\pgfsetdash{}{0pt}%
\pgfpathmoveto{\pgfqpoint{0.864350in}{1.795058in}}%
\pgfpathlineto{\pgfqpoint{0.848964in}{1.804091in}}%
\pgfpathlineto{\pgfqpoint{0.924357in}{1.753933in}}%
\pgfpathlineto{\pgfqpoint{0.938140in}{1.745972in}}%
\pgfpathlineto{\pgfqpoint{0.864350in}{1.795058in}}%
\pgfpathclose%
\pgfusepath{fill}%
\end{pgfscope}%
\begin{pgfscope}%
\pgfpathrectangle{\pgfqpoint{0.000000in}{0.000000in}}{\pgfqpoint{3.000000in}{3.000000in}}%
\pgfusepath{clip}%
\pgfsetbuttcap%
\pgfsetroundjoin%
\definecolor{currentfill}{rgb}{0.856506,0.000000,0.000000}%
\pgfsetfillcolor{currentfill}%
\pgfsetlinewidth{0.000000pt}%
\definecolor{currentstroke}{rgb}{0.000000,0.000000,0.000000}%
\pgfsetstrokecolor{currentstroke}%
\pgfsetdash{}{0pt}%
\pgfpathmoveto{\pgfqpoint{2.047370in}{1.693401in}}%
\pgfpathlineto{\pgfqpoint{2.059784in}{1.701326in}}%
\pgfpathlineto{\pgfqpoint{2.140772in}{1.744527in}}%
\pgfpathlineto{\pgfqpoint{2.126488in}{1.735611in}}%
\pgfpathlineto{\pgfqpoint{2.047370in}{1.693401in}}%
\pgfpathclose%
\pgfusepath{fill}%
\end{pgfscope}%
\begin{pgfscope}%
\pgfpathrectangle{\pgfqpoint{0.000000in}{0.000000in}}{\pgfqpoint{3.000000in}{3.000000in}}%
\pgfusepath{clip}%
\pgfsetbuttcap%
\pgfsetroundjoin%
\definecolor{currentfill}{rgb}{0.553476,0.000000,0.000000}%
\pgfsetfillcolor{currentfill}%
\pgfsetlinewidth{0.000000pt}%
\definecolor{currentstroke}{rgb}{0.000000,0.000000,0.000000}%
\pgfsetstrokecolor{currentstroke}%
\pgfsetdash{}{0pt}%
\pgfpathmoveto{\pgfqpoint{2.198171in}{1.778157in}}%
\pgfpathlineto{\pgfqpoint{2.212587in}{1.786109in}}%
\pgfpathlineto{\pgfqpoint{2.285000in}{1.840234in}}%
\pgfpathlineto{\pgfqpoint{2.269124in}{1.831175in}}%
\pgfpathlineto{\pgfqpoint{2.198171in}{1.778157in}}%
\pgfpathclose%
\pgfusepath{fill}%
\end{pgfscope}%
\begin{pgfscope}%
\pgfpathrectangle{\pgfqpoint{0.000000in}{0.000000in}}{\pgfqpoint{3.000000in}{3.000000in}}%
\pgfusepath{clip}%
\pgfsetbuttcap%
\pgfsetroundjoin%
\definecolor{currentfill}{rgb}{1.000000,0.407407,0.000000}%
\pgfsetfillcolor{currentfill}%
\pgfsetlinewidth{0.000000pt}%
\definecolor{currentstroke}{rgb}{0.000000,0.000000,0.000000}%
\pgfsetstrokecolor{currentstroke}%
\pgfsetdash{}{0pt}%
\pgfpathmoveto{\pgfqpoint{1.604872in}{1.543670in}}%
\pgfpathlineto{\pgfqpoint{1.606812in}{1.551688in}}%
\pgfpathlineto{\pgfqpoint{1.704473in}{1.560734in}}%
\pgfpathlineto{\pgfqpoint{1.699680in}{1.552458in}}%
\pgfpathlineto{\pgfqpoint{1.604872in}{1.543670in}}%
\pgfpathclose%
\pgfusepath{fill}%
\end{pgfscope}%
\begin{pgfscope}%
\pgfpathrectangle{\pgfqpoint{0.000000in}{0.000000in}}{\pgfqpoint{3.000000in}{3.000000in}}%
\pgfusepath{clip}%
\pgfsetbuttcap%
\pgfsetroundjoin%
\definecolor{currentfill}{rgb}{0.927807,0.015251,0.000000}%
\pgfsetfillcolor{currentfill}%
\pgfsetlinewidth{0.000000pt}%
\definecolor{currentstroke}{rgb}{0.000000,0.000000,0.000000}%
\pgfsetstrokecolor{currentstroke}%
\pgfsetdash{}{0pt}%
\pgfpathmoveto{\pgfqpoint{0.993007in}{1.712091in}}%
\pgfpathlineto{\pgfqpoint{0.979329in}{1.720892in}}%
\pgfpathlineto{\pgfqpoint{1.063208in}{1.680717in}}%
\pgfpathlineto{\pgfqpoint{1.074891in}{1.672862in}}%
\pgfpathlineto{\pgfqpoint{0.993007in}{1.712091in}}%
\pgfpathclose%
\pgfusepath{fill}%
\end{pgfscope}%
\begin{pgfscope}%
\pgfpathrectangle{\pgfqpoint{0.000000in}{0.000000in}}{\pgfqpoint{3.000000in}{3.000000in}}%
\pgfusepath{clip}%
\pgfsetbuttcap%
\pgfsetroundjoin%
\definecolor{currentfill}{rgb}{1.000000,0.175018,0.000000}%
\pgfsetfillcolor{currentfill}%
\pgfsetlinewidth{0.000000pt}%
\definecolor{currentstroke}{rgb}{0.000000,0.000000,0.000000}%
\pgfsetstrokecolor{currentstroke}%
\pgfsetdash{}{0pt}%
\pgfpathmoveto{\pgfqpoint{1.109790in}{1.647771in}}%
\pgfpathlineto{\pgfqpoint{1.098182in}{1.656405in}}%
\pgfpathlineto{\pgfqpoint{1.187907in}{1.625451in}}%
\pgfpathlineto{\pgfqpoint{1.197190in}{1.617603in}}%
\pgfpathlineto{\pgfqpoint{1.109790in}{1.647771in}}%
\pgfpathclose%
\pgfusepath{fill}%
\end{pgfscope}%
\begin{pgfscope}%
\pgfpathrectangle{\pgfqpoint{0.000000in}{0.000000in}}{\pgfqpoint{3.000000in}{3.000000in}}%
\pgfusepath{clip}%
\pgfsetbuttcap%
\pgfsetroundjoin%
\definecolor{currentfill}{rgb}{1.000000,0.349310,0.000000}%
\pgfsetfillcolor{currentfill}%
\pgfsetlinewidth{0.000000pt}%
\definecolor{currentstroke}{rgb}{0.000000,0.000000,0.000000}%
\pgfsetstrokecolor{currentstroke}%
\pgfsetdash{}{0pt}%
\pgfpathmoveto{\pgfqpoint{1.704473in}{1.560734in}}%
\pgfpathlineto{\pgfqpoint{1.709279in}{1.568643in}}%
\pgfpathlineto{\pgfqpoint{1.805231in}{1.585635in}}%
\pgfpathlineto{\pgfqpoint{1.797707in}{1.577257in}}%
\pgfpathlineto{\pgfqpoint{1.704473in}{1.560734in}}%
\pgfpathclose%
\pgfusepath{fill}%
\end{pgfscope}%
\begin{pgfscope}%
\pgfpathrectangle{\pgfqpoint{0.000000in}{0.000000in}}{\pgfqpoint{3.000000in}{3.000000in}}%
\pgfusepath{clip}%
\pgfsetbuttcap%
\pgfsetroundjoin%
\definecolor{currentfill}{rgb}{1.000000,0.291213,0.000000}%
\pgfsetfillcolor{currentfill}%
\pgfsetlinewidth{0.000000pt}%
\definecolor{currentstroke}{rgb}{0.000000,0.000000,0.000000}%
\pgfsetstrokecolor{currentstroke}%
\pgfsetdash{}{0pt}%
\pgfpathmoveto{\pgfqpoint{1.215694in}{1.600998in}}%
\pgfpathlineto{\pgfqpoint{1.206453in}{1.609459in}}%
\pgfpathlineto{\pgfqpoint{1.300509in}{1.587002in}}%
\pgfpathlineto{\pgfqpoint{1.307164in}{1.579143in}}%
\pgfpathlineto{\pgfqpoint{1.215694in}{1.600998in}}%
\pgfpathclose%
\pgfusepath{fill}%
\end{pgfscope}%
\begin{pgfscope}%
\pgfpathrectangle{\pgfqpoint{0.000000in}{0.000000in}}{\pgfqpoint{3.000000in}{3.000000in}}%
\pgfusepath{clip}%
\pgfsetbuttcap%
\pgfsetroundjoin%
\definecolor{currentfill}{rgb}{1.000000,0.116921,0.000000}%
\pgfsetfillcolor{currentfill}%
\pgfsetlinewidth{0.000000pt}%
\definecolor{currentstroke}{rgb}{0.000000,0.000000,0.000000}%
\pgfsetstrokecolor{currentstroke}%
\pgfsetdash{}{0pt}%
\pgfpathmoveto{\pgfqpoint{1.924262in}{1.635000in}}%
\pgfpathlineto{\pgfqpoint{1.934377in}{1.642814in}}%
\pgfpathlineto{\pgfqpoint{2.022620in}{1.676849in}}%
\pgfpathlineto{\pgfqpoint{2.010283in}{1.668192in}}%
\pgfpathlineto{\pgfqpoint{1.924262in}{1.635000in}}%
\pgfpathclose%
\pgfusepath{fill}%
\end{pgfscope}%
\begin{pgfscope}%
\pgfpathrectangle{\pgfqpoint{0.000000in}{0.000000in}}{\pgfqpoint{3.000000in}{3.000000in}}%
\pgfusepath{clip}%
\pgfsetbuttcap%
\pgfsetroundjoin%
\definecolor{currentfill}{rgb}{1.000000,0.407407,0.000000}%
\pgfsetfillcolor{currentfill}%
\pgfsetlinewidth{0.000000pt}%
\definecolor{currentstroke}{rgb}{0.000000,0.000000,0.000000}%
\pgfsetstrokecolor{currentstroke}%
\pgfsetdash{}{0pt}%
\pgfpathmoveto{\pgfqpoint{1.412651in}{1.548704in}}%
\pgfpathlineto{\pgfqpoint{1.408797in}{1.556870in}}%
\pgfpathlineto{\pgfqpoint{1.507355in}{1.550387in}}%
\pgfpathlineto{\pgfqpoint{1.508326in}{1.542406in}}%
\pgfpathlineto{\pgfqpoint{1.412651in}{1.548704in}}%
\pgfpathclose%
\pgfusepath{fill}%
\end{pgfscope}%
\begin{pgfscope}%
\pgfpathrectangle{\pgfqpoint{0.000000in}{0.000000in}}{\pgfqpoint{3.000000in}{3.000000in}}%
\pgfusepath{clip}%
\pgfsetbuttcap%
\pgfsetroundjoin%
\definecolor{currentfill}{rgb}{1.000000,0.407407,0.000000}%
\pgfsetfillcolor{currentfill}%
\pgfsetlinewidth{0.000000pt}%
\definecolor{currentstroke}{rgb}{0.000000,0.000000,0.000000}%
\pgfsetstrokecolor{currentstroke}%
\pgfsetdash{}{0pt}%
\pgfpathmoveto{\pgfqpoint{1.508326in}{1.542406in}}%
\pgfpathlineto{\pgfqpoint{1.507355in}{1.550387in}}%
\pgfpathlineto{\pgfqpoint{1.606812in}{1.551688in}}%
\pgfpathlineto{\pgfqpoint{1.604872in}{1.543670in}}%
\pgfpathlineto{\pgfqpoint{1.508326in}{1.542406in}}%
\pgfpathclose%
\pgfusepath{fill}%
\end{pgfscope}%
\begin{pgfscope}%
\pgfpathrectangle{\pgfqpoint{0.000000in}{0.000000in}}{\pgfqpoint{3.000000in}{3.000000in}}%
\pgfusepath{clip}%
\pgfsetbuttcap%
\pgfsetroundjoin%
\definecolor{currentfill}{rgb}{1.000000,0.349310,0.000000}%
\pgfsetfillcolor{currentfill}%
\pgfsetlinewidth{0.000000pt}%
\definecolor{currentstroke}{rgb}{0.000000,0.000000,0.000000}%
\pgfsetstrokecolor{currentstroke}%
\pgfsetdash{}{0pt}%
\pgfpathmoveto{\pgfqpoint{1.313803in}{1.570944in}}%
\pgfpathlineto{\pgfqpoint{1.307164in}{1.579143in}}%
\pgfpathlineto{\pgfqpoint{1.404934in}{1.564669in}}%
\pgfpathlineto{\pgfqpoint{1.408797in}{1.556870in}}%
\pgfpathlineto{\pgfqpoint{1.313803in}{1.570944in}}%
\pgfpathclose%
\pgfusepath{fill}%
\end{pgfscope}%
\begin{pgfscope}%
\pgfpathrectangle{\pgfqpoint{0.000000in}{0.000000in}}{\pgfqpoint{3.000000in}{3.000000in}}%
\pgfusepath{clip}%
\pgfsetbuttcap%
\pgfsetroundjoin%
\definecolor{currentfill}{rgb}{1.000000,0.233115,0.000000}%
\pgfsetfillcolor{currentfill}%
\pgfsetlinewidth{0.000000pt}%
\definecolor{currentstroke}{rgb}{0.000000,0.000000,0.000000}%
\pgfsetstrokecolor{currentstroke}%
\pgfsetdash{}{0pt}%
\pgfpathmoveto{\pgfqpoint{1.812774in}{1.593673in}}%
\pgfpathlineto{\pgfqpoint{1.820334in}{1.601393in}}%
\pgfpathlineto{\pgfqpoint{1.914169in}{1.626909in}}%
\pgfpathlineto{\pgfqpoint{1.904100in}{1.618522in}}%
\pgfpathlineto{\pgfqpoint{1.812774in}{1.593673in}}%
\pgfpathclose%
\pgfusepath{fill}%
\end{pgfscope}%
\begin{pgfscope}%
\pgfpathrectangle{\pgfqpoint{0.000000in}{0.000000in}}{\pgfqpoint{3.000000in}{3.000000in}}%
\pgfusepath{clip}%
\pgfsetbuttcap%
\pgfsetroundjoin%
\definecolor{currentfill}{rgb}{0.606952,0.000000,0.000000}%
\pgfsetfillcolor{currentfill}%
\pgfsetlinewidth{0.000000pt}%
\definecolor{currentstroke}{rgb}{0.000000,0.000000,0.000000}%
\pgfsetstrokecolor{currentstroke}%
\pgfsetdash{}{0pt}%
\pgfpathmoveto{\pgfqpoint{0.848964in}{1.804091in}}%
\pgfpathlineto{\pgfqpoint{0.833552in}{1.812944in}}%
\pgfpathlineto{\pgfqpoint{0.910549in}{1.761711in}}%
\pgfpathlineto{\pgfqpoint{0.924357in}{1.753933in}}%
\pgfpathlineto{\pgfqpoint{0.848964in}{1.804091in}}%
\pgfpathclose%
\pgfusepath{fill}%
\end{pgfscope}%
\begin{pgfscope}%
\pgfpathrectangle{\pgfqpoint{0.000000in}{0.000000in}}{\pgfqpoint{3.000000in}{3.000000in}}%
\pgfusepath{clip}%
\pgfsetbuttcap%
\pgfsetroundjoin%
\definecolor{currentfill}{rgb}{0.500000,0.000000,0.000000}%
\pgfsetfillcolor{currentfill}%
\pgfsetlinewidth{0.000000pt}%
\definecolor{currentstroke}{rgb}{0.000000,0.000000,0.000000}%
\pgfsetstrokecolor{currentstroke}%
\pgfsetdash{}{0pt}%
\pgfpathmoveto{\pgfqpoint{2.212587in}{1.786109in}}%
\pgfpathlineto{\pgfqpoint{2.227029in}{1.793897in}}%
\pgfpathlineto{\pgfqpoint{2.300901in}{1.849131in}}%
\pgfpathlineto{\pgfqpoint{2.285000in}{1.840234in}}%
\pgfpathlineto{\pgfqpoint{2.212587in}{1.786109in}}%
\pgfpathclose%
\pgfusepath{fill}%
\end{pgfscope}%
\begin{pgfscope}%
\pgfpathrectangle{\pgfqpoint{0.000000in}{0.000000in}}{\pgfqpoint{3.000000in}{3.000000in}}%
\pgfusepath{clip}%
\pgfsetbuttcap%
\pgfsetroundjoin%
\definecolor{currentfill}{rgb}{0.803030,0.000000,0.000000}%
\pgfsetfillcolor{currentfill}%
\pgfsetlinewidth{0.000000pt}%
\definecolor{currentstroke}{rgb}{0.000000,0.000000,0.000000}%
\pgfsetstrokecolor{currentstroke}%
\pgfsetdash{}{0pt}%
\pgfpathmoveto{\pgfqpoint{2.059784in}{1.701326in}}%
\pgfpathlineto{\pgfqpoint{2.072223in}{1.709034in}}%
\pgfpathlineto{\pgfqpoint{2.155083in}{1.753228in}}%
\pgfpathlineto{\pgfqpoint{2.140772in}{1.744527in}}%
\pgfpathlineto{\pgfqpoint{2.059784in}{1.701326in}}%
\pgfpathclose%
\pgfusepath{fill}%
\end{pgfscope}%
\begin{pgfscope}%
\pgfpathrectangle{\pgfqpoint{0.000000in}{0.000000in}}{\pgfqpoint{3.000000in}{3.000000in}}%
\pgfusepath{clip}%
\pgfsetbuttcap%
\pgfsetroundjoin%
\definecolor{currentfill}{rgb}{0.856506,0.000000,0.000000}%
\pgfsetfillcolor{currentfill}%
\pgfsetlinewidth{0.000000pt}%
\definecolor{currentstroke}{rgb}{0.000000,0.000000,0.000000}%
\pgfsetstrokecolor{currentstroke}%
\pgfsetdash{}{0pt}%
\pgfpathmoveto{\pgfqpoint{0.979329in}{1.720892in}}%
\pgfpathlineto{\pgfqpoint{0.965626in}{1.729464in}}%
\pgfpathlineto{\pgfqpoint{1.051501in}{1.688342in}}%
\pgfpathlineto{\pgfqpoint{1.063208in}{1.680717in}}%
\pgfpathlineto{\pgfqpoint{0.979329in}{1.720892in}}%
\pgfpathclose%
\pgfusepath{fill}%
\end{pgfscope}%
\begin{pgfscope}%
\pgfpathrectangle{\pgfqpoint{0.000000in}{0.000000in}}{\pgfqpoint{3.000000in}{3.000000in}}%
\pgfusepath{clip}%
\pgfsetbuttcap%
\pgfsetroundjoin%
\definecolor{currentfill}{rgb}{1.000000,0.116921,0.000000}%
\pgfsetfillcolor{currentfill}%
\pgfsetlinewidth{0.000000pt}%
\definecolor{currentstroke}{rgb}{0.000000,0.000000,0.000000}%
\pgfsetstrokecolor{currentstroke}%
\pgfsetdash{}{0pt}%
\pgfpathmoveto{\pgfqpoint{1.098182in}{1.656405in}}%
\pgfpathlineto{\pgfqpoint{1.086549in}{1.664763in}}%
\pgfpathlineto{\pgfqpoint{1.178601in}{1.633021in}}%
\pgfpathlineto{\pgfqpoint{1.187907in}{1.625451in}}%
\pgfpathlineto{\pgfqpoint{1.098182in}{1.656405in}}%
\pgfpathclose%
\pgfusepath{fill}%
\end{pgfscope}%
\begin{pgfscope}%
\pgfpathrectangle{\pgfqpoint{0.000000in}{0.000000in}}{\pgfqpoint{3.000000in}{3.000000in}}%
\pgfusepath{clip}%
\pgfsetbuttcap%
\pgfsetroundjoin%
\definecolor{currentfill}{rgb}{0.999109,0.073348,0.000000}%
\pgfsetfillcolor{currentfill}%
\pgfsetlinewidth{0.000000pt}%
\definecolor{currentstroke}{rgb}{0.000000,0.000000,0.000000}%
\pgfsetstrokecolor{currentstroke}%
\pgfsetdash{}{0pt}%
\pgfpathmoveto{\pgfqpoint{1.934377in}{1.642814in}}%
\pgfpathlineto{\pgfqpoint{1.944514in}{1.650367in}}%
\pgfpathlineto{\pgfqpoint{2.034982in}{1.685247in}}%
\pgfpathlineto{\pgfqpoint{2.022620in}{1.676849in}}%
\pgfpathlineto{\pgfqpoint{1.934377in}{1.642814in}}%
\pgfpathclose%
\pgfusepath{fill}%
\end{pgfscope}%
\begin{pgfscope}%
\pgfpathrectangle{\pgfqpoint{0.000000in}{0.000000in}}{\pgfqpoint{3.000000in}{3.000000in}}%
\pgfusepath{clip}%
\pgfsetbuttcap%
\pgfsetroundjoin%
\definecolor{currentfill}{rgb}{1.000000,0.349310,0.000000}%
\pgfsetfillcolor{currentfill}%
\pgfsetlinewidth{0.000000pt}%
\definecolor{currentstroke}{rgb}{0.000000,0.000000,0.000000}%
\pgfsetstrokecolor{currentstroke}%
\pgfsetdash{}{0pt}%
\pgfpathmoveto{\pgfqpoint{1.606812in}{1.551688in}}%
\pgfpathlineto{\pgfqpoint{1.608756in}{1.559340in}}%
\pgfpathlineto{\pgfqpoint{1.709279in}{1.568643in}}%
\pgfpathlineto{\pgfqpoint{1.704473in}{1.560734in}}%
\pgfpathlineto{\pgfqpoint{1.606812in}{1.551688in}}%
\pgfpathclose%
\pgfusepath{fill}%
\end{pgfscope}%
\begin{pgfscope}%
\pgfpathrectangle{\pgfqpoint{0.000000in}{0.000000in}}{\pgfqpoint{3.000000in}{3.000000in}}%
\pgfusepath{clip}%
\pgfsetbuttcap%
\pgfsetroundjoin%
\definecolor{currentfill}{rgb}{1.000000,0.291213,0.000000}%
\pgfsetfillcolor{currentfill}%
\pgfsetlinewidth{0.000000pt}%
\definecolor{currentstroke}{rgb}{0.000000,0.000000,0.000000}%
\pgfsetstrokecolor{currentstroke}%
\pgfsetdash{}{0pt}%
\pgfpathmoveto{\pgfqpoint{1.709279in}{1.568643in}}%
\pgfpathlineto{\pgfqpoint{1.714096in}{1.576211in}}%
\pgfpathlineto{\pgfqpoint{1.812774in}{1.593673in}}%
\pgfpathlineto{\pgfqpoint{1.805231in}{1.585635in}}%
\pgfpathlineto{\pgfqpoint{1.709279in}{1.568643in}}%
\pgfpathclose%
\pgfusepath{fill}%
\end{pgfscope}%
\begin{pgfscope}%
\pgfpathrectangle{\pgfqpoint{0.000000in}{0.000000in}}{\pgfqpoint{3.000000in}{3.000000in}}%
\pgfusepath{clip}%
\pgfsetbuttcap%
\pgfsetroundjoin%
\definecolor{currentfill}{rgb}{1.000000,0.233115,0.000000}%
\pgfsetfillcolor{currentfill}%
\pgfsetlinewidth{0.000000pt}%
\definecolor{currentstroke}{rgb}{0.000000,0.000000,0.000000}%
\pgfsetstrokecolor{currentstroke}%
\pgfsetdash{}{0pt}%
\pgfpathmoveto{\pgfqpoint{1.206453in}{1.609459in}}%
\pgfpathlineto{\pgfqpoint{1.197190in}{1.617603in}}%
\pgfpathlineto{\pgfqpoint{1.293838in}{1.594542in}}%
\pgfpathlineto{\pgfqpoint{1.300509in}{1.587002in}}%
\pgfpathlineto{\pgfqpoint{1.206453in}{1.609459in}}%
\pgfpathclose%
\pgfusepath{fill}%
\end{pgfscope}%
\begin{pgfscope}%
\pgfpathrectangle{\pgfqpoint{0.000000in}{0.000000in}}{\pgfqpoint{3.000000in}{3.000000in}}%
\pgfusepath{clip}%
\pgfsetbuttcap%
\pgfsetroundjoin%
\definecolor{currentfill}{rgb}{1.000000,0.349310,0.000000}%
\pgfsetfillcolor{currentfill}%
\pgfsetlinewidth{0.000000pt}%
\definecolor{currentstroke}{rgb}{0.000000,0.000000,0.000000}%
\pgfsetstrokecolor{currentstroke}%
\pgfsetdash{}{0pt}%
\pgfpathmoveto{\pgfqpoint{1.408797in}{1.556870in}}%
\pgfpathlineto{\pgfqpoint{1.404934in}{1.564669in}}%
\pgfpathlineto{\pgfqpoint{1.506381in}{1.558002in}}%
\pgfpathlineto{\pgfqpoint{1.507355in}{1.550387in}}%
\pgfpathlineto{\pgfqpoint{1.408797in}{1.556870in}}%
\pgfpathclose%
\pgfusepath{fill}%
\end{pgfscope}%
\begin{pgfscope}%
\pgfpathrectangle{\pgfqpoint{0.000000in}{0.000000in}}{\pgfqpoint{3.000000in}{3.000000in}}%
\pgfusepath{clip}%
\pgfsetbuttcap%
\pgfsetroundjoin%
\definecolor{currentfill}{rgb}{0.553476,0.000000,0.000000}%
\pgfsetfillcolor{currentfill}%
\pgfsetlinewidth{0.000000pt}%
\definecolor{currentstroke}{rgb}{0.000000,0.000000,0.000000}%
\pgfsetstrokecolor{currentstroke}%
\pgfsetdash{}{0pt}%
\pgfpathmoveto{\pgfqpoint{0.833552in}{1.812944in}}%
\pgfpathlineto{\pgfqpoint{0.818115in}{1.821624in}}%
\pgfpathlineto{\pgfqpoint{0.896713in}{1.769316in}}%
\pgfpathlineto{\pgfqpoint{0.910549in}{1.761711in}}%
\pgfpathlineto{\pgfqpoint{0.833552in}{1.812944in}}%
\pgfpathclose%
\pgfusepath{fill}%
\end{pgfscope}%
\begin{pgfscope}%
\pgfpathrectangle{\pgfqpoint{0.000000in}{0.000000in}}{\pgfqpoint{3.000000in}{3.000000in}}%
\pgfusepath{clip}%
\pgfsetbuttcap%
\pgfsetroundjoin%
\definecolor{currentfill}{rgb}{0.731729,0.000000,0.000000}%
\pgfsetfillcolor{currentfill}%
\pgfsetlinewidth{0.000000pt}%
\definecolor{currentstroke}{rgb}{0.000000,0.000000,0.000000}%
\pgfsetstrokecolor{currentstroke}%
\pgfsetdash{}{0pt}%
\pgfpathmoveto{\pgfqpoint{2.072223in}{1.709034in}}%
\pgfpathlineto{\pgfqpoint{2.084688in}{1.716538in}}%
\pgfpathlineto{\pgfqpoint{2.169419in}{1.761727in}}%
\pgfpathlineto{\pgfqpoint{2.155083in}{1.753228in}}%
\pgfpathlineto{\pgfqpoint{2.072223in}{1.709034in}}%
\pgfpathclose%
\pgfusepath{fill}%
\end{pgfscope}%
\begin{pgfscope}%
\pgfpathrectangle{\pgfqpoint{0.000000in}{0.000000in}}{\pgfqpoint{3.000000in}{3.000000in}}%
\pgfusepath{clip}%
\pgfsetbuttcap%
\pgfsetroundjoin%
\definecolor{currentfill}{rgb}{1.000000,0.349310,0.000000}%
\pgfsetfillcolor{currentfill}%
\pgfsetlinewidth{0.000000pt}%
\definecolor{currentstroke}{rgb}{0.000000,0.000000,0.000000}%
\pgfsetstrokecolor{currentstroke}%
\pgfsetdash{}{0pt}%
\pgfpathmoveto{\pgfqpoint{1.507355in}{1.550387in}}%
\pgfpathlineto{\pgfqpoint{1.506381in}{1.558002in}}%
\pgfpathlineto{\pgfqpoint{1.608756in}{1.559340in}}%
\pgfpathlineto{\pgfqpoint{1.606812in}{1.551688in}}%
\pgfpathlineto{\pgfqpoint{1.507355in}{1.550387in}}%
\pgfpathclose%
\pgfusepath{fill}%
\end{pgfscope}%
\begin{pgfscope}%
\pgfpathrectangle{\pgfqpoint{0.000000in}{0.000000in}}{\pgfqpoint{3.000000in}{3.000000in}}%
\pgfusepath{clip}%
\pgfsetbuttcap%
\pgfsetroundjoin%
\definecolor{currentfill}{rgb}{1.000000,0.175018,0.000000}%
\pgfsetfillcolor{currentfill}%
\pgfsetlinewidth{0.000000pt}%
\definecolor{currentstroke}{rgb}{0.000000,0.000000,0.000000}%
\pgfsetstrokecolor{currentstroke}%
\pgfsetdash{}{0pt}%
\pgfpathmoveto{\pgfqpoint{1.820334in}{1.601393in}}%
\pgfpathlineto{\pgfqpoint{1.827913in}{1.608816in}}%
\pgfpathlineto{\pgfqpoint{1.924262in}{1.635000in}}%
\pgfpathlineto{\pgfqpoint{1.914169in}{1.626909in}}%
\pgfpathlineto{\pgfqpoint{1.820334in}{1.601393in}}%
\pgfpathclose%
\pgfusepath{fill}%
\end{pgfscope}%
\begin{pgfscope}%
\pgfpathrectangle{\pgfqpoint{0.000000in}{0.000000in}}{\pgfqpoint{3.000000in}{3.000000in}}%
\pgfusepath{clip}%
\pgfsetbuttcap%
\pgfsetroundjoin%
\definecolor{currentfill}{rgb}{1.000000,0.291213,0.000000}%
\pgfsetfillcolor{currentfill}%
\pgfsetlinewidth{0.000000pt}%
\definecolor{currentstroke}{rgb}{0.000000,0.000000,0.000000}%
\pgfsetstrokecolor{currentstroke}%
\pgfsetdash{}{0pt}%
\pgfpathmoveto{\pgfqpoint{1.307164in}{1.579143in}}%
\pgfpathlineto{\pgfqpoint{1.300509in}{1.587002in}}%
\pgfpathlineto{\pgfqpoint{1.401061in}{1.572127in}}%
\pgfpathlineto{\pgfqpoint{1.404934in}{1.564669in}}%
\pgfpathlineto{\pgfqpoint{1.307164in}{1.579143in}}%
\pgfpathclose%
\pgfusepath{fill}%
\end{pgfscope}%
\begin{pgfscope}%
\pgfpathrectangle{\pgfqpoint{0.000000in}{0.000000in}}{\pgfqpoint{3.000000in}{3.000000in}}%
\pgfusepath{clip}%
\pgfsetbuttcap%
\pgfsetroundjoin%
\definecolor{currentfill}{rgb}{0.803030,0.000000,0.000000}%
\pgfsetfillcolor{currentfill}%
\pgfsetlinewidth{0.000000pt}%
\definecolor{currentstroke}{rgb}{0.000000,0.000000,0.000000}%
\pgfsetstrokecolor{currentstroke}%
\pgfsetdash{}{0pt}%
\pgfpathmoveto{\pgfqpoint{0.965626in}{1.729464in}}%
\pgfpathlineto{\pgfqpoint{0.951896in}{1.737820in}}%
\pgfpathlineto{\pgfqpoint{1.039768in}{1.695750in}}%
\pgfpathlineto{\pgfqpoint{1.051501in}{1.688342in}}%
\pgfpathlineto{\pgfqpoint{0.965626in}{1.729464in}}%
\pgfpathclose%
\pgfusepath{fill}%
\end{pgfscope}%
\begin{pgfscope}%
\pgfpathrectangle{\pgfqpoint{0.000000in}{0.000000in}}{\pgfqpoint{3.000000in}{3.000000in}}%
\pgfusepath{clip}%
\pgfsetbuttcap%
\pgfsetroundjoin%
\definecolor{currentfill}{rgb}{0.999109,0.073348,0.000000}%
\pgfsetfillcolor{currentfill}%
\pgfsetlinewidth{0.000000pt}%
\definecolor{currentstroke}{rgb}{0.000000,0.000000,0.000000}%
\pgfsetstrokecolor{currentstroke}%
\pgfsetdash{}{0pt}%
\pgfpathmoveto{\pgfqpoint{1.086549in}{1.664763in}}%
\pgfpathlineto{\pgfqpoint{1.074891in}{1.672862in}}%
\pgfpathlineto{\pgfqpoint{1.169274in}{1.640331in}}%
\pgfpathlineto{\pgfqpoint{1.178601in}{1.633021in}}%
\pgfpathlineto{\pgfqpoint{1.086549in}{1.664763in}}%
\pgfpathclose%
\pgfusepath{fill}%
\end{pgfscope}%
\begin{pgfscope}%
\pgfpathrectangle{\pgfqpoint{0.000000in}{0.000000in}}{\pgfqpoint{3.000000in}{3.000000in}}%
\pgfusepath{clip}%
\pgfsetbuttcap%
\pgfsetroundjoin%
\definecolor{currentfill}{rgb}{0.927807,0.015251,0.000000}%
\pgfsetfillcolor{currentfill}%
\pgfsetlinewidth{0.000000pt}%
\definecolor{currentstroke}{rgb}{0.000000,0.000000,0.000000}%
\pgfsetstrokecolor{currentstroke}%
\pgfsetdash{}{0pt}%
\pgfpathmoveto{\pgfqpoint{1.944514in}{1.650367in}}%
\pgfpathlineto{\pgfqpoint{1.954675in}{1.657676in}}%
\pgfpathlineto{\pgfqpoint{2.047370in}{1.693401in}}%
\pgfpathlineto{\pgfqpoint{2.034982in}{1.685247in}}%
\pgfpathlineto{\pgfqpoint{1.944514in}{1.650367in}}%
\pgfpathclose%
\pgfusepath{fill}%
\end{pgfscope}%
\begin{pgfscope}%
\pgfpathrectangle{\pgfqpoint{0.000000in}{0.000000in}}{\pgfqpoint{3.000000in}{3.000000in}}%
\pgfusepath{clip}%
\pgfsetbuttcap%
\pgfsetroundjoin%
\definecolor{currentfill}{rgb}{0.500000,0.000000,0.000000}%
\pgfsetfillcolor{currentfill}%
\pgfsetlinewidth{0.000000pt}%
\definecolor{currentstroke}{rgb}{0.000000,0.000000,0.000000}%
\pgfsetstrokecolor{currentstroke}%
\pgfsetdash{}{0pt}%
\pgfpathmoveto{\pgfqpoint{0.818115in}{1.821624in}}%
\pgfpathlineto{\pgfqpoint{0.802653in}{1.830141in}}%
\pgfpathlineto{\pgfqpoint{0.882852in}{1.776757in}}%
\pgfpathlineto{\pgfqpoint{0.896713in}{1.769316in}}%
\pgfpathlineto{\pgfqpoint{0.818115in}{1.821624in}}%
\pgfpathclose%
\pgfusepath{fill}%
\end{pgfscope}%
\begin{pgfscope}%
\pgfpathrectangle{\pgfqpoint{0.000000in}{0.000000in}}{\pgfqpoint{3.000000in}{3.000000in}}%
\pgfusepath{clip}%
\pgfsetbuttcap%
\pgfsetroundjoin%
\definecolor{currentfill}{rgb}{1.000000,0.175018,0.000000}%
\pgfsetfillcolor{currentfill}%
\pgfsetlinewidth{0.000000pt}%
\definecolor{currentstroke}{rgb}{0.000000,0.000000,0.000000}%
\pgfsetstrokecolor{currentstroke}%
\pgfsetdash{}{0pt}%
\pgfpathmoveto{\pgfqpoint{1.197190in}{1.617603in}}%
\pgfpathlineto{\pgfqpoint{1.187907in}{1.625451in}}%
\pgfpathlineto{\pgfqpoint{1.287150in}{1.601785in}}%
\pgfpathlineto{\pgfqpoint{1.293838in}{1.594542in}}%
\pgfpathlineto{\pgfqpoint{1.197190in}{1.617603in}}%
\pgfpathclose%
\pgfusepath{fill}%
\end{pgfscope}%
\begin{pgfscope}%
\pgfpathrectangle{\pgfqpoint{0.000000in}{0.000000in}}{\pgfqpoint{3.000000in}{3.000000in}}%
\pgfusepath{clip}%
\pgfsetbuttcap%
\pgfsetroundjoin%
\definecolor{currentfill}{rgb}{1.000000,0.291213,0.000000}%
\pgfsetfillcolor{currentfill}%
\pgfsetlinewidth{0.000000pt}%
\definecolor{currentstroke}{rgb}{0.000000,0.000000,0.000000}%
\pgfsetstrokecolor{currentstroke}%
\pgfsetdash{}{0pt}%
\pgfpathmoveto{\pgfqpoint{1.608756in}{1.559340in}}%
\pgfpathlineto{\pgfqpoint{1.610706in}{1.566650in}}%
\pgfpathlineto{\pgfqpoint{1.714096in}{1.576211in}}%
\pgfpathlineto{\pgfqpoint{1.709279in}{1.568643in}}%
\pgfpathlineto{\pgfqpoint{1.608756in}{1.559340in}}%
\pgfpathclose%
\pgfusepath{fill}%
\end{pgfscope}%
\begin{pgfscope}%
\pgfpathrectangle{\pgfqpoint{0.000000in}{0.000000in}}{\pgfqpoint{3.000000in}{3.000000in}}%
\pgfusepath{clip}%
\pgfsetbuttcap%
\pgfsetroundjoin%
\definecolor{currentfill}{rgb}{1.000000,0.233115,0.000000}%
\pgfsetfillcolor{currentfill}%
\pgfsetlinewidth{0.000000pt}%
\definecolor{currentstroke}{rgb}{0.000000,0.000000,0.000000}%
\pgfsetstrokecolor{currentstroke}%
\pgfsetdash{}{0pt}%
\pgfpathmoveto{\pgfqpoint{1.714096in}{1.576211in}}%
\pgfpathlineto{\pgfqpoint{1.718925in}{1.583460in}}%
\pgfpathlineto{\pgfqpoint{1.820334in}{1.601393in}}%
\pgfpathlineto{\pgfqpoint{1.812774in}{1.593673in}}%
\pgfpathlineto{\pgfqpoint{1.714096in}{1.576211in}}%
\pgfpathclose%
\pgfusepath{fill}%
\end{pgfscope}%
\begin{pgfscope}%
\pgfpathrectangle{\pgfqpoint{0.000000in}{0.000000in}}{\pgfqpoint{3.000000in}{3.000000in}}%
\pgfusepath{clip}%
\pgfsetbuttcap%
\pgfsetroundjoin%
\definecolor{currentfill}{rgb}{0.678253,0.000000,0.000000}%
\pgfsetfillcolor{currentfill}%
\pgfsetlinewidth{0.000000pt}%
\definecolor{currentstroke}{rgb}{0.000000,0.000000,0.000000}%
\pgfsetstrokecolor{currentstroke}%
\pgfsetdash{}{0pt}%
\pgfpathmoveto{\pgfqpoint{2.084688in}{1.716538in}}%
\pgfpathlineto{\pgfqpoint{2.097179in}{1.723849in}}%
\pgfpathlineto{\pgfqpoint{2.183782in}{1.770033in}}%
\pgfpathlineto{\pgfqpoint{2.169419in}{1.761727in}}%
\pgfpathlineto{\pgfqpoint{2.084688in}{1.716538in}}%
\pgfpathclose%
\pgfusepath{fill}%
\end{pgfscope}%
\begin{pgfscope}%
\pgfpathrectangle{\pgfqpoint{0.000000in}{0.000000in}}{\pgfqpoint{3.000000in}{3.000000in}}%
\pgfusepath{clip}%
\pgfsetbuttcap%
\pgfsetroundjoin%
\definecolor{currentfill}{rgb}{1.000000,0.291213,0.000000}%
\pgfsetfillcolor{currentfill}%
\pgfsetlinewidth{0.000000pt}%
\definecolor{currentstroke}{rgb}{0.000000,0.000000,0.000000}%
\pgfsetstrokecolor{currentstroke}%
\pgfsetdash{}{0pt}%
\pgfpathmoveto{\pgfqpoint{1.404934in}{1.564669in}}%
\pgfpathlineto{\pgfqpoint{1.401061in}{1.572127in}}%
\pgfpathlineto{\pgfqpoint{1.505405in}{1.565275in}}%
\pgfpathlineto{\pgfqpoint{1.506381in}{1.558002in}}%
\pgfpathlineto{\pgfqpoint{1.404934in}{1.564669in}}%
\pgfpathclose%
\pgfusepath{fill}%
\end{pgfscope}%
\begin{pgfscope}%
\pgfpathrectangle{\pgfqpoint{0.000000in}{0.000000in}}{\pgfqpoint{3.000000in}{3.000000in}}%
\pgfusepath{clip}%
\pgfsetbuttcap%
\pgfsetroundjoin%
\definecolor{currentfill}{rgb}{1.000000,0.116921,0.000000}%
\pgfsetfillcolor{currentfill}%
\pgfsetlinewidth{0.000000pt}%
\definecolor{currentstroke}{rgb}{0.000000,0.000000,0.000000}%
\pgfsetstrokecolor{currentstroke}%
\pgfsetdash{}{0pt}%
\pgfpathmoveto{\pgfqpoint{1.827913in}{1.608816in}}%
\pgfpathlineto{\pgfqpoint{1.835510in}{1.615961in}}%
\pgfpathlineto{\pgfqpoint{1.934377in}{1.642814in}}%
\pgfpathlineto{\pgfqpoint{1.924262in}{1.635000in}}%
\pgfpathlineto{\pgfqpoint{1.827913in}{1.608816in}}%
\pgfpathclose%
\pgfusepath{fill}%
\end{pgfscope}%
\begin{pgfscope}%
\pgfpathrectangle{\pgfqpoint{0.000000in}{0.000000in}}{\pgfqpoint{3.000000in}{3.000000in}}%
\pgfusepath{clip}%
\pgfsetbuttcap%
\pgfsetroundjoin%
\definecolor{currentfill}{rgb}{1.000000,0.291213,0.000000}%
\pgfsetfillcolor{currentfill}%
\pgfsetlinewidth{0.000000pt}%
\definecolor{currentstroke}{rgb}{0.000000,0.000000,0.000000}%
\pgfsetstrokecolor{currentstroke}%
\pgfsetdash{}{0pt}%
\pgfpathmoveto{\pgfqpoint{1.506381in}{1.558002in}}%
\pgfpathlineto{\pgfqpoint{1.505405in}{1.565275in}}%
\pgfpathlineto{\pgfqpoint{1.610706in}{1.566650in}}%
\pgfpathlineto{\pgfqpoint{1.608756in}{1.559340in}}%
\pgfpathlineto{\pgfqpoint{1.506381in}{1.558002in}}%
\pgfpathclose%
\pgfusepath{fill}%
\end{pgfscope}%
\begin{pgfscope}%
\pgfpathrectangle{\pgfqpoint{0.000000in}{0.000000in}}{\pgfqpoint{3.000000in}{3.000000in}}%
\pgfusepath{clip}%
\pgfsetbuttcap%
\pgfsetroundjoin%
\definecolor{currentfill}{rgb}{0.731729,0.000000,0.000000}%
\pgfsetfillcolor{currentfill}%
\pgfsetlinewidth{0.000000pt}%
\definecolor{currentstroke}{rgb}{0.000000,0.000000,0.000000}%
\pgfsetstrokecolor{currentstroke}%
\pgfsetdash{}{0pt}%
\pgfpathmoveto{\pgfqpoint{0.951896in}{1.737820in}}%
\pgfpathlineto{\pgfqpoint{0.938140in}{1.745972in}}%
\pgfpathlineto{\pgfqpoint{1.028010in}{1.702953in}}%
\pgfpathlineto{\pgfqpoint{1.039768in}{1.695750in}}%
\pgfpathlineto{\pgfqpoint{0.951896in}{1.737820in}}%
\pgfpathclose%
\pgfusepath{fill}%
\end{pgfscope}%
\begin{pgfscope}%
\pgfpathrectangle{\pgfqpoint{0.000000in}{0.000000in}}{\pgfqpoint{3.000000in}{3.000000in}}%
\pgfusepath{clip}%
\pgfsetbuttcap%
\pgfsetroundjoin%
\definecolor{currentfill}{rgb}{1.000000,0.233115,0.000000}%
\pgfsetfillcolor{currentfill}%
\pgfsetlinewidth{0.000000pt}%
\definecolor{currentstroke}{rgb}{0.000000,0.000000,0.000000}%
\pgfsetstrokecolor{currentstroke}%
\pgfsetdash{}{0pt}%
\pgfpathmoveto{\pgfqpoint{1.300509in}{1.587002in}}%
\pgfpathlineto{\pgfqpoint{1.293838in}{1.594542in}}%
\pgfpathlineto{\pgfqpoint{1.397178in}{1.579265in}}%
\pgfpathlineto{\pgfqpoint{1.401061in}{1.572127in}}%
\pgfpathlineto{\pgfqpoint{1.300509in}{1.587002in}}%
\pgfpathclose%
\pgfusepath{fill}%
\end{pgfscope}%
\begin{pgfscope}%
\pgfpathrectangle{\pgfqpoint{0.000000in}{0.000000in}}{\pgfqpoint{3.000000in}{3.000000in}}%
\pgfusepath{clip}%
\pgfsetbuttcap%
\pgfsetroundjoin%
\definecolor{currentfill}{rgb}{0.927807,0.015251,0.000000}%
\pgfsetfillcolor{currentfill}%
\pgfsetlinewidth{0.000000pt}%
\definecolor{currentstroke}{rgb}{0.000000,0.000000,0.000000}%
\pgfsetstrokecolor{currentstroke}%
\pgfsetdash{}{0pt}%
\pgfpathmoveto{\pgfqpoint{1.074891in}{1.672862in}}%
\pgfpathlineto{\pgfqpoint{1.063208in}{1.680717in}}%
\pgfpathlineto{\pgfqpoint{1.159926in}{1.647394in}}%
\pgfpathlineto{\pgfqpoint{1.169274in}{1.640331in}}%
\pgfpathlineto{\pgfqpoint{1.074891in}{1.672862in}}%
\pgfpathclose%
\pgfusepath{fill}%
\end{pgfscope}%
\begin{pgfscope}%
\pgfpathrectangle{\pgfqpoint{0.000000in}{0.000000in}}{\pgfqpoint{3.000000in}{3.000000in}}%
\pgfusepath{clip}%
\pgfsetbuttcap%
\pgfsetroundjoin%
\definecolor{currentfill}{rgb}{0.856506,0.000000,0.000000}%
\pgfsetfillcolor{currentfill}%
\pgfsetlinewidth{0.000000pt}%
\definecolor{currentstroke}{rgb}{0.000000,0.000000,0.000000}%
\pgfsetstrokecolor{currentstroke}%
\pgfsetdash{}{0pt}%
\pgfpathmoveto{\pgfqpoint{1.954675in}{1.657676in}}%
\pgfpathlineto{\pgfqpoint{1.964859in}{1.664754in}}%
\pgfpathlineto{\pgfqpoint{2.059784in}{1.701326in}}%
\pgfpathlineto{\pgfqpoint{2.047370in}{1.693401in}}%
\pgfpathlineto{\pgfqpoint{1.954675in}{1.657676in}}%
\pgfpathclose%
\pgfusepath{fill}%
\end{pgfscope}%
\begin{pgfscope}%
\pgfpathrectangle{\pgfqpoint{0.000000in}{0.000000in}}{\pgfqpoint{3.000000in}{3.000000in}}%
\pgfusepath{clip}%
\pgfsetbuttcap%
\pgfsetroundjoin%
\definecolor{currentfill}{rgb}{0.606952,0.000000,0.000000}%
\pgfsetfillcolor{currentfill}%
\pgfsetlinewidth{0.000000pt}%
\definecolor{currentstroke}{rgb}{0.000000,0.000000,0.000000}%
\pgfsetstrokecolor{currentstroke}%
\pgfsetdash{}{0pt}%
\pgfpathmoveto{\pgfqpoint{2.097179in}{1.723849in}}%
\pgfpathlineto{\pgfqpoint{2.109695in}{1.730977in}}%
\pgfpathlineto{\pgfqpoint{2.198171in}{1.778157in}}%
\pgfpathlineto{\pgfqpoint{2.183782in}{1.770033in}}%
\pgfpathlineto{\pgfqpoint{2.097179in}{1.723849in}}%
\pgfpathclose%
\pgfusepath{fill}%
\end{pgfscope}%
\begin{pgfscope}%
\pgfpathrectangle{\pgfqpoint{0.000000in}{0.000000in}}{\pgfqpoint{3.000000in}{3.000000in}}%
\pgfusepath{clip}%
\pgfsetbuttcap%
\pgfsetroundjoin%
\definecolor{currentfill}{rgb}{1.000000,0.116921,0.000000}%
\pgfsetfillcolor{currentfill}%
\pgfsetlinewidth{0.000000pt}%
\definecolor{currentstroke}{rgb}{0.000000,0.000000,0.000000}%
\pgfsetstrokecolor{currentstroke}%
\pgfsetdash{}{0pt}%
\pgfpathmoveto{\pgfqpoint{1.187907in}{1.625451in}}%
\pgfpathlineto{\pgfqpoint{1.178601in}{1.633021in}}%
\pgfpathlineto{\pgfqpoint{1.280446in}{1.608749in}}%
\pgfpathlineto{\pgfqpoint{1.287150in}{1.601785in}}%
\pgfpathlineto{\pgfqpoint{1.187907in}{1.625451in}}%
\pgfpathclose%
\pgfusepath{fill}%
\end{pgfscope}%
\begin{pgfscope}%
\pgfpathrectangle{\pgfqpoint{0.000000in}{0.000000in}}{\pgfqpoint{3.000000in}{3.000000in}}%
\pgfusepath{clip}%
\pgfsetbuttcap%
\pgfsetroundjoin%
\definecolor{currentfill}{rgb}{1.000000,0.175018,0.000000}%
\pgfsetfillcolor{currentfill}%
\pgfsetlinewidth{0.000000pt}%
\definecolor{currentstroke}{rgb}{0.000000,0.000000,0.000000}%
\pgfsetstrokecolor{currentstroke}%
\pgfsetdash{}{0pt}%
\pgfpathmoveto{\pgfqpoint{1.718925in}{1.583460in}}%
\pgfpathlineto{\pgfqpoint{1.723767in}{1.590411in}}%
\pgfpathlineto{\pgfqpoint{1.827913in}{1.608816in}}%
\pgfpathlineto{\pgfqpoint{1.820334in}{1.601393in}}%
\pgfpathlineto{\pgfqpoint{1.718925in}{1.583460in}}%
\pgfpathclose%
\pgfusepath{fill}%
\end{pgfscope}%
\begin{pgfscope}%
\pgfpathrectangle{\pgfqpoint{0.000000in}{0.000000in}}{\pgfqpoint{3.000000in}{3.000000in}}%
\pgfusepath{clip}%
\pgfsetbuttcap%
\pgfsetroundjoin%
\definecolor{currentfill}{rgb}{1.000000,0.233115,0.000000}%
\pgfsetfillcolor{currentfill}%
\pgfsetlinewidth{0.000000pt}%
\definecolor{currentstroke}{rgb}{0.000000,0.000000,0.000000}%
\pgfsetstrokecolor{currentstroke}%
\pgfsetdash{}{0pt}%
\pgfpathmoveto{\pgfqpoint{1.610706in}{1.566650in}}%
\pgfpathlineto{\pgfqpoint{1.612660in}{1.573640in}}%
\pgfpathlineto{\pgfqpoint{1.718925in}{1.583460in}}%
\pgfpathlineto{\pgfqpoint{1.714096in}{1.576211in}}%
\pgfpathlineto{\pgfqpoint{1.610706in}{1.566650in}}%
\pgfpathclose%
\pgfusepath{fill}%
\end{pgfscope}%
\begin{pgfscope}%
\pgfpathrectangle{\pgfqpoint{0.000000in}{0.000000in}}{\pgfqpoint{3.000000in}{3.000000in}}%
\pgfusepath{clip}%
\pgfsetbuttcap%
\pgfsetroundjoin%
\definecolor{currentfill}{rgb}{0.678253,0.000000,0.000000}%
\pgfsetfillcolor{currentfill}%
\pgfsetlinewidth{0.000000pt}%
\definecolor{currentstroke}{rgb}{0.000000,0.000000,0.000000}%
\pgfsetstrokecolor{currentstroke}%
\pgfsetdash{}{0pt}%
\pgfpathmoveto{\pgfqpoint{0.938140in}{1.745972in}}%
\pgfpathlineto{\pgfqpoint{0.924357in}{1.753933in}}%
\pgfpathlineto{\pgfqpoint{1.016228in}{1.709962in}}%
\pgfpathlineto{\pgfqpoint{1.028010in}{1.702953in}}%
\pgfpathlineto{\pgfqpoint{0.938140in}{1.745972in}}%
\pgfpathclose%
\pgfusepath{fill}%
\end{pgfscope}%
\begin{pgfscope}%
\pgfpathrectangle{\pgfqpoint{0.000000in}{0.000000in}}{\pgfqpoint{3.000000in}{3.000000in}}%
\pgfusepath{clip}%
\pgfsetbuttcap%
\pgfsetroundjoin%
\definecolor{currentfill}{rgb}{1.000000,0.233115,0.000000}%
\pgfsetfillcolor{currentfill}%
\pgfsetlinewidth{0.000000pt}%
\definecolor{currentstroke}{rgb}{0.000000,0.000000,0.000000}%
\pgfsetstrokecolor{currentstroke}%
\pgfsetdash{}{0pt}%
\pgfpathmoveto{\pgfqpoint{1.401061in}{1.572127in}}%
\pgfpathlineto{\pgfqpoint{1.397178in}{1.579265in}}%
\pgfpathlineto{\pgfqpoint{1.504426in}{1.572228in}}%
\pgfpathlineto{\pgfqpoint{1.505405in}{1.565275in}}%
\pgfpathlineto{\pgfqpoint{1.401061in}{1.572127in}}%
\pgfpathclose%
\pgfusepath{fill}%
\end{pgfscope}%
\begin{pgfscope}%
\pgfpathrectangle{\pgfqpoint{0.000000in}{0.000000in}}{\pgfqpoint{3.000000in}{3.000000in}}%
\pgfusepath{clip}%
\pgfsetbuttcap%
\pgfsetroundjoin%
\definecolor{currentfill}{rgb}{0.999109,0.073348,0.000000}%
\pgfsetfillcolor{currentfill}%
\pgfsetlinewidth{0.000000pt}%
\definecolor{currentstroke}{rgb}{0.000000,0.000000,0.000000}%
\pgfsetstrokecolor{currentstroke}%
\pgfsetdash{}{0pt}%
\pgfpathmoveto{\pgfqpoint{1.835510in}{1.615961in}}%
\pgfpathlineto{\pgfqpoint{1.843126in}{1.622843in}}%
\pgfpathlineto{\pgfqpoint{1.944514in}{1.650367in}}%
\pgfpathlineto{\pgfqpoint{1.934377in}{1.642814in}}%
\pgfpathlineto{\pgfqpoint{1.835510in}{1.615961in}}%
\pgfpathclose%
\pgfusepath{fill}%
\end{pgfscope}%
\begin{pgfscope}%
\pgfpathrectangle{\pgfqpoint{0.000000in}{0.000000in}}{\pgfqpoint{3.000000in}{3.000000in}}%
\pgfusepath{clip}%
\pgfsetbuttcap%
\pgfsetroundjoin%
\definecolor{currentfill}{rgb}{1.000000,0.175018,0.000000}%
\pgfsetfillcolor{currentfill}%
\pgfsetlinewidth{0.000000pt}%
\definecolor{currentstroke}{rgb}{0.000000,0.000000,0.000000}%
\pgfsetstrokecolor{currentstroke}%
\pgfsetdash{}{0pt}%
\pgfpathmoveto{\pgfqpoint{1.293838in}{1.594542in}}%
\pgfpathlineto{\pgfqpoint{1.287150in}{1.601785in}}%
\pgfpathlineto{\pgfqpoint{1.393285in}{1.586106in}}%
\pgfpathlineto{\pgfqpoint{1.397178in}{1.579265in}}%
\pgfpathlineto{\pgfqpoint{1.293838in}{1.594542in}}%
\pgfpathclose%
\pgfusepath{fill}%
\end{pgfscope}%
\begin{pgfscope}%
\pgfpathrectangle{\pgfqpoint{0.000000in}{0.000000in}}{\pgfqpoint{3.000000in}{3.000000in}}%
\pgfusepath{clip}%
\pgfsetbuttcap%
\pgfsetroundjoin%
\definecolor{currentfill}{rgb}{1.000000,0.233115,0.000000}%
\pgfsetfillcolor{currentfill}%
\pgfsetlinewidth{0.000000pt}%
\definecolor{currentstroke}{rgb}{0.000000,0.000000,0.000000}%
\pgfsetstrokecolor{currentstroke}%
\pgfsetdash{}{0pt}%
\pgfpathmoveto{\pgfqpoint{1.505405in}{1.565275in}}%
\pgfpathlineto{\pgfqpoint{1.504426in}{1.572228in}}%
\pgfpathlineto{\pgfqpoint{1.612660in}{1.573640in}}%
\pgfpathlineto{\pgfqpoint{1.610706in}{1.566650in}}%
\pgfpathlineto{\pgfqpoint{1.505405in}{1.565275in}}%
\pgfpathclose%
\pgfusepath{fill}%
\end{pgfscope}%
\begin{pgfscope}%
\pgfpathrectangle{\pgfqpoint{0.000000in}{0.000000in}}{\pgfqpoint{3.000000in}{3.000000in}}%
\pgfusepath{clip}%
\pgfsetbuttcap%
\pgfsetroundjoin%
\definecolor{currentfill}{rgb}{0.803030,0.000000,0.000000}%
\pgfsetfillcolor{currentfill}%
\pgfsetlinewidth{0.000000pt}%
\definecolor{currentstroke}{rgb}{0.000000,0.000000,0.000000}%
\pgfsetstrokecolor{currentstroke}%
\pgfsetdash{}{0pt}%
\pgfpathmoveto{\pgfqpoint{1.964859in}{1.664754in}}%
\pgfpathlineto{\pgfqpoint{1.975065in}{1.671614in}}%
\pgfpathlineto{\pgfqpoint{2.072223in}{1.709034in}}%
\pgfpathlineto{\pgfqpoint{2.059784in}{1.701326in}}%
\pgfpathlineto{\pgfqpoint{1.964859in}{1.664754in}}%
\pgfpathclose%
\pgfusepath{fill}%
\end{pgfscope}%
\begin{pgfscope}%
\pgfpathrectangle{\pgfqpoint{0.000000in}{0.000000in}}{\pgfqpoint{3.000000in}{3.000000in}}%
\pgfusepath{clip}%
\pgfsetbuttcap%
\pgfsetroundjoin%
\definecolor{currentfill}{rgb}{0.856506,0.000000,0.000000}%
\pgfsetfillcolor{currentfill}%
\pgfsetlinewidth{0.000000pt}%
\definecolor{currentstroke}{rgb}{0.000000,0.000000,0.000000}%
\pgfsetstrokecolor{currentstroke}%
\pgfsetdash{}{0pt}%
\pgfpathmoveto{\pgfqpoint{1.063208in}{1.680717in}}%
\pgfpathlineto{\pgfqpoint{1.051501in}{1.688342in}}%
\pgfpathlineto{\pgfqpoint{1.150556in}{1.654227in}}%
\pgfpathlineto{\pgfqpoint{1.159926in}{1.647394in}}%
\pgfpathlineto{\pgfqpoint{1.063208in}{1.680717in}}%
\pgfpathclose%
\pgfusepath{fill}%
\end{pgfscope}%
\begin{pgfscope}%
\pgfpathrectangle{\pgfqpoint{0.000000in}{0.000000in}}{\pgfqpoint{3.000000in}{3.000000in}}%
\pgfusepath{clip}%
\pgfsetbuttcap%
\pgfsetroundjoin%
\definecolor{currentfill}{rgb}{0.553476,0.000000,0.000000}%
\pgfsetfillcolor{currentfill}%
\pgfsetlinewidth{0.000000pt}%
\definecolor{currentstroke}{rgb}{0.000000,0.000000,0.000000}%
\pgfsetstrokecolor{currentstroke}%
\pgfsetdash{}{0pt}%
\pgfpathmoveto{\pgfqpoint{2.109695in}{1.730977in}}%
\pgfpathlineto{\pgfqpoint{2.122237in}{1.737931in}}%
\pgfpathlineto{\pgfqpoint{2.212587in}{1.786109in}}%
\pgfpathlineto{\pgfqpoint{2.198171in}{1.778157in}}%
\pgfpathlineto{\pgfqpoint{2.109695in}{1.730977in}}%
\pgfpathclose%
\pgfusepath{fill}%
\end{pgfscope}%
\begin{pgfscope}%
\pgfpathrectangle{\pgfqpoint{0.000000in}{0.000000in}}{\pgfqpoint{3.000000in}{3.000000in}}%
\pgfusepath{clip}%
\pgfsetbuttcap%
\pgfsetroundjoin%
\definecolor{currentfill}{rgb}{0.999109,0.073348,0.000000}%
\pgfsetfillcolor{currentfill}%
\pgfsetlinewidth{0.000000pt}%
\definecolor{currentstroke}{rgb}{0.000000,0.000000,0.000000}%
\pgfsetstrokecolor{currentstroke}%
\pgfsetdash{}{0pt}%
\pgfpathmoveto{\pgfqpoint{1.178601in}{1.633021in}}%
\pgfpathlineto{\pgfqpoint{1.169274in}{1.640331in}}%
\pgfpathlineto{\pgfqpoint{1.273726in}{1.615450in}}%
\pgfpathlineto{\pgfqpoint{1.280446in}{1.608749in}}%
\pgfpathlineto{\pgfqpoint{1.178601in}{1.633021in}}%
\pgfpathclose%
\pgfusepath{fill}%
\end{pgfscope}%
\begin{pgfscope}%
\pgfpathrectangle{\pgfqpoint{0.000000in}{0.000000in}}{\pgfqpoint{3.000000in}{3.000000in}}%
\pgfusepath{clip}%
\pgfsetbuttcap%
\pgfsetroundjoin%
\definecolor{currentfill}{rgb}{1.000000,0.116921,0.000000}%
\pgfsetfillcolor{currentfill}%
\pgfsetlinewidth{0.000000pt}%
\definecolor{currentstroke}{rgb}{0.000000,0.000000,0.000000}%
\pgfsetstrokecolor{currentstroke}%
\pgfsetdash{}{0pt}%
\pgfpathmoveto{\pgfqpoint{1.723767in}{1.590411in}}%
\pgfpathlineto{\pgfqpoint{1.728620in}{1.597082in}}%
\pgfpathlineto{\pgfqpoint{1.835510in}{1.615961in}}%
\pgfpathlineto{\pgfqpoint{1.827913in}{1.608816in}}%
\pgfpathlineto{\pgfqpoint{1.723767in}{1.590411in}}%
\pgfpathclose%
\pgfusepath{fill}%
\end{pgfscope}%
\begin{pgfscope}%
\pgfpathrectangle{\pgfqpoint{0.000000in}{0.000000in}}{\pgfqpoint{3.000000in}{3.000000in}}%
\pgfusepath{clip}%
\pgfsetbuttcap%
\pgfsetroundjoin%
\definecolor{currentfill}{rgb}{0.606952,0.000000,0.000000}%
\pgfsetfillcolor{currentfill}%
\pgfsetlinewidth{0.000000pt}%
\definecolor{currentstroke}{rgb}{0.000000,0.000000,0.000000}%
\pgfsetstrokecolor{currentstroke}%
\pgfsetdash{}{0pt}%
\pgfpathmoveto{\pgfqpoint{0.924357in}{1.753933in}}%
\pgfpathlineto{\pgfqpoint{0.910549in}{1.761711in}}%
\pgfpathlineto{\pgfqpoint{1.004420in}{1.716788in}}%
\pgfpathlineto{\pgfqpoint{1.016228in}{1.709962in}}%
\pgfpathlineto{\pgfqpoint{0.924357in}{1.753933in}}%
\pgfpathclose%
\pgfusepath{fill}%
\end{pgfscope}%
\begin{pgfscope}%
\pgfpathrectangle{\pgfqpoint{0.000000in}{0.000000in}}{\pgfqpoint{3.000000in}{3.000000in}}%
\pgfusepath{clip}%
\pgfsetbuttcap%
\pgfsetroundjoin%
\definecolor{currentfill}{rgb}{1.000000,0.175018,0.000000}%
\pgfsetfillcolor{currentfill}%
\pgfsetlinewidth{0.000000pt}%
\definecolor{currentstroke}{rgb}{0.000000,0.000000,0.000000}%
\pgfsetstrokecolor{currentstroke}%
\pgfsetdash{}{0pt}%
\pgfpathmoveto{\pgfqpoint{1.612660in}{1.573640in}}%
\pgfpathlineto{\pgfqpoint{1.614620in}{1.580332in}}%
\pgfpathlineto{\pgfqpoint{1.723767in}{1.590411in}}%
\pgfpathlineto{\pgfqpoint{1.718925in}{1.583460in}}%
\pgfpathlineto{\pgfqpoint{1.612660in}{1.573640in}}%
\pgfpathclose%
\pgfusepath{fill}%
\end{pgfscope}%
\begin{pgfscope}%
\pgfpathrectangle{\pgfqpoint{0.000000in}{0.000000in}}{\pgfqpoint{3.000000in}{3.000000in}}%
\pgfusepath{clip}%
\pgfsetbuttcap%
\pgfsetroundjoin%
\definecolor{currentfill}{rgb}{0.927807,0.015251,0.000000}%
\pgfsetfillcolor{currentfill}%
\pgfsetlinewidth{0.000000pt}%
\definecolor{currentstroke}{rgb}{0.000000,0.000000,0.000000}%
\pgfsetstrokecolor{currentstroke}%
\pgfsetdash{}{0pt}%
\pgfpathmoveto{\pgfqpoint{1.843126in}{1.622843in}}%
\pgfpathlineto{\pgfqpoint{1.850760in}{1.629479in}}%
\pgfpathlineto{\pgfqpoint{1.954675in}{1.657676in}}%
\pgfpathlineto{\pgfqpoint{1.944514in}{1.650367in}}%
\pgfpathlineto{\pgfqpoint{1.843126in}{1.622843in}}%
\pgfpathclose%
\pgfusepath{fill}%
\end{pgfscope}%
\begin{pgfscope}%
\pgfpathrectangle{\pgfqpoint{0.000000in}{0.000000in}}{\pgfqpoint{3.000000in}{3.000000in}}%
\pgfusepath{clip}%
\pgfsetbuttcap%
\pgfsetroundjoin%
\definecolor{currentfill}{rgb}{1.000000,0.175018,0.000000}%
\pgfsetfillcolor{currentfill}%
\pgfsetlinewidth{0.000000pt}%
\definecolor{currentstroke}{rgb}{0.000000,0.000000,0.000000}%
\pgfsetstrokecolor{currentstroke}%
\pgfsetdash{}{0pt}%
\pgfpathmoveto{\pgfqpoint{1.397178in}{1.579265in}}%
\pgfpathlineto{\pgfqpoint{1.393285in}{1.586106in}}%
\pgfpathlineto{\pgfqpoint{1.503444in}{1.578882in}}%
\pgfpathlineto{\pgfqpoint{1.504426in}{1.572228in}}%
\pgfpathlineto{\pgfqpoint{1.397178in}{1.579265in}}%
\pgfpathclose%
\pgfusepath{fill}%
\end{pgfscope}%
\begin{pgfscope}%
\pgfpathrectangle{\pgfqpoint{0.000000in}{0.000000in}}{\pgfqpoint{3.000000in}{3.000000in}}%
\pgfusepath{clip}%
\pgfsetbuttcap%
\pgfsetroundjoin%
\definecolor{currentfill}{rgb}{1.000000,0.116921,0.000000}%
\pgfsetfillcolor{currentfill}%
\pgfsetlinewidth{0.000000pt}%
\definecolor{currentstroke}{rgb}{0.000000,0.000000,0.000000}%
\pgfsetstrokecolor{currentstroke}%
\pgfsetdash{}{0pt}%
\pgfpathmoveto{\pgfqpoint{1.287150in}{1.601785in}}%
\pgfpathlineto{\pgfqpoint{1.280446in}{1.608749in}}%
\pgfpathlineto{\pgfqpoint{1.389383in}{1.592666in}}%
\pgfpathlineto{\pgfqpoint{1.393285in}{1.586106in}}%
\pgfpathlineto{\pgfqpoint{1.287150in}{1.601785in}}%
\pgfpathclose%
\pgfusepath{fill}%
\end{pgfscope}%
\begin{pgfscope}%
\pgfpathrectangle{\pgfqpoint{0.000000in}{0.000000in}}{\pgfqpoint{3.000000in}{3.000000in}}%
\pgfusepath{clip}%
\pgfsetbuttcap%
\pgfsetroundjoin%
\definecolor{currentfill}{rgb}{0.731729,0.000000,0.000000}%
\pgfsetfillcolor{currentfill}%
\pgfsetlinewidth{0.000000pt}%
\definecolor{currentstroke}{rgb}{0.000000,0.000000,0.000000}%
\pgfsetstrokecolor{currentstroke}%
\pgfsetdash{}{0pt}%
\pgfpathmoveto{\pgfqpoint{1.975065in}{1.671614in}}%
\pgfpathlineto{\pgfqpoint{1.985295in}{1.678268in}}%
\pgfpathlineto{\pgfqpoint{2.084688in}{1.716538in}}%
\pgfpathlineto{\pgfqpoint{2.072223in}{1.709034in}}%
\pgfpathlineto{\pgfqpoint{1.975065in}{1.671614in}}%
\pgfpathclose%
\pgfusepath{fill}%
\end{pgfscope}%
\begin{pgfscope}%
\pgfpathrectangle{\pgfqpoint{0.000000in}{0.000000in}}{\pgfqpoint{3.000000in}{3.000000in}}%
\pgfusepath{clip}%
\pgfsetbuttcap%
\pgfsetroundjoin%
\definecolor{currentfill}{rgb}{0.500000,0.000000,0.000000}%
\pgfsetfillcolor{currentfill}%
\pgfsetlinewidth{0.000000pt}%
\definecolor{currentstroke}{rgb}{0.000000,0.000000,0.000000}%
\pgfsetstrokecolor{currentstroke}%
\pgfsetdash{}{0pt}%
\pgfpathmoveto{\pgfqpoint{2.122237in}{1.737931in}}%
\pgfpathlineto{\pgfqpoint{2.134805in}{1.744720in}}%
\pgfpathlineto{\pgfqpoint{2.227029in}{1.793897in}}%
\pgfpathlineto{\pgfqpoint{2.212587in}{1.786109in}}%
\pgfpathlineto{\pgfqpoint{2.122237in}{1.737931in}}%
\pgfpathclose%
\pgfusepath{fill}%
\end{pgfscope}%
\begin{pgfscope}%
\pgfpathrectangle{\pgfqpoint{0.000000in}{0.000000in}}{\pgfqpoint{3.000000in}{3.000000in}}%
\pgfusepath{clip}%
\pgfsetbuttcap%
\pgfsetroundjoin%
\definecolor{currentfill}{rgb}{0.803030,0.000000,0.000000}%
\pgfsetfillcolor{currentfill}%
\pgfsetlinewidth{0.000000pt}%
\definecolor{currentstroke}{rgb}{0.000000,0.000000,0.000000}%
\pgfsetstrokecolor{currentstroke}%
\pgfsetdash{}{0pt}%
\pgfpathmoveto{\pgfqpoint{1.051501in}{1.688342in}}%
\pgfpathlineto{\pgfqpoint{1.039768in}{1.695750in}}%
\pgfpathlineto{\pgfqpoint{1.141164in}{1.660842in}}%
\pgfpathlineto{\pgfqpoint{1.150556in}{1.654227in}}%
\pgfpathlineto{\pgfqpoint{1.051501in}{1.688342in}}%
\pgfpathclose%
\pgfusepath{fill}%
\end{pgfscope}%
\begin{pgfscope}%
\pgfpathrectangle{\pgfqpoint{0.000000in}{0.000000in}}{\pgfqpoint{3.000000in}{3.000000in}}%
\pgfusepath{clip}%
\pgfsetbuttcap%
\pgfsetroundjoin%
\definecolor{currentfill}{rgb}{1.000000,0.175018,0.000000}%
\pgfsetfillcolor{currentfill}%
\pgfsetlinewidth{0.000000pt}%
\definecolor{currentstroke}{rgb}{0.000000,0.000000,0.000000}%
\pgfsetstrokecolor{currentstroke}%
\pgfsetdash{}{0pt}%
\pgfpathmoveto{\pgfqpoint{1.504426in}{1.572228in}}%
\pgfpathlineto{\pgfqpoint{1.503444in}{1.578882in}}%
\pgfpathlineto{\pgfqpoint{1.614620in}{1.580332in}}%
\pgfpathlineto{\pgfqpoint{1.612660in}{1.573640in}}%
\pgfpathlineto{\pgfqpoint{1.504426in}{1.572228in}}%
\pgfpathclose%
\pgfusepath{fill}%
\end{pgfscope}%
\begin{pgfscope}%
\pgfpathrectangle{\pgfqpoint{0.000000in}{0.000000in}}{\pgfqpoint{3.000000in}{3.000000in}}%
\pgfusepath{clip}%
\pgfsetbuttcap%
\pgfsetroundjoin%
\definecolor{currentfill}{rgb}{0.553476,0.000000,0.000000}%
\pgfsetfillcolor{currentfill}%
\pgfsetlinewidth{0.000000pt}%
\definecolor{currentstroke}{rgb}{0.000000,0.000000,0.000000}%
\pgfsetstrokecolor{currentstroke}%
\pgfsetdash{}{0pt}%
\pgfpathmoveto{\pgfqpoint{0.910549in}{1.761711in}}%
\pgfpathlineto{\pgfqpoint{0.896713in}{1.769316in}}%
\pgfpathlineto{\pgfqpoint{0.992587in}{1.723440in}}%
\pgfpathlineto{\pgfqpoint{1.004420in}{1.716788in}}%
\pgfpathlineto{\pgfqpoint{0.910549in}{1.761711in}}%
\pgfpathclose%
\pgfusepath{fill}%
\end{pgfscope}%
\begin{pgfscope}%
\pgfpathrectangle{\pgfqpoint{0.000000in}{0.000000in}}{\pgfqpoint{3.000000in}{3.000000in}}%
\pgfusepath{clip}%
\pgfsetbuttcap%
\pgfsetroundjoin%
\definecolor{currentfill}{rgb}{0.927807,0.015251,0.000000}%
\pgfsetfillcolor{currentfill}%
\pgfsetlinewidth{0.000000pt}%
\definecolor{currentstroke}{rgb}{0.000000,0.000000,0.000000}%
\pgfsetstrokecolor{currentstroke}%
\pgfsetdash{}{0pt}%
\pgfpathmoveto{\pgfqpoint{1.169274in}{1.640331in}}%
\pgfpathlineto{\pgfqpoint{1.159926in}{1.647394in}}%
\pgfpathlineto{\pgfqpoint{1.266989in}{1.621905in}}%
\pgfpathlineto{\pgfqpoint{1.273726in}{1.615450in}}%
\pgfpathlineto{\pgfqpoint{1.169274in}{1.640331in}}%
\pgfpathclose%
\pgfusepath{fill}%
\end{pgfscope}%
\begin{pgfscope}%
\pgfpathrectangle{\pgfqpoint{0.000000in}{0.000000in}}{\pgfqpoint{3.000000in}{3.000000in}}%
\pgfusepath{clip}%
\pgfsetbuttcap%
\pgfsetroundjoin%
\definecolor{currentfill}{rgb}{0.999109,0.073348,0.000000}%
\pgfsetfillcolor{currentfill}%
\pgfsetlinewidth{0.000000pt}%
\definecolor{currentstroke}{rgb}{0.000000,0.000000,0.000000}%
\pgfsetstrokecolor{currentstroke}%
\pgfsetdash{}{0pt}%
\pgfpathmoveto{\pgfqpoint{1.728620in}{1.597082in}}%
\pgfpathlineto{\pgfqpoint{1.733486in}{1.603490in}}%
\pgfpathlineto{\pgfqpoint{1.843126in}{1.622843in}}%
\pgfpathlineto{\pgfqpoint{1.835510in}{1.615961in}}%
\pgfpathlineto{\pgfqpoint{1.728620in}{1.597082in}}%
\pgfpathclose%
\pgfusepath{fill}%
\end{pgfscope}%
\begin{pgfscope}%
\pgfpathrectangle{\pgfqpoint{0.000000in}{0.000000in}}{\pgfqpoint{3.000000in}{3.000000in}}%
\pgfusepath{clip}%
\pgfsetbuttcap%
\pgfsetroundjoin%
\definecolor{currentfill}{rgb}{1.000000,0.116921,0.000000}%
\pgfsetfillcolor{currentfill}%
\pgfsetlinewidth{0.000000pt}%
\definecolor{currentstroke}{rgb}{0.000000,0.000000,0.000000}%
\pgfsetstrokecolor{currentstroke}%
\pgfsetdash{}{0pt}%
\pgfpathmoveto{\pgfqpoint{1.614620in}{1.580332in}}%
\pgfpathlineto{\pgfqpoint{1.616584in}{1.586743in}}%
\pgfpathlineto{\pgfqpoint{1.728620in}{1.597082in}}%
\pgfpathlineto{\pgfqpoint{1.723767in}{1.590411in}}%
\pgfpathlineto{\pgfqpoint{1.614620in}{1.580332in}}%
\pgfpathclose%
\pgfusepath{fill}%
\end{pgfscope}%
\begin{pgfscope}%
\pgfpathrectangle{\pgfqpoint{0.000000in}{0.000000in}}{\pgfqpoint{3.000000in}{3.000000in}}%
\pgfusepath{clip}%
\pgfsetbuttcap%
\pgfsetroundjoin%
\definecolor{currentfill}{rgb}{0.856506,0.000000,0.000000}%
\pgfsetfillcolor{currentfill}%
\pgfsetlinewidth{0.000000pt}%
\definecolor{currentstroke}{rgb}{0.000000,0.000000,0.000000}%
\pgfsetstrokecolor{currentstroke}%
\pgfsetdash{}{0pt}%
\pgfpathmoveto{\pgfqpoint{1.850760in}{1.629479in}}%
\pgfpathlineto{\pgfqpoint{1.858412in}{1.635884in}}%
\pgfpathlineto{\pgfqpoint{1.964859in}{1.664754in}}%
\pgfpathlineto{\pgfqpoint{1.954675in}{1.657676in}}%
\pgfpathlineto{\pgfqpoint{1.850760in}{1.629479in}}%
\pgfpathclose%
\pgfusepath{fill}%
\end{pgfscope}%
\begin{pgfscope}%
\pgfpathrectangle{\pgfqpoint{0.000000in}{0.000000in}}{\pgfqpoint{3.000000in}{3.000000in}}%
\pgfusepath{clip}%
\pgfsetbuttcap%
\pgfsetroundjoin%
\definecolor{currentfill}{rgb}{0.678253,0.000000,0.000000}%
\pgfsetfillcolor{currentfill}%
\pgfsetlinewidth{0.000000pt}%
\definecolor{currentstroke}{rgb}{0.000000,0.000000,0.000000}%
\pgfsetstrokecolor{currentstroke}%
\pgfsetdash{}{0pt}%
\pgfpathmoveto{\pgfqpoint{1.985295in}{1.678268in}}%
\pgfpathlineto{\pgfqpoint{1.995547in}{1.684728in}}%
\pgfpathlineto{\pgfqpoint{2.097179in}{1.723849in}}%
\pgfpathlineto{\pgfqpoint{2.084688in}{1.716538in}}%
\pgfpathlineto{\pgfqpoint{1.985295in}{1.678268in}}%
\pgfpathclose%
\pgfusepath{fill}%
\end{pgfscope}%
\begin{pgfscope}%
\pgfpathrectangle{\pgfqpoint{0.000000in}{0.000000in}}{\pgfqpoint{3.000000in}{3.000000in}}%
\pgfusepath{clip}%
\pgfsetbuttcap%
\pgfsetroundjoin%
\definecolor{currentfill}{rgb}{1.000000,0.116921,0.000000}%
\pgfsetfillcolor{currentfill}%
\pgfsetlinewidth{0.000000pt}%
\definecolor{currentstroke}{rgb}{0.000000,0.000000,0.000000}%
\pgfsetstrokecolor{currentstroke}%
\pgfsetdash{}{0pt}%
\pgfpathmoveto{\pgfqpoint{1.393285in}{1.586106in}}%
\pgfpathlineto{\pgfqpoint{1.389383in}{1.592666in}}%
\pgfpathlineto{\pgfqpoint{1.502460in}{1.585256in}}%
\pgfpathlineto{\pgfqpoint{1.503444in}{1.578882in}}%
\pgfpathlineto{\pgfqpoint{1.393285in}{1.586106in}}%
\pgfpathclose%
\pgfusepath{fill}%
\end{pgfscope}%
\begin{pgfscope}%
\pgfpathrectangle{\pgfqpoint{0.000000in}{0.000000in}}{\pgfqpoint{3.000000in}{3.000000in}}%
\pgfusepath{clip}%
\pgfsetbuttcap%
\pgfsetroundjoin%
\definecolor{currentfill}{rgb}{0.731729,0.000000,0.000000}%
\pgfsetfillcolor{currentfill}%
\pgfsetlinewidth{0.000000pt}%
\definecolor{currentstroke}{rgb}{0.000000,0.000000,0.000000}%
\pgfsetstrokecolor{currentstroke}%
\pgfsetdash{}{0pt}%
\pgfpathmoveto{\pgfqpoint{1.039768in}{1.695750in}}%
\pgfpathlineto{\pgfqpoint{1.028010in}{1.702953in}}%
\pgfpathlineto{\pgfqpoint{1.131751in}{1.667250in}}%
\pgfpathlineto{\pgfqpoint{1.141164in}{1.660842in}}%
\pgfpathlineto{\pgfqpoint{1.039768in}{1.695750in}}%
\pgfpathclose%
\pgfusepath{fill}%
\end{pgfscope}%
\begin{pgfscope}%
\pgfpathrectangle{\pgfqpoint{0.000000in}{0.000000in}}{\pgfqpoint{3.000000in}{3.000000in}}%
\pgfusepath{clip}%
\pgfsetbuttcap%
\pgfsetroundjoin%
\definecolor{currentfill}{rgb}{0.999109,0.073348,0.000000}%
\pgfsetfillcolor{currentfill}%
\pgfsetlinewidth{0.000000pt}%
\definecolor{currentstroke}{rgb}{0.000000,0.000000,0.000000}%
\pgfsetstrokecolor{currentstroke}%
\pgfsetdash{}{0pt}%
\pgfpathmoveto{\pgfqpoint{1.280446in}{1.608749in}}%
\pgfpathlineto{\pgfqpoint{1.273726in}{1.615450in}}%
\pgfpathlineto{\pgfqpoint{1.385470in}{1.598963in}}%
\pgfpathlineto{\pgfqpoint{1.389383in}{1.592666in}}%
\pgfpathlineto{\pgfqpoint{1.280446in}{1.608749in}}%
\pgfpathclose%
\pgfusepath{fill}%
\end{pgfscope}%
\begin{pgfscope}%
\pgfpathrectangle{\pgfqpoint{0.000000in}{0.000000in}}{\pgfqpoint{3.000000in}{3.000000in}}%
\pgfusepath{clip}%
\pgfsetbuttcap%
\pgfsetroundjoin%
\definecolor{currentfill}{rgb}{1.000000,0.116921,0.000000}%
\pgfsetfillcolor{currentfill}%
\pgfsetlinewidth{0.000000pt}%
\definecolor{currentstroke}{rgb}{0.000000,0.000000,0.000000}%
\pgfsetstrokecolor{currentstroke}%
\pgfsetdash{}{0pt}%
\pgfpathmoveto{\pgfqpoint{1.503444in}{1.578882in}}%
\pgfpathlineto{\pgfqpoint{1.502460in}{1.585256in}}%
\pgfpathlineto{\pgfqpoint{1.616584in}{1.586743in}}%
\pgfpathlineto{\pgfqpoint{1.614620in}{1.580332in}}%
\pgfpathlineto{\pgfqpoint{1.503444in}{1.578882in}}%
\pgfpathclose%
\pgfusepath{fill}%
\end{pgfscope}%
\begin{pgfscope}%
\pgfpathrectangle{\pgfqpoint{0.000000in}{0.000000in}}{\pgfqpoint{3.000000in}{3.000000in}}%
\pgfusepath{clip}%
\pgfsetbuttcap%
\pgfsetroundjoin%
\definecolor{currentfill}{rgb}{0.500000,0.000000,0.000000}%
\pgfsetfillcolor{currentfill}%
\pgfsetlinewidth{0.000000pt}%
\definecolor{currentstroke}{rgb}{0.000000,0.000000,0.000000}%
\pgfsetstrokecolor{currentstroke}%
\pgfsetdash{}{0pt}%
\pgfpathmoveto{\pgfqpoint{0.896713in}{1.769316in}}%
\pgfpathlineto{\pgfqpoint{0.882852in}{1.776757in}}%
\pgfpathlineto{\pgfqpoint{0.980729in}{1.729927in}}%
\pgfpathlineto{\pgfqpoint{0.992587in}{1.723440in}}%
\pgfpathlineto{\pgfqpoint{0.896713in}{1.769316in}}%
\pgfpathclose%
\pgfusepath{fill}%
\end{pgfscope}%
\begin{pgfscope}%
\pgfpathrectangle{\pgfqpoint{0.000000in}{0.000000in}}{\pgfqpoint{3.000000in}{3.000000in}}%
\pgfusepath{clip}%
\pgfsetbuttcap%
\pgfsetroundjoin%
\definecolor{currentfill}{rgb}{0.856506,0.000000,0.000000}%
\pgfsetfillcolor{currentfill}%
\pgfsetlinewidth{0.000000pt}%
\definecolor{currentstroke}{rgb}{0.000000,0.000000,0.000000}%
\pgfsetstrokecolor{currentstroke}%
\pgfsetdash{}{0pt}%
\pgfpathmoveto{\pgfqpoint{1.159926in}{1.647394in}}%
\pgfpathlineto{\pgfqpoint{1.150556in}{1.654227in}}%
\pgfpathlineto{\pgfqpoint{1.260236in}{1.628128in}}%
\pgfpathlineto{\pgfqpoint{1.266989in}{1.621905in}}%
\pgfpathlineto{\pgfqpoint{1.159926in}{1.647394in}}%
\pgfpathclose%
\pgfusepath{fill}%
\end{pgfscope}%
\begin{pgfscope}%
\pgfpathrectangle{\pgfqpoint{0.000000in}{0.000000in}}{\pgfqpoint{3.000000in}{3.000000in}}%
\pgfusepath{clip}%
\pgfsetbuttcap%
\pgfsetroundjoin%
\definecolor{currentfill}{rgb}{0.927807,0.015251,0.000000}%
\pgfsetfillcolor{currentfill}%
\pgfsetlinewidth{0.000000pt}%
\definecolor{currentstroke}{rgb}{0.000000,0.000000,0.000000}%
\pgfsetstrokecolor{currentstroke}%
\pgfsetdash{}{0pt}%
\pgfpathmoveto{\pgfqpoint{1.733486in}{1.603490in}}%
\pgfpathlineto{\pgfqpoint{1.738365in}{1.609652in}}%
\pgfpathlineto{\pgfqpoint{1.850760in}{1.629479in}}%
\pgfpathlineto{\pgfqpoint{1.843126in}{1.622843in}}%
\pgfpathlineto{\pgfqpoint{1.733486in}{1.603490in}}%
\pgfpathclose%
\pgfusepath{fill}%
\end{pgfscope}%
\begin{pgfscope}%
\pgfpathrectangle{\pgfqpoint{0.000000in}{0.000000in}}{\pgfqpoint{3.000000in}{3.000000in}}%
\pgfusepath{clip}%
\pgfsetbuttcap%
\pgfsetroundjoin%
\definecolor{currentfill}{rgb}{0.803030,0.000000,0.000000}%
\pgfsetfillcolor{currentfill}%
\pgfsetlinewidth{0.000000pt}%
\definecolor{currentstroke}{rgb}{0.000000,0.000000,0.000000}%
\pgfsetstrokecolor{currentstroke}%
\pgfsetdash{}{0pt}%
\pgfpathmoveto{\pgfqpoint{1.858412in}{1.635884in}}%
\pgfpathlineto{\pgfqpoint{1.866082in}{1.642069in}}%
\pgfpathlineto{\pgfqpoint{1.975065in}{1.671614in}}%
\pgfpathlineto{\pgfqpoint{1.964859in}{1.664754in}}%
\pgfpathlineto{\pgfqpoint{1.858412in}{1.635884in}}%
\pgfpathclose%
\pgfusepath{fill}%
\end{pgfscope}%
\begin{pgfscope}%
\pgfpathrectangle{\pgfqpoint{0.000000in}{0.000000in}}{\pgfqpoint{3.000000in}{3.000000in}}%
\pgfusepath{clip}%
\pgfsetbuttcap%
\pgfsetroundjoin%
\definecolor{currentfill}{rgb}{0.999109,0.073348,0.000000}%
\pgfsetfillcolor{currentfill}%
\pgfsetlinewidth{0.000000pt}%
\definecolor{currentstroke}{rgb}{0.000000,0.000000,0.000000}%
\pgfsetstrokecolor{currentstroke}%
\pgfsetdash{}{0pt}%
\pgfpathmoveto{\pgfqpoint{1.616584in}{1.586743in}}%
\pgfpathlineto{\pgfqpoint{1.618554in}{1.592891in}}%
\pgfpathlineto{\pgfqpoint{1.733486in}{1.603490in}}%
\pgfpathlineto{\pgfqpoint{1.728620in}{1.597082in}}%
\pgfpathlineto{\pgfqpoint{1.616584in}{1.586743in}}%
\pgfpathclose%
\pgfusepath{fill}%
\end{pgfscope}%
\begin{pgfscope}%
\pgfpathrectangle{\pgfqpoint{0.000000in}{0.000000in}}{\pgfqpoint{3.000000in}{3.000000in}}%
\pgfusepath{clip}%
\pgfsetbuttcap%
\pgfsetroundjoin%
\definecolor{currentfill}{rgb}{0.606952,0.000000,0.000000}%
\pgfsetfillcolor{currentfill}%
\pgfsetlinewidth{0.000000pt}%
\definecolor{currentstroke}{rgb}{0.000000,0.000000,0.000000}%
\pgfsetstrokecolor{currentstroke}%
\pgfsetdash{}{0pt}%
\pgfpathmoveto{\pgfqpoint{1.995547in}{1.684728in}}%
\pgfpathlineto{\pgfqpoint{2.005823in}{1.691003in}}%
\pgfpathlineto{\pgfqpoint{2.109695in}{1.730977in}}%
\pgfpathlineto{\pgfqpoint{2.097179in}{1.723849in}}%
\pgfpathlineto{\pgfqpoint{1.995547in}{1.684728in}}%
\pgfpathclose%
\pgfusepath{fill}%
\end{pgfscope}%
\begin{pgfscope}%
\pgfpathrectangle{\pgfqpoint{0.000000in}{0.000000in}}{\pgfqpoint{3.000000in}{3.000000in}}%
\pgfusepath{clip}%
\pgfsetbuttcap%
\pgfsetroundjoin%
\definecolor{currentfill}{rgb}{0.678253,0.000000,0.000000}%
\pgfsetfillcolor{currentfill}%
\pgfsetlinewidth{0.000000pt}%
\definecolor{currentstroke}{rgb}{0.000000,0.000000,0.000000}%
\pgfsetstrokecolor{currentstroke}%
\pgfsetdash{}{0pt}%
\pgfpathmoveto{\pgfqpoint{1.028010in}{1.702953in}}%
\pgfpathlineto{\pgfqpoint{1.016228in}{1.709962in}}%
\pgfpathlineto{\pgfqpoint{1.122316in}{1.673463in}}%
\pgfpathlineto{\pgfqpoint{1.131751in}{1.667250in}}%
\pgfpathlineto{\pgfqpoint{1.028010in}{1.702953in}}%
\pgfpathclose%
\pgfusepath{fill}%
\end{pgfscope}%
\begin{pgfscope}%
\pgfpathrectangle{\pgfqpoint{0.000000in}{0.000000in}}{\pgfqpoint{3.000000in}{3.000000in}}%
\pgfusepath{clip}%
\pgfsetbuttcap%
\pgfsetroundjoin%
\definecolor{currentfill}{rgb}{0.999109,0.073348,0.000000}%
\pgfsetfillcolor{currentfill}%
\pgfsetlinewidth{0.000000pt}%
\definecolor{currentstroke}{rgb}{0.000000,0.000000,0.000000}%
\pgfsetstrokecolor{currentstroke}%
\pgfsetdash{}{0pt}%
\pgfpathmoveto{\pgfqpoint{1.389383in}{1.592666in}}%
\pgfpathlineto{\pgfqpoint{1.385470in}{1.598963in}}%
\pgfpathlineto{\pgfqpoint{1.501474in}{1.591366in}}%
\pgfpathlineto{\pgfqpoint{1.502460in}{1.585256in}}%
\pgfpathlineto{\pgfqpoint{1.389383in}{1.592666in}}%
\pgfpathclose%
\pgfusepath{fill}%
\end{pgfscope}%
\begin{pgfscope}%
\pgfpathrectangle{\pgfqpoint{0.000000in}{0.000000in}}{\pgfqpoint{3.000000in}{3.000000in}}%
\pgfusepath{clip}%
\pgfsetbuttcap%
\pgfsetroundjoin%
\definecolor{currentfill}{rgb}{0.927807,0.015251,0.000000}%
\pgfsetfillcolor{currentfill}%
\pgfsetlinewidth{0.000000pt}%
\definecolor{currentstroke}{rgb}{0.000000,0.000000,0.000000}%
\pgfsetstrokecolor{currentstroke}%
\pgfsetdash{}{0pt}%
\pgfpathmoveto{\pgfqpoint{1.273726in}{1.615450in}}%
\pgfpathlineto{\pgfqpoint{1.266989in}{1.621905in}}%
\pgfpathlineto{\pgfqpoint{1.381547in}{1.605013in}}%
\pgfpathlineto{\pgfqpoint{1.385470in}{1.598963in}}%
\pgfpathlineto{\pgfqpoint{1.273726in}{1.615450in}}%
\pgfpathclose%
\pgfusepath{fill}%
\end{pgfscope}%
\begin{pgfscope}%
\pgfpathrectangle{\pgfqpoint{0.000000in}{0.000000in}}{\pgfqpoint{3.000000in}{3.000000in}}%
\pgfusepath{clip}%
\pgfsetbuttcap%
\pgfsetroundjoin%
\definecolor{currentfill}{rgb}{0.999109,0.073348,0.000000}%
\pgfsetfillcolor{currentfill}%
\pgfsetlinewidth{0.000000pt}%
\definecolor{currentstroke}{rgb}{0.000000,0.000000,0.000000}%
\pgfsetstrokecolor{currentstroke}%
\pgfsetdash{}{0pt}%
\pgfpathmoveto{\pgfqpoint{1.502460in}{1.585256in}}%
\pgfpathlineto{\pgfqpoint{1.501474in}{1.591366in}}%
\pgfpathlineto{\pgfqpoint{1.618554in}{1.592891in}}%
\pgfpathlineto{\pgfqpoint{1.616584in}{1.586743in}}%
\pgfpathlineto{\pgfqpoint{1.502460in}{1.585256in}}%
\pgfpathclose%
\pgfusepath{fill}%
\end{pgfscope}%
\begin{pgfscope}%
\pgfpathrectangle{\pgfqpoint{0.000000in}{0.000000in}}{\pgfqpoint{3.000000in}{3.000000in}}%
\pgfusepath{clip}%
\pgfsetbuttcap%
\pgfsetroundjoin%
\definecolor{currentfill}{rgb}{0.803030,0.000000,0.000000}%
\pgfsetfillcolor{currentfill}%
\pgfsetlinewidth{0.000000pt}%
\definecolor{currentstroke}{rgb}{0.000000,0.000000,0.000000}%
\pgfsetstrokecolor{currentstroke}%
\pgfsetdash{}{0pt}%
\pgfpathmoveto{\pgfqpoint{1.150556in}{1.654227in}}%
\pgfpathlineto{\pgfqpoint{1.141164in}{1.660842in}}%
\pgfpathlineto{\pgfqpoint{1.253466in}{1.634131in}}%
\pgfpathlineto{\pgfqpoint{1.260236in}{1.628128in}}%
\pgfpathlineto{\pgfqpoint{1.150556in}{1.654227in}}%
\pgfpathclose%
\pgfusepath{fill}%
\end{pgfscope}%
\begin{pgfscope}%
\pgfpathrectangle{\pgfqpoint{0.000000in}{0.000000in}}{\pgfqpoint{3.000000in}{3.000000in}}%
\pgfusepath{clip}%
\pgfsetbuttcap%
\pgfsetroundjoin%
\definecolor{currentfill}{rgb}{0.856506,0.000000,0.000000}%
\pgfsetfillcolor{currentfill}%
\pgfsetlinewidth{0.000000pt}%
\definecolor{currentstroke}{rgb}{0.000000,0.000000,0.000000}%
\pgfsetstrokecolor{currentstroke}%
\pgfsetdash{}{0pt}%
\pgfpathmoveto{\pgfqpoint{1.738365in}{1.609652in}}%
\pgfpathlineto{\pgfqpoint{1.743255in}{1.615580in}}%
\pgfpathlineto{\pgfqpoint{1.858412in}{1.635884in}}%
\pgfpathlineto{\pgfqpoint{1.850760in}{1.629479in}}%
\pgfpathlineto{\pgfqpoint{1.738365in}{1.609652in}}%
\pgfpathclose%
\pgfusepath{fill}%
\end{pgfscope}%
\begin{pgfscope}%
\pgfpathrectangle{\pgfqpoint{0.000000in}{0.000000in}}{\pgfqpoint{3.000000in}{3.000000in}}%
\pgfusepath{clip}%
\pgfsetbuttcap%
\pgfsetroundjoin%
\definecolor{currentfill}{rgb}{0.731729,0.000000,0.000000}%
\pgfsetfillcolor{currentfill}%
\pgfsetlinewidth{0.000000pt}%
\definecolor{currentstroke}{rgb}{0.000000,0.000000,0.000000}%
\pgfsetstrokecolor{currentstroke}%
\pgfsetdash{}{0pt}%
\pgfpathmoveto{\pgfqpoint{1.866082in}{1.642069in}}%
\pgfpathlineto{\pgfqpoint{1.873771in}{1.648047in}}%
\pgfpathlineto{\pgfqpoint{1.985295in}{1.678268in}}%
\pgfpathlineto{\pgfqpoint{1.975065in}{1.671614in}}%
\pgfpathlineto{\pgfqpoint{1.866082in}{1.642069in}}%
\pgfpathclose%
\pgfusepath{fill}%
\end{pgfscope}%
\begin{pgfscope}%
\pgfpathrectangle{\pgfqpoint{0.000000in}{0.000000in}}{\pgfqpoint{3.000000in}{3.000000in}}%
\pgfusepath{clip}%
\pgfsetbuttcap%
\pgfsetroundjoin%
\definecolor{currentfill}{rgb}{0.553476,0.000000,0.000000}%
\pgfsetfillcolor{currentfill}%
\pgfsetlinewidth{0.000000pt}%
\definecolor{currentstroke}{rgb}{0.000000,0.000000,0.000000}%
\pgfsetstrokecolor{currentstroke}%
\pgfsetdash{}{0pt}%
\pgfpathmoveto{\pgfqpoint{2.005823in}{1.691003in}}%
\pgfpathlineto{\pgfqpoint{2.016122in}{1.697103in}}%
\pgfpathlineto{\pgfqpoint{2.122237in}{1.737931in}}%
\pgfpathlineto{\pgfqpoint{2.109695in}{1.730977in}}%
\pgfpathlineto{\pgfqpoint{2.005823in}{1.691003in}}%
\pgfpathclose%
\pgfusepath{fill}%
\end{pgfscope}%
\begin{pgfscope}%
\pgfpathrectangle{\pgfqpoint{0.000000in}{0.000000in}}{\pgfqpoint{3.000000in}{3.000000in}}%
\pgfusepath{clip}%
\pgfsetbuttcap%
\pgfsetroundjoin%
\definecolor{currentfill}{rgb}{0.927807,0.015251,0.000000}%
\pgfsetfillcolor{currentfill}%
\pgfsetlinewidth{0.000000pt}%
\definecolor{currentstroke}{rgb}{0.000000,0.000000,0.000000}%
\pgfsetstrokecolor{currentstroke}%
\pgfsetdash{}{0pt}%
\pgfpathmoveto{\pgfqpoint{1.618554in}{1.592891in}}%
\pgfpathlineto{\pgfqpoint{1.620528in}{1.598791in}}%
\pgfpathlineto{\pgfqpoint{1.738365in}{1.609652in}}%
\pgfpathlineto{\pgfqpoint{1.733486in}{1.603490in}}%
\pgfpathlineto{\pgfqpoint{1.618554in}{1.592891in}}%
\pgfpathclose%
\pgfusepath{fill}%
\end{pgfscope}%
\begin{pgfscope}%
\pgfpathrectangle{\pgfqpoint{0.000000in}{0.000000in}}{\pgfqpoint{3.000000in}{3.000000in}}%
\pgfusepath{clip}%
\pgfsetbuttcap%
\pgfsetroundjoin%
\definecolor{currentfill}{rgb}{0.606952,0.000000,0.000000}%
\pgfsetfillcolor{currentfill}%
\pgfsetlinewidth{0.000000pt}%
\definecolor{currentstroke}{rgb}{0.000000,0.000000,0.000000}%
\pgfsetstrokecolor{currentstroke}%
\pgfsetdash{}{0pt}%
\pgfpathmoveto{\pgfqpoint{1.016228in}{1.709962in}}%
\pgfpathlineto{\pgfqpoint{1.004420in}{1.716788in}}%
\pgfpathlineto{\pgfqpoint{1.112859in}{1.679492in}}%
\pgfpathlineto{\pgfqpoint{1.122316in}{1.673463in}}%
\pgfpathlineto{\pgfqpoint{1.016228in}{1.709962in}}%
\pgfpathclose%
\pgfusepath{fill}%
\end{pgfscope}%
\begin{pgfscope}%
\pgfpathrectangle{\pgfqpoint{0.000000in}{0.000000in}}{\pgfqpoint{3.000000in}{3.000000in}}%
\pgfusepath{clip}%
\pgfsetbuttcap%
\pgfsetroundjoin%
\definecolor{currentfill}{rgb}{0.927807,0.015251,0.000000}%
\pgfsetfillcolor{currentfill}%
\pgfsetlinewidth{0.000000pt}%
\definecolor{currentstroke}{rgb}{0.000000,0.000000,0.000000}%
\pgfsetstrokecolor{currentstroke}%
\pgfsetdash{}{0pt}%
\pgfpathmoveto{\pgfqpoint{1.385470in}{1.598963in}}%
\pgfpathlineto{\pgfqpoint{1.381547in}{1.605013in}}%
\pgfpathlineto{\pgfqpoint{1.500485in}{1.597229in}}%
\pgfpathlineto{\pgfqpoint{1.501474in}{1.591366in}}%
\pgfpathlineto{\pgfqpoint{1.385470in}{1.598963in}}%
\pgfpathclose%
\pgfusepath{fill}%
\end{pgfscope}%
\begin{pgfscope}%
\pgfpathrectangle{\pgfqpoint{0.000000in}{0.000000in}}{\pgfqpoint{3.000000in}{3.000000in}}%
\pgfusepath{clip}%
\pgfsetbuttcap%
\pgfsetroundjoin%
\definecolor{currentfill}{rgb}{0.856506,0.000000,0.000000}%
\pgfsetfillcolor{currentfill}%
\pgfsetlinewidth{0.000000pt}%
\definecolor{currentstroke}{rgb}{0.000000,0.000000,0.000000}%
\pgfsetstrokecolor{currentstroke}%
\pgfsetdash{}{0pt}%
\pgfpathmoveto{\pgfqpoint{1.266989in}{1.621905in}}%
\pgfpathlineto{\pgfqpoint{1.260236in}{1.628128in}}%
\pgfpathlineto{\pgfqpoint{1.377615in}{1.610829in}}%
\pgfpathlineto{\pgfqpoint{1.381547in}{1.605013in}}%
\pgfpathlineto{\pgfqpoint{1.266989in}{1.621905in}}%
\pgfpathclose%
\pgfusepath{fill}%
\end{pgfscope}%
\begin{pgfscope}%
\pgfpathrectangle{\pgfqpoint{0.000000in}{0.000000in}}{\pgfqpoint{3.000000in}{3.000000in}}%
\pgfusepath{clip}%
\pgfsetbuttcap%
\pgfsetroundjoin%
\definecolor{currentfill}{rgb}{0.927807,0.015251,0.000000}%
\pgfsetfillcolor{currentfill}%
\pgfsetlinewidth{0.000000pt}%
\definecolor{currentstroke}{rgb}{0.000000,0.000000,0.000000}%
\pgfsetstrokecolor{currentstroke}%
\pgfsetdash{}{0pt}%
\pgfpathmoveto{\pgfqpoint{1.501474in}{1.591366in}}%
\pgfpathlineto{\pgfqpoint{1.500485in}{1.597229in}}%
\pgfpathlineto{\pgfqpoint{1.620528in}{1.598791in}}%
\pgfpathlineto{\pgfqpoint{1.618554in}{1.592891in}}%
\pgfpathlineto{\pgfqpoint{1.501474in}{1.591366in}}%
\pgfpathclose%
\pgfusepath{fill}%
\end{pgfscope}%
\begin{pgfscope}%
\pgfpathrectangle{\pgfqpoint{0.000000in}{0.000000in}}{\pgfqpoint{3.000000in}{3.000000in}}%
\pgfusepath{clip}%
\pgfsetbuttcap%
\pgfsetroundjoin%
\definecolor{currentfill}{rgb}{0.731729,0.000000,0.000000}%
\pgfsetfillcolor{currentfill}%
\pgfsetlinewidth{0.000000pt}%
\definecolor{currentstroke}{rgb}{0.000000,0.000000,0.000000}%
\pgfsetstrokecolor{currentstroke}%
\pgfsetdash{}{0pt}%
\pgfpathmoveto{\pgfqpoint{1.141164in}{1.660842in}}%
\pgfpathlineto{\pgfqpoint{1.131751in}{1.667250in}}%
\pgfpathlineto{\pgfqpoint{1.246680in}{1.639927in}}%
\pgfpathlineto{\pgfqpoint{1.253466in}{1.634131in}}%
\pgfpathlineto{\pgfqpoint{1.141164in}{1.660842in}}%
\pgfpathclose%
\pgfusepath{fill}%
\end{pgfscope}%
\begin{pgfscope}%
\pgfpathrectangle{\pgfqpoint{0.000000in}{0.000000in}}{\pgfqpoint{3.000000in}{3.000000in}}%
\pgfusepath{clip}%
\pgfsetbuttcap%
\pgfsetroundjoin%
\definecolor{currentfill}{rgb}{0.500000,0.000000,0.000000}%
\pgfsetfillcolor{currentfill}%
\pgfsetlinewidth{0.000000pt}%
\definecolor{currentstroke}{rgb}{0.000000,0.000000,0.000000}%
\pgfsetstrokecolor{currentstroke}%
\pgfsetdash{}{0pt}%
\pgfpathmoveto{\pgfqpoint{2.016122in}{1.697103in}}%
\pgfpathlineto{\pgfqpoint{2.026444in}{1.703037in}}%
\pgfpathlineto{\pgfqpoint{2.134805in}{1.744720in}}%
\pgfpathlineto{\pgfqpoint{2.122237in}{1.737931in}}%
\pgfpathlineto{\pgfqpoint{2.016122in}{1.697103in}}%
\pgfpathclose%
\pgfusepath{fill}%
\end{pgfscope}%
\begin{pgfscope}%
\pgfpathrectangle{\pgfqpoint{0.000000in}{0.000000in}}{\pgfqpoint{3.000000in}{3.000000in}}%
\pgfusepath{clip}%
\pgfsetbuttcap%
\pgfsetroundjoin%
\definecolor{currentfill}{rgb}{0.678253,0.000000,0.000000}%
\pgfsetfillcolor{currentfill}%
\pgfsetlinewidth{0.000000pt}%
\definecolor{currentstroke}{rgb}{0.000000,0.000000,0.000000}%
\pgfsetstrokecolor{currentstroke}%
\pgfsetdash{}{0pt}%
\pgfpathmoveto{\pgfqpoint{1.873771in}{1.648047in}}%
\pgfpathlineto{\pgfqpoint{1.881479in}{1.653830in}}%
\pgfpathlineto{\pgfqpoint{1.995547in}{1.684728in}}%
\pgfpathlineto{\pgfqpoint{1.985295in}{1.678268in}}%
\pgfpathlineto{\pgfqpoint{1.873771in}{1.648047in}}%
\pgfpathclose%
\pgfusepath{fill}%
\end{pgfscope}%
\begin{pgfscope}%
\pgfpathrectangle{\pgfqpoint{0.000000in}{0.000000in}}{\pgfqpoint{3.000000in}{3.000000in}}%
\pgfusepath{clip}%
\pgfsetbuttcap%
\pgfsetroundjoin%
\definecolor{currentfill}{rgb}{0.803030,0.000000,0.000000}%
\pgfsetfillcolor{currentfill}%
\pgfsetlinewidth{0.000000pt}%
\definecolor{currentstroke}{rgb}{0.000000,0.000000,0.000000}%
\pgfsetstrokecolor{currentstroke}%
\pgfsetdash{}{0pt}%
\pgfpathmoveto{\pgfqpoint{1.743255in}{1.615580in}}%
\pgfpathlineto{\pgfqpoint{1.748158in}{1.621288in}}%
\pgfpathlineto{\pgfqpoint{1.866082in}{1.642069in}}%
\pgfpathlineto{\pgfqpoint{1.858412in}{1.635884in}}%
\pgfpathlineto{\pgfqpoint{1.743255in}{1.615580in}}%
\pgfpathclose%
\pgfusepath{fill}%
\end{pgfscope}%
\begin{pgfscope}%
\pgfpathrectangle{\pgfqpoint{0.000000in}{0.000000in}}{\pgfqpoint{3.000000in}{3.000000in}}%
\pgfusepath{clip}%
\pgfsetbuttcap%
\pgfsetroundjoin%
\definecolor{currentfill}{rgb}{0.553476,0.000000,0.000000}%
\pgfsetfillcolor{currentfill}%
\pgfsetlinewidth{0.000000pt}%
\definecolor{currentstroke}{rgb}{0.000000,0.000000,0.000000}%
\pgfsetstrokecolor{currentstroke}%
\pgfsetdash{}{0pt}%
\pgfpathmoveto{\pgfqpoint{1.004420in}{1.716788in}}%
\pgfpathlineto{\pgfqpoint{0.992587in}{1.723440in}}%
\pgfpathlineto{\pgfqpoint{1.103381in}{1.685345in}}%
\pgfpathlineto{\pgfqpoint{1.112859in}{1.679492in}}%
\pgfpathlineto{\pgfqpoint{1.004420in}{1.716788in}}%
\pgfpathclose%
\pgfusepath{fill}%
\end{pgfscope}%
\begin{pgfscope}%
\pgfpathrectangle{\pgfqpoint{0.000000in}{0.000000in}}{\pgfqpoint{3.000000in}{3.000000in}}%
\pgfusepath{clip}%
\pgfsetbuttcap%
\pgfsetroundjoin%
\definecolor{currentfill}{rgb}{0.856506,0.000000,0.000000}%
\pgfsetfillcolor{currentfill}%
\pgfsetlinewidth{0.000000pt}%
\definecolor{currentstroke}{rgb}{0.000000,0.000000,0.000000}%
\pgfsetstrokecolor{currentstroke}%
\pgfsetdash{}{0pt}%
\pgfpathmoveto{\pgfqpoint{1.620528in}{1.598791in}}%
\pgfpathlineto{\pgfqpoint{1.622508in}{1.604458in}}%
\pgfpathlineto{\pgfqpoint{1.743255in}{1.615580in}}%
\pgfpathlineto{\pgfqpoint{1.738365in}{1.609652in}}%
\pgfpathlineto{\pgfqpoint{1.620528in}{1.598791in}}%
\pgfpathclose%
\pgfusepath{fill}%
\end{pgfscope}%
\begin{pgfscope}%
\pgfpathrectangle{\pgfqpoint{0.000000in}{0.000000in}}{\pgfqpoint{3.000000in}{3.000000in}}%
\pgfusepath{clip}%
\pgfsetbuttcap%
\pgfsetroundjoin%
\definecolor{currentfill}{rgb}{0.856506,0.000000,0.000000}%
\pgfsetfillcolor{currentfill}%
\pgfsetlinewidth{0.000000pt}%
\definecolor{currentstroke}{rgb}{0.000000,0.000000,0.000000}%
\pgfsetstrokecolor{currentstroke}%
\pgfsetdash{}{0pt}%
\pgfpathmoveto{\pgfqpoint{1.381547in}{1.605013in}}%
\pgfpathlineto{\pgfqpoint{1.377615in}{1.610829in}}%
\pgfpathlineto{\pgfqpoint{1.499493in}{1.602858in}}%
\pgfpathlineto{\pgfqpoint{1.500485in}{1.597229in}}%
\pgfpathlineto{\pgfqpoint{1.381547in}{1.605013in}}%
\pgfpathclose%
\pgfusepath{fill}%
\end{pgfscope}%
\begin{pgfscope}%
\pgfpathrectangle{\pgfqpoint{0.000000in}{0.000000in}}{\pgfqpoint{3.000000in}{3.000000in}}%
\pgfusepath{clip}%
\pgfsetbuttcap%
\pgfsetroundjoin%
\definecolor{currentfill}{rgb}{0.803030,0.000000,0.000000}%
\pgfsetfillcolor{currentfill}%
\pgfsetlinewidth{0.000000pt}%
\definecolor{currentstroke}{rgb}{0.000000,0.000000,0.000000}%
\pgfsetstrokecolor{currentstroke}%
\pgfsetdash{}{0pt}%
\pgfpathmoveto{\pgfqpoint{1.260236in}{1.628128in}}%
\pgfpathlineto{\pgfqpoint{1.253466in}{1.634131in}}%
\pgfpathlineto{\pgfqpoint{1.373672in}{1.616426in}}%
\pgfpathlineto{\pgfqpoint{1.377615in}{1.610829in}}%
\pgfpathlineto{\pgfqpoint{1.260236in}{1.628128in}}%
\pgfpathclose%
\pgfusepath{fill}%
\end{pgfscope}%
\begin{pgfscope}%
\pgfpathrectangle{\pgfqpoint{0.000000in}{0.000000in}}{\pgfqpoint{3.000000in}{3.000000in}}%
\pgfusepath{clip}%
\pgfsetbuttcap%
\pgfsetroundjoin%
\definecolor{currentfill}{rgb}{0.856506,0.000000,0.000000}%
\pgfsetfillcolor{currentfill}%
\pgfsetlinewidth{0.000000pt}%
\definecolor{currentstroke}{rgb}{0.000000,0.000000,0.000000}%
\pgfsetstrokecolor{currentstroke}%
\pgfsetdash{}{0pt}%
\pgfpathmoveto{\pgfqpoint{1.500485in}{1.597229in}}%
\pgfpathlineto{\pgfqpoint{1.499493in}{1.602858in}}%
\pgfpathlineto{\pgfqpoint{1.622508in}{1.604458in}}%
\pgfpathlineto{\pgfqpoint{1.620528in}{1.598791in}}%
\pgfpathlineto{\pgfqpoint{1.500485in}{1.597229in}}%
\pgfpathclose%
\pgfusepath{fill}%
\end{pgfscope}%
\begin{pgfscope}%
\pgfpathrectangle{\pgfqpoint{0.000000in}{0.000000in}}{\pgfqpoint{3.000000in}{3.000000in}}%
\pgfusepath{clip}%
\pgfsetbuttcap%
\pgfsetroundjoin%
\definecolor{currentfill}{rgb}{0.678253,0.000000,0.000000}%
\pgfsetfillcolor{currentfill}%
\pgfsetlinewidth{0.000000pt}%
\definecolor{currentstroke}{rgb}{0.000000,0.000000,0.000000}%
\pgfsetstrokecolor{currentstroke}%
\pgfsetdash{}{0pt}%
\pgfpathmoveto{\pgfqpoint{1.131751in}{1.667250in}}%
\pgfpathlineto{\pgfqpoint{1.122316in}{1.673463in}}%
\pgfpathlineto{\pgfqpoint{1.239877in}{1.645527in}}%
\pgfpathlineto{\pgfqpoint{1.246680in}{1.639927in}}%
\pgfpathlineto{\pgfqpoint{1.131751in}{1.667250in}}%
\pgfpathclose%
\pgfusepath{fill}%
\end{pgfscope}%
\begin{pgfscope}%
\pgfpathrectangle{\pgfqpoint{0.000000in}{0.000000in}}{\pgfqpoint{3.000000in}{3.000000in}}%
\pgfusepath{clip}%
\pgfsetbuttcap%
\pgfsetroundjoin%
\definecolor{currentfill}{rgb}{0.606952,0.000000,0.000000}%
\pgfsetfillcolor{currentfill}%
\pgfsetlinewidth{0.000000pt}%
\definecolor{currentstroke}{rgb}{0.000000,0.000000,0.000000}%
\pgfsetstrokecolor{currentstroke}%
\pgfsetdash{}{0pt}%
\pgfpathmoveto{\pgfqpoint{1.881479in}{1.653830in}}%
\pgfpathlineto{\pgfqpoint{1.889205in}{1.659426in}}%
\pgfpathlineto{\pgfqpoint{2.005823in}{1.691003in}}%
\pgfpathlineto{\pgfqpoint{1.995547in}{1.684728in}}%
\pgfpathlineto{\pgfqpoint{1.881479in}{1.653830in}}%
\pgfpathclose%
\pgfusepath{fill}%
\end{pgfscope}%
\begin{pgfscope}%
\pgfpathrectangle{\pgfqpoint{0.000000in}{0.000000in}}{\pgfqpoint{3.000000in}{3.000000in}}%
\pgfusepath{clip}%
\pgfsetbuttcap%
\pgfsetroundjoin%
\definecolor{currentfill}{rgb}{0.500000,0.000000,0.000000}%
\pgfsetfillcolor{currentfill}%
\pgfsetlinewidth{0.000000pt}%
\definecolor{currentstroke}{rgb}{0.000000,0.000000,0.000000}%
\pgfsetstrokecolor{currentstroke}%
\pgfsetdash{}{0pt}%
\pgfpathmoveto{\pgfqpoint{0.992587in}{1.723440in}}%
\pgfpathlineto{\pgfqpoint{0.980729in}{1.729927in}}%
\pgfpathlineto{\pgfqpoint{1.093880in}{1.691031in}}%
\pgfpathlineto{\pgfqpoint{1.103381in}{1.685345in}}%
\pgfpathlineto{\pgfqpoint{0.992587in}{1.723440in}}%
\pgfpathclose%
\pgfusepath{fill}%
\end{pgfscope}%
\begin{pgfscope}%
\pgfpathrectangle{\pgfqpoint{0.000000in}{0.000000in}}{\pgfqpoint{3.000000in}{3.000000in}}%
\pgfusepath{clip}%
\pgfsetbuttcap%
\pgfsetroundjoin%
\definecolor{currentfill}{rgb}{0.731729,0.000000,0.000000}%
\pgfsetfillcolor{currentfill}%
\pgfsetlinewidth{0.000000pt}%
\definecolor{currentstroke}{rgb}{0.000000,0.000000,0.000000}%
\pgfsetstrokecolor{currentstroke}%
\pgfsetdash{}{0pt}%
\pgfpathmoveto{\pgfqpoint{1.748158in}{1.621288in}}%
\pgfpathlineto{\pgfqpoint{1.753073in}{1.626788in}}%
\pgfpathlineto{\pgfqpoint{1.873771in}{1.648047in}}%
\pgfpathlineto{\pgfqpoint{1.866082in}{1.642069in}}%
\pgfpathlineto{\pgfqpoint{1.748158in}{1.621288in}}%
\pgfpathclose%
\pgfusepath{fill}%
\end{pgfscope}%
\begin{pgfscope}%
\pgfpathrectangle{\pgfqpoint{0.000000in}{0.000000in}}{\pgfqpoint{3.000000in}{3.000000in}}%
\pgfusepath{clip}%
\pgfsetbuttcap%
\pgfsetroundjoin%
\definecolor{currentfill}{rgb}{0.803030,0.000000,0.000000}%
\pgfsetfillcolor{currentfill}%
\pgfsetlinewidth{0.000000pt}%
\definecolor{currentstroke}{rgb}{0.000000,0.000000,0.000000}%
\pgfsetstrokecolor{currentstroke}%
\pgfsetdash{}{0pt}%
\pgfpathmoveto{\pgfqpoint{1.622508in}{1.604458in}}%
\pgfpathlineto{\pgfqpoint{1.624493in}{1.609903in}}%
\pgfpathlineto{\pgfqpoint{1.748158in}{1.621288in}}%
\pgfpathlineto{\pgfqpoint{1.743255in}{1.615580in}}%
\pgfpathlineto{\pgfqpoint{1.622508in}{1.604458in}}%
\pgfpathclose%
\pgfusepath{fill}%
\end{pgfscope}%
\begin{pgfscope}%
\pgfpathrectangle{\pgfqpoint{0.000000in}{0.000000in}}{\pgfqpoint{3.000000in}{3.000000in}}%
\pgfusepath{clip}%
\pgfsetbuttcap%
\pgfsetroundjoin%
\definecolor{currentfill}{rgb}{0.803030,0.000000,0.000000}%
\pgfsetfillcolor{currentfill}%
\pgfsetlinewidth{0.000000pt}%
\definecolor{currentstroke}{rgb}{0.000000,0.000000,0.000000}%
\pgfsetstrokecolor{currentstroke}%
\pgfsetdash{}{0pt}%
\pgfpathmoveto{\pgfqpoint{1.377615in}{1.610829in}}%
\pgfpathlineto{\pgfqpoint{1.373672in}{1.616426in}}%
\pgfpathlineto{\pgfqpoint{1.498499in}{1.608266in}}%
\pgfpathlineto{\pgfqpoint{1.499493in}{1.602858in}}%
\pgfpathlineto{\pgfqpoint{1.377615in}{1.610829in}}%
\pgfpathclose%
\pgfusepath{fill}%
\end{pgfscope}%
\begin{pgfscope}%
\pgfpathrectangle{\pgfqpoint{0.000000in}{0.000000in}}{\pgfqpoint{3.000000in}{3.000000in}}%
\pgfusepath{clip}%
\pgfsetbuttcap%
\pgfsetroundjoin%
\definecolor{currentfill}{rgb}{0.731729,0.000000,0.000000}%
\pgfsetfillcolor{currentfill}%
\pgfsetlinewidth{0.000000pt}%
\definecolor{currentstroke}{rgb}{0.000000,0.000000,0.000000}%
\pgfsetstrokecolor{currentstroke}%
\pgfsetdash{}{0pt}%
\pgfpathmoveto{\pgfqpoint{1.253466in}{1.634131in}}%
\pgfpathlineto{\pgfqpoint{1.246680in}{1.639927in}}%
\pgfpathlineto{\pgfqpoint{1.369720in}{1.621814in}}%
\pgfpathlineto{\pgfqpoint{1.373672in}{1.616426in}}%
\pgfpathlineto{\pgfqpoint{1.253466in}{1.634131in}}%
\pgfpathclose%
\pgfusepath{fill}%
\end{pgfscope}%
\begin{pgfscope}%
\pgfpathrectangle{\pgfqpoint{0.000000in}{0.000000in}}{\pgfqpoint{3.000000in}{3.000000in}}%
\pgfusepath{clip}%
\pgfsetbuttcap%
\pgfsetroundjoin%
\definecolor{currentfill}{rgb}{0.606952,0.000000,0.000000}%
\pgfsetfillcolor{currentfill}%
\pgfsetlinewidth{0.000000pt}%
\definecolor{currentstroke}{rgb}{0.000000,0.000000,0.000000}%
\pgfsetstrokecolor{currentstroke}%
\pgfsetdash{}{0pt}%
\pgfpathmoveto{\pgfqpoint{1.122316in}{1.673463in}}%
\pgfpathlineto{\pgfqpoint{1.112859in}{1.679492in}}%
\pgfpathlineto{\pgfqpoint{1.233057in}{1.650940in}}%
\pgfpathlineto{\pgfqpoint{1.239877in}{1.645527in}}%
\pgfpathlineto{\pgfqpoint{1.122316in}{1.673463in}}%
\pgfpathclose%
\pgfusepath{fill}%
\end{pgfscope}%
\begin{pgfscope}%
\pgfpathrectangle{\pgfqpoint{0.000000in}{0.000000in}}{\pgfqpoint{3.000000in}{3.000000in}}%
\pgfusepath{clip}%
\pgfsetbuttcap%
\pgfsetroundjoin%
\definecolor{currentfill}{rgb}{0.803030,0.000000,0.000000}%
\pgfsetfillcolor{currentfill}%
\pgfsetlinewidth{0.000000pt}%
\definecolor{currentstroke}{rgb}{0.000000,0.000000,0.000000}%
\pgfsetstrokecolor{currentstroke}%
\pgfsetdash{}{0pt}%
\pgfpathmoveto{\pgfqpoint{1.499493in}{1.602858in}}%
\pgfpathlineto{\pgfqpoint{1.498499in}{1.608266in}}%
\pgfpathlineto{\pgfqpoint{1.624493in}{1.609903in}}%
\pgfpathlineto{\pgfqpoint{1.622508in}{1.604458in}}%
\pgfpathlineto{\pgfqpoint{1.499493in}{1.602858in}}%
\pgfpathclose%
\pgfusepath{fill}%
\end{pgfscope}%
\begin{pgfscope}%
\pgfpathrectangle{\pgfqpoint{0.000000in}{0.000000in}}{\pgfqpoint{3.000000in}{3.000000in}}%
\pgfusepath{clip}%
\pgfsetbuttcap%
\pgfsetroundjoin%
\definecolor{currentfill}{rgb}{0.553476,0.000000,0.000000}%
\pgfsetfillcolor{currentfill}%
\pgfsetlinewidth{0.000000pt}%
\definecolor{currentstroke}{rgb}{0.000000,0.000000,0.000000}%
\pgfsetstrokecolor{currentstroke}%
\pgfsetdash{}{0pt}%
\pgfpathmoveto{\pgfqpoint{1.889205in}{1.659426in}}%
\pgfpathlineto{\pgfqpoint{1.896949in}{1.664847in}}%
\pgfpathlineto{\pgfqpoint{2.016122in}{1.697103in}}%
\pgfpathlineto{\pgfqpoint{2.005823in}{1.691003in}}%
\pgfpathlineto{\pgfqpoint{1.889205in}{1.659426in}}%
\pgfpathclose%
\pgfusepath{fill}%
\end{pgfscope}%
\begin{pgfscope}%
\pgfpathrectangle{\pgfqpoint{0.000000in}{0.000000in}}{\pgfqpoint{3.000000in}{3.000000in}}%
\pgfusepath{clip}%
\pgfsetbuttcap%
\pgfsetroundjoin%
\definecolor{currentfill}{rgb}{0.678253,0.000000,0.000000}%
\pgfsetfillcolor{currentfill}%
\pgfsetlinewidth{0.000000pt}%
\definecolor{currentstroke}{rgb}{0.000000,0.000000,0.000000}%
\pgfsetstrokecolor{currentstroke}%
\pgfsetdash{}{0pt}%
\pgfpathmoveto{\pgfqpoint{1.753073in}{1.626788in}}%
\pgfpathlineto{\pgfqpoint{1.758000in}{1.632091in}}%
\pgfpathlineto{\pgfqpoint{1.881479in}{1.653830in}}%
\pgfpathlineto{\pgfqpoint{1.873771in}{1.648047in}}%
\pgfpathlineto{\pgfqpoint{1.753073in}{1.626788in}}%
\pgfpathclose%
\pgfusepath{fill}%
\end{pgfscope}%
\begin{pgfscope}%
\pgfpathrectangle{\pgfqpoint{0.000000in}{0.000000in}}{\pgfqpoint{3.000000in}{3.000000in}}%
\pgfusepath{clip}%
\pgfsetbuttcap%
\pgfsetroundjoin%
\definecolor{currentfill}{rgb}{0.731729,0.000000,0.000000}%
\pgfsetfillcolor{currentfill}%
\pgfsetlinewidth{0.000000pt}%
\definecolor{currentstroke}{rgb}{0.000000,0.000000,0.000000}%
\pgfsetstrokecolor{currentstroke}%
\pgfsetdash{}{0pt}%
\pgfpathmoveto{\pgfqpoint{1.624493in}{1.609903in}}%
\pgfpathlineto{\pgfqpoint{1.626483in}{1.615141in}}%
\pgfpathlineto{\pgfqpoint{1.753073in}{1.626788in}}%
\pgfpathlineto{\pgfqpoint{1.748158in}{1.621288in}}%
\pgfpathlineto{\pgfqpoint{1.624493in}{1.609903in}}%
\pgfpathclose%
\pgfusepath{fill}%
\end{pgfscope}%
\begin{pgfscope}%
\pgfpathrectangle{\pgfqpoint{0.000000in}{0.000000in}}{\pgfqpoint{3.000000in}{3.000000in}}%
\pgfusepath{clip}%
\pgfsetbuttcap%
\pgfsetroundjoin%
\definecolor{currentfill}{rgb}{0.678253,0.000000,0.000000}%
\pgfsetfillcolor{currentfill}%
\pgfsetlinewidth{0.000000pt}%
\definecolor{currentstroke}{rgb}{0.000000,0.000000,0.000000}%
\pgfsetstrokecolor{currentstroke}%
\pgfsetdash{}{0pt}%
\pgfpathmoveto{\pgfqpoint{1.246680in}{1.639927in}}%
\pgfpathlineto{\pgfqpoint{1.239877in}{1.645527in}}%
\pgfpathlineto{\pgfqpoint{1.365758in}{1.627004in}}%
\pgfpathlineto{\pgfqpoint{1.369720in}{1.621814in}}%
\pgfpathlineto{\pgfqpoint{1.246680in}{1.639927in}}%
\pgfpathclose%
\pgfusepath{fill}%
\end{pgfscope}%
\begin{pgfscope}%
\pgfpathrectangle{\pgfqpoint{0.000000in}{0.000000in}}{\pgfqpoint{3.000000in}{3.000000in}}%
\pgfusepath{clip}%
\pgfsetbuttcap%
\pgfsetroundjoin%
\definecolor{currentfill}{rgb}{0.731729,0.000000,0.000000}%
\pgfsetfillcolor{currentfill}%
\pgfsetlinewidth{0.000000pt}%
\definecolor{currentstroke}{rgb}{0.000000,0.000000,0.000000}%
\pgfsetstrokecolor{currentstroke}%
\pgfsetdash{}{0pt}%
\pgfpathmoveto{\pgfqpoint{1.373672in}{1.616426in}}%
\pgfpathlineto{\pgfqpoint{1.369720in}{1.621814in}}%
\pgfpathlineto{\pgfqpoint{1.497503in}{1.613465in}}%
\pgfpathlineto{\pgfqpoint{1.498499in}{1.608266in}}%
\pgfpathlineto{\pgfqpoint{1.373672in}{1.616426in}}%
\pgfpathclose%
\pgfusepath{fill}%
\end{pgfscope}%
\begin{pgfscope}%
\pgfpathrectangle{\pgfqpoint{0.000000in}{0.000000in}}{\pgfqpoint{3.000000in}{3.000000in}}%
\pgfusepath{clip}%
\pgfsetbuttcap%
\pgfsetroundjoin%
\definecolor{currentfill}{rgb}{0.553476,0.000000,0.000000}%
\pgfsetfillcolor{currentfill}%
\pgfsetlinewidth{0.000000pt}%
\definecolor{currentstroke}{rgb}{0.000000,0.000000,0.000000}%
\pgfsetstrokecolor{currentstroke}%
\pgfsetdash{}{0pt}%
\pgfpathmoveto{\pgfqpoint{1.112859in}{1.679492in}}%
\pgfpathlineto{\pgfqpoint{1.103381in}{1.685345in}}%
\pgfpathlineto{\pgfqpoint{1.226221in}{1.656177in}}%
\pgfpathlineto{\pgfqpoint{1.233057in}{1.650940in}}%
\pgfpathlineto{\pgfqpoint{1.112859in}{1.679492in}}%
\pgfpathclose%
\pgfusepath{fill}%
\end{pgfscope}%
\begin{pgfscope}%
\pgfpathrectangle{\pgfqpoint{0.000000in}{0.000000in}}{\pgfqpoint{3.000000in}{3.000000in}}%
\pgfusepath{clip}%
\pgfsetbuttcap%
\pgfsetroundjoin%
\definecolor{currentfill}{rgb}{0.731729,0.000000,0.000000}%
\pgfsetfillcolor{currentfill}%
\pgfsetlinewidth{0.000000pt}%
\definecolor{currentstroke}{rgb}{0.000000,0.000000,0.000000}%
\pgfsetstrokecolor{currentstroke}%
\pgfsetdash{}{0pt}%
\pgfpathmoveto{\pgfqpoint{1.498499in}{1.608266in}}%
\pgfpathlineto{\pgfqpoint{1.497503in}{1.613465in}}%
\pgfpathlineto{\pgfqpoint{1.626483in}{1.615141in}}%
\pgfpathlineto{\pgfqpoint{1.624493in}{1.609903in}}%
\pgfpathlineto{\pgfqpoint{1.498499in}{1.608266in}}%
\pgfpathclose%
\pgfusepath{fill}%
\end{pgfscope}%
\begin{pgfscope}%
\pgfpathrectangle{\pgfqpoint{0.000000in}{0.000000in}}{\pgfqpoint{3.000000in}{3.000000in}}%
\pgfusepath{clip}%
\pgfsetbuttcap%
\pgfsetroundjoin%
\definecolor{currentfill}{rgb}{0.500000,0.000000,0.000000}%
\pgfsetfillcolor{currentfill}%
\pgfsetlinewidth{0.000000pt}%
\definecolor{currentstroke}{rgb}{0.000000,0.000000,0.000000}%
\pgfsetstrokecolor{currentstroke}%
\pgfsetdash{}{0pt}%
\pgfpathmoveto{\pgfqpoint{1.896949in}{1.664847in}}%
\pgfpathlineto{\pgfqpoint{1.904712in}{1.670099in}}%
\pgfpathlineto{\pgfqpoint{2.026444in}{1.703037in}}%
\pgfpathlineto{\pgfqpoint{2.016122in}{1.697103in}}%
\pgfpathlineto{\pgfqpoint{1.896949in}{1.664847in}}%
\pgfpathclose%
\pgfusepath{fill}%
\end{pgfscope}%
\begin{pgfscope}%
\pgfpathrectangle{\pgfqpoint{0.000000in}{0.000000in}}{\pgfqpoint{3.000000in}{3.000000in}}%
\pgfusepath{clip}%
\pgfsetbuttcap%
\pgfsetroundjoin%
\definecolor{currentfill}{rgb}{0.606952,0.000000,0.000000}%
\pgfsetfillcolor{currentfill}%
\pgfsetlinewidth{0.000000pt}%
\definecolor{currentstroke}{rgb}{0.000000,0.000000,0.000000}%
\pgfsetstrokecolor{currentstroke}%
\pgfsetdash{}{0pt}%
\pgfpathmoveto{\pgfqpoint{1.758000in}{1.632091in}}%
\pgfpathlineto{\pgfqpoint{1.762940in}{1.637207in}}%
\pgfpathlineto{\pgfqpoint{1.889205in}{1.659426in}}%
\pgfpathlineto{\pgfqpoint{1.881479in}{1.653830in}}%
\pgfpathlineto{\pgfqpoint{1.758000in}{1.632091in}}%
\pgfpathclose%
\pgfusepath{fill}%
\end{pgfscope}%
\begin{pgfscope}%
\pgfpathrectangle{\pgfqpoint{0.000000in}{0.000000in}}{\pgfqpoint{3.000000in}{3.000000in}}%
\pgfusepath{clip}%
\pgfsetbuttcap%
\pgfsetroundjoin%
\definecolor{currentfill}{rgb}{0.678253,0.000000,0.000000}%
\pgfsetfillcolor{currentfill}%
\pgfsetlinewidth{0.000000pt}%
\definecolor{currentstroke}{rgb}{0.000000,0.000000,0.000000}%
\pgfsetstrokecolor{currentstroke}%
\pgfsetdash{}{0pt}%
\pgfpathmoveto{\pgfqpoint{1.626483in}{1.615141in}}%
\pgfpathlineto{\pgfqpoint{1.628478in}{1.620180in}}%
\pgfpathlineto{\pgfqpoint{1.758000in}{1.632091in}}%
\pgfpathlineto{\pgfqpoint{1.753073in}{1.626788in}}%
\pgfpathlineto{\pgfqpoint{1.626483in}{1.615141in}}%
\pgfpathclose%
\pgfusepath{fill}%
\end{pgfscope}%
\begin{pgfscope}%
\pgfpathrectangle{\pgfqpoint{0.000000in}{0.000000in}}{\pgfqpoint{3.000000in}{3.000000in}}%
\pgfusepath{clip}%
\pgfsetbuttcap%
\pgfsetroundjoin%
\definecolor{currentfill}{rgb}{0.606952,0.000000,0.000000}%
\pgfsetfillcolor{currentfill}%
\pgfsetlinewidth{0.000000pt}%
\definecolor{currentstroke}{rgb}{0.000000,0.000000,0.000000}%
\pgfsetstrokecolor{currentstroke}%
\pgfsetdash{}{0pt}%
\pgfpathmoveto{\pgfqpoint{1.239877in}{1.645527in}}%
\pgfpathlineto{\pgfqpoint{1.233057in}{1.650940in}}%
\pgfpathlineto{\pgfqpoint{1.361785in}{1.632008in}}%
\pgfpathlineto{\pgfqpoint{1.365758in}{1.627004in}}%
\pgfpathlineto{\pgfqpoint{1.239877in}{1.645527in}}%
\pgfpathclose%
\pgfusepath{fill}%
\end{pgfscope}%
\begin{pgfscope}%
\pgfpathrectangle{\pgfqpoint{0.000000in}{0.000000in}}{\pgfqpoint{3.000000in}{3.000000in}}%
\pgfusepath{clip}%
\pgfsetbuttcap%
\pgfsetroundjoin%
\definecolor{currentfill}{rgb}{0.678253,0.000000,0.000000}%
\pgfsetfillcolor{currentfill}%
\pgfsetlinewidth{0.000000pt}%
\definecolor{currentstroke}{rgb}{0.000000,0.000000,0.000000}%
\pgfsetstrokecolor{currentstroke}%
\pgfsetdash{}{0pt}%
\pgfpathmoveto{\pgfqpoint{1.369720in}{1.621814in}}%
\pgfpathlineto{\pgfqpoint{1.365758in}{1.627004in}}%
\pgfpathlineto{\pgfqpoint{1.496503in}{1.618467in}}%
\pgfpathlineto{\pgfqpoint{1.497503in}{1.613465in}}%
\pgfpathlineto{\pgfqpoint{1.369720in}{1.621814in}}%
\pgfpathclose%
\pgfusepath{fill}%
\end{pgfscope}%
\begin{pgfscope}%
\pgfpathrectangle{\pgfqpoint{0.000000in}{0.000000in}}{\pgfqpoint{3.000000in}{3.000000in}}%
\pgfusepath{clip}%
\pgfsetbuttcap%
\pgfsetroundjoin%
\definecolor{currentfill}{rgb}{0.500000,0.000000,0.000000}%
\pgfsetfillcolor{currentfill}%
\pgfsetlinewidth{0.000000pt}%
\definecolor{currentstroke}{rgb}{0.000000,0.000000,0.000000}%
\pgfsetstrokecolor{currentstroke}%
\pgfsetdash{}{0pt}%
\pgfpathmoveto{\pgfqpoint{1.103381in}{1.685345in}}%
\pgfpathlineto{\pgfqpoint{1.093880in}{1.691031in}}%
\pgfpathlineto{\pgfqpoint{1.219368in}{1.661246in}}%
\pgfpathlineto{\pgfqpoint{1.226221in}{1.656177in}}%
\pgfpathlineto{\pgfqpoint{1.103381in}{1.685345in}}%
\pgfpathclose%
\pgfusepath{fill}%
\end{pgfscope}%
\begin{pgfscope}%
\pgfpathrectangle{\pgfqpoint{0.000000in}{0.000000in}}{\pgfqpoint{3.000000in}{3.000000in}}%
\pgfusepath{clip}%
\pgfsetbuttcap%
\pgfsetroundjoin%
\definecolor{currentfill}{rgb}{0.678253,0.000000,0.000000}%
\pgfsetfillcolor{currentfill}%
\pgfsetlinewidth{0.000000pt}%
\definecolor{currentstroke}{rgb}{0.000000,0.000000,0.000000}%
\pgfsetstrokecolor{currentstroke}%
\pgfsetdash{}{0pt}%
\pgfpathmoveto{\pgfqpoint{1.497503in}{1.613465in}}%
\pgfpathlineto{\pgfqpoint{1.496503in}{1.618467in}}%
\pgfpathlineto{\pgfqpoint{1.628478in}{1.620180in}}%
\pgfpathlineto{\pgfqpoint{1.626483in}{1.615141in}}%
\pgfpathlineto{\pgfqpoint{1.497503in}{1.613465in}}%
\pgfpathclose%
\pgfusepath{fill}%
\end{pgfscope}%
\begin{pgfscope}%
\pgfpathrectangle{\pgfqpoint{0.000000in}{0.000000in}}{\pgfqpoint{3.000000in}{3.000000in}}%
\pgfusepath{clip}%
\pgfsetbuttcap%
\pgfsetroundjoin%
\definecolor{currentfill}{rgb}{0.553476,0.000000,0.000000}%
\pgfsetfillcolor{currentfill}%
\pgfsetlinewidth{0.000000pt}%
\definecolor{currentstroke}{rgb}{0.000000,0.000000,0.000000}%
\pgfsetstrokecolor{currentstroke}%
\pgfsetdash{}{0pt}%
\pgfpathmoveto{\pgfqpoint{1.762940in}{1.637207in}}%
\pgfpathlineto{\pgfqpoint{1.767892in}{1.642146in}}%
\pgfpathlineto{\pgfqpoint{1.896949in}{1.664847in}}%
\pgfpathlineto{\pgfqpoint{1.889205in}{1.659426in}}%
\pgfpathlineto{\pgfqpoint{1.762940in}{1.637207in}}%
\pgfpathclose%
\pgfusepath{fill}%
\end{pgfscope}%
\begin{pgfscope}%
\pgfpathrectangle{\pgfqpoint{0.000000in}{0.000000in}}{\pgfqpoint{3.000000in}{3.000000in}}%
\pgfusepath{clip}%
\pgfsetbuttcap%
\pgfsetroundjoin%
\definecolor{currentfill}{rgb}{0.606952,0.000000,0.000000}%
\pgfsetfillcolor{currentfill}%
\pgfsetlinewidth{0.000000pt}%
\definecolor{currentstroke}{rgb}{0.000000,0.000000,0.000000}%
\pgfsetstrokecolor{currentstroke}%
\pgfsetdash{}{0pt}%
\pgfpathmoveto{\pgfqpoint{1.628478in}{1.620180in}}%
\pgfpathlineto{\pgfqpoint{1.630478in}{1.625032in}}%
\pgfpathlineto{\pgfqpoint{1.762940in}{1.637207in}}%
\pgfpathlineto{\pgfqpoint{1.758000in}{1.632091in}}%
\pgfpathlineto{\pgfqpoint{1.628478in}{1.620180in}}%
\pgfpathclose%
\pgfusepath{fill}%
\end{pgfscope}%
\begin{pgfscope}%
\pgfpathrectangle{\pgfqpoint{0.000000in}{0.000000in}}{\pgfqpoint{3.000000in}{3.000000in}}%
\pgfusepath{clip}%
\pgfsetbuttcap%
\pgfsetroundjoin%
\definecolor{currentfill}{rgb}{0.553476,0.000000,0.000000}%
\pgfsetfillcolor{currentfill}%
\pgfsetlinewidth{0.000000pt}%
\definecolor{currentstroke}{rgb}{0.000000,0.000000,0.000000}%
\pgfsetstrokecolor{currentstroke}%
\pgfsetdash{}{0pt}%
\pgfpathmoveto{\pgfqpoint{1.233057in}{1.650940in}}%
\pgfpathlineto{\pgfqpoint{1.226221in}{1.656177in}}%
\pgfpathlineto{\pgfqpoint{1.357803in}{1.636834in}}%
\pgfpathlineto{\pgfqpoint{1.361785in}{1.632008in}}%
\pgfpathlineto{\pgfqpoint{1.233057in}{1.650940in}}%
\pgfpathclose%
\pgfusepath{fill}%
\end{pgfscope}%
\begin{pgfscope}%
\pgfpathrectangle{\pgfqpoint{0.000000in}{0.000000in}}{\pgfqpoint{3.000000in}{3.000000in}}%
\pgfusepath{clip}%
\pgfsetbuttcap%
\pgfsetroundjoin%
\definecolor{currentfill}{rgb}{0.606952,0.000000,0.000000}%
\pgfsetfillcolor{currentfill}%
\pgfsetlinewidth{0.000000pt}%
\definecolor{currentstroke}{rgb}{0.000000,0.000000,0.000000}%
\pgfsetstrokecolor{currentstroke}%
\pgfsetdash{}{0pt}%
\pgfpathmoveto{\pgfqpoint{1.365758in}{1.627004in}}%
\pgfpathlineto{\pgfqpoint{1.361785in}{1.632008in}}%
\pgfpathlineto{\pgfqpoint{1.495501in}{1.623281in}}%
\pgfpathlineto{\pgfqpoint{1.496503in}{1.618467in}}%
\pgfpathlineto{\pgfqpoint{1.365758in}{1.627004in}}%
\pgfpathclose%
\pgfusepath{fill}%
\end{pgfscope}%
\begin{pgfscope}%
\pgfpathrectangle{\pgfqpoint{0.000000in}{0.000000in}}{\pgfqpoint{3.000000in}{3.000000in}}%
\pgfusepath{clip}%
\pgfsetbuttcap%
\pgfsetroundjoin%
\definecolor{currentfill}{rgb}{0.606952,0.000000,0.000000}%
\pgfsetfillcolor{currentfill}%
\pgfsetlinewidth{0.000000pt}%
\definecolor{currentstroke}{rgb}{0.000000,0.000000,0.000000}%
\pgfsetstrokecolor{currentstroke}%
\pgfsetdash{}{0pt}%
\pgfpathmoveto{\pgfqpoint{1.496503in}{1.618467in}}%
\pgfpathlineto{\pgfqpoint{1.495501in}{1.623281in}}%
\pgfpathlineto{\pgfqpoint{1.630478in}{1.625032in}}%
\pgfpathlineto{\pgfqpoint{1.628478in}{1.620180in}}%
\pgfpathlineto{\pgfqpoint{1.496503in}{1.618467in}}%
\pgfpathclose%
\pgfusepath{fill}%
\end{pgfscope}%
\begin{pgfscope}%
\pgfpathrectangle{\pgfqpoint{0.000000in}{0.000000in}}{\pgfqpoint{3.000000in}{3.000000in}}%
\pgfusepath{clip}%
\pgfsetbuttcap%
\pgfsetroundjoin%
\definecolor{currentfill}{rgb}{0.500000,0.000000,0.000000}%
\pgfsetfillcolor{currentfill}%
\pgfsetlinewidth{0.000000pt}%
\definecolor{currentstroke}{rgb}{0.000000,0.000000,0.000000}%
\pgfsetstrokecolor{currentstroke}%
\pgfsetdash{}{0pt}%
\pgfpathmoveto{\pgfqpoint{1.767892in}{1.642146in}}%
\pgfpathlineto{\pgfqpoint{1.772856in}{1.646917in}}%
\pgfpathlineto{\pgfqpoint{1.904712in}{1.670099in}}%
\pgfpathlineto{\pgfqpoint{1.896949in}{1.664847in}}%
\pgfpathlineto{\pgfqpoint{1.767892in}{1.642146in}}%
\pgfpathclose%
\pgfusepath{fill}%
\end{pgfscope}%
\begin{pgfscope}%
\pgfpathrectangle{\pgfqpoint{0.000000in}{0.000000in}}{\pgfqpoint{3.000000in}{3.000000in}}%
\pgfusepath{clip}%
\pgfsetbuttcap%
\pgfsetroundjoin%
\definecolor{currentfill}{rgb}{0.553476,0.000000,0.000000}%
\pgfsetfillcolor{currentfill}%
\pgfsetlinewidth{0.000000pt}%
\definecolor{currentstroke}{rgb}{0.000000,0.000000,0.000000}%
\pgfsetstrokecolor{currentstroke}%
\pgfsetdash{}{0pt}%
\pgfpathmoveto{\pgfqpoint{1.630478in}{1.625032in}}%
\pgfpathlineto{\pgfqpoint{1.632483in}{1.629706in}}%
\pgfpathlineto{\pgfqpoint{1.767892in}{1.642146in}}%
\pgfpathlineto{\pgfqpoint{1.762940in}{1.637207in}}%
\pgfpathlineto{\pgfqpoint{1.630478in}{1.625032in}}%
\pgfpathclose%
\pgfusepath{fill}%
\end{pgfscope}%
\begin{pgfscope}%
\pgfpathrectangle{\pgfqpoint{0.000000in}{0.000000in}}{\pgfqpoint{3.000000in}{3.000000in}}%
\pgfusepath{clip}%
\pgfsetbuttcap%
\pgfsetroundjoin%
\definecolor{currentfill}{rgb}{0.500000,0.000000,0.000000}%
\pgfsetfillcolor{currentfill}%
\pgfsetlinewidth{0.000000pt}%
\definecolor{currentstroke}{rgb}{0.000000,0.000000,0.000000}%
\pgfsetstrokecolor{currentstroke}%
\pgfsetdash{}{0pt}%
\pgfpathmoveto{\pgfqpoint{1.226221in}{1.656177in}}%
\pgfpathlineto{\pgfqpoint{1.219368in}{1.661246in}}%
\pgfpathlineto{\pgfqpoint{1.353810in}{1.641491in}}%
\pgfpathlineto{\pgfqpoint{1.357803in}{1.636834in}}%
\pgfpathlineto{\pgfqpoint{1.226221in}{1.656177in}}%
\pgfpathclose%
\pgfusepath{fill}%
\end{pgfscope}%
\begin{pgfscope}%
\pgfpathrectangle{\pgfqpoint{0.000000in}{0.000000in}}{\pgfqpoint{3.000000in}{3.000000in}}%
\pgfusepath{clip}%
\pgfsetbuttcap%
\pgfsetroundjoin%
\definecolor{currentfill}{rgb}{0.553476,0.000000,0.000000}%
\pgfsetfillcolor{currentfill}%
\pgfsetlinewidth{0.000000pt}%
\definecolor{currentstroke}{rgb}{0.000000,0.000000,0.000000}%
\pgfsetstrokecolor{currentstroke}%
\pgfsetdash{}{0pt}%
\pgfpathmoveto{\pgfqpoint{1.361785in}{1.632008in}}%
\pgfpathlineto{\pgfqpoint{1.357803in}{1.636834in}}%
\pgfpathlineto{\pgfqpoint{1.494497in}{1.627917in}}%
\pgfpathlineto{\pgfqpoint{1.495501in}{1.623281in}}%
\pgfpathlineto{\pgfqpoint{1.361785in}{1.632008in}}%
\pgfpathclose%
\pgfusepath{fill}%
\end{pgfscope}%
\begin{pgfscope}%
\pgfpathrectangle{\pgfqpoint{0.000000in}{0.000000in}}{\pgfqpoint{3.000000in}{3.000000in}}%
\pgfusepath{clip}%
\pgfsetbuttcap%
\pgfsetroundjoin%
\definecolor{currentfill}{rgb}{0.553476,0.000000,0.000000}%
\pgfsetfillcolor{currentfill}%
\pgfsetlinewidth{0.000000pt}%
\definecolor{currentstroke}{rgb}{0.000000,0.000000,0.000000}%
\pgfsetstrokecolor{currentstroke}%
\pgfsetdash{}{0pt}%
\pgfpathmoveto{\pgfqpoint{1.495501in}{1.623281in}}%
\pgfpathlineto{\pgfqpoint{1.494497in}{1.627917in}}%
\pgfpathlineto{\pgfqpoint{1.632483in}{1.629706in}}%
\pgfpathlineto{\pgfqpoint{1.630478in}{1.625032in}}%
\pgfpathlineto{\pgfqpoint{1.495501in}{1.623281in}}%
\pgfpathclose%
\pgfusepath{fill}%
\end{pgfscope}%
\begin{pgfscope}%
\pgfpathrectangle{\pgfqpoint{0.000000in}{0.000000in}}{\pgfqpoint{3.000000in}{3.000000in}}%
\pgfusepath{clip}%
\pgfsetbuttcap%
\pgfsetroundjoin%
\definecolor{currentfill}{rgb}{0.500000,0.000000,0.000000}%
\pgfsetfillcolor{currentfill}%
\pgfsetlinewidth{0.000000pt}%
\definecolor{currentstroke}{rgb}{0.000000,0.000000,0.000000}%
\pgfsetstrokecolor{currentstroke}%
\pgfsetdash{}{0pt}%
\pgfpathmoveto{\pgfqpoint{1.632483in}{1.629706in}}%
\pgfpathlineto{\pgfqpoint{1.634494in}{1.634211in}}%
\pgfpathlineto{\pgfqpoint{1.772856in}{1.646917in}}%
\pgfpathlineto{\pgfqpoint{1.767892in}{1.642146in}}%
\pgfpathlineto{\pgfqpoint{1.632483in}{1.629706in}}%
\pgfpathclose%
\pgfusepath{fill}%
\end{pgfscope}%
\begin{pgfscope}%
\pgfpathrectangle{\pgfqpoint{0.000000in}{0.000000in}}{\pgfqpoint{3.000000in}{3.000000in}}%
\pgfusepath{clip}%
\pgfsetbuttcap%
\pgfsetroundjoin%
\definecolor{currentfill}{rgb}{0.500000,0.000000,0.000000}%
\pgfsetfillcolor{currentfill}%
\pgfsetlinewidth{0.000000pt}%
\definecolor{currentstroke}{rgb}{0.000000,0.000000,0.000000}%
\pgfsetstrokecolor{currentstroke}%
\pgfsetdash{}{0pt}%
\pgfpathmoveto{\pgfqpoint{1.357803in}{1.636834in}}%
\pgfpathlineto{\pgfqpoint{1.353810in}{1.641491in}}%
\pgfpathlineto{\pgfqpoint{1.493490in}{1.632383in}}%
\pgfpathlineto{\pgfqpoint{1.494497in}{1.627917in}}%
\pgfpathlineto{\pgfqpoint{1.357803in}{1.636834in}}%
\pgfpathclose%
\pgfusepath{fill}%
\end{pgfscope}%
\begin{pgfscope}%
\pgfpathrectangle{\pgfqpoint{0.000000in}{0.000000in}}{\pgfqpoint{3.000000in}{3.000000in}}%
\pgfusepath{clip}%
\pgfsetbuttcap%
\pgfsetroundjoin%
\definecolor{currentfill}{rgb}{0.500000,0.000000,0.000000}%
\pgfsetfillcolor{currentfill}%
\pgfsetlinewidth{0.000000pt}%
\definecolor{currentstroke}{rgb}{0.000000,0.000000,0.000000}%
\pgfsetstrokecolor{currentstroke}%
\pgfsetdash{}{0pt}%
\pgfpathmoveto{\pgfqpoint{1.494497in}{1.627917in}}%
\pgfpathlineto{\pgfqpoint{1.493490in}{1.632383in}}%
\pgfpathlineto{\pgfqpoint{1.634494in}{1.634211in}}%
\pgfpathlineto{\pgfqpoint{1.632483in}{1.629706in}}%
\pgfpathlineto{\pgfqpoint{1.494497in}{1.627917in}}%
\pgfpathclose%
\pgfusepath{fill}%
\end{pgfscope}%
\end{pgfpicture}%
\makeatother%
\endgroup%

	\caption{Flamm's paraboloid from the first simulation. It represents $\gamma_{xx}$ in the equatorial plane where $\theta = \frac{\pi}{2}$. The slope at each point represents the magnitude of $\gamma_{xx}$.}
	\label{fig:0_data0}
\end{figure}

However, when the errors in the coordinate system of the lapse function and minor errors were corrected, the metric could no longer be considered static as shown in Figure 1.

\begin{figure}[H]
	\centering
	%% Creator: Matplotlib, PGF backend
%%
%% To include the figure in your LaTeX document, write
%%   \input{<filename>.pgf}
%%
%% Make sure the required packages are loaded in your preamble
%%   \usepackage{pgf}
%%
%% Also ensure that all the required font packages are loaded; for instance,
%% the lmodern package is sometimes necessary when using math font.
%%   \usepackage{lmodern}
%%
%% Figures using additional raster images can only be included by \input if
%% they are in the same directory as the main LaTeX file. For loading figures
%% from other directories you can use the `import` package
%%   \usepackage{import}
%%
%% and then include the figures with
%%   \import{<path to file>}{<filename>.pgf}
%%
%% Matplotlib used the following preamble
%%   \usepackage{fontspec}
%%   \setmainfont{DejaVuSerif.ttf}[Path=\detokenize{C:/boj/venv/Lib/site-packages/matplotlib/mpl-data/fonts/ttf/}]
%%   \setsansfont{DejaVuSans.ttf}[Path=\detokenize{C:/boj/venv/Lib/site-packages/matplotlib/mpl-data/fonts/ttf/}]
%%   \setmonofont{DejaVuSansMono.ttf}[Path=\detokenize{C:/boj/venv/Lib/site-packages/matplotlib/mpl-data/fonts/ttf/}]
%%
\begingroup%
\makeatletter%
\begin{pgfpicture}%
\pgfpathrectangle{\pgfpointorigin}{\pgfqpoint{5.000000in}{5.000000in}}%
\pgfusepath{use as bounding box, clip}%
\begin{pgfscope}%
\pgfsetbuttcap%
\pgfsetmiterjoin%
\definecolor{currentfill}{rgb}{1.000000,1.000000,1.000000}%
\pgfsetfillcolor{currentfill}%
\pgfsetlinewidth{0.000000pt}%
\definecolor{currentstroke}{rgb}{1.000000,1.000000,1.000000}%
\pgfsetstrokecolor{currentstroke}%
\pgfsetdash{}{0pt}%
\pgfpathmoveto{\pgfqpoint{0.000000in}{0.000000in}}%
\pgfpathlineto{\pgfqpoint{5.000000in}{0.000000in}}%
\pgfpathlineto{\pgfqpoint{5.000000in}{5.000000in}}%
\pgfpathlineto{\pgfqpoint{0.000000in}{5.000000in}}%
\pgfpathlineto{\pgfqpoint{0.000000in}{0.000000in}}%
\pgfpathclose%
\pgfusepath{fill}%
\end{pgfscope}%
\begin{pgfscope}%
\pgfsetbuttcap%
\pgfsetmiterjoin%
\definecolor{currentfill}{rgb}{1.000000,1.000000,1.000000}%
\pgfsetfillcolor{currentfill}%
\pgfsetlinewidth{0.000000pt}%
\definecolor{currentstroke}{rgb}{0.000000,0.000000,0.000000}%
\pgfsetstrokecolor{currentstroke}%
\pgfsetstrokeopacity{0.000000}%
\pgfsetdash{}{0pt}%
\pgfpathmoveto{\pgfqpoint{0.000000in}{0.250000in}}%
\pgfpathlineto{\pgfqpoint{4.500000in}{0.250000in}}%
\pgfpathlineto{\pgfqpoint{4.500000in}{4.750000in}}%
\pgfpathlineto{\pgfqpoint{0.000000in}{4.750000in}}%
\pgfpathlineto{\pgfqpoint{0.000000in}{0.250000in}}%
\pgfpathclose%
\pgfusepath{fill}%
\end{pgfscope}%
\begin{pgfscope}%
\pgfsetbuttcap%
\pgfsetmiterjoin%
\definecolor{currentfill}{rgb}{0.950000,0.950000,0.950000}%
\pgfsetfillcolor{currentfill}%
\pgfsetfillopacity{0.500000}%
\pgfsetlinewidth{1.003750pt}%
\definecolor{currentstroke}{rgb}{0.950000,0.950000,0.950000}%
\pgfsetstrokecolor{currentstroke}%
\pgfsetstrokeopacity{0.500000}%
\pgfsetdash{}{0pt}%
\pgfpathmoveto{\pgfqpoint{0.339781in}{1.359556in}}%
\pgfpathlineto{\pgfqpoint{1.825828in}{2.605190in}}%
\pgfpathlineto{\pgfqpoint{1.805171in}{4.401617in}}%
\pgfpathlineto{\pgfqpoint{0.248009in}{3.265271in}}%
\pgfusepath{stroke,fill}%
\end{pgfscope}%
\begin{pgfscope}%
\pgfsetbuttcap%
\pgfsetmiterjoin%
\definecolor{currentfill}{rgb}{0.900000,0.900000,0.900000}%
\pgfsetfillcolor{currentfill}%
\pgfsetfillopacity{0.500000}%
\pgfsetlinewidth{1.003750pt}%
\definecolor{currentstroke}{rgb}{0.900000,0.900000,0.900000}%
\pgfsetstrokecolor{currentstroke}%
\pgfsetstrokeopacity{0.500000}%
\pgfsetdash{}{0pt}%
\pgfpathmoveto{\pgfqpoint{1.825828in}{2.605190in}}%
\pgfpathlineto{\pgfqpoint{4.210403in}{1.912088in}}%
\pgfpathlineto{\pgfqpoint{4.295500in}{3.770391in}}%
\pgfpathlineto{\pgfqpoint{1.805171in}{4.401617in}}%
\pgfusepath{stroke,fill}%
\end{pgfscope}%
\begin{pgfscope}%
\pgfsetbuttcap%
\pgfsetmiterjoin%
\definecolor{currentfill}{rgb}{0.925000,0.925000,0.925000}%
\pgfsetfillcolor{currentfill}%
\pgfsetfillopacity{0.500000}%
\pgfsetlinewidth{1.003750pt}%
\definecolor{currentstroke}{rgb}{0.925000,0.925000,0.925000}%
\pgfsetstrokecolor{currentstroke}%
\pgfsetstrokeopacity{0.500000}%
\pgfsetdash{}{0pt}%
\pgfpathmoveto{\pgfqpoint{0.339781in}{1.359556in}}%
\pgfpathlineto{\pgfqpoint{2.867547in}{0.533987in}}%
\pgfpathlineto{\pgfqpoint{4.210403in}{1.912088in}}%
\pgfpathlineto{\pgfqpoint{1.825828in}{2.605190in}}%
\pgfusepath{stroke,fill}%
\end{pgfscope}%
\begin{pgfscope}%
\pgfsetrectcap%
\pgfsetroundjoin%
\pgfsetlinewidth{0.803000pt}%
\definecolor{currentstroke}{rgb}{0.000000,0.000000,0.000000}%
\pgfsetstrokecolor{currentstroke}%
\pgfsetdash{}{0pt}%
\pgfpathmoveto{\pgfqpoint{0.339781in}{1.359556in}}%
\pgfpathlineto{\pgfqpoint{2.867547in}{0.533987in}}%
\pgfusepath{stroke}%
\end{pgfscope}%
\begin{pgfscope}%
\definecolor{textcolor}{rgb}{0.000000,0.000000,0.000000}%
\pgfsetstrokecolor{textcolor}%
\pgfsetfillcolor{textcolor}%
\pgftext[x=1.197768in, y=0.479856in, left, base,rotate=341.912962]{\color{textcolor}\sffamily\fontsize{10.000000}{12.000000}\selectfont \(\displaystyle r/M\)}%
\end{pgfscope}%
\begin{pgfscope}%
\pgfsetbuttcap%
\pgfsetroundjoin%
\pgfsetlinewidth{0.803000pt}%
\definecolor{currentstroke}{rgb}{0.690196,0.690196,0.690196}%
\pgfsetstrokecolor{currentstroke}%
\pgfsetdash{}{0pt}%
\pgfpathmoveto{\pgfqpoint{2.814312in}{0.551374in}}%
\pgfpathlineto{\pgfqpoint{4.160400in}{1.926622in}}%
\pgfpathlineto{\pgfqpoint{4.243171in}{3.783655in}}%
\pgfusepath{stroke}%
\end{pgfscope}%
\begin{pgfscope}%
\pgfsetbuttcap%
\pgfsetroundjoin%
\pgfsetlinewidth{0.803000pt}%
\definecolor{currentstroke}{rgb}{0.690196,0.690196,0.690196}%
\pgfsetstrokecolor{currentstroke}%
\pgfsetdash{}{0pt}%
\pgfpathmoveto{\pgfqpoint{2.328293in}{0.710108in}}%
\pgfpathlineto{\pgfqpoint{3.703461in}{2.059436in}}%
\pgfpathlineto{\pgfqpoint{3.765195in}{3.904808in}}%
\pgfusepath{stroke}%
\end{pgfscope}%
\begin{pgfscope}%
\pgfsetbuttcap%
\pgfsetroundjoin%
\pgfsetlinewidth{0.803000pt}%
\definecolor{currentstroke}{rgb}{0.690196,0.690196,0.690196}%
\pgfsetstrokecolor{currentstroke}%
\pgfsetdash{}{0pt}%
\pgfpathmoveto{\pgfqpoint{1.851792in}{0.865733in}}%
\pgfpathlineto{\pgfqpoint{3.254717in}{2.189868in}}%
\pgfpathlineto{\pgfqpoint{3.296167in}{4.023692in}}%
\pgfusepath{stroke}%
\end{pgfscope}%
\begin{pgfscope}%
\pgfsetbuttcap%
\pgfsetroundjoin%
\pgfsetlinewidth{0.803000pt}%
\definecolor{currentstroke}{rgb}{0.690196,0.690196,0.690196}%
\pgfsetstrokecolor{currentstroke}%
\pgfsetdash{}{0pt}%
\pgfpathmoveto{\pgfqpoint{1.384532in}{1.018340in}}%
\pgfpathlineto{\pgfqpoint{2.813950in}{2.317982in}}%
\pgfpathlineto{\pgfqpoint{2.835840in}{4.140372in}}%
\pgfusepath{stroke}%
\end{pgfscope}%
\begin{pgfscope}%
\pgfsetbuttcap%
\pgfsetroundjoin%
\pgfsetlinewidth{0.803000pt}%
\definecolor{currentstroke}{rgb}{0.690196,0.690196,0.690196}%
\pgfsetstrokecolor{currentstroke}%
\pgfsetdash{}{0pt}%
\pgfpathmoveto{\pgfqpoint{0.926246in}{1.168016in}}%
\pgfpathlineto{\pgfqpoint{2.380949in}{2.443838in}}%
\pgfpathlineto{\pgfqpoint{2.383972in}{4.254907in}}%
\pgfusepath{stroke}%
\end{pgfscope}%
\begin{pgfscope}%
\pgfsetbuttcap%
\pgfsetroundjoin%
\pgfsetlinewidth{0.803000pt}%
\definecolor{currentstroke}{rgb}{0.690196,0.690196,0.690196}%
\pgfsetstrokecolor{currentstroke}%
\pgfsetdash{}{0pt}%
\pgfpathmoveto{\pgfqpoint{0.476678in}{1.314845in}}%
\pgfpathlineto{\pgfqpoint{1.955510in}{2.567496in}}%
\pgfpathlineto{\pgfqpoint{1.940334in}{4.367357in}}%
\pgfusepath{stroke}%
\end{pgfscope}%
\begin{pgfscope}%
\pgfsetrectcap%
\pgfsetroundjoin%
\pgfsetlinewidth{0.803000pt}%
\definecolor{currentstroke}{rgb}{0.000000,0.000000,0.000000}%
\pgfsetstrokecolor{currentstroke}%
\pgfsetdash{}{0pt}%
\pgfpathmoveto{\pgfqpoint{2.826081in}{0.563398in}}%
\pgfpathlineto{\pgfqpoint{2.790721in}{0.527272in}}%
\pgfusepath{stroke}%
\end{pgfscope}%
\begin{pgfscope}%
\definecolor{textcolor}{rgb}{0.000000,0.000000,0.000000}%
\pgfsetstrokecolor{textcolor}%
\pgfsetfillcolor{textcolor}%
\pgftext[x=2.712049in,y=0.320042in,,top]{\color{textcolor}\sffamily\fontsize{10.000000}{12.000000}\selectfont \(\displaystyle {0.0}\)}%
\end{pgfscope}%
\begin{pgfscope}%
\pgfsetrectcap%
\pgfsetroundjoin%
\pgfsetlinewidth{0.803000pt}%
\definecolor{currentstroke}{rgb}{0.000000,0.000000,0.000000}%
\pgfsetstrokecolor{currentstroke}%
\pgfsetdash{}{0pt}%
\pgfpathmoveto{\pgfqpoint{2.340306in}{0.721895in}}%
\pgfpathlineto{\pgfqpoint{2.304214in}{0.686481in}}%
\pgfusepath{stroke}%
\end{pgfscope}%
\begin{pgfscope}%
\definecolor{textcolor}{rgb}{0.000000,0.000000,0.000000}%
\pgfsetstrokecolor{textcolor}%
\pgfsetfillcolor{textcolor}%
\pgftext[x=2.225504in,y=0.481574in,,top]{\color{textcolor}\sffamily\fontsize{10.000000}{12.000000}\selectfont \(\displaystyle {\ensuremath{-}0.1}\)}%
\end{pgfscope}%
\begin{pgfscope}%
\pgfsetrectcap%
\pgfsetroundjoin%
\pgfsetlinewidth{0.803000pt}%
\definecolor{currentstroke}{rgb}{0.000000,0.000000,0.000000}%
\pgfsetstrokecolor{currentstroke}%
\pgfsetdash{}{0pt}%
\pgfpathmoveto{\pgfqpoint{1.864038in}{0.877291in}}%
\pgfpathlineto{\pgfqpoint{1.827247in}{0.842567in}}%
\pgfusepath{stroke}%
\end{pgfscope}%
\begin{pgfscope}%
\definecolor{textcolor}{rgb}{0.000000,0.000000,0.000000}%
\pgfsetstrokecolor{textcolor}%
\pgfsetfillcolor{textcolor}%
\pgftext[x=1.748516in,y=0.639932in,,top]{\color{textcolor}\sffamily\fontsize{10.000000}{12.000000}\selectfont \(\displaystyle {\ensuremath{-}0.2}\)}%
\end{pgfscope}%
\begin{pgfscope}%
\pgfsetrectcap%
\pgfsetroundjoin%
\pgfsetlinewidth{0.803000pt}%
\definecolor{currentstroke}{rgb}{0.000000,0.000000,0.000000}%
\pgfsetstrokecolor{currentstroke}%
\pgfsetdash{}{0pt}%
\pgfpathmoveto{\pgfqpoint{1.396998in}{1.029675in}}%
\pgfpathlineto{\pgfqpoint{1.359544in}{0.995621in}}%
\pgfusepath{stroke}%
\end{pgfscope}%
\begin{pgfscope}%
\definecolor{textcolor}{rgb}{0.000000,0.000000,0.000000}%
\pgfsetstrokecolor{textcolor}%
\pgfsetfillcolor{textcolor}%
\pgftext[x=1.280808in,y=0.795210in,,top]{\color{textcolor}\sffamily\fontsize{10.000000}{12.000000}\selectfont \(\displaystyle {\ensuremath{-}0.3}\)}%
\end{pgfscope}%
\begin{pgfscope}%
\pgfsetrectcap%
\pgfsetroundjoin%
\pgfsetlinewidth{0.803000pt}%
\definecolor{currentstroke}{rgb}{0.000000,0.000000,0.000000}%
\pgfsetstrokecolor{currentstroke}%
\pgfsetdash{}{0pt}%
\pgfpathmoveto{\pgfqpoint{0.938923in}{1.179135in}}%
\pgfpathlineto{\pgfqpoint{0.900837in}{1.145732in}}%
\pgfusepath{stroke}%
\end{pgfscope}%
\begin{pgfscope}%
\definecolor{textcolor}{rgb}{0.000000,0.000000,0.000000}%
\pgfsetstrokecolor{textcolor}%
\pgfsetfillcolor{textcolor}%
\pgftext[x=0.822109in,y=0.947497in,,top]{\color{textcolor}\sffamily\fontsize{10.000000}{12.000000}\selectfont \(\displaystyle {\ensuremath{-}0.4}\)}%
\end{pgfscope}%
\begin{pgfscope}%
\pgfsetrectcap%
\pgfsetroundjoin%
\pgfsetlinewidth{0.803000pt}%
\definecolor{currentstroke}{rgb}{0.000000,0.000000,0.000000}%
\pgfsetstrokecolor{currentstroke}%
\pgfsetdash{}{0pt}%
\pgfpathmoveto{\pgfqpoint{0.489555in}{1.325753in}}%
\pgfpathlineto{\pgfqpoint{0.450868in}{1.292983in}}%
\pgfusepath{stroke}%
\end{pgfscope}%
\begin{pgfscope}%
\definecolor{textcolor}{rgb}{0.000000,0.000000,0.000000}%
\pgfsetstrokecolor{textcolor}%
\pgfsetfillcolor{textcolor}%
\pgftext[x=0.372164in,y=1.096878in,,top]{\color{textcolor}\sffamily\fontsize{10.000000}{12.000000}\selectfont \(\displaystyle {\ensuremath{-}0.5}\)}%
\end{pgfscope}%
\begin{pgfscope}%
\pgfsetrectcap%
\pgfsetroundjoin%
\pgfsetlinewidth{0.803000pt}%
\definecolor{currentstroke}{rgb}{0.000000,0.000000,0.000000}%
\pgfsetstrokecolor{currentstroke}%
\pgfsetdash{}{0pt}%
\pgfpathmoveto{\pgfqpoint{4.210403in}{1.912088in}}%
\pgfpathlineto{\pgfqpoint{2.867547in}{0.533987in}}%
\pgfusepath{stroke}%
\end{pgfscope}%
\begin{pgfscope}%
\definecolor{textcolor}{rgb}{0.000000,0.000000,0.000000}%
\pgfsetstrokecolor{textcolor}%
\pgfsetfillcolor{textcolor}%
\pgftext[x=3.874376in, y=0.744198in, left, base,rotate=45.742112]{\color{textcolor}\sffamily\fontsize{10.000000}{12.000000}\selectfont \(\displaystyle t/M\)}%
\end{pgfscope}%
\begin{pgfscope}%
\pgfsetbuttcap%
\pgfsetroundjoin%
\pgfsetlinewidth{0.803000pt}%
\definecolor{currentstroke}{rgb}{0.690196,0.690196,0.690196}%
\pgfsetstrokecolor{currentstroke}%
\pgfsetdash{}{0pt}%
\pgfpathmoveto{\pgfqpoint{0.355690in}{3.343852in}}%
\pgfpathlineto{\pgfqpoint{0.442182in}{1.445390in}}%
\pgfpathlineto{\pgfqpoint{2.960462in}{0.629341in}}%
\pgfusepath{stroke}%
\end{pgfscope}%
\begin{pgfscope}%
\pgfsetbuttcap%
\pgfsetroundjoin%
\pgfsetlinewidth{0.803000pt}%
\definecolor{currentstroke}{rgb}{0.690196,0.690196,0.690196}%
\pgfsetstrokecolor{currentstroke}%
\pgfsetdash{}{0pt}%
\pgfpathmoveto{\pgfqpoint{0.715443in}{3.606383in}}%
\pgfpathlineto{\pgfqpoint{0.784684in}{1.732482in}}%
\pgfpathlineto{\pgfqpoint{3.270825in}{0.947850in}}%
\pgfusepath{stroke}%
\end{pgfscope}%
\begin{pgfscope}%
\pgfsetbuttcap%
\pgfsetroundjoin%
\pgfsetlinewidth{0.803000pt}%
\definecolor{currentstroke}{rgb}{0.690196,0.690196,0.690196}%
\pgfsetstrokecolor{currentstroke}%
\pgfsetdash{}{0pt}%
\pgfpathmoveto{\pgfqpoint{1.061228in}{3.858721in}}%
\pgfpathlineto{\pgfqpoint{1.114453in}{2.008901in}}%
\pgfpathlineto{\pgfqpoint{3.569053in}{1.253905in}}%
\pgfusepath{stroke}%
\end{pgfscope}%
\begin{pgfscope}%
\pgfsetbuttcap%
\pgfsetroundjoin%
\pgfsetlinewidth{0.803000pt}%
\definecolor{currentstroke}{rgb}{0.690196,0.690196,0.690196}%
\pgfsetstrokecolor{currentstroke}%
\pgfsetdash{}{0pt}%
\pgfpathmoveto{\pgfqpoint{1.393843in}{4.101449in}}%
\pgfpathlineto{\pgfqpoint{1.432188in}{2.275233in}}%
\pgfpathlineto{\pgfqpoint{3.855846in}{1.548225in}}%
\pgfusepath{stroke}%
\end{pgfscope}%
\begin{pgfscope}%
\pgfsetbuttcap%
\pgfsetroundjoin%
\pgfsetlinewidth{0.803000pt}%
\definecolor{currentstroke}{rgb}{0.690196,0.690196,0.690196}%
\pgfsetstrokecolor{currentstroke}%
\pgfsetdash{}{0pt}%
\pgfpathmoveto{\pgfqpoint{1.714026in}{4.335104in}}%
\pgfpathlineto{\pgfqpoint{1.738534in}{2.532019in}}%
\pgfpathlineto{\pgfqpoint{4.131848in}{1.831471in}}%
\pgfusepath{stroke}%
\end{pgfscope}%
\begin{pgfscope}%
\pgfsetrectcap%
\pgfsetroundjoin%
\pgfsetlinewidth{0.803000pt}%
\definecolor{currentstroke}{rgb}{0.000000,0.000000,0.000000}%
\pgfsetstrokecolor{currentstroke}%
\pgfsetdash{}{0pt}%
\pgfpathmoveto{\pgfqpoint{2.939241in}{0.636218in}}%
\pgfpathlineto{\pgfqpoint{3.002961in}{0.615570in}}%
\pgfusepath{stroke}%
\end{pgfscope}%
\begin{pgfscope}%
\definecolor{textcolor}{rgb}{0.000000,0.000000,0.000000}%
\pgfsetstrokecolor{textcolor}%
\pgfsetfillcolor{textcolor}%
\pgftext[x=3.139341in,y=0.442927in,,top]{\color{textcolor}\sffamily\fontsize{10.000000}{12.000000}\selectfont \(\displaystyle {0.00}\)}%
\end{pgfscope}%
\begin{pgfscope}%
\pgfsetrectcap%
\pgfsetroundjoin%
\pgfsetlinewidth{0.803000pt}%
\definecolor{currentstroke}{rgb}{0.000000,0.000000,0.000000}%
\pgfsetstrokecolor{currentstroke}%
\pgfsetdash{}{0pt}%
\pgfpathmoveto{\pgfqpoint{3.249895in}{0.954455in}}%
\pgfpathlineto{\pgfqpoint{3.312737in}{0.934622in}}%
\pgfusepath{stroke}%
\end{pgfscope}%
\begin{pgfscope}%
\definecolor{textcolor}{rgb}{0.000000,0.000000,0.000000}%
\pgfsetstrokecolor{textcolor}%
\pgfsetfillcolor{textcolor}%
\pgftext[x=3.446285in,y=0.765307in,,top]{\color{textcolor}\sffamily\fontsize{10.000000}{12.000000}\selectfont \(\displaystyle {0.02}\)}%
\end{pgfscope}%
\begin{pgfscope}%
\pgfsetrectcap%
\pgfsetroundjoin%
\pgfsetlinewidth{0.803000pt}%
\definecolor{currentstroke}{rgb}{0.000000,0.000000,0.000000}%
\pgfsetstrokecolor{currentstroke}%
\pgfsetdash{}{0pt}%
\pgfpathmoveto{\pgfqpoint{3.548409in}{1.260255in}}%
\pgfpathlineto{\pgfqpoint{3.610393in}{1.241190in}}%
\pgfusepath{stroke}%
\end{pgfscope}%
\begin{pgfscope}%
\definecolor{textcolor}{rgb}{0.000000,0.000000,0.000000}%
\pgfsetstrokecolor{textcolor}%
\pgfsetfillcolor{textcolor}%
\pgftext[x=3.741223in,y=1.075077in,,top]{\color{textcolor}\sffamily\fontsize{10.000000}{12.000000}\selectfont \(\displaystyle {0.04}\)}%
\end{pgfscope}%
\begin{pgfscope}%
\pgfsetrectcap%
\pgfsetroundjoin%
\pgfsetlinewidth{0.803000pt}%
\definecolor{currentstroke}{rgb}{0.000000,0.000000,0.000000}%
\pgfsetstrokecolor{currentstroke}%
\pgfsetdash{}{0pt}%
\pgfpathmoveto{\pgfqpoint{3.835481in}{1.554334in}}%
\pgfpathlineto{\pgfqpoint{3.896625in}{1.535993in}}%
\pgfusepath{stroke}%
\end{pgfscope}%
\begin{pgfscope}%
\definecolor{textcolor}{rgb}{0.000000,0.000000,0.000000}%
\pgfsetstrokecolor{textcolor}%
\pgfsetfillcolor{textcolor}%
\pgftext[x=4.024846in,y=1.372964in,,top]{\color{textcolor}\sffamily\fontsize{10.000000}{12.000000}\selectfont \(\displaystyle {0.06}\)}%
\end{pgfscope}%
\begin{pgfscope}%
\pgfsetrectcap%
\pgfsetroundjoin%
\pgfsetlinewidth{0.803000pt}%
\definecolor{currentstroke}{rgb}{0.000000,0.000000,0.000000}%
\pgfsetstrokecolor{currentstroke}%
\pgfsetdash{}{0pt}%
\pgfpathmoveto{\pgfqpoint{4.111756in}{1.837352in}}%
\pgfpathlineto{\pgfqpoint{4.172079in}{1.819695in}}%
\pgfusepath{stroke}%
\end{pgfscope}%
\begin{pgfscope}%
\definecolor{textcolor}{rgb}{0.000000,0.000000,0.000000}%
\pgfsetstrokecolor{textcolor}%
\pgfsetfillcolor{textcolor}%
\pgftext[x=4.297792in,y=1.659637in,,top]{\color{textcolor}\sffamily\fontsize{10.000000}{12.000000}\selectfont \(\displaystyle {0.08}\)}%
\end{pgfscope}%
\begin{pgfscope}%
\pgfsetrectcap%
\pgfsetroundjoin%
\pgfsetlinewidth{0.803000pt}%
\definecolor{currentstroke}{rgb}{0.000000,0.000000,0.000000}%
\pgfsetstrokecolor{currentstroke}%
\pgfsetdash{}{0pt}%
\pgfpathmoveto{\pgfqpoint{4.210403in}{1.912088in}}%
\pgfpathlineto{\pgfqpoint{4.295500in}{3.770391in}}%
\pgfusepath{stroke}%
\end{pgfscope}%
\begin{pgfscope}%
\definecolor{textcolor}{rgb}{0.000000,0.000000,0.000000}%
\pgfsetstrokecolor{textcolor}%
\pgfsetfillcolor{textcolor}%
\pgftext[x=4.839755in, y=2.711280in, left, base,rotate=87.378092]{\color{textcolor}\sffamily\fontsize{10.000000}{12.000000}\selectfont \(\displaystyle \int\gamma_{xx}\)}%
\end{pgfscope}%
\begin{pgfscope}%
\pgfsetbuttcap%
\pgfsetroundjoin%
\pgfsetlinewidth{0.803000pt}%
\definecolor{currentstroke}{rgb}{0.690196,0.690196,0.690196}%
\pgfsetstrokecolor{currentstroke}%
\pgfsetdash{}{0pt}%
\pgfpathmoveto{\pgfqpoint{4.215651in}{2.026679in}}%
\pgfpathlineto{\pgfqpoint{1.824552in}{2.716185in}}%
\pgfpathlineto{\pgfqpoint{0.334131in}{1.476884in}}%
\pgfusepath{stroke}%
\end{pgfscope}%
\begin{pgfscope}%
\pgfsetbuttcap%
\pgfsetroundjoin%
\pgfsetlinewidth{0.803000pt}%
\definecolor{currentstroke}{rgb}{0.690196,0.690196,0.690196}%
\pgfsetstrokecolor{currentstroke}%
\pgfsetdash{}{0pt}%
\pgfpathmoveto{\pgfqpoint{4.225771in}{2.247674in}}%
\pgfpathlineto{\pgfqpoint{1.822091in}{2.930163in}}%
\pgfpathlineto{\pgfqpoint{0.323231in}{1.703229in}}%
\pgfusepath{stroke}%
\end{pgfscope}%
\begin{pgfscope}%
\pgfsetbuttcap%
\pgfsetroundjoin%
\pgfsetlinewidth{0.803000pt}%
\definecolor{currentstroke}{rgb}{0.690196,0.690196,0.690196}%
\pgfsetstrokecolor{currentstroke}%
\pgfsetdash{}{0pt}%
\pgfpathmoveto{\pgfqpoint{4.235999in}{2.471030in}}%
\pgfpathlineto{\pgfqpoint{1.819605in}{3.146319in}}%
\pgfpathlineto{\pgfqpoint{0.312210in}{1.932083in}}%
\pgfusepath{stroke}%
\end{pgfscope}%
\begin{pgfscope}%
\pgfsetbuttcap%
\pgfsetroundjoin%
\pgfsetlinewidth{0.803000pt}%
\definecolor{currentstroke}{rgb}{0.690196,0.690196,0.690196}%
\pgfsetstrokecolor{currentstroke}%
\pgfsetdash{}{0pt}%
\pgfpathmoveto{\pgfqpoint{4.246337in}{2.696785in}}%
\pgfpathlineto{\pgfqpoint{1.817094in}{3.364684in}}%
\pgfpathlineto{\pgfqpoint{0.301066in}{2.163490in}}%
\pgfusepath{stroke}%
\end{pgfscope}%
\begin{pgfscope}%
\pgfsetbuttcap%
\pgfsetroundjoin%
\pgfsetlinewidth{0.803000pt}%
\definecolor{currentstroke}{rgb}{0.690196,0.690196,0.690196}%
\pgfsetstrokecolor{currentstroke}%
\pgfsetdash{}{0pt}%
\pgfpathmoveto{\pgfqpoint{4.256786in}{2.924978in}}%
\pgfpathlineto{\pgfqpoint{1.814558in}{3.585293in}}%
\pgfpathlineto{\pgfqpoint{0.289798in}{2.397491in}}%
\pgfusepath{stroke}%
\end{pgfscope}%
\begin{pgfscope}%
\pgfsetbuttcap%
\pgfsetroundjoin%
\pgfsetlinewidth{0.803000pt}%
\definecolor{currentstroke}{rgb}{0.690196,0.690196,0.690196}%
\pgfsetstrokecolor{currentstroke}%
\pgfsetdash{}{0pt}%
\pgfpathmoveto{\pgfqpoint{4.267349in}{3.155647in}}%
\pgfpathlineto{\pgfqpoint{1.811995in}{3.808182in}}%
\pgfpathlineto{\pgfqpoint{0.278402in}{2.634132in}}%
\pgfusepath{stroke}%
\end{pgfscope}%
\begin{pgfscope}%
\pgfsetbuttcap%
\pgfsetroundjoin%
\pgfsetlinewidth{0.803000pt}%
\definecolor{currentstroke}{rgb}{0.690196,0.690196,0.690196}%
\pgfsetstrokecolor{currentstroke}%
\pgfsetdash{}{0pt}%
\pgfpathmoveto{\pgfqpoint{4.278028in}{3.388835in}}%
\pgfpathlineto{\pgfqpoint{1.809405in}{4.033384in}}%
\pgfpathlineto{\pgfqpoint{0.266877in}{2.873456in}}%
\pgfusepath{stroke}%
\end{pgfscope}%
\begin{pgfscope}%
\pgfsetbuttcap%
\pgfsetroundjoin%
\pgfsetlinewidth{0.803000pt}%
\definecolor{currentstroke}{rgb}{0.690196,0.690196,0.690196}%
\pgfsetstrokecolor{currentstroke}%
\pgfsetdash{}{0pt}%
\pgfpathmoveto{\pgfqpoint{4.288823in}{3.624582in}}%
\pgfpathlineto{\pgfqpoint{1.806788in}{4.260937in}}%
\pgfpathlineto{\pgfqpoint{0.255221in}{3.115510in}}%
\pgfusepath{stroke}%
\end{pgfscope}%
\begin{pgfscope}%
\pgfsetrectcap%
\pgfsetroundjoin%
\pgfsetlinewidth{0.803000pt}%
\definecolor{currentstroke}{rgb}{0.000000,0.000000,0.000000}%
\pgfsetstrokecolor{currentstroke}%
\pgfsetdash{}{0pt}%
\pgfpathmoveto{\pgfqpoint{4.195580in}{2.032466in}}%
\pgfpathlineto{\pgfqpoint{4.255839in}{2.015090in}}%
\pgfusepath{stroke}%
\end{pgfscope}%
\begin{pgfscope}%
\definecolor{textcolor}{rgb}{0.000000,0.000000,0.000000}%
\pgfsetstrokecolor{textcolor}%
\pgfsetfillcolor{textcolor}%
\pgftext[x=4.469926in,y=2.062695in,,top]{\color{textcolor}\sffamily\fontsize{10.000000}{12.000000}\selectfont \(\displaystyle {25}\)}%
\end{pgfscope}%
\begin{pgfscope}%
\pgfsetrectcap%
\pgfsetroundjoin%
\pgfsetlinewidth{0.803000pt}%
\definecolor{currentstroke}{rgb}{0.000000,0.000000,0.000000}%
\pgfsetstrokecolor{currentstroke}%
\pgfsetdash{}{0pt}%
\pgfpathmoveto{\pgfqpoint{4.205589in}{2.253404in}}%
\pgfpathlineto{\pgfqpoint{4.266181in}{2.236200in}}%
\pgfusepath{stroke}%
\end{pgfscope}%
\begin{pgfscope}%
\definecolor{textcolor}{rgb}{0.000000,0.000000,0.000000}%
\pgfsetstrokecolor{textcolor}%
\pgfsetfillcolor{textcolor}%
\pgftext[x=4.481368in,y=2.283332in,,top]{\color{textcolor}\sffamily\fontsize{10.000000}{12.000000}\selectfont \(\displaystyle {50}\)}%
\end{pgfscope}%
\begin{pgfscope}%
\pgfsetrectcap%
\pgfsetroundjoin%
\pgfsetlinewidth{0.803000pt}%
\definecolor{currentstroke}{rgb}{0.000000,0.000000,0.000000}%
\pgfsetstrokecolor{currentstroke}%
\pgfsetdash{}{0pt}%
\pgfpathmoveto{\pgfqpoint{4.215706in}{2.476701in}}%
\pgfpathlineto{\pgfqpoint{4.276634in}{2.459674in}}%
\pgfusepath{stroke}%
\end{pgfscope}%
\begin{pgfscope}%
\definecolor{textcolor}{rgb}{0.000000,0.000000,0.000000}%
\pgfsetstrokecolor{textcolor}%
\pgfsetfillcolor{textcolor}%
\pgftext[x=4.492932in,y=2.506321in,,top]{\color{textcolor}\sffamily\fontsize{10.000000}{12.000000}\selectfont \(\displaystyle {75}\)}%
\end{pgfscope}%
\begin{pgfscope}%
\pgfsetrectcap%
\pgfsetroundjoin%
\pgfsetlinewidth{0.803000pt}%
\definecolor{currentstroke}{rgb}{0.000000,0.000000,0.000000}%
\pgfsetstrokecolor{currentstroke}%
\pgfsetdash{}{0pt}%
\pgfpathmoveto{\pgfqpoint{4.225931in}{2.702395in}}%
\pgfpathlineto{\pgfqpoint{4.287198in}{2.685550in}}%
\pgfusepath{stroke}%
\end{pgfscope}%
\begin{pgfscope}%
\definecolor{textcolor}{rgb}{0.000000,0.000000,0.000000}%
\pgfsetstrokecolor{textcolor}%
\pgfsetfillcolor{textcolor}%
\pgftext[x=4.504621in,y=2.731698in,,top]{\color{textcolor}\sffamily\fontsize{10.000000}{12.000000}\selectfont \(\displaystyle {100}\)}%
\end{pgfscope}%
\begin{pgfscope}%
\pgfsetrectcap%
\pgfsetroundjoin%
\pgfsetlinewidth{0.803000pt}%
\definecolor{currentstroke}{rgb}{0.000000,0.000000,0.000000}%
\pgfsetstrokecolor{currentstroke}%
\pgfsetdash{}{0pt}%
\pgfpathmoveto{\pgfqpoint{4.236266in}{2.930526in}}%
\pgfpathlineto{\pgfqpoint{4.297877in}{2.913868in}}%
\pgfusepath{stroke}%
\end{pgfscope}%
\begin{pgfscope}%
\definecolor{textcolor}{rgb}{0.000000,0.000000,0.000000}%
\pgfsetstrokecolor{textcolor}%
\pgfsetfillcolor{textcolor}%
\pgftext[x=4.516435in,y=2.959503in,,top]{\color{textcolor}\sffamily\fontsize{10.000000}{12.000000}\selectfont \(\displaystyle {125}\)}%
\end{pgfscope}%
\begin{pgfscope}%
\pgfsetrectcap%
\pgfsetroundjoin%
\pgfsetlinewidth{0.803000pt}%
\definecolor{currentstroke}{rgb}{0.000000,0.000000,0.000000}%
\pgfsetstrokecolor{currentstroke}%
\pgfsetdash{}{0pt}%
\pgfpathmoveto{\pgfqpoint{4.246713in}{3.161132in}}%
\pgfpathlineto{\pgfqpoint{4.308672in}{3.144665in}}%
\pgfusepath{stroke}%
\end{pgfscope}%
\begin{pgfscope}%
\definecolor{textcolor}{rgb}{0.000000,0.000000,0.000000}%
\pgfsetstrokecolor{textcolor}%
\pgfsetfillcolor{textcolor}%
\pgftext[x=4.528377in,y=3.189774in,,top]{\color{textcolor}\sffamily\fontsize{10.000000}{12.000000}\selectfont \(\displaystyle {150}\)}%
\end{pgfscope}%
\begin{pgfscope}%
\pgfsetrectcap%
\pgfsetroundjoin%
\pgfsetlinewidth{0.803000pt}%
\definecolor{currentstroke}{rgb}{0.000000,0.000000,0.000000}%
\pgfsetstrokecolor{currentstroke}%
\pgfsetdash{}{0pt}%
\pgfpathmoveto{\pgfqpoint{4.257275in}{3.394254in}}%
\pgfpathlineto{\pgfqpoint{4.319585in}{3.377985in}}%
\pgfusepath{stroke}%
\end{pgfscope}%
\begin{pgfscope}%
\definecolor{textcolor}{rgb}{0.000000,0.000000,0.000000}%
\pgfsetstrokecolor{textcolor}%
\pgfsetfillcolor{textcolor}%
\pgftext[x=4.540449in,y=3.422553in,,top]{\color{textcolor}\sffamily\fontsize{10.000000}{12.000000}\selectfont \(\displaystyle {175}\)}%
\end{pgfscope}%
\begin{pgfscope}%
\pgfsetrectcap%
\pgfsetroundjoin%
\pgfsetlinewidth{0.803000pt}%
\definecolor{currentstroke}{rgb}{0.000000,0.000000,0.000000}%
\pgfsetstrokecolor{currentstroke}%
\pgfsetdash{}{0pt}%
\pgfpathmoveto{\pgfqpoint{4.267952in}{3.629933in}}%
\pgfpathlineto{\pgfqpoint{4.330618in}{3.613867in}}%
\pgfusepath{stroke}%
\end{pgfscope}%
\begin{pgfscope}%
\definecolor{textcolor}{rgb}{0.000000,0.000000,0.000000}%
\pgfsetstrokecolor{textcolor}%
\pgfsetfillcolor{textcolor}%
\pgftext[x=4.552653in,y=3.657880in,,top]{\color{textcolor}\sffamily\fontsize{10.000000}{12.000000}\selectfont \(\displaystyle {200}\)}%
\end{pgfscope}%
\begin{pgfscope}%
\pgfpathrectangle{\pgfqpoint{0.000000in}{0.250000in}}{\pgfqpoint{4.500000in}{4.500000in}}%
\pgfusepath{clip}%
\pgfsetbuttcap%
\pgfsetroundjoin%
\pgfsetlinewidth{1.003750pt}%
\definecolor{currentstroke}{rgb}{0.121569,0.466667,0.705882}%
\pgfsetstrokecolor{currentstroke}%
\pgfsetdash{}{0pt}%
\pgfpathmoveto{\pgfqpoint{0.510977in}{3.000769in}}%
\pgfpathlineto{\pgfqpoint{0.557887in}{2.967375in}}%
\pgfpathlineto{\pgfqpoint{0.604856in}{2.933673in}}%
\pgfpathlineto{\pgfqpoint{0.651884in}{2.899650in}}%
\pgfpathlineto{\pgfqpoint{0.698970in}{2.865289in}}%
\pgfpathlineto{\pgfqpoint{0.746115in}{2.830575in}}%
\pgfpathlineto{\pgfqpoint{0.793318in}{2.795489in}}%
\pgfpathlineto{\pgfqpoint{0.840579in}{2.760012in}}%
\pgfpathlineto{\pgfqpoint{0.887896in}{2.724123in}}%
\pgfpathlineto{\pgfqpoint{0.935271in}{2.687800in}}%
\pgfpathlineto{\pgfqpoint{0.982702in}{2.651016in}}%
\pgfpathlineto{\pgfqpoint{1.030188in}{2.613746in}}%
\pgfpathlineto{\pgfqpoint{1.077730in}{2.575959in}}%
\pgfpathlineto{\pgfqpoint{1.125326in}{2.537621in}}%
\pgfpathlineto{\pgfqpoint{1.172976in}{2.498698in}}%
\pgfpathlineto{\pgfqpoint{1.220680in}{2.459148in}}%
\pgfpathlineto{\pgfqpoint{1.268435in}{2.418927in}}%
\pgfpathlineto{\pgfqpoint{1.316242in}{2.377985in}}%
\pgfpathlineto{\pgfqpoint{1.364099in}{2.336268in}}%
\pgfpathlineto{\pgfqpoint{1.412006in}{2.293712in}}%
\pgfpathlineto{\pgfqpoint{1.459960in}{2.250248in}}%
\pgfpathlineto{\pgfqpoint{1.507961in}{2.205796in}}%
\pgfpathlineto{\pgfqpoint{1.556006in}{2.160267in}}%
\pgfpathlineto{\pgfqpoint{1.604095in}{2.113558in}}%
\pgfpathlineto{\pgfqpoint{1.652225in}{2.065550in}}%
\pgfpathlineto{\pgfqpoint{1.700393in}{2.016106in}}%
\pgfpathlineto{\pgfqpoint{1.748598in}{1.965069in}}%
\pgfpathlineto{\pgfqpoint{1.796835in}{1.912252in}}%
\pgfpathlineto{\pgfqpoint{1.845101in}{1.857436in}}%
\pgfpathlineto{\pgfqpoint{1.893392in}{1.800361in}}%
\pgfpathlineto{\pgfqpoint{1.941703in}{1.740714in}}%
\pgfpathlineto{\pgfqpoint{1.990028in}{1.678116in}}%
\pgfpathlineto{\pgfqpoint{2.038360in}{1.612102in}}%
\pgfpathlineto{\pgfqpoint{2.086689in}{1.542097in}}%
\pgfpathlineto{\pgfqpoint{2.135006in}{1.467377in}}%
\pgfpathlineto{\pgfqpoint{2.183295in}{1.387020in}}%
\pgfpathlineto{\pgfqpoint{2.231540in}{1.299825in}}%
\pgfpathlineto{\pgfqpoint{2.279717in}{1.204206in}}%
\pgfpathlineto{\pgfqpoint{2.327795in}{1.098013in}}%
\pgfpathlineto{\pgfqpoint{2.375731in}{0.978257in}}%
\pgfpathlineto{\pgfqpoint{2.423465in}{0.840658in}}%
\pgfusepath{stroke}%
\end{pgfscope}%
\begin{pgfscope}%
\pgfpathrectangle{\pgfqpoint{0.000000in}{0.250000in}}{\pgfqpoint{4.500000in}{4.500000in}}%
\pgfusepath{clip}%
\pgfsetbuttcap%
\pgfsetroundjoin%
\pgfsetlinewidth{1.003750pt}%
\definecolor{currentstroke}{rgb}{0.121569,0.466667,0.705882}%
\pgfsetstrokecolor{currentstroke}%
\pgfsetdash{}{0pt}%
\pgfpathmoveto{\pgfqpoint{0.715718in}{3.155367in}}%
\pgfpathlineto{\pgfqpoint{0.762199in}{3.122431in}}%
\pgfpathlineto{\pgfqpoint{0.808736in}{3.089190in}}%
\pgfpathlineto{\pgfqpoint{0.855330in}{3.055631in}}%
\pgfpathlineto{\pgfqpoint{0.901979in}{3.021738in}}%
\pgfpathlineto{\pgfqpoint{0.948685in}{2.987494in}}%
\pgfpathlineto{\pgfqpoint{0.995446in}{2.952882in}}%
\pgfpathlineto{\pgfqpoint{1.042261in}{2.917883in}}%
\pgfpathlineto{\pgfqpoint{1.089131in}{2.882476in}}%
\pgfpathlineto{\pgfqpoint{1.136055in}{2.846638in}}%
\pgfpathlineto{\pgfqpoint{1.183032in}{2.810344in}}%
\pgfpathlineto{\pgfqpoint{1.230061in}{2.773567in}}%
\pgfpathlineto{\pgfqpoint{1.277143in}{2.736278in}}%
\pgfpathlineto{\pgfqpoint{1.324275in}{2.698444in}}%
\pgfpathlineto{\pgfqpoint{1.371458in}{2.660028in}}%
\pgfpathlineto{\pgfqpoint{1.418690in}{2.620992in}}%
\pgfpathlineto{\pgfqpoint{1.465971in}{2.581290in}}%
\pgfpathlineto{\pgfqpoint{1.513299in}{2.540873in}}%
\pgfpathlineto{\pgfqpoint{1.560672in}{2.499687in}}%
\pgfpathlineto{\pgfqpoint{1.608091in}{2.457669in}}%
\pgfpathlineto{\pgfqpoint{1.655552in}{2.414750in}}%
\pgfpathlineto{\pgfqpoint{1.703055in}{2.370852in}}%
\pgfpathlineto{\pgfqpoint{1.750596in}{2.325885in}}%
\pgfpathlineto{\pgfqpoint{1.798175in}{2.279746in}}%
\pgfpathlineto{\pgfqpoint{1.845788in}{2.232319in}}%
\pgfpathlineto{\pgfqpoint{1.893433in}{2.183468in}}%
\pgfpathlineto{\pgfqpoint{1.941106in}{2.133035in}}%
\pgfpathlineto{\pgfqpoint{1.988803in}{2.080836in}}%
\pgfpathlineto{\pgfqpoint{2.036519in}{2.026653in}}%
\pgfpathlineto{\pgfqpoint{2.084250in}{1.970227in}}%
\pgfpathlineto{\pgfqpoint{2.131989in}{1.911248in}}%
\pgfpathlineto{\pgfqpoint{2.179729in}{1.849340in}}%
\pgfpathlineto{\pgfqpoint{2.227459in}{1.784041in}}%
\pgfpathlineto{\pgfqpoint{2.275170in}{1.714780in}}%
\pgfpathlineto{\pgfqpoint{2.322846in}{1.640837in}}%
\pgfpathlineto{\pgfqpoint{2.370471in}{1.561297in}}%
\pgfpathlineto{\pgfqpoint{2.418021in}{1.474967in}}%
\pgfpathlineto{\pgfqpoint{2.465466in}{1.380272in}}%
\pgfpathlineto{\pgfqpoint{2.512766in}{1.275074in}}%
\pgfpathlineto{\pgfqpoint{2.559867in}{1.156407in}}%
\pgfpathlineto{\pgfqpoint{2.606688in}{1.020015in}}%
\pgfusepath{stroke}%
\end{pgfscope}%
\begin{pgfscope}%
\pgfpathrectangle{\pgfqpoint{0.000000in}{0.250000in}}{\pgfqpoint{4.500000in}{4.500000in}}%
\pgfusepath{clip}%
\pgfsetbuttcap%
\pgfsetroundjoin%
\pgfsetlinewidth{1.003750pt}%
\definecolor{currentstroke}{rgb}{0.121569,0.466667,0.705882}%
\pgfsetstrokecolor{currentstroke}%
\pgfsetdash{}{0pt}%
\pgfpathmoveto{\pgfqpoint{0.915333in}{3.322655in}}%
\pgfpathlineto{\pgfqpoint{0.961409in}{3.290161in}}%
\pgfpathlineto{\pgfqpoint{1.007538in}{3.257365in}}%
\pgfpathlineto{\pgfqpoint{1.053729in}{3.224037in}}%
\pgfpathlineto{\pgfqpoint{1.099973in}{3.190367in}}%
\pgfpathlineto{\pgfqpoint{1.146269in}{3.156338in}}%
\pgfpathlineto{\pgfqpoint{1.192619in}{3.121933in}}%
\pgfpathlineto{\pgfqpoint{1.239020in}{3.087132in}}%
\pgfpathlineto{\pgfqpoint{1.285473in}{3.051913in}}%
\pgfpathlineto{\pgfqpoint{1.331977in}{3.016253in}}%
\pgfpathlineto{\pgfqpoint{1.378530in}{2.980126in}}%
\pgfpathlineto{\pgfqpoint{1.425133in}{2.943505in}}%
\pgfpathlineto{\pgfqpoint{1.471785in}{2.906359in}}%
\pgfpathlineto{\pgfqpoint{1.518483in}{2.868656in}}%
\pgfpathlineto{\pgfqpoint{1.565229in}{2.830357in}}%
\pgfpathlineto{\pgfqpoint{1.612019in}{2.791423in}}%
\pgfpathlineto{\pgfqpoint{1.658854in}{2.751809in}}%
\pgfpathlineto{\pgfqpoint{1.705731in}{2.711465in}}%
\pgfpathlineto{\pgfqpoint{1.752649in}{2.670335in}}%
\pgfpathlineto{\pgfqpoint{1.799607in}{2.628357in}}%
\pgfpathlineto{\pgfqpoint{1.846603in}{2.585461in}}%
\pgfpathlineto{\pgfqpoint{1.893633in}{2.541568in}}%
\pgfpathlineto{\pgfqpoint{1.940697in}{2.496590in}}%
\pgfpathlineto{\pgfqpoint{1.987790in}{2.450423in}}%
\pgfpathlineto{\pgfqpoint{2.034911in}{2.402953in}}%
\pgfpathlineto{\pgfqpoint{2.082055in}{2.354044in}}%
\pgfpathlineto{\pgfqpoint{2.129219in}{2.303542in}}%
\pgfpathlineto{\pgfqpoint{2.176397in}{2.251267in}}%
\pgfpathlineto{\pgfqpoint{2.223584in}{2.197005in}}%
\pgfpathlineto{\pgfqpoint{2.270774in}{2.140505in}}%
\pgfpathlineto{\pgfqpoint{2.317959in}{2.081466in}}%
\pgfpathlineto{\pgfqpoint{2.365130in}{2.019523in}}%
\pgfpathlineto{\pgfqpoint{2.412276in}{1.954234in}}%
\pgfpathlineto{\pgfqpoint{2.459383in}{1.885047in}}%
\pgfpathlineto{\pgfqpoint{2.506435in}{1.811271in}}%
\pgfpathlineto{\pgfqpoint{2.553410in}{1.732023in}}%
\pgfpathlineto{\pgfqpoint{2.600282in}{1.646151in}}%
\pgfpathlineto{\pgfqpoint{2.647014in}{1.552122in}}%
\pgfpathlineto{\pgfqpoint{2.693559in}{1.447839in}}%
\pgfpathlineto{\pgfqpoint{2.739849in}{1.330347in}}%
\pgfpathlineto{\pgfqpoint{2.785789in}{1.195337in}}%
\pgfusepath{stroke}%
\end{pgfscope}%
\begin{pgfscope}%
\pgfpathrectangle{\pgfqpoint{0.000000in}{0.250000in}}{\pgfqpoint{4.500000in}{4.500000in}}%
\pgfusepath{clip}%
\pgfsetbuttcap%
\pgfsetroundjoin%
\pgfsetlinewidth{1.003750pt}%
\definecolor{currentstroke}{rgb}{0.121569,0.466667,0.705882}%
\pgfsetstrokecolor{currentstroke}%
\pgfsetdash{}{0pt}%
\pgfpathmoveto{\pgfqpoint{1.110211in}{3.502356in}}%
\pgfpathlineto{\pgfqpoint{1.155905in}{3.470288in}}%
\pgfpathlineto{\pgfqpoint{1.201651in}{3.437922in}}%
\pgfpathlineto{\pgfqpoint{1.247465in}{3.404597in}}%
\pgfpathlineto{\pgfqpoint{1.293329in}{3.370912in}}%
\pgfpathlineto{\pgfqpoint{1.339244in}{3.336851in}}%
\pgfpathlineto{\pgfqpoint{1.385208in}{3.302393in}}%
\pgfpathlineto{\pgfqpoint{1.431222in}{3.267517in}}%
\pgfpathlineto{\pgfqpoint{1.477283in}{3.232201in}}%
\pgfpathlineto{\pgfqpoint{1.523392in}{3.196420in}}%
\pgfpathlineto{\pgfqpoint{1.569548in}{3.160147in}}%
\pgfpathlineto{\pgfqpoint{1.615749in}{3.123353in}}%
\pgfpathlineto{\pgfqpoint{1.661994in}{3.086006in}}%
\pgfpathlineto{\pgfqpoint{1.708283in}{3.048071in}}%
\pgfpathlineto{\pgfqpoint{1.754614in}{3.009509in}}%
\pgfpathlineto{\pgfqpoint{1.800986in}{2.970279in}}%
\pgfpathlineto{\pgfqpoint{1.847396in}{2.930334in}}%
\pgfpathlineto{\pgfqpoint{1.893845in}{2.889621in}}%
\pgfpathlineto{\pgfqpoint{1.940329in}{2.848085in}}%
\pgfpathlineto{\pgfqpoint{1.986846in}{2.805662in}}%
\pgfpathlineto{\pgfqpoint{2.033395in}{2.762280in}}%
\pgfpathlineto{\pgfqpoint{2.079972in}{2.717859in}}%
\pgfpathlineto{\pgfqpoint{2.126574in}{2.672311in}}%
\pgfpathlineto{\pgfqpoint{2.173199in}{2.625533in}}%
\pgfpathlineto{\pgfqpoint{2.219843in}{2.577410in}}%
\pgfpathlineto{\pgfqpoint{2.266501in}{2.527811in}}%
\pgfpathlineto{\pgfqpoint{2.313168in}{2.476585in}}%
\pgfpathlineto{\pgfqpoint{2.359839in}{2.423555in}}%
\pgfpathlineto{\pgfqpoint{2.406507in}{2.368519in}}%
\pgfpathlineto{\pgfqpoint{2.453165in}{2.311237in}}%
\pgfpathlineto{\pgfqpoint{2.499804in}{2.251425in}}%
\pgfpathlineto{\pgfqpoint{2.546414in}{2.188741in}}%
\pgfpathlineto{\pgfqpoint{2.592982in}{2.122770in}}%
\pgfpathlineto{\pgfqpoint{2.639492in}{2.053002in}}%
\pgfpathlineto{\pgfqpoint{2.685926in}{1.978794in}}%
\pgfpathlineto{\pgfqpoint{2.732260in}{1.899323in}}%
\pgfpathlineto{\pgfqpoint{2.778465in}{1.813510in}}%
\pgfpathlineto{\pgfqpoint{2.824499in}{1.719895in}}%
\pgfpathlineto{\pgfqpoint{2.870308in}{1.616446in}}%
\pgfpathlineto{\pgfqpoint{2.915815in}{1.500217in}}%
\pgfpathlineto{\pgfqpoint{2.960904in}{1.366758in}}%
\pgfusepath{stroke}%
\end{pgfscope}%
\begin{pgfscope}%
\pgfpathrectangle{\pgfqpoint{0.000000in}{0.250000in}}{\pgfqpoint{4.500000in}{4.500000in}}%
\pgfusepath{clip}%
\pgfsetbuttcap%
\pgfsetroundjoin%
\pgfsetlinewidth{1.003750pt}%
\definecolor{currentstroke}{rgb}{0.121569,0.466667,0.705882}%
\pgfsetstrokecolor{currentstroke}%
\pgfsetdash{}{0pt}%
\pgfpathmoveto{\pgfqpoint{1.300766in}{3.692028in}}%
\pgfpathlineto{\pgfqpoint{1.346098in}{3.660374in}}%
\pgfpathlineto{\pgfqpoint{1.391480in}{3.628422in}}%
\pgfpathlineto{\pgfqpoint{1.436918in}{3.595894in}}%
\pgfpathlineto{\pgfqpoint{1.482405in}{3.562995in}}%
\pgfpathlineto{\pgfqpoint{1.527960in}{3.528652in}}%
\pgfpathlineto{\pgfqpoint{1.573561in}{3.493881in}}%
\pgfpathlineto{\pgfqpoint{1.619208in}{3.458658in}}%
\pgfpathlineto{\pgfqpoint{1.664899in}{3.422959in}}%
\pgfpathlineto{\pgfqpoint{1.710634in}{3.386757in}}%
\pgfpathlineto{\pgfqpoint{1.756411in}{3.350024in}}%
\pgfpathlineto{\pgfqpoint{1.802230in}{3.312727in}}%
\pgfpathlineto{\pgfqpoint{1.848089in}{3.274832in}}%
\pgfpathlineto{\pgfqpoint{1.893986in}{3.236303in}}%
\pgfpathlineto{\pgfqpoint{1.939920in}{3.197097in}}%
\pgfpathlineto{\pgfqpoint{1.985890in}{3.157170in}}%
\pgfpathlineto{\pgfqpoint{2.031893in}{3.116474in}}%
\pgfpathlineto{\pgfqpoint{2.077927in}{3.074954in}}%
\pgfpathlineto{\pgfqpoint{2.123990in}{3.032550in}}%
\pgfpathlineto{\pgfqpoint{2.170080in}{2.989197in}}%
\pgfpathlineto{\pgfqpoint{2.216194in}{2.944823in}}%
\pgfpathlineto{\pgfqpoint{2.262328in}{2.899345in}}%
\pgfpathlineto{\pgfqpoint{2.308480in}{2.852674in}}%
\pgfpathlineto{\pgfqpoint{2.354644in}{2.804709in}}%
\pgfpathlineto{\pgfqpoint{2.400817in}{2.755336in}}%
\pgfpathlineto{\pgfqpoint{2.446993in}{2.704427in}}%
\pgfpathlineto{\pgfqpoint{2.493168in}{2.651835in}}%
\pgfpathlineto{\pgfqpoint{2.539334in}{2.597396in}}%
\pgfpathlineto{\pgfqpoint{2.585485in}{2.540917in}}%
\pgfpathlineto{\pgfqpoint{2.631611in}{2.482176in}}%
\pgfpathlineto{\pgfqpoint{2.677703in}{2.420914in}}%
\pgfpathlineto{\pgfqpoint{2.723750in}{2.356821in}}%
\pgfpathlineto{\pgfqpoint{2.769737in}{2.289525in}}%
\pgfpathlineto{\pgfqpoint{2.815649in}{2.218570in}}%
\pgfpathlineto{\pgfqpoint{2.861465in}{2.143382in}}%
\pgfpathlineto{\pgfqpoint{2.907161in}{2.063224in}}%
\pgfpathlineto{\pgfqpoint{2.952704in}{1.977113in}}%
\pgfpathlineto{\pgfqpoint{2.998052in}{1.883694in}}%
\pgfpathlineto{\pgfqpoint{3.043144in}{1.781018in}}%
\pgfpathlineto{\pgfqpoint{3.087894in}{1.666145in}}%
\pgfpathlineto{\pgfqpoint{3.132167in}{1.534408in}}%
\pgfusepath{stroke}%
\end{pgfscope}%
\begin{pgfscope}%
\pgfpathrectangle{\pgfqpoint{0.000000in}{0.250000in}}{\pgfqpoint{4.500000in}{4.500000in}}%
\pgfusepath{clip}%
\pgfsetbuttcap%
\pgfsetroundjoin%
\pgfsetlinewidth{1.003750pt}%
\definecolor{currentstroke}{rgb}{0.121569,0.466667,0.705882}%
\pgfsetstrokecolor{currentstroke}%
\pgfsetdash{}{0pt}%
\pgfpathmoveto{\pgfqpoint{1.487400in}{3.886553in}}%
\pgfpathlineto{\pgfqpoint{1.532384in}{3.855299in}}%
\pgfpathlineto{\pgfqpoint{1.577417in}{3.823750in}}%
\pgfpathlineto{\pgfqpoint{1.622445in}{3.795208in}}%
\pgfpathlineto{\pgfqpoint{1.667530in}{3.766099in}}%
\pgfpathlineto{\pgfqpoint{1.712744in}{3.731223in}}%
\pgfpathlineto{\pgfqpoint{1.758000in}{3.695874in}}%
\pgfpathlineto{\pgfqpoint{1.803297in}{3.660028in}}%
\pgfpathlineto{\pgfqpoint{1.848635in}{3.623656in}}%
\pgfpathlineto{\pgfqpoint{1.894011in}{3.586729in}}%
\pgfpathlineto{\pgfqpoint{1.939426in}{3.549215in}}%
\pgfpathlineto{\pgfqpoint{1.984877in}{3.511079in}}%
\pgfpathlineto{\pgfqpoint{2.030362in}{3.472284in}}%
\pgfpathlineto{\pgfqpoint{2.075880in}{3.432790in}}%
\pgfpathlineto{\pgfqpoint{2.121430in}{3.392551in}}%
\pgfpathlineto{\pgfqpoint{2.167008in}{3.351520in}}%
\pgfpathlineto{\pgfqpoint{2.212613in}{3.309645in}}%
\pgfpathlineto{\pgfqpoint{2.258242in}{3.266868in}}%
\pgfpathlineto{\pgfqpoint{2.303892in}{3.223127in}}%
\pgfpathlineto{\pgfqpoint{2.349561in}{3.178354in}}%
\pgfpathlineto{\pgfqpoint{2.395244in}{3.132474in}}%
\pgfpathlineto{\pgfqpoint{2.440939in}{3.085405in}}%
\pgfpathlineto{\pgfqpoint{2.486640in}{3.037056in}}%
\pgfpathlineto{\pgfqpoint{2.532344in}{2.987327in}}%
\pgfpathlineto{\pgfqpoint{2.578045in}{2.936109in}}%
\pgfpathlineto{\pgfqpoint{2.623737in}{2.883278in}}%
\pgfpathlineto{\pgfqpoint{2.669415in}{2.828698in}}%
\pgfpathlineto{\pgfqpoint{2.715070in}{2.772214in}}%
\pgfpathlineto{\pgfqpoint{2.760695in}{2.713653in}}%
\pgfpathlineto{\pgfqpoint{2.806281in}{2.652818in}}%
\pgfpathlineto{\pgfqpoint{2.851817in}{2.589481in}}%
\pgfpathlineto{\pgfqpoint{2.897291in}{2.523374in}}%
\pgfpathlineto{\pgfqpoint{2.942688in}{2.454181in}}%
\pgfpathlineto{\pgfqpoint{2.987994in}{2.381513in}}%
\pgfpathlineto{\pgfqpoint{3.033187in}{2.304883in}}%
\pgfpathlineto{\pgfqpoint{3.078242in}{2.223656in}}%
\pgfpathlineto{\pgfqpoint{3.123127in}{2.136969in}}%
\pgfpathlineto{\pgfqpoint{3.167798in}{2.043589in}}%
\pgfpathlineto{\pgfqpoint{3.212192in}{1.941664in}}%
\pgfpathlineto{\pgfqpoint{3.256212in}{1.828250in}}%
\pgfpathlineto{\pgfqpoint{3.299702in}{1.698408in}}%
\pgfusepath{stroke}%
\end{pgfscope}%
\begin{pgfscope}%
\pgfpathrectangle{\pgfqpoint{0.000000in}{0.250000in}}{\pgfqpoint{4.500000in}{4.500000in}}%
\pgfusepath{clip}%
\pgfsetbuttcap%
\pgfsetroundjoin%
\pgfsetlinewidth{1.003750pt}%
\definecolor{currentstroke}{rgb}{0.121569,0.466667,0.705882}%
\pgfsetstrokecolor{currentstroke}%
\pgfsetdash{}{0pt}%
\pgfpathmoveto{\pgfqpoint{1.670315in}{4.087263in}}%
\pgfpathlineto{\pgfqpoint{1.714967in}{4.056398in}}%
\pgfpathlineto{\pgfqpoint{1.759665in}{4.025240in}}%
\pgfpathlineto{\pgfqpoint{1.804379in}{3.996347in}}%
\pgfpathlineto{\pgfqpoint{1.849145in}{3.967080in}}%
\pgfpathlineto{\pgfqpoint{1.893952in}{3.938262in}}%
\pgfpathlineto{\pgfqpoint{1.938822in}{3.907736in}}%
\pgfpathlineto{\pgfqpoint{1.983782in}{3.870992in}}%
\pgfpathlineto{\pgfqpoint{2.028778in}{3.833659in}}%
\pgfpathlineto{\pgfqpoint{2.073808in}{3.795704in}}%
\pgfpathlineto{\pgfqpoint{2.118870in}{3.757090in}}%
\pgfpathlineto{\pgfqpoint{2.163962in}{3.717780in}}%
\pgfpathlineto{\pgfqpoint{2.209083in}{3.677730in}}%
\pgfpathlineto{\pgfqpoint{2.254230in}{3.636897in}}%
\pgfpathlineto{\pgfqpoint{2.299400in}{3.595232in}}%
\pgfpathlineto{\pgfqpoint{2.344593in}{3.552683in}}%
\pgfpathlineto{\pgfqpoint{2.389803in}{3.509192in}}%
\pgfpathlineto{\pgfqpoint{2.435029in}{3.464701in}}%
\pgfpathlineto{\pgfqpoint{2.480267in}{3.419142in}}%
\pgfpathlineto{\pgfqpoint{2.525514in}{3.372444in}}%
\pgfpathlineto{\pgfqpoint{2.570764in}{3.324532in}}%
\pgfpathlineto{\pgfqpoint{2.616015in}{3.275322in}}%
\pgfpathlineto{\pgfqpoint{2.661261in}{3.224725in}}%
\pgfpathlineto{\pgfqpoint{2.706496in}{3.172645in}}%
\pgfpathlineto{\pgfqpoint{2.751716in}{3.118977in}}%
\pgfpathlineto{\pgfqpoint{2.796913in}{3.063608in}}%
\pgfpathlineto{\pgfqpoint{2.842080in}{3.006414in}}%
\pgfpathlineto{\pgfqpoint{2.887210in}{2.947261in}}%
\pgfpathlineto{\pgfqpoint{2.932295in}{2.886002in}}%
\pgfpathlineto{\pgfqpoint{2.977324in}{2.822472in}}%
\pgfpathlineto{\pgfqpoint{3.022287in}{2.756486in}}%
\pgfpathlineto{\pgfqpoint{3.067172in}{2.687831in}}%
\pgfpathlineto{\pgfqpoint{3.111966in}{2.616255in}}%
\pgfpathlineto{\pgfqpoint{3.156652in}{2.541452in}}%
\pgfpathlineto{\pgfqpoint{3.201212in}{2.463030in}}%
\pgfpathlineto{\pgfqpoint{3.245622in}{2.380470in}}%
\pgfpathlineto{\pgfqpoint{3.289850in}{2.293032in}}%
\pgfpathlineto{\pgfqpoint{3.333854in}{2.199617in}}%
\pgfpathlineto{\pgfqpoint{3.377568in}{2.098472in}}%
\pgfpathlineto{\pgfqpoint{3.420887in}{1.986642in}}%
\pgfpathlineto{\pgfqpoint{3.463630in}{1.858877in}}%
\pgfusepath{stroke}%
\end{pgfscope}%
\begin{pgfscope}%
\pgfpathrectangle{\pgfqpoint{0.000000in}{0.250000in}}{\pgfqpoint{4.500000in}{4.500000in}}%
\pgfusepath{clip}%
\pgfsetbuttcap%
\pgfsetroundjoin%
\pgfsetlinewidth{1.003750pt}%
\definecolor{currentstroke}{rgb}{0.121569,0.466667,0.705882}%
\pgfsetstrokecolor{currentstroke}%
\pgfsetdash{}{0pt}%
\pgfpathmoveto{\pgfqpoint{1.850067in}{4.262818in}}%
\pgfpathlineto{\pgfqpoint{1.894369in}{4.232339in}}%
\pgfpathlineto{\pgfqpoint{1.938716in}{4.201568in}}%
\pgfpathlineto{\pgfqpoint{1.983340in}{4.139373in}}%
\pgfpathlineto{\pgfqpoint{2.028014in}{4.066093in}}%
\pgfpathlineto{\pgfqpoint{2.071946in}{4.120444in}}%
\pgfpathlineto{\pgfqpoint{2.116296in}{4.129039in}}%
\pgfpathlineto{\pgfqpoint{2.160929in}{4.091163in}}%
\pgfpathlineto{\pgfqpoint{2.205591in}{4.052620in}}%
\pgfpathlineto{\pgfqpoint{2.250281in}{4.013369in}}%
\pgfpathlineto{\pgfqpoint{2.294996in}{3.973371in}}%
\pgfpathlineto{\pgfqpoint{2.339735in}{3.932581in}}%
\pgfpathlineto{\pgfqpoint{2.384495in}{3.890951in}}%
\pgfpathlineto{\pgfqpoint{2.429273in}{3.848429in}}%
\pgfpathlineto{\pgfqpoint{2.474067in}{3.804961in}}%
\pgfpathlineto{\pgfqpoint{2.518873in}{3.760489in}}%
\pgfpathlineto{\pgfqpoint{2.563687in}{3.714951in}}%
\pgfpathlineto{\pgfqpoint{2.608507in}{3.668279in}}%
\pgfpathlineto{\pgfqpoint{2.653327in}{3.620403in}}%
\pgfpathlineto{\pgfqpoint{2.698145in}{3.571249in}}%
\pgfpathlineto{\pgfqpoint{2.742954in}{3.520738in}}%
\pgfpathlineto{\pgfqpoint{2.787749in}{3.468789in}}%
\pgfpathlineto{\pgfqpoint{2.832526in}{3.415316in}}%
\pgfpathlineto{\pgfqpoint{2.877278in}{3.360230in}}%
\pgfpathlineto{\pgfqpoint{2.921998in}{3.303437in}}%
\pgfpathlineto{\pgfqpoint{2.966680in}{3.244844in}}%
\pgfpathlineto{\pgfqpoint{3.011316in}{3.184350in}}%
\pgfpathlineto{\pgfqpoint{3.055898in}{3.121854in}}%
\pgfpathlineto{\pgfqpoint{3.100418in}{3.057245in}}%
\pgfpathlineto{\pgfqpoint{3.144867in}{2.990410in}}%
\pgfpathlineto{\pgfqpoint{3.189234in}{2.921219in}}%
\pgfpathlineto{\pgfqpoint{3.233510in}{2.849528in}}%
\pgfpathlineto{\pgfqpoint{3.277680in}{2.775163in}}%
\pgfpathlineto{\pgfqpoint{3.321733in}{2.697907in}}%
\pgfpathlineto{\pgfqpoint{3.365649in}{2.617470in}}%
\pgfpathlineto{\pgfqpoint{3.409410in}{2.533442in}}%
\pgfpathlineto{\pgfqpoint{3.452985in}{2.445209in}}%
\pgfpathlineto{\pgfqpoint{3.496332in}{2.351789in}}%
\pgfpathlineto{\pgfqpoint{3.539387in}{2.251522in}}%
\pgfpathlineto{\pgfqpoint{3.582034in}{2.141423in}}%
\pgfpathlineto{\pgfqpoint{3.624065in}{2.015928in}}%
\pgfusepath{stroke}%
\end{pgfscope}%
\begin{pgfscope}%
\pgfpathrectangle{\pgfqpoint{0.000000in}{0.250000in}}{\pgfqpoint{4.500000in}{4.500000in}}%
\pgfusepath{clip}%
\pgfsetbuttcap%
\pgfsetroundjoin%
\pgfsetlinewidth{1.003750pt}%
\definecolor{currentstroke}{rgb}{0.121569,0.466667,0.705882}%
\pgfsetstrokecolor{currentstroke}%
\pgfsetdash{}{0pt}%
\pgfpathmoveto{\pgfqpoint{0.510977in}{3.000769in}}%
\pgfpathlineto{\pgfqpoint{0.715718in}{3.155367in}}%
\pgfpathlineto{\pgfqpoint{0.915333in}{3.322655in}}%
\pgfpathlineto{\pgfqpoint{1.110211in}{3.502356in}}%
\pgfpathlineto{\pgfqpoint{1.300766in}{3.692028in}}%
\pgfpathlineto{\pgfqpoint{1.487400in}{3.886553in}}%
\pgfpathlineto{\pgfqpoint{1.670315in}{4.087263in}}%
\pgfpathlineto{\pgfqpoint{1.850067in}{4.262818in}}%
\pgfusepath{stroke}%
\end{pgfscope}%
\begin{pgfscope}%
\pgfpathrectangle{\pgfqpoint{0.000000in}{0.250000in}}{\pgfqpoint{4.500000in}{4.500000in}}%
\pgfusepath{clip}%
\pgfsetbuttcap%
\pgfsetroundjoin%
\pgfsetlinewidth{1.003750pt}%
\definecolor{currentstroke}{rgb}{0.121569,0.466667,0.705882}%
\pgfsetstrokecolor{currentstroke}%
\pgfsetdash{}{0pt}%
\pgfpathmoveto{\pgfqpoint{0.557887in}{2.967375in}}%
\pgfpathlineto{\pgfqpoint{0.762199in}{3.122431in}}%
\pgfpathlineto{\pgfqpoint{0.961409in}{3.290161in}}%
\pgfpathlineto{\pgfqpoint{1.155905in}{3.470288in}}%
\pgfpathlineto{\pgfqpoint{1.346098in}{3.660374in}}%
\pgfpathlineto{\pgfqpoint{1.532384in}{3.855299in}}%
\pgfpathlineto{\pgfqpoint{1.714967in}{4.056398in}}%
\pgfpathlineto{\pgfqpoint{1.894369in}{4.232339in}}%
\pgfusepath{stroke}%
\end{pgfscope}%
\begin{pgfscope}%
\pgfpathrectangle{\pgfqpoint{0.000000in}{0.250000in}}{\pgfqpoint{4.500000in}{4.500000in}}%
\pgfusepath{clip}%
\pgfsetbuttcap%
\pgfsetroundjoin%
\pgfsetlinewidth{1.003750pt}%
\definecolor{currentstroke}{rgb}{0.121569,0.466667,0.705882}%
\pgfsetstrokecolor{currentstroke}%
\pgfsetdash{}{0pt}%
\pgfpathmoveto{\pgfqpoint{0.604856in}{2.933673in}}%
\pgfpathlineto{\pgfqpoint{0.808736in}{3.089190in}}%
\pgfpathlineto{\pgfqpoint{1.007538in}{3.257365in}}%
\pgfpathlineto{\pgfqpoint{1.201651in}{3.437922in}}%
\pgfpathlineto{\pgfqpoint{1.391480in}{3.628422in}}%
\pgfpathlineto{\pgfqpoint{1.577417in}{3.823750in}}%
\pgfpathlineto{\pgfqpoint{1.759665in}{4.025240in}}%
\pgfpathlineto{\pgfqpoint{1.938716in}{4.201568in}}%
\pgfusepath{stroke}%
\end{pgfscope}%
\begin{pgfscope}%
\pgfpathrectangle{\pgfqpoint{0.000000in}{0.250000in}}{\pgfqpoint{4.500000in}{4.500000in}}%
\pgfusepath{clip}%
\pgfsetbuttcap%
\pgfsetroundjoin%
\pgfsetlinewidth{1.003750pt}%
\definecolor{currentstroke}{rgb}{0.121569,0.466667,0.705882}%
\pgfsetstrokecolor{currentstroke}%
\pgfsetdash{}{0pt}%
\pgfpathmoveto{\pgfqpoint{0.651884in}{2.899650in}}%
\pgfpathlineto{\pgfqpoint{0.855330in}{3.055631in}}%
\pgfpathlineto{\pgfqpoint{1.053729in}{3.224037in}}%
\pgfpathlineto{\pgfqpoint{1.247465in}{3.404597in}}%
\pgfpathlineto{\pgfqpoint{1.436918in}{3.595894in}}%
\pgfpathlineto{\pgfqpoint{1.622445in}{3.795208in}}%
\pgfpathlineto{\pgfqpoint{1.804379in}{3.996347in}}%
\pgfpathlineto{\pgfqpoint{1.983340in}{4.139373in}}%
\pgfusepath{stroke}%
\end{pgfscope}%
\begin{pgfscope}%
\pgfpathrectangle{\pgfqpoint{0.000000in}{0.250000in}}{\pgfqpoint{4.500000in}{4.500000in}}%
\pgfusepath{clip}%
\pgfsetbuttcap%
\pgfsetroundjoin%
\pgfsetlinewidth{1.003750pt}%
\definecolor{currentstroke}{rgb}{0.121569,0.466667,0.705882}%
\pgfsetstrokecolor{currentstroke}%
\pgfsetdash{}{0pt}%
\pgfpathmoveto{\pgfqpoint{0.698970in}{2.865289in}}%
\pgfpathlineto{\pgfqpoint{0.901979in}{3.021738in}}%
\pgfpathlineto{\pgfqpoint{1.099973in}{3.190367in}}%
\pgfpathlineto{\pgfqpoint{1.293329in}{3.370912in}}%
\pgfpathlineto{\pgfqpoint{1.482405in}{3.562995in}}%
\pgfpathlineto{\pgfqpoint{1.667530in}{3.766099in}}%
\pgfpathlineto{\pgfqpoint{1.849145in}{3.967080in}}%
\pgfpathlineto{\pgfqpoint{2.028014in}{4.066093in}}%
\pgfusepath{stroke}%
\end{pgfscope}%
\begin{pgfscope}%
\pgfpathrectangle{\pgfqpoint{0.000000in}{0.250000in}}{\pgfqpoint{4.500000in}{4.500000in}}%
\pgfusepath{clip}%
\pgfsetbuttcap%
\pgfsetroundjoin%
\pgfsetlinewidth{1.003750pt}%
\definecolor{currentstroke}{rgb}{0.121569,0.466667,0.705882}%
\pgfsetstrokecolor{currentstroke}%
\pgfsetdash{}{0pt}%
\pgfpathmoveto{\pgfqpoint{0.746115in}{2.830575in}}%
\pgfpathlineto{\pgfqpoint{0.948685in}{2.987494in}}%
\pgfpathlineto{\pgfqpoint{1.146269in}{3.156338in}}%
\pgfpathlineto{\pgfqpoint{1.339244in}{3.336851in}}%
\pgfpathlineto{\pgfqpoint{1.527960in}{3.528652in}}%
\pgfpathlineto{\pgfqpoint{1.712744in}{3.731223in}}%
\pgfpathlineto{\pgfqpoint{1.893952in}{3.938262in}}%
\pgfpathlineto{\pgfqpoint{2.071946in}{4.120444in}}%
\pgfusepath{stroke}%
\end{pgfscope}%
\begin{pgfscope}%
\pgfpathrectangle{\pgfqpoint{0.000000in}{0.250000in}}{\pgfqpoint{4.500000in}{4.500000in}}%
\pgfusepath{clip}%
\pgfsetbuttcap%
\pgfsetroundjoin%
\pgfsetlinewidth{1.003750pt}%
\definecolor{currentstroke}{rgb}{0.121569,0.466667,0.705882}%
\pgfsetstrokecolor{currentstroke}%
\pgfsetdash{}{0pt}%
\pgfpathmoveto{\pgfqpoint{0.793318in}{2.795489in}}%
\pgfpathlineto{\pgfqpoint{0.995446in}{2.952882in}}%
\pgfpathlineto{\pgfqpoint{1.192619in}{3.121933in}}%
\pgfpathlineto{\pgfqpoint{1.385208in}{3.302393in}}%
\pgfpathlineto{\pgfqpoint{1.573561in}{3.493881in}}%
\pgfpathlineto{\pgfqpoint{1.758000in}{3.695874in}}%
\pgfpathlineto{\pgfqpoint{1.938822in}{3.907736in}}%
\pgfpathlineto{\pgfqpoint{2.116296in}{4.129039in}}%
\pgfusepath{stroke}%
\end{pgfscope}%
\begin{pgfscope}%
\pgfpathrectangle{\pgfqpoint{0.000000in}{0.250000in}}{\pgfqpoint{4.500000in}{4.500000in}}%
\pgfusepath{clip}%
\pgfsetbuttcap%
\pgfsetroundjoin%
\pgfsetlinewidth{1.003750pt}%
\definecolor{currentstroke}{rgb}{0.121569,0.466667,0.705882}%
\pgfsetstrokecolor{currentstroke}%
\pgfsetdash{}{0pt}%
\pgfpathmoveto{\pgfqpoint{0.840579in}{2.760012in}}%
\pgfpathlineto{\pgfqpoint{1.042261in}{2.917883in}}%
\pgfpathlineto{\pgfqpoint{1.239020in}{3.087132in}}%
\pgfpathlineto{\pgfqpoint{1.431222in}{3.267517in}}%
\pgfpathlineto{\pgfqpoint{1.619208in}{3.458658in}}%
\pgfpathlineto{\pgfqpoint{1.803297in}{3.660028in}}%
\pgfpathlineto{\pgfqpoint{1.983782in}{3.870992in}}%
\pgfpathlineto{\pgfqpoint{2.160929in}{4.091163in}}%
\pgfusepath{stroke}%
\end{pgfscope}%
\begin{pgfscope}%
\pgfpathrectangle{\pgfqpoint{0.000000in}{0.250000in}}{\pgfqpoint{4.500000in}{4.500000in}}%
\pgfusepath{clip}%
\pgfsetbuttcap%
\pgfsetroundjoin%
\pgfsetlinewidth{1.003750pt}%
\definecolor{currentstroke}{rgb}{0.121569,0.466667,0.705882}%
\pgfsetstrokecolor{currentstroke}%
\pgfsetdash{}{0pt}%
\pgfpathmoveto{\pgfqpoint{0.887896in}{2.724123in}}%
\pgfpathlineto{\pgfqpoint{1.089131in}{2.882476in}}%
\pgfpathlineto{\pgfqpoint{1.285473in}{3.051913in}}%
\pgfpathlineto{\pgfqpoint{1.477283in}{3.232201in}}%
\pgfpathlineto{\pgfqpoint{1.664899in}{3.422959in}}%
\pgfpathlineto{\pgfqpoint{1.848635in}{3.623656in}}%
\pgfpathlineto{\pgfqpoint{2.028778in}{3.833659in}}%
\pgfpathlineto{\pgfqpoint{2.205591in}{4.052620in}}%
\pgfusepath{stroke}%
\end{pgfscope}%
\begin{pgfscope}%
\pgfpathrectangle{\pgfqpoint{0.000000in}{0.250000in}}{\pgfqpoint{4.500000in}{4.500000in}}%
\pgfusepath{clip}%
\pgfsetbuttcap%
\pgfsetroundjoin%
\pgfsetlinewidth{1.003750pt}%
\definecolor{currentstroke}{rgb}{0.121569,0.466667,0.705882}%
\pgfsetstrokecolor{currentstroke}%
\pgfsetdash{}{0pt}%
\pgfpathmoveto{\pgfqpoint{0.935271in}{2.687800in}}%
\pgfpathlineto{\pgfqpoint{1.136055in}{2.846638in}}%
\pgfpathlineto{\pgfqpoint{1.331977in}{3.016253in}}%
\pgfpathlineto{\pgfqpoint{1.523392in}{3.196420in}}%
\pgfpathlineto{\pgfqpoint{1.710634in}{3.386757in}}%
\pgfpathlineto{\pgfqpoint{1.894011in}{3.586729in}}%
\pgfpathlineto{\pgfqpoint{2.073808in}{3.795704in}}%
\pgfpathlineto{\pgfqpoint{2.250281in}{4.013369in}}%
\pgfusepath{stroke}%
\end{pgfscope}%
\begin{pgfscope}%
\pgfpathrectangle{\pgfqpoint{0.000000in}{0.250000in}}{\pgfqpoint{4.500000in}{4.500000in}}%
\pgfusepath{clip}%
\pgfsetbuttcap%
\pgfsetroundjoin%
\pgfsetlinewidth{1.003750pt}%
\definecolor{currentstroke}{rgb}{0.121569,0.466667,0.705882}%
\pgfsetstrokecolor{currentstroke}%
\pgfsetdash{}{0pt}%
\pgfpathmoveto{\pgfqpoint{0.982702in}{2.651016in}}%
\pgfpathlineto{\pgfqpoint{1.183032in}{2.810344in}}%
\pgfpathlineto{\pgfqpoint{1.378530in}{2.980126in}}%
\pgfpathlineto{\pgfqpoint{1.569548in}{3.160147in}}%
\pgfpathlineto{\pgfqpoint{1.756411in}{3.350024in}}%
\pgfpathlineto{\pgfqpoint{1.939426in}{3.549215in}}%
\pgfpathlineto{\pgfqpoint{2.118870in}{3.757090in}}%
\pgfpathlineto{\pgfqpoint{2.294996in}{3.973371in}}%
\pgfusepath{stroke}%
\end{pgfscope}%
\begin{pgfscope}%
\pgfpathrectangle{\pgfqpoint{0.000000in}{0.250000in}}{\pgfqpoint{4.500000in}{4.500000in}}%
\pgfusepath{clip}%
\pgfsetbuttcap%
\pgfsetroundjoin%
\pgfsetlinewidth{1.003750pt}%
\definecolor{currentstroke}{rgb}{0.121569,0.466667,0.705882}%
\pgfsetstrokecolor{currentstroke}%
\pgfsetdash{}{0pt}%
\pgfpathmoveto{\pgfqpoint{1.030188in}{2.613746in}}%
\pgfpathlineto{\pgfqpoint{1.230061in}{2.773567in}}%
\pgfpathlineto{\pgfqpoint{1.425133in}{2.943505in}}%
\pgfpathlineto{\pgfqpoint{1.615749in}{3.123353in}}%
\pgfpathlineto{\pgfqpoint{1.802230in}{3.312727in}}%
\pgfpathlineto{\pgfqpoint{1.984877in}{3.511079in}}%
\pgfpathlineto{\pgfqpoint{2.163962in}{3.717780in}}%
\pgfpathlineto{\pgfqpoint{2.339735in}{3.932581in}}%
\pgfusepath{stroke}%
\end{pgfscope}%
\begin{pgfscope}%
\pgfpathrectangle{\pgfqpoint{0.000000in}{0.250000in}}{\pgfqpoint{4.500000in}{4.500000in}}%
\pgfusepath{clip}%
\pgfsetbuttcap%
\pgfsetroundjoin%
\pgfsetlinewidth{1.003750pt}%
\definecolor{currentstroke}{rgb}{0.121569,0.466667,0.705882}%
\pgfsetstrokecolor{currentstroke}%
\pgfsetdash{}{0pt}%
\pgfpathmoveto{\pgfqpoint{1.077730in}{2.575959in}}%
\pgfpathlineto{\pgfqpoint{1.277143in}{2.736278in}}%
\pgfpathlineto{\pgfqpoint{1.471785in}{2.906359in}}%
\pgfpathlineto{\pgfqpoint{1.661994in}{3.086006in}}%
\pgfpathlineto{\pgfqpoint{1.848089in}{3.274832in}}%
\pgfpathlineto{\pgfqpoint{2.030362in}{3.472284in}}%
\pgfpathlineto{\pgfqpoint{2.209083in}{3.677730in}}%
\pgfpathlineto{\pgfqpoint{2.384495in}{3.890951in}}%
\pgfusepath{stroke}%
\end{pgfscope}%
\begin{pgfscope}%
\pgfpathrectangle{\pgfqpoint{0.000000in}{0.250000in}}{\pgfqpoint{4.500000in}{4.500000in}}%
\pgfusepath{clip}%
\pgfsetbuttcap%
\pgfsetroundjoin%
\pgfsetlinewidth{1.003750pt}%
\definecolor{currentstroke}{rgb}{0.121569,0.466667,0.705882}%
\pgfsetstrokecolor{currentstroke}%
\pgfsetdash{}{0pt}%
\pgfpathmoveto{\pgfqpoint{1.125326in}{2.537621in}}%
\pgfpathlineto{\pgfqpoint{1.324275in}{2.698444in}}%
\pgfpathlineto{\pgfqpoint{1.518483in}{2.868656in}}%
\pgfpathlineto{\pgfqpoint{1.708283in}{3.048071in}}%
\pgfpathlineto{\pgfqpoint{1.893986in}{3.236303in}}%
\pgfpathlineto{\pgfqpoint{2.075880in}{3.432790in}}%
\pgfpathlineto{\pgfqpoint{2.254230in}{3.636897in}}%
\pgfpathlineto{\pgfqpoint{2.429273in}{3.848429in}}%
\pgfusepath{stroke}%
\end{pgfscope}%
\begin{pgfscope}%
\pgfpathrectangle{\pgfqpoint{0.000000in}{0.250000in}}{\pgfqpoint{4.500000in}{4.500000in}}%
\pgfusepath{clip}%
\pgfsetbuttcap%
\pgfsetroundjoin%
\pgfsetlinewidth{1.003750pt}%
\definecolor{currentstroke}{rgb}{0.121569,0.466667,0.705882}%
\pgfsetstrokecolor{currentstroke}%
\pgfsetdash{}{0pt}%
\pgfpathmoveto{\pgfqpoint{1.172976in}{2.498698in}}%
\pgfpathlineto{\pgfqpoint{1.371458in}{2.660028in}}%
\pgfpathlineto{\pgfqpoint{1.565229in}{2.830357in}}%
\pgfpathlineto{\pgfqpoint{1.754614in}{3.009509in}}%
\pgfpathlineto{\pgfqpoint{1.939920in}{3.197097in}}%
\pgfpathlineto{\pgfqpoint{2.121430in}{3.392551in}}%
\pgfpathlineto{\pgfqpoint{2.299400in}{3.595232in}}%
\pgfpathlineto{\pgfqpoint{2.474067in}{3.804961in}}%
\pgfusepath{stroke}%
\end{pgfscope}%
\begin{pgfscope}%
\pgfpathrectangle{\pgfqpoint{0.000000in}{0.250000in}}{\pgfqpoint{4.500000in}{4.500000in}}%
\pgfusepath{clip}%
\pgfsetbuttcap%
\pgfsetroundjoin%
\pgfsetlinewidth{1.003750pt}%
\definecolor{currentstroke}{rgb}{0.121569,0.466667,0.705882}%
\pgfsetstrokecolor{currentstroke}%
\pgfsetdash{}{0pt}%
\pgfpathmoveto{\pgfqpoint{1.220680in}{2.459148in}}%
\pgfpathlineto{\pgfqpoint{1.418690in}{2.620992in}}%
\pgfpathlineto{\pgfqpoint{1.612019in}{2.791423in}}%
\pgfpathlineto{\pgfqpoint{1.800986in}{2.970279in}}%
\pgfpathlineto{\pgfqpoint{1.985890in}{3.157170in}}%
\pgfpathlineto{\pgfqpoint{2.167008in}{3.351520in}}%
\pgfpathlineto{\pgfqpoint{2.344593in}{3.552683in}}%
\pgfpathlineto{\pgfqpoint{2.518873in}{3.760489in}}%
\pgfusepath{stroke}%
\end{pgfscope}%
\begin{pgfscope}%
\pgfpathrectangle{\pgfqpoint{0.000000in}{0.250000in}}{\pgfqpoint{4.500000in}{4.500000in}}%
\pgfusepath{clip}%
\pgfsetbuttcap%
\pgfsetroundjoin%
\pgfsetlinewidth{1.003750pt}%
\definecolor{currentstroke}{rgb}{0.121569,0.466667,0.705882}%
\pgfsetstrokecolor{currentstroke}%
\pgfsetdash{}{0pt}%
\pgfpathmoveto{\pgfqpoint{1.268435in}{2.418927in}}%
\pgfpathlineto{\pgfqpoint{1.465971in}{2.581290in}}%
\pgfpathlineto{\pgfqpoint{1.658854in}{2.751809in}}%
\pgfpathlineto{\pgfqpoint{1.847396in}{2.930334in}}%
\pgfpathlineto{\pgfqpoint{2.031893in}{3.116474in}}%
\pgfpathlineto{\pgfqpoint{2.212613in}{3.309645in}}%
\pgfpathlineto{\pgfqpoint{2.389803in}{3.509192in}}%
\pgfpathlineto{\pgfqpoint{2.563687in}{3.714951in}}%
\pgfusepath{stroke}%
\end{pgfscope}%
\begin{pgfscope}%
\pgfpathrectangle{\pgfqpoint{0.000000in}{0.250000in}}{\pgfqpoint{4.500000in}{4.500000in}}%
\pgfusepath{clip}%
\pgfsetbuttcap%
\pgfsetroundjoin%
\pgfsetlinewidth{1.003750pt}%
\definecolor{currentstroke}{rgb}{0.121569,0.466667,0.705882}%
\pgfsetstrokecolor{currentstroke}%
\pgfsetdash{}{0pt}%
\pgfpathmoveto{\pgfqpoint{1.316242in}{2.377985in}}%
\pgfpathlineto{\pgfqpoint{1.513299in}{2.540873in}}%
\pgfpathlineto{\pgfqpoint{1.705731in}{2.711465in}}%
\pgfpathlineto{\pgfqpoint{1.893845in}{2.889621in}}%
\pgfpathlineto{\pgfqpoint{2.077927in}{3.074954in}}%
\pgfpathlineto{\pgfqpoint{2.258242in}{3.266868in}}%
\pgfpathlineto{\pgfqpoint{2.435029in}{3.464701in}}%
\pgfpathlineto{\pgfqpoint{2.608507in}{3.668279in}}%
\pgfusepath{stroke}%
\end{pgfscope}%
\begin{pgfscope}%
\pgfpathrectangle{\pgfqpoint{0.000000in}{0.250000in}}{\pgfqpoint{4.500000in}{4.500000in}}%
\pgfusepath{clip}%
\pgfsetbuttcap%
\pgfsetroundjoin%
\pgfsetlinewidth{1.003750pt}%
\definecolor{currentstroke}{rgb}{0.121569,0.466667,0.705882}%
\pgfsetstrokecolor{currentstroke}%
\pgfsetdash{}{0pt}%
\pgfpathmoveto{\pgfqpoint{1.364099in}{2.336268in}}%
\pgfpathlineto{\pgfqpoint{1.560672in}{2.499687in}}%
\pgfpathlineto{\pgfqpoint{1.752649in}{2.670335in}}%
\pgfpathlineto{\pgfqpoint{1.940329in}{2.848085in}}%
\pgfpathlineto{\pgfqpoint{2.123990in}{3.032550in}}%
\pgfpathlineto{\pgfqpoint{2.303892in}{3.223127in}}%
\pgfpathlineto{\pgfqpoint{2.480267in}{3.419142in}}%
\pgfpathlineto{\pgfqpoint{2.653327in}{3.620403in}}%
\pgfusepath{stroke}%
\end{pgfscope}%
\begin{pgfscope}%
\pgfpathrectangle{\pgfqpoint{0.000000in}{0.250000in}}{\pgfqpoint{4.500000in}{4.500000in}}%
\pgfusepath{clip}%
\pgfsetbuttcap%
\pgfsetroundjoin%
\pgfsetlinewidth{1.003750pt}%
\definecolor{currentstroke}{rgb}{0.121569,0.466667,0.705882}%
\pgfsetstrokecolor{currentstroke}%
\pgfsetdash{}{0pt}%
\pgfpathmoveto{\pgfqpoint{1.412006in}{2.293712in}}%
\pgfpathlineto{\pgfqpoint{1.608091in}{2.457669in}}%
\pgfpathlineto{\pgfqpoint{1.799607in}{2.628357in}}%
\pgfpathlineto{\pgfqpoint{1.986846in}{2.805662in}}%
\pgfpathlineto{\pgfqpoint{2.170080in}{2.989197in}}%
\pgfpathlineto{\pgfqpoint{2.349561in}{3.178354in}}%
\pgfpathlineto{\pgfqpoint{2.525514in}{3.372444in}}%
\pgfpathlineto{\pgfqpoint{2.698145in}{3.571249in}}%
\pgfusepath{stroke}%
\end{pgfscope}%
\begin{pgfscope}%
\pgfpathrectangle{\pgfqpoint{0.000000in}{0.250000in}}{\pgfqpoint{4.500000in}{4.500000in}}%
\pgfusepath{clip}%
\pgfsetbuttcap%
\pgfsetroundjoin%
\pgfsetlinewidth{1.003750pt}%
\definecolor{currentstroke}{rgb}{0.121569,0.466667,0.705882}%
\pgfsetstrokecolor{currentstroke}%
\pgfsetdash{}{0pt}%
\pgfpathmoveto{\pgfqpoint{1.459960in}{2.250248in}}%
\pgfpathlineto{\pgfqpoint{1.655552in}{2.414750in}}%
\pgfpathlineto{\pgfqpoint{1.846603in}{2.585461in}}%
\pgfpathlineto{\pgfqpoint{2.033395in}{2.762280in}}%
\pgfpathlineto{\pgfqpoint{2.216194in}{2.944823in}}%
\pgfpathlineto{\pgfqpoint{2.395244in}{3.132474in}}%
\pgfpathlineto{\pgfqpoint{2.570764in}{3.324532in}}%
\pgfpathlineto{\pgfqpoint{2.742954in}{3.520738in}}%
\pgfusepath{stroke}%
\end{pgfscope}%
\begin{pgfscope}%
\pgfpathrectangle{\pgfqpoint{0.000000in}{0.250000in}}{\pgfqpoint{4.500000in}{4.500000in}}%
\pgfusepath{clip}%
\pgfsetbuttcap%
\pgfsetroundjoin%
\pgfsetlinewidth{1.003750pt}%
\definecolor{currentstroke}{rgb}{0.121569,0.466667,0.705882}%
\pgfsetstrokecolor{currentstroke}%
\pgfsetdash{}{0pt}%
\pgfpathmoveto{\pgfqpoint{1.507961in}{2.205796in}}%
\pgfpathlineto{\pgfqpoint{1.703055in}{2.370852in}}%
\pgfpathlineto{\pgfqpoint{1.893633in}{2.541568in}}%
\pgfpathlineto{\pgfqpoint{2.079972in}{2.717859in}}%
\pgfpathlineto{\pgfqpoint{2.262328in}{2.899345in}}%
\pgfpathlineto{\pgfqpoint{2.440939in}{3.085405in}}%
\pgfpathlineto{\pgfqpoint{2.616015in}{3.275322in}}%
\pgfpathlineto{\pgfqpoint{2.787749in}{3.468789in}}%
\pgfusepath{stroke}%
\end{pgfscope}%
\begin{pgfscope}%
\pgfpathrectangle{\pgfqpoint{0.000000in}{0.250000in}}{\pgfqpoint{4.500000in}{4.500000in}}%
\pgfusepath{clip}%
\pgfsetbuttcap%
\pgfsetroundjoin%
\pgfsetlinewidth{1.003750pt}%
\definecolor{currentstroke}{rgb}{0.121569,0.466667,0.705882}%
\pgfsetstrokecolor{currentstroke}%
\pgfsetdash{}{0pt}%
\pgfpathmoveto{\pgfqpoint{1.556006in}{2.160267in}}%
\pgfpathlineto{\pgfqpoint{1.750596in}{2.325885in}}%
\pgfpathlineto{\pgfqpoint{1.940697in}{2.496590in}}%
\pgfpathlineto{\pgfqpoint{2.126574in}{2.672311in}}%
\pgfpathlineto{\pgfqpoint{2.308480in}{2.852674in}}%
\pgfpathlineto{\pgfqpoint{2.486640in}{3.037056in}}%
\pgfpathlineto{\pgfqpoint{2.661261in}{3.224725in}}%
\pgfpathlineto{\pgfqpoint{2.832526in}{3.415316in}}%
\pgfusepath{stroke}%
\end{pgfscope}%
\begin{pgfscope}%
\pgfpathrectangle{\pgfqpoint{0.000000in}{0.250000in}}{\pgfqpoint{4.500000in}{4.500000in}}%
\pgfusepath{clip}%
\pgfsetbuttcap%
\pgfsetroundjoin%
\pgfsetlinewidth{1.003750pt}%
\definecolor{currentstroke}{rgb}{0.121569,0.466667,0.705882}%
\pgfsetstrokecolor{currentstroke}%
\pgfsetdash{}{0pt}%
\pgfpathmoveto{\pgfqpoint{1.604095in}{2.113558in}}%
\pgfpathlineto{\pgfqpoint{1.798175in}{2.279746in}}%
\pgfpathlineto{\pgfqpoint{1.987790in}{2.450423in}}%
\pgfpathlineto{\pgfqpoint{2.173199in}{2.625533in}}%
\pgfpathlineto{\pgfqpoint{2.354644in}{2.804709in}}%
\pgfpathlineto{\pgfqpoint{2.532344in}{2.987327in}}%
\pgfpathlineto{\pgfqpoint{2.706496in}{3.172645in}}%
\pgfpathlineto{\pgfqpoint{2.877278in}{3.360230in}}%
\pgfusepath{stroke}%
\end{pgfscope}%
\begin{pgfscope}%
\pgfpathrectangle{\pgfqpoint{0.000000in}{0.250000in}}{\pgfqpoint{4.500000in}{4.500000in}}%
\pgfusepath{clip}%
\pgfsetbuttcap%
\pgfsetroundjoin%
\pgfsetlinewidth{1.003750pt}%
\definecolor{currentstroke}{rgb}{0.121569,0.466667,0.705882}%
\pgfsetstrokecolor{currentstroke}%
\pgfsetdash{}{0pt}%
\pgfpathmoveto{\pgfqpoint{1.652225in}{2.065550in}}%
\pgfpathlineto{\pgfqpoint{1.845788in}{2.232319in}}%
\pgfpathlineto{\pgfqpoint{2.034911in}{2.402953in}}%
\pgfpathlineto{\pgfqpoint{2.219843in}{2.577410in}}%
\pgfpathlineto{\pgfqpoint{2.400817in}{2.755336in}}%
\pgfpathlineto{\pgfqpoint{2.578045in}{2.936109in}}%
\pgfpathlineto{\pgfqpoint{2.751716in}{3.118977in}}%
\pgfpathlineto{\pgfqpoint{2.921998in}{3.303437in}}%
\pgfusepath{stroke}%
\end{pgfscope}%
\begin{pgfscope}%
\pgfpathrectangle{\pgfqpoint{0.000000in}{0.250000in}}{\pgfqpoint{4.500000in}{4.500000in}}%
\pgfusepath{clip}%
\pgfsetbuttcap%
\pgfsetroundjoin%
\pgfsetlinewidth{1.003750pt}%
\definecolor{currentstroke}{rgb}{0.121569,0.466667,0.705882}%
\pgfsetstrokecolor{currentstroke}%
\pgfsetdash{}{0pt}%
\pgfpathmoveto{\pgfqpoint{1.700393in}{2.016106in}}%
\pgfpathlineto{\pgfqpoint{1.893433in}{2.183468in}}%
\pgfpathlineto{\pgfqpoint{2.082055in}{2.354044in}}%
\pgfpathlineto{\pgfqpoint{2.266501in}{2.527811in}}%
\pgfpathlineto{\pgfqpoint{2.446993in}{2.704427in}}%
\pgfpathlineto{\pgfqpoint{2.623737in}{2.883278in}}%
\pgfpathlineto{\pgfqpoint{2.796913in}{3.063608in}}%
\pgfpathlineto{\pgfqpoint{2.966680in}{3.244844in}}%
\pgfusepath{stroke}%
\end{pgfscope}%
\begin{pgfscope}%
\pgfpathrectangle{\pgfqpoint{0.000000in}{0.250000in}}{\pgfqpoint{4.500000in}{4.500000in}}%
\pgfusepath{clip}%
\pgfsetbuttcap%
\pgfsetroundjoin%
\pgfsetlinewidth{1.003750pt}%
\definecolor{currentstroke}{rgb}{0.121569,0.466667,0.705882}%
\pgfsetstrokecolor{currentstroke}%
\pgfsetdash{}{0pt}%
\pgfpathmoveto{\pgfqpoint{1.748598in}{1.965069in}}%
\pgfpathlineto{\pgfqpoint{1.941106in}{2.133035in}}%
\pgfpathlineto{\pgfqpoint{2.129219in}{2.303542in}}%
\pgfpathlineto{\pgfqpoint{2.313168in}{2.476585in}}%
\pgfpathlineto{\pgfqpoint{2.493168in}{2.651835in}}%
\pgfpathlineto{\pgfqpoint{2.669415in}{2.828698in}}%
\pgfpathlineto{\pgfqpoint{2.842080in}{3.006414in}}%
\pgfpathlineto{\pgfqpoint{3.011316in}{3.184350in}}%
\pgfusepath{stroke}%
\end{pgfscope}%
\begin{pgfscope}%
\pgfpathrectangle{\pgfqpoint{0.000000in}{0.250000in}}{\pgfqpoint{4.500000in}{4.500000in}}%
\pgfusepath{clip}%
\pgfsetbuttcap%
\pgfsetroundjoin%
\pgfsetlinewidth{1.003750pt}%
\definecolor{currentstroke}{rgb}{0.121569,0.466667,0.705882}%
\pgfsetstrokecolor{currentstroke}%
\pgfsetdash{}{0pt}%
\pgfpathmoveto{\pgfqpoint{1.796835in}{1.912252in}}%
\pgfpathlineto{\pgfqpoint{1.988803in}{2.080836in}}%
\pgfpathlineto{\pgfqpoint{2.176397in}{2.251267in}}%
\pgfpathlineto{\pgfqpoint{2.359839in}{2.423555in}}%
\pgfpathlineto{\pgfqpoint{2.539334in}{2.597396in}}%
\pgfpathlineto{\pgfqpoint{2.715070in}{2.772214in}}%
\pgfpathlineto{\pgfqpoint{2.887210in}{2.947261in}}%
\pgfpathlineto{\pgfqpoint{3.055898in}{3.121854in}}%
\pgfusepath{stroke}%
\end{pgfscope}%
\begin{pgfscope}%
\pgfpathrectangle{\pgfqpoint{0.000000in}{0.250000in}}{\pgfqpoint{4.500000in}{4.500000in}}%
\pgfusepath{clip}%
\pgfsetbuttcap%
\pgfsetroundjoin%
\pgfsetlinewidth{1.003750pt}%
\definecolor{currentstroke}{rgb}{0.121569,0.466667,0.705882}%
\pgfsetstrokecolor{currentstroke}%
\pgfsetdash{}{0pt}%
\pgfpathmoveto{\pgfqpoint{1.845101in}{1.857436in}}%
\pgfpathlineto{\pgfqpoint{2.036519in}{2.026653in}}%
\pgfpathlineto{\pgfqpoint{2.223584in}{2.197005in}}%
\pgfpathlineto{\pgfqpoint{2.406507in}{2.368519in}}%
\pgfpathlineto{\pgfqpoint{2.585485in}{2.540917in}}%
\pgfpathlineto{\pgfqpoint{2.760695in}{2.713653in}}%
\pgfpathlineto{\pgfqpoint{2.932295in}{2.886002in}}%
\pgfpathlineto{\pgfqpoint{3.100418in}{3.057245in}}%
\pgfusepath{stroke}%
\end{pgfscope}%
\begin{pgfscope}%
\pgfpathrectangle{\pgfqpoint{0.000000in}{0.250000in}}{\pgfqpoint{4.500000in}{4.500000in}}%
\pgfusepath{clip}%
\pgfsetbuttcap%
\pgfsetroundjoin%
\pgfsetlinewidth{1.003750pt}%
\definecolor{currentstroke}{rgb}{0.121569,0.466667,0.705882}%
\pgfsetstrokecolor{currentstroke}%
\pgfsetdash{}{0pt}%
\pgfpathmoveto{\pgfqpoint{1.893392in}{1.800361in}}%
\pgfpathlineto{\pgfqpoint{2.084250in}{1.970227in}}%
\pgfpathlineto{\pgfqpoint{2.270774in}{2.140505in}}%
\pgfpathlineto{\pgfqpoint{2.453165in}{2.311237in}}%
\pgfpathlineto{\pgfqpoint{2.631611in}{2.482176in}}%
\pgfpathlineto{\pgfqpoint{2.806281in}{2.652818in}}%
\pgfpathlineto{\pgfqpoint{2.977324in}{2.822472in}}%
\pgfpathlineto{\pgfqpoint{3.144867in}{2.990410in}}%
\pgfusepath{stroke}%
\end{pgfscope}%
\begin{pgfscope}%
\pgfpathrectangle{\pgfqpoint{0.000000in}{0.250000in}}{\pgfqpoint{4.500000in}{4.500000in}}%
\pgfusepath{clip}%
\pgfsetbuttcap%
\pgfsetroundjoin%
\pgfsetlinewidth{1.003750pt}%
\definecolor{currentstroke}{rgb}{0.121569,0.466667,0.705882}%
\pgfsetstrokecolor{currentstroke}%
\pgfsetdash{}{0pt}%
\pgfpathmoveto{\pgfqpoint{1.941703in}{1.740714in}}%
\pgfpathlineto{\pgfqpoint{2.131989in}{1.911248in}}%
\pgfpathlineto{\pgfqpoint{2.317959in}{2.081466in}}%
\pgfpathlineto{\pgfqpoint{2.499804in}{2.251425in}}%
\pgfpathlineto{\pgfqpoint{2.677703in}{2.420914in}}%
\pgfpathlineto{\pgfqpoint{2.851817in}{2.589481in}}%
\pgfpathlineto{\pgfqpoint{3.022287in}{2.756486in}}%
\pgfpathlineto{\pgfqpoint{3.189234in}{2.921219in}}%
\pgfusepath{stroke}%
\end{pgfscope}%
\begin{pgfscope}%
\pgfpathrectangle{\pgfqpoint{0.000000in}{0.250000in}}{\pgfqpoint{4.500000in}{4.500000in}}%
\pgfusepath{clip}%
\pgfsetbuttcap%
\pgfsetroundjoin%
\pgfsetlinewidth{1.003750pt}%
\definecolor{currentstroke}{rgb}{0.121569,0.466667,0.705882}%
\pgfsetstrokecolor{currentstroke}%
\pgfsetdash{}{0pt}%
\pgfpathmoveto{\pgfqpoint{1.990028in}{1.678116in}}%
\pgfpathlineto{\pgfqpoint{2.179729in}{1.849340in}}%
\pgfpathlineto{\pgfqpoint{2.365130in}{2.019523in}}%
\pgfpathlineto{\pgfqpoint{2.546414in}{2.188741in}}%
\pgfpathlineto{\pgfqpoint{2.723750in}{2.356821in}}%
\pgfpathlineto{\pgfqpoint{2.897291in}{2.523374in}}%
\pgfpathlineto{\pgfqpoint{3.067172in}{2.687831in}}%
\pgfpathlineto{\pgfqpoint{3.233510in}{2.849528in}}%
\pgfusepath{stroke}%
\end{pgfscope}%
\begin{pgfscope}%
\pgfpathrectangle{\pgfqpoint{0.000000in}{0.250000in}}{\pgfqpoint{4.500000in}{4.500000in}}%
\pgfusepath{clip}%
\pgfsetbuttcap%
\pgfsetroundjoin%
\pgfsetlinewidth{1.003750pt}%
\definecolor{currentstroke}{rgb}{0.121569,0.466667,0.705882}%
\pgfsetstrokecolor{currentstroke}%
\pgfsetdash{}{0pt}%
\pgfpathmoveto{\pgfqpoint{2.038360in}{1.612102in}}%
\pgfpathlineto{\pgfqpoint{2.227459in}{1.784041in}}%
\pgfpathlineto{\pgfqpoint{2.412276in}{1.954234in}}%
\pgfpathlineto{\pgfqpoint{2.592982in}{2.122770in}}%
\pgfpathlineto{\pgfqpoint{2.769737in}{2.289525in}}%
\pgfpathlineto{\pgfqpoint{2.942688in}{2.454181in}}%
\pgfpathlineto{\pgfqpoint{3.111966in}{2.616255in}}%
\pgfpathlineto{\pgfqpoint{3.277680in}{2.775163in}}%
\pgfusepath{stroke}%
\end{pgfscope}%
\begin{pgfscope}%
\pgfpathrectangle{\pgfqpoint{0.000000in}{0.250000in}}{\pgfqpoint{4.500000in}{4.500000in}}%
\pgfusepath{clip}%
\pgfsetbuttcap%
\pgfsetroundjoin%
\pgfsetlinewidth{1.003750pt}%
\definecolor{currentstroke}{rgb}{0.121569,0.466667,0.705882}%
\pgfsetstrokecolor{currentstroke}%
\pgfsetdash{}{0pt}%
\pgfpathmoveto{\pgfqpoint{2.086689in}{1.542097in}}%
\pgfpathlineto{\pgfqpoint{2.275170in}{1.714780in}}%
\pgfpathlineto{\pgfqpoint{2.459383in}{1.885047in}}%
\pgfpathlineto{\pgfqpoint{2.639492in}{2.053002in}}%
\pgfpathlineto{\pgfqpoint{2.815649in}{2.218570in}}%
\pgfpathlineto{\pgfqpoint{2.987994in}{2.381513in}}%
\pgfpathlineto{\pgfqpoint{3.156652in}{2.541452in}}%
\pgfpathlineto{\pgfqpoint{3.321733in}{2.697907in}}%
\pgfusepath{stroke}%
\end{pgfscope}%
\begin{pgfscope}%
\pgfpathrectangle{\pgfqpoint{0.000000in}{0.250000in}}{\pgfqpoint{4.500000in}{4.500000in}}%
\pgfusepath{clip}%
\pgfsetbuttcap%
\pgfsetroundjoin%
\pgfsetlinewidth{1.003750pt}%
\definecolor{currentstroke}{rgb}{0.121569,0.466667,0.705882}%
\pgfsetstrokecolor{currentstroke}%
\pgfsetdash{}{0pt}%
\pgfpathmoveto{\pgfqpoint{2.135006in}{1.467377in}}%
\pgfpathlineto{\pgfqpoint{2.322846in}{1.640837in}}%
\pgfpathlineto{\pgfqpoint{2.506435in}{1.811271in}}%
\pgfpathlineto{\pgfqpoint{2.685926in}{1.978794in}}%
\pgfpathlineto{\pgfqpoint{2.861465in}{2.143382in}}%
\pgfpathlineto{\pgfqpoint{3.033187in}{2.304883in}}%
\pgfpathlineto{\pgfqpoint{3.201212in}{2.463030in}}%
\pgfpathlineto{\pgfqpoint{3.365649in}{2.617470in}}%
\pgfusepath{stroke}%
\end{pgfscope}%
\begin{pgfscope}%
\pgfpathrectangle{\pgfqpoint{0.000000in}{0.250000in}}{\pgfqpoint{4.500000in}{4.500000in}}%
\pgfusepath{clip}%
\pgfsetbuttcap%
\pgfsetroundjoin%
\pgfsetlinewidth{1.003750pt}%
\definecolor{currentstroke}{rgb}{0.121569,0.466667,0.705882}%
\pgfsetstrokecolor{currentstroke}%
\pgfsetdash{}{0pt}%
\pgfpathmoveto{\pgfqpoint{2.183295in}{1.387020in}}%
\pgfpathlineto{\pgfqpoint{2.370471in}{1.561297in}}%
\pgfpathlineto{\pgfqpoint{2.553410in}{1.732023in}}%
\pgfpathlineto{\pgfqpoint{2.732260in}{1.899323in}}%
\pgfpathlineto{\pgfqpoint{2.907161in}{2.063224in}}%
\pgfpathlineto{\pgfqpoint{3.078242in}{2.223656in}}%
\pgfpathlineto{\pgfqpoint{3.245622in}{2.380470in}}%
\pgfpathlineto{\pgfqpoint{3.409410in}{2.533442in}}%
\pgfusepath{stroke}%
\end{pgfscope}%
\begin{pgfscope}%
\pgfpathrectangle{\pgfqpoint{0.000000in}{0.250000in}}{\pgfqpoint{4.500000in}{4.500000in}}%
\pgfusepath{clip}%
\pgfsetbuttcap%
\pgfsetroundjoin%
\pgfsetlinewidth{1.003750pt}%
\definecolor{currentstroke}{rgb}{0.121569,0.466667,0.705882}%
\pgfsetstrokecolor{currentstroke}%
\pgfsetdash{}{0pt}%
\pgfpathmoveto{\pgfqpoint{2.231540in}{1.299825in}}%
\pgfpathlineto{\pgfqpoint{2.418021in}{1.474967in}}%
\pgfpathlineto{\pgfqpoint{2.600282in}{1.646151in}}%
\pgfpathlineto{\pgfqpoint{2.778465in}{1.813510in}}%
\pgfpathlineto{\pgfqpoint{2.952704in}{1.977113in}}%
\pgfpathlineto{\pgfqpoint{3.123127in}{2.136969in}}%
\pgfpathlineto{\pgfqpoint{3.289850in}{2.293032in}}%
\pgfpathlineto{\pgfqpoint{3.452985in}{2.445209in}}%
\pgfusepath{stroke}%
\end{pgfscope}%
\begin{pgfscope}%
\pgfpathrectangle{\pgfqpoint{0.000000in}{0.250000in}}{\pgfqpoint{4.500000in}{4.500000in}}%
\pgfusepath{clip}%
\pgfsetbuttcap%
\pgfsetroundjoin%
\pgfsetlinewidth{1.003750pt}%
\definecolor{currentstroke}{rgb}{0.121569,0.466667,0.705882}%
\pgfsetstrokecolor{currentstroke}%
\pgfsetdash{}{0pt}%
\pgfpathmoveto{\pgfqpoint{2.279717in}{1.204206in}}%
\pgfpathlineto{\pgfqpoint{2.465466in}{1.380272in}}%
\pgfpathlineto{\pgfqpoint{2.647014in}{1.552122in}}%
\pgfpathlineto{\pgfqpoint{2.824499in}{1.719895in}}%
\pgfpathlineto{\pgfqpoint{2.998052in}{1.883694in}}%
\pgfpathlineto{\pgfqpoint{3.167798in}{2.043589in}}%
\pgfpathlineto{\pgfqpoint{3.333854in}{2.199617in}}%
\pgfpathlineto{\pgfqpoint{3.496332in}{2.351789in}}%
\pgfusepath{stroke}%
\end{pgfscope}%
\begin{pgfscope}%
\pgfpathrectangle{\pgfqpoint{0.000000in}{0.250000in}}{\pgfqpoint{4.500000in}{4.500000in}}%
\pgfusepath{clip}%
\pgfsetbuttcap%
\pgfsetroundjoin%
\pgfsetlinewidth{1.003750pt}%
\definecolor{currentstroke}{rgb}{0.121569,0.466667,0.705882}%
\pgfsetstrokecolor{currentstroke}%
\pgfsetdash{}{0pt}%
\pgfpathmoveto{\pgfqpoint{2.327795in}{1.098013in}}%
\pgfpathlineto{\pgfqpoint{2.512766in}{1.275074in}}%
\pgfpathlineto{\pgfqpoint{2.693559in}{1.447839in}}%
\pgfpathlineto{\pgfqpoint{2.870308in}{1.616446in}}%
\pgfpathlineto{\pgfqpoint{3.043144in}{1.781018in}}%
\pgfpathlineto{\pgfqpoint{3.212192in}{1.941664in}}%
\pgfpathlineto{\pgfqpoint{3.377568in}{2.098472in}}%
\pgfpathlineto{\pgfqpoint{3.539387in}{2.251522in}}%
\pgfusepath{stroke}%
\end{pgfscope}%
\begin{pgfscope}%
\pgfpathrectangle{\pgfqpoint{0.000000in}{0.250000in}}{\pgfqpoint{4.500000in}{4.500000in}}%
\pgfusepath{clip}%
\pgfsetbuttcap%
\pgfsetroundjoin%
\pgfsetlinewidth{1.003750pt}%
\definecolor{currentstroke}{rgb}{0.121569,0.466667,0.705882}%
\pgfsetstrokecolor{currentstroke}%
\pgfsetdash{}{0pt}%
\pgfpathmoveto{\pgfqpoint{2.375731in}{0.978257in}}%
\pgfpathlineto{\pgfqpoint{2.559867in}{1.156407in}}%
\pgfpathlineto{\pgfqpoint{2.739849in}{1.330347in}}%
\pgfpathlineto{\pgfqpoint{2.915815in}{1.500217in}}%
\pgfpathlineto{\pgfqpoint{3.087894in}{1.666145in}}%
\pgfpathlineto{\pgfqpoint{3.256212in}{1.828250in}}%
\pgfpathlineto{\pgfqpoint{3.420887in}{1.986642in}}%
\pgfpathlineto{\pgfqpoint{3.582034in}{2.141423in}}%
\pgfusepath{stroke}%
\end{pgfscope}%
\begin{pgfscope}%
\pgfpathrectangle{\pgfqpoint{0.000000in}{0.250000in}}{\pgfqpoint{4.500000in}{4.500000in}}%
\pgfusepath{clip}%
\pgfsetbuttcap%
\pgfsetroundjoin%
\pgfsetlinewidth{1.003750pt}%
\definecolor{currentstroke}{rgb}{0.121569,0.466667,0.705882}%
\pgfsetstrokecolor{currentstroke}%
\pgfsetdash{}{0pt}%
\pgfpathmoveto{\pgfqpoint{2.423465in}{0.840658in}}%
\pgfpathlineto{\pgfqpoint{2.606688in}{1.020015in}}%
\pgfpathlineto{\pgfqpoint{2.785789in}{1.195337in}}%
\pgfpathlineto{\pgfqpoint{2.960904in}{1.366758in}}%
\pgfpathlineto{\pgfqpoint{3.132167in}{1.534408in}}%
\pgfpathlineto{\pgfqpoint{3.299702in}{1.698408in}}%
\pgfpathlineto{\pgfqpoint{3.463630in}{1.858877in}}%
\pgfpathlineto{\pgfqpoint{3.624065in}{2.015928in}}%
\pgfusepath{stroke}%
\end{pgfscope}%
\end{pgfpicture}%
\makeatother%
\endgroup%

	\caption{The unidirectional component of Flamm's paraboloid as a function of time. Even in a very short time interval, errors gradually accumulate over time, especially distortions can be seen.}
	\label{fig:data_by_time}
\end{figure}


